\section{Fundamental Groups}  \label{sectFundGr}

All introductions to algebraic topology start with a study of
fundamental groups associated to a topological spaces even if this is
not the end goal.  It is a very visual introduction to the subject of
algebraic topology.  We provide only a brief overview of this subject,
leaving many of the stated results without proofs.
As we said in the introduction to this chapter, the reader may find
proofs of all the results that we state without proofs in \cite{GH,MUat,ST}
xofor instance.

The fundamental concept in this section is the notion of homotopy
between continuous functions.  We have already defined the concept of
homotopic functions in Definition~\ref{defnSmoothHomot} in the context
of maps between manifolds.  We generalize this concept to any
topological spaces.

\begin{focus}{Note}
From now on, when we talk about a topological space, we assume that it
is a Hausdorff space.
\end{focus}

\begin{defn}
Let $X$ and $Y$ be two topological spaces and
$f,g:X \to Y$ be two continuous functions.  We say that $f$ is
{\bfseries homotopic}\index{Homotopic Functions} to $g$ if there exists a
continuous map $H:X\times[0,1] \to Y$ such that
$H(x,0) = f(x)$ and $H(x,1) = g(x)$ for all $x \in X$.  We write
$f \sim g$.  The function $H$ is called an
{\bfseries homotopy}\index{Homotopy} from $f$ to $g$.
\end{defn}

This definition of homotopic functions differs from the definition
given in Definition~\ref{defnSmoothHomot} because we now require only that
functions be continuous.   Moreover, manifolds are replaced by any
topological spaces.   The concept of homotopy between continuous
functions defines an equivalence relation. Namely, if $f,g,h:X\to Y$
are continuous functions between two topological spaces $X$ and $Y$, then
\begin{enumerate}
\item $f \sim f$,
\item $f\sim g$ if and only if $g \sim f$, and
\item $f\sim g$ and $g \sim h$ imply that $f \sim h$.
\end{enumerate}

\begin{defn}
A topological space $X$ is {\bfseries contractible}\index{Contractible} if
there exists $x_0 \in X$ such that the identity function
$\Id_X:X \to X$ is homotopic to the constant function $f:X \to X$
defined by $f(x) = x_0$ for all $x \in X$. 
\end{defn}

As for the previous definition for homotopy between functions, we only
require a continuous homotopy for the definition of contractible
spaces instead of a smooth homotopy as in definition~\ref{defnCtoP}.

We also slightly modify our definition of path between two points
given in Definition~\ref{defnPathV1} for the context where we require
only that functions be continuous.

\begin{defn}
Let $X$ be a topological space.  A {\bfseries path from $x_0 \in X$ to
$x_1 \in X$}\index{Path} is a continuous function
$\alpha:[0,1] \to X$ such that $\alpha(0) = x_0$ and $\alpha(1) = x_1$.
\end{defn}

\begin{defn}
Let $X$ be a topological space.
If $\alpha$ is a path from $x_0\in X$ to $x_1 \in X$ and
$\beta$ is a path from $x_1 \in X$ to $x_2 \in X$, the {\bfseries
product}\index{Product of Paths} of $\alpha$ and $\beta$, denoted
$\alpha \beta$, is the path from $x_0$ to $x_2$ defined by
\[
(\alpha \beta) (t) = \begin{cases}
\alpha(2t) & \quad \text{if} \ 0 \leq t \leq 1/2 \\
\beta(2t-1) & \quad \text{if} \ 1/2 < t \leq 1
\end{cases}
\]
for $0 \leq t \leq 1$.
Also, $\displaystyle \alpha^{-1}$ is the path from $x_1$ to $x_0$ defined by
$\displaystyle \alpha^{-1}(t) = \alpha(1-t)$ for $0 \leq t \leq 1$.
\end{defn}

The product $\alpha \beta$ is what we informally called the union of the paths
$\alpha$ and $\beta$ in Chapter~\ref{chaptCVC}.  In this section, we
are really strict in preserving the interval $[0,1]$ of definition of
a path.  We are not just interested in the image of $\alpha$ and
$\beta$ as in Chapter~\ref{chaptCVC}. 

\begin{defn} \label{defnHomPaths}
Let $X$ be a topological space.  Suppose that $\alpha$ and $\beta$ are
two paths from $x_0 \in X$ to $x_1 \in X$.  We say that $\alpha$ is
{\bfseries homotopic}\index{Homotopic} to $\beta$ if there exists a
continuous map $H:[0,1] \times [0,1] \to X$ such that
$H(t,0) = \alpha(t)$ and $H(t,1) = \beta(t)$ for $0 \leq t \leq 1$,
and $H(0,s) = x_0$ and $H(1,s) = x_1$ for $0 \leq s \leq 1$.
We write $\alpha \dotsim[X] \beta$.  If there is no risk of confusion
for the range $X$ of $H$, we may simply write $\alpha \dotsim \beta$. 
\end{defn}

The path $\alpha$ is deformed to the path $\beta$ without changing the
end points (Figure \ref{FundGr1}).

\pdfF{alg_top/fundgr1}{Homotopic paths}{The path $\alpha$ from
$x_0$ to $x_1$ is homotopic to the path $\beta$ from $x_0$ to $x_1$.
We have $0 < s_1, s_2 < 1$ in the figure.}{FundGr1}

The homotopy of paths between two fixed points $x_0$ and $x_1$ of a
topological space $X$ defines an equivalence relation.
Suppose that $\alpha$, $\beta$ and $\gamma$ are paths from $x_0$ to $x_1$.
Then
\begin{enumerate}
\item $\alpha \dotsim \alpha$,
\item $\alpha \dotsim \beta$ if and only if $\beta \dotsim \alpha$, and
\item $\alpha \dotsim \beta$ and $\beta \dotsim \gamma$ imply
that $\alpha \dotsim \gamma$.
\end{enumerate}
We prove that last condition and left the proof of the two others to
the reader.  Since $\alpha \dotsim \beta$, there exists a continuous
function $H_1:[0,1] \times [0,1] \to X$ such that
$H_1(t,0) = \alpha(t)$ and $H_1(t,1) = \beta(t)$ for $0 \leq t \leq 1$,
and $H_1(0,s) = x_0$ and $H_1(1,s) = x_1$ for $0 \leq s \leq 1$.
Since $\beta \dotsim \gamma$, there exists a continuous function
$H_2:[0,1] \times [0,1] \to X$ such that
$H_2(t,0) = \beta(t)$ and $H_2(t,1) = \gamma(t)$ for $0 \leq t \leq 1$,
and $H_2(0,s) = x_0$ and $H_2(1,s) = x_1$ for $0 \leq s \leq 1$.
Let
\[
H(t,s) = \begin{cases}
H_1(t,2s) & \quad \text{if} \ 0 \leq s < 1/2 \ \text{and} \ 0\leq t \leq 1 \\
H_1(t,2s-1) & \quad \text{if} \ 1/2 \leq s \leq 1
\ \text{and} \ 0\leq t \leq 1
\end{cases}
\]
Then $H:[0,1] \times [0,1] \to X$ is a continuous function such that
$H(t,0) = \alpha(t)$ and $H(t,1) = \gamma(t)$ for $0 \leq t \leq 1$,
and $H(0,s) = x_0$ and $H(1,s) = x_1$ for $0 \leq s \leq 1$.  Thus
$\alpha \dotsim \gamma$.

It is not hard to prove that if $\alpha_1,\alpha_2$ are two paths from
$x_0\in X$ to $x_1 \in X$ such that $\alpha_1 \dotsim \alpha_2$, and
$\beta_1,\beta_2$ are two paths from $x_1 \in X$ to $x_2 \in X$ such
that $\beta_1 \dotsim \beta_2$, then $\alpha_1 \beta_1$ and
$\alpha_2\beta_2$ are two paths from $x_0$ to $x_2$ such that
$\alpha_1 \beta_1 \dotsim \alpha_2 \beta_2$.  Moreover,
$\displaystyle \alpha_1^{-1},\alpha_2^{-1}$ are two paths from
$x_1\in X$ to $x_0 \in X$ such that
$\displaystyle \alpha_1^{-1} \dotsim \alpha_2^{-1}$.

\begin{defn}
Let $X$ be a topological space and $x_0 \in X$.  A closed path in $X$ from
$x_0$ to $x_0$ is called a {\bfseries loop}\index{Loop} at $x_0$.
\end{defn}

Let $X$ be a topological space and $x_0 \in X$.  The set of
equivalence classes $[\alpha]$ of homotopic loops $\alpha$ at $x_0$ is
denoted $\pi_1(X,x_0)$.  It is a group where the product is defined by
$[\alpha]\,[\beta] = [\alpha\,\beta]$ for all loops $\alpha$ and
$\beta$ at $x_0$, the inverse is defined by 
$\displaystyle [\alpha]^{-1} = [\alpha^{-1}]$ for all loops $\alpha$
at $x_0$, and the identity element is defined by $[e_{x_0}]$ for the
path $e_{x_0}$ given by $e_{x_0}(t) = x_0$ for $0 \leq t \leq 1$. 
All these algebraic operations are well defined according to what we
said in the paragraph preceding the definition above.

\begin{defn} \label{defnFGpione}
The group $\pi_1(X,x_0)$ is called the
{\bfseries fundamental group}\index{Fundamental Group} or {\bfseries
$\displaystyle \mathbf{1^{st}}$ homotopy
group}\index{$\displaystyle 1^{st}$ Homotopy Group|see{Fundamental Group}} of
the pair $(X,x_0)$.
\end{defn}

\begin{prop}
Suppose that $f:X \to Y$ is a continuous function between two
path-connected topological spaces and that $x_0 \in X$.  Then the map
$f_\ast:\pi_1(X,x_0) \to \pi_1(Y,f(x_0))$ defined by
$f_\ast([\alpha]) = [f\circ \alpha]$ for all
$[\alpha] \in \pi_1(X,x_0)$ is an homomorphism.
\end{prop}

\begin{proof}
\stage{i} The map $f_\ast$ is well defined.  If $\sigma_0$ and
$\sigma_1$ are two loops in $X$ at $x_0$ such that
$\sigma_0 \dotsim \sigma_1$, then there exists
$H:[0,1]\times[0,1] \to X$ such that
$H(0,s) = H(1,s) = x_0$ for $0\leq s \leq 1$, and
$H(t,0) = \sigma_0(t)$ and $H(t,1) = \sigma_1(t)$ for $0\leq t \leq 1$.

We have that $f\circ \sigma_0$ and $f\circ \sigma_1$ are two loops in
$Y$ at $f(x_0)$.  Let $\tilde{H} = f \circ H$.  Then 
$\tilde{H}:[0,1]\times[0,1] \to X$ is such that
$\tilde{H}(0,s) = f(H(0,s)) = f(x_0)$ and
$\tilde{H}(1,s) = f(H(1,s)) = f(x_0)$ for $0\leq s \leq 1$, and
$\tilde{H}(t,0) = f(H(t,0)) = (f\circ \sigma_0)(t)$ and
$\tilde{H}(t,1) = f(H(t,1)) = (f\circ \sigma_1)(t)$ for $0\leq t \leq 1$.
Thus $f\circ \sigma_0 \dotsim f\circ \sigma_1$.

Hence $f_\ast([\sigma])$ is
independent of the representative path of $[\sigma]$ selected.

\stage{ii} The map $f_\ast$ is an homomorphism because
\begin{align*}
f_\ast([\alpha_0][\alpha_1]) &= f_\ast([\alpha_0\,\alpha_1])
= [f\circ (\alpha_0\,\alpha_1)] = [(f\circ \alpha_0)(f\circ\alpha_1)] \\
&= [f\circ \alpha_0][f\circ\alpha_1]
= f_\ast([\alpha_0]) f_\ast([\alpha_1])
\end{align*}
for all loops $\sigma_0$ and $\sigma_1$ in $X$ at $x_0$.
\end{proof}

Suppose that $X$, $Y$ and $Z$ are three path-connected topological
spaces, that $f:X \to Y$ and $g:Y\to Z$ are continuous functions, and
that $x_0 \in X$.  Then it is easy to prove that
$g_\ast \circ f_\ast = (g\circ f)_\ast: \pi_1(X,x_0) \to \pi_1(Z,g(f(x_0)))$.

\begin{prop}
Suppose that $f:X \to Y$ is a homeomorphism between two
path-connected topological spaces and that $x_0 \in X$.  Then the map
$f_\ast:\pi_1(X,x_0) \to \pi_1(Y,f(x_0))$ is an isomorphism.
\end{prop}

\begin{proof}
It follows from $\displaystyle f^{-1} \circ f = \Id_X$ that
$\displaystyle (f^{-1})_\ast \circ f_\ast = (f^{-1} \circ f)_\ast = (\Id_X)_\ast
= \Id_{\pi_1(X,x_0)}$ and
from $\displaystyle f \circ f^{-1} = \Id_Y$ that
$\displaystyle f_\ast \circ (f^{-1})_\ast = (f \circ f^{-1})_\ast =
(\Id_Y)_\ast = \Id_{\pi_1(Y,f(x_0))}$.  Thus $f_\ast$ has an inverse given by
$\displaystyle (f_\ast)^{-1} = (f^{-1})_\ast$.
\end{proof}

\begin{prop}
Suppose that $X$ is a path-connected topological space and that
$x_0 ,x_1 \in X$.  Then $\pi_1(X,x_0) \cong \pi_1(X,x_1)$.
\end{prop}

\begin{proof}
Since $X$ is path-connected, there exits a path $\beta$ in $X$ from
$x_0$ to $x_1$.
Consider $\beta_\sharp : \pi_1(X,x_0) \to \pi_1(X,x_1)$ defined by
$\displaystyle \beta_\sharp([\sigma]) = [\beta^{-1} \sigma \beta]$ for all
$\sigma \in \pi_1(X,x_0)$.

\stage{i}  We first note that $\beta_\sharp$ is well defined.
If $\sigma_0$ and $\sigma_0$ are two loops in $X$ at $x_0$ such that
$\sigma_0 \dotsim \sigma_1$, then there exists
$H:[0,1]\times[0,1] \to X$ such that $H(0,s) = H(1,s) = x_0$
for $0\leq s \leq 1$, and $H(t,0) = \sigma_0(t)$ and
$H(t,1) = \sigma_1(t)$ for $0\leq t \leq 1$.

We have that $\displaystyle \beta^{-1} \sigma_0 \beta$ and
$\displaystyle \beta^{-1}\sigma_1 \beta$ are two loops in $X$ at $x_1$.
Let $\tilde{H}:[0,1]\times[0,1] \to X$ be the function defined
by
\[
\tilde{H}(t,s) = \begin{cases}
\beta^{-1}(t) & \quad \text{if} \ 0 \leq t \leq 1/4
\ \text{and} \ 0\leq s \leq 1 \\
H(4t -1,s) & \quad \text{if} \ 1/4 < t < 1/2
\ \text{and} \ 0\leq s \leq 1 \\
\beta(t) & \quad \text{if} \ 1/2 \leq t \leq 1
\ \text{and} \ 0\leq s \leq 1
\end{cases}
\]
Then $\displaystyle \tilde{H}(0,s) = \beta^{-1}(0) = x_1$ and
$\tilde{H}(1,s) = \beta(1) = x_1$ for $0\leq s \leq 1$, and
\begin{align*}
\tilde{H}(t,0) &= \begin{cases}
\beta^{-1}(t) & \quad 0 \leq t \leq 1/4 \\
\sigma_0(4t -1) & \quad \text{if} \ 1/4 < t < 1/2 \\
\beta(t) & \quad \text{if} \ 1/2 \leq t \leq 1
\end{cases} \\
&= (\beta^{-1} \sigma_0\beta)(t)
\end{align*}
and
\begin{align*}
\tilde{H}(t,1) &= \begin{cases}
\beta^{-1}(t) & \quad 0 \leq t \leq 1/4 \\
\sigma_1(4t -1) & \quad \text{if} \ 1/4 < t < 1/2 \\
\beta(t) & \quad \text{if} \ 1/2 \leq t \leq 1
\end{cases} \\
&= (\beta^{-1} \sigma_1\beta)(t)
\end{align*}
for $0\leq t \leq 1$.
Thus
$\displaystyle \beta^{-1} \sigma_0 \beta \dotsim \beta^{-1}\sigma_1 \beta$.

\stage{ii} The map $\beta_\sharp$ is a homomorphism because
\begin{align*}
\beta_\sharp([\sigma_0][\sigma_1])
&= \beta_\sharp([\sigma_0\sigma_1])
= [\beta^{-1} \sigma_0\, \sigma_1 \beta]
= [\beta^{-1} \sigma_0 \, \underbrace{\beta \beta^{-1}}_{\dotsim e_{x_0}} \,
\sigma_1 \beta] \\
&= [\beta^{-1} \sigma_0 \beta]\,[\beta^{-1}\sigma_1 \beta]
= \beta_\sharp([\sigma_0])\beta_\sharp([\sigma_1])
\end{align*}
for all loops $\sigma_0$ and $\sigma_1$ in $X$ at $x_0$.

\stage{iii} The map $\beta_\sharp$ is an isomorphism.  In fact, it is
easy to verify that $\displaystyle \beta_\sharp^{-1} = (\beta^{-1})_\sharp$.
\end{proof}

If $X$ is a path-connected topological space, it follows from the
previous proposition that the fundamental group of a pair $(X,x_0)$ is,
up to isomorphism, the same group independently of the base point
$x_0 \in X$ chosen.  Hence, the following definition is justified.

\begin{defn}
Let $X$ be a path-connected topological space.  The
{\bfseries fundamental group}\index{Fundamental Group}
of $X$, denoted $\pi_1(X)$, is the group $\pi_1(X,x_0)$ for any
base point $x_0 \in X$.
\end{defn}

\subsection{Covering Spaces}

In this section, we prove only the results that will be used later.
The missing proofs and more information about covering spaces can be
found in \cite{GH,ST}.

\begin{defn}
A topological space $X$ is locally path-connected if for every
$x \in X$ and open neighbourhood $V \subset X$ of $x$, there exists an
open neighbourhood $U \subset V$ such that $U$ is path-connected.
\end{defn}

We insist on the fact that $U$ is path-connected in the
previous definition.  Namely, given any $x_1, x_2 \in X$, there exists
a path $\alpha$ in $U$ from $x_1$ to $x_2$.  The fact that
$\alpha(t) \in U$ for $0\leq t \leq 1$ is crucial.  A topological
space could be path-connected without being locally path connected as
can be seen in the following figure.
\pdfbox{alg_top/lpathc}
The set $X$ is a curve in $\displaystyle \RR^2$ with the induced
topology from $\displaystyle \RR^2$.  There is no open neighbourhood
of $\VEC{x}$ that is path-connected. 

The notion of locally path-connected may be new to the reader but it
is going to be essential for the next important concept.

\begin{defn}  \label{defnCovering}
Let $X$ and $Q$ be two path-connected and locally
path-connected topological spaces.  Suppose that $p:Q \to X$
is a continuous and surjective function such that, for all $x \in X$,
there exists an open neighbourhood $U \subset X$ of $x$ satisfying the
following conditions:
\begin{enumerate}
\item $\displaystyle p^{-1}(U) = \bigcup_{\tau \in T} V_\tau$ with
$V_\tau\subset Q$ open sets and $V_{\tau_1} \cap V_{\tau_2} = \emptyset$
for $\tau_1 \neq \tau_2$, and
\item $p\big|_{V_\tau}:V_\tau \to U$ is a homeomorphism for all $\tau$.
\end{enumerate}
Then $(Q,p)$ is called a {\bfseries covering}\index{Covering}
of $X$ (Figure \ref{FundGr2}).
\end{defn}

\pdfF{alg_top/fundgr2}{Covering of $S^1$}{Two illustrations of the covering
of $\displaystyle S^1$.  In the lower illustration, if we assume that
$\displaystyle S^1$ is represented by a circle of radius $1$ centred
at the origin, $p$ is given by $\displaystyle p(t) = e^{2\pi t i}$ for
$t \in \RR$.}{FundGr2}

\begin{prop}
If $(Q,p)$ is a covering of a topological space $X$, then $p:Q \to X$
is an open mapping.
\end{prop}

\begin{proof}
Let $W$ be an open subset of $Q$.  Given $x = p(W)$, there exists an
open neighbourhood $U \subset X$ of $x$ satisfying
Definition~\ref{defnCovering}.  Choose $y \in W$ such that $p(y) = x$.
Then $y \in V_\tau$ for some $\tau$.  Since $W \cap V_\tau$ is an open subset
of $V_\tau$ and $p\big|_{V_\tau}:V_\tau \to U$ is a homeomorphism, we have that
$B = p(W\cap V_\tau)$ is an open neighbourhood of $x$ in $U$ and also in $X$
because $U$ is open.  Thus $B$ is an open neighbourhood of $x$ with
$B \subset p(W)$.  Since $x \in p(W)$ is arbitrary, this proves that
$p(W)$ is an open subset of $X$.
\end{proof}

\begin{prop} \label{propUnLift}
Let $(Q,p)$ be a covering of a topological space $X$.
If $\alpha, \beta : Y \to Q$ are two continuous functions defined
on a connected space $Y$ such that
$p \circ \alpha = p \circ \beta$ on $Y$ and $\alpha(\tilde{y})
= \beta(\tilde{y})$ for a point $\tilde{y} \in Y$, then $\alpha = \beta$.  
\end{prop}

\begin{proof}
Let $h:Y \to Q \times Q$ be the function defined by
$h(y) =(\alpha(y),\beta(y))$ for $y \in Y$.  We have that $h$ is a
continuous function.

Let $D = \{(q,q) : q \in Q\}$.  Since $D$ is a closed set
\footnote{In fact $D$ is closed if and only if $Q$ is hausdorff.  One
part of the proof consists in proving that $Q$ Hausdorff implies that
$(Q\times Q) \setminus D$ is open.  The proof of the other direction
is even more direct.  See \cite{Du}.}, we have that
$\displaystyle Z = \{ \VEC{y} \in Y : \alpha(y)
= \beta(y) \} = h^{-1}(D)$ is closed.
We also have that $Z \neq \emptyset$ because $\tilde{y} \in Z$.  Since
$Y$ is connected, it suffices to prove that $Z$ is also open to
conclude that $Z=Y$.

To prove that $Z$ is open, choose $\breve{y} \in Z$ and let
$x = p(\alpha(\breve{y})) = p(\beta(\breve{y}))$.  There exists an
open neighbourhood $U \subset X$ of $x$ satisfying
Definition~\ref{defnCovering}.
Then $\alpha(\breve{y}) = \beta(\breve{y}) \in V_\tau$ for some $\tau$.
Since $\alpha$ is continuous, there exists an open neighbourhood
$\displaystyle W_1 \subset \alpha^{-1}(V_\tau)$ of $\breve{y}$.
Similarly, since $\beta$ is continuous, there exists an open neighbourhood
$\displaystyle W_2 \subset \beta^{-1}(V_\tau)$ of $\breve{y}$.
Since $p\big|_{V_\tau}:V_\tau \to U$ is one-to-one and
$p \circ \alpha = p \circ \beta$ on $W_1 \cap W_2$, we get that
$\alpha = \beta$ on the open set $W_1 \cap W_2$.  Thus
$W_1 \cap W_2 \subset Z$.
Since $\breve{y} \in Z$ is arbitrary, this proves that $Z$ is open.
\end{proof}

The following theorem has deep consequences.

\begin{theorem}[Covering Homotopy Theorem]
Suppose that $(Q,p)$ is a covering of a topological space $X$
and that $Y$ is a compact and connected space.
Moreover, suppose that $f,g:Y \to Q$ are two continuous
functions and that $F:Y \times [0,1] \to X$ is a homotopy between 
$p\circ f$ and $p \circ g$.  Then there exists a homotopy
$G:Y \times [0,1]\to Q$ between $f$ and $g$ satisfying the
following conditions.
\begin{enumerate}
\item $p \circ G = F$ on $Y \times [0,1]$.
\item If there exists $y \in Y$ and an interval $E \subset [0,1]$
such that $F(y,s)$ is constant for $s \in E$, then
$G(y,s)$ is constant for $s \in E$.
\end{enumerate}
\end{theorem}

\begin{proof}
\stage{i} The set $Y \times [0,1]$ is compact because $Y$ and $[0,1]$
are compact.  For each $x \in X$, let $\displaystyle U_x \subset X$
be an open neighbourhood of $x$ satisfying Definition~\ref{defnCovering}.
Since the collection $\displaystyle \{ U_x \}_{x\in X}$
is an open cover of $F(Y\times[0,1])$, we have that the collection
$\displaystyle \big\{ F^{-1}(U_x) \big\}_{x\in X}$
is an open cover of the compact set $Y\times[0,1]$.  Hence, there
exists a finite subcover
$\displaystyle \big\{ F^{-1}(U_{x_j}) \big\}_{1 \leq j \leq J}$
of $Y\times [0,1]$.

We can cover $Y \times [0,1]$ with open sets of the form
$W_\alpha \times I_\alpha$ where $W_\alpha \subset Y$ is a connected
open set, $I_\alpha$ is an open interval, and
$\displaystyle W_\alpha \times I_\alpha \subset F^{-1}(U_{x_j})$ for some
$j \in \{1,2,\ldots, J\}$.  Again, because $Y \times [0,1]$ is
compact, there exists a finite subcover
$\displaystyle \{ W_{\alpha_k} \times I_{\alpha_k} \}_{1\leq k \leq K}$
of $Y \times [0,1]$.  By splitting the intervals $I_{\alpha_k}$, we
get a finite subcover of $Y \times [0,1]$ of the form
$\displaystyle \{ W_{\alpha_k} \times [s_{i-1}, s_i]
\}_{1\leq k \leq K,1\leq i \leq I}$ where
$s_0 =0 < s_1 < s_2 < \ldots < s_I = 1$.  We still have that
$\displaystyle F\big(W_{\alpha_k} \times [s_{i-1}, s_i]\big) \subset
U_{x_{j(k,i)}}$ for some $j(k,i) \in \{1,2,\ldots, J\}$.

\stage{ii}  We define continuous function
$G_i:Y \times [s_{i-1},s_i] \to Q$ such that
\[
p \circ G_i = F\Big|_{Y \times [s_{i-1},s_i]}\quad  \text{and} \quad
G_i\big|_{Y \times \{s_i\}} = G_{i-1}\big|_{Y \times \{s_i\}}
\]
for $1 \leq i \leq I$.
Because of the second condition, we will have that
$G:Y \times [0,1] \to Q$ defined by
$G(y,s) = G_i(y,s)$ for $y\in Y$ and $t \in [s_{i-1},s_i]$ is a well
defined continuous function satisfying (1) in the statement of the
theorem.

\stage{iii}
Assume that $1 \leq i \leq I$.  Let
\[
G_i(y,s_{i-1}) =
\begin{cases}
f(y) & \quad \text{if}\ i = 1 \\
G_{i-1}(y,s_{i-1}) & \quad \text{if} \ 1 < i \leq I
\end{cases}
\]
for $y \in Y$.  Since $W_{\alpha_k}$ is connected, we have that
\[
G_i(W_{\alpha_k},s_{i-1})
= \begin{cases}
f(W_{\alpha_k}) & \quad \text{if}\ i = 1 \\
G_{i-1}(W_{\alpha_k},s_{i-1}) & \quad \text{if} \ 1 < i \leq I
\end{cases}
\]
is connected.  Since
$F(W_{\alpha_k},[s_{i-1},s_i]) \subset U_{x_{j(k,i)}}$ for some
$j(k,i) \in \{1,2,\ldots,J\}$ and
\begin{align*}
p \circ G_i\big|_{W_{\alpha_k} \times \{s_{i-1}\}} &=
\begin{cases}
p \circ f\big|_{W_{\alpha_k}} & \quad \text{if} \ i = 0 \\
p \circ G_{i-1}\big|_{W_{\alpha_k} \times \{s_{i-1}\}} &
\quad \text{if} \ 1 < i \leq I
\end{cases} \\
&= F\big|_{W_{\alpha_k} \times\{s_{i-1}\}} \ ,
\end{align*}
we have that
$G_i(W_{\alpha_k},s_{i-1}) \subset V_{x_{j(k,i)},\tau}$ for some
$\tau \in T$, where
$\displaystyle p^{-1}(U_{x_{j(k,i)}})
= \bigcup_{\tau \in T} V_{x_{j(k,i)},\tau}$ with
$V_{x_{j(k,i)},\tau_1} \cap V_{x_{j(k,i)},\tau_2} = \emptyset$ if
$\tau_1 \neq \tau_2$
as given in Definition~\ref{defnCovering}.  More precisely,
$G_i(W_{\alpha_k},s_{i-1})$ is a connected subset
of $\displaystyle p^{-1}(U_{x_{j(k,i)}})
= \bigcup_{\tau \in T} V_{x_{j(k,i)},\tau}$
where the $V_{x_{j(k,i)},\tau}$ are distinct open sets.  Therefore
$G_i(W_{\alpha_k},s_{i-1})$ is a subset of only one of them.

Since $p\big|_{V_{x_{j(k,i)},\tau}} :V_{x_{j(k,i)},\tau} \to U_{x_{j(k,i)}}$ is a
homeomorphism, we may set\\
$\displaystyle G_{i,k}(y,s) = p^{-1}(F(y,s))$ for
$(y,s) \in W_{\alpha_k} \times [s_{i-1},s_i]$.
We define $G_i:Y \times [s_{i-1},s_i] \to Q$ by
$G_i(y,s) = G_{i,k}(y,s)$ if
$(y,s) \in W_{\alpha_k} \times [s_{i-1},s_i]$.  To prove that $G_i$
is well defined and continuous on $Y \times [s_{i-1},s_i]$, we
prove in the next step that
$G_{i,k_1}(y,s) = G_{i,k_2}(y,s)$ for
$(y,s) \in (W_{\alpha_{k_1}} \cap W_{\alpha_{k_2}}) \times [s_{i-1},s_i]$.

\stage{iv}  Suppose that
$W_{\alpha_{k_1}} \cap W_{\alpha_{k_2}} \neq \emptyset$.  We have by
definition of $G_{i,k}$ that
\[
G_{i,k_m}(W_{\alpha_{k_m}},[s_{i-1},s_i]) \subset
V_{x_{j(k_m,i)},\tau_m}
\]
for some $j(k_m,i) \in \{1,2,\ldots,J\}$ and $\tau_m \in T$ with
$1\leq m \leq 2$.  We also have
\[
G_{i,k_1}(y,s_{i-1}) = G_{i,k_2}(y,s_{i-1})
= \begin{cases}
f(y) & \quad \text{if} \ i = 1 \\
G_{i-1}(y,s_{i-1}) & \quad \text{if} \ 1 < i \leq I
\end{cases}
\]
for all $(y,s) \in (W_{\alpha_{k_1}} \cap W_{\alpha_{k_2}})
\times [s_{i-1},s_i]$.  Thus
$G_{i,k_1}(y,s_{i-1}) = G_{i,k_2}(y,s_{i-1}) \in 
V_{x_{j(k_1,i)},\tau_1} \cap V_{x_{j(k_2,i)},\tau_2}$ for all
$(y,s) \in (W_{\alpha_{k_1}} \cap W_{\alpha_{k_2}}) \times [s_{i-1},s_i]$.

Consider $(y,s) \in (W_{\alpha_{k_1}} \cap W_{\alpha_{k_2}})
\times [s_{i-1},s_i]$ and the two functions
\begin{align*}
\sigma_1:[0,1] &\to Q \\
t & \mapsto G_{i,k_1}(y,t s + (1-t) s_{i-1})
\end{align*}
and
\begin{align*}
\sigma_2:[0,1] &\to Q \\
t & \mapsto G_{i,k_2}(y,t s + (1-t) s_{i-1})
\end{align*}
We have that $\sigma_1(0) = \sigma_2(0)$ and
\begin{align*}
(p\circ \sigma_1)(t) &= (p \circ G_{i,k_1}(y,t s + (1-t) s_{i-1})
= F(y, t s + (1-t) s_{i-1}) \\
&= (p \circ G_{i,k_2}(y, t s + (1-t) s_{i-1})
= (p\circ \sigma_2)(t)
\end{align*}
for $0 \leq t \leq 1$.  It follows from Proposition~\ref{propUnLift}
that $\sigma_1(t) = \sigma_2(t)$ for $0\leq t \leq 1$.  Thus
$G_{i,k_1}(y,s) = \sigma_1(1) = \sigma_2(1) = G_{i,k_2}(y,s)$.

\stage{v} Repeating recursively (iii) and (iv) from $i=1$ to $i=I$
yields the functions $G_i$ in (ii).

\stage{vi} Recall that $G_{i,k}$ is defined by
$\displaystyle G_{i,k}(y,s) = p^{-1}(F(y,s))$ for
$(y,s) \in W_{\alpha_k} \times [s_{i-1},s_i]$ where
$p\big|_{V_{x_{j(k,i)},\tau}} :V_{x_{j(k,i)},\tau} \to U_{x_{j(k,i)}}$ is a
homeomorphism.  Therefore $G_{i,k}(y,s)$ is constant with respect to
$s$ if $F(y,s)$ is constant with respect to $s$ because
$p\big|_{V_{x_{j(k,i)},\tau}} :V_{x_{j(k,i)},\tau} \to U_{x_{j(k,i)}}$
is one-to-one.
\end{proof}

It follows from Prop~\ref{propUnLift} that if $(Q,p)$ is a covering of
a topological space $X$ and $q \in Q$, then
$p_\ast: \pi_1(Q,q) \to \pi_1(X, p(q))$ is one-to-one.  If $\alpha$ is
a path in $X$ such that $\alpha(0) = p(q)$, then there exists a unique
path $\beta$ in $Q$ such that $p \circ \beta = \alpha$ and $\beta(0) = q$.

There are several interesting and important corollaries to this
theorem.  For instance, if $x \in X$ and $x = p(q)$, there exists a
one-to-one map from the quotient group\\
$\displaystyle \pi_1(X,x) / p_\ast(\pi_1(Q,q))$
onto $\displaystyle p^{-1}(\{x\})$.  To prove this result, we first
construct a map $M$ from $\displaystyle \pi_1(X,x)$ onto
$\displaystyle p^{-1}(\{x\})$ as it follows.  Given
$[\alpha] \in \pi_1(X,x)$, we have from the result stated in the previous 
paragraph that there exists a unique path $\beta$ in $Q$ such that
$p \circ \beta = \alpha$ with $\beta(0) = q$.  We set
$M([\alpha]) = \beta(1)$.  The rest of the proof consists in proving
that the definition of $M$ is independent of the representative $\alpha$
chosen, that $M$ is onto and that the kernel of $M$ is
$p_\ast(\pi_1(Q,q))$.

\begin{defn}
Let $X$ be a path-connected and locally path-connected topological space.
We say that $X$ is {\bfseries simply connected}\index{Simply Connected}
if $\pi_1(X)$, the fundamental group of $X$, is trivial.
\end{defn}

It follows that if $X$ is simply connected and $x \in X$, then
$\alpha \dotsim e_x$ for all loops $\alpha$ in $X$ at $x$, where $e_x$ is the
path defined by $e_x(t) = x$ for $0 \leq t \leq 1$.

\begin{defn}
A topological space $X$ is
{\bfseries locally simply connected}\index{Locally Simply Connected}
if for every $x \in X$ there exists an open neighbourhood $U$ of $x$ such
$\alpha \dotsim[U] e_x$ for all loops $\alpha$ in $U$ at $x$.
\end{defn}

We insist on the fact that the range of the homotopy in the previous
definition is only $U$.

\begin{defn}
Let $(Q,p)$ be a covering of a topological space $X$.  We say
that $(Q,p)$ is a {\bfseries universal covering}\index{Universal Covering}
of $X$ if $Q$ is simply connected
\end{defn}

\begin{theorem}  \label{thCoveringEx}
Let $X$ be a path-connected, locally path-connected and locally simply
connected topological space.  Given $x \in X$, let $H$ be a subgroup
of $\pi_1(X,x)$.  Then there exists a covering $(Q,p)$ of $X$
such that $p_\ast(\pi_1(Q,q)) = H$ for $q \in Q$ such that $p(q) = x$.
\end{theorem}

In particular, if $H = \{[e_x]\}$ for $x \in X$, we get a universal
covering $(Q,p)$ of $X$ because, as we stated before,
$p_\ast:\pi_1(Q,q) \to \pi_1(X,p(q))$ is one-to-one.

It can be proved that if $(Q,p)$ is a covering of
a topological space $X$ and $x \in X$, then
$p_\ast(\pi_1(Q,q_1))$ and $p_\ast(\pi_1(Q,q_2))$ with
$p(q_1)= p(q_2) = x$ are two conjugate subgroups in
$\pi_1(X,x)$; namely, there exists $[\beta] \in \pi_1(X,x)$ such that
$\displaystyle [\beta]\, p_\ast(p_1(Q,q_1))\, [\beta]^{-1}= p_\ast(p_1(Q,q_2))$.
In fact, if $\gamma$ is a path in $Q$ from $q_1$ to $q_2$,
then $\beta = p \circ \gamma$.

\begin{defn}
Two coverings $(Q_1,p_1)$ and $(Q_1,p_1)$ of a
topological space $X$ are {\bfseries isomorphic}\index{Isomorphic Coverings}
if there exists a homeomorphism $h:Q_1 \to Q_2$ such that $p_2 \circ h = p_1$. 
\end{defn}

Another result that can relatively easily be proved states that
if $(Q_1,p_1)$ and $(Q_2,p_2)$ are two coverings of a locally simply
connected space $X$ such that $p_\ast(\pi_1(Q_1,q_1)) = p_\ast(\pi_1(Q_2,q_2))$
for any $q_1 \in Q_1$ and $q_2 \in Q_2$ with
$p(q_1) = p(q_2)$, then $(Q_1,p_1)$ and $(Q_2,p_2)$ are isomorphic coverings.

\begin{defn}
Let $(Q,p)$ be a covering of a topological space $X$.
A {\bfseries covering transformation}\index{Covering Transformation}
of $(Q,p)$ is a homeomorphism $h:Q \to Q$ such that $p \circ h = p$.
\end{defn}

The set of covering transformations of $(Q,p)$ form a group
under composition of functions.  This group is denoted
${\cal G}(Q,p)$.

If $U$ is in open set as in the definition of covering,
Definition~\ref{defnCovering}, then a covering transformation $h$
permutes the open sets $V_j$.

\begin{defn}
A {\bfseries regular covering}\index{Regular Covering} of a
topological space $X$ is a covering $(Q,p)$ of $X$ such that
$p_\ast(\pi_1(Q,q))$ is a normal subgroup of $\pi_1(X,p(q))$ for
$q \in Q$; namely,\\
$\displaystyle [\beta]\, p_\ast(\pi_1(Q,q))\, [\beta]^{-1} =
p_\ast(\pi_1(Q,q))$ for all $[\beta] \in \pi_1(X,p(q))$.
\end{defn}

It follows from the paragraph following Theorem~\ref{thCoveringEx}
that the definition of regularity is independent of the point
$q \in Q$ selected.

The next theorem is the fundamental theorem of this subsection.

\begin{theorem}
Suppose that $(Q,p)$ is a regular covering of a locally simply
connected space $X$ and that $q \in Q$.
Then ${\cal G}(Q,p)$ is group isomorphic
to $\displaystyle \pi_1(X,p(q)) / p_\ast (\pi_1(Q,q))$.
In particular, if the covering is also universal, then
${\cal G}(Q,p)$ is group isomorphic to $\pi_1(X,p(q))$.
\end{theorem}

\begin{prop} \label{propFGSone}
$\displaystyle \pi_1(S^1,1) \cong \ZZ$.  In fact, since
$\displaystyle S^1$ is path-connected, we have that
$\displaystyle \pi_1(S^1) \cong \ZZ$.
\end{prop}

\begin{proof}
In the statement of the theorem, we assume the complex representation
of $\displaystyle S^1$; namely,
$\displaystyle S^1 = \{z \in \CC : |z| = 1 \}$.

Suppose that $(Q,p)$ is the covering described in
Figure~\ref{FundGr2}.  We have that $Q = \RR$ and $(Q,p)$ is an
universal and regular covering of $\displaystyle S^1$.
If $h:\RR \to \RR$ is a covering transformation, 
then $p \circ h = p$ implies that
\[
e^{2\pi h(q) i} = e^{2\pi q i} \iff e^{2\pi (h(q)-q) i} = 1
\]
for all $q \in \RR$.  Thus $h(q)-q$ is a continuous function defined
on $\RR$, a connected set, whose range is $\ZZ$.  This is possible
only if there exists $n\in \ZZ$ such that $h(q) - q = n$ for all
$q \in \RR$.  This proves that the covering transformations are all
given by integer translations.  Hence ${\cal G}(Q,p) \cong \ZZ$ and
from the previous theorem
$\displaystyle \pi_1(S^1) \cong {\cal G}(Q,p) \cong \ZZ$.
\end{proof}

\begin{prop}
Suppose that $X$ and $Y$ are path-connected topological spaces, and that
$x_0 \in X$ and $y_0 \in Y$.  Then
$\pi_1(X\times Y, (x_0,y_0)) \cong \pi_1(X,x_0) \times \pi_1(Y,y_0)$
\end{prop}

\begin{proof}
It is easy to verify that the two projections
$P:X\times Y \to X$ and $Q:X\times Y \to Y$ induces two homomorphism
$P_\ast: \pi_1(X\times Y, (x_0,y_0)) \to \pi_1(X,x_0)$
defined by $P_\ast([\sigma]) = [P \circ \sigma]$
and $Q_\ast:\pi_1(X\times Y, (x_0,y_0)) \to \pi_1(Y,y_0)$
defined by $Q_\ast([\sigma]) = [Q \circ \sigma]$
for $[\sigma] \in \pi_1(X\times Y, (x_0,y_0))$.
Thus $(P_\ast,Q_\ast): \pi_1(X\times Y, (x_0,y_0)) \to
\pi_1(X,x_0) \times \pi_1(Y,y_0)$ defined by
$(P_\ast,Q_\ast)([\sigma]) = (P_\ast([\sigma]),Q_\ast([\sigma]))$
for $[\sigma] \in \pi_1(X\times Y, (x_0,y_0))$ is an homomorphism.

The map $(P_\ast,Q_\ast)$ is also an isomorphism because it has an
inverse.  If $\sigma_X$ is a loop in $X$ at $x_0$ and
$\sigma_Y$ is a loop in $Y$ at $y_0$, then $(\sigma_X,\sigma_Y)$
is a loop in $X \times Y$ at $(x_0,y_0)$.
Let $R: \pi_1(X,x_0) \times \pi_1(Y,y_0) \to \pi_1(X\times Y, (x_0,y_0))$
be the map defined by
$\displaystyle R([\sigma_X],[\sigma_Y]) = [(\sigma_X,\sigma_Y)]$ 
for $[\sigma_X] \in \pi_1(X,x_0)$ and
$[\sigma_Y] \in \pi_1(Y,y_0)$.
The map $R$ is well defined because
$\sigma_X \dotsim \tilde{\sigma}_X$ and
$\sigma_Y \dotsim \tilde{\sigma}_Y$ imply
$(\sigma_X,\sigma_Y) \dotsim (\tilde{\sigma}_X,\tilde{\sigma}_Y)$
\footnote{We leave it to the reader to construction the map
$H:[0,1]\times[0,1] \to X \times Y$ needed to prove that
$(\sigma_X,\sigma_Y) \dotsim (\tilde{\sigma}_X,\tilde{\sigma}_Y)$.}.
Since all loops $\sigma$ in $X \times Y$ at $(x_0,y_0)$ are of the form
$\sigma = (\sigma_X,\sigma_Y)$ where $\sigma_X = P\circ \sigma$
is a loop in $X$ at $x_0$ and $\sigma_Y = Q\circ \sigma$ is a loop
in $Y$ at $y_0$,  we have that $R$ is onto.  The map $R$ is also
one-to-one because 
$\sigma \dotsim \tilde{\sigma}$ in $X \times Y$ implies that
$P\circ \sigma \dotsim P\circ \tilde{\sigma}$ in $X$ and
$Q\circ \sigma \dotsim Q\circ \tilde{\sigma}_Y$ in $Y$
\footnote{Again, we leave it to the reader to construction the map
$H_P:[0,1]\times[0,1] \to X$ and $H_Q:[0,1]\times[0,1] \to Y$
needed to prove that
$P\circ \sigma \dotsim P\circ \tilde{\sigma}$ in $X$ and
$Q\circ \sigma \dotsim Q\circ \tilde{\sigma}_Y$ in $Y$.}.

It is now easy to verify that
$\displaystyle R\circ (P_\ast,Q_\ast) = \Id_{\pi_1(X\times Y, (x_0,y_0))}$ and \\
$\displaystyle (P_\ast,Q_\ast) \circ R = \Id_{\pi_1(X, x_0) \times \pi_1(Y,y_0)}$.
\end{proof}

\begin{egg}
The torus $\displaystyle \torus{2} \subset \RR^3$         \label{eggFGtorus}
is isomorphic to $\displaystyle S^1 \times S^1$.
This can be seen from the parametric representation of the torus.
Hence, it follows from the previous proposition that
$\displaystyle \pi_1(\torus{2}) \cong \pi_1(S^1) \times \pi_1(S^1) \cong
\ZZ \times \ZZ$ since $\torus{2}$ is path-connected.
\end{egg}

To conclude this section, we plan to prove a limited version of the
famous Brouwer fixed point theorem.  Namely, a version limited to
$\displaystyle \RR^2$.  We first need a new definition and a lemma.

\begin{defn} \label{defnRetract}
Let $Y$ be a subset of a topological space $X$.
A {\bfseries retraction}\index{Retraction}
of $X$ onto $Y$ is a continuous function $r:X \to Y$ such that $r(x) = x$
for all $x \in Y$ or, equivalently, $r \circ \iota = \Id_Y$ for the
inclusion map $\iota :Y \to X$.  We then say that $Y$ is a
{\bfseries retract}\index{Retract} of $X$.
\end{defn}

\begin{lemma}  \label{lenNoRetract}
Let $\displaystyle D = \{ \VEC{x} \in \RR^2 : \|\VEC{x}\| \leq 1 \}$.
There does not exist any retraction of $D$ onto
$\displaystyle S^1 = \{\VEC{x} \in \RR^2 : \|\VEC{x}\|=1 \}$.
\end{lemma}

\begin{proof}
Suppose that there is a retraction $\displaystyle g:D \to S^1$.
Let $\displaystyle \iota:S^1 \to D$ be the inclusion map.  
Then $g \circ \iota = \Id_{S^1}$ and
\begin{equation} \label{lemBrouwerEq1}
g_\ast \circ \iota_\ast = (g \circ \iota)_\ast
= \Id_{\pi_1(S^1,\VEC{e}_1)} \ .
\end{equation}
However, $\pi_1(D,\VEC{e}_1) = \{[e_{\VEC{e}_1}]\}$ where
$e_{\VEC{e}_1}(t) = \VEC{e}_1$ for $0 \leq t \leq 1$
because $D$ is contractible.  Thus
$\displaystyle \iota_\ast(\pi_1(S^1,\VEC{e}_1)) \subset \pi_1(D,\VEC{e}_1) =
\{[e_{\VEC{e}_1}]\}$ implies that
$\displaystyle (g_\ast \circ \iota_\ast)(\pi_1(S^1,\VEC{e}_1))
= \{[e_{\VEC{e}_1}]\}$.  This is a contradiction of
Proposition~\ref{propFGSone} because, according to
(\ref{lemBrouwerEq1}), the image of $g_\ast \circ \iota_\ast$ should
be $\displaystyle \pi_1(S^1,\VEC{e}_1) \cong \ZZ \not\cong
\{[e_{\VEC{e}_1}]\}$.
\end{proof}

\begin{theorem}[Brouwer Fixed Point Theorem]
Let $\displaystyle D = \{ \VEC{x} \in \RR^2 : \|\VEC{x}\| \leq 1 \}$.
Suppose that $f:D \to D$ is a continuous function.  Then there exists
$\VEC{y} \in D$ such that $f(\VEC{y}) = \VEC{y}$; namely, $f$ has
a {\bfseries fixed point}\index{Fixed Point} in $D$.
\end{theorem}

\begin{proof}
Suppose that there does not exist $\VEC{y} \in D$ such that
$f(\VEC{y}) = \VEC{y}$.
Let $\displaystyle g:D \to S^1$ be the function defined by
$\displaystyle g(\VEC{x})$ is the intersection with the circle
$\displaystyle S^1$ of the line through $\VEC{x}$ originating from
$f(\VEC{x})$.  Such a line is well defined because
$f(\VEC{x}) \neq \VEC{x}$ for all $\VEC{x} \in D$.  It
is also a continuous function because $f$ is continuous.
Moreover $g(\VEC{x}) = \VEC{x}$ for all $\displaystyle \VEC{x} \in S^1$.
Thus, $g$ is a retraction of $D$ onto $\displaystyle S^1$.
This is a contradiction of Lemma~\ref{lenNoRetract}.
\end{proof}

The proof of the Brouwer fixed point theorem is standard and can be
generalized to $\displaystyle \RR^{n+1}$ with $n>1$ using a group
structure on $\displaystyle S^n$ other than the fundamental group of
$\displaystyle S^n$ because
$\displaystyle \pi_1(S^n,\VEC{e}_1) = \{[ e_{\VEC{e}_1}]\}$ for $n>1$.

\begin{rmk}
It is possible to define higher order homotopy groups as it is
explained in \cite{GH}.  We however follow the tradition of instead
introducing singular homology as we will do in
Section~\ref{sectSingHom}.   As we will see in
Subsection~\ref{subsectP1EquH1} later, it is a natural way to
generalize the concept of fundamental group.
\end{rmk}

%%% Local Variables:
%%% mode: latex
%%% TeX-master: "notes"
%%% End:
