\chapter{Differential Forms}

\section{Multilinear Algebra}\label{stokesMLABRA}

Before undertaking the study of differential forms and the integration
of these forms, we need to introduce some concepts that are sometime
covered in books on linear algebra.

\begin{defn}
Let $H$ be a vector space over $\RR$.  A map
$\displaystyle \tau:H^k \to \RR$ is
{\bfseries multilinear}\index{Multilinear Mapping} if
\[
\tau(\VEC{v}_1,\VEC{v}_2, \ldots, a \VEC{v}_i + b \tilde{\VEC{v}}_i,
\ldots , \VEC{v}_k)
= a \tau(\VEC{v}_1,\VEC{v}_2, \ldots,\VEC{v}_i, \ldots , \VEC{v}_k)
+ b \tau(\VEC{v}_1,\VEC{v}_2, \ldots, \tilde{\VEC{v}}_i, \ldots , \VEC{v}_k)
\]
for all $a, b \in \RR$ and $0 \leq i \leq k$.
A {\bfseries $\mathbf{k}$-tensor}\index{Tensor!$k$-Tensor} on $H$ is a
multilinear map $\displaystyle \tau:H^k\to \RR$.  The set of all
$k$-tensors on $V$ is denoted $\displaystyle \T^k(H)$.
\end{defn}

The space $\displaystyle \T^k(H)$ is a vector space over $\RR$ where
the sum of vectors and the product by a scalar are defined by
\[
(a \tau_1+ b \tau_2)(\VEC{v}_1,\VEC{v}_2, \ldots,\VEC{v}_k) =
a \tau_1(\VEC{v}_1,\VEC{v}_2, \ldots,\VEC{v}_k)
+ b \tau_2(\VEC{v}_1,\VEC{v}_2, \ldots,\VEC{v}_k)
\]
for all $\displaystyle \tau_1, \tau_2 \in \T^k(H)$ and all $a, b \in \RR$.
It is also important to note that $\displaystyle \T^1(H) = H^\ast$,
the dual space of $H$.

\begin{defn}
The {\bfseries product of two tensors}\index{Tensor!Product}
$\displaystyle \tau_1 \in \T^{k_1}(H)$ and
$\displaystyle \tau_2 \in \T^{k_2}(H)$ is the tensor
$\displaystyle \tau_1\otimes \tau_2 \in \T^{k_1+k_2}(H)$ defined by
\[
(\tau_1\otimes \tau_2)(\VEC{v}_1,\VEC{v}_2, \ldots, \VEC{v}_{k_1+k_2})
= \tau_1(\VEC{v}_1,\VEC{v}_2, \ldots, \VEC{v}_{k_1})
\, \tau_1(\VEC{v}_{k_1+1},\VEC{v}_{k_1+2}, \ldots, \VEC{v}_{k_1+k_2})  \ .
\]
\end{defn}

\begin{theorem} \label{stokesBT}
Let $\displaystyle \B = \{ \VEC{v}_i \}_{i=1}^n$ be a basis of $H$
and let $\displaystyle \B^\ast = \{ \phi_i \}_{i=1}^n$ be
the dual basis associated to $\B$ \footnotemark.  Then the set
$\displaystyle
\MM = \{ \phi_{i_1} \otimes \phi_{i_2} \otimes \ldots \otimes \phi_{i_k} :
1 \leq i_m \leq n , \ 1 \leq m \leq k\}$
is a basis of $\displaystyle \T^k(H)$.
\end{theorem}

\footnotetext{Namely, $\phi_j(\VEC{v}_i) = \delta_{i,j}$
for all $i$ and $j$, where $\delta_{i,j}$ is the Kronecker delta
function.}

\begin{proof}
The proof is a standard proof in linear algebra.  We note that
\begin{align}
(\phi_{i_1} \otimes \phi_{i_2} \otimes \ldots \otimes \phi_{i_k})
( \VEC{v}_{j_1}, \VEC{v}_{j_2}, \ldots, \VEC{v}_{j_k})
&= \prod_{m=1}^k \phi_{i_m}(\VEC{v}_{j_m})
= \prod_{m=1}^k \delta_{i_m,j_m} \nonumber \\
&= \begin{cases}
1 & \quad \text{if} \ j_m = i_m \ \text{for} \ 1 \leq m \leq k \\
0 & \quad \text{otherwise}  
\end{cases} \label{stokesBTEq1}
\end{align}

\stage{i} We prove that any $\displaystyle \tau \in \T^k(H)$ can be expressed
as a linear combination of elements in $\MM$.

Given $\VEC{w}_j \in H$ for $1 \leq j \leq k$.  Since $\B$ is a basis
of $H$, we may express each $\VEC{w}_j$ as
$\displaystyle \VEC{w}_j = \sum_{m_j=1}^n a_{m_j,j} \VEC{v}_{m_j}$ for some
$a_{m_j,j} \in \RR$.  We get from (\ref{stokesBTEq1}) that
\begin{align*}
&(\phi_{i_1} \otimes \phi_{i_2} \otimes \ldots \otimes \phi_{i_k})
(\VEC{w}_1, \VEC{w}_2, \ldots, \VEC{w}_k) \\
&\ = \sum_{\substack{1 \leq m_j \leq n\\1\leq j \leq k}}
a_{m_1,1} a_{m_2,2} \cdots a_{m_k,k}
(\phi_{i_1} \otimes \phi_{i_2} \otimes \ldots \otimes \phi_{i_k})
(\VEC{v}_{m_1}, \VEC{v}_{m_2}, \ldots, \VEC{v}_{m_k})
= a_{i_1,1} a_{i_2,2} \cdots a_{i_k,k} \ .
\end{align*}
Hence
\begin{align*}
\tau(\VEC{w}_1, \VEC{w}_2, \ldots ,\VEC{w}_k)
&= \sum_{\substack{1 \leq i_m \leq n\\1\leq m \leq k}}
a_{i_1,1} a_{i_2,2} \cdots a_{i_k,k}
\tau (\VEC{v}_{i_1}, \VEC{v}_{i_2}, \ldots, \VEC{v}_{i_k}) \\
&= \sum_{\substack{1 \leq i_m \leq n\\1\leq m \leq k}}
\tau (\VEC{v}_{i_1}, \VEC{v}_{i_2}, \ldots, \VEC{v}_{i_k})
(\phi_{i_1} \otimes \phi_{i_2} \otimes \ldots \otimes \phi_{i_k})
(\VEC{w}_1, \VEC{w}_2, \ldots, \VEC{w}_k) \ .
\end{align*}
Since this is true for all $\VEC{w}_i \in H$ with $1 \leq i \leq k$,
we get
\[
\tau
= \sum_{\substack{1 \leq i_m \leq n\\1\leq m \leq k}}
\tau (\VEC{v}_{i_1}, \VEC{v}_{i_2}, \ldots, \VEC{v}_{i_k})
(\phi_{i_1} \otimes \phi_{i_2} \otimes \ldots \otimes \phi_{i_k}) \ .
\]

\stage{ii} We prove that the elements of $\MM$ are linearly
independent.  Suppose that
\[
\sum_{\substack{1 \leq i_m \leq n\\1\leq m \leq k}}
a_{i_1,i_2,\ldots,i_k} \,
\phi_{i_1} \otimes \phi_{i_2} \otimes \ldots \otimes \phi_{i_k}
= 0 \ .
\]
If we evaluate this expression at
$(\VEC{v}_{m_1}, \VEC{v}_{m_2}, \ldots, \VEC{v}_{m_k})$ and use
(\ref{stokesBTEq1}), then we get $a_{m_1,m_2,\ldots,m_k}=0$
for $1 \leq m_j \leq n$ and $1 \leq j \leq k$.
\end{proof}

\begin{defn}
An {\bfseries inner product}\index{Tensor!Inner Product} on $H$ is a
$2$-tensor $\displaystyle \tau \in \T^2(H)$ such that $\tau$ is
{\bfseries symmetric}\index{Tensor!Symmetric} and
{\bfseries positive definite}\index{Tensor!Positive Defined} \footnotemark.
\end{defn}

\footnotetext{The $2$-tensor $\tau$ is symmetric if
$\tau(\VEC{v}_1,\VEC{v}_2) = \tau(\VEC{v}_2,\VEC{v}_1)$ for all
$\VEC{v}_1$ and $\VEC{v}_2$ in $H$.  The $2$-tensor
$\tau$ is positive definite if $\tau(\VEC{v},\VEC{v})>0$ for all
$\VEC{v}\in H$ with $\VEC{v} \neq \VEC{0}$.}

This definition of the inner product corresponds to the standard
definition of inner product on a vector space.

\begin{theorem} \label{thONBtau}
Suppose that $\displaystyle \tau\in \T^2(H)$ is an inner product on a
vector space $H$ of dimension $n$ over the real.  There exists a basis
$\displaystyle \B = \{ \VEC{v}_i \}_{i=1}^n$ of $H$ such that
$\tau(\VEC{v}_i,\VEC{v}_j) = \delta_{i,j}$ for all $i$ and $j$.
\end{theorem}

The theorem is proved exactly as it is done in linear algebra using
Gram-Schmidt orthogonalization process with the inner product
$\ps{\VEC{v}_1}{\VEC{v}_2} \equiv \tau(\VEC{v}_1,\VEC{v}_2)$ for
$\VEC{v}_1,\VEC{v}_2 \in H$ and the norm
$\|\VEC{v}\| \equiv \sqrt{\ps{\VEC{v}}{\VEC{v}}}$ for $\VEC{v} \in H$.

\begin{defn}
A basis $\B$ as in the statement of the Theorem~\ref{thONBtau} is
called an {\bfseries orthonormal basis}\index{Tensor!Orthonormal Basis}
with respect to the inner product $\tau$.
\end{defn}

Let $\displaystyle \B = \{ \VEC{v}_i \}_{i=1}^n$ be a basis of $H$, it
is possible to choose an inner product on $H$ such that $\B$ is an
orthonormal basis.  Let $\displaystyle \E = \{ \VEC{e}_i \}_{i=1}^n$
be the canonical basis of $\displaystyle \RR^n$ and
$\ps{}{}$ be the standard inner product on $\displaystyle \RR^n$
defined by
$\displaystyle \ps{\VEC{x}_1}{\VEC{x}_2} = \sum_{j=1}^n x_{j,1}x_{j,2}$
for $\displaystyle \VEC{x}_i = \sum_{j=1}^n x_{j,i}\VEC{e}_j$ and
$1 \leq i \leq 2$.
Moreover, let $\displaystyle f:H \to \RR^n$ be the linear
isomorphism defined by $f(\VEC{v}_i) = \VEC{e}_i$ for $1 \leq i \leq n$.
We define the inner product $\displaystyle \tau \in \T^2(H)$ by
\begin{align*}
\tau(\VEC{w}_1,\VEC{w}_2) &= \ps{f(\VEC{w}_1)}{f(\VEC{w}_2)}
= \ps{\sum_{j=1}^n w_{j,1}f(\VEC{v}_j)}{\sum_{j=1}^n w_{j,2}f(\VEC{v}_j)}
= \ps{\sum_{j=1}^n w_{j,1}\VEC{e}_j}{\sum_{j=1}^n w_{j,2}\VEC{e}_j} \\
&= \sum_{j=1}^n w_{j,1}w_{j,2}
\end{align*}
for $\displaystyle \VEC{w}_i = \sum_{j=1}^n w_{j,i}\VEC{v}_j$ and
$1 \leq i \leq 2$.  The basis $\B$ is an orthogonal basis with respect
to the inner product $\tau$.

\begin{defn}
A {\bfseries permutation}\index{Permuation} $\sigma$ on
$J_k=\{1,2,\ldots,k\}$ is a one-to-one and onto
function $\sigma:J_k \to J_k$.  A
{\bfseries transposition}\index{Transposition} on $J_k$ is a permutation
on $J_k$ that interchanges only two elements of $J_k$ and
leaves the others fixed.
\end{defn}

The set of all permutation on $J_k$ forms a group with respect to
the composition of functions that we denote $S_k$.  There is a famous
result in algebra that says that any permutation on $J_k$ may be
expressed as the composition of transpositions on $J_k$.  Moreover,
there exists a function $\sgn$ acting on the permutations on $J_k$
such that $\sgn(\sigma) = -1$ if $\sigma$ is a transposition on $J_k$
and $\sgn(\sigma_1\circ \sigma_2) = \sgn(\sigma_1)\sgn(\sigma_2)$ for
all permutations $\sigma_1$ and $\sigma_2$ on $J_k$.  In particular,
for a permutation $\sigma$ on $J_k$, $\sgn(\sigma) = 1$ if and only if
$\sigma$ is the composition of an even number of transpositions.
Therefore $\sgn(\sigma) = -1$ if and only if $\sigma$ is the
composition of an odd number of transpositions.  All these results
are proved in standard books in (linear) algebra, for instance in
\cite{La}.

\begin{defn}
A $k$-tensor $\displaystyle \tau \in \T^k(H)$ is called
{\bfseries alternating}\index{Tensor!Alternating} if
\[
\tau(\VEC{v}_1, \VEC{v}_2, \ldots ,\VEC{v}_k) =
-\tau(\VEC{v}_{\sigma(1)}, \VEC{v}_{\sigma(2)}, \ldots ,\VEC{v}_{\sigma(k)})
\]
for any transposition $\sigma$ on $J_k$.  The set of all alternating
$k$-tensor on $H$ is denoted $\displaystyle \Omega^k(H)$.
\end{defn}

It is easy to verify that $\displaystyle \Omega^k(H)$ is a linear
subspace of $\displaystyle \T^k(H)$.

\begin{prop} \label{propAltDefn}
Given $\displaystyle \tau \in \T^k(H)$, let
\[
\alt(\tau)(\VEC{v}_1, \VEC{v}_2, \ldots ,\VEC{v}_k)
= \frac{1}{k!} \sum_{\sigma\in S_k} \sgn(\sigma)
\tau(\VEC{v}_{\sigma(1)}, \VEC{v}_{\sigma(2)}, \ldots ,\VEC{v}_{\sigma(k)})
\]
for $\VEC{v}_i \in H$ and $1\leq i \leq k$,
where $S_k$ is the group of permutations on $J_k = \{1,2,\ldots,k\}$.
Then $\displaystyle \alt:\T^k(H) \to \Omega^k(H)$.
\end{prop}

\begin{proof}
Suppose that $\displaystyle \tau \in \T^k(H)$ and $\tilde{\sigma}$ is a
fixed transposition on $J_k$.  We have that
$\displaystyle \tilde{\sigma}^{-1} = \tilde{\sigma}$ and
\begin{align*}
\tilde{\sigma} : S_k & \to S_k \\
\sigma &\mapsto \sigma \circ \tilde{\sigma}
\end{align*}
is one-to-one and onto.  Hence
\begin{align*}
&\alt(\tau)(\VEC{v}_{\tilde{\sigma}(1)}, \VEC{v}_{\tilde{\sigma}(2)},
\ldots,\VEC{v}_{\tilde{\sigma}(k)})
= \frac{1}{k!} \sum_{\sigma\in S_k} \sgn(\sigma)
\tau(\VEC{v}_{(\sigma\circ \tilde{\sigma})(1)},
\VEC{v}_{(\sigma\circ\tilde{\sigma})(2)}, \ldots ,
\VEC{v}_{(\sigma\circ\tilde{\sigma})(k)}) \\
&\quad = \frac{1}{k!} \sum_{\sigma\circ\tilde{\sigma}\in S_k}
\sgn(\sigma\circ\tilde{\sigma})
\tau(\VEC{v}_{\sigma(1)},\VEC{v}_{\sigma(2)}, \ldots,\VEC{v}_{\sigma(k)}) \\
&\quad = \frac{1}{k!} \sum_{\sigma\in S_k} \sgn(\sigma)\sgn(\tilde{\sigma})
\tau(\VEC{v}_{\sigma(1)},\VEC{v}_{\sigma(2)}, \ldots , \VEC{v}_{\sigma(k)})
= -\frac{1}{k!} \sum_{\sigma\in S_k} \sgn(\sigma)
\tau(\VEC{v}_{\sigma(1)},\VEC{v}_{\sigma(2)}, \ldots , \VEC{v}_{\sigma(k)}) \\
&\quad = - \alt(\tau)(\VEC{v}_1, \VEC{v}_2, \ldots,\VEC{v}_k)
\end{align*}
for all $\VEC{v}_i \in H$ and $1 \leq i \leq k$.
\end{proof}

\begin{prop} \label{propAltR}
\begin{enumerate}
\item If $\displaystyle \tau \in \Omega^k(H)$, then $\alt(\tau)=\tau$.
\item If $\displaystyle \tau \in \T^k(H)$, then $\alt(\alt(\tau))=\alt(\tau)$.
\item If $\displaystyle \tau_i \in \Omega^{k_i}(H)$ for $1 \leq i \leq 2$
and $\alt(\tau_1) = 0$, then $\alt(\tau_1 \otimes \tau_2) =
\alt(\tau_2 \otimes \tau_1) = 0$.
\item If $\displaystyle \tau_i \in \Omega^{k_i}(H)$ for $1 \leq i \leq 3$,
then $\alt(\alt(\tau_1\otimes \tau_2) \otimes \tau_3)
= \alt(\tau_1\otimes \alt(\tau_2 \otimes \tau_3))
= \alt(\tau_1\otimes \tau_2 \otimes \tau_3)$.
\end{enumerate}
\end{prop}

\begin{proof}
\stage{1}  Since $\sigma \in S_k$ is the product of transpositions, we
have that
$\tau(\VEC{v}_{\sigma(1)}, \VEC{v}_{\sigma(2)},\ldots, \VEC{v}_{\sigma(k)})
= \sgn(\sigma) \tau(\VEC{v}_1, \VEC{v}_2,\ldots, \VEC{v}_k)$
for all $\VEC{v}_i \in H$ and $1 \leq i \leq k$.
Moreover, the number of different functions from $J_k$ to $J_k$ is $k!$.
Namely, the number of elements of $S_k$, denoted $|S_k|$, is $k!$.
Hence, for $\displaystyle \tau \in \Omega^k(H)$, we have
\begin{align*}
&\alt(\tau)(\VEC{v}_1, \VEC{v}_2, \ldots ,\VEC{v}_k)
= \frac{1}{k!} \sum_{\sigma\in S_k} \sgn(\sigma)
\tau(\VEC{v}_{\sigma(1)}, \VEC{v}_{\sigma(2)}, \ldots ,\VEC{v}_{\sigma(k)}) \\
&\qquad
= \left(\frac{1}{k!} \sum_{\sigma\in S_k}
\underbrace{\sgn(\sigma) \sgn(\sigma)}_{=1}\right)
\tau(\VEC{v}_1, \VEC{v}_2, \ldots ,\VEC{v}_k)
= \tau(\VEC{v}_1, \VEC{v}_2, \ldots ,\VEC{v}_k)
\end{align*}
for all $\VEC{v}_i \in H$ and $1 \leq i \leq k$.

\stage{2} This follows from (1) because
$\displaystyle \alt(\tau) \in \Omega^k(H)$.

\stage{3} Let $\tilde{S}_{k_1} \subset S_{k_1+k_2}$ be the set of all
permutations $\tilde{\sigma}$ on $J_{k_1+k_2}$ such that
$\tilde{\sigma}(i) = i$ for $k_1+1 \leq i \leq k_2$.  Namely,
$\tilde{\sigma}$ permutes only the elements of the set $\{1,2,\ldots,k_1\}$.
We have that $\tilde{S}_{k_1}$ is a subgroup of $S_{k_1+k_2}$.
We choose a representative $\sigma_j \in S_{k_1+k_2}$ for each of the
elements (i.e.\ equivalence classes) of the quotient group
$S_{k_1+k_2}/\tilde{S}_{k_1}$.  We end up with a set
of permutations $S = \{\sigma_j: 1 \leq j \leq J\}$.
We have that $\displaystyle S_{k_1+k_2} = \bigcup_{1\leq j \le J}
\{ \sigma_j \circ \tilde{\sigma} : \tilde{\sigma} \in \tilde{S}_{k_1} \}$
and each element of $S_{k_1+k_2}$ has a unique representation of the form
$\sigma_j \circ \tilde{\sigma}$ with $\tilde{\sigma} \in \tilde{S}_{k_1}$
and $\sigma_j \in S$.  Hence
\begin{align*}
&\alt(\tau_1\otimes \tau_2)(\VEC{v}_1, \VEC{v}_2, \ldots ,\VEC{v}_{k_1+k_2})
= \frac{1}{(k_1+k_2)!} \sum_{\sigma\in S_{k_1+k_2}} \sgn(\sigma)
\tau(\VEC{v}_{\sigma(1)}, \VEC{v}_{\sigma(2)}, \ldots,\VEC{v}_{\sigma(k_1+k_2)})
\\
&\quad = \frac{1}{(k_1+k_2)!} \sum_{j=1}^J
\sum_{\tilde{\sigma}\in \tilde{S}_{k_1}} \sgn(\sigma_j\circ\tilde{\sigma})
(\tau_1\otimes \tau_2)(\VEC{v}_{(\sigma_j\circ\tilde{\sigma})(1)},
\VEC{v}_{(\sigma_j\circ\tilde{\sigma})(2)}, \ldots ,
\VEC{v}_{(\sigma_j\circ\tilde{\sigma})(k_1+k_2)}) \\
&\quad = \frac{1}{(k_1+k_2)!} \sum_{j=1}^J
\sum_{\tilde{\sigma}\in \tilde{S}_{k_1}} \sgn(\sigma_j\circ\tilde{\sigma})
\sgn(\sigma_j) (\tau_1\otimes \tau_2)
(\VEC{v}_{\tilde{\sigma}(1)},\VEC{v}_{\tilde{\sigma}(2)}, \ldots ,
\VEC{v}_{\tilde{\sigma}(k_1+k_2)}) \\
&\quad = \frac{1}{(k_1+k_2)!} \sum_{j=1}^J
\sum_{\tilde{\sigma}\in \tilde{S}_{k_1}} \sgn(\tilde{\sigma})
(\tau_1 \otimes \tau_2)
(\VEC{v}_{\tilde{\sigma}(1)},\VEC{v}_{\tilde{\sigma}(2)}, \ldots ,
\VEC{v}_{\tilde{\sigma}(k_1+k_2)}) \\
&\quad = \frac{k_1!}{(k_1+k_2)!} \sum_{j=1}^J
\left( \frac{1}{k_1!}\sum_{\tilde{\sigma}\in \tilde{S}_{k_1}} \sgn(\tilde{\sigma})
\tau_1(\VEC{v}_{\tilde{\sigma}(1)},\VEC{v}_{\tilde{\sigma}(2)}, \ldots,
\VEC{v}_{\tilde{\sigma}(k_1)}) \right)
\tau_2(\VEC{v}_{k_1+1},\VEC{v}_{k_1+2}, \ldots , \VEC{v}_{k_1+k_2}) \\
&\quad = \frac{J\,k_1!}{(k_1+k_2)!}
\alt(\tau_1)(\VEC{v}_1,\VEC{v}_2, \ldots, \VEC{v}_{k_1})
\tau_2(\VEC{v}_{k_1+1},\VEC{v}_{k_1+2}, \ldots , \VEC{v}_{k_1+k_2})
= 0
\end{align*}
for all $\VEC{v}_i\in H$ and $1 \leq i \leq k_1+k_2$.
The proof that $\displaystyle \alt(\tau_2 \otimes \tau_1) = 0$ is
similar.  It suffices to use the subgroup
$\breve{S}_{k_1} \subset S_{k_1+k_2}$ of all
permutations $\breve{\sigma}$ on $J_{k_1+k_2}$ such that
$\breve{\sigma}(i) = i$ for $1 \leq i \leq k_2$.

\stage{4}
It follows from (2) that
$\alt(\alt(\tau_2 \otimes \tau_3) - \tau_2\otimes \tau_3) 
= \alt(\tau_2\otimes \tau_3) - \alt(\tau_2\otimes \tau_3)  = 0$. 
Hence, we get from (3) that
\begin{align*}
0 & =\alt\big( \tau_1 \otimes \big(\alt(\tau_2 \otimes \tau_3)
- \tau_2\otimes \tau_3\big)\big)
= \alt\big( \tau_1 \otimes \alt(\tau_2 \otimes \tau_3)
- \tau_1 \otimes \tau_2\otimes \tau_3 \big) \\
&= \alt( \tau_1 \otimes \alt(\tau_2 \otimes \tau_3))
- \alt(\tau_1 \otimes \tau_2\otimes \tau_3 \big) \ .
\end{align*}
Similarly, we can prove that
$\alt( \alt(\tau_1 \otimes \tau_2) \otimes \tau_3)
- \alt(\tau_1 \otimes \tau_2\otimes \tau_3 \big) = 0$.
\end{proof}

\begin{defn} \label{defnWwedgeP}
The {\bfseries wedge product}\index{Tensor!Wedge Product}
$\tau_1 \wedge \tau_2$ of
$\displaystyle \tau_1 \in \Omega^{k_1}(H)$ and
$\displaystyle \tau_2\in \Omega^{k_2}(H)$ is the tensor
$\displaystyle \tau_1\wedge \tau_2 \in \Omega^{k_1+k_2}(H)$ defined by
\[
\tau_1 \wedge \tau_2 = \frac{(k_1+k_2)!}{k_1! k_2!}
\alt(\tau_1\otimes \tau_2) \ .
\]
\end{defn}

Let $H_1$ and $H_2$ be two vector spaces over $\RR$.  Given a linear map
$f:H_1\to H_2$, we can define a map
$\displaystyle f^\ast: \T^k(H_2)\to \T^k(H_1)$ by
\[
(f^\ast(\tau))(\VEC{v}_1,\VEC{v}_2, \ldots,\VEC{v}_k) = 
\tau(f(\VEC{v}_1),f(\VEC{v}_2), \ldots,f(\VEC{v}_k))
\]
for all $\displaystyle \tau \in \T^k(H_2)$ and $\VEC{v}_i \in H_1$ for
$1 \leq i \leq k$.  We say that
$\displaystyle f^\ast(\tau)$ is the {\bfseries pull-back}\index{Pull-Back}
of $\tau$ by $f$.

The wedge product satisfies many relations.  Some of them are listed in
the next proposition.

\begin{prop} \label{propWedgeR}
\begin{enumerate}
\item If $\displaystyle \tau_1,\tau_2 \in \Omega^{k_1}$
and $\displaystyle \tau_3 \in \Omega^{k_2}(H)$, then
$(\tau_1 + \tau_2)\wedge \tau_3 = \tau_1 \wedge \tau_3 + \tau_2
\wedge \tau_3$.
\item If $\displaystyle \tau_1\in \Omega^{k_1}(H)$ and
$\displaystyle \tau_2,\tau_3 \in \Omega^{k_2}(H)$, then
$\tau_1\wedge(\tau_2+ \tau_3) = \tau_1 \wedge \tau_2 + \tau_1
\wedge \tau_3$.
\item If $\displaystyle \tau_i \in \Omega^{k_i}(H)$
for $1 \leq i \leq 2$ and $a \in \RR$, then
$(a \tau_1) \wedge \tau_2 = \tau_1 \wedge (a\tau_2) = 
a (\tau_1 \wedge \tau_2)$.
\item If $\displaystyle \tau_i \in \Omega^{k_i}(H)$
for $1 \leq i \leq 2$, then
$\displaystyle \tau_1 \wedge \tau_2 = (-1)^{k_1 k_2} \tau_2 \wedge \tau_1$.
\item If $\displaystyle \tau_i \in \Omega^{k_i}(H)$
for $1 \leq i \leq 2$ and $f:H_1\to H_2$ is a linear map from the
vector space $H_1$ to the vector space $H_2$, then 
$\displaystyle f^\ast(\tau_1 \wedge \tau_2) = f^\ast\tau_1 \wedge f^\ast \tau_2$.
\item If $\displaystyle \tau_i\in \Omega^{k_i}(H)$ for $1\leq i \leq 3$, then
\[
(\tau_1\wedge \tau_2) \wedge \tau_3 = \tau_1 \wedge ( \tau_2 \wedge \tau_3)
= \frac{(k_1+k_2+k_3)!}{k_1! k_2! k_3!}
\alt(\tau_1\otimes \tau_2 \otimes \tau_3) \ .
\]
\end{enumerate}
\end{prop}

\begin{proof}
We leave the proof of (1), (2), (3) and (5) to the reader.

\stage{4} Given $\sigma \in S_{k_1+k_2}$, we have
\begin{align*}
&(\tau_1 \otimes \tau_2)(\VEC{v}_{\sigma(1)}, \VEC{v}_{\sigma(2)}, \ldots,
\VEC{v}_{\sigma(k_1+k_2)}) \\
&\qquad = \tau_1(\VEC{v}_{\sigma(1)}, \VEC{v}_{\sigma(2)}, \ldots,
\VEC{v}_{\sigma(k_1)})\,
\tau_2(\VEC{v}_{\sigma(k_1+1)}, \VEC{v}_{\sigma(k_1+2)}, \ldots,
\VEC{v}_{\sigma(k_1+k_2)}) \\
&\qquad = \tau_2(\VEC{v}_{\sigma(k_1+1)}, \VEC{v}_{\sigma(k_1+2)}, \ldots,
\VEC{v}_{\sigma(k_1+k_2)})\,
\tau_1(\VEC{v}_{\sigma(1)}, \VEC{v}_{\sigma(2)}, \ldots,
\VEC{v}_{\sigma(k_1)}) \\
&\qquad = \tau_2(\VEC{v}_{\tilde{\sigma}(1)}, \VEC{v}_{\tilde{\sigma}(2)},
\ldots, \VEC{v}_{\tilde{\sigma}(k_2)})\,
\tau_1(\VEC{v}_{\tilde{\sigma}(k_2+1)}, \VEC{v}_{\tilde{\sigma}(k_2+2)}, \ldots,
\VEC{v}_{\tilde{\sigma}(k_2+k_1)}) \\
&\qquad = (\tau_2 \otimes \tau_1)(\VEC{v}_{\tilde{\sigma}(1)},
\VEC{v}_{\tilde{\sigma}(2)}, \ldots, \VEC{v}_{\tilde{\sigma}(k_1+k_2)}) \ ,
\end{align*}
where $\tilde{\sigma} \in S_{k_1+k_2}$ is defined by
\[
\tilde{\sigma}(i) = \begin{cases}
\sigma(k_1+i) & \quad \text{if} \ 1 \leq i \leq k_2 \\
\sigma(i-k_2) & \quad \text{if} \ k_2+1 \leq i \leq k_1 + k_2
\end{cases}
\]
Note that $\sigma \mapsto \tilde{\sigma}$ for $\sigma \in S_{k_1+k_2}$
defines a one-to-one and onto map from $S_{k_1+k_2}$ into itself.

We need to determine the relation between $\sgn(\tilde{\sigma})$ and
$\sgn(\sigma)$.  Let $\sigma_{i_1,i_2} \in S_{k_1+k_2}$ be the
transposition such that $\sigma_{i_1,i_2}(i_1) = i_2$,
$\sigma_{i_1,i_2}(i_2) = i_1$ and $\sigma_{i_1,i_2}(i) = i$ for $i
\neq i_1,i_2$.

Let $\tilde{\sigma}_0 = \sigma$ and
$\tilde{\sigma}_j = \tilde{\sigma}_{j-1} \circ
\sigma_{k_1-j+k_2,k_1-j+k_2+1}\circ \ldots
\circ \sigma_{k_1-j+2,k_1-j+3}\circ \sigma_{k_1-j+1,k_1-j+2}$ for
$1 \leq j \leq k_1$.  We have
\begin{align*}
\sgn(\tilde{\sigma}_j) &= \sgn(\tilde{\sigma}_{j-1})\,
\underbrace{\sgn(\sigma_{k_1-j+k_2,k_1-j+k_2+1}) \cdots
\sgn(\sigma_{k_1-j+2,k_1-j+3}) \sgn(\sigma_{k_1-j+1,k_1-j+2})}_{k_2\text{ times}} \\
&= (-1)^{k_2} \sgn(\tilde{\sigma}_{j-1})
\end{align*}
for $1 \leq j \leq k_1$.  By induction, it is easy to verify that
$\displaystyle \sgn(\tilde{\sigma}_{k_1}) = (-1)^{k_1 k_2}
\sgn(\tilde{\sigma}_0) = (-1)^{k_1 k_2} \sgn(\sigma)$.

We claim that $\tilde{\sigma}_{k_1} = \tilde{\sigma}$.  We take a
close look at $\tilde{\sigma}_j$.  For $1\leq i < k_1 - j +1$ or
$k_1 -j +k_2 +1 < i \leq k_1+k_2$, we have
\begin{align*}
i & \xrightarrow{\sigma_{k_1-j+1,k_1-j+2}}
i \xrightarrow{\sigma_{k_1-j+2,k_1-j+3}}
i \xrightarrow{\sigma_{k_1-j+3,k_1-j+4}}
\ldots \\
& \xrightarrow{\sigma_{k_1-j+k_2-1,k_1-j+k_2}}
i \xrightarrow{\sigma_{k_1-j+k_2,k_1-j+k_2+1}}
i \xrightarrow{\tilde{\sigma}_{j-1}} \tilde{\sigma}_{j-1}(i) \ .
\end{align*}
For $k_1 -j +1 < i \leq k_1-j+k_2+1$, say $i = k_1 -j+m +1$ with
$1\leq m \leq k_2$, we have
\begin{align*}
i & \xrightarrow{\sigma_{k_1-j+1,k_1-j+2}}
i \xrightarrow{\sigma_{k_1-j+2,k_1-j+3}}
i \xrightarrow{\sigma_{k_1-j+3,k_1-j+4}}
\ldots \xrightarrow{\sigma_{k_1-j+m-1,k_1-j+m}}
i \xrightarrow{\sigma_{k_1-j+m,k_1-j+m+1}} \\
& i-1 \xrightarrow{\sigma_{k_1-j+m+1,k_1-j+m+2}}
i-1 \xrightarrow{\sigma_{k_1-j+m+2,k_1-j+m+3}}
\ldots \xrightarrow{\sigma_{k_1-j+k_2-1,k_1-j+k_2}} \\
&i-1 \xrightarrow{\sigma_{k_1-j+k_2,k_1-j+k_2+1}}
 i-1 \xrightarrow{\tilde{\sigma}_{j-1}} \tilde{\sigma}_{j-1}(i-1) \ .
\end{align*}
Finally, for $i = k_1-j+1$, we have
\begin{align*}
i &\xrightarrow{\sigma_{k_1-j+1,k_1-j+2}}
i+1 \xrightarrow{\sigma_{k_1-j+2,k_1-j+3}}
i+2 \xrightarrow{\sigma_{k_1-j+3,k_1-j+4}}
\ldots \xrightarrow{\sigma_{k_1-j+k_2-1,k_1-j+k_2}} \\
&i+ k_2-1 \xrightarrow{\sigma_{k_1-j+k_2,k_1-j+k_2+1}}
i+ k_2 \xrightarrow{\tilde{\sigma}_{j-1}} \tilde{\sigma}_{j-1}(i+k_2) \ .
\end{align*}
To summarize, we have shown that
\[
\tilde{\sigma}_j(i) = \begin{cases}
\tilde{\sigma}_{j-1}(i) & \quad \text{if} \ 1 \leq i \leq k_1-j \\
\tilde{\sigma}_{j-1}(i+1) & \quad \text{if} \ k_1-j+ 1 \leq i <
k_1 -j + k_2 + 1 \\
\tilde{\sigma}_{j-1}(i-k_2) & \quad \text{if} \ k_1-j+k_2+1 \leq
i \leq  k_1 + k_2
\end{cases}
\]
With this information, a simple proof by induction shows that
\[
\tilde{\sigma}_j(i) = \begin{cases}
\sigma(i) & \quad \text{if} \ 1 \leq i \leq k_1-j \\
\sigma(i+j) & \quad \text{if} \ k_1-j+1 \leq i \leq k_1 + k_2 -j \\
\sigma(i - k_2) & \quad \text{if} \ k_1+k_2 -j +1 \leq i \leq  k_1 + k_2
\end{cases}
\]
for $1 \leq j \leq k_1$.  In particular,
$\tilde{\sigma}_{k_1} = \tilde{\sigma}$ as claimed.

We therefore have that
$\displaystyle \sgn(\sigma) = (-1)^{k_1 k_2}\sgn(\tilde{\sigma})$.
Hence
\begin{align*}
&(\tau_1\wedge\tau_2)(\VEC{v}_1, \VEC{v}_, \ldots, \VEC{v}_{k_1+k_2})
= \frac{1}{k_1!\,k_2!} \sum_{\sigma \in S_{k_1+k_2}}
\sgn(\sigma) (\tau_1 \otimes \tau_2)
(\VEC{v}_{\sigma(1)}, \VEC{v}_{\sigma(2)}, \ldots,
\VEC{v}_{\sigma(k_1+k_2)}) \\
&\qquad = \frac{1}{k_1!\,k_2!} \sum_{\tilde{\sigma} \in S_{k_1+k_2}}
\sgn(\sigma) (\tau_2 \otimes \tau_1)
(\VEC{v}_{\tilde{\sigma}(1)}, \VEC{v}_{\tilde{\sigma}(2)}, \ldots,
\VEC{v}_{\tilde{\sigma}(k_1+k_2)}) \\
&\qquad = (-1)^{k_1 k_2} \frac{1}{k_1!\,k_2!} \sum_{\tilde{\sigma} \in S_{k_1+k_2}}
\sgn(\tilde{\sigma}) (\tau_2 \otimes \tau_1)
(\VEC{v}_{\tilde{\sigma}(1)}, \VEC{v}_{\tilde{\sigma}(2)}, \ldots,
\VEC{v}_{\tilde{\sigma}(k_1+k_2)}) \\
&\qquad = (-1)^{k_1 k_2} (\tau_2\wedge \tau_1)(\VEC{v}_1, \VEC{v}_,
\ldots, \VEC{v}_{k_1+k_2})                                        
\end{align*}
for all $\VEC{v}_i \in H$ and $1 \leq i \leq k_1 + k_2$.

\stage{6} We have
\begin{align*}
\tau_1 \wedge ( \tau_2 \wedge \tau_3)
&= \frac{(k_1+k_2 +k_3)!}{k_1!\, (k_2+k_3)!} \alt(\tau_1 \otimes
(\tau_2 \wedge \tau_3)) \\
&= \frac{(k_1+k_2 +k_3)!}{k_1!\, (k_2+k_3)!} \alt\left(\tau_1 \otimes
\left(\frac{(k_2+k_3)!}{k_2! \, k_3!} \alt(\tau_2 \otimes \tau_3\right)\right)
\\
&= \frac{(k_1+k_2 +k_3)!}{k_1!\, (k_2+k_3)!}\,
\frac{(k_2+k_3)!}{k_2! \, k_3!} \alt\left(\tau_1 \otimes
\left(\alt(\tau_2 \otimes \tau_3\right)\right) \\
&= \frac{(k_1+k_2+k_3)!}{k_1! k_2! k_3!}
\alt(\tau_1\otimes \tau_2 \otimes \tau_3) \ ,
\end{align*}
where we have used the definition of the wedge product to get the first
two equality and (4) of Proposition~\ref{propAltR} to get the last
equality.  The other relation in (6) is similarly proved.
\end{proof}

From (6) of the previous theorem, we have associativity of the wedge product.
Hence, we may defined $\tau_1\wedge \tau_2 \wedge \tau_3$ as
$(\tau_1\wedge \tau_2) \wedge \tau_3 = \tau_1 \wedge ( \tau_2 \wedge \tau_3)$
for all $\displaystyle \tau_i \in \Omega^{k_i}(H)$ with $1\leq i \leq 3$.
Inductively, we may define
$\tau_1\wedge \tau_2 \wedge \ldots \wedge \tau_j$ for all
$\displaystyle \tau_i\in \Omega^{k_i}(H)$ with $1\leq i \leq j$.

\begin{cor} \label{corMwedgesR}
$\displaystyle \tau_1\wedge \tau_2 \wedge \ldots \wedge \tau_j
= \frac{(k_1+k_2+\ldots + k_j)!}{k_1! k_2! \cdots k_j!}
\alt(\tau_1\otimes \tau_2 \otimes \ldots \otimes \tau_j)$ for
$\tau_i \in \Omega^{k_i}(H)$ and $1\leq i \leq j$.
\end{cor}

The proof is left to the reader.

\begin{prop} \label{propTssT}
Given $\displaystyle \tau_i \in \Omega^1(H)$ for $1\leq i \leq k$, we have
\begin{equation} \label{wedgeIeq1}
\tau_{\rho(1)} \wedge \tau_{\rho(2)} \wedge \ldots \wedge \tau_{\rho(k)}
= \sgn(\rho)\, \tau_1 \wedge \tau_2 \wedge \ldots \wedge \tau_k
\end{equation}
for all $\rho \in S_k$, the group of permutations on
$J_k = \{1,2,\ldots,k\}$.
In particular, if $\tau_p = \tau_q$ for some $1 \leq p < q \leq k$, then
$\tau_1 \wedge \tau_2 \wedge \ldots \wedge \tau_k = 0$.
\end{prop}

\begin{proof}
We first note that the map from $S_k$ to itself defined by
$\displaystyle \sigma \mapsto \sigma\circ\rho^{-1}$ for
$\sigma \in S_k$ is one-to-one
and onto.  Moreover, for all $1\leq m \leq k$, we have that
$\tau_{\rho(m)}(\VEC{w}_{\sigma(m)})
= \tau_p(\VEC{w}_{\sigma(\rho^{-1}(p))})$ for $p = \rho(m)$
and $\VEC{w}_i \in H$ and $1 \leq i \leq k$.
Hence, we get from Corollary~\ref{corMwedgesR} that
\begin{align*}
&(\tau_{\rho(1)}\wedge \tau_{\rho(2)} \wedge \ldots \wedge
\tau_{\rho(k)}) (\VEC{w}_1, \VEC{w}_2, \ldots, \VEC{w}_k) \\
&\qquad\qquad = k! \alt(\tau_{\rho(1)}\otimes \tau_{\rho(2)} \otimes \ldots
\otimes \tau_{\rho(k)}) (\VEC{w}_1, \VEC{w}_2, \ldots, \VEC{w}_k) \\
&\qquad\qquad = \sum_{\sigma\in S_k} \sgn(\sigma)\,
\tau_{\rho(1)}(\VEC{w}_{\sigma(1)}) \tau_{\rho(2)}(\VEC{w}_{\sigma(2)})
\cdots \tau_{\rho(k)}(\VEC{w}_{\sigma(k)}) \\
&\qquad\qquad =\sum_{\sigma\in S_k} \sgn(\sigma)\,
\tau_1(\VEC{w}_{(\sigma\circ\rho^{-1})(1)})
\tau_2(\VEC{w}_{(\sigma\circ\rho^{-1})(2)})
\cdots \tau_k(\VEC{w}_{(\sigma\circ\rho^{-1})(k)}) \\
&\qquad\qquad =\sum_{\tilde{\sigma}\in S_k}
\sgn\left(\tilde{\sigma}\circ \rho\right)\,
\tau_1(\VEC{w}_{\tilde{\sigma}(1)})
\tau_2(\VEC{w}_{\tilde{\sigma}(2)})
\cdots \tau_k(\VEC{w}_{\tilde{\sigma}()}) \\
% &\qquad\qquad =\sum_{\tilde{\sigma}\in S_k}
% \sgn(\tilde{\sigma})\sgn(\rho)\,
% \tau_1(\VEC{w}_{\tilde{\sigma}(1)})
% \tau_2(\VEC{w}_{\tilde{\sigma}(2)})
% \cdots \tau_k(\VEC{w}_{\tilde{\sigma}(k)}) \\
&\qquad\qquad = \sgn(\rho) \, \sum_{\tilde{\sigma}\in S_k}
\sgn(\tilde{\sigma})\,\tau_1(\VEC{w}_{\tilde{\sigma}(1)})
\tau_2(\VEC{w}_{\tilde{\sigma}(2)})
\cdots \tau_k(\VEC{w}_{\tilde{\sigma}(k)}) \\
&\qquad\qquad = \sgn(\rho)\, k! \alt(\tau_1\otimes\tau_2\otimes\ldots
\otimes \tau_k)(\VEC{w}_2, \VEC{w}_2, \ldots, \VEC{w}_k) \\
&\qquad\qquad = \sgn(\rho)\, (\tau_1\wedge \tau_2\wedge \ldots
\wedge \tau_k) (\VEC{w}_1, \VEC{w}_2, \ldots, \VEC{w}_k)
\end{align*}
for all $\VEC{w}_i \in H$ and $1 \leq i \leq k$.

Suppose the $\tau_p = \tau_q$ for some $1 \leq p < q \leq k$.  As usual, let
$\sigma_{p,q} \in S_k$ be the transposition of $p$ and $q$.
Then
\begin{align*}
&\tau_1 \wedge \tau_2 \wedge \ldots \wedge \tau_k
= \tau_1 \wedge \tau_2 \wedge \ldots \wedge \tau_q
\wedge \ldots \wedge \tau_p \wedge \ldots \wedge \tau_k \\
&= \tau_{\sigma_{p,q}(1)} \wedge \tau_{\sigma_{p,q}(2)}
\wedge \ldots \wedge \tau_{\sigma_{p,q}(k)}
= \sgn(\sigma_{p,q})\, \tau_1 \wedge \tau_2
\wedge \ldots \wedge \tau_j
= - \tau_1 \wedge \tau_2 \wedge \ldots \wedge \tau_j \ ,
\end{align*}
where the first equality comes from $\tau_p = \tau_q$.  Thus
$\tau_1 \wedge \tau_2 \wedge \ldots \wedge \tau_j  = 0$.
\end{proof}

\begin{theorem}  \label{thOkbasis}
As in Theorem~\ref{stokesBT}, let
$\displaystyle \B^\ast = \{ \phi_i \}_{i=1}^n$ be
the dual basis associated to a basis
$\displaystyle \B = \{ \VEC{v}_i \}_{i=1}^n$ of $H$.
The set $\displaystyle \NNN =
\{ \phi_{i_1} \wedge \phi_{i_2} \wedge \ldots \wedge \phi_{i_k} :
1 \leq i_1 < i_2 < \ldots < i_k \leq n \}$
is a basis of $\displaystyle \Omega^k(H)$.  In particular, the dimension of
$\displaystyle \Omega^k(H)$ is
$\displaystyle \binom{n}{k} = \frac{n!}{k!(n-k)!}$.
\end{theorem}

\begin{proof}
\stage{i} We first show that any $\displaystyle \tau \in \Omega^k(H)$
can be written as a linear combination of elements in $\NNN$.

It follows from Theorem~\ref{stokesBT} that we may write $\tau$ as
$\displaystyle \tau = \sum_{\substack{1 \leq i_j \leq n\\1\leq j \leq k}}
a_{i_1,i_2,\ldots,i_k}
\phi_{i_1} \otimes \phi_{i_2} \otimes \ldots \otimes \phi_{i_k}$
for some $a_{i_1,i_2,\ldots,i_k} \in \RR$.  Hence, using (1) of
Proposition~\ref{propAltR}, Corollary~\ref{corMwedgesR} and
Proposition~\ref{propTssT} (with $\tau_j = \phi_{i_j}$ for
$1\leq j \leq k$), we get
\begin{align*}
\tau = \alt(\tau) &= \sum_{\substack{1 \leq i_j \leq n\\1\leq j \leq k}}
a_{i_1,i_2,\ldots,i_k}
\alt(\phi_{i_1} \otimes \phi_{i_2} \otimes \ldots \otimes \phi_{i_k}) \\
&= k! \sum_{\substack{1 \leq i_j \leq n\\1\leq j \leq k}}
a_{i_1,i_2,\ldots,i_k}
(\phi_{i_1} \wedge \phi_{i_2} \wedge \ldots \wedge \phi_{i_k}) \\
&= \sum_{1\leq i_1 < i_2 < \ldots < i_k \leq n}
\left( k! \sum_{\sigma \in S_k} a_{\sigma(i_1),\sigma(i_2),\ldots,\sigma(i_k)}
\, \phi_{\sigma(i_1)} \wedge \phi_{\sigma(i_2)} \wedge \ldots \wedge
\phi_{\sigma(i_k)} \right)\\
&= \sum_{1\leq i_1 < i_2 < \ldots < i_k \leq n}
\left( k! \sum_{\sigma \in S_k} \sgn(\sigma)\, a_{\sigma(i_1),\sigma(i_2),\ldots,
\sigma(i_k)} \right) \,
\phi_{i_1} \wedge \phi_{i_2} \wedge \ldots \wedge \phi_{i_k} \ .
\end{align*}

\stage{ii} The proof that the elements of $\NNN$ are linearly
independent is similar to the proof that the elements of $\MM$ are
linearly independent in Theorem~\ref{stokesBT}.
We first note that
\begin{align*}
&(\phi_{i_1} \wedge \phi_{i_2} \wedge \ldots \wedge \phi_{i_k})
(\VEC{v}_{j_1}, \VEC{v}_{j_2}, \ldots, \VEC{v}_{j_k}) \\
&\qquad = \sum_{\sigma \in S_k}
\sgn(\sigma) \, \phi_{i_1}(\VEC{v}_{\sigma(j_1)}) \otimes
\phi_{i_2}(\VEC{v}_{\sigma(j_2)}) \otimes \ldots \otimes
\phi_{i_k}(\VEC{v}_{\sigma(j_k)}) \nonumber \\
&\qquad = \begin{cases}
\sgn(\sigma) & \quad \text{if} \
\sigma(j_m) = i_m \ \text{for} \ 1\leq m \leq k \\
0 & \quad \text{otherwise}
\end{cases}
\end{align*}
for $1 \leq i_m,j_m \leq n$ and $1\leq m \leq k$.
In particular, for $1\leq i_1 < i_2 < \ldots < i_k \leq n$ and
$1\leq j_1 < j_2 < \ldots < j_k \leq n$, we get
\begin{align}
(\phi_{i_1} \wedge \phi_{i_2} \wedge \ldots \wedge \phi_{i_k})
(\VEC{v}_{j_1}, \VEC{v}_{j_2}, \ldots, \VEC{v}_{j_k})
= \begin{cases}
1 & \quad \text{if} \ j_m = i_m \ \text{for} \ 1\leq m \leq k \\
0 & \quad \text{otherwise}
\end{cases}  \label{thOkbasisEq1}
\end{align}

Suppose that
\[
\sum_{1 \leq i_1 < i_2 < \ldots < i_k \leq n}
a_{i_1,i_2,\ldots,i_k} \,
\phi_{i_1} \wedge \phi_{i_2} \wedge \ldots \wedge \phi_{i_k} = 0 \ .
\]
If we evaluate this expression at
$(\VEC{v}_{m_1}, \VEC{v}_{m_2}, \ldots, \VEC{v}_{m_k})$
with $1 \leq m_1 < m_2 < \ldots < m_k \leq n$ and use
(\ref{thOkbasisEq1}), then we get $a_{m_1,m_2,\ldots,m_k} =0$.

\stage{iii}
To determine the number of elements in the basis $\NNN$, we first note
that there are $n (n-1) (n-2)\cdots (n-k+1) = n!/(n-k)!$ ways to select
$k$ distinct $\phi_i$ from $\displaystyle \BB^\ast$.  However, for
each set of indices $1 \leq i_1 < i_2 < \ldots < i_k\leq n$, all $k!$
combinations
$\displaystyle \{ \phi_{i_{\sigma(j)}} \}_{j=1}^k$ for $\sigma \in S_k$
are included in the previous count while only the ordering
$i_1 < i_2 < \ldots <i_k$ matters for the basis $\NNN$.
Therefore, the number of elements of $\NNN$ is
$n!/((n-k)!k!)$.
\end{proof}

\begin{theorem} \label{stokesTBE}
Let $\displaystyle \B = \{ \VEC{v}_i \}_{i=1}^n$ be a basis of the
vector space $H$ over $\RR$.  If $\displaystyle \tau \in \Omega^n(H)$ and
$\displaystyle \VEC{w}_i = \sum_{i=1}^n a_{j,i} \VEC{v}_j$
for $1 \leq i \leq n$ are $n$ vectors in $H$, then
\begin{equation}\label{stokesBE}
\tau(\VEC{w}_1, \VEC{w}_2, \ldots ,\VEC{w}_n) = \det(A)
\, \tau(\VEC{v}_1, \VEC{v}_2, \ldots ,\VEC{v}_n) \ ,
\end{equation}
where the matrix $A$ has entries $a_{j,i}$ for $1\leq i,j \leq n$.
\end{theorem}

\begin{proof}
The crux of this proof is the basic result from linear algebra that
all alternating multilinear maps
\[
\rho: \underbrace{\RR^n \times \RR^n \times \cdots \times
\RR^n}_{n \text{ times}} \to \RR^n \ ,
\]
namely $\displaystyle \rho \in \Omega^n(\RR^n)$,
are of the form $\rho(\VEC{x}_1,\VEC{x}_2, \ldots, \VEC{x}_n) = b\,
\det \begin{pmatrix} \VEC{x}_1 & \VEC{x}_2 & \ldots & \VEC{x}_n \end{pmatrix}$
for some $b \in \RR$.

Consider the alternating multilinear map
$\displaystyle \rho:\RR^n \times \ldots \times \RR^n \to \RR$ defined by
\[
\rho\left(\begin{pmatrix} a_{1,1} \\ a_{2,1} \\ \vdots \\ a_{n,1} \end{pmatrix},
\begin{pmatrix} a_{1,2} \\ a_{2,2} \\ \vdots \\ a_{n,2} \end{pmatrix},
\ldots,
\begin{pmatrix} a_{1,n} \\ a_{2,n} \\ \vdots \\ a_{n,n} \end{pmatrix}
\right)
= \tau\left( \sum_{i=1}^n a_{i,1} \VEC{v}_i , \sum_{i=1}^n a_{i,2} \VEC{v}_i,
\ldots, \sum_{i=1}^n a_{i,n} \VEC{v}_i \right) \ .
\]
Then, from our initial remark,
\[
\tau\left( \sum_{i=1}^n a_{i,1} \VEC{v}_i , \sum_{i=1}^n a_{i,2} \VEC{v}_i,
\ldots, \sum_{i=1}^n a_{i,n} \VEC{v}_i \right)
= b\, \det \begin{pmatrix} a_{1,1} & a_{1,2} & \ldots & a_{1,n} \\
a_{2,1} & a_{2,2} & \ldots & a_{2,n} \\
\vdots & \vdots & \ddots & \vdots \\
a_{n,1} & a_{n,2} & \ldots & a_{n,n} \end{pmatrix}
\]
for some $b\in \RR$.   With $a_{j,i} = \delta_{j,i}$, we get
$\tau( \VEC{v}_1 , \VEC{v}_2, \ldots, \VEC{v}_n) = b\, \det \Id_n = b$.
\end{proof}

\begin{cor}
Suppose that $\displaystyle \tau \in \Omega^k(H)$ and
$\VEC{w}_i \in H$ for $1\leq i \leq n$.  Then
\begin{enumerate}
\item $\tau(\VEC{w}_{\sigma(1)}, \VEC{w}_{\sigma(2)}, \ldots,
\VEC{w}_{\sigma(n)})
= \sgn(\sigma)\tau(\VEC{w}_1, \VEC{w}_2, \ldots, \VEC{w}_n)$ for all
$\sigma \in S_n$, and
\item $\tau(\VEC{w}_1, \VEC{w}_2, \ldots, \VEC{w}_n) = 0$ if
$\VEC{w}_p = \VEC{w}_q$ for some $1\leq p < q \leq n$.
\end{enumerate}
\end{cor}

\begin{proof}
\stage{1} A permutation on $J_n =\{1,2,\ldots,n\}$ may be represented
by a matrix with only one $1$ per row and column, and
all the other entries being null.  For instance, the permutation 
$\sigma \in S_4$ defined by $\sigma(1) = 2$, $\sigma(2) = 4$,
$\sigma(3) = 3$ and $\sigma(4)=1$ is represented by the matrix
$\displaystyle M_\sigma = \begin{pmatrix} 0 & 0 & 0 & 1 \\ 1 & 0 & 0 & 0 \\
0 & 0 & 1 & 0 \\ 0 & 1 & 0 & 0 \end{pmatrix}$
\footnote{The matrix representation of $\sigma$ may also be
represented by the transpose of $M_\sigma$.  It all depends if the
multiplication with a vector is from the left or the right.  We have
chosen from the right.}.  The transposition
$\sigma \in S_4$ defined by $\sigma(1) = 3$, $\sigma(2) = 2$,
$\sigma(3) = 1$ and $\sigma(4)=4$ is represented by the matrix
$\displaystyle M_\sigma = \begin{pmatrix} 0 & 0 & 1 & 0 \\ 0 & 1 & 0 & 0 \\
1 & 0 & 0 & 0 \\ 0 & 0 & 0 & 1 \end{pmatrix}$.
We have that $\sgn(\sigma) = \det(M_\sigma)$ \footnote{The reader may
wonder why we did not define the sign of a permutation using the
determinant of its matrix representation.  But, by doing so, we will have
created a vicious circle because the sign of permutations is used to
define the determinant.}.  With this information, the proof of (1) is
now a simple consequence of the previous theorem.

Let $\displaystyle \B = \{ \VEC{v}_i \}_{i=1}^n$ be a basis of the
vector space $H$ over $\RR$.  We may write
$\displaystyle \VEC{w}_i = \sum_{i=1}^n a_{j,i} \VEC{v}_j$
for $1 \leq i \leq n$ and some $a_{j,i} \in \RR$.
Let $A$ be the matrix with entry in position $(j,i)$ given by
$a_{j,i}$ for $1\leq i,j \leq n$.
Since $\displaystyle \VEC{w}_{\sigma(i)} = \sum_{i=1}^n a_{j,\sigma(i)} \VEC{v}_j$
for $1 \leq i \leq n$, we have that the matrix $\tilde{A}$ with entry
in position $(j,i)$ given by $a_{j,\sigma(i)}$ is $\tilde{A} = A M_\sigma$.

Hence, we get from Theorem~\ref{stokesTBE} that
\begin{align*}
\tau(\VEC{w}_{\sigma(1)}, \VEC{w}_{\sigma(2)}, \ldots, \VEC{w}_{\sigma(n)})
&= \det(\tilde{A}) \, \tau(\VEC{v}_1, \VEC{v}_2, \ldots ,\VEC{v}_n)
= \det(A) \det(M_\sigma)  \, \tau(\VEC{v}_1, \VEC{v}_2, \ldots,\VEC{v}_n) \\
&= \sgn{\sigma} \det(A) \, \tau(\VEC{v}_1, \VEC{v}_2, \ldots ,\VEC{v}_n)
= \sgn(\sigma) \tau(\VEC{w}_1, \VEC{w}_2, \ldots ,\VEC{w}_n) \ .
\end{align*}

\stage{2} If $\sigma = \sigma_{p,q}$ is the transposition of $p$ and
$q$, then we get form (1) that
\begin{align*}
\tau(\VEC{w}_1, \VEC{w}_2, \ldots ,\VEC{w}_n)
&= \tau(\VEC{w}_{\sigma(1)}, \VEC{w}_{\sigma(2)}, \ldots, \VEC{w}_{\sigma(n)})
= \sgn(\sigma) \tau(\VEC{w}_1, \VEC{w}_2, \ldots ,\VEC{w}_n) \\
&= - \tau(\VEC{w}_1, \VEC{w}_2, \ldots ,\VEC{w}_n) \ ,
\end{align*}
where the first equality comes from $\VEC{w}_p = \VEC{w}_q$ and
$\sigma(i) = i$ for $i \neq p, q$.
\end{proof}

\section{Orientation} \label{sectOrientRn}

If $\displaystyle \C = \{ \VEC{w}_i \}_{i=1}^n$ in
Theorem~\ref{stokesTBE} is a basis of $H$, then $A$ is the matrix of
change of bases from $\C$ to $\B$.  In particular, $\det(A) \neq 0$.
Hence (\ref{stokesBE}) divides the bases of $H$ into two groups.  Two
bases like $\B$ and $\C$ are in the same group if (\ref{stokesBE}) is
satisfied with $\det(A)>0$.

\begin{defn}
We define an {\bfseries orientation}\index{Orientation} on a vector
space $H$ by selecting one of the two groups of bases described above.
\end{defn}

If $\displaystyle \B = \{ \VEC{v}_i \}_{i=1}^n$ is a basis
belonging to the chosen group for the orientation on $H$, then we denote this
orientation by $[\VEC{v}_1, \VEC{v}_2, \ldots, \VEC{v}_n]$.
The order of the vector is fundamental.  The group of bases not
belonging to the orientation
$[\VEC{v}_1, \VEC{v}_2, \ldots, \VEC{v}_n]$ on $H$ is the opposite
orientation and is denoted $-[\VEC{v}_1, \VEC{v}_2, \ldots, \VEC{v}_n]$.

\begin{defn}
The {\bfseries standard orientation}\index{Standard Orientation} on
$\displaystyle \RR^n$ is given by
$[\VEC{e}_1, \VEC{e}_2, \ldots, \VEC{e}_n]$ where
$\displaystyle \E = \{ \VEC{e}_i \}_{i=1}^n$ is the canonical basis of
$\displaystyle \RR^n$.
\end{defn}

If $\displaystyle \B = \{ \VEC{v}_i \}_{i=1}^n$
and $\displaystyle \C = \{ \VEC{w}_i \}_{i=1}^n$ are two orthonormal bases
with respect to an inner product $\displaystyle \tau\in \T^2(H)$, then
$A$ in Theorem~\ref{stokesTBE} is an orthogonal matrix, namely
$\displaystyle A^\top A = \Id_n$.  This follows from
\begin{align*}
\delta_{i,j} &= \tau(\VEC{w}_i,\VEC{w}_j)
= \tau\left(\sum_{k_1=1}^n a_{k_1,i} \VEC{v}_{k_1},
\sum_{k_2=1}^n a_{k_2,j} \VEC{v}_{k_2}\right)
= \sum_{k_1,k_2=1}^n a_{k_1,i} a_{k_2,j}
\tau\left(\VEC{v}_{k_1}, \VEC{v}_{k_2}\right) \\
&= \sum_{k_1,k_2=1}^n a_{k_1,i} a_{k_2,j} \delta_{k_1,k_2}
= \sum_{k=1}^n a_{k,i} a_{k,j} \ .
\end{align*}
Hence, if $\displaystyle \B = \{ \VEC{v}_i \}_{i=1}^n$
and $\displaystyle \C = \{ \VEC{w}_i \}_{i=1}^n$ are two orthonormal bases
with respect to the inner product $\displaystyle \tau \in \T^2(H)$,
then $\det A = \pm 1$ in Theorem~\ref{stokesTBE}.

Consider the orientation $[\VEC{v}_1, \VEC{v}_2, \ldots, \VEC{v}_n]$
associated to a given orthonormal basis
$\displaystyle \B = \{ \VEC{v}_i \}_{i=1}^n$ on $H$ with respect to
the inner product $\displaystyle \tau \in \T^2(H)$.
Since $\displaystyle \Omega^n(H)$ is of dimension one, there exists a unique
$\displaystyle \nu \in \Omega^n(H)$ such that
$\displaystyle \nu(\VEC{v}_1,\VEC{v}_2, \ldots, \VEC{v}_n) = 1$.
It follows from the previous paragraph that
$\displaystyle \nu(\VEC{w}_1,\VEC{w}_2, \ldots, \VEC{w}_n) = 1$ for all
orthonormal basis $\displaystyle \C = \{ \VEC{w}_i \}_{i=1}^n$ belonging to
the orientation $[\VEC{v}_1, \VEC{v}_2, \ldots, \VEC{v}_n]$.

\begin{defn}
Let $\displaystyle \B = \{ \VEC{v}_i \}_{i=1}^n$ be an orthonormal
basis on $S$ with respect to the inner product
$\displaystyle \tau \in \T^2(H)$.  The unique
$\displaystyle \nu \in \Omega^n(H)$
such that $\displaystyle \nu(\VEC{v}_1,\VEC{v}_2, \ldots, \VEC{v}_n) = 1$
is called the {\bfseries volume element}\index{Volume Element} on $H$
determined by the inner product $\displaystyle \tau \in \T^2(H)$ and
the orientation $[\VEC{v}_1, \VEC{v}_2, \ldots, \VEC{v}_n]$.
\end{defn}

A particular case of volume element is given by 
$\displaystyle \nu = \det \in \Omega^n(\RR^n)$ seen in the proof of
Theorem~\ref{stokesTBE}.  We have that $\det$ is the volume element on
$\displaystyle \RR^n$ determined by the standard inner product
$\tau \equiv \ps{}{}$ on $\displaystyle \RR^n$
and the orientation $[\VEC{e}_1, \VEC{e}_2, \ldots, \VEC{e}_n]$,
where $\displaystyle \E = \{ \VEC{e}_i \}_{i=1}^n$ is the canonical basis of
$\displaystyle \RR^n$.

To conclude this section, we extend the classical cross product in
$\displaystyle \RR^3$ to $\displaystyle \RR^n$ for $n>3$.
Given $\displaystyle \VEC{w}_i \in \RR^n$ for $1\leq i \leq n-1$, let
$\displaystyle g \in \Omega^1(\RR^n)$ be the linear functional defined by
\[
g(\VEC{v}) = \det
\begin{pmatrix} \VEC{w}_1 & \VEC{w}_2 & \ldots & \VEC{w}_{n-1} & \VEC{v}
\end{pmatrix}
% = \det
% \begin{pmatrix} \VEC{w}_1^\top \\ \VEC{w}_2^\top \\ \vdots \\
% \VEC{w}_{n-1}^\top \\ \VEC{v}^\top \end{pmatrix}
\]
for $\displaystyle \VEC{v} \in \RR^n$.
From the theory of representation of linear functionals on
$\displaystyle \RR^n$,
there exists a unique $\displaystyle \VEC{z}\in \RR^n$ such that 
$g(\VEC{v})=\ps{\VEC{v}}{\VEC{z}}$ for all
$\displaystyle \VEC{v}\in \RR^n$.

\begin{defn}
The {\bfseries cross product}\index{Cross Product}
$\VEC{w}_1\times \VEC{w}_2 \times \ldots \times \VEC{w}_{n-1}$
is the unique vector $\VEC{z}$ such that
$\ps{\VEC{v}}{\VEC{z}} =
\det \begin{pmatrix} \VEC{w}_1 & \VEC{w}_2 & \ldots & \VEC{w}_{n-1} & \VEC{v}
\end{pmatrix}$ for all $\displaystyle \VEC{v} \in \RR^n$.
\end{defn}

We have by construction that $\VEC{z}$ is orthogonal to $\VEC{w}_i$ for
$1\leq i \leq n-1$.  The cross product is a multilinear map.  The
reader may be interested to prove that
$\VEC{w}_{\sigma(1)}\times \VEC{w}_{\sigma(2)} \times \ldots \times
\VEC{w}_{\sigma(n-1)} = \sgn(\sigma) \,
\VEC{w}_1\times \VEC{w}_2 \times \ldots \times \VEC{w}_{n-1}$
for $\sigma \in S_{n-1}$.

\section{Tangent Spaces} \label{stokes_TSVS}

\begin{defn}
Let $\displaystyle V \subset \RR^n$ be an open set.  Given $\VEC{v} \in V$, the
{\bfseries tangent space}\index{Tangent Space} of $V$ at $\VEC{v}$ is
the set
\[
\TS_{\VEC{v}} V = \left\{ (\VEC{v},\VEC{x}) : \VEC{x} \in \RR^n \right\} \ .
\]
\end{defn}

The tangent space $\TS_{\VEC{v}} V$ is given the structure of a vector
space by setting
\[
a_1 (\VEC{v},\VEC{x}_1) + a_2 (\VEC{v},\VEC{x}_2)
= (\VEC{v}, a_1\VEC{x}_1 + a_2\VEC{x}_2)
\]
for all $a_i \in \RR$ and $\displaystyle \VEC{x}_i \in \RR^n$.
For each $\VEC{v} \in V$, we have that $\TS_{\VEC{v}}V$ is linearly
isomorphic to $\displaystyle \RR^n$.  The canonical basis
$\displaystyle \{ \VEC{e}_i \}_{i=1}^n$ of $\displaystyle \RR^n$ yields a
{\bfseries canonical basis}\index{Canonical Basis}
$\E$ on $\TS_{\VEC{v}} V$ defined by
$\displaystyle \E = \left\{ (\VEC{v}, \VEC{e}_i) \right\}_{i=1}^n$.

This definition of tangent space for an open subset of $\displaystyle \RR^n$ 
may seem to be unnecessary and cumbersome but we will see later, when
working on manifolds, that it is very useful.

\begin{defn}
Let $\displaystyle V \subset \RR^n$ be an open set and $\VEC{v} \in V$.
We define an inner product $\ps{}{}_{\VEC{v}}$ on $\TS_{\VEC{v}}V$ by
\[
\ps{ (\VEC{v},\VEC{x}_1) }{ (\VEC{v},\VEC{x}_2) }_{\VEC{v}}
= \tau(\VEC{x}_1,\VEC{x}_2)
\]
for all $\displaystyle \VEC{x}_i\in \RR^n$, where 
$\displaystyle \tau \in \T^2(\RR^n)$ is an inner product.
\end{defn}

The most frequent situation in this chapter is when $\tau$ is the
standard inner product on $\displaystyle \RR^n$.

The concept of tensors can be naturally extended to $\TS_{\VEC{v}}V$. 

\begin{defn}
Let $\displaystyle V \subset \RR^n$ be an open set and $\VEC{v} \in V$.
A $k$-tensors $\displaystyle \tau_{\VEC{v}} \in \T^k( \TS_{\VEC{v}} V)$
is defined by
\[
\tau_{\VEC{v}}\big( (\VEC{v},\VEC{x}_1), (\VEC{v},\VEC{x}_2), \ldots,
(\VEC{v},\VEC{x}_k) \big) = \tau(\VEC{x}_1, \VEC{x}_2, \ldots, \VEC{x}_k)
\]
for all $\displaystyle \VEC{x}_i \in \RR^n$, where
$\displaystyle \tau\in \T^k(\RR^n)$.  We say that
$\tau_{\VEC{v}}$ is generated by $\tau$.
\end{defn}

The space $\displaystyle \T^k(\TS_{\VEC{v}}V)$ is a vector
space over $\RR$.  If
$\displaystyle \tau_{\VEC{v}}^{[1]}, \tau_{\VEC{v}}^{[2]} \in \T^k(\TS_{\VEC{v}}V)$
are generated by $\displaystyle \tau^{[1]},\tau^{[2]} \in \T^k(\RR^n)$
respectively, and $a, b \in \RR$, then
$\displaystyle a \tau_{\VEC{v}}^{[1]}+ b \tau_{\VEC{v}}^{[2]}
\in \T(\TS_{\VEC{v}}V)$
is generated by $\displaystyle a \tau^{[1]} + b \tau^{[2]} \in \T(\RR^n)$.
This follows from
\begin{align*}
&(a \tau_{\VEC{v}}^{[1]}+ b \tau_{\VEC{v}}^{[2]})\big( (\VEC{v},\VEC{x}_1),
(\VEC{v},\VEC{x}_2), \ldots, (\VEC{v}, \VEC{x}_k)\big) \\
&\quad = a \tau_{\VEC{v}}^{[1]}\big( (\VEC{v},\VEC{x}_1),
(\VEC{v},\VEC{x}_2), \ldots, (\VEC{v}, \VEC{x}_k)\big)
+ b \tau_{\VEC{v}}^{[2]} \big( (\VEC{v},\VEC{x}_1),
(\VEC{v},\VEC{x}_2), \ldots, (\VEC{v}, \VEC{x}_k)\big) \\
&\quad = a \tau^{[1]}\big(\VEC{x}_1,\VEC{x}_2, \ldots,\VEC{x}_k\big)
+ b \tau^{[2]} \big( \VEC{x}_1, \VEC{x}_2, \ldots, \VEC{x}_k\big)
= (a \tau^{[1]} + b \tau^{[2]})
\big( \VEC{x}_1, \VEC{x}_2, \ldots, \VEC{x}_k\big)
\end{align*}
for all $\displaystyle \VEC{x}_i \in \RR^n$ and $1 \leq i \leq k$.

We can also extend the product of tensors to $\TS_{\VEC{v}}V$ as it
follows.

\begin{defn}
Let $\displaystyle V \subset \RR^n$ be an open set and $\VEC{v} \in V$.
The {\bfseries product of two tensors}\index{Tensor!Product}
$\displaystyle \tau_{\VEC{v}}^{[1]} \in \T^{k_1}(\TS_{\VEC{v}}V)$ and
$\displaystyle \tau_{\VEC{v}}^{[2]} \in \T^{k_2}(\TS_{\VEC{v}}V)$ is the tensor
$\displaystyle \tau_{\VEC{v}}^{[1]}\otimes \tau_{\VEC{v}}^{[2]} \in
\T^{k_1+k_2}(\TS_{\VEC{v}}V)$ defined by
\begin{align*}
&(\tau_{\VEC{v}}^{[1]}\otimes \tau_{\VEC{v}}^{[2]}) \big( (\VEC{v},\VEC{x}_1),
(\VEC{v},\VEC{x}_2), \ldots, (\VEC{v}, \VEC{x}_{k+1+k_2})\big) \\
&\quad = \tau_{\VEC{v}}^{[1]}
\big((\VEC{v},\VEC{x}_1), (\VEC{v},\VEC{x}_2),
\ldots, (\VEC{v},\VEC{x}_{k_1}) \big)
\, \tau_{\VEC{v}}^{[2]} \big( (\VEC{v},\VEC{x}_{k_1+1}),
(\VEC{v},\VEC{x}_{k_1+2}), \ldots, (\VEC{v},\VEC{x}_{k_1+k_2}) \big)
\end{align*}
for all $\displaystyle \VEC{x}_i \in \RR^n$.
\end{defn}

If $\displaystyle \tau_{\VEC{v}}^{[i]} \in \T^{k_i}(\TS_{\VEC{v}}V)$
is generated by $\displaystyle \tau^{[i]} \in \T^{k_1}(\RR^n)$
for $1\leq i \leq 2$, then
$\displaystyle \tau_{\VEC{v}}^{[1]}\otimes \tau_{\VEC{v}}^{[2]} \in
\T^{k_1+k_2}(\TS_{\VEC{v}}V)$
is generated by
$\displaystyle \tau^{[1]}\otimes \tau^{[2]} \in \T^{k_1+k_2}(\RR^n)$.
This follows from
\begin{align*}
&(\tau_{\VEC{v}}^{[1]}\otimes \tau_{\VEC{v}}^{[2]})\big( (\VEC{v},\VEC{x}_1),
(\VEC{v},\VEC{x}_2), \ldots, (\VEC{v}, \VEC{x}_{k+1+k_2})\big) \\
&\quad = \tau_{\VEC{v}}^{[1]}\big( (\VEC{v},\VEC{x}_1),
(\VEC{v},\VEC{x}_2), \ldots, (\VEC{v}, \VEC{x}_k)\big)
\, \tau_{\VEC{v}}^{[2]} \big( (\VEC{v},\VEC{x}_1),
(\VEC{v},\VEC{x}_2), \ldots, (\VEC{v}, \VEC{x}_k)\big) \\
&\quad = \tau^{[1]}\big(\VEC{x}_1,\VEC{x}_2, \ldots,\VEC{x}_k\big)
\, \tau^{[2]} \big( \VEC{x}_1, \VEC{x}_2, \ldots, \VEC{x}_k\big)
= (\tau_1 \otimes \tau_2) \big( \VEC{x}_1, \VEC{x}_2, \ldots, \VEC{x}_k\big)
\end{align*}
for all $\displaystyle \VEC{x}_i \in \RR^n$.

With our definition of tensors on $\TS_{\VEC{v}}V$, we may extend the
wedge product of alternating tensors on $\TS_{\VEC{v}}V$.

\begin{defn}
Let $\displaystyle V \subset \RR^n$ be an open set and $\VEC{v} \in V$.
The {\bfseries wedge product of two alternating tensors}\index{Tensor!Product}
$\displaystyle \tau_{\VEC{v}}^{[1]} \in \Omega^{k_1}(\TS_{\VEC{v}}V)$ and
$\displaystyle \tau_{\VEC{v}}^{[2]} \in \Omega^{k_2}(\TS_{\VEC{v}}V)$
is the alternating tensor
$\displaystyle \tau_{\VEC{v}}^{[1]}\wedge \tau_{\VEC{v}}^{[2]} \in
\Omega^{k_1+k_2}(\TS_{\VEC{v}}V)$ defined by
\begin{align*}
&(\tau_{\VEC{v}}^{[1]}\wedge \tau_{\VEC{v}}^{[2]}) \big( (\VEC{v},\VEC{x}_1),
(\VEC{v},\VEC{x}_2), \ldots, (\VEC{v}, \VEC{x}_{k+1+k_2})\big) \\
&\quad = \frac{(k_1+k_2)!}{k_1! k_2!} \alt(\tau_{\VEC{v}}^{[1]}
\otimes \tau_{\VEC{v}}^{[2]})
\big((\VEC{v},\VEC{x}_1), (\VEC{v},\VEC{x}_2), \ldots,
 (\VEC{v},\VEC{x}_{k_1+k_2}) \big)
\end{align*}
for all $\displaystyle \VEC{x}_i \in \RR^n$.
\end{defn}

Suppose that
$\displaystyle \tau_{\VEC{v}}^{[i]} \in \Omega^{k_i}(\TS_{\VEC{v}}V)$ is
generated by $\displaystyle \tau^{[i]} \in \Omega^{k_i}(\RR^n)$ for
$1\leq i \leq 2$.
Then $\displaystyle \tau_{\VEC{v}}^{[1]} \wedge \tau_{\VEC{v}}^{[2]} \in
\Omega^{k_1+k_2}(\TS_{\VEC{v}}V)$ is generated by
$\displaystyle \tau^{[1]} \wedge \tau^{[2]} \in \Omega^{k_1+k_2}(\RR^n)$.
This follows from
\begin{align*}
& (\tau_{\VEC{v}}^{[1]} \wedge \tau_{\VEC{v}}^{[2]})\big( (\VEC{v},\VEC{x}_1),
(\VEC{v},\VEC{x}_2), \ldots, (\VEC{v},\VEC{x}_{k_1+k_2}) \big) \\
&\quad =\frac{(k_1+k_2)!}{k_1!\,k_2!}
\alt(\tau_{\VEC{v}}^{[1]} \otimes \tau_{\VEC{v}}^{[2]})\big( (\VEC{v},\VEC{x}_1),
(\VEC{v},\VEC{x}_2), \ldots, (\VEC{v},\VEC{x}_{k_1+k_2}) \big) \\
&\quad = \frac{1}{k_1!\,k_2!}
\sum_{\sigma \in S_{k_1+k_2}} (\tau_{\VEC{v}}^{[1]} \otimes \tau_{\VEC{v}}^{[2]})
\big((\VEC{v},\VEC{x}_{\sigma(1)}), (\VEC{v},\VEC{x}_{\sigma(2)}),
\ldots, (\VEC{v},\VEC{x}_{\sigma(k_1+k_2)}) \big) \\
&\quad = \frac{1}{k_1!\,k_2!}
\sum_{\sigma \in S_{k_1+k_2}} \tau_{\VEC{v}}^{[1]}
\big((\VEC{v},\VEC{x}_{\sigma(1)}), (\VEC{v},\VEC{x}_{\sigma(2)}),
\ldots, (\VEC{v},\VEC{x}_{\sigma(k_1)}) \big) \\
&\hspace{8em} \tau_{\VEC{v}}^{[2]} \big( (\VEC{v},\VEC{x}_{\sigma(k_1+1)}),
(\VEC{v},\VEC{x}_{\sigma(k_1+2)}), \ldots,
(\VEC{v},\VEC{x}_{\sigma(k_1+k_2)}) \big) \\
&\quad = \frac{1}{k_1!\,k_2!}
\sum_{\sigma \in S_{k_1+k_2}} \tau^{[1]}
\big(\VEC{x}_{\sigma(1)}, \VEC{x}_{\sigma(2)}, \ldots,
\VEC{x}_{\sigma(k_1)} \big)
\, \tau^{[2]} \big( \VEC{x}_{\sigma(k_1+1)}, \VEC{x}_{\sigma(k_1+2)}, \ldots,
\VEC{x}_{\sigma(k_1+k_2)} \big) \\
&\quad
= \frac{(k_1+k_2)!}{k_1!\,k_2!}
\alt(\tau^{[1]} \otimes \tau^{[2]})\big(\VEC{x}_1, \VEC{x}_2,
\ldots, \VEC{x}_{k_1+k_2}\big)
= (\tau^{[1]} \wedge \tau^{[2]})\big(\VEC{x}_1, \VEC{x}_2,
\ldots, \VEC{x}_{k_1+k_2}\big)
\end{align*}
for all $\displaystyle \VEC{x}_i \in \RR^n$.

Suppose that $\displaystyle V\subset \RR^n$ and
$\displaystyle W \subset \RR^m$ are two open sets.
If $f:V \to W$ is a function of class $\displaystyle C^1$ and
$\VEC{v}\in V$, then $\diff f(\VEC{v})$ can be used to define the
following linear map.
\begin{align*}
f_\ast: \TS_{\VEC{v}} V &\to \TS_{f(\VEC{v})} W \\
(\VEC{v},\VEC{x}) &\mapsto \big(f(\VEC{v}) , \diff f(\VEC{v}) \VEC{x}\big)
\end{align*}
for each $\VEC{v} \in V$.  This linear map induces a map
$\displaystyle f^\ast: \T^k(\TS_{f(\VEC{v})} W) \to \T^k(\TS_{\VEC{v}} V)$
defined by
\begin{align*}
f^\ast(\tau_{f(\VEC{v})})\big((\VEC{v},\VEC{x}_1), (\VEC{v},\VEC{x}_2), \ldots,
(\VEC{v},\VEC{x}_k) \big)
= \tau_{f(\VEC{v})}\big( f_\ast(\VEC{v},\VEC{x}_1), f_\ast(\VEC{v},\VEC{x}_2),
\ldots, f_\ast(\VEC{v},\VEC{x}_k) \big) \ .
\end{align*}
for all $\displaystyle \VEC{x}_i \in \RR^n$.
We say that $\displaystyle f^\ast(\tau_{f(\VEC{v})})$ is the
{\bfseries pull-back}\index{Pull-Back} of $\tau_{f(\VEC{v})}$ by $f$.

\begin{prop}  \label{propAsttauR}
Suppose that $\displaystyle V\subset \RR^n$ and
$\displaystyle W \subset \RR^m$ are two open sets.
If $f:V \to W$ is a function of class $\displaystyle C^1$ and
$\VEC{v} \in V$, then
\begin{enumerate}
\item $\displaystyle f^\ast(\tau_{f(\VEC{v})}^{[1]} + \tau_{f(\VEC{v})}^{[2]})
= f^\ast(\tau_{f(\VEC{v})}^{[1]}) + f^\ast(\tau_{f(\VEC{v})}^{[2]})$ for all
$\displaystyle \tau_{f(\VEC{v})}^{[i]} \in \T^k(\TS_{f(\VEC{v})}W)$.
\item $\displaystyle f^\ast( a\, \tau_{f(\VEC{v})}) =
a f^\ast(\tau_{f(\VEC{v})})$
for all $\displaystyle \tau_{f(\VEC{v})} \in \T^k(\TS_{f(\VEC{v})}W)$
and $a \in \RR$.
\item $\displaystyle f^\ast( \tau_{f(\VEC{v})}^{[1]} \otimes
\tau_{f(\VEC{v})}^{[2]})
= f^\ast(\tau_{f(\VEC{v})}^{[1]}) \otimes f^\ast(\tau_{f(\VEC{v})}^{[2]})$ for
all $\displaystyle \tau_{f(\VEC{v})}^{[i]} \in \T^{k_i}(\TS_{f(\VEC{v})} W)$.
\item $\displaystyle f^\ast( \tau_{f(\VEC{v})}^{[1]}
\wedge \tau_{f(\VEC{v})}^{[2]})
= f^\ast(\tau_{f(\VEC{v})}^{[1]}) \wedge f^\ast(\tau_{f(\VEC{v})}^{[2]})$ for
all $\displaystyle \tau_{f(\VEC{v})}^{[i]} \in \Omega^{k_i}(\TS_{f(\VEC{v})} W)$.
\end{enumerate}
\end{prop}

\begin{proof}
We prove (2) and (4) and leave the proof of (1) and (3) to the reader.

\stage{2} For $\VEC{v} \in V$ fixed, we have
\begin{align*}
&f^\ast(a\, \tau_{f(\VEC{v})})
\big( (\VEC{v},\VEC{x}_1) , (\VEC{v},\VEC{x}_2) , \ldots,
(\VEC{v},\VEC{x}_k)\big)
= (a\, \tau_{f(\VEC{v})})
\big( f_\ast(\VEC{v},\VEC{x}_1) , f_\ast(\VEC{v},\VEC{x}_2) , \ldots,
f_\ast(\VEC{v},\VEC{x}_k)\big) \\
&\quad = a\left( \tau_{f(\VEC{v})}\big( f_\ast(\VEC{v},\VEC{x}_1) ,
f_\ast(\VEC{v},\VEC{x}_2) , \ldots, f_\ast(\VEC{v},\VEC{x}_k)\big) \right) \\
&\quad = a \left( f^\ast(\tau_{f(\VEC{v})})
\big( (\VEC{v},\VEC{x}_1) , (\VEC{v},\VEC{x}_2) , \ldots,
(\VEC{v},\VEC{x}_k)\big) \right)
\end{align*}
for all $\displaystyle \VEC{x}_i \in \RR^n$.

\stage{4} For $\VEC{v} \in V$ fixed, we have
\begin{align*}
&f^\ast(\tau_{f(\VEC{v})}^{[1]} \wedge \tau_{f(\VEC{v})}^{[2]})
\big( (\VEC{v},\VEC{x}_1) , (\VEC{v},\VEC{x}_2) , \ldots,
(\VEC{v},\VEC{x}_{k_1+k_2})\big) \\
&\quad = \big(\tau_{f(\VEC{v})}^{[1]}\wedge \tau_{f(\VEC{v})}^{[2]} \big)
\big( f_\ast(\VEC{v},\VEC{x}_1) , f_\ast(\VEC{v},\VEC{x}_2) , \ldots,
f_\ast(\VEC{v},\VEC{x}_{k_1+k_2})\big) \\
&\quad = \frac{1}{k_1!\,k_2!} \sum_{\sigma \in S_{k_1+k_2}}
\big( \tau_{f(\VEC{v})}^{[1]} \otimes \tau_{f(\VEC{v})}^{[2]} \big)
\big( f_\ast(\VEC{v},\VEC{x}_{\sigma(1)}),
f_\ast(\VEC{v},\VEC{x}_{\sigma(2)}), \ldots,
f_\ast(\VEC{v},\VEC{x}_{\sigma(k_1+k_2)})\big) \\
&\quad = \frac{1}{k_1!\,k_2!} \sum_{\sigma \in S_{k_1+k_2}}
\tau_{f(\VEC{v})}^{[1]} \big( f_\ast(\VEC{v},\VEC{x}_{\sigma(1)}),
f_\ast(\VEC{v},\VEC{x}_{\sigma(2)}), \ldots,
f_\ast(\VEC{v},\VEC{x}_{\sigma(k_1)})\big) \\
& \hspace{8em}
\tau_{f(\VEC{v})}^{[2]}\big( f_\ast(\VEC{v},\VEC{x}_{\sigma(k_1+1)}),
f_\ast(\VEC{v},\VEC{x}_{\sigma(k_1+2)}), \ldots,
f_\ast(\VEC{v},\VEC{x}_{\sigma(k_1+k_2)})\big) \\
&\quad = \frac{1}{k_1!\,k_2!} \sum_{\sigma \in S_{k_1+k_2}}
f^\ast(\tau_{f(\VEC{v})}^{[1]})
\big( (\VEC{v},\VEC{x}_{\sigma(1)}), (\VEC{v},\VEC{x}_{\sigma(2)}), \ldots,
(\VEC{v},\VEC{x}_{\sigma(k_1)})\big) \\
& \hspace{8em}
f^\ast(\tau_{f(\VEC{v})}^{[2]}) \big( (\VEC{v},\VEC{x}_{\sigma(k_1+1)}),
(\VEC{v},\VEC{x}_{\sigma(k_1+2)}), \ldots,
(\VEC{v},\VEC{x}_{\sigma(k_1+k_2)})\big) \\
&=\quad 
\big( f^\ast(\tau_{f(\VEC{v})}^{[1]}) \wedge
f^\ast(\tau_{f(\VEC{v})}^{[2]}) \big)\, \big( (\VEC{v},\VEC{x}_1),
(\VEC{v},\VEC{x}_2), \ldots, (\VEC{v},\VEC{x}_{\sigma(k_1+k_2)})\big)
\end{align*}
for all $\displaystyle \VEC{x}_i \in \RR^n$.
\end{proof}

\section{Vector Fields}

\begin{defn}
A {\bfseries vector field}\index{Vector Field} on an open set
$\displaystyle V \subset \RR^n$ is a map
\begin{align*}
F : V & \to \bigcup_{\VEC{v}\in V} \TS_{\VEC{v}} V \\
 \VEC{v} & \mapsto (\VEC{v}, f(\VEC{v}) ) \in \TS_{\VEC{v}} V
\end{align*}
where $\displaystyle f:V \to \RR^n$.
\end{defn}

It is important to note that $F(\VEC{v}) \in \TS_{\VEC{v}} V$ for each
$\VEC{v} \in V$.  We have
\[
F(\VEC{v}) = \left(\VEC{v} , \sum_{i=1}^n f_i(\VEC{v}) \VEC{e}_i\right)
= \sum_{i=1}^n f_i(\VEC{v}) (\VEC{v}, \VEC{e}_i)
\]
for all $\VEC{v} \in V$, where $f_i:V \to \RR$ for $1 \leq i \leq n$.
We say that the vector field $F$ is of class
$\displaystyle C^k$ if each function $f_i$ is of class
$\displaystyle C^k$.

The sum of vector fields, the product of a vector field by a real
valued function on $V$, the inner product of vector fields and the cross
product of vector fields are defined by the respective operation in
$\TS_{\VEC{v}} V$ for every $\VEC{v} \in V$.  For instance, if
$\displaystyle F,G:V \to \bigcup_{\VEC{v} \in V} \TS_{\VEC{v}} V$ are two
vector fields, then $\ps{F}{G}$ is defined by
$\ps{F}{G}(\VEC{v})
= \ps{F(\VEC{v})}{G(\VEC{v})}_{\VEC{v}} = \tau(f(\VEC{v}),g(\VEC{v}))$
for all $\VEC{v} \in V$, where $\displaystyle \tau \in \T^2(\RR^n)$ is
an inner product.

We have introduced this notation for vector fields because it will extend
naturally to vector fields on manifolds as we will see later.

\section{Differential Forms} \label{stokesDefDiffF}

\begin{defn}
Let $V$ be an open subset of $\displaystyle \RR^n$.  A
{\bfseries differential $\mathbf{k}$-form}\index{Differential Form!$k$-Form}
on $V$ is a map
\begin{align*}
\omega: V & \to \bigcup_{\VEC{v}\in V} \Omega^k\left( \TS_{\VEC{v}} V \right) \\
\VEC{v} & \mapsto \omega(\VEC{v}) \in \Omega^k\left( \TS_{\VEC{v}} V \right)
\end{align*}
\end{defn}

It is important to note that $\displaystyle \omega(\VEC{v}) \in
\Omega^k(\TS_{\VEC{v}} V)$ for each $\VEC{v} \in V$.

Let $\displaystyle \left\{\phi_i\right\}_{i=1}^n \subset (\RR^n)^\ast$
be the dual basis associated to a basis
$\displaystyle \left\{\VEC{u}_j\right\}_{j=1}^n$ of $\displaystyle \RR^n$.
For each $i$, we define a differential $1$-form $\tilde{\phi}_i$ on
$V$ by
$\displaystyle \tilde{\phi}_i(\VEC{v})\big( (\VEC{v}, \VEC{x}) \big)
= \phi_i(\VEC{x})$ for each $\VEC{v} \in V$ and all
$\displaystyle \VEC{x} \in \RR^n$.
For each $\VEC{v} \in V$, the set
$\displaystyle \left\{\tilde{\phi}_i(\VEC{v})\right\}_{i=1}^n
\subset (\TS_{\VEC{v}}V)^\ast = \Omega^1(\TS_{\VEC{v}}V)$ is the dual basis
associated to the basis
$\displaystyle \left\{ (\VEC{v}, \VEC{u}_j) \right\}_{j=1}^n$ of
$\displaystyle \TS_{\VEC{v}} V$ because
$\tilde{\phi}_i(\VEC{v})\big( (\VEC{v}, \VEC{u}_j) \big)
= \phi_i(\VEC{u}_j) = \delta_{i,j}$
for $1 \leq i,j \leq n$.
It follows from Theorem~\ref{thOkbasis} that every differential
$k$-form $\omega$ on $V$ can be expressed as
\[
\omega = \sum_{i_1< i_2< \ldots< i_k} \omega_{i_1,i_2,\ldots,i_k}
\tilde{\phi}_{i_1} \wedge \tilde{\phi}_{i_2} \wedge \ldots \wedge
\tilde{\phi}_{i_k} \ ,
\]
where $\displaystyle \omega_{i_1,i_2,\ldots,i_k}:V\to \RR$ for all indices.
We say that the differential $k$-form is of class $\displaystyle C^m$
if each $\displaystyle \omega_{i_1,i_2,\ldots,i_k}$ is of class
$\displaystyle C^m$.

The sum of differential $k$-forms, the product of a differential
$k$-form by a real valued function on $V$ and the wedge product of a
differential $k_1$-form with a differential $k_2$-form
are defined by the respective operation on $\TS_{\VEC{v}} V$ for each
$\VEC{v} \in V$.
For instance, if $\omega_i$ is a differential $k_i$-form on $V$ for
$1\leq i \leq 2$, then $\omega_1 \wedge \omega_2$ is the differential
$(k_1+k_2)$-form on $V$ defined by $(\omega_1 \wedge \omega_2)(\VEC{v}) = 
\omega_1(\VEC{v}) \wedge \omega_2(\VEC{v})$
for all $\VEC{v} \in V$.

If $f:V \to \RR$ is differentiable, then
$\displaystyle \diff f (\VEC{v}) \in (\RR^n)^\ast$.
As we have done with $\phi_i$ above, we may define the following
differential $1$-form on $V$.

\begin{defn} \label{defnd01}
Given $f:V\to \RR$ is of class $\displaystyle C^1$, the differential $1$-form
$\displaystyle \df{f} :V \to \bigcup_{\VEC{v}\in V}
\Omega^1\left( \TS_{\VEC{v}} V \right)$ is defined by
$\df{f}(\VEC{v}) \, \big((\VEC{v},\VEC{x})\big)
= \diff f (\VEC{v}) \VEC{x}$
for each $\VEC{v}\in V$ and all $\displaystyle \VEC{x} \in \RR^n$.
\end{defn}

If $\displaystyle \pi_i:\RR^n \to \RR$ is the projection defined by
$\pi_i(\VEC{x}) = x_i$ for all $\displaystyle \VEC{x} \in \RR^n$, then
the differential $1$-form $\df{\pi_i}$ is defined by
$\displaystyle \df{\pi_i}(\VEC{v}) \big( (\VEC{v},\VEC{x}) \big) =
\diff \pi_i (\VEC{v}) \VEC{x} = \pi_i(\VEC{x}) = x_i$
for each $\VEC{v} \in V$ and all $\displaystyle \VEC{x} \in \RR^n$.
Thus $\displaystyle \df{\pi_i}(\VEC{v}) \in \Omega^1(\TS_\VEC{v} V)$ 
is generated by $\displaystyle \pi_i \in \Omega^1(\RR^n)$ for all
$\VEC{v} \in V$.

For each $\VEC{v} \in V$, the dual basis of the canonical basis
$\displaystyle \E = \left\{ (\VEC{v}, \VEC{e}_i) \right\}_{i=1}^n \subset
\TS_{\VEC{v}} V$ is
$\displaystyle \left\{ \df{\pi_i}(\VEC{v}) \right\}_{i=1}^n \subset
(\TS_{\VEC{v}}V)^\ast$.
By tradition, the differential $1$-form $\df{\pi_i}$ is denoted
$\df{x_i}$.  Hence, every differential $k$-form $\omega$ on $V$ can be
expressed as
\[
\omega = \sum_{i_1< i_2< \ldots< i_k} \omega_{i_1,i_2,\ldots,i_k}
\df{x_{i_1}} \wedge \df{x_{i_2}} \wedge \ldots \wedge \df{x_{i_k}} \ ,
\]
where $\displaystyle \omega_{i_1,i_2,\ldots,i_k}:V\to \RR$ for all indices.

It is important to note that
\begin{equation} \label{dxPropr1}
\df{x_{i_1}} \wedge \df{x_{i_2}} \wedge \ldots \wedge \df{x_{i_k}} = 0
\end{equation}
if there are two identical indices, and that
\begin{equation} \label{dxPropr2}
\df{x_{\sigma(i_1)}} \wedge \df{x_{\sigma(i_2)}} \wedge \ldots
\wedge \df{x_{\sigma(i_k)}} = \sgn(\sigma) 
\df{x_{i_1}} \wedge \df{x_{i_2}} \wedge \ldots \wedge \df{x_{i_k}}
\end{equation}
for all permutations $\sigma \in S_k$.  These results follow from
Proposition~\ref{propTssT}.  For instance,
\begin{align*}
&\big( \df{x_{\sigma(i_1)}} \wedge \df{x_{\sigma(i_2)}} \wedge \ldots
\wedge \df{x_{\sigma(i_k)}} \big)(\VEC{v}) \, \big( (\VEC{v},\VEC{w}_1), 
(\VEC{v},\VEC{w}_2), \ldots , (\VEC{v},\VEC{w}_k) \big) \\
&\qquad = \big( \pi_{\sigma(i_1)} \wedge \pi_{\sigma(i_2)} \wedge \ldots
\wedge \pi_{\sigma(i_k)} \big)\, \big( \VEC{w}_1, 
\VEC{w}_2, \ldots , \VEC{w}_k \big) \\
&\qquad =\sgn{\sigma} \big( \pi_{i_1} \wedge \pi_{i_2} \wedge \ldots
\wedge \pi_{i_k} \big)\, \big( \VEC{w}_1, 
\VEC{w}_2, \ldots , \VEC{w}_k \big) \\
&\qquad = \sgn(\sigma)
\big( \df{x_{i_1}} \wedge \df{x_{i_2}} \wedge \ldots
\wedge \df{x_{i_k}} \big)(\VEC{v}) \, \big( (\VEC{v},\VEC{w}_1), 
(\VEC{v},\VEC{w}_2), \ldots , (\VEC{v},\VEC{w}_k) \big)
\end{align*}
for all $\VEC{v}\in V$ and $\displaystyle \VEC{w}_i \in \RR^n$,
where we have used Proposition~\ref{propTssT} to obtain the second
equality.

\begin{theorem}  \label{thmdefn01}
Suppose that $\displaystyle V\subset \RR^n$ is an open subset and
$f:V \to \RR$ is of class $\displaystyle C^1$, then
\[
\df{f} = \sum_{j=1}^n \pdydx{f}{x_j} \df{x_j} \ .
\]
\end{theorem}

\begin{proof}
By definition, for every $\VEC{v} \in V$,
\[
\df{f}(\VEC{v})\big( (\VEC{v}, \VEC{x}) \big)
= \diff f(\VEC{v}) \VEC{x}
= \sum_{j=1}^n \pdydx{f}{x_j}(\VEC{v}) x_j
= \sum_{j=1}^n \pdydx{f}{x_j}(\VEC{v})
\df{x_j}(\VEC{v})\big( (\VEC{v}, \VEC{x}) \big)
\]
for all $\displaystyle \VEC{x} \in \RR^n$.
\end{proof}

Suppose that $\displaystyle V\subset \RR^n$ and
$\displaystyle W \subset \RR^m$ are two open sets.
If $f:V \to W$ is a function of class $\displaystyle C^1$,
then we can extend $\displaystyle f^\ast$ to a map from
differential $k$-forms on $W$ to differential $k$-forms on $V$ by
$\displaystyle f^\ast(\omega)(\VEC{v}) = f^\ast(\omega(f(\VEC{v})))$ for
every differential $k$-forms $\omega$ on $W$ and $\VEC{v} \in V$.
We say that $\displaystyle f^\ast(\omega)$ is the
{\bfseries pull-back}\index{Pull-Back} of the differential
$k$-form $\omega$ by $f$.

\begin{prop}  \label{propAstoR}
Suppose that $\displaystyle V\subset \RR^n$ and
$\displaystyle W \subset \RR^m$ are two open sets.
If $f:V \to W$ is a function of class $\displaystyle C^1$, then
\begin{enumerate}
\item $\displaystyle f^\ast(\df{x_i}) = \sum_{j=1}^n \pdydx{f_i}{x_j} \df{x_j}$
for $1 \leq i \leq m$.
\item $\displaystyle f^\ast(\omega_1 + \omega_2) = f^\ast(\omega_1) +
f^\ast(\omega_2)$ for all differential $k$-forms $\omega_i$ on $W$.
\item $\displaystyle f^\ast( g\, \omega) = ( g\circ f)f^\ast(\omega)$
for all differential $k$-form $\omega$ on $W$ and functions
$g:W\to \RR$ of class $\displaystyle C^1$.
\item $\displaystyle f^\ast( \omega_1 \wedge \omega_2) = f^\ast(\omega_1)
\wedge f^\ast(\omega_2)$ for all differential $k_1$-forms $\omega_1$
and differential $k_2$-form $\omega_2$ on $W$
\end{enumerate}
\end{prop}

\begin{proof}
\stage{1} For every $\VEC{v} \in V$, we have
\begin{align*}
&f^\ast(\df{x_i})(\VEC{v})\big( (\VEC{v},\VEC{x}) \big)
= \df{x_i}(f(\VEC{v}))\big( f_\ast(\VEC{v},\VEC{x}) \big)
= \df{x_i}(f(\VEC{v}))\big( f(\VEC{v}) , \diff f(\VEC{v}) \VEC{x} \big)
= \pi_i\big( \diff f(\VEC{v}) \VEC{x} \big) \\
&\quad
= \pi_i\left( \sum_{r=1}^m \left( \sum_{j=1}^n \pdydx{f_r}{x_j}(\VEC{v}) x_j
\right) \VEC{e}_r \right)
= \sum_{j=1}^n \pdydx{f_i}{x_j}(\VEC{v}) x_j
= \sum_{j=1}^n \pdydx{f_i}{x_j}(\VEC{v})
\df{x_j}(\VEC{v})\big( (\VEC{v},\VEC{x}) \big)
\end{align*}
for all $\displaystyle \VEC{x}\in \RR^n$.

\stage{2} This proof is left to the reader.

\stage{3} We have
\begin{align*}
f^\ast(g\, \omega)(\VEC{v})
&= f^\ast \big((g\, \omega)(f(\VEC{v}))\big)
= f^\ast \big(g(f(\VEC{v})) \omega(f(\VEC{v}))\big)
= g(f(\VEC{v})) f^\ast(\omega(f(\VEC{v}))) \\
&= g(f(\VEC{v}))\, (f^\ast \omega)(\VEC{v})
\end{align*}
for all $\VEC{v} \in V$, where we have used (2) of
Proposition~\ref{propAsttauR} for the third equality.

\stage{4} We have
\begin{align*}
f^\ast(\omega_1 \wedge \omega_2)(\VEC{v})
&= f^\ast \big((\omega_1 \wedge \omega_2)(f(\VEC{v}))\big)
= f^\ast \big(\omega_1(f(\VEC{v})) \wedge \omega_2(f(\VEC{v}))\big) \\
&= f^\ast(\omega_1(f(\VEC{v}))) \wedge f^\ast(\omega_2(f(\VEC{v}))) 
=(f^\ast\omega_1)(\VEC{v}) \wedge (f^\ast\omega_2)(\VEC{v}) 
\end{align*}
for all $\VEC{v} \in V$,  where we have used (4) of
Proposition~\ref{propAsttauR} for the third equality.
\end{proof}

\begin{theorem} \label{stokesCofV}
Suppose that $V$ and $W$ are two open subsets of $\displaystyle \RR^n$.
Let $h:W\to \RR$ and $f:V \to W$ be functions of class
$\displaystyle C^1$.  Then
\[
f^\ast\big( h \df{x_1} \wedge \df{x_2} \wedge \ldots \wedge \df{x_n} \big)
= (h \circ f) (\det \diff f)
\df{x_1} \wedge \df{x_2} \wedge \ldots \wedge \df{x_n} \ .
\]
\end{theorem}

\begin{proof}
We have from (3) of Proposition~\ref{propAstoR} that
\begin{equation} \label{stokesCofVEq1}
f^\ast( h \df{x_1} \wedge \df{x_2} \wedge \ldots \wedge \df{x_n} )
= (h \circ f) f^\ast( \df{x_1} \wedge \df{x_2} \wedge \ldots
\wedge \df{x_n}) \ .
\end{equation}

Let $Q = \diff f(\VEC{v})$; namely, the components $q_{i,j}$ of $Q$ for
$1 \leq i, j \leq n$ are given by
$\displaystyle q_{i,j} = \dydx{f_i}{x_j}(\VEC{v})$.
Moreover, let
$\displaystyle \E_{\VEC{v}} = \left\{ (\VEC{v}, \VEC{e}_i) \right\}_{i=1}^n$
be the canonical basis of $\TS_{\VEC{v}}V$.
Recall that $\displaystyle \df{x_i}(\VEC{v}) = \df{\pi_i}(\VEC{v}) \in
\Omega^1(\TS_{\VEC{v}}V)$ for $1\leq i \leq n$
is generated by the projection $\displaystyle \pi_i:\RR^n \to \RR$ defined by
$\pi_i(\VEC{x}) = x_i$ for all $\displaystyle \VEC{x} \in \RR^n$.

Given $\displaystyle \VEC{w}_k \in \RR^n$ for $1\leq k \leq n$, we may write
$\displaystyle \VEC{w}_k = \sum_{j=1}^n a_{j,k} \VEC{e}_j$
for $1 \leq k \leq n$ and some $a_{j,k} \in \RR$ where
$\displaystyle \left\{ \VEC{e}_i \right\}_{i=1}^n$ is the canonical
basis on $\displaystyle \RR^n$.  Let $A$ be the matrix with entries $a_{j,k}$.

We get from Theorem~\ref{stokesTBE} that
\begin{align*}
&\big( f^\ast ( \df{x_1} \wedge \df{x_2} \wedge \ldots
\wedge \df{x_n} ) \big)(\VEC{v}) \, \big( (\VEC{v},\VEC{w}_1), 
(\VEC{v},\VEC{w}_2), \ldots , (\VEC{v},\VEC{w}_n) \big) \\
&\quad = f^\ast\big( (\df{x_1} \wedge \df{x_2} \wedge \ldots
\wedge \df{x_n}) (f(\VEC{v}))\big) \, \big( (\VEC{v},\VEC{w}_1), 
(\VEC{v},\VEC{w}_2), \ldots , (\VEC{v},\VEC{w}_n) \big) \\
&\quad = \big( (\df{x_1} \wedge \df{x_2} \wedge \ldots
\wedge \df{x_n}) (f(\VEC{v}))\big) \, \big( f_\ast(\VEC{v},\VEC{w}_1), 
f_\ast (\VEC{v},\VEC{w}_2), \ldots , f_\ast (\VEC{v},\VEC{w}_n) \big) \\
&\quad = \big( (\df{x_1} \wedge \df{x_2} \wedge \ldots
\wedge \df{x_n}) (f(\VEC{v}))\big) \\
&\hspace{7em} \big( (f(\VEC{v}),\diff f(\VEC{v}) \VEC{w}_1), 
(f(\VEC{v}),\diff f(\VEC{v}) \VEC{w}_2), \ldots ,
(f(\VEC{v}),\diff f(\VEC{v}) \VEC{w}_n) \big) \\
&\quad = ( \pi_1 \wedge \pi_2 \wedge \ldots
\wedge \pi_n) \left( \sum_{j=1}^n a_{j,1} \diff f(\VEC{v}) \VEC{e}_j,
\sum_{j=1}^n a_{j,2} \diff f(\VEC{v}) \VEC{e}_j, \ldots,
\sum_{j=1}^n a_{j,n} \diff f(\VEC{v}) \VEC{e}_j \right) \\
&\quad = \big( \pi_1 \wedge \pi_2 \wedge \ldots \wedge \pi_n\big)
\left( \sum_{i=1}^n \left(\sum_{j=1}^n a_{j,1}
q_{i,j}\right) \VEC{e}_i, \sum_{i=1}^n\left(\sum_{j=1}^n a_{j,2} q_{i,j}\right)
\VEC{e}_i, \ldots, \right. \\
&\hspace{15em} \left. \sum_{i=1}^n \left(\sum_{j=1}^n a_{j,n} q_{i,j}\right)
\VEC{e}_i \right) \\
&\quad = \det(Q A)\, \big( \pi_1 \wedge \pi_2 \wedge \ldots \wedge \pi_n\big)
\, \big( \VEC{e}_1, \VEC{e}_2, \ldots , \VEC{e}_n \big) \\
&\quad = \det(Q)\, \det(A)\,
\big(\pi_1 \wedge \pi_2 \wedge \ldots \wedge \pi_n\big)
\, \big( \VEC{e}_1, \VEC{e}_2, \ldots , \VEC{e}_n \big) \\
&\quad = \det(Q)\,\big( \pi_1 \wedge \pi_2 \wedge \ldots \wedge \pi_n\big)
\, \big( \sum_{j=1}^n a_{j,1} \VEC{e}_j, \sum_{j=1}^n a_{j,2} \VEC{e}_j, \ldots ,
\sum_{j=1}^n a_{j,n} \VEC{e}_j \big) \\
&\quad = \det(Q)\, \big( \pi_1 \wedge \pi_2 \wedge \ldots \wedge \pi_n\big)
\, \big( \VEC{w}_1, \VEC{w}_2, \ldots , \VEC{w}_n \big) \\
&\quad = \det(Q)\, \big( (\df{x_1} \wedge \df{x_2} \wedge \ldots \wedge
\df{x_n})(\VEC{v}) \big) 
\, \big( (\VEC{v},\VEC{w}_1), (\VEC{v},\VEC{w}_2), \ldots ,
(\VEC{v},\VEC{w}_n)  \big)
\end{align*}
for all $\VEC{v} \in V$ and $\displaystyle \VEC{w}_i \in \RR^n$.
Thus
\[
f^\ast( \df{x_1} \wedge \df{x_2} \wedge \ldots
\wedge \df{x_n}) = (\det(\diff f))\, \df{x_1} \wedge \df{x_2} \wedge
\ldots \wedge \df{x_n} \ . \qedhere
\]
\end{proof}

We now define an operator $\df{}$ that will map differential $k$-forms
to differential $(k+1)$-forms.  We have already defined this operator
for function $f:V\to \RR$ in Definition~\ref{defnd01} and
Theorem~\ref{thmdefn01}.

\begin{defn}
Suppose that
\[
\omega = \sum_{1\leq i_1<i_2< \ldots< i_k\leq n} \omega_{i_1,i_2,\ldots,i_k}
\df{x_{i_1}} \wedge \df{x_{i_2}} \wedge \ldots \wedge \df{x_{i_k}}
\]
is a differential $k$-form of class $\displaystyle C^1$ on an open set
$\displaystyle V \subset \RR^n$.  We define a
differential $k+1$-form $\df{\omega}$ on $V$ by
\begin{align*}
\df{\omega} &= \sum_{1\leq i_1 < i_2 < \ldots < i_k\leq n}
\df{\omega_{i_1,i_2,\ldots,i_k}} \wedge
\df{x_{i_1}} \wedge \df{x_{i_2}} \wedge \ldots \wedge \df{x_{i_k}} \\
&= \sum_{1\leq i_1< i_2 < \ldots < i_k} \sum_{j=1}^n
\pdydx{\omega_{i_1,i_2,\ldots,i_k}}{x_j} \df{x_j} \wedge
\df{x_{i_1}} \wedge \df{x_{i_2}} \wedge \ldots \wedge \df{x_{i_k}} \ .
\end{align*}
\end{defn}

\begin{theorem}\label{stokesDF}
Suppose that $\displaystyle V\subset \RR^n$ and
$\displaystyle W \subset \RR^m$ are two open subsets.
\begin{enumerate}
\item $\displaystyle \df{(\omega + \mu )} = \df{\omega} + \df{\mu}$
for all differential $k$-forms $\omega$ and $\mu$ of class
$\displaystyle C^1$ on $V$.
\item $\displaystyle \df{(\omega \wedge \mu)} =
\df{\omega}\wedge \mu + (-1)^p \omega \wedge \df{\mu}$
for all differential $p$-forms $\omega$ and differential $q$-form
$\mu$ of class $\displaystyle C^1$ on $V$.
\item $\df[2]{\omega} \equiv \df{(\df{\omega})} = 0$
for all differential $k$-forms $\omega$ of class
$\displaystyle C^2$ on $V$.
\item $\displaystyle f^\ast (\df{\omega}) = \df{f^\ast(\omega)}$ for all
differential $k$-forms $\omega$ of class $\displaystyle C^1$ on $W$ and all
functions $f:V\to W$ of class $\displaystyle C^1$.
\end{enumerate}
\end{theorem}

\begin{proof}
\stage{1} This proof is left to the reader.

\stage{2} We have that
$\displaystyle \omega = \sum_{1\leq i_1 <i_2 < \ldots < i_p \leq n}
\omega_{i_1,i_2,\ldots,i_p} \df{x_{i_1}} \wedge \df{x_{i_2}} \wedge \ldots
\wedge \df{x_{i_p}}$ and \\
$\displaystyle \mu = \sum_{1\leq j_1 <j_2 < \ldots < j_q \leq n}
\mu_{j_1,j_2,\ldots,j_q} \df{x_{j_1}} \wedge \df{x_{j_2}} \wedge \ldots
\wedge \df{x_{j_q}}$.  Hence
\begin{align*}
&\omega \wedge \mu = \sum_{\substack{1\leq i_1 <i_2 < \ldots < i_p \leq n\\
1\leq j_1 <j_2 < \ldots < j_q \leq n}}
\omega_{i_1,i_2,\ldots,i_p}\, \mu_{j_1,j_2,\ldots,j_q}  \\
&\hspace{8em} \df{x_{i_1}} \wedge \df{x_{i_2}} \wedge \ldots
\wedge \df{x_{i_p}} \wedge \df{x_{j_1}} \wedge \df{x_{j_2}} \wedge \ldots
\wedge \df{x_{j_q}} \ .
\end{align*}
From the definition of the operator $\df{}$, we get
\begin{align*}
\df{\omega} &= \sum_{1\leq i_1 <i_2 < \ldots < i_p \leq n}
\left( \sum_{m=1}^n \pdydx{\omega_{i_1,i_2,\ldots,i_p}}{x_m} \df{x_m}
\right) \wedge \df{x_{i_1}} \wedge \df{x_{i_2}} \wedge \ldots
\wedge \df{x_{i_p}}
\intertext{and}
\df{\mu} &= \sum_{1\leq j_1 <j_2 < \ldots < j_q \leq n}
\left(\sum_{m=1}^n \pdydx{\mu_{j_1,j_2,\ldots,j_q}}{x_m} \df{x_m} \right)
\wedge \df{x_{j_1}} \wedge \df{x_{j_2}} \wedge \ldots
\wedge \df{x_{j_q}} \ .
\end{align*}
Using (1), we get
\begin{align*}
&\df{(\omega \wedge \mu)}
= \sum_{\substack{1\leq i_1 <i_2 < \ldots < i_p \leq n\\
1\leq j_1 <j_2 < \ldots < j_q \leq n}}
\df{\big(\omega_{i_1,i_2,\ldots,i_p} \mu_{j_1,j_2,\ldots,j_q} \\
&\hspace{8em} \df{x_{i_1}} \wedge \df{x_{i_2}} \wedge \ldots
\wedge \df{x_{i_p}} \wedge \df{x_{j_1}} \wedge \df{x_{j_2}} \wedge \ldots
\wedge \df{x_{j_q}}\big)} \\
&\quad = \sum_{\substack{1\leq i_1 <i_2 < \ldots < i_p \leq n\\
1\leq j_1 <j_2 < \ldots < j_q \leq n}}
\bigg( \sum_{m=1}^n \left(\pdydx{\omega_{i_1,i_2,\ldots,i_p}}{x_m}
\mu_{j_1,j_2,\ldots,j_q} + \omega_{i_1,i_2,\ldots,i_p}
\pdydx{\mu_{j_1,j_2,\ldots,j_q}}{x_m} \right) \\
&\hspace{8em}
\df{x_m} \wedge \df{x_{i_1}} \wedge \df{x_{i_2}} \wedge \ldots
\wedge \df{x_{i_p}} \wedge \df{x_{j_1}} \wedge \df{x_{j_2}} \wedge \ldots
\wedge \df{x_{j_q}} \bigg) \\
&\quad = \sum_{\substack{1\leq i_1 <i_2 < \ldots < i_p \leq n\\
1\leq j_1 <j_2 < \ldots < j_q \leq n}}
\left( \sum_{m=1}^n \pdydx{\omega_{i_1,i_2,\ldots,i_p}}{x_m}
\df{x_m} \wedge \df{x_{i_1}} \wedge \df{x_{i_2}} \wedge \ldots
\wedge \df{x_{i_p}}\right) \\
&\hspace{8em} \wedge \left( \mu_{j_1,j_2,\ldots,j_q}
\df{x_{j_1}} \wedge \df{x_{j_2}} \wedge \ldots \wedge \df{x_{j_q}} \right) \\
& \qquad\quad + (-1)^p \sum_{\substack{1\leq i_1 <i_2 < \ldots < i_p \leq n\\
1\leq j_1 <j_2 < \ldots < j_q \leq n}} \left( \omega_{i_1,i_2,\ldots,i_p}
\wedge \df{x_{i_1}} \wedge \df{x_{i_2}} \wedge \ldots
\wedge \df{x_{i_p}} \right) \\
&\hspace{8em} \wedge \left( \sum_{m=1}^n \pdydx{\mu_{j_1,j_2,\ldots,j_q}}{x_m}
\df{x_m} \wedge \df{x_{j_1}} \wedge \df{x_{j_2}} \wedge \ldots
\wedge \df{x_{j_q}}\right) \\
&\quad = \df{\omega} \wedge \mu + (-1)^p \omega \wedge \df{\mu} \ ,
\end{align*}
where the factor of $\displaystyle (-1)^p$ comes from
$\displaystyle \df{x_m} \wedge \df{x_{i_1}} \wedge \df{x_{i_2}} \wedge \ldots
\wedge \df{x_{i_p}} \wedge \df{x_{j_1}} \wedge \df{x_{j_2}} \wedge \ldots
\wedge \df{x_{j_q}} = (-1)^p
\df{x_{i_1}} \wedge \df{x_{i_2}} \wedge \ldots
\wedge \df{x_{i_p}} \wedge \df{x_m} \wedge \df{x_{j_1}} \wedge
\df{x_{j_2}} \wedge \ldots \wedge \df{x_{j_q}}$
according to (\ref{dxPropr2}).

\stage{3} If
$\displaystyle \omega = \sum_{1\leq i_1 <i_2 < \ldots < i_p \leq n}
\omega_{i_1,i_2,\ldots,i_k} \df{x_{i_1}} \wedge \df{x_{i_2}} \wedge \ldots
\wedge \df{x_{i_k}}$, then
\[
\df{\omega} = \sum_{1\leq i_1 <i_2 < \ldots < i_p \leq n}
\sum_{j=1}^n \pdydx{\omega_{i_1,i_2,\ldots,i_k}}{x_j} \df{x_j}
\wedge \df{x_{i_1}} \wedge \df{x_{i_2}} \wedge \ldots
\wedge \df{x_{i_k}}
\]
and
\begin{align*}
\df{(\df{\omega})} &= \sum_{1\leq i_1 <i_2 < \ldots < i_p \leq n}
\sum_{m=1}^n \sum_{j=1}^n
\pdydxnm{\omega_{i_1,i_2,\ldots,i_k}}{x_j}{x_m}{2}{}{}
\df{x_m} \wedge \df{x_j} \wedge \df{x_{i_1}} \wedge
\df{x_{i_2}} \wedge \ldots \wedge \df{x_{i_k}} \\
&=\sum_{1\leq i_1 <i_2 < \ldots < i_p \leq n}
\left( \sum_{m=1}^n \sum_{j=1}^n
\pdydxnm{\omega_{i_1,i_2,\ldots,i_k}}{x_j}{x_m}{2}{}{}
\df{x_m} \wedge \df{x_j} \right) \wedge \df{x_{i_1}} \wedge
  \df{x_{i_2}} \wedge \ldots \wedge \df{x_{i_k}} = 0
\end{align*}
because
$\displaystyle \pdydxnm{\omega_{i_1,i_2,\ldots,i_k}}{x_j}{x_m}{2}{}{}
= \pdydxnm{\omega_{i_1,i_2,\ldots,i_k}}{x_m}{x_j}{2}{}{}$ for all $0
\leq j, m \leq n$ since $\displaystyle \omega_{i_1,i_2,\ldots,i_k} \in C^2(V)$
implies that
\[
\pdydxnm{\omega_{i_1,i_2,\ldots,i_k}}{x_j}{x_m}{2}{}{}
\df{x_m} \wedge \df{x_j}
= -\pdydxnm{\omega_{i_1,i_2,\ldots,i_k}}{x_m}{x_j}{2}{}{}
\df{x_j} \wedge \df{x_m}
\]
for all $1 \leq j,m \leq n$ according to (\ref{dxPropr2}); in
particular, both sides are null when $m=j$ according to (\ref{dxPropr1}).

\stage{4} We prove this result using a proof by induction on the order
of the differential form.  Using the chain rule, it is easy to verify
that the result is true if $\omega$ is a $0$-form; namely, if
$\omega:V \to \RR$ is a function of class $\displaystyle C^1$.
Suppose that the result is true for all differential $k$-form.  If
$\omega$ is a differential $(k+1)$-from, then we can write
$\displaystyle \omega = \sum_{j=1}^n \mu_j \wedge \df{x_j}$ for some
$k$-forms $\mu_j$.  From (1), (2) and (3) above, we get
\begin{align*}
\df{\omega} &= \sum_{j=1}^n\df{(\mu_j \wedge \df{x_j})}
= \sum_{j=1}^n\left(\df{\mu_j} \wedge \df{x_j} + (-i)^k\mu_j \wedge
\df[2]{x_j}\right)
= \sum_{j=1}^n\left(\df{\mu_j} \wedge \df{x_j}\right) \ .
\end{align*}
It follows from (2) and (4) of Proposition~\ref{propAstoR} that
$\displaystyle f^\ast \df{\omega}
= \sum_{j=1}^n\left(f^\ast \df{\mu_j} \wedge f^\ast \df{x_j}\right)$.
Recall that $\df{x_j} = \df{\pi_j}$ where $\pi_j:\RR^n \to \RR$ is a
projection.  Since $\pi_j$ is a $0$-form, we have that
$\displaystyle \df{ (f^\ast \df{x_j})} = \df{ (f^\ast \df{\pi_j})}
= \df{(\df{(f^\ast(\pi_j))})} = \df[2]{(\pi_j\circ f)}
= \df[2]{ f_j} = 0$ for all $j$ because of (3).
Hence
\[
\df{(f^\ast \mu_j \wedge f^\ast \df{x_j})}
= \df{f^\ast \mu_j} \wedge f^\ast \df{x_j}
+ (-1)^k f^\ast \mu_j \wedge \df{(f^\ast \df{x_j})}
= \df{f^\ast \mu_j} \wedge f^\ast \df{x_j} \ .
\]
Finally, using this relation and our hypothesis of induction, we get
\begin{align*}
f^\ast \df{\omega}
&= \sum_{j=1}^n\left(f^\ast \df{\mu_j} \wedge f^\ast \df{x_j}\right)
= \sum_{j=1}^n\left(\df{f^\ast \mu_j} \wedge f^\ast \df{x_j}\right)
= \sum_{j=1}^n\df{(f^\ast \mu_j \wedge f^\ast \dx{x_j})} \\
&= \df{\left( \sum_{j=1}^nf^\ast \mu_j \wedge f^\ast \dx{x_j}\right)}
= \df{\left(f^\ast \left( \sum_{j=1}^n\mu_j \wedge \dx{x_j}\right)\right)}
= \df{(f^\ast \omega)} \ ,
\end{align*}
where we have used (1) above to get the fourth equality, and
(2) and (4) of Proposition~\ref{propAstoR} to get the second to last
equality.  This complete the proof by induction.
\end{proof}

From now on, we will assume that each differential form that we consider is
of class $\displaystyle C^j$ for $j$ large enough to meet the
requirements of the statement about this differential form.

\begin{defn}
A differential $k$-form $\omega$ on the open set
$\displaystyle V \subset \RR^n$ is
{\bfseries closed}\index{Differential Form!Closed}
if $\df{\omega}=0$ on $V$.  A differential $k$-form $\omega$ on the open set
$\displaystyle V \subset \RR^n$ is
{\bfseries exact}\index{Differential Form!Exact}
if there exists a differential $(k-1)$-form $\mu$ on $V$ such that
$\df{\mu} = \omega$.
\end{defn}

It follows from item (3) of Theorem~\ref{stokesDF} that exact
differential $k$-forms are closed.

\begin{egg}
Are closed differential $k$-forms exact?    \label{CnotEpart1}
The following example is from \cite{S} though it is a classical
example that can be found in almost all textbooks on differential
geometry.

In general, the answer is negative.  Consider the differential $1$-form
\[
\omega = \frac{-x_2}{x_1^2+x_2^2} \df{x_1} + \frac{x_1}{x_1^2+x_2^2} \df{x_2}
\]
defined on $\displaystyle V= \RR^2 \setminus \{ \VEC{0}\}$ (see
warning below).

We have
\begin{align*}
\df{\omega} &= \left( \frac{2x_1 x_2}{(x_1^2+x_2^2)^2} \df{x_1}
+ \frac{x_2^2-x_1^2}{(x_1^2+x_2^2)^2} \df{x_2} \right) \wedge \df{x_1} \\
&\qquad \qquad + \left( \frac{x_2^2-x_1^2}{(x_1^2+x_2^2)^2} \df{x_1}
- \frac{2x_1 x_2}{(x_1^2+x_2^2)^2} \df{x_2} \right) \wedge \df{x_2} \\
&= \frac{x_2^2-x_1^2}{(x_1^2+x_2^2)^2} \df{x_2} \wedge \df{x_1}
+ \frac{x_2^2-x_1^2}{(x_1^2+x_2^2)^2} \df{x_1} \wedge \df{x_2} \\
&= -\frac{x_2^2-x_1^2}{(x_1^2+x_2^2)^2} \df{x_1} \wedge \df{x_2}
+ \frac{x_2^2-x_1^2}{(x_1^2+x_2^2)^2} \df{x_1} \wedge \df{x_2}
= 0 \ .
\end{align*}
However, there does not exist any function $\mu$ on $V$ such that
$\df{\mu} = \omega$.  To prove this statement, let
$\displaystyle W = \RR^2\setminus \{(x_1,0) : x_1 \geq 0 \}$ and
consider the function $\theta: W \to \RR$ defined by
\[
\theta(\VEC{x}) =
\begin{cases}
\arctan(x_2/x_1) & \quad \text{if} \ x_1,x_2 >0 \\
\pi/2 & \quad \text{if} \ x_1=0,\ x_2 >0 \\
\pi + \arctan(x_2/x_1) & \quad \text{if} \ x_1 <0 \\
3\pi/2 & \quad \text{if} \ x_1=0,\ x_2 <0 \\
2\pi + \arctan(x_2/x_1) & \quad \text{if} \ x_1>0, \ x_2 <0
\end{cases}
\]
\pdfbox{diff_forms/notExact}
The image of $\theta$ is $]0,2\pi[$.  We have that $\df{\theta}=\omega$ on
$W$.  If there was a function $\mu:V\to \RR$ such that
$\df{\mu} = \omega$ on $V$, then we would have that
$\displaystyle \pdydx{\mu}{x_1} = \pdydx{\theta}{x_1} =
\frac{-x_2}{x_1^2+x_2^2}$ and
$\displaystyle \pdydx{\mu}{x_2} = \pdydx{\theta}{x_2} =
\frac{x_1}{x_1^2+x_2^2}$ on $W$.  Thus $\mu = \theta + C$ on $W$ for some
constant $C$.  However, at $\VEC{x}_0 = (x_1,0)$ with $x_1>0$, we have
that $\mu(\VEC{x}) = \theta(\VEC{x}) + C \to C$ as $\VEC{x}\to \VEC{x}_0$
with $x_2>0$ and $\mu(\VEC{x}) = \theta(\VEC{x}) +C \to 2\pi + C$ as
$\VEC{x}\to \VEC{x}_0$ with $x_2<0$.  This is a contradiction that $\mu$ is
continuous at $\VEC{x}_0 \in V$.  Thus $\omega$ cannot be exact on
$V$.

We provide another proof that $\omega$ cannot be exact in $V$ in
Example~\ref{CnotEpart2} later.
\end{egg}

\begin{rmk}[Warning]
The expression             \label{rmkWarning1}
\[
\omega = \frac{-x_2}{x_1^2+x_2^2} \df{x_1} + \frac{x_1}{x_1^2+x_2^2} \df{x_2}
\]
is an abbreviation for $\omega = f_1 \df{x_1} + f_2 \df{x_2}$ where
$\displaystyle f_1: \RR^2 \setminus \{\VEC{0}\} \to \RR$ is the function defined
by $\displaystyle f_1(\VEC{x}) = -x_2/(x_1^2 + x_2^2)$ and
$\displaystyle f_2: \RR^2 \setminus \{\VEC{0}\} \to \RR$ is the function defined
by $\displaystyle f_2(\VEC{x}) = x_1/(x_1^2 + x_2^2)$.  To introduce
the variable $\VEC{x}$ and be rigorously correct, we should have
written instead
\[
\omega(\VEC{x}) = \frac{-x_2}{x_1^2+x_2^2} \df{x_1}(\VEC{x}) +
\frac{x_1}{x_1^2+x_2^2} \df{x_2}(\VEC{x})
\]
but that would have been a uglier expression.
We will regularly use this type of abbreviation from now on.
\end{rmk}

Suppose that $\displaystyle \omega = \sum_{i=1}^n \omega_i \df{x_i}$ is a
closed differential $1$-form on $\displaystyle \RR^n$.  Let
\begin{equation}  \label{introPLEq0}
\mu(\VEC{x}) = \sum_{j=1}^n \int_0^1 \omega_j(t \VEC{x}) x_j \dx{t}
\end{equation}
for $1\leq j \leq n$ and $\displaystyle \VEC{x} \in \RR^n$.
Then
\begin{equation} \label{introPLEq1}
\pdydx{\mu}{x_i}(\VEC{x}) =
\sum_{\substack{j=1\\i\neq j}}^n \int_0^1
\left( t x_j \pdydx{\omega_j}{x_i}(t \VEC{x}) \right) \dx{t} 
+ \int_0^1 \left( t x_i \pdydx{\omega_i}{x_i}(t \VEC{x})
+ \omega_i(t\VEC{x}) \right) \dx{t} \ .
\end{equation}
Since $\omega$ is a closed differential form, we get that
$\displaystyle
\df{\omega} = \sum_{i=1}^n \sum_{j=1}^n \pdydx{\omega_i}{x_j} \df{x_j}
\wedge \df{x_i} = 0$.
Therefore, with $\displaystyle (\VEC{x},\VEC{y}) \in \TS_{\VEC{x}} \RR^n$ 
where $y_i = 0$ for $i\neq p,q$ and $y_i = 1$ for $i=p,q$, we get from
$\df{\omega}(\VEC{x})(\VEC{x},\VEC{y}) = 0$ that
$\displaystyle \pdydx{\omega_p}{x_q}(\VEC{x}) 
= \pdydx{\omega_q}{x_p}(\VEC{x})$ for $1 \leq p < q \leq n$.
Thus (\ref{introPLEq1}) yields
\begin{align*}
\pdydx{\mu}{x_i}(\VEC{x}) &= \sum_{\substack{j=1\\i\neq j}}^n \int_0^1
\left( t x_j \pdydx{\omega_i}{x_j}(t \VEC{x}) \right) \dx{t} 
+ \int_0^1 \left( t x_i \pdydx{\omega_i}{x_i}(t \VEC{x})
+ \omega_i(t\VEC{x}) \right) \dx{t} \\
&= \int_0^1 \left( \sum_{j=1}^n  t x_j \pdydx{\omega_i}{x_j}(t \VEC{x}) 
+ \omega_i(t\VEC{x}) \right) \dx{t}
= \int_0^1 \dfdx{\left( t \omega_i(t\VEC{x}) \right)}{t} \dx{t}
= \omega_i(\VEC{x})
\end{align*}
for $\displaystyle \VEC{x} \in \RR^n$.  Hence
$\df{\mu} = \omega$.  This method to find a differential $0$-form
$\mu$ such that $\df{\mu} = \omega$ works because the line
from $\VEC{0}$ to any $\displaystyle \VEC{x} \in \RR^n$ is
included in the domain of $\omega$.  This is required in the
definition of $\mu$ in (\ref{introPLEq0}).  To expend this method, we
consider domains with a similar property.

\begin{defn} \label{defnStarShaped}
A set $\displaystyle V \subset \RR^n$ is
{\bfseries star-shaped}\index{Star-Shaped} if there exists a point
$\VEC{y} \in V$ such that the line segment
$\{ \VEC{y} + t(\VEC{x}-\VEC{y}) : 0\leq 0 \leq 1\} \subset V$ for all
$\VEC{x} \in V$.
\end{defn}

If $\omega$ is a closed differential $1$-form defined on a star-shaped
domain $V$, then the integral formula in (\ref{introPLEq0}) (with
appropriate shifting) can be used to find $\mu:V\to \RR$ such that
$\df{\mu} = \omega$.  A general form of this integral formula is used
to prove the following theorem.

\begin{theorem}[Poincaré Lemma]\label{closedexact}
If $\displaystyle V\subset \RR^n$ is an open set which is star-shaped
and $\omega$ is a closed differential $k$-form on $V$ with $k>0$, then
$\omega$ is exact on $V$.
\end{theorem}

\begin{proof}
Without loss of generality, we may assume that $\VEC{0}\in V$ and every point
$\VEC{x} \in V$ can be joined by a straight line in $V$ to the origin.
If it is not, then we consider $\tilde{V} = V - \{\VEC{y}\}
= \{ \VEC{x} - \VEC{y}: \VEC{x} \in V \}$ and
$\tilde{\omega}(\VEC{x}) = \omega(\VEC{x}+\VEC{y})$ for
$\VEC{x} \in \tilde{V}$ where $\VEC{y}$ is the "centre" of the
star-shaped set described in Definition~\ref{defnStarShaped}.

For every positive integer $k$, we prove the existence of a function
$L$ mappings differential $k$-forms on $V$ to differential
$(k-1)$-forms on $V$ such that $L(0)=0$ and
\begin{equation} \label{stokesEXCL}
\tau = [L,\df{}] \tau \equiv L(\df{\tau}) + \df{L(\tau)}
\end{equation}
for all differential $k$-forms $\tau$ on $V$ \footnote{The
expression $[L,\df{}]$ is called a bracket of $L$ and $\df{}$.
Expressions of that form are frequent in differential geometry and, in
fact, in many other fields of mathematics.}.
Assuming that $L$ exists,
(\ref{stokesEXCL}) with $\tau = \omega$ yields $\omega = \df{L(\omega)}$
because $\df{\omega} = 0$.  Thus $\omega$ is exact.

Suppose that
\[
\tau = \sum_{1\leq i_i < i_2 < \ldots < i_k\leq n} \tau_{i_1,i_2,\ldots,i_k}
\df{x_{i_1}} \wedge \df{x_{i_2}} \wedge \ldots \wedge \df{x_{i_k}} \ .
\]
Let $\displaystyle g_{i_1,i_2,\ldots,i_k,j}: V \to \RR$ for
$1\leq i_1<i_2<\ldots<i_k\leq n$ and $1\leq j \leq k$
be the functions defined by
\[
g_{i_1,i_2,\ldots,i_k,j}(\VEC{x}) =
(-1)^{j-1} x_{i_j} \int_0^1 t^{k-1} \tau_{i_1,i_2,\ldots,i_k}(t\VEC{x}) \dx{t}
\]
for $\VEC{x} \in V$.  We define $L(\tau)$ by
\begin{align*}
L(\tau) &= \sum_{1\leq i_1<i_2<\ldots<i_k\leq n} \bigg(
\sum_{j=1}^k \bigg( g_{i_1,i_2,\ldots,i_k,j}
 \df{x_{i_1}} \wedge \df{x_{i_2}} \wedge \ldots \wedge
\widehat{\df{x_{i_j}}} \wedge \ldots \wedge \df{x_{i_k}}
\bigg) \bigg)
\end{align*}
where the hat over a $\df{x_{i_j}}$ indicates that this
factor has been removed from the wedge product.  The integral is well
defined because $t\VEC{x} \in V$ for $0\leq t \leq 1$.
For the clarity of the proof, it is probably better to use the
abbreviation mentioned in Remark~\ref{rmkWarning1} and simply write
\begin{align*}
L(\tau) &= \sum_{1\leq i_1<i_2<\ldots<i_k\leq n} \bigg(
\sum_{j=1}^k \bigg( (-1)^{j-1} x_{i_j} \left( \int_0^1 t^{k-1}
\tau_{i_1,i_2,\ldots,i_k}(t\VEC{x}) \dx{t} \right) \\
&\hspace{10em} \df{x_{i_1}} \wedge \df{x_{i_2}} \wedge \ldots \wedge
\widehat{\df{x_{i_j}}} \wedge \ldots \wedge \df{x_{i_k}}
\bigg) \bigg) \ .
\end{align*}
We have
\begin{align}
\df{L(\tau)} &= \df{\big(L(\tau)\big)} \nonumber \\
&= \sum_{i_1<i_2<\ldots<i_k} \bigg( \sum_{j=1}^k \bigg( (-1)^{j-1} 
\sum_{m=1}^n \bigg( \pdfdx{ \left( x_{i_j} \int_0^1 t^{k-1}
\tau_{i_1,i_2,\ldots,i_k}(t\VEC{x}) \dx{t} \right)}{x_m}
\nonumber \\
&\hspace{8em} \df{x_m}\wedge \df{x_{i_1}} \wedge \df{x_{i_2}} \wedge \ldots
\wedge \widehat{\df{x_{i_j}}} \wedge \ldots \wedge \df{x_{i_k}} \bigg)
\bigg) \bigg) \nonumber \\
&= \sum_{i_1<i_2<\ldots<i_k} \bigg( \sum_{j=1}^k \bigg( (-1)^{j-1} 
\left( \int_0^1 t^{k-1} \tau_{i_1,i_2,\ldots,i_k}(t\VEC{x}) \dx{t} \right)
\nonumber \\
&\hspace{8em} 
\df{x_{i_j}} \wedge \df{x_{i_1}} \wedge \df{x_{i_2}} \wedge \ldots
\wedge \widehat{\df{x_{i_j}}} \wedge \ldots \wedge \df{x_{i_k}} \bigg) \bigg)
\nonumber \\
&\qquad + \sum_{i_1<i_2<\ldots<i_k} \bigg( \sum_{j=1}^k \bigg( (-1)^{j-1} 
x_{i_j} \sum_{m=1}^n \bigg( \left(\int_0^1 t^k
\pdydx{\tau_{i_1,i_2,\ldots,i_k}}{x_m}(t\VEC{x}) \dx{t} \right)
\nonumber \\
&\hspace{8em}
\df{x_m} \wedge \df{x_{i_1}} \wedge \df{x_{i_2}} \wedge \ldots \wedge
\widehat{\df{x_{i_j}}} \wedge \ldots \wedge \df{x_{i_k}} \bigg) \bigg) \bigg)
\nonumber \\
&= \sum_{i_1<i_2<\ldots<i_k} \bigg(
\left( \int_0^1 k t^{k-1} \tau_{i_1,i_2,\ldots,i_k}(t\VEC{x}) \dx{t} \right)
\df{x_{i_1}} \wedge \df{x_{i_2}} \wedge \ldots \wedge \df{x_{i_k}} \bigg)
\nonumber \\
&\qquad + \sum_{i_1<i_2<\ldots<i_k} \bigg( \sum_{j=1}^k \bigg( (-1)^{j-1} 
x_{i_j} \sum_{m=1}^n \bigg( \left( \int_0^1 t^k
\pdydx{\tau_{i_1,i_2,\ldots,i_k}}{x_m}(t\VEC{x}) \dx{t} \right)
\nonumber \\
&\hspace{8em}
\df{x_m} \wedge \df{x_{i_1}} \wedge \df{x_{i_2}} \wedge \ldots \wedge
\widehat{\df{x_{i_j}}} \wedge \ldots \wedge \df{x_{i_k}} \bigg) \bigg) \bigg)
\ , \label{stokesEq1}
\end{align}
where we have used
$\displaystyle \df{x_{i_j}} \wedge \df{x_{i_1}} \wedge \df{x_{i_2}} \wedge \ldots
\wedge \widehat{\df{x_{i_j}}} \wedge \ldots \wedge \df{x_{i_k}}
= (-1)^{j-1} \df{x_{i_1}} \wedge \df{x_{i_2}} \wedge \ldots
\wedge \df{x_{i_j}} \wedge \ldots \wedge \df{x_{i_k}}$ to get the last
equality.

Moreover, if we apply $L$ to the differential $(k+1)$-form 
\[
\df{\tau} = \sum_{i_i < i_2 < \ldots < i_k}
\sum_{m=1}^n \pdydx{\tau_{i_1,i_2,\ldots,i_k}}{x_m}
\df{x_m} \wedge \df{x_{i_1}} \wedge \df{x_{i_2}} \wedge \ldots
\wedge \df{x_{i_k}} \ ,
\]
then we get
\begin{align}
L(\df{\tau}) &= \sum_{i_1<i_2<\ldots<i_k} \left(
\sum_{m=1}^n \left( x_m \int_0^1 t^{k}
\pdydx{\tau_{i_1,i_2,\ldots,i_k}}{x_m}(t\VEC{x}) \dx{t} \right)
\df{x_{i_1}} \wedge \df{x_{i_2}} \wedge \ldots \wedge \df{x_{i_k}}
\right) \nonumber \\
&\qquad +
\sum_{i_1<i_2<\ldots<i_k} \bigg(
\sum_{j=1}^k \bigg( (-1)^j x_{i_j}
\sum_{m=1}^n  \bigg( \left( \int_0^1 t^{k}
\pdydx{\tau_{i_1,i_2,\ldots,i_k}}{x_m} (t\VEC{x}) \dx{t} \right)
\nonumber \\
&\hspace{8em}
\df{x_m} \wedge \df{x_{i_1}} \wedge \df{x_{i_2}} \wedge \ldots
\wedge \widehat{\df{x_{i_j}}} \wedge \ldots \wedge \df{x_{i_k}}
\bigg) \bigg) \bigg) \label{stokesEq2} \ .
\end{align}
Adding (\ref{stokesEq1}) and (\ref{stokesEq2}) gives
\begin{align*}
\df{L(\tau)} + L(\df{\tau}) &=  \sum_{i_1<i_2<\ldots<i_k} \bigg(
\int_0^1 k t^{k-1} \tau_{i_1,i_2,\ldots,i_k}(t\VEC{x}) \dx{t} \\
&\qquad \qquad + \sum_{m=1}^n x_m \int_0^1 t^{k}
\pdydx{\tau_{i_1,i_2,\ldots,i_k}}{x_m}(t\VEC{x}) \dx{t} \bigg)
\df{x_{i_1}} \wedge \df{x_{i_2}} \wedge \ldots \wedge \df{x_{i_k}} \\
&=  \sum_{i_1<i_2<\ldots<i_k} \bigg(
\int_0^1 \dfdx{ \left( t^k \tau_{i_1,i_2,\ldots,i_k}(t\VEC{x}) \right)}{t}
\dx{t} \bigg)
\df{x_{i_1}} \wedge \df{x_{i_2}} \wedge \ldots \wedge \df{x_{i_k}} \\
&=  \sum_{i_1<i_2<\ldots<i_k}
\tau_{i_1,i_2,\ldots,i_k}(\VEC{x})
\df{x_{i_1}} \wedge \df{x_{i_2}} \wedge \ldots \wedge \df{x_{i_k}}
= \tau(\VEC{x}) \ .  \qedhere
\end{align*}
\end{proof}

\begin{rmk}
The previous theorem can be generalized to domains $V$ that are
{\bfseries (smoothly) contractible to a point}\index{Contractible to a
Point}; namely, there exist a point $\VEC{y} \in V$ and a continuous
differentiable map $H:V \times [0,1] \to V$ such that
$H(\VEC{x},1) = \VEC{x}$ and $H(\VEC{x},0) = \VEC{y}$ for all
$\VEC{x} \in V$.  A star-shaped set $V$ as given in
Definition~\ref{defnStarShaped} is contractible to the point $\VEC{y} \in V$.
It suffices to use the map $\displaystyle H:V \times [0,1] \to V$
defined by $H(\VEC{x},t) = \VEC{y} + t (\VEC{x} - \VEC{y})$
for $\VEC{x} \in V$ and $t \in [0,1]$.

The proof of the generalized version of Theorem~\ref{closedexact} is
given in Section~\ref{sectPrelCohom}.
\end{rmk}

\section{Domains of Integration and their Boundary} \label{stokesBDRS}

Before stating the general form of the fundamental theorem of calculus, we
have to introduce the domains of integration that will be considered.

\begin{defn}
A {\bfseries singular $\mathbf{m}$-cube}\index{Singular $m$-Cube}
in $\displaystyle \RR^n$ is a continuous function
$\displaystyle \sigma :I_m \to \RR^n$, where $I_m$ is the unit cube
$\displaystyle I_m = \{(x_1,x_2,\ldots,x_m): 0 \leq x_j\leq 1
\text{ for } 1\leq j \leq m \}\subset \RR^m$.  For $m=0$, we define
$\displaystyle \RR^m$ and $I_m$ as $\{0\}$.
\end{defn}

A singular $0$-cube is simply a map
$\displaystyle \sigma: \{0\} \to \RR^n$.
The image of a singular $0$-cube $\sigma$ is a point in
$\displaystyle \RR^n$.  A
singular $1$-cube is a map $\displaystyle \sigma: [0,1] \to \RR^n$;
namely, a {\bfseries curve}\index{Curve} in $\displaystyle \RR^n$.

\begin{defn}
The {\bfseries standard $\mathbf{m}$-cube}\index{Standard $m$-Cube}
is the function $\displaystyle \stc_m : I_m \to \RR^m$ defined by
$\stc_m(\VEC{x}) = \VEC{x}$ for all $\VEC{x} \in I_m$.
\end{defn}

\begin{defn}
An {\bfseries $\mathbf{m}$-chain}\index{$m$-Chain} in $\displaystyle \RR^n$
is an expression of the form
$\displaystyle \sum_{j=1}^J a_j \sigma_j$, where $a_j \in \ZZ$ and
$\sigma_j$ is a singular $m$-cube in $\displaystyle \RR^n$ for
$1 \leq j \leq J$ with $J$ a positive integer.
\end{defn}

A singular $m$-cube $\sigma$ can be seen as the $m$-chain $1 \sigma$.
The addition of $m$-chain is defined by
\[
\sum_{j=1}^J a_j \sigma_j + \sum_{j=1}^J b_j \sigma_j
= \sum_{j=1}^J (a_j + b_j) \sigma_j \ .
\]
We have only added the coefficients associated to the same singular
$m$-cubes.  When adding $m$-chains, we may assume that they are sums
over the same singular $m$-cubes by adding terms of the form
$0 \sigma_j$ if necessary.  The multiplication of an $m$-chain by an
integer $c$ is defined by
\[
c \sum_{j=1}^J a_j \sigma_j = \sum_{j=1}^J (c a_j) \sigma_j \ .
\]

We use the standard $m$-cube
$\stc_m$ to define the singular $(m-1)$-cubes
$\stc_{m,j,i} : I_{m-1} \to \RR^m$ for $1\leq j \leq m$ and
$i=0$ or $i=1$ by $\stc_{m,j,i}(\VEC{x}) = \stc_m\big(
\begin{pmatrix} x_1 & x_2 & \ldots & x_{j-1} & i & x_j & \ldots & x_{m-1}
\end{pmatrix}^\top\big)$ for all $\VEC{x} \in I_{m-1}$.

\begin{defn}
For $1\leq j \leq m$ and $0 \leq i \leq 1$, the singular $(m-1)$-cube
$\stc_{m,j,i}$ is called the
{\bfseries $\mathbf{(j,i)}$ face}\index{$(j,i)$ Face} of
the standard $m$-cube $\stc_m$.
The {\bfseries boundary}\index{Boundary} of the
standard $m$-cube $\stc_m$, denoted $\partial \stc_m$, is defined by
the $(m-1)$-chain
\[
\partial \stc_m = \sum_{j=1}^m (-1)^{j-1}
\left( \stc_{m,j,1} - \stc_{m,j,0} \right) \ .
\]
\end{defn}

We have drawn the boundary of the standard $m$-cube for $m=1$
and $m=2$ in Figure~\ref{StdCube}.  In particular,
$\displaystyle \stc_{1,1,1}:\{0\} \to \RR^n$ is defined by
$\stc_{1,1,1}(0) = \stc_1(1)$
and $\displaystyle \stc_{1,1,0}:\{0\} \to \RR^n$ is defined by
$\stc_{1,1,0}(0) = \stc_1(0)$.

\pdfF{diff_forms/stdmcube}{Some examples of boundaries of standard $m$-cubes}
{We have drawn a standard $m$-cube with its boundary for $m=1$ on the left 
and $m=2$ on the right.  We have that
$\partial \stc_1 = \stc_{1,1,1} - \stc_{1,1,0}$ and
$\partial \stc_2 = \stc_{2,1,1} - \stc_{2,1,0}
+ \stc_{2,2,0} - \stc_{2,2,1}$.  We have added the sign of the 
faces to the standard $m$-cube.}{StdCube}

The $(j,i)$ face of a general singular $m$-cube $\sigma$ is defined by
$\sigma_{j,i} = \sigma \circ i_{m,j,i}$ for $1\leq j \leq m$ and
$i =0$ or $i=1$.

\begin{defn}     \label{defnBryCube}
Let $\sigma$ be a singular $m$-cube.  The
{\bfseries boundary}\index{Boundary} of $\sigma$, 
denoted $\partial \sigma$, is defined by the $(m-1)$-chain 
\[
\partial \sigma = \sum_{j=1}^m (-1)^{j-1} \left( \sigma_{j,1} - \sigma_{j,0}
\right) \ .
\]
Let $\sigma$ be a $m$-chain given by
$\displaystyle \sigma = \sum_{j=1}^J a_j \sigma_j$.
The {\bfseries boundary}\index{Boundary} of $\sigma$, also denote
$\partial \sigma$, is defined by the $(m-1)$-chain
$\displaystyle \partial \sigma = \sum_{j=1}^J a_j \partial \sigma_j$.
\end{defn}

We have drawn the boundary of a singular $2$-cube in Figure~\ref{SingCube}.

\pdfF{diff_forms/singmcube}{A singular $2$-cube in $\RR^3$}
{We have drawn a singular $2$-cube in $\displaystyle \RR^3$ with its boundary
We have that $\partial \sigma_2 = \sigma_{1,1} - \sigma_{1,0}
+ \sigma_{2,0} - \sigma_{2,1}$.  We have added the sign of the 
faces to the singular $2$-cube.}{SingCube}

\begin{theorem}
if $\sigma$ is a $m$-chain, then $\partial(\partial \sigma) = 0$
where $0$ denotes the empty $(m-2)$-chain.
\end{theorem}

\begin{proof}
Since the boundary of $m$-chain is the sum of the boundary of the
$m$-cubes that are parts of the chain, it is enough to proof that
$\partial(\partial \sigma) = 0$ for a $m$-cube $\sigma$.

Since $\displaystyle 
\partial \sigma = \sum_{j=1}^m (-1)^{j-1} \left( \sigma_{j,1} - \sigma_{j,0}
\right)$, we get
\begin{align*}
\partial (\partial \sigma)
&= \sum_{j=1}^m (-1)^{j-1} \left(
\sum_{i=1}^{m-1} (-1)^{i-1} \left( (\sigma_{j,1})_{i,1}
- (\sigma_{j,1})_{i,0} \right)
- \sum_{i=1}^{m-1} (-1)^{i-1} \left( (\sigma_{j,0})_{i,1} 
- (\sigma_{j,0})_{i,0} \right) \right) \\
&= \sum_{j=1}^m \sum_{i=1}^{m-1} (-1)^{j+i-2}
\big( (\sigma_{j,1})_{i,1} - (\sigma_{j,1})_{i,0}
- (\sigma_{j,0})_{i,1}  + (\sigma_{j,0})_{i,0} \big) \\
&= \sum_{j=1}^m \sum_{i=1}^{m-1} \sum_{s_1,s_2=0}^1 (-1)^{j+i+s_1+s_2}
(\sigma_{j,s_2})_{i,s_1} \ .
\end{align*}
If $1 \leq i < j \leq m$, then
\begin{align}
(\sigma_{j,s_2})_{i,s_1} (\VEC{y})
&= \sigma_{j,s_2} \big( \begin{pmatrix} y_1 & y_2 & \ldots & y_{i-1} & s_1 &
y_i & \ldots & y_{m-2} \end{pmatrix}^\top \big) \nonumber \\
&= \left(\begin{array}{cccccccccccc}
y_1 & y_2 & \ldots & y_{i-1} & s_1 & y_i & \ldots
& y_{j-2} & s_2 & y_{j-1} & \ldots & y_{m-2} \end{array}\right)^\top
\nonumber \\
&= \sigma_{i,s_1} \big( \begin{pmatrix} y_1 & y_2 & \ldots & y_{j-2} & s_2 &
y_{j-1} & \ldots & y_{m-2} \end{pmatrix}^\top \big)
= (\sigma_{i,s_1})_{j-1,s_2} (\VEC{y})   \label{bdbdE1e}
\end{align}
for all $\VEC{y} \in I_{m-2}$.  If $i = j-1$, then the section
$y_i\ \ldots\ y_{j-2}$ is removed from the second row of (\ref{bdbdE1e}).
The terms of the form $(\sigma_{j,s_2})_{i,s_1} (\VEC{y})$ with $j=i$ are
included in (\ref{bdbdE1e}) because, when $i = k$ and $j= k+1$ for
$1\leq k < m$, we get $(\sigma_{k,s_2})_{k,s_1} (\VEC{y})$
for $1\leq k <m$ on the right side of (\ref{bdbdE1e}).
Similarly, the terms of the form $(\sigma_{j,s_2})_{i,s_1} (\VEC{y})$
with $1 \leq j<i<m$ are also included in (\ref{bdbdE1e})
because we have $1 \leq i \leq j-1 < m$ on the right side of
(\ref{bdbdE1e}).

Since the coefficient of $(\sigma_{j,s_2})_{i,s_1} (\VEC{y})$ is
$\displaystyle (-1)^{j+i+s_1+s_2}$ and the coefficient of
$(\sigma_{i,s_1})_{j-1,s_2} (\VEC{y})$ is
$\displaystyle (-1)^{j-1+i+s_1+s_2}$, they have opposite signs and thus
cancel each other.  Therefore $\partial (\partial \sigma) = 0$.
\end{proof}

\begin{rmk}
It is worth emphasizing the difference between the topological
boundary of a cube and the boundary of a $m$-cube.  For instance, the
topological boundary of the cube
$I_3 = \{ (x_1,x_2,x_3) : 0 \leq x_j \leq 1 \text{ for } 1\leq j \leq 3 \}$
is the set
\begin{align*}
\partial I_3 &=
\bigcup_{m=0}^1 \Big( \{ (m,x_2,x_3) : 0 \leq x_j \leq 1 \
\text{for}\ j=2,3 \}
\cup \{ (x_1,m,x_3) : 0 \leq x_j \leq 1\ \text{for}\  j=1,3 \} \\
&\hspace{10em} \cup \{ (x_1,x_2,m) : 0 \leq x_j \leq 1 \ \text{for} \
j=1,2 \} \Big)
\end{align*}
and the topological boundary of $\partial I_3$ is
$\partial (\partial I_3) = \partial I_3$.  The issue about boundary
will get even more confusing when we will consider manifolds later.
The boundary of the manifold $I_3$ is still $\partial I_3$ defined
above and denoted $\partial I_3$.  However, the boundary of
$\partial I_3$ as a manifold, still denoted $\partial (\partial I_3)$,
is the empty set $\emptyset$.  As a manifold, $\partial I_3$ does not have a
boundary.  Therefore, the reader should be very careful of not mixing
all these definitions of boundary.  The context should determine which
definition of boundary is used.  It is unfortunate that by tradition
we use the same symbol $\partial$ for all definitions of boundary.
\end{rmk}

\section{Integration on Chains} \label{sectFTC}

We begin with the definition of the integral of a differential
$k$-forms on an $k$-chain.  From now on all differential $k$-forms
$\omega$ on an open set $\displaystyle V\subset \RR^n$ and all
singular $k$-cubes $\sigma: I_k \to V$ are assumed to be of class
$\displaystyle C^j$ for $j$ as sufficiently large.  Recall that
$\sigma:I_k \to V$ of class $\displaystyle C^j$ means that there
exist an open set $U \supset I_k$ and a function
$\tilde{\sigma}:U \to \RR^n$ of class $\displaystyle C^j$ such that
$\tilde{\sigma}\big|_{I_k} = \sigma$.

\begin{defn} \label{defStcInt}
Let $k$ be a positive integer and
$\omega = f \df{x_1}\wedge \df{x_2}\wedge \ldots \wedge \df{x_k}$
be a differential $k$-form on $\displaystyle \RR^k$.  The
{\bfseries integral of $\omega$}\index{Integral of a Differential Form}
on $I_k$ is defined by
$\displaystyle \int_{I_k} \omega = \int_{I_k} f(\VEC{x}) \dx{\VEC{x}}$.
Let $\omega$ be a differential $k$-form defined on an open set
$\displaystyle V \subset \RR^n$ and
$\sigma$ be a singular $k$-cube with image in $V$.  The
{\bfseries integral of $\omega$ on $\sigma$}\index{Integral of a
Differential Form} is defined by 
$\displaystyle \int_{\sigma} \omega = \int_{I_k} \sigma^\ast(\omega)$.
\end{defn}

Obviously, the previous definition include the case where
$\omega = f \df{x_1}\wedge \df{x_2}\wedge \ldots \wedge \df{x_k}$ is a
differential $k$-form on $\displaystyle \RR^k$ and $\sigma$ is the standard
$k$-cube $\stc_m$.  We then have that
$\displaystyle \int_{\stc_k} \omega = \int_{I_k} \stc_k^\ast(\omega)
= \int_{I_k} \omega = \int_{I_k} f(\VEC{x}) \dx{\VEC{x}}$.

\begin{defn}
Let $\omega$ be a $0$-form on an open set
$\displaystyle V \subset \RR^n$ and $\sigma$ is a singular $0$-cube
with image in $V$.  Namely, $\omega:V \to \RR$ and $\sigma(0) \in V$.
The {\bfseries integral of $\omega$ on $\sigma$}\index{Integral
of a Differential Form} is defined by
$\displaystyle \int_{\sigma} \omega = \omega(\sigma(0))$.
\end{defn}

\begin{defn}
Let $\omega$ be a differential $k$-form on an open set
$\displaystyle V \subset \RR^n$ and
$\displaystyle \sigma = \sum_{j=1}^J a_j \sigma_j$
be a $k$-chain in $V$; namely, the image of each singular $k$-cube
$\sigma_j$ is in $V$.  The {\bfseries integral of $\omega$ on
$\sigma$}\index{Integral of a Differential Form} is defined by
$\displaystyle \int_{\sigma} \omega = \sum_{j=1}^J a_j \int_{\sigma_j} \omega
= \sum_{j=1}^J a_j \int_{I_k} \sigma_j^\ast(\omega)$.
\end{defn}

\begin{egg}
We give a second proof that the differential $1$-form      \label{CnotEpart2}
\[
\omega = \frac{-x_2}{x_1^2+x_2^2} \df{x_1} + \frac{x_1}{x_1^2+x_2^2} \df{x_2}
\]
defined on $\displaystyle V= \RR^2 \setminus \{ \VEC{0}\}$ and introduced
in Example~\ref{CnotEpart1} cannot be exact.

Suppose that $\omega = \df{f}$ where $f:V \to \RR$ is a differentiable
function on $V$.  Let $\sigma$ be the singular $1$-cube defined by
$\sigma(\theta) = \big(\cos(2\pi \theta),\sin(2\pi \theta)\big)$ for
$\theta \in [0,1]$.  It follows that
\begin{align*}
\int_{\sigma} \omega &= \int_{\sigma} \df{f} = \int_{[0,1]} \sigma^\ast(\df{f})
= \int_{[0,1]} \df{(\sigma^\ast(f))} = \int_{[0,1]} \df{(f(\sigma))} \\
&= \int_{[0,1]} \dfdx{\big( f(\sigma(\theta)) \big)}{\theta} \dx{\theta}
= f(\sigma(1)) - f(\sigma(0)) = 0
\end{align*}
because $\sigma(0) = \sigma(2\pi) = (1,0)$.
However, if we compute directly the integral $\int_\sigma \omega$,
then we get
\[
\int_{\sigma} \omega = \int_{[0,1]} \sigma^\ast(\omega)
= 2\pi \int_{[0,1]} \dx{\theta} = 2\pi \neq 0 \ .
\]
Therefore, we cannot have that $\omega = \df{f}$ with $f:V \to \RR$.
\end{egg}

The next theorem is often called the {\bfseries fundamental
theorem of calculus}\index{Fundamental Theorem of Calculus}
in higher dimension.  In fact, the classical
fundamental theorem of calculus is the case $m=1$ and $\omega = \stc_1$
in the next theorem.

Before stating and proving the next theorem, it is going to be useful
to evaluate $\displaystyle \stc_{k,j,s}^\ast \df{x_i}$ for
$1 \leq i , j \leq k$ and $s=0$ or $s=1$.
Recall that $\df{x_i} = \df{\pi_i}$.  Hence
\begin{align*}
\big(\stc_{k,j,s}^\ast \df{x_i}\big)(\VEC{w})\big( (\VEC{w},\VEC{y}) \big)
& = \big(\stc_{k,j,s}^\ast \df{\pi_i}\big)(\VEC{w})\big( (\VEC{w},\VEC{y}) \big)
= \big(\df{(\pi_i\circ \stc_{k.j,s})}\big)(\VEC{w})\big( (\VEC{w},\VEC{y}) \big)
\\
&= \big(\diff( \pi_i \circ \stc_{k,j,s})(\VEC{w})\big)(\VEC{y}) \\
&= \big( \pi_i \circ \stc_{k,j,0}\big)(\VEC{y})
= \pi_i (\stc_{k,j,0} (\VEC{y}))
= \begin{cases}
y_i & \quad \text{if} \ i < j \\
0 & \quad \text{if} \ i = j \\
y_{i-1} & \quad \text{if} \ i > j
\end{cases}
\end{align*}
for all $\VEC{w} \in I_{k-1}$ and $\displaystyle \VEC{y} \in \RR^{k-1}$.
Therefore,
\[
\stc_{k,j,s}^\ast \df{x_i} = \begin{cases}
\df{y_i} & \quad \text{if} \ i < j \\
0 & \quad \text{if} \ i = j \\
\df{y_{i-1}} & \quad \text{if} \ i > j
\end{cases}
\]
A more computational but lest intuitive approach to get the previous
relation would have been to use (1) of Proposition~\ref{propAstoR}.
We will use the computational approach in the future but it is wise to
also keep in mind the intuitive approach above.

As a consequence of this, we have
\begin{align*}
\stc_{k,j,s}^\ast\left(
\df{x_1} \wedge \ldots \wedge \widehat{\df{x_i}} \wedge \ldots \wedge
\df{x_k} \right)
& = \stc_{k,j,s}^\ast \df{x_1} \wedge \ldots \wedge
\widehat{\stc_{k,j,s}^\ast \df{x_i}}
\wedge \ldots \wedge \stc_{k,j,s}^\ast \df{x_k} \\
&=  \begin{cases}
0 & \quad \text{if} \ i \neq j \\
\df{y_1} \wedge \df{y_2} \wedge \ldots \wedge \df{y_{k-1}} &
\quad \text{if} \ i=j
\end{cases}
\end{align*}
where, as we stated before, the hat over $\df{x_i}$ for
$1\leq i \leq k$ means that this differential $1$-form has been
removed from the wedge product.

\begin{theorem}[Stokes] \label{stokesStokes}
Suppose that $\omega$ is a differential $(k-1)$-form on an open set
$\displaystyle V \subset \RR^n$ and
that $\sigma$ is an $k$-chain in $V$ with $\displaystyle k \in \NNp$.  Then
\[
\int_\sigma \df{\omega} = \int_{\partial \sigma} \omega \ .
\]
\end{theorem}

\begin{proof}
\stage{i} We first prove the theorem for $\sigma = \stc_k$ and
$\omega$ a differential $(k-1)$-form on $\displaystyle \RR^k$.  We have that
$\omega$ is the sum of terms of the form
\begin{equation} \label{stokesFTC1}
f \df{x_1} \wedge \ldots \wedge \widehat{\df{x_i}} \wedge \ldots \wedge
\df{x_k} \ ,
\end{equation}
By linearity of the integral, it suffices to prove the theorem for
(\ref{stokesFTC1}).

Since $\displaystyle
\partial \stc_k = \sum_{j=1}^k \sum_{s=0}^1 (-1)^{j+s} \stc_{k,j,s}$
and
\begin{align*}
& \int_{I_{k-1}} \stc_{k,j,s}^\ast
\left( f \df{x_1} \wedge \ldots \wedge \widehat{\df{x_i}} \wedge \ldots \wedge
\df{x_k} \right) \\
&\qquad = \begin{cases}
0 & \quad \text{if} \ i \neq j \\
\displaystyle
\int_{I_{k-1}} f\circ \stc_{k,j,s} \, \df{y_1} \wedge \df{y_2} \wedge
\ldots \wedge \df{y_{k-1}} &
\quad \text{if} \ i = j
\end{cases} \\
& \qquad =
\begin{cases}
0 & \quad \text{if} \ i \neq j \\
\displaystyle \int_{I_{k-1}} f(y_1,\ldots, y_{i-1}, s, y_i, \ldots, y_{k-1})
\df{y_1}\df{y_2}\ldots \df{y_{k-1}} & \quad \text{if} \ i=j
\end{cases}
\end{align*}
for $s=0$ or $s=1$, we get
\begin{align}
& \int_{\partial \stc_k} f \df{x_1} \wedge \ldots \wedge
\widehat{\df{x_i}} \wedge \ldots \wedge \df{x_k} =
\sum_{j=1}^k \sum_{s=0}^1 (-1)^{j+s} 
\int_{\stc_{k,j,s}} f \df{x_1} \wedge \ldots \wedge \widehat{\df{x_i}} \wedge
\ldots \wedge \df{x_k} \nonumber \\
&\quad = \sum_{j=1}^k \sum_{s=0}^1 (-1)^{j+s} \int_{I_{k-1}} \stc_{k,j,s}^\ast
\left( f \df{x_1} \wedge \ldots \wedge \widehat{\df{x_i}} \wedge \ldots \wedge
\df{x_k} \right) \nonumber \\
&\quad = \sum_{s=0}^1 (-1)^{i+s}
\int_{I_{k-1}} f(y_1,\ldots, y_{i-1}, s, y_i, \ldots, y_{k-1})
\df{y_1}\df{y_2}\ldots \df{y_{k-1}} \ . \label{stokesFTC2}
\end{align}

Moreover
\begin{align*}
& \int_{\stc_k} \df{\left(f \df{x_1} \wedge \ldots \wedge \widehat{\df{x_i}}
\wedge \ldots \wedge \df{x_k}\right)}
= \int_{\stc_k} \df{f}\wedge \df{x_1} \wedge \ldots \wedge \widehat{\df{x_i}}
\wedge \ldots \wedge \df{x_k} \\
&= \int_{\stc_k} \sum_{j=1}^k \pdydx{f}{x_j} \df{x_j}\wedge \df{x_1} \wedge
\ldots \wedge \widehat{\df{x_i}} \wedge \ldots \wedge \df{x_k} \\
&= (-1)^{i-1} \int_{\stc_k} \pdydx{f}{x_i} \df{x_1} \wedge \df{x_2} \wedge
\ldots \wedge \df{x_k}
= (-1)^{i-1} \int_{I_k} \pdydx{f}{x_i}(\VEC{x}) \df{\VEC{x}} \ .
\end{align*}
It follows from Fubini's theorem and from the classical fundamental theorem
of calculus in one dimension that
\begin{align}
& \int_{\stc_k} \df{\left(f \df{x_1} \wedge \ldots \wedge \widehat{\df{x_i}}
\wedge \ldots \wedge \df{x_k}\right)} \nonumber \\
&\quad = (-1)^{i-1} \int_0^1 \ldots \int_0^1 \left( \int_0^1
\pdydx{f}{x_i}(x_1, x_2, \ldots, x_k) \df{x_i}\right)
\df{x_1} \ldots \widehat{\df{x_i}}\ldots \df{x_k} \nonumber \\
&\quad = (-1)^{i-1} \int_0^1 \ldots \int_0^1
f(x_1, \ldots, x_{i-1}, x_i, x_{i+1} ,\ldots, x_k)\bigg|_{x_i=0}^{x_i=1}
\dx{x_1} \ldots \widehat{\dx{x_i}}\ldots \dx{x_k} \nonumber \\
&\quad =\sum_{s=0}^1 (-1)^{i+s}
\int_{I_{k-1}} f(y_1,\ldots, y_{i-1}, s, y_i, \ldots, y_{k-1})
\dx{y_1}\dx{y_2}\ldots \dx{y_{k-1}}  \ . \label{stokesFTC3}
\end{align}
If we combine (\ref{stokesFTC2}) and (\ref{stokesFTC3}), then we get
\[
\int_{\partial \stc_k} f \df{x_1} \wedge \ldots \wedge \widehat{\df{x_i}} \wedge
\ldots \wedge \df{x_k} =
\int_{\stc_k} \df{\left(f \df{x_1} \wedge \ldots \wedge \widehat{\df{x_i}}
\wedge \ldots \wedge \df{x_k}\right)} \ .
\]
Hence,
\[
\int_{\partial \stc_k} \omega = \int_{\stc_k} \df{\omega}
\]
for all differential $(k-1)$-forms $\omega$ on $\displaystyle \RR^k$.

\subQ{ii}
Suppose that $\sigma$ is a general singular $k$-cube in an open set
$\displaystyle V \subset \RR^n$ and that $\omega$ is a differential
$(k-1)$-form on $V$.  Since
\[
\partial \sigma = \sum_{j=1}^k \sum_{s=0}^1 (-1)^{i+s} \sigma_{j,s}
= \sum_{j=1}^k \sum_{s=0}^1 (-1)^{i+s} \sigma \circ \stc_{k,j,s} \ ,
\]
we have
\begin{align*}
\int_{\partial \sigma} \omega
&= \sum_{j=1}^k \sum_{s=0}^1 (-1)^{i+s} \int_{\sigma_{j,s}} \omega
= \sum_{j=1}^k \sum_{s=0}^1 (-1)^{i+s} \int_{I_{k-1}}
\sigma_{j,s}^\ast(\omega) \\
& = \sum_{j=1}^k \sum_{s=0}^1 (-1)^{i+s} \int_{I_{k-1}}
(\sigma \circ \stc_{k,j,s})^\ast(\omega)
= \sum_{j=1}^k \sum_{s=0}^1 (-1)^{i+s} \int_{I_{k-1}}
\stc_{k,j,s}^\ast(\sigma^\ast(\omega)) \\
& = \sum_{j=1}^k \sum_{s=0}^1 (-1)^{i+s} \int_{\stc_{k,j,s}}
\sigma^\ast(\omega)
= \int_{\partial \stc_k} \sigma^\ast(\omega) \ .
\end{align*}
It follows from (i) applied to $\displaystyle \sigma^\ast(\omega)$ and
(4) of Proposition~\ref{propAstoR} that
\[
\int_{\sigma} \df{\omega} = \int_{I_k} \sigma^\ast(\df{\omega})
= \int_{I_k} \df{(\sigma^\ast(\omega))}
= \int_{\stc_k} \df{(\sigma^\ast(\omega))}
= \int_{\partial \stc_k} \sigma^\ast(\omega)
= \int_{\partial \sigma} \omega \ .
\]

\subQ{iii}
Finally, if $\sigma = \displaystyle \sum_{j=1}^J a_j \sigma_j$ is a $k$-chain
in an open set $\displaystyle V \subset \RR^n$ and $\omega$ is a
differential $(k-1)$-form in $V$, then
\[
\int_{\mu} \df{\omega} = \sum_{j=1}^J a_j \int_{\sigma_j} \df{\omega}
= \sum_{j=1}^J a_j \int_{\partial \sigma_j} \omega
= \int_{\partial \sigma} \omega \ .  \qedhere
\]
\end{proof}

%%% Local Variables: 
%%% mode: latex
%%% TeX-master: "notes"
%%% End: 
