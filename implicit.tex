\chapter{Inverse and Implicit Function
Theorems} \label{chaptInvImplTh}

\section{Review}

We review two of the main results about differentiation that will be
very useful later.  The proofs of these results can be found in any
good textbooks in analysis including \cite{S}. 

\begin{theorem}[Inverse Function Theorem]
Let $V$ be an open subset of $\DS \RR^n$ and
$\DS f:V \to \RR^n$ be a continuously
differentiable function.  Suppose that $\diff f(\VEC{a}) \neq 0$ for some
$\VEC{a} \in V$.  Then, there exist an open neighbourhood $U \subset V$ of
$\VEC{a}$ and an open neighbourhood $W$ of $f(\VEC{a})$ such that $f:V\to W$
is one-to-one and onto, the function $\DS f^{-1}:W\to V$
is continuously differentiable and
\[
\diff f^{-1} (\VEC{y}) = \big( \diff f(f^{-1}(\VEC{y})) \big)^{-1}
\]
for all $\VEC{y} \in W$.
\end{theorem}

A consequence of the Inverse Function Theorem is the following theorem.

\begin{theorem}[Implicit Function Theorem]
Let $\DS f:U\times V\to \RR^m$ be a continuously
differentiable function where $\DS U\subset \RR^n$ and
$\DS V \subset \RR^m$ are open sets.  Suppose that
$f(\VEC{a}, \VEC{b}) = \VEC{0}$ for some $(\VEC{a},\VEC{b}) \in U\times V$.
Let
\[
A = \begin{pmatrix}
\DS \pdydx{f_1}{x_{1}}(\VEC{a},\VEC{b}) & \ldots &
\DS \pdydx{f_1}{x_{n}}(\VEC{a},\VEC{b}) \\[1em]
\vdots & \ddots & \vdots \\
\DS \pdydx{f_m}{x_{1}}(\VEC{a},\VEC{b}) & \ldots &
\DS \pdydx{f_m}{x_{n}}(\VEC{a},\VEC{b})
\end{pmatrix}
\quad \text{and} \quad
B =
\begin{pmatrix}
\DS \pdydx{f_1}{x_{n+1}}(\VEC{a},\VEC{b}) & \ldots &
\DS \pdydx{f_1}{x_{n+m}}(\VEC{a},\VEC{b}) \\[1em]
\vdots & \ddots & \vdots \\
\DS \pdydx{f_m}{x_{n+1}}(\VEC{a},\VEC{b}) & \ldots &
\DS \pdydx{f_m}{x_{n+m}}(\VEC{a},\VEC{b})
\end{pmatrix} \ .
\]
If $\det B \neq 0$, then there exist open neighbourhoods
$U_1 \subset U$ of $\VEC{a}$ and
$V_1 \subset V$ of $\VEC{b}$ with the property that for each
$\VEC{x} \in U_1$ there exists a unique $g(\VEC{x}) \in V_1$ such that
$\DS f(\VEC{x}, g(\VEC{x}))=0$.  Moreover, the function
$g:U_1\to V_1$ is continuously differentiable and
$\DS \diff g(\VEC{a}) = - B^{-1} A$.
\end{theorem}

The Implicit Function Theorem is in fact equivalent to the Inverse Function
Theorem.  One is a consequence of the other.

\section{Local Immersion Theorem}

\begin{theorem}[Local Immersion Theorem]\label{implLImT}
Let $V$ be an open subset of $\DS \RR^n$ and
$\DS f:V\to \RR^m$ be a continuously
differentiable function with $m \leq n$.  Suppose that there exists
$\VEC{a} \in V$ such that $f(\VEC{a}) = 0$ and $\diff f(\VEC{a})$ has rank
$m$.  Then there exist an open set $\DS W \subset \RR^n$, an
open neighbourhood $U \subset V$ of $\VEC{a}$ and a diffeomorphism
$h:W \to U$ such that $\DS f(h(\VEC{x}))
= \begin{pmatrix}x_1 & x_2 & \ldots &  x_m\end{pmatrix}^\top$
for $\VEC{x} \in W$.
\end{theorem}

In the previous theorem, the function $h$ is a local change of coordinates
near $\VEC{a}$.

\begin{proof}
Since $\diff f(\VEC{a})$ has rank $m$, there exists an
$(m \times m)$-submatrix of $\diff f(\VEC{a})$ with non-zero determinant.  We
choose a permutation of the coordinates
$\DS P:\RR^n \to \RR^n$ such that the
first $m$ columns of $\diff f(\VEC{a})\, P$ formed an \nm{m}{m}-matrix with
non-zero determinant.  Since $P$ is a permutation, it is its own inverse. 

Let $\VEC{b} = P(\VEC{a})$.  The first $m$ columns of
$\diff (f \circ P)(\VEC{b})$ form an \nm{m}{m}-matrix with
non-zero determinant because
$\DS \diff (f\circ P) (\VEC{b}) = \diff f(P(\VEC{b})) P =
\diff f(\VEC{a})P$.

Let
\[
M = 
\begin{pmatrix}
\DS \pdydx{(f_1\circ P)}{x_1}(\VEC{b}) & \ldots &
\DS \pdydx{(f_1\circ P)}{x_m}(\VEC{b}) \\
\vdots & \ddots & \vdots \\
\DS \pdydx{(f_m\circ P)}{x_1}(\VEC{b}) & \ldots &
\DS \pdydx{(f_m\circ P)}{x_m}(\VEC{b})
\end{pmatrix} \ .
\]
It is the \nm{m}{m}-submatrix of $\diff (f\circ P)(\VEC{b})$ formed by
the first $m$ columns.  Consider the map
\begin{align*}
H: P(V) & \to \RR^n \\
\VEC{x} & \mapsto
\begin{pmatrix} (f_1 \circ P)(\VEC{x}) & \ldots & (f_m\circ P)(\VEC{x})
& x_{m+1} & x_{m+2}  & \ldots & x_n \end{pmatrix}^\top
\end{align*}
We have
$\DS H(\VEC{b}) = \begin{pmatrix}
0 & \ldots & 0 & b_{m+1} & b_{m+2} & \ldots & b_n \end{pmatrix}^\top$
because $(f\circ P)(\VEC{b}) = f(P(\VEC{b})) = f(\VEC{a}) = \VEC{0}$, and
\[
\det \diff H(\VEC{b}) =
\det \begin{pmatrix}
M & * \\
0 & \Id_{n-m}
\end{pmatrix} = \det M \neq 0 \ .
\]
It follows from the Inverse Function Theorem that there exist
an open neighbourhood
$\DS U_1 \subset P(V) \subset \RR^n$ of $\VEC{b}$ and
an open neighbourhood $\DS W_1 \subset \RR^n$ of
$\DS \begin{pmatrix} 0 &\ldots &\ 0 &
b_{m+1} & b_{m+2} & \ldots & b_n \end{pmatrix}^\top$ such that
$H:U_1 \to W_1$ is a diffeomorphism.  Let
$\DS \pi:\RR^n \to \RR^m$ be the
projection on the first $m$ components.  We have
\[
(f\circ P)(H^{-1}(\VEC{x})) = (\pi \circ H)(H^{-1}(\VEC{x}))
= \pi(\VEC{x}) =
\begin{pmatrix} x_1 & x_2 & \ldots & x_m \end{pmatrix}^\top
\]
for $\VEC{x} \in W_1$.  We have the following commutative diagram.
\[
\begin{CD}
V @>{f}>> \RR^m \\
@A{P}AA @AA{\pi}A \\
U_1 @<<{H^{-1}}< W_1
\end{CD}
\]
The conclusion of the theorem is given by $U = P(U_1)$, $W=W_1$ and
$\DS h = P \circ H^{-1}$.
\end{proof}

%%% Local Variables: 
%%% mode: latex
%%% TeX-master: "notes"
%%% End: 
