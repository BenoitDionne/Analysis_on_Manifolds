\documentclass[titlepage,twoside,12pt]{book}

%%%%%%%%%%%%%%%%%%%%%%%%%%%%%%%%%%%%%%%%%%%%%%%%%%%%%%%%%%%%%%%%%%%%%%
% Package required for the logical manipulations
%%%%%%%%%%%%%%%%%%%%%%%%%%%%%%%%%%%%%%%%%%%%%%%%%%%%%%%%%%%%%%%%%%%%%%
\usepackage{ifthen}

%%%%%%%%%%%%%%%%%%%%%%%%%%%%%%%%%%%%%%%%%%%%%%%%%%%%%%%%%%%%%%%%%%%%%%
% AMS packages
%%%%%%%%%%%%%%%%%%%%%%%%%%%%%%%%%%%%%%%%%%%%%%%%%%%%%%%%%%%%%%%%%%%%%%
\usepackage{amsmath}
\usepackage{amsthm}
\usepackage{amssymb}
% \usepackage{txfonts}   % I prefer the output without these fonts
\usepackage{bm}          % Using \bm in the title of a chapter,
                         % section ... may fail with an error message
                         % like "Illegal parameter number in
                         % definition of \BKM@title."  \mathbf{}
                         % should be used in this situation.  It is
                         % not a perfect fix but it works in this
                         % document.
% \newcommand\bmmar{0}   % to prevent pre-allocating math alphabets.
\usepackage{amscd}       % For simple commutative diagrams
\usepackage[all]{xy}     % For fancy commutative diagrams

\allowdisplaybreaks
% \CompileMatrices       % It is supposed to speed up successive
                         % compilations of xymatrix

%%%%%%%%%%%%%%%%%%%%%%%%%%%%%%%%%%%%%%%%%%%%%%%%%%%%%%%%%%%%%%%%%%%%%%
% Structure of the theorems, propositions, ...
%%%%%%%%%%%%%%%%%%%%%%%%%%%%%%%%%%%%%%%%%%%%%%%%%%%%%%%%%%%%%%%%%%%%%%

\usepackage[breakable]{tcolorbox}
\tcbuselibrary{skins}

\newcommand{\dfn}{Definition}
\newcommand{\prp}{Proposition}
\newcommand{\thm}{Theorem}
\newcommand{\crl}{Corollary}
\newcommand{\lmm}{Lemma}
\newcommand{\alg}{Algorithm}
\newcommand{\cod}{Code}
\newcommand{\qst}{Question(s)}

\newcounter{focus}[section]
\renewcommand\thefocus{\thesection.\arabic{focus}}
\newcommand{\focuscolor}{violet}

\newenvironment{focus}[2][TTT]{%
  \refstepcounter{focus}%
  \ifthenelse{\equal{\dfn}{#2}}{\renewcommand{\focuscolor}{purple}}{%
    \ifthenelse{\equal{\alg}{#2}}{\renewcommand{\focuscolor}{red}}{%
      \ifthenelse{\equal{\cod}{#2}}{\renewcommand{\focuscolor}{yellow}}{}}}%
  \begin{tcolorbox}[enhanced,breakable,colback=\focuscolor!5!white,colframe=\focuscolor!50!white,colbacktitle=\focuscolor!50!white,coltitle=black,drop fuzzy shadow=black!50!white,title={\bfseries #2 \thefocus}%
    \ifthenelse{\equal{TTT}{#1}}{}{\ ({\bfseries #1})}]%
    \renewcommand{\labelitemi}{\textbullet}%
  }{\end{tcolorbox}}

\newenvironment{focus*}[2][TTT]{%
  \ifthenelse{\equal{\dfn}{#2}}{\renewcommand{\focuscolor}{purple}}{%
    \ifthenelse{\equal{\alg}{#2}}{\renewcommand{\focuscolor}{red}}{%
      \ifthenelse{\equal{\cod}{#2}}{\renewcommand{\focuscolor}{yellow}}{
        \ifthenelse{\equal{\qst}{#2}}{\renewcommand{\focuscolor}{black}}{}}}}%
  \begin{tcolorbox}[enhanced,breakable,colback=\focuscolor!5!white,colframe=\focuscolor!50!white,colbacktitle=\focuscolor!50!white,coltitle=black,drop fuzzy shadow=black!50!white,title={\bfseries #2}%
    \ifthenelse{\equal{TTT}{#1}}{}{\ ({\bfseries #1})}]%
    \renewcommand{\labelitemi}{\textbullet}%
  }{\end{tcolorbox}}

\newenvironment{defn}[1][TTT]{%
\ifthenelse{\equal{TTT}{#1}}{\begin{focus}{\dfn}}{\begin{focus}[#1]{\dfn}}%
}{\end{focus}}
\newenvironment{prop}[1][TTT]{%
\ifthenelse{\equal{TTT}{#1}}{\begin{focus}{\prp}}{\begin{focus}[#1]{\prp}}%
}{\end{focus}}
\newenvironment{theorem}[1][TTT]{%
\ifthenelse{\equal{TTT}{#1}}{\begin{focus}{\thm}}{\begin{focus}[#1]{\thm}}%
}{\end{focus}}
\newenvironment{cor}[1][TTT]{%
\ifthenelse{\equal{TTT}{#1}}{\begin{focus}{\crl}}{\begin{focus}[#1]{\crl}}%
}{\end{focus}}
\newenvironment{lemma}[1][TTT]{%
\ifthenelse{\equal{TTT}{#1}}{\begin{focus}{\lmm}}{\begin{focus}[#1]{\lmm}}%
}{\end{focus}}
\newenvironment{algo}[1][TTT]{%
\ifthenelse{\equal{TTT}{#1}}{\begin{focus}{\alg}}{\begin{focus}[#1]{\alg}}%
}{\end{focus}}
\newenvironment{code}[1][TTT]{%
\ifthenelse{\equal{TTT}{#1}}{\begin{focus}{\cod}}{\begin{focus}[#1]{\cod}}%
}{\end{focus}}
\newenvironment{Quest}[1][TTT]{%
\ifthenelse{\equal{TTT}{#1}}{\begin{focus}{\qst}}{\begin{focus}[#1]{\qst}}%
}{\end{focus}}

\newenvironment{defn*}[1][TTT]{%
\ifthenelse{\equal{TTT}{#1}}{\begin{focus*}{\dfn}}{\begin{focus*}[#1]{\dfn}}%
}{\end{focus*}}
\newenvironment{prop*}[1][TTT]{%
\ifthenelse{\equal{TTT}{#1}}{\begin{focus*}{\prp}}{\begin{focus*}[#1]{\prp}}%
}{\end{focus*}}
\newenvironment{theorem*}[1][TTT]{%
\ifthenelse{\equal{TTT}{#1}}{\begin{focus*}{\thm}}{\begin{focus*}[#1]{\thm}}%
}{\end{focus*}}
\newenvironment{cor*}[1][TTT]{%
\ifthenelse{\equal{TTT}{#1}}{\begin{focus*}{\crl}}{\begin{focus*}[#1]{\crl}}%
}{\end{focus*}}
\newenvironment{lemma*}[1][TTT]{%
\ifthenelse{\equal{TTT}{#1}}{\begin{focus*}{\lmm}}{\begin{focus*}[#1]{\lmm}}%
}{\end{focus*}}
\newenvironment{algo*}[1][TTT]{%
\ifthenelse{\equal{TTT}{#1}}{\begin{focus*}{\alg}}{\begin{focus*}[#1]{\alg}}%
}{\end{focus*}}
\newenvironment{code*}[1][TTT]{%
\ifthenelse{\equal{TTT}{#1}}{\begin{focus*}{\cod}}{\begin{focus*}[#1]{\cod}}%
}{\end{focus*}}
\newenvironment{Quest*}[1][TTT]{%
\ifthenelse{\equal{TTT}{#1}}{\begin{focus*}{\qst}}{\begin{focus*}[#1]{\qst}}%
}{\end{focus*}}

%%%%%%%%%%%%%%%%%%%%%%%%%%%%%%%%%%%%%%%%%%%%%%%%%%%%%%
% With the packages  mdframed

% \usepackage{mdframed}

% \newtheoremstyle{newstyle}{\topsep}{\topsep}{\rm}{}{\bfseries}{}{\newline}{#1 #2 (#3)}
% \theoremstyle{newstyle}

% \mdfdefinestyle{mythmstyle}{leftmargin = 0pt,%
% rightmargin = 0pt,%
% skipabove = 1em,%
% skipbelow = 1em,%
% innerleftmargin = 0pt,%
% innerrightmargin = 0pt,%
% innertopmargin = 0em,%
% innerbottommargin = 0.5em,%
% linewidth = 1.5pt,%
% innerlinewidth = 0pt,%
% middlelinewidth = 0pt,%
% outerlinewidth = 0pt,%
% roundcorner = 0pt,%
% linecolor = black,%
% innerlinecolor = black,%
% middlelinecolor = black,%
% outerlinecolor = black,%
% backgroundcolor = white,%
% fontcolor = black,%
% topline = true,%
% rightline = false,%
% leftline = false,%
% bottomline = true,%
% shadow = false}

% \mdtheorem[style=mythmstyle]{theorem}{Theorem}[section]
% \mdtheorem[style=mythmstyle]{lemma}[theorem]{Lemma}
% \mdtheorem[style=mythmstyle]{defn}[theorem]{Definition}
% \mdtheorem[style=mythmstyle]{cor}[theorem]{Corollary}
% \mdtheorem[style=mythmstyle]{prop}[theorem]{Proposition}
% \mdtheorem[style=mythmstyle]{code}[theorem]{Code}
% \mdtheorem[style=mythmstyle]{algo}[theorem]{Algorithm}

%%%%%%%%%%%%%%%%%%%%%%%%%%%%%%%%%%%%%%%%%%%%%%%%%%%%%%
% With the packages  ntheorem 
%
% \usepackage[amsmath]{ntheorem}

% \theoremstyle{break}
% %\theorempostskipamount{}
% %\theorempreskipamount{}
% \theoremprework{\smallskip\rule[-1ex]{0.9\textwidth}{1pt}}
% \theorempostwork{\vspace{-1.5em}\rule{0.9\textwidth}{1pt}\newline}
% \theorembodyfont{\rm}
% \newtheorem{theorem}{Theorem}[section]

% \theoremprework{\smallskip\rule[-1ex]{0.9\textwidth}{1pt}}
% \theorempostwork{\vspace{-1.5em}\rule{0.9\textwidth}{1pt}\newline}
% \newtheorem{lemma}[theorem]{Lemma}

% \theoremprework{\smallskip\rule[-1ex]{0.9\textwidth}{1pt}}
% \theorempostwork{\vspace{-1.5em}\rule{0.9\textwidth}{1pt}\newline}
% \newtheorem{defn}[theorem]{Definition}

% \theoremprework{\smallskip\rule[-1ex]{0.9\textwidth}{1pt}}
% \theorempostwork{\vspace{-1.5em}\rule{0.9\textwidth}{1pt}\newline}
% \newtheorem{cor}[theorem]{Corollary}

% \theoremprework{\smallskip\rule[-1ex]{0.9\textwidth}{1pt}}
% \theorempostwork{\vspace{-1.5em}\rule{0.9\textwidth}{1pt}\newline}
% \newtheorem{prop}[theorem]{Proposition}

% \theoremprework{\smallskip\rule[-1ex]{0.9\textwidth}{1pt}}
% \theorempostwork{\vspace{-1.5em}\rule{0.9\textwidth}{1pt}\newline}
% \newtheorem{method}[theorem]{Method}

%%%%%%%%%%%%%%%%%%%%%%%%%%%%%%%%%%%%%%%%%%%%%%%%%%%%%%
% Without the packages  ntheorem  or  mdframed
%
% \usepackage{amsthm}
%
% \newtheoremstyle{newstyle}{\topsep}{\topsep}{\rm}{}{\bfseries}{}{\newline}{#1 #2 (#3)}
% \theoremstyle{newstyle}
% \newtheorem{theorem}{Theorem}[section]
% \newtheorem{lemma}[theorem]{Lemma}
% \newtheorem{defn}[theorem]{Definition}
% \newtheorem{cor}[theorem]{Corollary}
% \newtheorem{prop}[theorem]{Proposition}
% \newtheorem{method}[theorem]{Method}

\numberwithin{equation}{section}

%%%%%%%%%%%%%%%%%%%%%%%%%%%%%%%%%%%%%%%%%%%%%%%%%%%%%%%%%%%%%%%%%%%%%%
% Structure of the chapters, sections, ...
%%%%%%%%%%%%%%%%%%%%%%%%%%%%%%%%%%%%%%%%%%%%%%%%%%%%%%%%%%%%%%%%%%%%%%
\setcounter{secnumdepth}{2}

%%%%%%%%%%%%%%%%%%%%%%%%%%%%%%%%%%%%%%%%%%%%%%%%%%%%%%%%%%%%%%%%%%%%%%
% Format of the page
%%%%%%%%%%%%%%%%%%%%%%%%%%%%%%%%%%%%%%%%%%%%%%%%%%%%%%%%%%%%%%%%%%%%%%
\raggedbottom
\usepackage[letterpaper,lmargin=1in,height=9in,width=6.5in,twoside,top=1in]{geometry}
\setlength\parskip{1ex plus 0.3ex minus 0ex}
\setlength\parindent{3ex}
\addtolength{\skip\footins}{2ex}

%%%%%%%%%%%%%%%%%%%%%%%%%%%%%%%%%%%%%%%%%%%%%%%%%%%%%%%%%%%%%%%%%%%%%%
% Header
%
% 1) the page number on the left followed by the title of the
%    chapter for the even numbered pages (left pages).
% 2) the page number on the right preceded by the title of the
%    section for the odd numbered pages (right pages).
% For a one-sided document, only 2 is applied.
%%%%%%%%%%%%%%%%%%%%%%%%%%%%%%%%%%%%%%%%%%%%%%%%%%%%%%%%%%%%%%%%%%%%%%
\usepackage{fancyhdr}
\pagestyle{fancy}
\setlength{\headheight}{1.1\baselineskip}
\setlength{\headsep}{1.5\baselineskip}
\renewcommand{\chaptermark}[1]{\markboth{\thechapter.\ #1}{}}
\renewcommand{\sectionmark}[1]{\markright{\thesection.\ #1}}
\renewcommand{\headrulewidth}{0.5pt}
\renewcommand{\plainheadrulewidth}{0pt}
\fancyhf{}
\fancyhead[LE,RO]{\small\thepage}
\fancyhead[LO]{\small\rightmark}
\fancyhead[RE]{\small\leftmark}

%%%%%%%%%%%%%%%%%%%%%%%%%%%%%%%%%%%%%%%%%%%%%%%%%%%%%%%%%%%%%%%%%%%%%%
% Hyper references
%%%%%%%%%%%%%%%%%%%%%%%%%%%%%%%%%%%%%%%%%%%%%%%%%%%%%%%%%%%%%%%%%%%%%%
\usepackage[pdfa=true]{hyperref}
\hypersetup{colorlinks=true,linkcolor=blue,citecolor=blue,urlcolor=blue}

%%%%%%%%%%%%%%%%%%%%%%%%%%%%%%%%%%%%%%%%%%%%%%%%%%%%%%%%%%%%%%%%%%%%%%
% For chapters without a chapter number
%
% You will have to add after the title of the chapter the
% following commands if you want your tables to be numbered  A.1, ...
%
% \setcounter{table}{0}
% \renewcommand{\thetable}{A.\arabic{table}}
%
% Add similar commands for the number of the equations, the sections,
% etc.
%%%%%%%%%%%%%%%%%%%%%%%%%%%%%%%%%%%%%%%%%%%%%%%%%%%%%%%%%%%%%%%%%%%%%%
\newcommand{\nonumchapter}[1]{
  \chapter*{#1}
  \markboth{#1}{#1}
  \addcontentsline{toc}{chapter}{#1}
}

%%%%%%%%%%%%%%%%%%%%%%%%%%%%%%%%%%%%%%%%%%%%%%%%%%%%%%%%%%%%%%%%%%%%%%
% For sections without a number
%
% You will have to add after the title of the chapter the
% following commands if you want your equations to be numbered
% chapter_number.equation_number
% 
% \renewcommand{\theequation}{\thechapter.\arabic{equation}}
%
% Add similar commands for the number of the tables, etc.
%%%%%%%%%%%%%%%%%%%%%%%%%%%%%%%%%%%%%%%%%%%%%%%%%%%%%%%%%%%%%%%%%%%%%%
\newcommand{\nonumsection}[1]{
  \section*{#1}
  \addcontentsline{toc}{section}{#1}
}

%%%%%%%%%%%%%%%%%%%%%%%%%%%%%%%%%%%%%%%%%%%%%%%%%%%%%%%%%%%%%%%%%%%%%%
% Chapter heading for the part of the document for the solutions to
% the exercises
%%%%%%%%%%%%%%%%%%%%%%%%%%%%%%%%%%%%%%%%%%%%%%%%%%%%%%%%%%%%%%%%%%%%%%
% \makeatletter
% \newcounter{chapterS}

% \let\chapterBackup=\chapter
% \newcommand\CsolPartStart{\renewcommand\chapter{ \@afterindentfalse%
%     \secdef\Csol\sCsol}%
%   \setcounter{chapterS}{0}%
%   \renewcommand\thechapterS{\arabic{chapterS}}}

% \newcommand\Csol[2][?]{%
%   \refstepcounter{chapterS}%
%   \addcontentsline{toc}{chapter}%
%   {\protect\numberline{\thechapterS}#1}%
%   {\vspace*{2em}\noindent\large\bfseries \chaptername\ \thechapterS\quad {#2}\vspace*{1em}}%
%   \sectionmark{#1}%
%   % \@afterheading\addvspace{1em}   %  Cannot be used in horizontal mode ???
% }

% \newcommand\sCsol[1]{%
%   {\noindent\large\bfseries \chaptername\ {#1}}%
%   \@afterheading\addvspace{\baselineskip}}

% \newcommand\CsolPartEnd{\let\chapter=\chapterBackup}

% \makeatother

%%%%%%%%%%%%%%%%%%%%%%%%%%%%%%%%%%%%%%%%%%%%%%%%%%%%%%%%%%%%%%%%%%%%%%
% Table of contents, list of figures, list of tables, ...
% If your manuscript has section numbers larger than 9, many tables or
% many figures, the numbers in the table of contents, the list of tables
% or the list of figures may overlap with the titles/descriptions. This
% should solve the problem.
%%%%%%%%%%%%%%%%%%%%%%%%%%%%%%%%%%%%%%%%%%%%%%%%%%%%%%%%%%%%%%%%%%%%%%
\makeatletter
\renewcommand\l@section{\@dottedtocline{1}{1em}{4em}}
\renewcommand\l@subsection{\@dottedtocline{2}{2em}{4em}}
\renewcommand\@pnumwidth{3em}
\renewcommand\@tocrmarg{4em}
\renewcommand\l@figure[2]{\@dottedtocline{1}{1em}{4em}{#1}{#2}}
\renewcommand\l@table[2]{\@dottedtocline{1}{1em}{4em}{#1}{#2}}
\makeatother

% Stuff for the table of contents (requires titlesec because \filright, ...
% are used)
\usepackage{titlesec}
\usepackage[dotinlabels]{titletoc}

% The content label doesn't appear.  It is overwritten by something
% else.   BUG
\titlecontents{part}[0pt]{\addvspace{2em}\bfseries\titlerule[1pt]\filright}
{\contentslabel{{\large\partname\ \thecontentslabel\quad}}{1em}}
{}{\hfill\contentspage}[\addvspace{1ex}]

\titlecontents{chapter}[0pt]{\addvspace{2ex}\bfseries}
{{\large\chaptername\ \thecontentslabel\quad}}
{}{\hfill\contentspage}[\addvspace{1ex}]

\newcommand{\UOtableofcontents}{
  \tableofcontents
  \contentsfinish
  \cleardoublepage
}

\newcommand{\UOtableslist}{
  \addcontentsline{toc}{chapter}{List of Tables}
  \listoftables
  \cleardoublepage
}

\newcommand{\UOfigureslist}{
  \addcontentsline{toc}{chapter}{Table of Figures}
  \listoffigures
  \cleardoublepage
}

%%%%%%%%%%%%%%%%%%%%%%%%%%%%%%%%%%%%%%%%%%%%%%%%%%%%%%%%%%%%%%%%%%%%%%
% To produce questions and solutions for the exercises
%%%%%%%%%%%%%%%%%%%%%%%%%%%%%%%%%%%%%%%%%%%%%%%%%%%%%%%%%%%%%%%%%%%%%%
\newcounter{questNBR}[chapter]
\renewcommand\thequestNBR{\thechapter.\arabic{questNBR}}

\newenvironment{question}[1][TTT]{%
  \refstepcounter{questNBR}\noindent%
  {\bfseries Question \arabic{chapter}.\arabic{questNBR}}%
  \ifthenelse{\equal{TTT}{#1}}{}{\ ({\bfseries #1})} \\ \noindent}{}

\newenvironment{sol}[1][TTT]{\noindent{\bfseries Solution}\ %
\ifthenelse{\equal{TTT}{#1}}{\\ \noindent}{({\bfseries #1}) \\ \noindent} %
\normalfont}{\hspace*{\fill} $\heartsuit$}

%%%%%%%%%%%%%%%%%%%%%%%%%%%%%%%%%%%%%%%%%%%%%%%%%%%%%%%%%%%%%%%%%%%%%%
% Proof environment
%%%%%%%%%%%%%%%%%%%%%%%%%%%%%%%%%%%%%%%%%%%%%%%%%%%%%%%%%%%%%%%%%%%%%%

% The filled box corresponding to the empty box suggested by the AMS
% \renewcommand{\qedsymbol}{$\blacksquare$}
% I prefer this box
\renewcommand{\qedsymbol}{\rule{0.4em}{0.7em}}

% I prefer bold characters.
\renewcommand*{\proofname}{{\normalfont\bfseries Proof}}

\makeatletter
\renewenvironment{proof}[1][\proofname]{\par\pushQED{\qed}%
\normalfont \topsep6\p@\@plus6\p@\relax
\trivlist
\item\relax {\normalfont\bfseries #1\@addpunct{.}}\newline
}{\popQED\endtrivlist\@endpefalse}
\makeatother

%%%%%%%%%%%%%%%%%%%%%%%%%%%%%%%%%%%%%%%%%%%%%%%%%%%%%%%%%%%%%%%%%%%%%%
% To be able to draw circles, ovals, ... in the picture environment.
%%%%%%%%%%%%%%%%%%%%%%%%%%%%%%%%%%%%%%%%%%%%%%%%%%%%%%%%%%%%%%%%%%%%%%
\setlength{\unitlength}{1cm}
% \usepackage[pdftex]{pict2e}

%%%%%%%%%%%%%%%%%%%%%%%%%%%%%%%%%%%%%%%%%%%%%%%%%%%%%%%%%%%%%%%%%%%%%%
% To get double angle bracket, on may use the package MnSymbol.
% However, this package conflicts with ams symbols.
% Here is the code in MnSymbol to produce \llangle and \rrangle
%%%%%%%%%%%%%%%%%%%%%%%%%%%%%%%%%%%%%%%%%%%%%%%%%%%%%%%%%%%%%%%%%%%%%%

\makeatletter
\DeclareFontFamily{OMX}{MnSymbolE}{}
\DeclareSymbolFont{MnLargeSymbols}{OMX}{MnSymbolE}{m}{n}
\SetSymbolFont{MnLargeSymbols}{bold}{OMX}{MnSymbolE}{b}{n}
\DeclareFontShape{OMX}{MnSymbolE}{m}{n}{
    <-6>  MnSymbolE5
   <6-7>  MnSymbolE6
   <7-8>  MnSymbolE7
   <8-9>  MnSymbolE8
   <9-10> MnSymbolE9
  <10-12> MnSymbolE10
  <12->   MnSymbolE12
}{}
\DeclareFontShape{OMX}{MnSymbolE}{b}{n}{
    <-6>  MnSymbolE-Bold5
   <6-7>  MnSymbolE-Bold6
   <7-8>  MnSymbolE-Bold7
   <8-9>  MnSymbolE-Bold8
   <9-10> MnSymbolE-Bold9
  <10-12> MnSymbolE-Bold10
  <12->   MnSymbolE-Bold12
}{}

\let\llangle\@undefined
\let\rrangle\@undefined
\DeclareMathDelimiter{\llangle}{\mathopen}%
                     {MnLargeSymbols}{'164}{MnLargeSymbols}{'164}
\DeclareMathDelimiter{\rrangle}{\mathclose}%
                     {MnLargeSymbols}{'171}{MnLargeSymbols}{'171}
\makeatother

%%%%%%%%%%%%%%%%%%%%%%%%%%%%%%%%%%%%%%%%%%%%%%%%%%%%%%%%%%%%%%%%%%%%%%
% Package to properly format some parts of the text
%%%%%%%%%%%%%%%%%%%%%%%%%%%%%%%%%%%%%%%%%%%%%%%%%%%%%%%%%%%%%%%%%%%%%%
\usepackage{ulem}        % To properly underline words, sentences, ...
\usepackage{url}         % To get normal url
\usepackage{verbatim}
\usepackage[margin=0.05\textwidth]{caption}

%%%%%%%%%%%%%%%%%%%%%%%%%%%%%%%%%%%%%%%%%%%%%%%%%%%%%%%%%%%%%%%%%%%%%%
% For minipage
%%%%%%%%%%%%%%%%%%%%%%%%%%%%%%%%%%%%%%%%%%%%%%%%%%%%%%%%%%%%%%%%%%%%%%
\newlength{\miniwidth}
\setlength{\miniwidth}{\textwidth}
\addtolength{\miniwidth}{-2cm}

%%%%%%%%%%%%%%%%%%%%%%%%%%%%%%%%%%%%%%%%%%%%%%%%%%%%%%%%%%%%%%%%%%%%%%
% For tables that extent over the next page.
%%%%%%%%%%%%%%%%%%%%%%%%%%%%%%%%%%%%%%%%%%%%%%%%%%%%%%%%%%%%%%%%%%%%%%
% \usepackage{longtable}

%%%%%%%%%%%%%%%%%%%%%%%%%%%%%%%%%%%%%%%%%%%%%%%%%%%%%%%%%%%%%%%%%%%%%%
% For (horizontal and more) lists inside a paragraph
%%%%%%%%%%%%%%%%%%%%%%%%%%%%%%%%%%%%%%%%%%%%%%%%%%%%%%%%%%%%%%%%%%%%%%
\usepackage{paralist}

%%%%%%%%%%%%%%%%%%%%%%%%%%%%%%%%%%%%%%%%%%%%%%%%%%%%%%%%%%%%%%%%%%%%%%
% Important variables supplied by the user
%%%%%%%%%%%%%%%%%%%%%%%%%%%%%%%%%%%%%%%%%%%%%%%%%%%%%%%%%%%%%%%%%%%%%%
\newcommand{\UO}{University of Ottawa}
\newcommand{\setUO}[1]{\renewcommand{\UO}{#1}}

\newcommand{\UOfac}{Faculty of Science}
\newcommand{\setUOfac}[1]{\renewcommand{\UOfac}{#1}}

\newcommand{\UOdept}{Department of Mathematics and Statistics}
\newcommand{\setUOdept}[1]{\renewcommand{\UOdept}{#1}}

\newcommand{\UOauthor}{Benoit Dionne}     % author
\newcommand{\setUOauthor}[1]{\renewcommand{\UOauthor}{#1}}

\newcommand{\UOtitle}{Analysis on Manifolds} % title
\newcommand{\setUOtitle}[1]{\renewcommand{\UOtitle}{#1}}

\newcommand{\UOyear}{2025}  % year
\newcommand{\setUOyear}[1]{\renewcommand{\UOyear}{#1}}

%%%%%%%%%%%%%%%%%%%%%%%%%%%%%%%%%%%%%%%%%%%%%%%%%%%%%%%%%%%%%%%%%%%%%%
% To print today's date in Canadian format
%%%%%%%%%%%%%%%%%%%%%%%%%%%%%%%%%%%%%%%%%%%%%%%%%%%%%%%%%%%%%%%%%%%%%%
\newcommand*{\dateCAN}{\renewcommand*{\today}{%
    \ifcase\day \or
    01\or 02\or 03\or 04\or 05\or 06\or 07\or 08\or 09\or 10\or
    11\or 12\or 13\or 14\or 15\or 16\or 17\or 18\or 19\or 20\or
    21\or 22\or 23\or 24\or 25\or 26\or 27\or 28\or 29\or 30\or
    31\fi/\ifcase\month \or
    01\or 02\or 03\or 04\or 05\or 06\or 07\or 08\or 09\or 10\or
    11\or 12\fi/\number\year}}

%%%%%%%%%%%%%%%%%%%%%%%%%%%%%%%%%%%%%%%%%%%%%%%%%%%%%%%%%%%%%%%%%%%%%%
% The title page
%%%%%%%%%%%%%%%%%%%%%%%%%%%%%%%%%%%%%%%%%%%%%%%%%%%%%%%%%%%%%%%%%%%%%%
\newcommand{\UOlogoCLloc}{images/UOlogo_headCL}
                                          % Location of the logo
\newcommand{\UOlogoCLopt}{3cm}            % Options for includegraphics
\newcommand{\setUOlogoCLloc}[2]{
  \renewcommand{\UOlogoCLloc}{#1}
  \renewcommand{\UOlogoCLopt}{#2}
}

\usepackage{incgraph}
\usetikzlibrary{fadings}

\newcommand{\titlePage}{
  \thispagestyle{empty}
  \begin{titlepage}
    \newsavebox{\biblio}
    \savebox{\biblio}{\includegraphics[width=7in]{images/cover_biblio}}
    \newsavebox{\building}
    \savebox{\building}{\includegraphics[width=2.5in]{images/cover_building}}
    \begin{inctext}[paper=current, target=mytarget]
      \begin{tikzpicture}
        \coordinate (A) at (0,0);
        \coordinate (B) at (8.5in,11in);
        \fill[use as bounding box,color=lime!4] (A) rectangle (B);
        \coordinate (C) at ([xshift=0.5in,yshift=0.5in]A);
        \coordinate (D) at ([xshift=-0.5in,yshift=-0.5in]B);
        \draw[rounded corners=0.25in,very thick,black] (C) rectangle (D);
        \node[inner sep=0pt,above right] (biblio) at (0.75in,2.5in) {\usebox{\biblio}};
        \fill[lime!4,path fading=south] (0.75in,4.9in) rectangle (7.80in,6in);
        \node[inner sep=0pt,below right] (stairs) at (0.75in,10in) {\usebox{\building}};
        \fill[lime!4,path fading=north] (0.75in,6.05in) rectangle (3.3in,6.8in);
        \fill[lime!4,path fading=west] (2.8in,6in) rectangle (3.26in,10in);
        \node[text width=10cm,align=flush center,font=\Huge\bfseries] at (5.5in,8in) {\UOtitle};
        \node[inner sep=0pt,above right] (UOlogo) at (1in,1in) {\includegraphics[width=\UOlogoCLopt]{\UOlogoCLloc}};
        \node[text width=6cm,above left,font=\large\bfseries] (author) at (7.5in,1in) {\UOauthor \\ \UO};
      \end{tikzpicture}
    \end{inctext}
  \end{titlepage}
  \pagecolor{white}
  \cleardoublepage
}

%%%%%%%%%%%%%%%%%%%%%%%%%%%%%%%%%%%%%%%%%%%%%%%%%%%%%%%%%%%%%%%%%%%%%%
% The open source page
%%%%%%%%%%%%%%%%%%%%%%%%%%%%%%%%%%%%%%%%%%%%%%%%%%%%%%%%%%%%%%%%%%%%%%
\newcommand{\UOopenloc}{images/open_source}     % Location of the logo
\newcommand{\UOopenopt}{2cm}             % Options for includegraphics

\newcommand{\opensource}{
  \thispagestyle{empty}

  \noindent \copyright\ \UOauthor, \UOyear\ (\UO)
  
  \noindent Introduction to the analysis on manifolds / modern
  differential geometry for an advanced undergraduate or graduate course.

  \noindent This document is available on the following sites.\\
  uO Research: \\ % http://hdl.handle.net/10393/45600 \\
  GitHub: https://github.com/BenoitDionne/Analysis\_on\_Manifolds

  \vspace*{1cm}
  
  \noindent \includegraphics[width=\UOopenopt]{\UOopenloc}

  \noindent Unless otherwise stated, these lecture notes are made
  available under the terms of the license
  \href{https://creativecommons.org/licenses/by-nc-sa/4.0/deed.en}{Creative Commons Attribution - Non Commercial-Share Alike 4.0 International} (CC BY-NC-SA 4.0)

  \vfill

  \noindent Cover Page:\\
  Casa Mil\`a (popularly known as La Pedrera) by the architect Antoni
  Gaudi, Barcelona, photo by Louise Oegema.\\ 
  Facade of the Roman library of Celsus in Ephenus, Anatolia (Asia
  Minor) in the Izmir province, photo by Jean Dionne.

  \cleardoublepage
}

%%%%%%%%%%%%%%%%%%%%%%%%%%%%%%%%%%%%%%%%%%%%%%%%%%%%%%%%%%%%%%%%%%%%%%
% The bibliography
%%%%%%%%%%%%%%%%%%%%%%%%%%%%%%%%%%%%%%%%%%%%%%%%%%%%%%%%%%%%%%%%%%%%%%
\newcommand{\UObibliography}[1]{
  % \setcounter{chapter}{100}
  \addcontentsline{toc}{chapter}{Bibliography}
  \include{#1}
  \cleardoublepage
}

%%%%%%%%%%%%%%%%%%%%%%%%%%%%%%%%%%%%%%%%%%%%%%%%%%%%%%%%%%%%%%%%%%%%%%
% The index
%%%%%%%%%%%%%%%%%%%%%%%%%%%%%%%%%%%%%%%%%%%%%%%%%%%%%%%%%%%%%%%%%%%%%%
\usepackage{makeidx}
\makeindex
\newcommand{\UOindex}{
  \setcounter{chapter}{100}
  \addcontentsline{toc}{chapter}{\indexname}
  \printindex
}

%%%%%%%%%%%%%%%%%%%%%%%%%%%%%%%%%%%%%%%%%%%%%%%%%%%%%%%%%%%%%%%%%%%%%%
% graphics, ...
%%%%%%%%%%%%%%%%%%%%%%%%%%%%%%%%%%%%%%%%%%%%%%%%%%%%%%%%%%%%%%%%%%%%%%
\usepackage{graphicx}
\usepackage{rotating}    % To rotate figures, tables, ...
\usepackage{color}
% \usepackage{tikz}
% \usepackage{pst-node,pst-plot,auto-pst-pdf}
% \usepackage{subfig}

% By default figures will be placed "here" if possible otherwise at
% the "top."
\makeatletter
\renewcommand*{\fps@figure}{ht}
\makeatother

% Methods for jpg, png, ... figures.

\newcommand{\figbox}[2]{\begin{center}%
\includegraphics[width=#2]{#1}\end{center}}

% Methods for pdf figures.
\newcommand{\pdfbox}[1]{\begin{center}\input{#1.pdf_t}\end{center}}

\newcommand{\pdfF}[5][TTT]{%
  \ifthenelse{\equal{TTT}{#1}}{\begin{figure}}{\begin{figure}[#1]}%
      {\color{purple}\rule{\textwidth}{1pt}}\begin{center}%
        \input{#2.pdf_t}\end{center}\caption[#3]{#4 \label{#5}}%
      {\color{purple}\rule{\textwidth}{1pt}}%
    \end{figure}}

\newcommand{\pdfFD}[6][TTT]{%
  \ifthenelse{\equal{TTT}{#1}}{\begin{figure}}{\begin{figure}[#1]}%
      {\color{purple}\rule{\textwidth}{1pt}}\begin{center} %
        \input{#2.pdf_t}\end{center} %
      \begin{center}\input{#3.pdf_t}\end{center} %
      \caption[#4]{#5 \label{#6}}%
      {\color{purple}\rule{\textwidth}{1pt}}%
    \end{figure}}

% Methods for png, jpg, ... figures.

\newcommand{\mathF}[6][TTT]{%
  \ifthenelse{\equal{TTT}{#1}}{\begin{figure}}{\begin{figure}[#1]}%
      {\color{purple}\rule{\textwidth}{1pt}}\begin{center} %
        \includegraphics[width=#3]{#2}\end{center} %
      \caption[#4]{#5 \label{#6}}%
      {\color{purple}\rule{\textwidth}{1pt}}%
    \end{figure}}

\newcommand{\mathFD}[7]{\begin{figure}%
    {\color{purple}\rule{\textwidth}{1pt}}\begin{center}%
      \includegraphics[width=#2]{#1}\includegraphics[width=#4]{#3}\end{center}%
    \caption[#5]{#6 \label{#7}}%
    {\color{purple}\rule{\textwidth}{1pt}}\end{figure}}

% Formats for the remarks, examples, ...
\newenvironment{rmk}[1][TTT]%
{\refstepcounter{focus} \noindent %
{\bfseries Remark \arabic{chapter}.\arabic{section}.\arabic{focus}}\ %
\ifthenelse{\equal{TTT}{#1}}{}{({\bfseries #1})}\\ \noindent %
}{\hspace*{\fill} $\spadesuit$}

\newenvironment{rmkList}%   % special case for remarks starting with a list
{\refstepcounter{focus} \noindent %
{\bfseries Remark \arabic{chapter}.\arabic{section}.\arabic{focus}}\\ %
\vspace*{-2\topskip}\noindent}%
{\vspace*{-\topskip}\hspace*{\fill} $\spadesuit$}

\newenvironment{egg}[1][TTT]{%
  \refstepcounter{focus}\noindent%
  {\bfseries Example \arabic{chapter}.\arabic{section}.\arabic{focus}}\ %
  \ifthenelse{\equal{TTT}{#1}}{}{({\bfseries #1})}\\ \noindent %
}{\hspace*{\fill} $\clubsuit$}

\newenvironment{eggList}[1][TTT]{ % special case for ex. starting with a list
  \refstepcounter{focus}\noindent%
  {\bfseries Example \arabic{chapter}.\arabic{section}.\arabic{focus}}\ %
  \ifthenelse{\equal{TTT}{#1}}{}{ ({\bfseries #1})} \\ %
  \vspace*{-2\topskip} \noindent %
}{\vspace*{-\topskip}\hspace*{\fill} $\clubsuit$}

% Title for the multiple choices, the parts of a proof, etc.
\newcommand{\subQ}[1]{\noindent {\bfseries #1})\ }
\newcommand{\subI}[1]{\noindent {\bfseries #1}:\ }
\newcommand{\stage}[1]{\noindent {\bfseries #1}) \ }

%%%%%%%%%%%%%%%%%%%%%%%%%%%%%%%%%%%%%%%%%%%%%%%%%%%%%%%%%%%%%%%%%%%%%%
% Mathematical symbols, short cuts, ...
%%%%%%%%%%%%%%%%%%%%%%%%%%%%%%%%%%%%%%%%%%%%%%%%%%%%%%%%%%%%%%%%%%%%%%
\newcommand{\NN}{\mathbb{N}}
\newcommand{\NNp}{\mathbb{N}^+}
\newcommand{\ZZ}{\mathbb{Z}}
\newcommand{\RR}{\mathbb{R}}
\newcommand{\QQ}{\mathbb{Q}}
\newcommand{\CC}{\mathbb{C}}

\newcommand{\BB}{\mathcal{B}}
\newcommand{\EE}{\mathcal{E}}
\newcommand{\FF}{\mathcal{F}}
\newcommand{\GG}{\mathcal{G}}
\newcommand{\II}{\mathcal{I}}
\newcommand{\JJ}{\mathcal{J}}
\newcommand{\LL}{\mathcal{L}}   % For linear space of bounded operators, ...
\newcommand{\KK}{\mathcal{K}}   % For linear space of compact operators, ...
\renewcommand{\SS}{\mathcal{S}} % For rapidly decreasing functions, ...
\newcommand{\TT}{\mathcal{T}}   % For trace operators on open sets, ...
\newcommand{\DD}{\mathcal{D}}   % For test functions, ...
\newcommand{\DO}{\mathcal{D}}   % For the domain of a function, ...
\newcommand{\SH}{\mathcal{S}_{\mathcal{H}}} % The set of subharmonic functions.
\newcommand{\B}{\mathcal{B}}
\newcommand{\D}{\mathcal{D}}
\newcommand{\E}{\mathcal{E}}
\newcommand{\F}{\mathcal{F}}
\newcommand{\Q}{\mathcal{Q}}
\newcommand{\U}{\mathcal{U}}
\newcommand{\X}{\mathcal{X}}
\newcommand{\Y}{\mathcal{Y}}
\newcommand{\MM}{\mathcal{M}}
\newcommand{\OO}{\mathcal{O}}
\newcommand{\PP}{\mathcal{P}}
\newcommand{\IMG}{\mathcal{R}}      % For the range of a function, ...
\newcommand{\KE}{\mathcal{N}}       % For the kernel of a function, ...
\newcommand{\GR}{\mathcal{G}}       % For the graph of a function, ...

\DeclareMathOperator*{\esssup}{ess\;sup}
\DeclareMathOperator*{\essinf}{ess\;inf}
\DeclareMathOperator{\arcsec}{arcsec}
\DeclareMathOperator{\tr}{tr}
\DeclareMathOperator{\Id}{Id}
\DeclareMathOperator{\sgn}{sgn}
\DeclareMathOperator{\diff}{D}
\DeclareMathOperator{\supp}{\,supp}
\DeclareMathOperator{\curL}{curl}
\DeclareMathOperator{\diV}{div}
\DeclareMathOperator{\graD}{\nabla}
\DeclareMathOperator{\rank}{rank}
\DeclareMathOperator{\RE}{Re}      % For the real part of a number, ...
\DeclareMathOperator{\IM}{Im}      % For the imaginary par of a number, ...
\DeclareMathOperator{\Fix}{Fix}
\DeclareMathOperator{\Per}{Per}
\DeclareMathOperator{\sep}{sep}    % For the section on ``Melnikov
                                   % Function'', file melnikov.tex
\DeclareMathOperator{\Span}{span}
\DeclareMathOperator{\diam}{diam}
\DeclareMathOperator{\mesh}{mesh}  % The mesh of a simplicial complex
\DeclareMathOperator{\St}{St}      % The star of a vertex
\DeclareMathOperator{\Sd}{Sd}      % Barycentric subdivision operator
\DeclareMathOperator{\Dm}{d}       % A measure on lin. comb. of affine maps

\newcommand{\Chi}{\mathcal{X}}        % \chi is to small for subscripts.

\newcommand{\VEC}[1]{{\bm{#1}}}
\newcommand{\ps}[2]{ \left\langle{#1} , {#2}\right\rangle }
\newcommand{\psP}[2]{ \left\langle{#1} : {#2}\right\rangle }
\newcommand{\conj}[1]{\overline{#1}}
\newcommand{\HH}{$H_{\VEC{w}}$}
\newcommand{\AP}{AP}
\newcommand{\SN}[1]{\mathrm{S}_{#1}}
\newcommand{\Sone}{S^1}
\newcommand{\torus}[1]{\mathrm{T}^{#1}}
\newcommand{\dist}[2]{\mathrm{dist}\left(#1,#2\right)}
\newcommand{\sgm}[1]{\overline{#1}}
\newcommand{\ii}{\VEC{i}}
\newcommand{\jj}{\VEC{j}}
\newcommand{\kk}{\VEC{k}}
\newcommand{\nn}{$n\times n$\,}
\newcommand{\nm}[2]{${#1}\times{#2}$\,}
\newcommand{\intpt}[1]{\lfloor{#1}\rfloor}
\newcommand{\HW}{H_{\VEC{w}}}

\newcommand{\dx}[1]{\,\mathrm{d}{#1}}
\newcommand{\dydx}[2]{\frac{\mathrm{d}#1}{\mathrm{d}{#2}}}
\newcommand{\dydxn}[3]{\frac{\displaystyle \mathrm{d}^{#3}{#1}}%
{\displaystyle \mathrm{d}{#2}^{#3}}}
\newcommand{\dfdx}[2]{\frac{\mathrm{d}}{\mathrm{d}{#2}}{#1}}
\newcommand{\dfdxn}[3]{\frac{\displaystyle \mathrm{d}^{#3}}%
{\displaystyle \mathrm{d}{#2}^{#3}}{#1}}

\newcommand{\pdydx}[2]{\frac{\partial{#1}}{\partial #2}}
\newcommand{\pdydxn}[3]{\frac{\displaystyle \partial^{#3}{#1}}%
{\displaystyle \partial #2^{#3}}}
\newcommand{\pdydxnm}[6]{\frac{\displaystyle \partial^{#4}{#1}}%
{\displaystyle \partial #2^{#5} \partial #3^{#6}}}
\newcommand{\pdydxdots}[8]{\frac{\displaystyle \partial^{#5}#1}%
{\displaystyle \partial{#2}^{#6} \partial #3^{#7} \ldots \partial #4^{#8}}}

\newcommand{\pdfdx}[2]{\frac{\partial}{\partial #2}{#1}}
\newcommand{\pdfdxn}[3]{\frac{\displaystyle \partial^{#3}}%
{\displaystyle \partial #2^{#3}}{#1}}
\newcommand{\pdfdxnm}[6]{\frac{\displaystyle \partial^{#4}}%
{\displaystyle \partial #2^{#5} \partial #3^{#6}}{#1}}

\newcommand{\fdiff}[2]{\mathrm{D}^{#2}{#1}}

\newcommand{\dtx}[1]{\Delta{#1}}
\newcommand{\dtxn}[2]{\Delta^{#1}{#2}}

\newcommand{\dsss}[2]{\,\mathrm{d}{#1}_{#2}}
\newcommand{\dss}[2]{\dsss{#1}{\VEC{#2}}}
\newcommand{\pdydxS}[2]{\frac{\tilde{\partial}{#1}}{\tilde{\partial}{#2}}}
\newcommand{\pdydxnS}[3]{\frac{\tilde{\partial}^{#3}{#1}}{\tilde{\partial}{#2}^{#3}}}

\DeclareMathOperator{\fl}{fl}
\DeclareMathOperator{\Bot}{\,\bot\,}
\newcommand{\tildev}[1]{\tilde{\VEC{#1}}}
\newcommand{\pps}[2]{\left\llangle{#1} , {#2}\right\rrangle } % pseudo scalar product

% For the section of Fast Fourier Transform
\newcommand{\fct}[1]{{\bm{#1}}}
\newcommand{\fctt}[1]{\tilde{{\bm{#1}}}}

\newcommand{\NNN}{\mathcal{N}}
\newcommand{\GL}[1]{\mathrm{GL}(#1)}

\newcommand{\sigmaU}{\underline{\sigma}}   % I have exhausted
                                           % \hat{\sigma}, ...
\newcommand{\A}{\mathcal{A}}
\newcommand{\C}{\mathcal{C}}
\newcommand{\R}{\mathcal{R}}
\newcommand{\T}{\mathcal{T}}
\newcommand{\TS}{\mathrm{T}}           % Tangent space
\newcommand{\stc}{\,\mathsf{i}}        % Standard cube
\DeclareMathOperator{\alt}{Alt}        % For tensors
\DeclareMathOperator{\Int}{Int}        % For the interior of a manifold

\newcommand{\dfNext}[2][TTT]{\ifthenelse{\equal{TTT}{#1}}{\,\mathrm{d}#2}%
{\,\mathrm{d}^{#1}#2}}
\newcommand{\df}[1][TTT]{\ifthenelse{\equal{TTT}{#1}}{\dfNext}{\dfNext[#1]}}
                                     % For differential forms
\newcommand{\dfC}{\mathrm{d}}        % differential operator for
                                     % simplicial and singular cohomology

\newcommand{\rint}{\hspace{1ex}{\mathcal R}\hspace{-1.15em}\int}
                                     % Riemann integral

\newcommand{\dotsim}[1][TTT]{\ifthenelse{\equal{TTT}{#1}}{\,\dot\sim\,}{%
\,\underset{#1}{\,\dot\sim\,}}}                % homotopic paths, ... 
\newcommand{\circsim}{\overset{\circ}{\sim}}   % contiguous equivalence, ... 
\newcommand{\hsim}{\overset{h}{\sim}}          % homologous equivalence, ... 
\newcommand{\rsim}{\overset{rel}{\sim}}    % relative homotopic equivalence, ...

% Simple double bracket.  They cannot be combined with \left, \right,
% \big, ...  For that, one needs to use \delcode to define a new
% delimiter and that requires new fonts since the double brackets do
% not exist in the standard font family.
% More fancy double brackets can be found in the package stmaryrd but
% there is a cost to it.  You have to use totally new fonts.
\newcommand{\llbracket}{\hbox{$[\kern-1pt[$}}
\newcommand{\rrbracket}{\hbox{$]\kern-1pt]$}}

\newcommand{\relC}[2][TTT]{\ifthenelse{\equal{TTT}{#1}}%
{\llbracket{#2}\rrbracket}{\llbracket{#2}\rrbracket_{#1}}}
                        % equivalence class in relative homology module, ...
% \newcommand{\edge}[2]{\llbracket{#1} , {#2}\rrbracket}
%                                      % edge equivalence
% \newcommand{\edgesim}{\overset{\triangle}{\sim}}   % edge equivalence, ... 

\newcommand{\os}[5]{\llbracket {#1}%
  \ifthenelse{\equal{}{#2}}{}{,{#2}}%
  \ifthenelse{\equal{}{#3}}{}{,{#3},\ldots}%
  \ifthenelse{\equal{}{#4}}{}{,{#4},\ldots}%
  \ifthenelse{\equal{}{#5}}{\rrbracket}{,{#5}\rrbracket}}
\newcommand{\osscript}[5]{{\text{\scriptsize$\llbracket$}} {#1}%
  \ifthenelse{\equal{}{#2}}{}{,{#2}}%
  \ifthenelse{\equal{}{#3}}{}{,{#3},\ldots}%
  \ifthenelse{\equal{}{#4}}{}{,{#4},\ldots}%
  \ifthenelse{\equal{}{#5}}{{\text{\scriptsize$\rrbracket$}}}%
  {,{#5}{\text{\scriptsize$\rrbracket$}}}}
                   % simplicial cohomology class
                   % The script version of \os is to be used in
                   % subscripts or superscripts because \llbracket and
                   % \rrbracket are not properly rescaled for a reason
                   % unknown to me.

% The \bigoplus is too big for my taste.  I have redefined it.
\let\oldbigoplus\bigoplus
\let\bigoplus\relax
\DeclareMathOperator*{\bigoplus}{\scalebox{0.9}{$\oldbigoplus$}}

%%%%%%%%%%%%%%%%%%%%%%%%%%%%%%%%%%%%%%%%%%%%%%%%%%%%%%%%%%%%%%%%%%%%%%
% Trouble
%%%%%%%%%%%%%%%%%%%%%%%%%%%%%%%%%%%%%%%%%%%%%%%%%%%%%%%%%%%%%%%%%%%%%%
\newcommand{\MORE}[1][TTT]{\begin{center}%
\ifthenelse{\equal{TTT}{#1}}{\fbox{\color{red}\bfseries TO BE COMPLETED}}{%
\fbox{\begin{minipage}{0.8\textwidth}%
\color{red}{\bfseries TO BE COMPLETED} \\ #1 \end{minipage}}}%
\end{center}}
\newcommand{\BUG}{\begin{center}\fbox{\color{red}\bfseries TO BE FIXED}\end{center}}
