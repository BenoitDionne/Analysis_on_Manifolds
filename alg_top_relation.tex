\section{Relation Between Simplicial and Singular Theory}
\label{sectRelSandS}

This section is not needed for the next chapter and, in a sense, is
not related to the goal of these lecture notes.  However, after having
introduced both the simplicial and singular homology theory, we feel
that it is important to know how they are related and thus how they
may interact.  After having read the sections about the simplicial
and singular theories, the reader has probably realized that there are
a lot of similarity between the theoretical results and techniques
used in both theories.  As we demonstrate in this section, this in not
really surprising.

The style of this section is a little lest detailed than in the rest
of these lecture notes.

\subsection{Relation Between Simplicial Homology and Singular
Homology}

We follow the presentation in \cite{MUat}.

Through out this section, $L$ is a subcomplex of a simplicial complex
$K$ and $R$ is an integral domain.  We assume that
an indexing of the vertices of $K$ has been selected.  More
specifically, we assume that $\VEC{x}_0$, $\VEC{x}_1$, $\VEC{x}_2$,
\ldots are the vertices of $K$.  In particular, we assume that there
could be countably infinitely many open simplices in $K$  This
is more general than in Sections~\ref{sectSimplHomol} and
\ref{sectSimplCohom} where we assumed that there were only a finite
number of open simplices in $K$.   Most of what we said in
Sections~\ref{sectSimplHomol} and \ref{sectSimplCohom} is true
in the case where $K$ contains countably infinitely many open
simplices.  However, there are a few exceptions.  Barycentric
subdivisions and simplicial approximation are two concepts that need
to be generalized to address the issue that $K$ may contain countably
infinitely many open simplices.  The reader may find more information
about this subject in \cite{MUat}.

Adding the possibility that the simplicial complex $K$ could contain
countably infinitely many open simplices does not really increase the
level of complexity in this section.  The only extra work is in part
(II) of the proof of Proposition~\ref{propHkEHkv1}.

\subsubsection{Ordered Simplicial Homology}

We need to enlarge $C_k(K;R)$ to be able to define a mapping
onto $S_k([K];R)$.  For instance, we need a set large enough to be
able to map an element of this set to a singular $k$-simplex
$\sigma:\Delta_k \to [K]$ representing a constant function into $[K]$.

In addition to the $k$-simplices $(s) \in K$, we
also consider elements of the form\\
$(\VEC{y}_0,\VEC{y}_1, \ldots, \VEC{y}_k)$
where $\VEC{y}_0$, $\VEC{y}_1$, \ldots, $\VEC{y}_k$ are vertices of a
$k$-simplex $(s) \in K$, and the $\VEC{y}_j$ may not be all distinct.

\begin{defn}
Elements of the form $(\VEC{y}_0,\VEC{y}_1, \ldots, \VEC{y}_k)$ as
described above are called
{\bfseries ordered $\mathbf{k}$-simplices}\index{Ordered $k$-Simplex}
of $K$.
\end{defn}

\begin{defn}
Let $\tilde{C}_k(K;R)$ be the free abelian group generated by the
elements $(\VEC{y}_0,\VEC{y}_1, \ldots, \VEC{y}_k)$ as described
above.  Elements of $\tilde{C}_k(K;R)$ are called
{\bfseries ordered $\mathbf{k}$-chains}\index{Ordered $k$-Chain}
of $K$.
\end{defn}

We define the boundary operator $\tilde{\partial}_k$ on $\tilde{C}_K(K;R)$ as
we did for the boundary operator $\partial_k$ on $C_K(K;R)$; namely,
\[
\tilde{\partial}_k (\VEC{y}_0,\VEC{y}_1, \ldots, \VEC{y}_k)
= \sum_{i=0}^k (-1)^i
(\VEC{y}_0,\VEC{y}_1,\ldots,\widehat{\VEC{y}_i}, \ldots,
\VEC{y}_k) \ .
\]

In analogy to $H_k(K;R) = \KE(\partial_k) / \IMG(\partial_{k-1})$,
we can define $\tilde{H}_k(K;R) = \KE(\tilde{\partial}_k)
/ \IMG(\tilde{\partial}_{k-1})$.  Likewise, we define the
quotient space
$\displaystyle \tilde{C}_k(K,L;R) = \tilde{C}_k(K;R)/\tilde{C}_k(L;R)$.
Since $\tilde{\partial}_k:\tilde{C}_k(L;R) \to \tilde{C}_{k-1}(L;R)$,
we may define the operator
$\overline{\tilde{\partial}}_k:\tilde{C}_k(K,L;R) \to \tilde{C}_{k-1}(K,L;R)$
by $\overline{\tilde{\partial}}_k(\relC[K,L]{c}) =
\relC[K,L]{\tilde{\partial}_k(c)}$ for all $c \in \tilde{C}_k(K;R)$, where
$\relC[K,L]{\cdot}$ represents an equivalence class in
$\tilde{C}_k(K,L;R)$.   We also define
$\displaystyle \tilde{H}_k(K,L;R) = \KE\big(\overline{\tilde{\partial}}_k\big) /
\IMG\big(\overline{\tilde{\partial}}_{k-1}\big)$.

\begin{rmk}
We will use the same notation $[\cdot]_K$ to denote
equivalence classes in $H_k(K;R)$, $\tilde{H}_k(K;R)$ and their reduced
form.  Similarly, we will use the same notation $[\cdot]_{K,L}$ to denote
equivalence classes in $H_k(K,L;R)$, $\tilde{H}_k(K,L;R)$ and there reduced
form.  This will simplify the notation.  The context is generally
sufficient to determine in which space the equivalence class belongs to.
\end{rmk}

\begin{defn}
Let $K_1$ and $K_2$ be two simplicial complexes, $R$ be an integral
domain, and $f$ be a simplicial map between $K_1$ and $K_2$.  The
function $\tilde{C}_k(f): \tilde{C}_k(K_1;R) \to \tilde{C}_k(K_2;R)$
is the linear function defined by 
$\tilde{C}_k(f)\big((\VEC{y}_0,\VEC{y}_1,\ldots,\VEC{y}_k)\big)
= (f(\VEC{y}_0),f(\VEC{y}_1),\ldots,f(\VEC{y}_k)\big)$
for all ordered $k$-simplices
$(\VEC{y}_0,\VEC{y}_1,\ldots,\VEC{y}_k)$ and
extended linearly to all of $\tilde{C}_k(K_1;R)$.
\end{defn}

Since $\displaystyle \tilde{C}_k(f)$ commutes with
$\displaystyle \tilde{\partial}_k$, we get the following proposition.

\begin{prop}
In the context of the previous definition, the map
$\displaystyle \tilde{H}_k(f): \tilde{H}_k(K_1;R) \to \tilde{H}_k(K_2,R)$
defined by
$\displaystyle \tilde{H}_k(f)([c]_{K_1}) = [ \tilde{C}_k(f)(c)]_{K_2}$ for all
$\displaystyle [c]_{K_1} \in \tilde{H}_k(K_1;R)$ is an homomorphism.
\end{prop}

The following result is easy to proof.

\begin{prop}
Suppose that $L_i$ is a subcomplex of a simplicial complex
$K_i$ for $i =1,2$, and that $R$ is an integral domain.
If $f:(K_1,L_1) \to (K_2,L_2)$, then the map
$\displaystyle \tilde{H}_k(f): \tilde{H}_k(K_1,L_1;R) \to
\tilde{H}_k(K_2,L_2,R)$ defined by
$\displaystyle \tilde{H}_k(f)(\relC[K_1,L_1]{c}) =
\relC[K_2,L_2]{\tilde{C}_k(f)(c)}$ for all
$\displaystyle \relC[K_1,L_1]{c} \in \tilde{H}_k(K_1,L_1;R)$ is an
homomorphism.
\end{prop}

The theory of reduced and relative simplicial homology presented
in Section~\ref{ssectSimplRDh} can be replicated to obtain the same
theoretical results for reduced and relative ordered simplicial
homology.

The free abelian group $\tilde{C}_k(K;R)$ is a useful tool to proof the
relation between simplicial and singular homology but it is not useful
for computations because it is much bigger than $C_k(K;R)$.

We first construct an augmentation-preserving chain map
$\F = \{ f_k\}_{k\in \NN}$ between the chain complex
$\tilde{\C} = \{ \tilde{C}_k(K;R), \tilde{\partial}_k \}_{k\in \NN}$ and the
chain complex $\SS = \{ S_k([K];R), \partial_k \}_{k\in \NN}$.
Recall that $\displaystyle [K] = \bigcup_{(s) \in K} (s)$.
Given an ordered $k$-simplex
$(s) = (\VEC{y}_0,\VEC{y}_1, \ldots,\VEC{y}_k)$,
we set $F\big((\VEC{y}_0,\VEC{y}_1, \ldots,\VEC{y}_k)\big)
= \delta_{(s)}$, where $\delta_{(s)}:\Delta_k \to [K]$
is the affine map defined by the conditions 
$\delta_{(s)}(\VEC{e}_i) = \VEC{y}_i$ for
$0 \leq i \leq k$.  We extend $F$ to $\tilde{C}_k(K;R)$ by linearity;
namely,
\[
F\Big(\sum_{\substack{(s) \text{ an ordered }\\
k\text{-simplex}}} a_{(s)} (s)\Big) = \sum_{\substack{(s) \text{ an ordered }\\
k\text{-simplex}}} a_{(s)} F\big((s)\big) \ ,
\]
where $a_{(s)} \in R$ and the sum is finite.  The chain map
$\F = \{ f_k\}_{k\in \ZZ}$ is defined by $f_k = F$ for all $k$.
We also have that $\F = \{ f_k\}_{k\in \NN}$ is a chain map
between the chain complex
$\{ \tilde{C}_k(L;R), \tilde{\partial}_k \}_{k\in \NN}$ and the
chain complex $\{ S_k([L];R), \partial_k \}_{k\in \NN}$.
Since the chain map $\F$ is augmentation preserving, $\F$ is also a
chain map between the augmented chain complex
$\displaystyle \tilde{\C}^\sharp
= \{ \tilde{C}_k^\sharp (K;R), \tilde{\partial}_k^\sharp \}_{k\in \NN}$
and the augmented chain complex
$\displaystyle \SS^\sharp
= \{ S_k^\sharp([K];R), \partial_k^\sharp \}_{k\in \NN}$.

Using $F$, we may define the following homomorphisms:
{
%\renewcommand{\labelitemi}{$\circ$}
\begin{itemize}
\item
$\tilde{H}_k(F) : \tilde{H}_k(K;R) \to H_k([K];R)$ by
$\tilde{H}_k(F)([c]_K) = [F(c)]_{[K]}$ for all $[c]_K \in
\tilde{H}_k(K;R)$,
\item $\tilde{H}_k(F) : \tilde{H}_k(K,L;R) \to H_k([K],[L];R)$ by
$\tilde{H}_k(F)([c]_{K,L}) = [F(c)]_{[K],[L]}$ for all
$[c]_{K,L} \in \tilde{H}_k(K,L;R)$, and
\item $\displaystyle \tilde{H}_k^\sharp(F) : \tilde{H}_k^\sharp(K;R) \to
H_k^\sharp([K];R)$ by
$\displaystyle \tilde{H}_k^\sharp(F)([c]_K) = [F(c)]_{[K]}$ for all
$\displaystyle [c]_K \in \tilde{H}_k^\sharp(K;R)$.
\end{itemize}
}

\begin{prop} \label{propHkEHkv1}
Let $L$ be a subcomplex of a simplicial complex $K$ and $R$ be an
integral domain. Then
\begin{enumerate}
\item $\tilde{H}_k(F): \tilde{H}_k(K;R) \to H_k([K];R)$ is an isomorphism,
\item $\tilde{H}_k^\sharp(F) : \tilde{H}_k^\sharp(K;R) \to
H_k^\sharp([K];R)$ is an isomorphism, and
\item $\tilde{H}_k(F) : \tilde{H}_k(K,L;R) \to H_k([K],[L];R)$ is an
isomorphism.
\end{enumerate}
\end{prop}

\begin{proof}
It follows from Proposition~\ref{propCsCtfauHsH} that (1) is
equivalent to (2).  So, we only need to prove (2) and (3).

\stage{I} We first assume that $K$ is a finite set.  The proof is by
induction on the number of elements in $K$.

\stage{I.i} If $K$ has only one element, then $K = \{ (\VEC{x}_i) \}$ for some
$\VEC{x}_i\in \RR^n$.  We have that $\tilde{C}_k(K;R)$ is generated by the
ordered $k$-simplex $(s) = (\VEC{x}_i,\VEC{x}_i, \ldots, \VEC{x}_i)$ and
$\delta_{(s)}:\Delta_k \to [K] = \{\VEC{x}_i\}$ is the affine map
defined by $\delta_{(s)}(\VEC{x}) = \VEC{x}_i$ for all
$\VEC{x} \in \Delta_k$.  Therefore $F:\tilde{C}_k(K;R) \to S_k([K];R)$ is
given by $F(r (s)) = r \delta_{s}$ for all $r \in R$.  Hence $F$ is
an isomorphism.  It follows that (2) is true.  The statement (3) is
obviously true because the only non-trivial relative homology modules
are $\tilde{H}_k(K,\emptyset;R) = \tilde{H}_k(K;R)$ and
$H_k([K],\emptyset;R) = H_k([K];R)$.

\stage{I.ii} We assume that (2) and (3) are true if $K$ has less
than $N$ elements and prove that it is also true if $K$ has $N$ elements.

It suffices to prove that
$\displaystyle \tilde{H}_k^\sharp(F): \tilde{H}_k^\sharp(K;R) \to
H_k^\sharp([K];R)$ is an isomorphism.
Suppose that this is true.  We get the following commuting diagram from
Proposition~\ref{propConnectingH} \footnote{There are versions of
Proposition~\ref{propConnectingH} for the reduced homology and the
ordered simplicial homology.}
\[
\xymatrix@C+2em{
\ar[r]^-{C_{k+1}} & \tilde{H}_k(L;R) \ar[r]^{\tilde{H}_k(\iota)}
\ar[d]^{\tilde{H}_k(F)} & \tilde{H}_k(K;R)
\ar[r]^{\tilde{H}_k(\Id_X)}  \ar[d]^{\tilde{H}_k(F)}
& \tilde{H}_k(K,L;R) \ar[d]^{\tilde{H}_k(F)} \\
\ar[r]^-{C_{k+1}} & H_k([L];R) \ar[r]^{H_k(\iota)}
& H_k([K];R) \ar[r]^{H_k(\Id_X)}  & H_k([K],[L];R) \\
\ar[r]^-{C_k} & \tilde{H}_{k-1}(L;R)
\ar[r]^{\tilde{H}_{k-1}(\iota)} \ar[d]^{\tilde{H}_{k-1}(F)}
& \tilde{H}_{k-1}(K;R) \ar[r]^-{\tilde{H}_{k-1}(\Id_X)}
\ar[d]^{\tilde{H}_{k-1}(F)} & \\
\ar[r]^-{C_k} & H_{k-1}([L];R) \ar[r]^{H_{k-1}(\iota)}
& H_{k-1}([K];R) \ar[r]^-{H_{k-1}(\Id_X)} &
}
\]
where each row is an exact sequence.
Since $\displaystyle \tilde{H}_k^\sharp(F): \tilde{H}_k^\sharp(L;R) \to
H_k^\sharp([L];R)$ is an isomorphism because $L$ has less than $N$
elements, and
$\displaystyle \tilde{H}_k^\sharp(F): \tilde{H}_k^\sharp(K;R) \to
H_k^\sharp([K];R)$ is an isomorphism by assumption,
we get from Proposition~\ref{propCsCtfauHsH} that
$\tilde{H}_k(F): \tilde{H}_k(L;R) \to H_k([L];R)$ and
$\tilde{H}_k(F): \tilde{H}_k(K;R) \to H_k([K];R)$
are isomorphism.  Hence, we get from the Five Lemma that
$\tilde{H}_k(F) : \tilde{H}_k(K,L;R) \to H_k([K],[L];R)$ 
is an isomorphism.

We now prove that
$\displaystyle \tilde{H}_k^\sharp(F): \tilde{H}_k^\sharp(K;R) \to
H_k^\sharp([K];R)$ is an isomorphism.
Suppose that $(s) \in K$ is a simplex which is not
the face of any other simplex but itself.  Then
$K_1 = K \setminus \{s\}$ is simplicial subcomplex of $K$ and $K_1$
has $N-1$ elements.  Let $K_2$ be the simplicial complex which is the
collection of $(s)$ and all its faces, and $K_3 = K_2 \setminus \{(s)\}$
(Figure~\ref{HkEHkfig}).

\pdfF{alg_top/hkehk}{Isomorphism between singular and simplicial homology}
{Figure associated to the proof of Proposition~\ref{propHkEHkv1}.
The set $[K_1]$ is the region in grey including the continuous black lines.
$[K_2]$ is the region in blue including the dashed blue lines forming
the boundary.  $[K_3]$ is the union of the dashed blue lines.  $V_1$
is the region in pale green excluding the dashed green lines forming
the boundary.  Finally, $V$ is the region in darker green excluding
the dashed green lines forming the boundary.  The arrows in black
describe the retraction $r$ of $[K_2] \setminus V$ onto $[K_3]$.}
{HkEHkfig}

We get the following commutative diagram.
\[
\xymatrix@C+2em{
\tilde{H}_k^\sharp(K;R) \ar[r]^{\tilde{H}_k^\sharp(F)}
\ar[d]_{\tilde{H}_k(\Id)} & H_k^\sharp([K];R) \ar[d]^{H_k(\Id)} \\
\tilde{H}_k(K,K_2;R) \ar[r]^-{\tilde{H}_k(F)} & H_k([K],[K_2];R) \\
\tilde{H}_k(K_1;K_3;R) \ar[r]^-{\tilde{H}_k(F)} \ar[u]^{\tilde{H}_k(\iota)}
& H_k([K_1],[K_3];R) \ar[u]_{H_k(\iota)}
}
\]
where $\iota$ denotes an inclusion map.  We have by
induction that
$\tilde{H}_k(F): \tilde{H}_k(K_1;K_3;R) \to H_k([K_1],[K_3];R)$
is an isomorphism.  To prove that
$\tilde{H}_k^\sharp(F): \tilde{H}_k^\sharp(K;R) \to H_k^\sharp([K];R)$
is an isomorphism, we prove
that all the maps $\tilde{H}_k(\iota)$, $H_k(\iota)$,
$\tilde{H}_k(\Id)$ and $H_k(\Id)$ are isomorphism.

Since $[K_2] = [s]$ is contractible, we get from
Proposition~\ref{propContrXeXA} that
$\displaystyle H_k(\Id):H_k^\sharp([K];R) \to H_k^\sharp([K],[K_2];R)$
is an isomorphism for $k \geq 0$.  Since 
$\displaystyle H_k^\sharp([K],[K_2];R) = H_k([K],[K_2];R)$ for $k>0$,
we get that
$\displaystyle H_k(\Id):H_k^\sharp([K];R) \to H_k([K],[K_2];R)$
is an isomorphism for $k>0$.  For $k=0$, we assume that
$[K]$ is the disjoint union of path-connected components 
$X_j$ for $j \in J$.  There exists $j_0 \in J$ such that
$(s) \subset X_{j_0}$.  We have from Propositions~\ref{propXAopXjAj} 
and \ref{propH0XA0} that
\begin{align*}
H_0([K],[K_2];R)
&= \Big( \bigoplus_{j \in J \setminus \{j_0\}} H_0(X_j;R) \Big) \oplus 
H_0(X_{j_0},[K_2];R) \\
&= \Big( \bigoplus_{j \in J \setminus \{j_0\}} H_0(X_j;R) \Big) \oplus 0 \ .
\end{align*}
We get from Propositions~\ref{propHkeHkJ} and \ref{propHkEquHskpR}
that
\[
\Big( \bigoplus_{j \in J \setminus \{j_0\}} H_0(X_j;R) \Big) \oplus 
H_0(X_{j_0};R) = H_0([K];R) \cong H_0^\sharp([K];R) \oplus R \ .
\]
Since $H_0(X_j;R) \cong R$ for all $j$ according to
Corollary~\ref{corContrHk}, we get that
$\displaystyle H_0([K],[K_2];R)
= \bigoplus_{j \in J \setminus \{j_0\}} H_0(X_j;R)
\cong H_0^\sharp([K];R)$.  Since the previous isomorphism is deduced
from the identity map, we again have that
$\displaystyle H_0(\Id):H_0^\sharp([K];R) \to H_0([K],[K_2];R)$
is an isomorphism.  The same reasoning shows that
$\displaystyle \tilde{H}_k(\Id):\tilde{H}_k^\sharp(K;R) \to
\tilde{H}_k(K,K_2;R)$ is an isomorphism for $k \geq 0$
\footnote{The courageous reader is invited to repeat for
relative and reduced simplicial homology what we have done for relative
and reduced singular homology.}.

Consider the inclusion map
$\iota : \tilde{C}_k(K_1;R)/\tilde{C}_k(K_3;R) \to
\tilde{C}_k(K;R)/\tilde{C}_k(K_2;R)$.  The map $\iota$ is an isomorphism
because $K_1 \setminus K_3 = K \setminus K_2$.  More
precisely, if\\
$\relC[K,K_2]{c} \in \tilde{C}_k(K;R)/\tilde{C}_k(K_2;R)$, then
$c - b \in \tilde{C}_k(K_2;R)$ for some $b \in \tilde{C}_k(K_1;R)$ because
$(s) \in K_2$.  Hence $\iota(\relC[K_1,K_3]{b}) = \relC[K,K_2]{c}$ and $\iota$
is onto.  If $\relC[K,K_2]{c_1} = \iota(\relC[K_1,K_3]{c_1})
= \iota(\relC[K_1,K_3]{c_2})= \relC[K,K_2]{c_2}$ with
$c_1,c_2 \in \tilde{C}_k(K_1;R)$, then
$c_1 - c_2 = b \in \tilde{C}_k(K_2;R) \cap \tilde{C}_k(K_1;R)$.
Thus $c_1 - c_2 = b \in \tilde{C}_k(K_3;R)$ and therefore
$\relC[K_1,K_3]{c_1}= \relC[K_1,K_3]{c_2}$.
This implies that
$\tilde{H}_k(\iota) : \tilde{H}_k(K_1,K_3;R) \to \tilde{H}_k(K,K_2;R)$
is also an isomorphism.

The prove that $H_k(\iota) : H_k([K_1],[K_3];R) \to H_k([K],[K_2];R)$
is an isomorphism is not as simple as the proof that
$\tilde{H}_k(\iota)$ is an isomorphism given above.  We plan to use excision to
prove that $H_k(\iota) : H_k([K_1],[K_3];R) \to H_k([K],[K_2];R)$
is an isomorphism.  Unfortunately, the excision theorem,
Theorem~\ref{thmExcis}, cannot be used directly because we have the
situation where the set that we want to excise, namely $(s)$, is such
that $\overline{(s)} = [s]$ is not a subset of
$\displaystyle [K_2]^\circ = (s)$,
where closure and interior of sets are relative to the induce topology
on $[K]$ from $\displaystyle \RR^n$.
We therefore need to use Proposition~\ref{propVUAX}.

Suppose that $(s)$ is a $k$-simplex and that $(s_0)$, $(s_1)$, \ldots,
$(s_k)$ are the faces of $(s)$ which are $(k-1)$-simplices.
Let $V_1$ be the $k$-simplex
$(\VEC{b}_{(s_0)},\VEC{b}_{(s_1)}, \ldots, \VEC{b}_{(s_k)})$ where
$\VEC{b}_{(s_j)}$ is the barycentre of $(s_j)$ for $0 \leq j \leq k$.
Let $\VEC{p}$ be the barycentre of $V_1$. 
Referring to the statement of Proposition~\ref{propVUAX}, the set
$V \subset (s)$ that we consider is given by
\[
V = \VEC{p} + \epsilon (V_1 - \VEC{p}) =
\{ \VEC{p} + \epsilon (\VEC{x} - \VEC{p}) : \VEC{x} \in V_1 \}
\]
for $0 < \epsilon <1$ fixed (Figure~\ref{HkEHkfig}).

We have that $\overline{V} \subset (s)^\circ = (s)$.
There is a retraction $r$ of $[K_2] \setminus V$ onto $[K_3]$
(Figure~\ref{HkEHkfig}).  By defining $r$ to be the identity on
$[K_1]\setminus [K_2]$, we get a deformation retraction of
$([K] \setminus V, [K_2] \setminus V)$ onto
$([K_1],[K_3]) = ([K]\setminus (s),[K_2] \setminus (s))$.  We may
therefore use Proposition~\ref{propVUAX} to conclude that
$H_k(\iota):H_k([K_1],[K_3];R) \to H_k([K] \setminus V, [K_2] \setminus V;R)$
is an isomorphism.

Moreover, since $\overline{V} \subset (s)^\circ = (s)$, we may use the
excision theorem, Theorem~\ref{thmExcis}, to conclude that
$H_k(\iota):H_k([K] \setminus V, [K_2] \setminus V;R) \to H_k([K], [K_2];R)$.
is an isomorphism.  It follows from the previous paragraph that
$H_k(\iota) : H_k([K_1],[K_3];R) \to H_k([K],[K_2];R)$
is an isomorphism.

\stage{II} The fact that we have proved the proposition when $K$ has
any finite number of elements does not automatically implies that the
proposition is true when $K$ has infinitely many elements.  We do not
have a concept of limit for the spaces $C_k(K;R)$, $S_k(K;R)$, and so on,
as the size of $K$ increases.  We now assume that $K$ has countably
infinitely many elements.

It still follows from Proposition~\ref{propCsCtfauHsH} that (1) is
equivalent to (2).  Moreover, if we prove that
$\displaystyle \tilde{H}_k^\sharp(F): \tilde{H}_k^\sharp(K;R) \to
H_k^\sharp([K];R)$ is an isomorphism for all $k$, then we can use long
exact sequences and the Five Lemma exactly as we have done in (I.ii) to
show that this implies that (3) is true for all $k$.

Thus, we only have to prove that
$\displaystyle \tilde{H}_k^\sharp(F): \tilde{H}_k^\sharp(K;R) \to
H_k^\sharp([K];R)$ where $K$ has infinitely many elements.

\stage{II.i}  Suppose that $c \in S_k([K];R)$.  We may express $c$ as a
finite sum $\displaystyle c = \sum_{j\in J} a_j \sigma_j$, where
$a_j \in R$ and $\sigma_j$ is a singular $k$-simplex in $[K]$ for all $j$.
We prove that there exists a simplicial
subcomplex $L_c$ of $K$ such that $L_c$ has a finite number of elements and
$\displaystyle Q = \bigcup_{j\in J} \sigma_j(\Delta_k) \subset [L_c]$.
Since the $\sigma_j$ are continuous functions and $\Delta_k$ is
compact, we have that $\sigma_j(\Delta_k)$ is a compact set.
Thus $Q$ is a compact set because it is the finite union of compact
sets.  Suppose that there is no finite number of simplices
in $K$ that cover $Q$.  Then there exists an
infinite collection $\displaystyle \{ (s_i) \}_{i\in \NN}$ of distinct
simplices in $K$ such that $(s_i) \cap Q \neq \emptyset$
for all $i \in \NN$ \footnote{These simplices are obtained inductively using
the fact that $\displaystyle Q \subset [K] = \bigcup_{(s) \in K} (s)$.}.
Choose $\VEC{z}_i \in (s_i) \cap Q$ for each $i \in \NN$.
Let $\displaystyle U_m = Q \setminus \bigcup_{i\neq m} \{ \VEC{z}_i \}$
for all $m \in \NN$.  The sets $U_m$ are open sets (see Remark~\ref{rmkUmOpen}
below) and $\displaystyle Q = \bigcup_{m \in \NN} U_m$.  Thus
$\{ U_m\}_{m\in \NN}$ is an open cover of the compact set
$Q$.  Therefore, there exists a finite subcover
$\{ U_{m_t}\}_{0\leq t \leq T}$ of $Q$.  But
$\displaystyle Q \varsupsetneqq \bigcup_{0\leq t\leq T} U_{m_t}$ because
$\VEC{z}_i \not\in \bigcup_{0\leq t \leq T} U_{m_t}$ if $i \neq m_t$ for
$0 \leq t\leq T$.  This is a contradiction.   Hence, there are a
finite number of simplices in $K$ whose union cover $Q$.
If we add all the faces of these simplices, we get a finite simplicial
subcomplex $L_c$ of $K$ with $Q \subset [L_c]$.

\stage{II.ii}  We first prove that
$\displaystyle \tilde{H}_k^\sharp(F): \tilde{H}_k^\sharp(K;R) \to
H_k^\sharp([K];R)$ is onto.  Suppose that
$\displaystyle [c]_K \in H_k^\sharp([K];R)$ and that
$\displaystyle c = \sum_{j\in J} a_j \sigma_j$, where
$a_j \in R$ and $\sigma_j$ is a singular $k$-simplex in $[K]$ for all
$j \in J$.
Using (II.i), we get a finite simplicial subcomplex $L_c$ of $K$ such that
$\displaystyle c \in S_k^\sharp([L_c];R)$ and thus
$\displaystyle [c]_{L_c} \in H_k^\sharp([L_c];R)$.
we consider the commutative diagram
\[
\xymatrix@C+2em{
\tilde{H}_k^\sharp(L_c;R) \ar[r]^-{\tilde{H}_k^\sharp(F)}
\ar[d]_{\tilde{H}_k(\iota)}
& H_k^\sharp([L_c];R) \ar[d]^{H_k(\iota)} \\
\tilde{H}_k^\sharp(K;R) \ar[r]^-{\tilde{H}_k^\sharp(F)}
& H_k^\sharp([K];R)
}
\]
We have from (I) that
$\displaystyle \tilde{H}_k^\sharp(F) : \tilde{H}_k^\sharp(L_c;R) \to
H_k^\sharp([L_c];R)$ is an isomorphism.  Thus, there
exists $\displaystyle [b]_{L_c} \in \tilde{H}_k^\sharp(L_c;R)$ such that
$\displaystyle \tilde{H}_k^\sharp(F)([b]_{L_c}) = [c]_{[L_c]}$.  we
therefore get that\\
$\displaystyle \tilde{H}_k^\sharp(F)\big(\tilde{H}_k(\iota)
([b]_{L_c})\big) = H_k(\iota)\big( \tilde{H}_k^\sharp(F)([b]_{L_c}) \big)
= [c]_{[K]}$.  Hence $[c]_{[K]}$ is in the image of
$\displaystyle \tilde{H}_k^\sharp(F): \tilde{H}_k^\sharp(K;R) \to
H_k^\sharp([K];R)$.

To prove that $\displaystyle \tilde{H}_k^\sharp(F): \tilde{H}_k^\sharp(K;R) \to
H_k^\sharp([K];R)$ is one-to-one, suppose that
$\displaystyle \tilde{H}_k^\sharp(F)([c]_K) = [F(c)]_{[K]} = [0]_{[K]}$ for some
$\displaystyle [c]_K \in \tilde{H}_k^\sharp(K;R)$.  This means that
$\displaystyle F(c) = \partial^\sharp(b)$ for some $b \in S_{k+1}([K];R)$.
Using (II.i), we get a finite simplicial subcomplex $L_c$ of $K$ such that
$\displaystyle F(c) \in S_k^\sharp([L_c];R)$ and a finite simplicial
subcomplex $L_b$ of $K$ such that
$\displaystyle b \in S_{k+1}^\sharp([L_b];R)$.
Let $L = L_c \cup L_b$.  We have that $L$ is a finite simplicial
subcomplex of $K$ with $\displaystyle F(c) \in S_k^\sharp([L];R)$ and
$\displaystyle b \in S_{k+1}([L];R)$.
We also have that $\displaystyle c \in \tilde{C}_k^\sharp(L;R)$
because $c$ is a finite linear combinations of ordered $k$-simplices
which are included in $L$ by construction \footnote{It follows from
the definition of $F$ that If $(s)$ is one of the ordered
$k$-simplices involved in the linear combination for $c$, then $(s)$
must comes from one of the initial simplices that cover
$Q$ in (II.i).  Recall that if two open simplices
intersect, then they must be equal.}.
We consider the commutative diagram
\[
\xymatrix@C+2em{
\tilde{H}_k^\sharp(L;R) \ar[r]^-{\tilde{H}_k^\sharp(F)}
\ar[d]_{\tilde{H}_k(\iota)}
& H_k^\sharp([L];R) \ar[d]^{H_k(\iota)} \\
\tilde{H}_k^\sharp(K;R) \ar[r]^-{\tilde{H}_k^\sharp(F)}
& H_k^\sharp([K];R)
}
\]
Since $\displaystyle F(c) = \partial^\sharp(b) \in S_k^\sharp([L];R)$
with $\displaystyle b \in S_{k+1}^\sharp([L];R)$, we have that
$\displaystyle [F(c)]_{[L]}= [0]_{[L]} \in \tilde{H}_k^\sharp([L];R)$.
Since $\displaystyle \tilde{H}_k^\sharp(F) : \tilde{H}_k^\sharp(L;R) \to
H_k^\sharp([L];R)$ is an isomorphism according to (I), we have that
$\displaystyle [c]_L = [0]_L \in \tilde{H}_k^\sharp(L;R)$.  Thus
$\displaystyle [c]_K = \tilde{H}(\iota)([c]_L) = [0]_K \in 
\tilde{H}_k^\sharp(K;R)$.
\end{proof}

HERE

\begin{rmk}
We prove that the sets $U_m$ in part (II.i) of          \label{rmkUmOpen}
the previous proof are open sets.  By adding the missing faces if
needed to $\displaystyle \{ (s_i) \}_{i\in \NN}$, we get a
a simplicial subcomplex $L$ of $K$.  Consider
$\VEC{z} \in U_m$.  If $\VEC{z} \in U_m \setminus [L]$, 
then $V = Q \setminus [L]$ can be used as an open neighbourhood of
$\VEC{z}$ such that $V \subset U_m$.  If
$\displaystyle \VEC{z} \in [L] = \bigcup_{(s) \in L} (s)$, then
there exists a unique $(s) \in L$ such that $\VEC{z} \in (s)$.
The finite union $\displaystyle W =
\bigcup_{\substack{(t) \in K\\ [t]\cap [s] \neq \emptyset}} (t)$ 
is an open neighbourhood of $\VEC{z}$ in the induce topology on $[K]$.
Moreover, $W$ contains only a finite number of $\VEC{z}_i$ with
$i\neq m$ because the open simplices are distinct and we have only one
$\VEC{z}_i$ per $(s_i)$; We are referring to
the $\VEC{z}_i$ with $i \neq m$ that may be in one of the $(t)$ listed
in the union.  We do not want to reject $\VEC{z}_m$ if it is in $W$;
in particular if $\VEC{z} = \VEC{z}_m$.
If $\VEC{z}_{m_s}$ for $m_s \neq m$ and $0\leq s \leq S$ are the
$\VEC{z}_i \in W$, then $\displaystyle Q \cap \big( W \setminus
\{ \VEC{z}_{m_s} : 0\leq s \leq S\}\big) \subset U_m$ is an open
neighbourhood of $\VEC{z}$.  
\end{rmk}

\subsubsection{Definition of the Isomorphism}

We really want a version of the previous proposition for oriented
simplicial homology.  Instead of ordered simplices, we want to work
with oriented simplices.  Instead of the map $F$ defined previously,
we would like to use the following map.  Given an oriented $k$-simplex
$\os{s}{}{}{}{} = \os{\VEC{x}_{i_0}}{}{\VEC{x}_{i_1}}{}{\VEC{x}_{i_k}}$,
we set $G(\os{s}{}{}{}{}) = \delta_{\osscript{s}{}{}{}{}}$ where as before
$\delta_{\osscript{s}{}{}{}{}}:\Delta_k \to [K]$
is the affine map defined by
$\delta_{\osscript{s}{}{}{}{}}(\VEC{e}_j) = \VEC{x}_{i_j}$ for
$0 \leq j \leq k$.  We extend $G$ to $C_k(K;R)$ by linearity;
namely,
$\displaystyle G\Big(\sum_{\substack{(s) \in K\\\dim(s)=k}} a_{(s)}
\os{s}{}{}{}{}\Big) = \sum_{\substack{(s) \in K\\\dim(s)=k}} a_{(s)}
G(\os{s}{}{}{}{})$ where $a_{(s)} \in R$ and the sum is finite.
The rest of this subsection is devoted to justify how we may
``replace'' $F$ by $G$.

Let $\displaystyle \phi:C_k(K;R) \to \tilde{C}_k(K;R)$
be the function defined by \\
$\phi(\os{\VEC{x}_{j_0}}{}{\VEC{x}_{j_1}}{}{\VEC{x}_{j_k}})
= (\VEC{x}_{j_0},\VEC{x}_{j_1}, \ldots, \VEC{x}_{j_k})$ for all 
oriented $k$-simplices with $j_0 < j_1 < \ldots < j_k$ and
extended to $\displaystyle C_k(K;R)$ by linearly,
and $\displaystyle \psi:\tilde{C}_k(K;R) \to C_k(K;R)$
be the function defined by
\[
\psi\big((\VEC{y}_0,\VEC{y}_1,\ldots,\VEC{y}_k)\big)
= \begin{cases}
\os{\VEC{y}_0}{}{\VEC{y}_1}{}{\VEC{y}_k}
& \quad \text{if} \ (\VEC{y}_0,\VEC{y}_1,\ldots,\VEC{y}_k) \in K \\
0 & \quad \text{otherwise}
\end{cases}
\]
for all ordered $k$-simplices and extended to
$\displaystyle \tilde{C}_k(K;R)$ by linearly.
We note that $\psi((s)) = 0$ when $(s)$ has at least two equal vertices.
It is clear that $\tilde{\partial}_k \circ \phi = \phi \circ \partial_k$.
It is also true that
$\psi \circ \tilde{\partial}_k = \partial_k \circ \psi$ but the
proof is not completely obvious.  If $(s) \in \tilde{C}_k(K;R)$ does
not have any equal vertices, then the result is obvious.  If
$(s) = \big(\VEC{y}_0,\VEC{y}_1,\ldots,\VEC{y}_k\big)$ has two equal
vertices, say $\VEC{y}_m = \VEC{y}_{m+1}$ for some $0 \leq m < k$,
then $\partial_k \big(\psi\big((s)\big)\big) = \partial_K(0) = 0$ and
\begin{align*}
\psi\big(\tilde{\partial}_k(s)\big)
&= \sum_{i=0}^k (-1)^i \psi\big(
(\VEC{y}_0,\VEC{y}_1,\ldots,\widehat{\VEC{y}_i}, \ldots,
\VEC{y}_k)\big) \\
&= (-1)^m \psi\big(
(\VEC{y}_0,\VEC{y}_1,\ldots,\widehat{\VEC{y}_m}, \ldots, \VEC{y}_k)\big)
+ (-1)^{m+1} \psi\big(
(\VEC{y}_0,\VEC{y}_1,\ldots,\widehat{\VEC{y}_{m+1}}, \ldots, \VEC{y}_k)\big)
= 0
\end{align*}
because $(\VEC{y}_0,\VEC{y}_1,\ldots,\widehat{\VEC{y}_m}, \ldots, \VEC{y}_k)
= (\VEC{y}_0,\VEC{y}_1,\ldots,\widehat{\VEC{y}_{m+1}}, \ldots, \VEC{y}_k)$.
If $(s) = \big(\VEC{y}_0,\VEC{y}_1,\ldots,\VEC{y}_k\big)$ has more
than two equal vertices, then
$\partial_k \big(\psi\big((s)\big)\big) = \partial_K(0) = 0$ and
\[
\psi\big(\tilde{\partial}_k(s)\big)
= \sum_{i=0}^k (-1)^i \psi\big(
(\VEC{y}_0,\VEC{y}_1,\ldots,\widehat{\VEC{y}_i}, \ldots,
\VEC{y}_k)\big)
= 0
\]
because each $(\VEC{y}_0,\VEC{y}_1,\ldots,\widehat{\VEC{y}_i}, \ldots,
\VEC{y}_k)$ has at last two equal vertices.

We have the following commutative diagram
\[
\xymatrix{
\ar[r]^(0.2){\partial_{k+2}}
& C_{k+1}(K;R)  \ar[r]^-{\partial_{k+1}} \ar@/^/[d]^-{\phi}
& C_k(K;R) \ar[r]^-{\partial_k} \ar@/^/[d]^-{\phi}
& C_{k-1}(K;R) \ar[r]^(0.7){\partial_{k-1}} \ar@/^/[d]^-{\phi} & \\
\ar[r]_(0.2){\tilde{\partial}_{k+2}}
& \tilde{C}_{k+1}(K;R) \ar[r]_-{\tilde{\partial}_{k+1}}
\ar@/^/[u]^-{\psi}
& \tilde{C}_k(K;R) \ar[r]_-{\tilde{\partial}_k} \ar@/^/[u]^-{\psi}
& \tilde{C}_{k-1}(K;R) \ar[r]_(0.7){\tilde{\partial}_{k-1}}
\ar@/^/[u]^-{\psi} &
}
\]
Namely, $\{\phi_k\}_{k\in \ZZ}$ with $\phi_k = \phi$ for all $k$ is a
chain map from the chain complex
$\displaystyle \C = \{(C_k(K;R),\partial_k)\}_{k\in \ZZ}$
to the chain complex
$\displaystyle \tilde{\C}
= \{(\tilde{C}_k(K;R),\tilde{\partial}_k)\}_{k\in \ZZ}$ and
$\{\psi_k\}_{k\in \ZZ}$ with $\psi_k = \psi$ for all $k$ is a
chain map from $\displaystyle \tilde{\C}$ to $\displaystyle \C$.

Since
$\displaystyle \phi: \KE(\partial_k) \to \KE(\tilde{\partial}_k)$
and
$\displaystyle \phi: \IMG(\partial_{k-1}) \to
\IMG(\tilde{\partial}_{k-1})$, we can
define the homomorphism
$\displaystyle \tilde{H}_k(\phi):H_k(K;R) \to \tilde{H}_k(K;R)$
by
$\displaystyle \tilde{H}_k(\phi)([c]_K) = [\phi(c)]_K$ for all
$\displaystyle c \in C_k(K;R)$,
where as usual $[\cdot]_K$ represents equivalence classes in both
$\displaystyle H_k(K;R)$ and
$\displaystyle \tilde{H}_k(K;R)$.
Similarly, since
$\displaystyle \psi:\KE(\tilde{\partial}_k) \to \KE(\partial_k)$
and
$\displaystyle \psi:\IMG(\tilde{\partial}_{k-1}) \to
\IMG(\partial_{k-1})$, we may
define the homomorphism
$\displaystyle \tilde{H}_k(\psi):\tilde{H}_k(K;R) \to
H_k(K;R)$ by
$\displaystyle \tilde{H}_k(\psi)([c]_K) = [\psi(c)]_K$ for all
$\displaystyle c \in \tilde{C}_k(K;R)$.

We can repeat all the previous discussion for the reduced
homology (i.e.\ with $\partial_k$ replaced by
$\displaystyle \partial_k^\sharp$ and $\tilde{\partial}_k$
replaced by $\displaystyle \tilde{\partial}_k^\sharp$) and
define
$\displaystyle \tilde{H}_k^\sharp(\phi):H_k^\sharp(K;R) \to
\tilde{H}_k^\sharp(K;R)$ and
$\displaystyle \tilde{H}_k^\sharp(\psi):\tilde{H}_k^\sharp(K;R) \to
H_k^\sharp(K;R)$.  We assume that
$\displaystyle \tilde{C}_{-1}^\sharp(K;R) = R$ and
$\displaystyle C_{-1}^\sharp(K;R) = R$, and that
$\displaystyle \phi:C_{-1}^\sharp(K;R) \to \tilde{C}_{-1}^\sharp(K;R)$
and
$\displaystyle \psi:\tilde{C}_{-1}^\sharp(K;R) \to C_{-1}^\sharp(K;R)$
are the identity maps.

According to Proposition~\ref{propCsCtfauHsH}, if we prove that
$\displaystyle \tilde{H}_k^\sharp(\phi)$ is an isomorphism, then 
we will have that $\displaystyle \tilde{H}_k(\phi)$ is an isomorphism.
To prove that $\displaystyle \tilde{H}_k^\sharp(\phi)$ is an
isomorphism, we prove that $\displaystyle \tilde{H}_k^\sharp(\psi)$ is
the inverse of $\displaystyle \tilde{H}_k^\sharp(\phi)$.

It is clear that $\psi \circ \phi = \Id$ on
$\displaystyle C_k^\sharp(K;R)$.  Thus
$\displaystyle \tilde{H}_k^\sharp(\psi) \circ \tilde{H}_k^\sharp(\phi) = \Id$
on $\displaystyle H_k^\sharp(K;R)$.
To prove that
$\displaystyle \tilde{H}_k^\sharp(\phi) \circ \tilde{H}_k^\sharp(\psi) = \Id$
requires a little bit more work because $\phi \circ \psi$ is not the identity.
We need to prove that there exists a chain homotopy between
$\F = \{ f_k\}_{k\in \ZZ}$ with
$\displaystyle f_k = \Id:\tilde{C}_k^\sharp(K;R) \to \tilde{C}_k^\sharp(K;R)$
and
$\GG = \{ g_k\}_{k\in \ZZ}$ with
$\displaystyle g_k = \phi\circ \psi :\tilde{C}_k^\sharp(K;R) \to
\tilde{C}_k^\sharp(K;R)$.
Namely, we need to find $\DD = \{ D_k \}_{k\in \ZZ}$ with
$\displaystyle D_k : \tilde{C}_k^\sharp(K;R) \to \tilde{C}_{k+1}^\sharp(K;R)$
such that
$\displaystyle \tilde{\partial}_{k+1}^\sharp \circ D_k +
D_{k-1} \circ \tilde{\partial}_k^\sharp =
\phi\circ \psi - \Id$ on $\tilde{C}_k^\sharp(K;R)$.
After this is done, we will be able to use
Proposition~\ref{propCHiFeG} \footnote{To be precise, a version of
this proposition for ordered simplicial homology.}
to conclude that
$\displaystyle \tilde{H}_k^\sharp(\Id) = \tilde{H}_k^\sharp(\phi\circ \psi)$
for all $k$; namely,
$\displaystyle \tilde{H}_k^\sharp(\phi) \circ \tilde{H}_k^\sharp(\psi)
= \Id:\tilde{H}_k^\sharp(K;R) \to \tilde{H}_k^\sharp(K;R)$.

\begin{prop}  \label{propOOhomCM}
There exists $\DD = \{ D_k \}_{k\in \ZZ}$ with
$\displaystyle D_k : \tilde{C}_k^\sharp(K;R) \to \tilde{C}_{k+1}^\sharp(K;R)$
such that
$\displaystyle \tilde{\partial}_{k+1}^\sharp \circ D_k
+ D_{k-1} \circ \tilde{\partial}_k^\sharp =
\phi\circ \psi - \Id$ on $\displaystyle \tilde{C}_k^\sharp(K;R)$.
\end{prop}

\begin{proof}
\stage{i} Suppose that $(\VEC{p},[K])$ is an general position.  We
consider the cone $\VEC{p} \ast K$.  Recall that it is a simplicial
complex consisting of all the simplices of the form
$(\VEC{p},\VEC{x}_{i_0}, \VEC{x}_{i_1}, \ldots , \VEC{x}_{i_k})$ and
all their faces where
$(\VEC{x}_{i_0}, \VEC{x}_{i_1}, \ldots , \VEC{x}_{i_k}) \in K$.
Hence $K \subset \VEC{p} \ast K$.

We prove that the chain complex
$\displaystyle \{(\tilde{C}_k^\sharp(\VEC{p}\ast K;R),
\tilde{\partial}_k^\sharp)\}_{k\in \ZZ}$ is
acyclic.  Let
$\displaystyle Q_k: \tilde{C}_k^\sharp(\VEC{p}\ast K;R) \to
\tilde{C}_{k+1}^\sharp(\VEC{p}\ast K;R)$ be
the mapping define by 
$Q_k\big( (\VEC{y}_0, \VEC{y}_1, \ldots, \VEC{y}_k) \big)
=(\VEC{p}, \VEC{y}_0, \VEC{y}_1, \ldots, \VEC{y}_k)$
for all $(\VEC{y}_0, \VEC{y}_1, \ldots, \VEC{y}_k) \in \VEC{p} \ast K$, and 
extended by linearity to $\displaystyle \tilde{C}_k^\sharp(\VEC{p}\ast K;R)$.

Proceeding as we did in Section~\ref{subsectExcis}, we find that
$\displaystyle \tilde{\partial}_1^\sharp(Q_0(c))
= c - \tilde{\partial}_0^\sharp(c) (\VEC{p})$ for
all $\displaystyle c \in \tilde{C}_0^\sharp(\VEC{p} \ast K;R)$ and
$\displaystyle \tilde{\partial}_{k+1}^\sharp(Q_k(c))
= c - Q_{k-1}(\tilde{\partial}_k^\sharp(c))$
for all $\displaystyle c \in \tilde{C}_k^\sharp(\VEC{p} \ast K;R)$ if $k>0$.

If $\displaystyle c \in \tilde{Z}_0^\sharp(\VEC{p} \ast K;R)$, then
$\displaystyle \tilde{\partial}_0^\sharp(c) =0$.
Therefore $\displaystyle c = \tilde{\partial}_1^\sharp(Q_0(c))$.
Hence $\displaystyle c \in \tilde{B}_0^\sharp(\VEC{p} \ast K;R)$.  Thus
$[c]_{\VEC{p}\ast K} = [0]_{\VEC{p}\ast K}$.  We get that
$\displaystyle \tilde{H}_0^\sharp(\VEC{p} \ast K;R) = 0$.

If $\displaystyle c \in \tilde{Z}_k^\sharp(\VEC{p} \ast K;R)$ with $k>0$, then
$\displaystyle \tilde{\partial}_k^\sharp(c) = 0$.  Therefore
$\displaystyle c = \tilde{\partial}_{k+1}^\sharp(Q_k(c))$.
Hence $\displaystyle c \in \tilde{B}_k^\sharp(\VEC{p} \ast K;R)$.
Thus $[c]_{\VEC{p}\ast K} = [0]_{\VEC{p}\ast K}$.  We get that
$\displaystyle \tilde{H}_k^\sharp(\VEC{p} \ast K;R) = 0$.

\stage{ii}  Let $\eta = \phi\circ \psi$.
Given $\sigma = (\VEC{y}_0,\VEC{y}_1,\ldots,\VEC{y}_k) \in \VEC{p} \ast K$,
Let $L_\sigma$ be the smallest subcomplex of $\VEC{p}\ast K$ containing
the vertices $\VEC{y}_0$, $\VEC{y}_1$, \ldots, $\VEC{y}_k$,
$\eta(\VEC{y}_0)$, $\eta(\VEC{y}_1)$, \ldots,$\eta(\VEC{y}_{k-1}$ and
$\eta(\VEC{y}_k)$.
We have that $\displaystyle \tilde{\partial}_k^\sharp(\sigma) \in
\tilde{C}_k^\sharp(L_\sigma;R)$.
Moreover, if $\displaystyle \sigma \in \tilde{C}_k^\sharp(K;R)$,
then $L_\sigma \subset K$ because
$\displaystyle \eta: \tilde{C}_k^\sharp(K;R) \to \tilde{C}_k^\sharp(K;R)$.

We use a proof by induction to construct
$\displaystyle D_k : \tilde{C}_k^\sharp(\VEC{p} \ast K;R) \to
\tilde{C}_{k+1}^\sharp(\VEC{p} \ast K;R)$
such that
$\displaystyle \tilde{\partial}_{k+1}^\sharp \circ D_k
= D_{k-1} \circ \tilde{\partial}_k^\sharp =
\eta - \Id$ on $\displaystyle \tilde{C}_k^\sharp(\VEC{p} \ast K;R)$
and such that
$\displaystyle D_k \big(\tilde{C}_k^\sharp(Q;R)\big) \subset
\tilde{C}_{k+1}^\sharp(Q;R)$ for all simplicial subcomplexes of
$\VEC{p} \ast K$; in particular, for $Q = K$.

\stage{$\mathbf{k=-1}$}
We define $D_{-1} : \tilde{C}_{-1}^\sharp(\VEC{p} \ast K;R) = R \to
\tilde{C}_0^\sharp(\VEC{p} \ast K;R)$ by $D_{-1}(r) = 0$ for $r \in R$.

\stage{$\mathbf{k=0}$}
Consider
$\displaystyle \sigma = (\VEC{y}) \in \tilde{C}_1^\sharp(\VEC{p} \ast K;R)$.
Thus $\VEC{y}$ is a vertex of $\VEC{p} \ast K$.
Since
\[
\partial_0^\sharp \big( \eta(\sigma) - \Id(\sigma) \big)
= \eta(\partial_0^\sharp(\sigma)) - \Id(\partial_0^\sharp(\sigma))
= \eta(1) - \Id(1) = 1 -1 = 0 \ ,
\]
we get that $\displaystyle \eta(\sigma) - \Id(\sigma) \in
\tilde{Z}_0^\sharp(\VEC{p} \ast K;R)$.
Since $\displaystyle \tilde{H}_0^\sharp(\VEC{p} \ast K;R) = 0$
by (i), we have that $\displaystyle \eta(\sigma)
- \Id(\sigma) \in \tilde{B}_0^\sharp(\VEC{p} \ast K;R)$.  Thus,
there exists $\displaystyle b_\sigma \in \tilde{C}_1^\sharp(\VEC{p}\ast K;R)$
such that
$\displaystyle \tilde{\partial}_1^\sharp(b_\sigma) = \eta(\sigma) - \Id(\sigma)$.
We may assume that
$\displaystyle b_\sigma \in \tilde{C}_1^\sharp(L_\sigma;R)$ because
$\displaystyle \eta(\sigma) - \Id(\sigma) \in \tilde{C}_0^\sharp(L_\sigma;R)$.

We set $D_0(\sigma) = b_\sigma$ for all ordered $0$-simplices $\sigma$
of $\VEC{p} \ast K$, and extend $D_0$ by linearity to
$\displaystyle \tilde{C}_0^\sharp(\VEC{p} \ast K;R)$.

By linearity, we have that
\[
\tilde{\partial}_1^\sharp(D_0(c)) + D_{-1}(\tilde{\partial}_0^\sharp(c))
= \tilde{\partial}_1^\sharp(D_0(c)) = \eta(c) - \Id(c)
\]
for all $\displaystyle c \in \tilde{C}_0^\sharp(\VEC{p} \ast K;R)$.
If $\displaystyle c \in \tilde{C}_0^\sharp(Q;R)$ for some simplicial
subcomplex $Q$ of $\VEC{p} \ast K$, then $c$ is the sum (after
simplification) of oriented $0$-simplices $\sigma$ in $Q$.  Since
$D_0(\sigma) \in \tilde{C}_1^\sharp(L_\sigma;R) \subset \tilde{C}_1^\sharp(Q,R)$
for all of these oriented $0$-simplices because $L_\sigma \subset Q$
for each of them, we have that $D_0(c) \in \tilde{C}_1^\sharp(Q,R)$.
In particular, $\displaystyle D_0 \big(\tilde{C}_0^\sharp(K;R)\big) \subset
\tilde{C}_1^\sharp(K;R)$.

\stage{$\mathbf{k>0}$}
We assume for $q <k$ that
$\displaystyle \tilde{\partial}_{q+1}^\sharp \circ D_q
+ D_{q-1} \circ \tilde{\partial}_q^\sharp = \eta - \Id$ on
$\displaystyle \tilde{C}_q^\sharp(\VEC{p} \ast K;R)$
and
$\displaystyle D_q \big(\tilde{C}_q^\sharp(Q;R)\big) \subset
\tilde{C}_{q+1}^\sharp(Q;R)$ for all simplicial
subcomplexes $Q$ of $\VEC{p} \ast K$.

Suppose that $\sigma$ is an ordered $k$-simplex of $\VEC{p} \ast K$.
Let $\displaystyle c_\sigma = \eta(\sigma) - \Id(\sigma) -
D_{k-1}(\tilde{\partial}_k^\sharp(\sigma))$.
We have that $\displaystyle c_\sigma \in
\tilde{C}_k^\sharp(L_\sigma;R)$ because
$\displaystyle D_{k-1}\big(\tilde{C}_{k-1}^\sharp(L_\sigma;R)\big)
\subset \tilde{C}_k^\sharp(L_\sigma;R)$ by hypothesis of induction.

From our hypothesis of induction, we also have that
\begin{align*}
\tilde{\partial}_k^\sharp(c_\sigma)
&= \tilde{\partial}_k^\sharp\big( \eta(\sigma) \big)
- \tilde{\partial}_k^\sharp\big(\Id(\sigma)\big) -
\tilde{\partial}_k^\sharp\big(D_{k-1}(\tilde{\partial}_k^\sharp(\sigma))\big) \\
&= \tilde{\partial}_k^\sharp\big( \eta(\sigma) \big)
- \tilde{\partial}_k^\sharp\big(\Id(\sigma)\big)
- \big( - D_{k-2}\big(\tilde{\partial}_{k-1}^\sharp
(\tilde{\partial}_k^\sharp(\sigma))\big)
+ \eta(\tilde{\partial}_k^\sharp(\sigma))
- \Id(\tilde{\partial}_k^\sharp(\sigma))\big) \\
&= \eta(\tilde{\partial}_k^\sharp(\sigma))
- (\Id)(\tilde{\partial}_k^\sharp(\sigma))
+ D_{k-2}\big(\tilde{\partial}_{k-1}^\sharp
(\tilde{\partial}_k^\sharp(\sigma))\big)
- \eta(\tilde{\partial}_k^\sharp(\sigma))
+ \Id(\tilde{\partial}_k^\sharp(\sigma)) \\
&= 0
\end{align*}
because
$\displaystyle \tilde{\partial}_{k-1}^\sharp(\tilde{\partial}_k^\sharp(\sigma))
= 0$.
Thus $\displaystyle c_\sigma \in \tilde{Z}_k^\sharp(\VEC{p} \ast K;R)$.
Since $\displaystyle \tilde{H}_{k+1}^\sharp(\VEC{p} \ast K;R) = 0$,
we have that
$\displaystyle c_\sigma \in \tilde{B}_k^\sharp(\VEC{p} \ast K;R)$.
Therefore, there exists
$\displaystyle b_\sigma \in \tilde{C}_{k+1}^\sharp(\VEC{p} \ast K;R)$
such that $\displaystyle \tilde{\partial}_{k+1}^\sharp(b_\sigma) = c_\sigma$.
We may assume that
$\displaystyle b_\sigma \in \tilde{C}_{k+1}^\sharp(L_\sigma;R)$
because $\displaystyle c_\sigma \in \tilde{C}_k^\sharp(L_\sigma;R)$.

As expected, we set $D_k(\sigma) = b_\sigma$ for all ordered
$k$-simplices $\sigma$ of $\VEC{p} \ast K$, and extend $D_k$ by
linearity to $\displaystyle \tilde{C}_k^\sharp(\VEC{p} \ast K;R)$.

By linearity, we have that
\begin{align*}
\tilde{\partial}_{k+1}^\sharp(D_k(c)) + D_{k-1}(\tilde{\partial}_k^\sharp(c))
&= \big( \eta(c) - \Id(c) -
D_{k-1}(\tilde{\partial}_k^\sharp(c)) \big)
+ D_{k-1}(\tilde{\partial}_k^\sharp(c)) \\
&= \eta(c) - \Id(c)
\end{align*}
for all $\displaystyle c \in \tilde{C}_k^\sharp(\VEC{p} \ast K;R)$.
If $\displaystyle c \in \tilde{C}_k^\sharp(Q;R)$ for some simplicial
subcomplex $Q$ of $\VEC{p} \ast K$, then $c$ is the sum (after
simplification) of oriented $k$-simplices $\sigma$ in $Q$.  Since,
$D_k(\sigma) \in \tilde{C}_{k+1}^\sharp(L_\sigma;R) \subset
\tilde{C}_{k+1}^\sharp(Q,R)$
for all of these oriented $k$-simplices because $L_\sigma \subset Q$
for each of them, we have that $D_k(c) \in \tilde{C}_{k+1}^\sharp(Q,R)$.
In particular, $\displaystyle D_0 \big(\tilde{C}_k^\sharp(K;R)\big) \subset
\tilde{C}_{k+1}^\sharp(K;R)$.

The operator $D_k$ that we are looking for in the statement of the
proposition is $D_K$ defined above restricted to
$\displaystyle \tilde{C}_k^\sharp(K;R)$.
\end{proof}

We have that $G = F \circ \phi$.  Therefore,
$\displaystyle H_k^\sharp(G) = \tilde{H}_k^\sharp(F) \circ
\tilde{H}_k^\sharp(\phi) : H_k^\sharp(K;R) \to H_k^\sharp([K];R)$ is
an homomorphism.  Since
$\displaystyle \tilde{H}_k^\sharp(\phi): H_k^\sharp(K;R) \to
\tilde{H}_k^\sharp(K;R)$ is an isomorphism and
$\displaystyle \tilde{H}_k^\sharp(F): \tilde{H}_k^\sharp(K;R) \to
H_k^\sharp([K];R)$ is an isomorphism according to the
Proposition~\ref{propHkEHkv1}, we have that
$\displaystyle H_k^\sharp(G) : H_k^\sharp(K;R) \to H_k^\sharp([K];R)$ is an
isomorphism.  For similar reasons,
$\displaystyle H_k(G) : H_k(K;R) \to H_k([K];R)$ is an
isomorphism.

We can repeat the previous discussion in the context of relative
homology.
Since $\displaystyle \phi:C_k(L;R) \to \tilde{C}_k(L;R)$,
we may define the homomorphism
$\displaystyle \phi:C_k(K,L;R) \to \tilde{C}_k(K,L;R)$ by
$\phi(\relC[K,L]{c}) = \relC[K,L]{\phi(c)}$ for all
$\displaystyle \relC[K,L]{c} \in C_k(K,L;R)$, where as usual
$\relC[K,L]{\cdot}$ represents the
equivalence classes in both $\displaystyle C_k(K,L;R)$,
and $\displaystyle \tilde{C}_k(K,L;R)$.
Similarly,
Since $\displaystyle \psi:\tilde{C}_k(L;R) \to C_k(L;R)$,
we may define the homomorphism
$\displaystyle \psi:\tilde{C}_k(K,L;R) \to C_k(K,L;R)$ by
$\psi(\relC[K,L]{c}) = \relC[K,L]{\psi(c)}$ for all
$\displaystyle \relC[K,L]{c} \in \tilde{C}_k(K,L;R)$.
We have the following commutative diagram 
\[
\xymatrix@C+2ex{
\ar[r]^(0.2){\overline{\partial}_{k+2}}
& C_{k+1}(K,L;R)  \ar[r]^-{\overline{\partial}_{k+1}}
\ar@/^/[d]^-{\phi}
& C_k(K,L;R) \ar[r]^-{\overline{\partial}_k} \ar@/^/[d]^-{\phi}
& C_{k-1}(K,L;R) \ar[r]^(0.8){\overline{\partial}_{k-1}}
\ar@/^/[d]^-{\phi} & \\
\ar[r]_(0.2){\overline{\tilde{\partial}}_{k+2}}
& \tilde{C}_{k+1}(K,L;R) \ar[r]_-{\overline{\tilde{\partial}}_{k+1}}
\ar@/^/[u]^-{\psi}
& \tilde{C}_k(K,L;R) \ar[r]_-{\overline{\tilde{\partial}}_k}
 \ar@/^/[u]^-{\psi}
& \tilde{C}_{k-1}(K,L;R) \ar[r]_(0.8){\overline{\tilde{\partial}}_{k-1}}
\ar@/^/[u]^-{\psi} &
}
\]
Namely, $\{\phi_k\}_{k\in \ZZ}$ with $\phi_k = \phi$ for all $k$ is a
chain map from the chain complex
$\displaystyle \C = \{(C_k(K,L;R),
\overline{\partial}_k)\}_{k\in \ZZ}$
to the chain complex
$\tilde{\C} = \{(\tilde{C}_k(K,L;R),
\overline{\tilde{\partial}}_k)\}_{k\in \ZZ}$ and
$\{\psi_k\}_{k\in \ZZ}$ with $\psi_k = \psi$ for all $k$ is a
chain map from $\displaystyle \tilde{\C}$ to $\displaystyle \C$.

We may define the homomorphism
$\displaystyle \tilde{H}_k(\phi):H_k(K,L;R) \to \tilde{H}_k(K,L;R)$ by\\
$\displaystyle \tilde{H}_k(\phi)([c]_{K,L}) = [\phi(c)]_{K,L}$ for all
$\displaystyle [c]_{K,L} \in H_k(K,L;R)$,
where as usual $[\cdot]_{K,L}$ represents the equivalence classes in both
$\displaystyle H_k(K,L;R)$, and $\displaystyle \tilde{H}_k(K,L;R)$
As well, we may define the homomorphism
$\displaystyle \tilde{H}_k(\psi):\tilde{H}_k(K,L;R) \to H_k(K,L;R)$ by
$\displaystyle \tilde{H}_k(\psi)([c]_{K,L}) = [\psi(c)]_{K,L}$ for all
$\displaystyle [c]_{K,L} \in \tilde{H}_k(K,L;R)$.

We can repeat all the previous discussion for the reduced
homology and define
$\displaystyle \tilde{H}_k^\sharp(\phi):H_k^\sharp(K,L;R) \to
\tilde{H}_k^\sharp(K,L;R)$ and
$\displaystyle \tilde{H}_k^\sharp(\psi):\tilde{H}_k^\sharp(K,L;R) \to
H_k^\sharp(K,L;R)$.

As we did for the regular homology, we can use
Proposition~\ref{propCsCtfauHsH} to conclude that
$\displaystyle \tilde{H}_k^\sharp(\phi)$ is an isomorphism if and only
if $\displaystyle \tilde{H}_k(\phi)$ is an isomorphism.
To prove that $\displaystyle \tilde{H}_k^\sharp(\phi)$ is an
isomorphism, we prove that $\displaystyle \tilde{H}_k^\sharp(\psi)$ is
the inverse of $\displaystyle \tilde{H}_k^\sharp(\phi)$.  The proof is almost
identical to the proof that we gave above for the reduced homology.
The only difference is that we use the following proposition instead
of Proposition~\ref{propOOhomCM}.

\begin{prop}
There exists $\DD = \{ D_k \}_{k\in \ZZ}$ with
$\displaystyle D_k : \tilde{C}_k^\sharp(K,L;R) \to
\tilde{C}_{k+1}^\sharp(K,L;R)$ such that
$\displaystyle \tilde{\partial}_{k+1}^{\,\sharp} \circ D_k
+ D_{k-1} \circ \tilde{\partial}_k^{\,\sharp} =
\phi\circ \psi - \Id$ on $\displaystyle \tilde{C}_k^\sharp(K,L;R)$.
\end{prop}

\begin{proof}[Proof (Sketch)]
We can use the chain homotopy $\DD = \{ D_k \}_{k\in \ZZ}$ obtained in
Proposition~\ref{propOOhomCM} to induce a
chain homotopy between $\F = \{ f_k\}_{k\in \ZZ}$ with
$\displaystyle f_k = \Id:\tilde{C}_k^\sharp(K,L;R) \to \tilde{C}_k^\sharp(K,L;R)$
and $\GG = \{ g_k\}_{k\in \ZZ}$ with
$\displaystyle g_k = \phi\circ \psi :\tilde{C}_k^\sharp(K,L;R) \to
\tilde{C}_k^\sharp(K,L;R)$.

It suffices to note that $\displaystyle D_k: \tilde{C}_k^\sharp(K;R)
\to \tilde{C}_{k+1}^\sharp(K;R)$ was defined in such a way to have
that $\displaystyle D_k: \tilde{C}_k^\sharp(L;R)
\to \tilde{C}_{k+1}^\sharp(L;R)$.  Thus we can define
$\displaystyle D_k: \tilde{C}_k^\sharp(K,L;R)
\to \tilde{C}_{k+1}^\sharp(K,L;R)$ in a natural way as we have done
with $\displaystyle \tilde{\partial}_k^{\,\sharp}$ for instance to get
$\displaystyle \overline{\tilde{\partial}}_k^{\,\sharp}$.
\end{proof}

We get from $G = F \circ \phi$ that
$\displaystyle H_k(G) = \tilde{H}_k(F) \circ
\tilde{H}_k(\phi) : H_k(K,L;R) \to H_k([K],[L];R)$ is
an homomorphism.  Since
$\displaystyle \tilde{H}_k(\phi): H_k(K,L;R) \to \tilde{H}_k(K,L;R)$
is an isomorphism and
$\displaystyle \tilde{H}_k(F): \tilde{H}_k(K,L;R) \to H_k([K],[L];R)$
is an isomorphism according to the Proposition~\ref{propHkEHkv1}, we have that
$\displaystyle H_k(G) : H_k(K,L;R) \to H_k([K],[L];R)$ is an isomorphism.

In summary, we get the following result.

\begin{prop}  \label{propIsoSShom}
Let $L$ be a subcomplex of a simplicial complex $K$ and $R$ is an
integral domain. Then
$H_k(G): H_k(K;R) \to H_k([K];R)$
and $H_k(G) : H_k(K,L;R) \to H_k([K],[L];R)$ are 
isomorphisms.  We have a similar result for the reduced homology.
\end{prop}

To conclude this subsection, we should mention the following result.
Suppose that $f:([K_1],[L_1]) \to ([K_2],[L_2])$ is a continuous maps
where $L_i$ is a simplicial subcomplex of the simplicial complex
$K_i$ for $i =1,2$.  Then we have the following commutative
diagram.
\[
\xymatrix@C+2em{
H_k(K_1,L_1;R) \ar[r]^{H_k(f)}
\ar[d]_{H_k(G)} & H_k(K_2,L_2;R) \ar[d]^{H_k(G)} \\
H_k([K_1],[L_1];R) \ar[r]^-{{H}_k(f)} & H_k([K_2],[L_2];R)
}
\]
The proof of this result is interesting.  It makes use of simplicial
approximation.  This is a naturality result for those who may know
about category theory.

\subsection{Relation Between Simplicial Cohomology and Singular
Cohomology}

Suppose that $L$ is a simplicial subcomplex of a simplicial complex
$K$, and that $R$ is an integral domain.  We have defined in the
previous subsection an homomorphism
$\displaystyle G: C_k(K;R) \to S_k([K];R)$ and shown that it
induces isomorphism $H_k(G):H_k(K;R) \to H_k([K];R)$ and 
$H_k(G):H_k(K,L;R) \to H_k([K],[L];R)$.

By duality, we get the following result.

\begin{prop}
Let $L$ be a subcomplex of a simplicial complex $K$ and $R$ is an
integral domain. Then
$\displaystyle H^k(G): H^k([K];R) \to H^k(K;R)$
and $\displaystyle H^k(G) : H^k([K],[L];R) \to H^k(K,L;R)$ are 
isomorphisms.
\end{prop}

As for $H_k(G)$, suppose that $f:([K_1],[L_1]) \to ([K_2],[L_2])$ is a
continuous maps where $L_i$ is a simplicial subcomplex of the
simplicial complex $K_i$ for $i =1,2$.  Then we have the
following commutative diagram.
\[
\xymatrix@C+2em{
H^k([K_2],[L_2];R) \ar[r]^{H^k(f)}
\ar[d]_{H^k(G)} & H^k([K_1],[L_1];R) \ar[d]^{H^k(G)} \\
H^k(K_2,L_2;R) \ar[r]^-{{H}^k(f)} & H^k(K_1,L_1;R)
}
\]

%%% Local Variables:
%%% mode: latex
%%% TeX-master: "notes"
%%% End:
