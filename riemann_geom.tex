\chapter{Riemannian Geometry} \label{ChapRGeom}

To be precise, this chapter is an introduction to the Riemannian
geometry of surfaces (i.e. $2$-dimensional manifolds).  For a more
complete study, the reader should consult \cite{ST,Sv1}.  This chapter
is inspired in large part on \cite{ST}.

One of the major results of this chapter is Gauss-Bonnet theorem,
Theorem~\ref{thmGaussBonnet}, that relates integration of the
curvature over a manifold to the Euler characteristic of the
simplicial complex used to triangulate the manifold.  We will conclude
this chapter and these lecture notes by relating the sign of the
curvature of a manifold with constant curvature to the type of
geometry on the manifold.  The study of the possible types of
geometry is a really fascinating subject with deep historical
background.

\section{Riemannian Manifold}

\begin{defn}
A {\bfseries Riemannian manifold}\index{Riemannian Manifold} is a
manifold $S$ of class $\displaystyle C^\infty$ for which
there exists a map
$\displaystyle \tau: S \to \bigcup_{\VEC{u}\in S} \T^2(\TS_{\VEC{u}} S)$
of class $\displaystyle C^\infty$ such that
$\ps{}{}_{\VEC{u}} = \tau(\VEC{u})$ is an inner product
on $\TS_{\VEC{u}}S$ (See Definition~\ref{defnCmInnerP}).  The map
$\tau$ is called a {\bfseries Riemannian metric}\index{Riemannian Metric}
on $S$.
\end{defn}

The map $\tau$ in the previous definition is said to be smooth.
An equivalent way to define smoothness is as it follows.
We say that $\tau$ is smooth if the map defined by 
$\VEC{u} \mapsto \ps{F_1(\VEC{u})}{F_2(\VEC{u})}_{\VEC{u}}$ for $\VEC{u} \in S$
is of class $\displaystyle C^\infty$ for all smooth vector fields
$F_1$ and $F_2$ on $S$.

It follows from Proposition~\ref{propEIPonS} that it is always
possible to define a Riemannian metric on a manifold of $\displaystyle \RR^n$.
If $M$ is a submanifold of a Riemannian manifold $S$, then $M$
inherits a Riemannian metric by restricting
$\displaystyle \tau: S \to \bigcup_{\VEC{u}\in S} \T^2(\TS_{\VEC{u}} S)$
to $M$.

\begin{defn} \label{defnIsomS1S2}
Let $S_1$ and $S_2$ be two Riemannian manifolds.  An
{\bfseries isometry}\index{Isometry} between $S_1$ and $S_2$ is an
isomorphism $f:S_1\to S_2$ such that $f$ and
$\displaystyle f^{-1}:S_2 \to S_1$ are of class
$\displaystyle C^\infty$, and
$\ps{f_\ast(\VEC{u},\VEC{x}_1)}{f_\ast(\VEC{u},\VEC{x}_2)}_{f(\VEC{u})}
= \ps{(\VEC{u},\VEC{x}_1)}{(\VEC{u},\VEC{x}_2)}_{\VEC{u}}$
for all $(\VEC{u},\VEC{x}_1), (\VEC{u},\VEC{x}_2) \in \TS_{\VEC{u}} S_1$
and all $\VEC{u} \in S_1$.
\end{defn}

\begin{egg}
There may be more than one Riemannian metric on a manifold $S$.
Moreover, $S$ equipped with one of these Riemannian metric may not
be isometric to $S$ equipped with another Riemannian metric. 

For instance, the torus $\torus{2}$ seen as a submanifold of
$\displaystyle \RR^3$
inherits the Riemannian metric of $\displaystyle \RR^3$.  Namely,
$\ps{ (\VEC{u},\VEC{x}_1)}{(\VEC{u},\VEC{x}_2)}_{\VEC{u}}
= \ps{\VEC{x}_1}{\VEC{x}_2}$ for all
$(\VEC{u},\VEC{x}_1),(\VEC{u},\VEC{x}_2) \in \TS_{\VEC{u}} \torus{2}$ and
$\VEC{u} \in \torus{2}$ where the second inner product is the standard inner
product in $\displaystyle \RR^3$.

We also have that $\displaystyle \torus{2} \cong S^1 \times S^1 \subset \RR^4$.
Thus we may define a Riemannian metric on $\torus{2}$ based on the
Riemannian metric on $\displaystyle S^1$.  If
$\displaystyle \tau^\circ:S^1 \to
\bigcup_{\VEC{u} \in S^1} \T^2(\TS_{\VEC{u}} S^1)$
is a Riemannian metric on $\displaystyle S^1$ and
$\displaystyle \ps{}{}^\circ_{\VEC{u}} = \tau^\circ(\VEC{u})$ for all
$\displaystyle \VEC{u} \in S^1$, then we can define the
Riemannian metric
$\displaystyle \tau :S^1 \times S^1 \to
\bigcup_{(\VEC{u}_1,\VEC{u}_2) \in S^1 \times S^1}
\T^2\big(\TS_{(\VEC{u}_1,\VEC{u}_2)}\, (S^1 \times S^1)\big)$
on $\displaystyle \torus{2} \cong S^1 \times S^1$ as it follows.
We set
$\tau(\VEC{u}_1,\VEC{u}_2) = \ps{}{}_{(\VEC{u}_1,\VEC{u}_2)}$
for all $\displaystyle (\VEC{u}_1,\VEC{u}_2) \in S^1 \times S^1$ where
\begin{align*}
& \ps{\big((\VEC{u}_1,\VEC{u}_2),(\VEC{x}_{1,1},\VEC{x}_{2,1})\big)}
{\big((\VEC{u}_1,\VEC{u}_2),
(\VEC{x}_{1,2},\VEC{x}_{2,2})\big)}_{(\VEC{u}_1,\VEC{u}_2)} \\
&\hspace{5em}
= \ps{(\VEC{u}_1,\VEC{x}_{1,1})}{(\VEC{u}_1,\VEC{x}_{1,2})}_{\VEC{u}_1}^\circ
+ \ps{(\VEC{u}_2,\VEC{x}_{2,1})}{(\VEC{u}_2,\VEC{y}_{2,2})}_{\VEC{u}_2}^\circ
\end{align*}
for $\displaystyle (\VEC{u}_i,\VEC{x}_{i,j}) \in \TS_{\VEC{u}_i} S^1$
with $1 \leq i,j \leq 2$.  These two metrics are not equivalent because
$\displaystyle S^1 \times S^1$ with its Riemannian metric cannot be
imbedded as a submanifold of $\displaystyle \RR^3$.  The proof is
based on the fact that the ``curvature'' of $\displaystyle S^1 \times S^1$
is everywhere null.  Consult \cite{ST} for more information.
\end{egg}

\section{Preliminaries}

The first section of this chapter is technical and tedious but clear
the path for the study of Riemannian geometry later by giving us some
of the basic elements that we will play with later.  So, it is worth
the pain.

\subsection{Group of Rotations} \label{subsectGrRot}

Let $S$ be a $2$-dimension Riemannian manifold with a
Riemannian metric\\
$\displaystyle \tau:S \to \bigcup_{\VEC{u}\in S} \T^2(\TS_{\VEC{u}} S)$.
As usual, we let $\ps{}{}_{\VEC{u}} = \tau(\VEC{u})$ and
$\displaystyle \|\cdot\|_{\VEC{u}} = \ps{}{}_{\VEC{u}}^{1/2}$.

For $\VEC{u} \in S$, we define the action of the group of rotations
$\displaystyle S^1$ on $\TS_{\VEC{u}}S$ as it follows.  Given
$\displaystyle \eta = e^{i\theta} \in S^1 = \{ z \in \CC : |z|=1 \}$ and
$(\VEC{u},\VEC{x}) \in \TS_{\VEC{u}} S$, we set
$\eta(\VEC{u},\VEC{x}) = (\VEC{u},\theta.\VEC{x})$ where
$\theta.\VEC{x}$ is the rotation by $\theta$ radians about the origin 
of the vector $\VEC{x}$ in the $2$-dimension space $\TS_{\VEC{u}}S$.

We first prove that the function
\begin{align*}
\Phi : \TS S \times S^1 & \to \TS S \\
\big((\VEC{u},\VEC{x}), \eta) &\to \eta(\VEC{u},\VEC{x}) = (\VEC{u},
  \theta.\VEC{x})
\end{align*}
is a smooth function.  We have seen that if $(W,U,\phi)$ is a local
chart of $S$, then $(\tilde{W},\tilde{U}, \tilde{\phi})$ is a local
chart of $\TS S$ where $\displaystyle \tilde{W} = \TS\, W = W \times \RR^2$,
$\displaystyle \tilde{U} = \TS\, U = \bigcup_{\VEC{u} \in U} \TS_{\VEC{u}} S$,
and $\tilde{\phi}:\tilde{W} \to \tilde{U}$ is defined by
$\tilde{\phi}(\VEC{w}, \VEC{y}) = \phi_\ast(\VEC{w}, \VEC{y})
= \big(\phi(\VEC{w}),\diff \phi(\VEC{w}) \VEC{y} \big)$
for all $(\VEC{w},\VEC{y}) \in \tilde{W}$.
We consider a local chart of $\displaystyle \TS S \times S^1$ 
of the form
$\big(\tilde{W} \times I,\tilde{U} \times C,(\tilde{\phi},\psi)\big)$
where $I \subset \RR$ is an open interval of length less than $2 \pi$,
the function $\displaystyle \psi:\RR \to S^1$ is defined by
$\displaystyle \psi(\theta) = e^{i \theta}$ for $\theta \in \RR$,
and $C = \psi(I)$.
The condition that $I$ be of length less than $2 \pi$ is to ensure
that $\tilde{\psi}$ is one-to-one.

Since we assume that local chart $(W,U,\phi)$ of $S$ is orientation
preserving, we may assume that the orientation on
$\TS_{\phi(\VEC{w})} S$ is given by
$\mu_{\phi(\VEC{w})} = [(\phi(\VEC{w}),\diff \phi(\VEC{w}) \VEC{e}_1),
(\phi(\VEC{w}),\diff \phi(\VEC{w}) \VEC{e}_2)]$ while the
orientation on $\TS_{\VEC{w}} W$ is given by
$\mu_{\VEC{w}} = [(\VEC{w},\VEC{e}_1),(\VEC{w},\VEC{e}_2)]$.
Do not forget that $\displaystyle \diff \phi(\VEC{w}) : \RR^2 \to \RR^3$
is of rank $2$ for all $\VEC{w} \in W$.

We use Gram-Schmidt orthogonalization and change of bases to
express $\Phi$ in local coordinates.  Let
\[
\VEC{p}_1(\VEC{w}) = \diff \phi(\VEC{w}) \VEC{e}_1
\]
and
\[
\VEC{p}_2(\VEC{w}) = \diff \phi(\VEC{w}) \VEC{e}_2 -
\frac{\ps{\diff \phi(\VEC{w}) \VEC{e}_1}{\diff \phi(\VEC{w})
\VEC{e}_2}_{\phi(\VEC{w})}}{\ps{\diff \phi(\VEC{w}) \VEC{e}_1}
{\diff \phi(\VEC{w}) \VEC{e}_1}_{\phi(\VEC{w})}}
\diff \phi(\VEC{w}) \VEC{e}_1 \ .
\]
Let $\A = \{(\phi(\VEC{w}),\diff \phi(\VEC{w}) \VEC{e}_1)
, (\phi(\VEC{w}),\diff \phi(\VEC{w}) \VEC{e}_2)\}$
and $\B = \{ (\phi(\VEC{w}), \VEC{q}_1(\VEC{w})),
(\phi(\VEC{w}), \VEC{q}_2(\VEC{w}) )\}$ where
$\displaystyle \VEC{q}_i(\VEC{w}) = \|\VEC{p}_i(\VEC{w})\|_{\phi(\VEC{w})}^{-1}
\VEC{q}_i(\VEC{w})$
for $i=1,2$.  We have that $\B$ is orthonormal basis
for $\TS_{\phi(\VEC{w})}S$.
If $\displaystyle Q(\VEC{w}) = [\Id]_{\B}^{\A}$ is the matrix of
change of bases from $\A$ to $\B$, then
\begin{equation}  \label{smRootEq}
[\theta.(\diff \phi(\VEC{w}) \VEC{y})]_{\A}
= (Q(\VEC{w}))^{-1} \begin{pmatrix}
\cos(\theta) & -\sin(\theta) \\ \sin(\theta) & \cos(\theta) \end{pmatrix}
Q(\VEC{w}) \underbrace{ [\diff \phi(\VEC{w}) \VEC{y}]_{\A}}_{=\VEC{y}}
\end{equation}
for $\displaystyle \VEC{y} \in \RR^2$.
Hence, the local representation $\tilde{\Phi}$ of $\Phi$
satisfies
\[
\tilde{\phi}\big(\tilde{\Phi}\big((\VEC{w},\VEC{y}),\theta\big)
= \Phi\big((\tilde{\phi},\psi) \big((\VEC{w},\VEC{y}),\theta\big)
= \Phi\big((\phi(\VEC{w}), \diff \phi(\VEC{w}) \VEC{y}),\theta\big)
= \big( \phi(\VEC{w}), \theta.(\diff \phi(\VEC{w}) \VEC{y}) \big)
\]
for $\big((\VEC{w},\VEC{y}),\theta\big) \in \tilde{W} \times I$. 
Since the right hand side in (\ref{smRootEq}) is a smooth function,
we deduce that $\Phi$ is a smooth function.

Now that we have dealt with the smoothness of $\Phi$, we should focus
on the geometric aspect of the action of $\displaystyle \eta \in S^1$ on
$\TS_{\VEC{u}} S$ for all $\VEC{u}$.  This is the essential element
for the rest of this chapter.  Suppose that
$\B = \{ (\VEC{u},\VEC{v}_1), (\VEC{u},\VEC{v}_2)\}$ is an
oriented and orthonormal basis of $\TS_{\VEC{u}}S$.
Every element $(\VEC{u},\VEC{x}) \in \TS_{\VEC{u}} S$ can be
expressed as $(\VEC{u},\VEC{x})
=(\VEC{u}, a_1 \VEC{v}_1 + a_2\VEC{v}_2)$ for $a_1,a_2 \in \RR$.
Then $[\VEC{x}]_\B = \begin{pmatrix} a_1 \\ a_2 \end{pmatrix}$.
As we saw above, a rotation by $\theta \in \RR$ of $[\VEC{x}]_\B$ is
represented in the basis $\B$ by
\begin{equation} \label{txBeq1}
[\theta.\VEC{x}]_\B = \theta.[\VEC{x}]_\B = \begin{pmatrix}
\cos(\theta) & -\sin(\theta) \\ \sin(\theta) & \cos(\theta)
\end{pmatrix} \begin{pmatrix} a_1 \\ a_2 \end{pmatrix}
= \begin{pmatrix}
a_1 \cos(\theta) - a_ 2\sin(\theta) \\ a_1 \sin(\theta) + a_2 \cos(\theta)
\end{pmatrix} \ .
\end{equation}
Thus
$\theta.\VEC{x} = (a_1 \cos(\theta) - a_ 2\sin(\theta)) \VEC{v}_1
+ (a_1 \sin(\theta) + a_2 \cos(\theta)) \VEC{v}_2$.

The norm on $\TS_{\VEC{u}} S$ is defined by
$\displaystyle \| (\VEC{u},\VEC{x})\|_{\VEC{u}} = \left(
\ps{(\VEC{u},\VEC{x})}{(\VEC{u},\VEC{x})}_{\VEC{u}}\right)^{1/2}$
for all $(\VEC{u},\VEC{x}) \in \TS_{\VEC{u}} S$.
Since $\{ (\VEC{u},\VEC{v}_1), (\VEC{u},\VEC{v}_2)\}$ is an
orthonormal basis, we have that
$\displaystyle \| (\VEC{u},\VEC{x})\|_{\VEC{u}}
= \| (\VEC{u},a_1 \VEC{v}_1 + a_2 \VEC{v}_2)\|_{\VEC{u}} =
\sqrt{a_1^2 + a_2^2}$.   We also have that
$\displaystyle \| (\VEC{u},\theta.\VEC{x})\|_{\VEC{u}} = \sqrt{a_1^2 + a_2^2}$.
In particular, if\\
$\| (\VEC{u},\VEC{x})\|_{\VEC{u}}=1$, then
$\| (\VEC{u},\theta.\VEC{x})\|_{\VEC{u}} = 1$.

For $(\VEC{u},\VEC{x})$ fixed, we may use (\ref{txBeq1}) to compute
the derivative of the map from $R:\RR \to \TS_{\VEC{u}} S$ defined by
$R(\theta) = \eta(\VEC{u},\VEC{x}) = (\VEC{u},\theta.\VEC{x})$.  We
find that
\begin{equation}  \label{txBeq2}
R'(\theta) = (\VEC{u},(\pi/2 + \theta).\VEC{x}) =
\upsilon (\eta(\VEC{u},\VEC{x}))
\end{equation}
where $\displaystyle \upsilon = e^{i\pi/2} = i$.  Note that this
relation can easily be deduced if we express (\ref{txBeq1}) in
$\displaystyle \CC \cong \RR^2$.

\subsection{The Tangent Bundle $\TS(\TS\,S)$} \label{subsectTTS}

Let $S$ be a $2$-dimensional Riemannian manifold.
Recall that $\TS S$ is a $4$-dimensional manifold.  In particular, if
$(W,U,\phi)$ is a local chart of $S$, then
$(\tilde{W},\tilde{U}, \tilde{\phi})$ is a local chart of $\TS\, S$ where
$\displaystyle \tilde{W} = W \times \RR^2$,
$\displaystyle \tilde{U} = \TS\, U = \bigcup_{\VEC{u} \in U} \TS_{\VEC{u}} S$,
and $\tilde{\phi}:\tilde{W} \to \tilde{U}$ is defined by
$\tilde{\phi}(\VEC{w}, \VEC{y}) = \phi_\ast(\VEC{w}, \VEC{y})
= \big(\phi(\VEC{w}),\diff \phi(\VEC{w}) \VEC{y} \big)$
for all $(\VEC{w},\VEC{y}) \in \tilde{W}$.

Using the procedure used to obtain the local chart
$(\tilde{W},\tilde{U},\tilde{\phi})$ above, we find that a local chart
of $\TS (\TS\,S)$ is given by $(\breve{W},\breve{U}, \breve{\phi})$ where
$\displaystyle \breve{W} = (W \times \RR^2) \times (\RR^2 \times \RR^2)$,
$\displaystyle \breve{U} = \TS\, \tilde{U}
= \bigcup_{(\VEC{u},\VEC{x}) \in \tilde{U}} \TS_{(\VEC{u},\VEC{x})} (\TS\,S)$, and
$\breve{\phi}:\breve{W} \to \breve{U}$ is defined by
\begin{align*}
\breve{\phi}\big((\VEC{w}, \VEC{y}),(\VEC{r},\VEC{s})\big) &=
\tilde{\phi}_\ast\big((\VEC{w}, \VEC{y}),(\VEC{r},\VEC{s})\big)
= \left( \tilde{\phi}(\VEC{w}, \VEC{y}),
\diff_{\VEC{w},\VEC{y}} \tilde{\phi}(\VEC{w},\VEC{y})
\begin{pmatrix} \VEC{r} \\ \VEC{s} \end{pmatrix} \right) \\
&= \big( (\phi(\VEC{w}), \diff \phi(\VEC{w}) \VEC{y}),
\big( \diff \phi(\VEC{w}) \, \VEC{r} ,
\diff_{\VEC{w}} (\diff \phi(\VEC{w}) \, \VEC{y} ) \VEC{r} 
+  \diff \phi(\VEC{w}) \, \VEC{s} \big) \big)
\end{align*}
for all $\big((\VEC{w},\VEC{y}),(\VEC{r},\VEC{s})\big) \in \breve{W}$.
Note that $(\VEC{y},\VEC{r}) \mapsto \diff_{\VEC{w}} (\diff
\phi(\VEC{w}) \, \VEC{y} ) \VEC{r}$ is a bilinear map.

The important information to remember is that
$\TS (\TS\,S)$ is a $8$-dimensional manifold,
that $\TS_{\VEC{u},\VEC{x}} (\TS \, S)$ is a $4$-dimensional vector space
for all $(\VEC{u},\VEC{x}) \in \TS\,S$ and that the
elements of $\TS_{\VEC{u},\VEC{x}} (\TS \, S)$ are of the form
$\big( (\VEC{u},\VEC{x}), (\VEC{p},\VEC{q}) \big)$ where
$(\VEC{u},\VEC{p}) \in \TS_{\VEC{u}}\,S$.

Let $\pi_S : \TS S \to S$ be the projection defined by
$\pi_S(\VEC{u},\VEC{x}) = \VEC{u}$ for all
$(\VEC{u},\VEC{x}) \in \TS_{\VEC{u}} S$.  This is obviously a smooth
map since the representation of $\pi_S$ with respect to a local charts
$(W,U,\phi)$ and $(\tilde{W},\tilde{U},\tilde{\phi})$ defined above is
$\pi(\VEC{w},\VEC{y}) = \VEC{w}$ for all
$\displaystyle (\VEC{w},\VEC{y}) \in W \times \RR^2$.

\begin{lemma}  \label{lemmaPsast}
We have that $(\pi_S)_\ast: \TS (\TS\, S) \to \TS S$ is given by
$(\pi_S)_\ast( (\VEC{u},\VEC{x}), (\VEC{p},\VEC{q}) ) = (\VEC{u},\VEC{p})$
for all $((\VEC{u},\VEC{x}), (\VEC{p},\VEC{q})) \in
\TS_{(\VEC{u},\VEC{x})} (\TS\, S)$ and $(\VEC{u},\VEC{x}) \in \TS\, S$.
\end{lemma}

\begin{proof}
Suppose that $(\tilde{W},\tilde{U}, \tilde{\phi})$ and
$(\breve{W},\breve{U}, \breve{\phi})$ are the local charts of
$\TS\,S$ and $\TS (\TS\,S)$ respectively defined above.  Moreover, suppose that
$\big((\VEC{w}, \VEC{y}),(\VEC{r},\VEC{s})\big) \in \breve{W}$
is such that
$\breve{\phi}\big((\VEC{w}, \VEC{y}),(\VEC{r},\VEC{s})\big) =
\big((\VEC{u},\VEC{x}), (\VEC{p},\VEC{q})\big) \in \breve{U}$
with $\VEC{u} = \phi(\VEC{w})$ and $\VEC{p} = \diff \phi(\VEC{w}) \VEC{r}$.

If $\displaystyle \pi:W \times \RR^2 \to \RR^2$ is the projection
defined by $\pi(\VEC{w},\VEC{y}) = \VEC{w}$ for all
$\displaystyle (\VEC{w},\VEC{y}) \in W \times \RR^2$, then
$\pi_\ast \big( (\VEC{w},\VEC{y}),(\VEC{r},\VEC{s})\big)
= \big( \pi(\VEC{w},\VEC{y}), \diff \pi(\VEC{w})
(\VEC{r},\VEC{s})\big)
= \big( \VEC{w}, \pi(\VEC{r},\VEC{s})\big)
= (\VEC{w}, \VEC{r})$
for all $((\VEC{w},\VEC{y}), (\VEC{r},\VEC{s})) \in \breve{W}$.

Since $\phi \circ \pi = \pi_S \circ \tilde{\phi}$ on $W$, we get that
$\displaystyle
\tilde{\phi} \circ \pi_\ast = \phi_\ast \circ \pi_\ast
=(\pi_S)_\ast \circ \tilde{\phi}_\ast
=(\pi_S)_\ast \circ \breve{\phi}$ on $\tilde{W} = \TS\, W$.
Hence
\begin{align*}
(\pi_S)_\ast \big((\VEC{u},\VEC{x}), (\VEC{p},\VEC{q})\big)
&= (\pi_S)_\ast \big( \breve{\phi} ((\VEC{w},\VEC{y}),
(\VEC{r},\VEC{s})) \big)
= \tilde{\phi} \big(\pi_\ast((\VEC{w},\VEC{y}),
(\VEC{r},\VEC{s})) \big)
= \tilde{\phi}(\VEC{w},\VEC{r}) \\
&= (\phi(\VEC{w}),\diff \phi(\VEC{w}) \VEC{r}) = (\VEC{u},\VEC{p}) \ .
\qedhere
\end{align*}
\end{proof}

\begin{prop} \label{proTSprpts}
Suppose that $S_1$ and $S_2$ are two $2$-dimensional Riemannian
manifolds and that $f:S_1 \to S_2$ is an isometry.  Let
$\tilde{f} = f_\ast:\TS\, S_1 \to \TS\, S_2$.  Then
\begin{enumerate}
\item $\pi_{S_2} \circ \tilde{f} = f \circ \pi_{S_1}$ on $\TS\, S_1$, and
\item $(\pi_{S_2})_\ast \circ \tilde{f}_\ast = \tilde{f} \circ
(\pi_{S_1})_\ast$ on $\TS\, (\TS\, S_1)$.
\end{enumerate}
Moreover, if $S_1$ and $S_2$ are oriented and $f$ preserves the
orientation, then
\begin{enumerate} \addtocounter{enumi}{2}
\item $\eta \circ \tilde{f} = \tilde{f} \circ \eta$ for all
$\displaystyle \eta \in S^1$ acting on $\TS\, S_1$ and $\TS\, S_2$.
\end{enumerate}
\end{prop}

\begin{proof}
Let $(W_i,U_i,\phi_i)$, $(\tilde{W}_i,\tilde{U}_i,\tilde{\phi}_i)$ and 
$(\breve{W}_i,\breve{U}_i, \breve{\phi}_i)$ be the previously defined
local charts of $S_i$, $\TS\, S_i$ and $\TS\, (\TS\, S_i)$ respectively for
$i =1,2$.

Let $g:W_1 \to W_2$ be the local representation of $f$;
namely, $f \circ \phi_1 = \phi_2 \circ g$ on $W_1$.  We then have that
$\tilde{f} \circ (\phi_1)_\ast = f_\ast \circ (\phi_1)_\ast
= (\phi_2)_\ast \circ g_\ast$ on $\TS\, W_1 = \tilde{W}_1$.

\stage{1} Given $(\VEC{u},\VEC{x}) \in \tilde{U}_1 \subset \TS_{\VEC{u}} S_1$,
we have that $(\VEC{u},\VEC{x}) = (\phi_1)_\ast(\VEC{w},\VEC{y})$ for some
$(\VEC{w},\VEC{y}) \in \tilde{W}_1$.  Hence
\begin{align*}
(\pi_{S_2} \circ \tilde{f})(\VEC{u},\VEC{x})
&= \pi_{S_2}\big( (\tilde{f} \circ (\phi_1)_\ast)(\VEC{w},\VEC{y}) \big)
= \pi_{S_2}\big( (\phi_2)_\ast \circ g_\ast)(\VEC{w},\VEC{y}) \big) \\
&= \pi_{S_2}\big( (\phi_2 \circ g)_\ast(\VEC{w},\VEC{y}) \big)
= (\phi_2 \circ g)(\VEC{w})
= (\phi_2 \circ g \circ \phi_1^{-1})(\phi_1(\VEC{w})) \\
&= f(\VEC{x}) = (f\circ \pi_{S_1})(\VEC{u},\VEC{x}) \ .
\end{align*}

\stage{2}
Since $\pi_{S_2} \circ \tilde{f} = f \circ \pi_{S_1}$, we get
\[
 (\pi_{S_2})_\ast \circ \tilde{f}_\ast 
= (\pi_{S_2} \circ \tilde{f})_\ast
= (f \circ \pi_{S_1})_\ast
= f_\ast \circ (\pi_{S_1})_\ast
= \tilde{f} \circ (\pi_{S_1})_\ast \ .
\]

\stage{3} We need to show that
$\eta\big(\tilde{f}(\VEC{u},\VEC{x})\big)
= \tilde{f}\big(\eta(\VEC{u},\VEC{x})\big)$
for all $(\VEC{u},\VEC{x}) \in \TS_{\VEC{u}} S_1$ and
$\displaystyle \eta \in S^1$.

Since
\[
\ps{\tilde{f}\big(\eta(\VEC{u},\VEC{x})\big)}
{\tilde{f}(\VEC{u},\VEC{x})}_{f(\VEC{u})} = 
\ps{\eta(\VEC{u},\VEC{x})}{(\VEC{u},\VEC{x})}_{\VEC{u}} \ ,
\]
the angle between $\tilde{f}\big(\eta(\VEC{u},\VEC{x})\big)$ and
$\tilde{f}(\VEC{u},\VEC{x})$ is equal modulo the direction of the rotation
to the angle between $\eta(\VEC{u},\VEC{x})$ and $(\VEC{u},\VEC{x})$;
namely, it is equal to $\eta$ modulo the direction of the rotation.
But we also assume that $f$ is orientation preserving, thus
$\tilde{f}\big(\eta(\VEC{u},\VEC{x})\big)
= \eta\big(\tilde{f}(\VEC{u},\VEC{x})\big)$ 
for all $(\VEC{u},\VEC{x}) \in \TS_{\VEC{u}} S_1$ and
$\displaystyle \eta \in S^1$.
\end{proof}

\begin{prop} \label{propfisoRot}
Let $S$ be a $2$-dimensional oriented and Riemannian
manifold, and $f:S\to S$ be an orientation preserving isometry
such that $f(\VEC{u}) = \VEC{u}$ for some $\VEC{u} \in S$, then
$f_\ast: \TS_{\VEC{u}} S \to \TS_{\VEC{u}} S$ is a simple rotation by
an angle $\displaystyle \eta \in S^1$.
\end{prop}

\begin{proof}
Choose $(\VEC{u},\VEC{x}) \in \TS_{\VEC{u}} S$.  Since $f_\ast$ preserve
the length, we have that
$\|f_\ast(\VEC{u},\VEC{x})\|_{\VEC{u}} = \|(\VEC{u},\VEC{x})\|_{\VEC{u}}$.
Let $\displaystyle \eta \in S^1$ be the angle
such that $f_\ast(\VEC{u},\VEC{x}) = \eta(\VEC{u}, \VEC{x})$.

Given $(\VEC{u},\VEC{y}) \in \TS_{\VEC{u}} S$, we have that
$(\VEC{u},\VEC{y}) = a (\VEC{u},\tilde{\VEC{y}})
= (\VEC{u}, a\tilde{\VEC{y}})$ where $a \in \RR$ and
$\|(\VEC{u},\tilde{\VEC{y}})\|_{\VEC{u}} = \|(\VEC{u},\VEC{x})\|_{\VEC{u}}$.
There exists $\displaystyle \mu \in S^1$ such that
$(\VEC{u},\tilde{\VEC{y}}) = \mu(\VEC{u},\VEC{x})$.  Hence, it follows from
the third item of Proposition~\ref{proTSprpts} that
\begin{align*}
f_\ast(\VEC{u},\VEC{y}) &= f_\ast(\VEC{u},a \tilde{\VEC{y}})
= a f_\ast(\VEC{u},\tilde{\VEC{y}})
= a f_\ast\big(\mu(\VEC{u},\VEC{x})\big)
= a \mu\big( f_\ast(\VEC{u},\VEC{x})\big)
= a \mu\big( \eta(\VEC{u},\VEC{x})\big) \\
&= a \eta\big(\mu(\VEC{u},\VEC{x})\big)
= a \eta(\VEC{u},\tilde{\VEC{y}})
= \eta(\VEC{u}, a \tilde{\VEC{y}})
= \eta(\VEC{u}, \VEC{y})
\end{align*}
where we have used the fact that
$f_\ast\big|_{\TS_{\VEC{u}} S}$ is a linear mapping.
\end{proof}

\subsection{The Tangent Bundle $\TS S_c$} \label{subsectTSc}

We begin with a definition.

\begin{defn}
Let $S$ be a $2$-dimensional Riemannian manifold.  The
{\bfseries circle bundle}\index{Circle Bundle} is the set defined
by $\displaystyle S_c = \bigcup_{\VEC{u} \in S} S_c(\VEC{u})$, where
$S_c(\VEC{u}) = \big\{ (\VEC{u},\VEC{x}) \in \TS_{\VEC{u}} S :
\ps{(\VEC{u},\VEC{x})}{(\VEC{u},\VEC{x})}_{\VEC{u}} = 1 \}$ for all
$\VEC{u} \in S$.
\end{defn}

For each $\VEC{u} \in S$ fixed, the set $S_c(\VEC{u})$ is the unit circle in
$\TS_{\VEC{u}} S$ centred at the origin.  Therefore, given
$\displaystyle \eta = e^{i\theta} \in S^1$,  we have that
$\eta : S_c \to S_c$.  This defines a smooth mapping.

\begin{prop}
If $S$ be a $2$-dimensional Riemannian manifold, then
$S_c$ is a $3$-dimensional manifold of class $\displaystyle C^\infty$
\end{prop}

\begin{proof}
We could prove this proposition by explicitly providing smooth local
charts as we will do later.  Instead, we consider the map
$\displaystyle g:\TS S \to \RR$ defined by
$g\big((\VEC{u},\VEC{x})\big)
= \ps{(\VEC{u},\VEC{x})}{(\VEC{u},\VEC{x})}_{\VEC{u}}$ for all
$(\VEC{u},\VEC{x}) \in \TS_{\VEC{x}} S$ and $\VEC{u} \in S$ and prove
that $\displaystyle S_c = g^{-1}(\{1\})$ is a smooth manifold.

Suppose that $(W,U,\phi)$ is a local chart of the manifold $S$.  The local
representation of $g$ is given by
\[
\phi^\ast(g) (\VEC{w},\VEC{y})
= g\big( \phi_\ast(\VEC{w},\VEC{y})\big)
= \ps{(\phi(\VEC{w}),\diff \phi(\VEC{w})\, \VEC{y})}
{(\phi(\VEC{w}),\diff \phi(\VEC{w})\, \VEC{y})}_{\phi(\VEC{w})}
\]
for all $\displaystyle
(\VEC{w},\VEC{y}) \in W \times \RR^2 \subset \RR^2 \times \RR^2$.
Since
$\displaystyle \diff_{\VEC{w},\VEC{y}} \phi^\ast(g)(\VEC{w},\VEC{y})$
is of rank $1$ for all
$\displaystyle (\VEC{w},\VEC{y}) \in W \times \RR^2$ such that
$\displaystyle \phi^\ast(g)(\VEC{w},\VEC{y}) = 1$, We get from
Theorem~\ref{thmIFTmanif} that
$\displaystyle (\phi^\ast(g))^{-1}(\{1\})$ is a
$3$-dimensional manifold in $\displaystyle W \times \RR^2$.

If $(\tilde{W},\tilde{U},\tilde{\phi})$ is a local
chart of $\displaystyle (\phi^\ast(g))^{-1}(\{1\}) \subset W \times \RR^2$,
then $(\tilde{W},\phi_\ast(\tilde{U}), \phi_\ast \circ \tilde{\phi})$ is
local chart of $\displaystyle S_c = g^{-1}(\{1\})$.
\end{proof}

Let $S$ be a $2$-dimensional Riemannian manifold.  We first define
local charts of $S_c$.  Let $(W,U,\phi)$ be a local chart of $S$.  We
define the smooth function
$\displaystyle E:W \to \bigcup_{\VEC{w} \in W} \diff \phi(\VEC{w}) (\RR^2)$
by
\[
E(\VEC{w}) = \| \diff \phi(\VEC{w})
(\VEC{e}_1) \|_{\phi(\VEC{w})}^{-1} \diff \phi(\VEC{w}) (\VEC{e}_1)
\]
for $\VEC{w} \in W$.  Recall that $\diff \phi(\VEC{w})$ is of rank $2$
for all $\VEC{w} \in W$ by definition of local charts.  Therefore, the
norm is never null.  A local chart of $S_c$ is defined by
$(\tilde{W},\tilde{U},\tilde{\phi})$ where
$\tilde{W} = W \times I$ with $I \subset \RR$ an open interval of
length less than $2\pi$, $\tilde{\phi}$ is defined by
\[
\tilde{\phi}(\VEC{w},\theta) = \big(\phi(\VEC{w}), \theta.E(\VEC{w}) \big)
\]
for $(\VEC{w},\theta) \in \tilde{W}$,
and $\displaystyle \tilde{U} = \tilde{\phi}(\tilde{W})
\subset \bigcup_{\VEC{u} \in U} S_c(\VEC{u})$.
The condition that $I$ be of length less than $2 \pi$ is to ensure
that $\tilde{\phi}$ is one-to-one.  The proof of the smoothness of the
function $\tilde{\phi}$ is similar to the proof of the smoothness of
the function $\Phi$ in Subsection~\ref{subsectGrRot}.

\begin{rmk}
We have that                  \label{rmkUS1eScU}
$\displaystyle S_c(U) = \bigcup_{\VEC{u} \in U} S_c(\VEC{u})
\cong W \times S^1$.  To prove this statement,
we define the mapping $\displaystyle g:W \times S^1 \to S_c(U)$ by
$g(\VEC{w}, \eta) = \eta\big(\phi(\VEC{w}),E(\VEC{w})\big)$
for $\VEC{w} \in W$ and $\displaystyle \eta \in S^1$.
The map $g$ is an isomorphism between $\displaystyle W\times S^1$ and
$S_c(U)$.

Since $\phi:W \to U$ is an isomorphism, we have that
\[
G(\VEC{u},\eta) = g(\phi^{-1}(\VEC{u}), \eta)
= \eta\big(\VEC{u},E(\phi^{-1}(\VEC{u}))\big)
\]
for $\VEC{u} \in U$ and $\displaystyle \eta \in S^1$
defines an isomorphism between $\displaystyle U \times S^1$ and $S_c(U)$.

However, it is generally false that $\displaystyle S_c \cong S \times S^1$.
For instance, we cannot have $\displaystyle S_c \cong S \times S^1$
for $\displaystyle S = S^2$ because the isomorphism implies the
existence of a non-null vector field on $S$ and we know from
Proposition~\ref{propNNVFodd} that this is not possible on
$\displaystyle S^2$.
\end{rmk}

Since $S_c$ is a manifold, then $\TS \, S_c$ is well define.
Using the procedure to obtain local charts of the previous subsection,
a local chart of $\TS S_c$ is given by
$(\breve{W},\breve{U},\breve{\phi})$ where
$\displaystyle \breve{W} = (W \times I) \times (\RR^2 \times \RR)$
with $I \subset \RR$ an open interval of length less than
$2\pi$, $\breve{U} = \tilde{\phi}_\ast(\breve{W})$, and
$\breve{\phi} = \tilde{\phi}_\ast$.  In particular, we have
\begin{align}
\breve{\phi}\big((\VEC{w},\theta),(\VEC{r},t)\big) 
&= \tilde{\phi}_\ast\big((\VEC{w},\theta),(\VEC{r},t)\big) 
= \left( \tilde{\phi}(\VEC{w},\theta) ,
\diff_{\VEC{w},\theta} \tilde{\phi}(\VEC{w},\theta)\,
\begin{pmatrix} \VEC{r} \\ t \end{pmatrix} \right) \nonumber \\
&= \left( \tilde{\phi}(\VEC{w},\theta), \left( \diff \phi(\VEC{w}) \, \VEC{r}, 
\diff_{\VEC{w},\theta} (\theta. E(\VEC{w})) \,
\begin{pmatrix} \VEC{r} \\ t \end{pmatrix} \right) \right)\nonumber  \\
&= \Big( \tilde{\phi}(\VEC{w},\theta) , \Big( \diff \phi(\VEC{w}) \, \VEC{r}, 
\diff_{\VEC{w}} (\theta. E(\VEC{w})) \, \VEC{r} +
 t\, \pdfdx{(\theta. E(\VEC{w}))}{\theta} \Big) \Big) \nonumber \\
&= \big( \tilde{\phi}(\VEC{w},\theta) , \big( \diff \phi(\VEC{w}) \, \VEC{r}, 
\diff_{\VEC{w}} (\theta. E(\VEC{w})) \, \VEC{r} +
 t\, (\pi/2 + \theta). E(\VEC{w}) \big) \big) \label{defnbphiEq}
\end{align}
for all $\big((\VEC{w},\theta),(\VEC{r},t)\big) \in \breve{W}$ where
we have used (\ref{txBeq2}) to obtain the last equality.

We define a vector field $H$ on $S_c$ using local charts.  Let
\begin{equation} \label{HpisAstEq}
H(\VEC{u},\VEC{x}) = \tilde{\phi}_\ast\big((\VEC{w},\theta),(\VEC{0},1)\big)
\end{equation}
for $(\VEC{u},\VEC{x}) = \tilde{\phi}(\VEC{w},\theta)$.  This is a
smooth vector field on $S_c$ because its local representation
is given by
\[
\big((\tilde{\phi}_\ast)^{-1} \circ H \circ \tilde{\phi}\big)
(\VEC{w},\theta) = \big((\VEC{w},\theta),(\VEC{0},1)\big)
\]
for $(\VEC{w},\theta) \in \tilde{W} = W \times I$.  Note that
\begin{equation} \label{defnHonSc}
\begin{split}
H(\VEC{u},\VEC{x}) &= \tilde{\phi}_\ast\big((\VEC{w},\theta),(\VEC{0},1)\big)
= \Big( (\VEC{u},\theta.\tilde{E}(\VEC{u})),
\Big( \VEC{0}, \pdfdx{ (\nu.\tilde{E}(\VEC{u}))}{\nu}\Big|_{\nu=\theta}
\Big)\Big) \\
&= \Big( (\VEC{u},\VEC{x}),
\Big( \VEC{0}, \pdfdx{ (\nu.\tilde{E}(\VEC{u}))}{\nu}\Big|_{\nu=\theta}
\Big)\Big)
\end{split}
\end{equation}
where $\displaystyle \tilde{E}(\VEC{u}) = E(\phi^{-1}(\VEC{u})) = E(\VEC{w})$.

\begin{rmk}
Using the local charts    \label{rmkHeptheta}
$(\tilde{W},\tilde{U},\tilde{\phi})$ of $S_c$
and the interpretation of the tangent space of $S_c$ has a space of
differential linear operators on $S_c$ given in
Subsection~\ref{secTSasDiffop}, we observe that
$H(\VEC{u},\VEC{x})$ is represented locally as
$\displaystyle \pdydx{}{\theta}\Big|_{(\VEC{w},\theta)}$
where $(\VEC{u},\VEC{x}) = \tilde{\phi}(\VEC{w},\theta)$.
\end{rmk}

\begin{rmk}
It is possible to define a basis   \label{RMKsubsectTSc}
for $\TS_{(\VEC{u},\VEC{x})} S_c$.
Suppose that $(\tilde{W},\tilde{U},\tilde{\phi})$ is a local chart of
$S_c$ as defined above with $(\VEC{u},\VEC{x}) \in S_c$.
Suppose that $(\VEC{u},\VEC{x}) = \tilde{\phi}(\VEC{w},\psi)$
for $(\VEC{w},\psi) \in \tilde{W} = W \times I$.
Let $\rho_i:]-a,a[ \to W$ for $i =1,2$ be two curves such that
$\rho_1(0) = \rho_2(0) = \VEC{w}$, and
$\displaystyle \rho_1'(0), \rho_2'(0) \in \RR^2$ are linearly
independent.  Consider the curves
$\sigma_i:]-a,a[ \to W\times I$ for $i =1,2$ defined by
$\sigma_i(t) = (\rho_i(t),\psi)$ for $-a < t < a$.
Then $\tilde{\sigma}_i = \tilde{\phi} \circ \sigma_i : ]-a,a[ \to S_c$
for $i = 1,2$ are two curves in $S_c$ such that
$\tilde{\sigma}_i(0) = \tilde{\phi}(\rho_i(0),\psi) =
\tilde{\phi}(\VEC{w},\psi) = (\VEC{u},\VEC{x})$
and
\begin{align*}
(\tilde{\sigma}_i)_\ast(0,1)
&= \Big( \tilde{\sigma}_i(0) , \dfdx{\tilde{\sigma}_i(t)}{t}\Big|_{t=0} \Big)
= \Big((\VEC{u},\VEC{x}),\Big(\dfdx{\phi(\rho_i(t))}{t}\Big|_{t=0},
\VEC{q}_i\Big) \Big) \\
&= \big((\VEC{u},\VEC{x}), (\diff \phi(\VEC{w}) \rho_i'(0),\VEC{q}_i)\big)
\end{align*}
for some $\displaystyle \VEC{q}_i \in \RR^n$ \footnote{We do not need
to know the complete expression for the $\VEC{q}_i$.}.
Hence $(\tilde{\sigma}_1)_\ast(0,1),(\tilde{\sigma}_2)_\ast(0,1) \in
\TS_{(\VEC{u},\VEC{x})} S_c$ are linearly independent because
$\displaystyle \diff \phi(\VEC{w}) \rho_1'(0), \diff \phi(\VEC{w})
\rho_2'(0) \in \RR^n$ are linearly independent since $\diff \phi(\VEC{w})$
is of rank $2$.

Let $\sigma_3:]-a,a[ \to W\times I$ be the curve defined by
$\sigma_3(t) = (\VEC{w}, \psi + t)$ for $-a < t < a$.  We assume that
$a$ is small enough to have $[\psi- a,\psi+ a] \subset I$.
Then $\tilde{\sigma}_3 = \tilde{\phi} \circ \sigma_3 : ]-a,a[ \to S_c$
is a curve in $S_c$ such that
$\tilde{\sigma}_3(0) = \tilde{\phi}(\VEC{w},\psi) = (\VEC{u},\VEC{x})$
and
\begin{align*}
(\tilde{\sigma}_3)_\ast(0,1)
&= \Big( \tilde{\sigma}_3(0), \dfdx{\tilde{\sigma}_3(t)}{t}\Big|_{t=0}
\Big)
= \Big((\VEC{u},\VEC{x}),\Big(\VEC{0},
\dfdx{(\psi+t).E(\VEC{w})}{t}\Big|_{t = 0}\Big) \\
&= \Big((\VEC{u},\VEC{x}),\Big(\VEC{0},
\dfdx{(\nu.E(\VEC{w}))}{\nu}\Big|_{\nu = \psi}\Big)
=H(\VEC{u},\VEC{x}) \ .
\end{align*}
We have that
$\{ (\tilde{\sigma}_1)_\ast(0,1), (\tilde{\sigma}_2)_\ast(0,1),
(\tilde{\sigma}_3)_\ast(0,1)\}$ is a basis of the $3$-dimensional
space\\ $\TS_{(\VEC{u},\VEC{x})} S_c$.

It is interesting to note that
$(\pi_S)_\ast:\TS_{(\VEC{u},\VEC{x})} S_c \to \TS_{\VEC{u}} S$
is onto.  This follows from\\
$(\pi_S)_\ast\big( (\tilde{\sigma}_i)_\ast(0,1) \big)
= \big(\VEC{u},\diff \phi(\VEC{w}) \rho_i'(0) \big)$ for
$i =1,2$ because $\displaystyle \rho_1'(0)$ and
$\rho_2'(0)$ are linearly independent.  It is also interesting to note
that the vector $\displaystyle
\dfdx{(\nu.E(\VEC{w})\big)}{\nu}\Big|_{\nu = \psi} \in \RR^n$
is of norm $1$ because it is the derivative of the parametric
representation $\theta \mapsto \theta.E(\VEC{w})$ of a unit circle in
$\displaystyle \RR^n$ centred at the origin where $\theta$ represents
the arc length.
\end{rmk}

\begin{rmk}
For the readers who have difficulties to visualize the tangent
space $\TS_{(\VEC{u},\VEC{x})} S_c$, it may help to use the
presentation of tangent spaces given in Subsection~\ref{subSTSequiCC}.
Thus $\TS_{(\VEC{u},\VEC{x})} S_c$ is the set of all equivalent class
of differentiable curves $\sigma:]-a,a[\to S_c \subset \TS\, S$
at $(\VEC{u},\VEC{x}) \in S_c$.
It is now clear that $\TS_{(\VEC{u},\VEC{x})} S_c$ is a subspace of
$\TS_{(\VEC{u},\VEC{x})}(\TS S)$ which is the set
of all equivalent class of differentiable curves
$\sigma:]-a,a[\to \TS S$ at $(\VEC{u},\VEC{x}) \in \TS\, S$.

The definition of tangent spaces given in
Subsection~\ref{subSTSequiCC} is really nice to visualize tangent
spaces.  However, it is not really convenient to prove smoothness of
tangent spaces.
\end{rmk}

\begin{prop} \label{proEtaPrtps}
Let $S$ be a $2$-dimensional Riemannian manifold.
\begin{enumerate}
\item $\eta_\ast (H(\VEC{u},\VEC{x})) = H(\eta(\VEC{u},\VEC{x}))$
for all $(\VEC{u},\VEC{x}) \in S_c$ and $\displaystyle \eta \in S^1$.
\item $\eta_\ast$ is invertible and
$\displaystyle (\eta_\ast)^{-1} = (\eta^{-1})_\ast$
for all $\displaystyle \eta \in S^1$.
\end{enumerate}
\end{prop}

\begin{proof}
\stage{1} Let $(W,U,\phi)$, $(\tilde{W},\tilde{U},\tilde{\phi})$ and 
$(\breve{W},\breve{U}, \breve{\phi})$ be the previously defined local charts
of $S$, $S_c$ and $\TS S_c$ respectively.

Suppose that $\displaystyle \eta = e^{i\mu}$ for $\mu \in \RR$.
Since $\eta : S_c \to S_c$, we get that $\eta_\ast : \TS\, S_c \to \TS\, S_c$.  

Given $(\VEC{u},\VEC{x}) \in \tilde{U} \subset S_c$, we have
that $(\VEC{u},\VEC{x}) = \tilde{\phi}(\VEC{w},\theta)$ for some
$(\VEC{w},\theta) \in \tilde{W}$; namely,
$(\VEC{u},\VEC{x}) = (\phi(\VEC{w}), \theta.E(\VEC{w}))$.

Since $\eta \circ \tilde{\phi} = \tilde{\phi} \circ \eta$ where the
first $\eta$ acts on $\tilde{U}$ and the second on $\tilde{W}$, we get
that $\eta_\ast \circ \tilde{\phi}_\ast = \tilde{\phi}_\ast \circ \eta_\ast$
where the first $\eta_\ast$ acts on $\TS\, \tilde{U}$ and the
second one acts on $\TS\, \tilde{W}$.

Since the action of $\eta$ on $\tilde{W}$ gives
$\eta(\VEC{w},\theta) = (\VEC{w},\theta + \mu)$ \footnote{We
may assume that $0 \leq \mu < 2\pi$ and that the open interval $I$
in the definition of $\tilde{W}$ is such that $\theta, \mu \in I$.},
we have
\[
\eta_\ast\big( (\VEC{w},\theta), (\VEC{r},s)\big)
= \left( \eta(\VEC{w},\theta) , \diff_{\VEC{w},\theta} \eta(\VEC{w},\theta)
\, \begin{pmatrix} \VEC{r} \\ s \end{pmatrix} \right)
= \big( (\VEC{w},\theta+ \mu) , (\VEC{r}, s) \big)
\]
for all $\big((\VEC{w},\theta),(\VEC{r},s)\big) \in
\TS_{(\VEC{w},\theta)} \tilde{W}$.

It then follows from (\ref{HpisAstEq}) that
\begin{align*}
\eta_\ast\big( H(\VEC{u},\VEC{x}) \big)
&= \eta_\ast \big(\tilde{\phi}_\ast\big((\VEC{w},\theta),(\VEC{0},1)\big)\big)
= \tilde{\phi}_\ast\big(\eta_\ast\big((\VEC{w},\theta),(\VEC{0},1)\big)\big) \\
&= \tilde{\phi}_\ast\big((\VEC{w},\mu+\theta),(\VEC{0},1)\big)
= \tilde{\phi}_\ast\big( \eta(\VEC{w},\theta),(\VEC{0},1)\big)
= H(\eta(\VEC{u},\VEC{x}))
\end{align*}
because $\tilde{\phi}(\eta(\VEC{w},\theta)) = 
\eta(\tilde{\phi}(\VEC{w},\theta)) = \eta(\VEC{u},\VEC{x})$.

\stage{2} Since $\eta : S_c \to S_c$ and
$\displaystyle \eta^{-1}:S_c \to  S_c$, we
have that $\eta_\ast: \TS\,S_c \to \TS\,S_c$ and
$\displaystyle (\eta^{-1})_\ast: \TS\,S_c \to \TS\,S_c$.  Moreover, since
$\displaystyle \eta \circ \eta^{-1} = \Id_{S_c}$ and
$\displaystyle \eta^{-1} \circ \eta = \Id_{S_c}$, we have that
$\displaystyle \eta_\ast \circ (\eta^{-1})_\ast = (\eta \circ \eta^{-1})_\ast
= (\Id_{S_c})_\ast = \Id_{\TS\,S_c}$ and
$\displaystyle (\eta^{-1})_\ast \circ \eta_\ast = (\eta^{-1} \circ \eta)_\ast
= (\Id_{S_c})_\ast = \Id_{\TS\,S_c}$.  Thus
$\eta_\ast$ is invertible and
$\displaystyle (\eta_\ast)^{-1} = (\eta^{-1})_\ast$.
\end{proof}

\begin{rmk}
Suppose that            \label{rmkEtaProps}
$(W,U,\phi)$, $(\tilde{W},\tilde{U},\tilde{\phi})$ and 
$(\breve{W},\breve{U}, \breve{\phi})$ are the previously defined local
charts of $S$, $S_c$ and $\TS S_c$
respectively.  Moreover, suppose that
$\displaystyle \eta = e^{i\mu}$ for $\mu \in \RR$.

If $\big((\VEC{u},\VEC{x}),(\VEC{p},\VEC{q})\big)
= \breve{\phi}\big((\VEC{w},\theta),(\VEC{r},t)\big)
= \tilde{\phi}_\ast\big((\VEC{w},\theta),(\VEC{r},t)\big)$
for
$\big((\VEC{w},\theta),(\VEC{r},t)\big) \in \TS_{(\VEC{w},\theta)} \tilde{W}$.
then
\begin{align*}
& \eta_\ast\big( (\VEC{w},\VEC{x}),(\VEC{p} , \VEC{q}) \big)
= \eta_\ast \big(\breve{\phi}\big((\VEC{w},\theta),(\VEC{r},t)\big)
= \eta_\ast \big(\tilde{\phi}_\ast\big((\VEC{w},\theta),(\VEC{r},t)\big)\big)
\\
&\qquad = \tilde{\phi}_\ast
\big(\eta_\ast\big( (\VEC{w},\theta),(\VEC{r},t)\big)\big)
= \tilde{\phi}_\ast\big( \eta(\VEC{w},\theta),(\VEC{r},t)\big) \\
&\qquad = \left( \tilde{\phi}(\eta(\VEC{w},\theta)),
\diff_{\VEC{v},\nu} \tilde{\phi}(\VEC{v},\nu)
\big|_{(\VEC{v},\nu) = \eta(\VEC{w},\theta)}
\, \begin{pmatrix} \VEC{r} \\ t \end{pmatrix} \right) \\
&\qquad = \Big( \tilde{\phi}(\eta(\VEC{w},\theta)),
\Big( \diff \phi(\VEC{w})\,\VEC{r},
\diff_{\VEC{w}} \big((\mu+\theta).E(\VEC{w})\big) \, \VEC{r}
+ t\, (\pi/2 + \mu + \theta).E(\VEC{w}) \Big)\Big) \\
&\qquad = \big( \eta(\VEC{u},\VEC{x}) , (\VEC{p},\tilde{\VEC{q}}) \big) 
\end{align*}
for some $\tilde{\VEC{q}}$.  This simple remark will be useful later.
\end{rmk}

\begin{prop} \label{profastMap}
If $S_1$ and $S_2$ are two $2$-dimensional Riemannian
manifolds and $f:S_1 \to S_2$ is an isometry, then
$f_\ast: (S_1)_c \to (S_2)_c$.
\end{prop}

\begin{proof}
Since $f$ is an isometry, we have that
$\|f_\ast(\VEC{u},\VEC{x})\|_{f(\VEC{u})} =
\|(\VEC{u},\VEC{x})\|_{\VEC{u}}$ for all
$(\VEC{u},\VEC{x}) \in \TS_{\VEC{u}} S_1$.  In particular, if
$(\VEC{u},\VEC{x}) \in (S_1)_c(\VEC{u})$, then
$f_\ast(\VEC{u},\VEC{x}) \in (S_2)_c(f(\VEC{u}))$.  Thus
$f_\ast:(S_1)_c(\VEC{u}) \to (S_2)_c(f(\VEC{u}))$ for all
$\VEC{u} \in S_1$.
\end{proof}

\begin{prop} \label{proFPrpts}
Let $S_1$ and $S_2$ be two $2$-dimensional oriented and Riemannian
manifold, and $f:S_1 \to S_2$ be an orientation preserving 
isometry.  Then $\tilde{f}_\ast \circ H = H \circ \tilde{f}$ on $(S_1)_c$
where $\tilde{f} = f_\ast$.
\end{prop}

\begin{proof}
Let $(W_i,U_i,\phi_i)$, $(\tilde{W}_i,\tilde{U}_i,\tilde{\phi}_i)$ and 
$(\breve{W}_i,\breve{U}_i, \breve{\phi}_i)$ be the previously defined
local charts of $S_i$, $(S_i)_c$ and $\TS (S_i)_c$ respectively for
$i =1,2$.  We may assume that $f(U_1) \subset U_2$.  Suppose that
$g:W_1 \to W_2$ is the local representation of $f:U_1 \to U_2$;
namely, $f\circ \phi_1 = \phi_2 \circ g$ on $W_1$.

Suppose that
$(\VEC{u},\VEC{x}) = \tilde{\phi}_1(\VEC{w},\theta)$ for some $\theta \in \RR$.
Since $\tilde{f} \circ \tilde{\phi}_1 = \tilde{\phi}_2 \circ \tilde{g}$ where 
$\tilde{g} = g_\ast : \tilde{W}_1 \to \tilde{W}_2$ is the local representation
of $\tilde{f}$, we get that
$\tilde{f}_\ast \circ (\tilde{\phi}_1)_\ast
= (\tilde{\phi}_2)_\ast \circ \tilde{g}_\ast$ on $\TS \tilde{W}_1$.

Since $f$ is an orientation preserving isometry, we have
that $\tilde{f} = f_\ast$ preserves the angle and direction of
rotation between vectors.  To be more precise, let
$\displaystyle \nu \in S^1$ be the angle
such that $\tilde{f}\big(\VEC{u},\tilde{E}(\VEC{u})\big)
= \nu\big(f(\VEC{u}), \tilde{E}(f(\VEC{u}))\big)$
where $\displaystyle \tilde{E}(\VEC{u}) = E(\phi^{-1}(\VEC{u}))$ for all
$\VEC{u} \in U_1$.  Then, we get from 
the third item of Proposition~\ref{proTSprpts} that
\[
\tilde{f}\big(\mu(\VEC{u},\tilde{E}(\VEC{u}))\big)
= \mu \big(\tilde{f}(\VEC{u},\tilde{E}(\VEC{u}))\big)
= \mu \big(\nu \big(f(\VEC{u}),\tilde{E}(f(\VEC{u})) \big)\big)
\]
for all $\displaystyle \mu \in S^1$.
If $\displaystyle \nu = e^{i \alpha}$ and $\displaystyle \mu = e^{i \beta}$,
then the previous statement yields
\begin{align*}
\tilde{f}\big(\tilde{\phi}_1(\VEC{w},\beta)\big)
&= \tilde{f}\big(\VEC{u},\beta.\tilde{E}(\VEC{u})\big)
= \big(f(\VEC{u}),(\alpha + \beta).\tilde{E}(f(\VEC{u}))\big) \\
&= \big( f(\phi_1(\VEC{w})),
(\alpha +\beta).\tilde{E}(f(\phi_1(\VEC{w})))\big)
= \tilde{\phi}_2( g(\VEC{w}), \alpha + \beta)
\end{align*}
for all $\beta \in \RR$.  Thus
$\tilde{g}(\VEC{w},\beta) = (g(\VEC{w}), \alpha + \beta)$ for all
$(\VEC{w},\beta) \in \tilde{W}$.
Hence $\displaystyle \pdydx{\tilde{g}}{\beta}(\VEC{w},\beta) = (\VEC{0},1)$
for all $\beta \in \RR$.
It then follows from (\ref{HpisAstEq}) that
\begin{align*}
&\tilde{f}_\ast\big( H(\VEC{u},\VEC{x}) \big)
= \tilde{f}_\ast \big((\tilde{\phi}_1)_\ast
\big((\VEC{w},\theta),(\VEC{0},1)\big)\big)
= (\tilde{\phi}_2)_\ast
\big(\tilde{g}_\ast\big((\VEC{w},\theta),(\VEC{0},1)\big)\big) \\
&\qquad = (\tilde{\phi}_2)_\ast\left( \tilde{g}(\VEC{w},\theta),
\diff \tilde{g}(\VEC{w},\theta)
\begin{pmatrix} \VEC{0} \\ 1 \end{pmatrix} \right)
= (\tilde{\phi}_2)_\ast\Big( \tilde{g}(\VEC{w},\theta),
\pdydx{\tilde{g}}{\theta}(\VEC{w},\theta) \Big) \\
&\qquad
= (\tilde{\phi}_2)_\ast\big( \tilde{g}(\VEC{w},\theta), (\VEC{0}, 1) \big)
= H(\tilde{f}(\VEC{u},\VEC{x})) \ ,
\end{align*}
because
$\tilde{f}(\VEC{u},\VEC{x}) = \tilde{\phi}_2(\tilde{g}(\VEC{w},\theta))$.
\end{proof}

\section{Parallel Translation and Connection}

We have a clear idea of what translation of a vector in
$\displaystyle \RR^n$ implies.  The direction of the vector does
not change and the length of the vector does not change.  The goal is
to generalize this concept of translation to the context of a
manifold.  It is unlikely that the spacial direction of a tangent
vector to a Riemannian manifold $S$ can be preserved when it is
translated along the manifold $S$.  However, we should certainly try
to preserve the length measured using the Riemann metric of a tangent
vector to $S$ when it is translated.
Moreover, we should also try to preserve the angle between two tangent
vectors at a point of $S$ when they are translated along $S$. 

Throughout this section, $S$ will be a $2$-dimensional Riemannian
manifold in $\displaystyle \RR^n$.

Suppose that $(\VEC{u},\VEC{x}) \in S_c(\VEC{u})$ and that
$\sigma:[a,b] \to S$ is a smooth curve such that
$\sigma(a) = \VEC{u}$.  We seek smooth curves
$\tilde{\sigma}_{(\VEC{u},\VEC{x})}:[a,b] \to S_c$
(Figure~\ref{ParTranslOne}) such that
\begin{enumerate}
\item $\tilde{\sigma}_{(\VEC{u},\VEC{x})}(a) = (\VEC{u},\VEC{x})$,
\item $\pi_S \circ \tilde{\sigma}_{(\VEC{u},\VEC{x})} = \sigma$ on $[a,b]$, and
\item $\tilde{\sigma}_{\eta(\VEC{u},\VEC{x})}(t)
= \eta(\tilde{\sigma}_{(\VEC{u},\VEC{x})}(t))$ for $t \in [a,b]$ and
$\displaystyle \eta \in S^1$ (Figure~\ref{ParTranslTwo}).
\end{enumerate}

The third condition means
that the map $(\VEC{u},\VEC{x}) \mapsto \tilde{\sigma}_{(\VEC{u},\VEC{x})}(t)$
commutes with the action of $\displaystyle S^1$ on $S_c$.
Note that $(\tilde{\sigma}_{(\VEC{u},\VEC{x})})_\ast(t,1) \in
\TS_{\tilde{\sigma}_{(\VEC{u},\VEC{x})}(t)} S_c$ and it follows from
(2) that
\[
(\pi_S)_\ast\big((\tilde{\sigma}_{(\VEC{u},\VEC{x})})_\ast(t,1)\big)
= (\pi_S \circ \tilde{\sigma}_{(\VEC{u},\VEC{x})})_\ast((t,1)
= \sigma_\ast(t,1) = (\sigma(t),\sigma'(t))
\]
for $t \in [a,b]$.
The vector space $\TS_{\tilde{\sigma}_{(\VEC{u},\VEC{x})}(t)} S_c$ is a
$3$-dimensional subspace of the $4$-dimensional vector space
$\TS_{\tilde{\sigma}_{(\VEC{u},\VEC{x})}(t)} (\TS\, S)$.

\pdfF{riemann_geom/partransl1}{First schematic representation of parallel
translation}{This is a schematic representation of parallel
translation.  We have that $\tilde{\sigma}_{(\VEC{u},\VEC{x})}(t)
\in S_c(\sigma(t)) \subset \TS_{\sigma(t)} S$ for $a \leq t \leq b$.
Elements of the form $\tilde{\sigma}_{(\VEC{u},\VEC{x})}(t) \in S_c(\sigma(t))$
are represented by points on the unit circle
$C \subset \TS_{\sigma(t)} S$ (in blue) centred at the origin which is
associated to $\sigma(t)$ in the figure above.} 
{ParTranslOne}

\begin{defn} \label{defnLiftId}
A curve $\tilde{\sigma}_{(\VEC{u},\VEC{x})}:[a,b] \to S_c$ that
satisfies (2) above is called a {\bfseries lift}\index{Lift} of $\sigma$.
\end{defn}

\begin{defn}
The {\bfseries parallel translation}\index{Parallel Translation} of the
unit vector $(\VEC{u},\VEC{x}) \in S_c(\VEC{u})$ along $\sigma$ to
$\sigma(t)$ with $a \leq t \leq b$ is defined as the vector
$\tilde{\sigma}_{(\VEC{u},\VEC{x})}(t) \in S_c(\sigma(t))$.
\end{defn}

\pdfF{riemann_geom/partransl2}{Angle under parallel translation}{This
is a schematic representation of parallel translation that aims to
illustrate the effect of the translation on the angle between two
vectors in $S_c$.  In the figure $\displaystyle \eta = e^{i\theta} \in S^1$.
As in the previous figure, elements of the form
$\tilde{\sigma}_{(\VEC{u},\VEC{x})}(t) \in S_c(\sigma(t))$ are represented by
points on the unit circle $C \subset \TS_{\sigma(t)} S$ (in blue) 
centred at the origin which is associated to $\sigma(t)$in the figure above.}
{ParTranslTwo}

However, the three conditions listed above are not enough to determine
uniquely $\tilde{\sigma}_{(\VEC{u},\VEC{x})}$.  To justify this
statement, we need a little lemma.

\begin{lemma}
Given $(\VEC{u},\VEC{x}) \in S_c(\VEC{u})$, the space
$\displaystyle P_{(\VEC{u},\VEC{x})}
= ((\pi_S)_\ast)^{-1}\big(\{(\VEC{u},\VEC{0})\}\big) \cap
\TS_{(\VEC{u},\VEC{x})} S_c$
is a $1$ dimensional subspace of $\TS_{(\VEC{u},\VEC{x})} S_c$.
\end{lemma}

\begin{proof}
We assume that $(\VEC{u},\VEC{x}) = \eta(\VEC{u},\tilde{E}(\VEC{u}))
= (\VEC{u}, \theta.\tilde{E}(\VEC{u}))$ for some
$\displaystyle \eta = e^{i\theta} \in S^1$.

Consider the vector field $H$ on $S_c$ defined in (\ref{defnHonSc}).
We have that $H(\VEC{u},\VEC{x}) \in \TS_{\VEC{u},\VEC{x}} S_c$,
$H(\VEC{u},\VEC{x}) = \big( (\VEC{u},\VEC{x}), (\VEC{0},\VEC{q}) \big)$
with $\VEC{q} \neq \VEC{0}$, and
$(\pi_S)_\ast( H(\VEC{u},\VEC{x}) ) = (\VEC{u},\VEC{0})$.
Thus $P_{(\VEC{u},\VEC{x})}$ is of dimension at least $1$.
The space $P_{(\VEC{u},\VEC{x})}$ is in fact of dimension $1$ because
\[
(\pi_S)_\ast\big( (\tilde{\sigma}_i)_\ast(0,1)  \big) 
= \big(\VEC{u}, \diff \phi(\VEC{w}) \rho_i'(0)\big)
\neq (\VEC{u},\VEC{0})
\]
for $i = 1,2$, where
$\{ (\tilde{\sigma}_i)_\ast(0,1) \}_{1\leq i \leq 3}$
is the basis of $\TS_{(\VEC{u},\VEC{x})} S_c$ introduced in
Remark~\ref{RMKsubsectTSc}.  The reader
may recall that
$(\tilde{\sigma}_3)_\ast(0,1) = H(\tilde{\sigma}_3(0))
= H(\VEC{u},\VEC{x})$.

Therefore
$\displaystyle P_{(\VEC{u},\VEC{x})}$ is the $1$-dimensional subspace of
$\TS_{(\VEC{u},\VEC{x})} S_c$ given by
\begin{equation} \label{defnPux}
P_{(\VEC{u},\VEC{x})} = \{ t H(\VEC{u},\VEC{x}) : t \in \RR \}
= \Big\{\Big( (\VEC{u},\VEC{x}), t\Big( \VEC{0},
\dfdx{(\nu.\tilde{E}(\VEC{u}))}{\nu}\Big|_{\nu=\theta}
 \Big)\Big) : t \in \RR\Big\}
\end{equation}
(Figure~\ref{ParTranslThree}).
\end{proof}

\pdfF{riemann_geom/partransl3}{Schematic representation of connection}
{This is a schematic representation of a connection.  We have
projected the subspaces $\displaystyle \TS_{\VEC{u}}S$,
$\displaystyle M_{(\VEC{u},\VEC{x})}$ and
$P_{(\VEC{u},\VEC{x})}$ into $\displaystyle \RR^n$.  We also have
projected $H(\VEC{u},\VEC{x})$ into $\displaystyle \RR^n$.  Normally,
they all live in different spaces.  We have added the potential direction
$\big(\tilde{\sigma}_{(\VEC{u},\VEC{x})}\big)_\ast(a,1)
\in M_{(\VEC{u},\VEC{x})}$
of the lift $\tilde{\sigma}_{(\VEC{u},\VEC{x})}$ of $\sigma$ at $t=a$.
As usual, $(\VEC{u},\VEC{x})$ is represented by the point
$\VEC{u} + \VEC{x}$ on the unit circle $C$ (in black) in the
tangent plane $\TS_{\VEC{u}} S$ (in grey) centred at the origin which in the
figure above is associated to $\VEC{u}$.  Be aware that
$P_{(\VEC{u},\VEC{x})}$ does not generally represent a normal
direction to $M_{(\VEC{u},\VEC{x})}$ as it seems to be in the figure.
We only can say that $P_{(\VEC{u},\VEC{x})}$ is perpendicular to
$M_{(\VEC{u},\VEC{x})} \cap \TS_{(\VEC{u},\VEC{x})} S$.}{ParTranslThree}

Hence, if
$(\pi_S)_\ast\big((\tilde{\sigma}_{(\VEC{u},\VEC{x})})_\ast(t,1)\big)
= \sigma_\ast(t,1)$, then
$(\pi_S)_\ast\big(\big(\tilde{\sigma}_{(\VEC{u},\VEC{x})}(t),
(\VEC{p},\VEC{q})\big) + (\tilde{\sigma}_{(\VEC{u},\VEC{x})})_\ast(t,1)\big)
= \sigma_\ast(t,1)$ for all
$\big(\tilde{\sigma}_{(\VEC{u},\VEC{x})}(t),(\VEC{p},\VEC{q})\big) \in
P_{\tilde{\sigma}_{(\VEC{u},\VEC{x})}(t)}$
because $(\pi_S)_\ast\big(\tilde{\sigma}_{(\VEC{u},\VEC{x})}(t),
(\VEC{p},\VEC{q})\big) = (\sigma(t),0)$.
Therefore, there is a $1$-dimensional subspace of the
$3$-dimensional vector space
$\TS_{\tilde{\sigma}_{(\VEC{u},\VEC{x})}(t)} S_c$ such that the image
of this subspace by the projection $(\pi_S)_\ast : 
\TS_{\tilde{\sigma}_{(\VEC{u},\VEC{x})}(t)} S_c \to \TS_{\sigma(t)} S$ is 
$\{ \sigma_\ast(t,1) \}$.  We need to add conditions on the lift associated
to a parallel translations to uniquely determine the value of
$(\tilde{\sigma}_{(\VEC{u},\VEC{x})})_\ast(t,1)$.

\begin{defn}  \label{defnConnection}
Let $S$ be a $2$-dimensional Riemannian manifold. A
{\bfseries connection}\index{Connection} on $S_c$ is a collection
$\displaystyle \MM_c = \{ M_{(\VEC{u},\VEC{x})}\}_{(\VEC{u},\VEC{x}) \in S_c}$
such that
\begin{enumerate}
\item $M_{(\VEC{u},\VEC{x})}$ is a $2$-dimensional subspace of
$\TS_{(\VEC{u},\VEC{x})} S_c$ for all $(\VEC{u},\VEC{x}) \in S_c$,
\item $\displaystyle \TS_{(\VEC{u},\VEC{x})} S_c = M_{(\VEC{u},\VEC{x})} \oplus
P_{(\VEC{u},\VEC{x})}$ for all $(\VEC{u},\VEC{x}) \in S_c$
(Figure~\ref{ParTranslThree}),
\item $\eta_\ast (M_{(\VEC{u},\VEC{x})})
= M_{\eta(\VEC{u},\VEC{x})}$ for all $(\VEC{u},\VEC{x}) \in S_c$
and $\displaystyle \eta \in S^1$ (Figure~\ref{ParTranslFour}), and
\item $S_c$ can be covered by open sets $\tilde{U} \subset S_c$ such that
there exist two smooth vector fields $F,G :\tilde{U} \to \TS S_c$ with
the property that $\{ F(\VEC{u},\VEC{x}), G(\VEC{u},\VEC{x}) \}$ is a basis of
$M_{(\VEC{u},\VEC{x})}$ for all $(\VEC{u},\VEC{x}) \in \tilde{U}$.
\end{enumerate}
\end{defn}

In the previous definition, we emphasize that
$\eta :S_c \to S_c$ for all fixed $\displaystyle \eta \in S^1$.
Thus $\eta_\ast: \TS S_c \to \TS S_c$.

\pdfF{riemann_geom/partransl4}{Schematic representation of the
rotation of $\TS_{(\VEC{u},\VEC{x})} S_c$}{Given $(\VEC{u},\VEC{x})
\in S_c$, we give a schematic representation of the
rotation of $H(\VEC{u},\VEC{x})$ and $M_{(\VEC{u},\VEC{x})}$
by an angle $\displaystyle \eta \in S^1$.}{ParTranslFour}

\begin{egg}
It is not hard to define a connection locally on $S_c$.  \label{eggConnPAp1}

Suppose that $(W,U,\phi)$ is local chart of $S$.  Consider the map
$\displaystyle G:U\times S^1 \to S_c(U)$ defined by
$G(\VEC{u},\eta) = \eta\big(\VEC{u}, \tilde{E}(\VEC{u}) \big)
= \big(\VEC{u} , \theta.\tilde{E}(\VEC{u}) \big)$
for $(\VEC{u},\theta) \in U \times \RR$ and
$\displaystyle \eta = e^{i\theta}$, where
$\displaystyle \tilde{E}(\VEC{u}) = E(\phi^{-1}(\VEC{u}))$ for all
$\VEC{u} \in U$.  This is the famous map that we have used to prove that
$\displaystyle S_c(U) \cong U \times S^1$ in Remark~\ref{rmkUS1eScU}.

For $\eta$ fixed, we have that
$(G_\eta)_\ast: \TS_{\VEC{u}} S \to
\TS_{\eta(\VEC{u}, \tilde{E}(\VEC{u}))} S_c$.
Let $\displaystyle \check{M}_{\eta(\VEC{u},\tilde{E}(\VEC{u}))}
= (G_\eta)_\ast\big(\TS_{\VEC{u}}S\big)$.
Since $(\pi_S)_\ast\circ (G_\eta)_\ast
= (\pi_S \circ G_\eta)_\ast = (\Id_S)_\ast$, we get that
$(\pi_S)_\ast
\big(\check{M}_{\eta(\VEC{u},\tilde{E}(\VEC{u}))}\big) = \TS_{\VEC{u}} S$
for $\VEC{u} \in U$.  Moreover
$(\pi_S)_\ast\big(H(\eta(\VEC{u},\tilde{E}(\VEC{u})))\big)
= (\VEC{u},\VEC{0})$ for all $\VEC{u} \in U$.
Thus
$(\pi_S)_\ast:\check{M}_{\eta(\VEC{u},\tilde{E}(\VEC{u}))} \to \TS_{\VEC{u}} S$
is a (linear) isomorphism and so
$\displaystyle \check{M}_{\eta(\VEC{u},\tilde{E}(\VEC{u}))}$ is of dimension $2$.
It follows that
$\TS_{\eta(\VEC{u},\tilde{E}(\VEC{u}))} = \check{M}_{\eta(\VEC{u},\tilde{E}(\VEC{u}))}
\oplus P_{\eta(\VEC{u},\tilde{E}(\VEC{u}))}$.

We have that
$\displaystyle \check{M}_{\eta(\VEC{u},\tilde{E}(\VEC{u}))}
= \eta_\ast \big(\check{M}_{(\VEC{u},\tilde{E}(\VEC{u}))}\big)$
for $\displaystyle \eta \in S^1$ because $G_\eta = \eta \circ G_1$
We may define
$\displaystyle \check{\MM}_c
= \{ \check{M}_{(\VEC{u},\VEC{x})}\}_{(\VEC{u},\VEC{x}) \in S_c(U)}$
because every $(\VEC{u},\VEC{x}) \in S_c(U)$ is of the form
$\eta\big(\VEC{u},\tilde{E}(\VEC{u})\big)$ for some
$\displaystyle \eta \in S^1$.

The existence of two smooth vector fields $P,Q :S_c(U) \to \TS S_c$ with
the property that $\{ P(\VEC{u},\VEC{x}), Q(\VEC{u},\VEC{x}) \}$ is a basis of
$\check{M}_{(\VEC{u},\VEC{x})}$ for all $(\VEC{u},\VEC{x}) \in S_c(U)$ follows
from the isomorphism between
$\check{M}_{(\VEC{u},\VEC{x})}$ and $\TS_{\VEC{u}} S$
for $(\VEC{u},\VEC{x}) \in S_c(U)$.  Two
obvious vector fields $\tilde{P},\tilde{Q} :U \to \TS S$ with the property that
$\{ \tilde{P}(\VEC{u}), \tilde{Q}(\VEC{u}) \}$ is a basis of
$\TS_{\VEC{u}} S$ for all $\VEC{u} \in U$ are given by
$\tilde{P}(\VEC{u}) = \big(\VEC{u}, \tilde{E}(\VEC{u}) \big)$
and
$\tilde{Q}(\VEC{u}) = \big(\VEC{u}, (\pi/2).\tilde{E}(\VEC{u}) \big)$
for $\VEC{u} \in U$.  We may use
$P\big( \eta(\VEC{u},\tilde{E}(\VEC{u}))\big)
= (G_\eta)_\ast(\tilde{P}(\VEC{u}))$
and
$Q\big(\eta(\VEC{u},\tilde{E}(\VEC{u}))\big)
= (G_\eta)_\ast(\tilde{Q}(\VEC{u}))$
for $\VEC{u} \in U$ and $\displaystyle \eta \in S^1$.
\end{egg}

The following concept is fundamental in the proof of existence of
connections.

\begin{prop} \label{thmConn1form}
Let $S$ be a $2$-dimensional Riemannian manifold and
$\MM_c = \displaystyle \{ M_{(\VEC{u},\VEC{x})}\}_{(\VEC{u},\VEC{x})
\in S_c} \}$
be a connection on $S_c$.  There exists a smooth differential $1$-form $\rho$
on $S_c$ such that
\begin{equation} \label{defnConnRho}
\pi_{P_{(\VEC{u},\VEC{x})}}\big((\VEC{u},\VEC{x}),(\VEC{p},\VEC{q})\big)
= \rho(\VEC{u},\VEC{x})
\big((\VEC{u},\VEC{x}),(\VEC{p},\VEC{q})\big) H(\VEC{u},\VEC{x})
\end{equation}
for all $\big((\VEC{u},\VEC{x}),(\VEC{p},\VEC{q})\big) \in
\TS S_c$, where $\pi_{P_{(\VEC{u},\VEC{x})}}$ is the
projection from $M_{(\VEC{u},\VEC{x})} \oplus P_{(\VEC{u},\VEC{x})}$ onto
$P_{(\VEC{u},\VEC{x})}$ with $P_{(\VEC{u},\VEC{x})}$ defined in
(\ref{defnPux}).
The smooth differential $1$-form $\rho$ is called the
{\bfseries connection $\mathbf{1}$-form}\index{Connection $1$-form}
associated to the connection $\MM_c$.
\end{prop}

\begin{proof}
Every element $\displaystyle \big((\VEC{u},\VEC{x}), (\VEC{p},\VEC{q})\big)
\in \TS_{(\VEC{u},\VEC{x})} S_c$ can be expressed in an unique way as
\[
\big((\VEC{u},\VEC{x}), (\VEC{p},\VEC{q})\big)
= \lambda\big((\VEC{u},\VEC{x}),(\VEC{p},\VEC{q})\big) H(\VEC{u},\VEC{x})
+ \big( (\VEC{u},\VEC{x}), (\tilde{\VEC{p}},\tilde{\VEC{q}})\big)
\]
for some
$\big( (\VEC{u},\VEC{x}), (\tilde{\VEC{p}},\tilde{\VEC{q}})\big)
\in M_{(\VEC{u},\VEC{x})}$ and
$\lambda\big((\VEC{u},\VEC{x}),(\VEC{p},\VEC{q})\big) \in \RR$.

We set $\rho(\VEC{u},\VEC{x})
\big((\VEC{u},\VEC{x}),(\VEC{p},\VEC{q})\big) = 
\lambda\big((\VEC{u},\VEC{x}),(\VEC{p},\VEC{q})\big)$ for all
$(\VEC{u},\VEC{x}) \in S_c$ and
$\big((\VEC{u},\VEC{x}),(\VEC{p},\VEC{q})\big) \in
\TS_{(\VEC{u},\VEC{x})} S_c$.

To prove that $\rho$ is a smooth differential $1$-form, we consider an
open set $\tilde{U} \subset S_c$ such that there exist two smooth vector
fields $F,G :\tilde{U} \to \TS S_c$ with the property that \\
$\{ F(\VEC{v},\VEC{x}), G(\VEC{u},\VEC{x}) \}$ is a basis of
$M_{(\VEC{u},\VEC{x})}$ for all $(\VEC{u},\VEC{x}) \in \tilde{U}$.
Since
$\displaystyle \TS_{(\VEC{u},\VEC{x})} S_c = M_{(\VEC{u},\VEC{x})} \oplus
P_{(\VEC{u},\VEC{x})}$ for all $(\VEC{u},\VEC{x}) \in \tilde{U}$, we have that
$\{ F(\VEC{v},\VEC{x}), G(\VEC{u},\VEC{x}) , H(\VEC{u},\VEC{x})\}$ is
a basis of $\TS_{(\VEC{u},\VEC{x})} S_c$ for all
$(\VEC{u},\VEC{x}) \in \tilde{U}$.
Let $\{ \tilde{F}(\VEC{v},\VEC{x}), \tilde{G}(\VEC{u},\VEC{x}),
\tilde{H}(\VEC{u},\VEC{x})\}$ be the dual basis associated to\\
$\{ F(\VEC{v},\VEC{x}), G(\VEC{u},\VEC{x}) , H(\VEC{u},\VEC{x})\}$
for all $(\VEC{u},\VEC{x}) \in \tilde{U}$.
We have by construction that $\tilde{F}$, $\tilde{G}$ and $\tilde{H}$
are smooth differential $1$-forms on $\tilde{U}$.

Since $\tilde{H}(\VEC{u},\VEC{x})$ is null on 
$M_{(\VEC{u},\VEC{x})}$ and
$\tilde{H}((\VEC{u},\VEC{x})\big( H(\VEC{u},\VEC{x}) \big) = 1$, we
get
\[
\tilde{H}(\VEC{u},\VEC{x})\big( (\VEC{u},\VEC{x}), (\VEC{p},\VEC{q})\big)
= \rho(\VEC{u},\VEC{x})\big((\VEC{u},\VEC{x}),(\VEC{p},\VEC{q})\big)
\]
for all $(\VEC{u},\VEC{x}) \in \tilde{U}$ and
$\displaystyle \big((\VEC{u},\VEC{x}), (\VEC{p},\VEC{q})\big) \in
\TS_{(\VEC{u},\VEC{x})} \tilde{U}$.  Thus $\rho = \tilde{H}$ on $\tilde{U}$.
Since $\tilde{U}$ is arbitrary, this proves that $\rho$ is a smooth
differential $1$-form on $S_c$.
\end{proof}

\begin{egg}[Example~\ref{eggConnPAp1} (Continued)]
According to the previous proposition,      \label{eggConnPAp2}
the connection $1$-form $\check{\rho}$
on $S_c(U)$ satisfies (\ref{defnConnRho}) with $\rho$ replaced by
$\check{\rho}$ and $(\VEC{u},\VEC{x}) = \eta.(\VEC{u},\tilde{E}(\VEC{u}))$ 
for $\displaystyle \eta \in S^1$, where
$\displaystyle \tilde{E}(\VEC{u}) = E(\phi^{-1}(\VEC{u}))$ for all
$\VEC{u} \in U$.

We consider
$\displaystyle S^1 \cong \{ e^{i\theta} \in \CC : \theta \in \RR \}$.
We have that
$\displaystyle \TS_{\eta} S^1 = \{ (\eta, t\, i \eta) : t \in \RR \}$
for all $\displaystyle \eta = e^{i\theta} \in S^1$.
We define a differential $1$-form $\omega$ on
$\displaystyle S^1$ by $\omega(\eta)(\eta, t\, i\eta) = t$
for all $\displaystyle \eta \in S^1$ and $t \in \RR$.

We now prove that the connection $1$-form $\check{\rho}$ associated to
$\check{\MM}_c$ is given by
$\displaystyle \check{\rho} = (G^{-1})^\ast(\pi_2^\ast(\omega))$ where
$\displaystyle \pi_2: U \times S^1 \to S^1$ is the
projection defined by $\pi_2(\VEC{u},\eta) = \eta$ for all
$\displaystyle (\VEC{u},\eta) \in U \times S^1$, and $G$ is the
function defined in Remark~\ref{rmkUS1eScU}.
Since $\displaystyle G:U \times S^1 \to S_c(U)$ is an isomorphism, we
have that $\displaystyle G_\ast:\TS\,(U\times S^1) \to \TS\, S_c(U)$ is an
isomorphism.  Therefore, given
$\big(\eta(\VEC{u},\tilde{E}(\VEC{u})),(\VEC{p},\VEC{q})\big)
\in \TS_{\eta(\VEC{u}, \tilde{E}(\VEC{u}))} S_c$ with
$\displaystyle \eta = e^{i\theta}$, there exists
$\displaystyle \big((\VEC{u},\eta),(\VEC{r},t\, i\eta)\big)
\in \TS_{(\VEC{u},\eta)}(U\times S^1)$  such that
\begin{align*}
\big(\eta(\VEC{u},\tilde{E}(\VEC{u})),(\VEC{p},\VEC{q})\big)
&= G_\ast\big((\VEC{u},\eta),(\VEC{r},t\, i\eta)\big)
= (G_\eta)_\ast(\VEC{u},\VEC{r})
+ (G_{\VEC{u}})_\ast(\eta,t\, i \eta) \\
&= (G_\eta)_\ast(\VEC{u},\VEC{r})
+ \Big( \eta(\VEC{u},\tilde{E}(\VEC{u})), 
\Big( \VEC{0}, t \dfdx{(\theta.\tilde{E}(\VEC{u}))}{\theta}\Big) \Big) \\
&= (G_\eta)_\ast(\VEC{u},\VEC{r})
+ t H(\eta(\VEC{u},\tilde{E}(\VEC{u})))
\end{align*}
where the second to last equality comes form the fact 
that $\displaystyle \pdfdx{G_{\VEC{u}}(\eta)}{\eta}
= \pdfdx{G(\VEC{u},\eta)}{\eta}$ for $\VEC{u}$ constant is defined
with the help of the local coordinate $\displaystyle \eta = e^{i\theta}$ as
\[
\pdfdx{G(\VEC{u},\eta)}{\eta} =
\pdfdx{G(\VEC{u},e^{i\theta})}{\theta}
\left( \dfdx{e^{i\theta}}{\theta}\right)^{-1}
= \frac{1}{i\eta} \dfdx{\big(\VEC{u},\theta.\tilde{E}(\VEC{u})\big)}{\theta}
= \frac{1}{i\eta} \big( \VEC{0} ,\dfdx{\theta.\tilde{E}(\VEC{u})}{\theta}\big)
\ .
\]
Hence
\begin{align*}
&\check{\rho}\big(\eta(\VEC{u},\tilde{E}(\VEC{u}))\big)
\big(\eta(\VEC{u},\tilde{E}(\VEC{u})),(\VEC{p},\VEC{q})\big)
=
(G^{-1})^\ast(\pi_2^\ast(\omega)) \big(\eta(\VEC{u},\tilde{E}(\VEC{u}))\big)
\big(\eta(\VEC{u},\tilde{E}(\VEC{u})),(\VEC{p},\VEC{q})\big) \\
&\qquad =
\pi_2^\ast(\omega)(\VEC{u},\eta)\big((\VEC{u},\eta),(\VEC{r},t\, i\eta)\big)
= \omega\big(\pi_2(\VEC{u},\eta)\big)
\big((\pi_2)_\ast\big((\VEC{u},\eta), (\VEC{r},t\, i \eta)\big)\big) \\
&\qquad = \omega(\eta)(\eta, t\, i \eta) = t \ .
\end{align*}
We leave it to the reader to verify
that $(\pi_2)_\ast\big( (\VEC{u}, \eta),(\VEC{r},t\, i\eta)\big) =
(\eta,t\, i\eta)$ for all\\
$\displaystyle \big((\VEC{u}, \eta),(\VEC{r},t\, i \eta)\big) \in
\TS_{(\VEC{u},\eta)} (U \times S^1)$.
Since $(G_\eta)_\ast(\VEC{u},\VEC{r}) \in
\check{M}_{\eta(\VEC{u},\tilde{E}(\VEC{u}))}$, we get
\begin{align*}
&\pi_{P_{\eta(\VEC{u},\tilde{E}(\VEC{u}))}}
\big(\eta(\VEC{u},\tilde{E}(\VEC{u})),(\VEC{p},\VEC{q})\big)
= \pi_{P_{\eta(\VEC{u},\tilde{E}(\VEC{u}))}}
\big((G_\eta)_\ast(\VEC{u},\VEC{r}) \big) + t
\pi_{P_{\eta(\VEC{u},\tilde{E}(\VEC{u}))}}
\big(H(\eta(\VEC{u},\tilde{E}(\VEC{u})))\big)
\\
&\qquad = t \, H(\eta(\VEC{u},\tilde{E}(\VEC{u})))
= \check{\rho}\big(\eta(\VEC{u},\tilde{E}(\VEC{u}))\big)
\big(\eta(\VEC{u},\tilde{E}(\VEC{u})),(\VEC{p},\VEC{q})\big)
\, H(\eta(\VEC{u},\tilde{E}(\VEC{u})))
\end{align*}
for all
$\big(\eta(\VEC{u},\tilde{E}(\VEC{u})),(\VEC{p},\VEC{q})\big)
\in \TS\, S_c(U)$ as required in (\ref{defnConnRho}).

Note that $\df{\check{\rho}} = 0$ because
$\displaystyle \df{(G^{-1})^\ast(\pi_2^\ast(\omega))}
= (G^{-1})^\ast(\pi_2^\ast(\df{\omega}))$
according to Proposition~\ref{manifSDFitem4} and
$\df{\omega} = 0$ since there is no non-trivial differential
$2$-form on a $1$-dimensional manifold; in the present case, on
$\displaystyle S^1$.
\end{egg}

If $\rho$ is the connection $1$-form associated to a connection
$\displaystyle \MM_c = \{ M_{(\VEC{u},\VEC{x})}\}_{(\VEC{u},\VEC{x})}$,
then we get from the previous proposition that $\rho(\VEC{u},\VEC{x})$
is null on $M_{(\VEC{u},\VEC{x})}$ for all $(\VEC{u},\VEC{x})\in S_c$.
Moreover, $\rho(\VEC{u},\VEC{x}) \big( H(\VEC{u},\VEC{x}) \big) = 1$.
We can also prove the following lemma.

\begin{prop} \label{proERequR}
Let $S$ be a $2$-dimensional Riemannian manifold and
$\MM_c = \displaystyle \{ M_{(\VEC{u},\VEC{x})}\}_{(\VEC{u},\VEC{x})\in S_c} \}$
be a connection on $S_c$ where $\rho$ is the connection $1$-form
associated to $\MM_c$.  Then
$\displaystyle \eta^\ast(\rho)(\VEC{u},\VEC{x}) = \rho(\VEC{u},\VEC{x})$
for all $\displaystyle \eta\in S^1$ and $(\VEC{u},\VEC{x})\in S_c$.
\end{prop}

\begin{proof}
It follows from Proposition~\ref{proEtaPrtps} that
$\eta_\ast\big( H(\VEC{u},\VEC{x}) \big) = H(\eta(\VEC{u},\VEC{x}))$
for all $\displaystyle \eta \in S^1$.
Hence, since every element of $\TS\, S_c$ can be expressed as
$\displaystyle \big((\VEC{u},\VEC{x}), (\VEC{p},\VEC{q})\big)
= \lambda H(\VEC{u},\VEC{x})
+ \big( (\VEC{u},\VEC{x}), (\tilde{\VEC{p}},\tilde{\VEC{q}})\big)$
with $\lambda =
\rho(\VEC{u},\VEC{x})\big((\VEC{u},\VEC{x}),(\VEC{p},\VEC{q})\big)$
and
$\big( (\VEC{u},\VEC{x}), (\tilde{\VEC{p}},\tilde{\VEC{q}})\big)
\in M_{(\VEC{u},\VEC{x})}$, we get
\begin{align*}
\eta^\ast(\rho)(\VEC{u},\VEC{x})\big((\VEC{u},\VEC{x}), (\VEC{p},\VEC{q})\big)
&= \eta^\ast(\rho)(\VEC{u},\VEC{x})\big(\lambda H(\VEC{u},\VEC{x}) \big)
+ \eta^\ast(\rho)(\VEC{u},\VEC{x})
\big((\VEC{u},\VEC{x}),(\tilde{\VEC{p}},\tilde{\VEC{q}})\big) \\
&= \rho(\eta(\VEC{u},\VEC{x}))
\big( \lambda \eta_\ast\big(H(\VEC{u},\VEC{x})\big) \big)
= \rho(\eta(\VEC{u},\VEC{x}))
\big(\lambda H(\eta(\VEC{u},\VEC{x})) \big) \\
&= \lambda \rho(\eta(\VEC{u},\VEC{x})) \big(H(\eta(\VEC{u},\VEC{x}))\big)
= \lambda = \rho(\VEC{u},\VEC{x})\big((\VEC{u},\VEC{x}),(\VEC{p},\VEC{q})\big)
\end{align*}
because $\eta_\ast(M_{(\VEC{u},\VEC{x})}) = M_{\eta(\VEC{u},\VEC{x})}$
and $\rho = 0$ on $M_{(\VEC{u},\VEC{x})}$ for all $(\VEC{u},\VEC{x}) \in S_c$
implies that
\[
\eta^\ast(\rho)(\VEC{u},\VEC{x})
\big((\VEC{u},\VEC{x}),(\tilde{\VEC{p}},\tilde{\VEC{q}})\big)
= \rho\big(\eta(\VEC{u},\VEC{x})\big)\big(\eta_\ast
\big((\VEC{u},\VEC{x}),(\tilde{\VEC{p}},\tilde{\VEC{q}})\big)\big)
= 0
\]
since $\eta_\ast \big((\VEC{u},\VEC{x}),(\tilde{\VEC{p}},\tilde{\VEC{q}})\big)
\in M_{\eta(\VEC{u},\VEC{x})}$.
\end{proof}

There is kind of a converse to Theorem~\ref{thmConn1form}.

\begin{prop} \label{propCIfInv}
Let $S$ be a $2$-dimensional Riemannian manifold.  Suppose that $\rho$
is a smooth differential $1$-form on $S_c$ such that 
$\rho(\VEC{u},\VEC{x})( H(\VEC{u},\VEC{x}) ) = 1$ for all
$(\VEC{u},\VEC{x}) \in S_c$ and $\displaystyle \eta^\ast(\rho) = \rho$ for all
$\displaystyle \eta \in S^1$.  Them
$\displaystyle \MM_c = \{ M_{(\VEC{u},\VEC{x})}
\}_{(\VEC{u},\VEC{x})\in S_c}$ with
$\displaystyle M_{(\VEC{u},\VEC{x})} = (\rho(\VEC{u},\VEC{x}))^{-1}
\big(\big\{\big((\VEC{u},\VEC{x}), (\VEC{0},\VEC{0})\big)\big\}\big)$
for all $(\VEC{u},\VEC{x}) \in S_c$ is a connection on $S_c$ with the 
property that its connection $1$-form is $\rho$.
\end{prop}

\begin{proof}
For each $(\VEC{u},\VEC{y}) \in S_c$,
$\rho(\VEC{u},\VEC{x}):\TS_{(\VEC{u},\VEC{x})} S_c \to \RR$ is a
linear functional.  Since $\TS_{(\VEC{u},\VEC{x})} S_c$ is a
$3$-dimensional vector space and $\rho(H(\VEC{u},\VEC{x})) \neq 0$, we
have that
\[
M_{(\VEC{u},\VEC{x})} =
(\rho(\VEC{u},\VEC{x}))^{-1}
\big(\{((\VEC{u},\VEC{x}), (\VEC{0},\VEC{0}))\}\big)
\]
is a $2$-dimensional subspace of $\TS_{(\VEC{u},\VEC{x})} S_c$
satisfying (1) and (2) of Definition~\ref{defnConnection}.
Since $\displaystyle \eta^\ast(\rho) = \rho$, we get
\[
\rho(\eta(\VEC{u},\VEC{x}))
\big(\eta_\ast\big((\VEC{u},\VEC{x}),(\VEC{p},\VEC{q})\big)\big)
= \eta^\ast(\rho)(\VEC{u},\VEC{x})\big((\VEC{u},\VEC{x}),(\VEC{p},\VEC{q})\big)
= \rho(\VEC{u},\VEC{x})\big((\VEC{u},\VEC{x}),(\VEC{p},\VEC{q})\big) = 0
\]
for all $\big((\VEC{u},\VEC{x}),(\VEC{p},\VEC{q}) \in M_{(\VEC{u},\VEC{x})}$.
Thus $\eta_\ast\big((\VEC{u},\VEC{x}),(\VEC{p},\VEC{q})\big)
\in M_{\eta(\VEC{u},\VEC{x})}$ for
all $\big((\VEC{u},\VEC{x}),(\VEC{p},\VEC{q}) \in M_{(\VEC{u},\VEC{x})}$.
Hence (3) of Definition~\ref{defnConnection} is satisfied.

To prove (4) of Definition~\ref{defnConnection}, we use the fact that 
$\rho$ is a smooth differential $1$-form on $S_c$.  Suppose that 
$(\tilde{W},\tilde{U},\tilde{\phi})$ is a local chart of $S_c$ as in
Subsection~\ref{subsectTSc}.  The local representation of $\rho$ is
given by $\displaystyle \tilde{\phi}^\ast(\rho)$.
It suffices to construct two smooth vector fields
$\displaystyle \tilde{F},\tilde{G} : \tilde{W} \to \tilde{W} \times \RR^3$
such that
$\displaystyle
\tilde{\phi}^\ast(\rho)(\tilde{\VEC{w}})(\tilde{F}(\tilde{\VEC{w}}))
= \tilde{\phi}^\ast(\rho)(\tilde{\VEC{w}})(\tilde{G}(\tilde{\VEC{w}})) = 0$
and $\displaystyle \tilde{F}(\tilde{\VEC{w}}), \tilde{G}(\tilde{\VEC{w}})
\in \TS_{\tilde{\VEC{w}}} \tilde{W} = \{\tilde{\VEC{w}}\}\times \RR^3$
are linearly independent for all $\tilde{\VEC{w}} \in \tilde{W}$.
The vectors fields
$\displaystyle F = \tilde{\phi}_\ast \circ \tilde{F} \circ \tilde{\phi}^{-1}$
and $\displaystyle G
= \tilde{\phi}_\ast \circ \tilde{G} \circ \tilde{\phi}^{-1}$ on
$\tilde{U}$ will then satisfy (4) of Definition~\ref{defnConnection}.
Note that the kernel of $\displaystyle \tilde{\phi}^\ast(\rho)$ is of
dimension $2$ for each $\tilde{\VEC{w}} \in \tilde{W}$ because
$M_{\tilde{\phi}(\tilde{\VEC{w}})}$ is a subspace of dimension $2$.

We have that $\TS_{(\VEC{u},\VEC{x})} S_c = M_{(\VEC{u},\VEC{x})}
\oplus P_{(\VEC{u},\VEC{x})}$ because $H(\VEC{u},\VEC{x}) \not\in
\KE( \rho(\VEC{u},\VEC{x}) )$.  Thus every 
$\big((\VEC{u},\VEC{x}),(\VEC{p},\VEC{q})\big) \in
\TS_{(\VEC{u},\VEC{x})} S_c$ can be expressed in an unique way as
\[
\big((\VEC{u},\VEC{x}),(\VEC{p},\VEC{q})\big) =
\lambda\big((\VEC{u},\VEC{x}),(\VEC{p},\VEC{q})\big) H(\VEC{u},\VEC{x})
+ \big((\VEC{u},\VEC{x}),(\tilde{\VEC{p}},\tilde{\VEC{q}})\big)
\]
with $\lambda\big((\VEC{u},\VEC{x}),(\VEC{p},\VEC{q})\big)  \in \RR$
and $\big((\VEC{u},\VEC{x}),(\tilde{\VEC{p}},\tilde{\VEC{q}})\big) \in
M_{(\VEC{u},\VEC{x})}$.  Hence,
$\rho(\VEC{u},\VEC{x})\big((\VEC{u},\VEC{x}),(\VEC{p},\VEC{q})\big) = 
\lambda\big((\VEC{u},\VEC{x}),(\VEC{p},\VEC{q})\big)$ for all
$(\VEC{u},\VEC{x}) \in S_c$ and
$\big((\VEC{u},\VEC{x}),(\VEC{p},\VEC{q})\big) \in
\TS_{(\VEC{u},\VEC{x})} S_c$.   Therefore,
\[
\pi_{P_{(\VEC{u},\VEC{x})}}\big((\VEC{u},\VEC{x}),(\VEC{p},\VEC{q})\big)
= \rho(\VEC{u},\VEC{x})\big((\VEC{u},\VEC{x}),(\VEC{p},\VEC{q})\big)
H(\VEC{u},\VEC{x})
\]
for all $(\VEC{u},\VEC{x}) \in S_c$ and
$\big((\VEC{u},\VEC{x}),(\VEC{p},\VEC{q})\big) \in
\TS_{(\VEC{u},\VEC{x})} S_c$.
Thus $\rho$ is a connection $1$-form associated to $\MM_c$.
\end{proof}

We need a couple of preliminary results to prove that, given a
connection\\
$\displaystyle \MM_c = \{ M_{(\VEC{u},\VEC{x})}\}_{(\VEC{u},\VEC{x}) \in S_c}$
with an associate connection $1$-form, there exists a unique lift
$\tilde{\sigma}_{(\VEC{u},\VEC{x})}:[a,b] \to S_c$ such that
$(\tilde{\sigma}_{(\VEC{u},\VEC{x})})_\ast(t,1) \in
M_{\tilde{\sigma}_{(\VEC{u},\VEC{x})}(t)}$ for all $t \in [a,b]$.

\begin{prop} \label{prorPiAom}
Let $S$ be a $2$-dimensional Riemannian manifold.
Suppose that $\xi$ is a differential $1$-from on $S_c$ such that
$\displaystyle \xi(\VEC{u},\VEC{x})\big(H(\VEC{u},\VEC{x})\big) = 0$
for all $(\VEC{u},\VEC{x}) \in S_c$ and $\displaystyle \eta^\ast(\xi) = \xi$
for all $\displaystyle \eta \in S^1$.
Then $\displaystyle \xi = \pi_S^\ast(\omega)$ for
some smooth differential $1$-form $\omega$ on $S$.
\end{prop}

\begin{proof}
The differential $1$-form $\omega$ on $S$ is defined as follows.
Choose $(\VEC{u},\VEC{p}) \in S_c$ arbitrary but fixed.
For $(\VEC{u},\VEC{x}) \in \TS_{\VEC{u}} S$, we set
\[
\omega(\VEC{u})(\VEC{u},\VEC{x}) =
\xi(\VEC{u},\VEC{p})\big((\VEC{u},\VEC{p}),
(\VEC{x},\VEC{q})\big)
\]
for some $\big((\VEC{u},\VEC{p}),(\VEC{x},\VEC{q})\big) \in
\TS_{(\VEC{u},\VEC{p})} S_c$.  Such an element exists because
$(\pi_S)_\ast: \TS_{(\VEC{u},\VEC{p})} S_c \to \TS_{\VEC{u}} S$ is onto.
The differential $1$-form $\omega$ is well defined because the
definition is independent of the choice of
$\big((\VEC{u},\VEC{p}),(\VEC{x},\VEC{q})\big) \in
\TS_{(\VEC{u},\VEC{p})} S_c$.
If $\big((\VEC{u},\VEC{p}),(\VEC{x},\tilde{\VEC{q}})\big)$
is another element of $\TS_{(\VEC{u},\VEC{p})} S_c$, then
$\big((\VEC{u},\VEC{p}),(\VEC{x},\VEC{q})\big) -
\big((\VEC{u},\VEC{p}),(\VEC{x},\tilde{\VEC{q}})\big)
= \lambda H(\VEC{u},\VEC{p})$ for some $\lambda \in \RR$ because
$\big((\VEC{u},\VEC{p}),(\VEC{x},\VEC{q})\big) -
\big((\VEC{u},\VEC{p}),(\VEC{x},\tilde{\VEC{q}})\big)
\in P_{(\VEC{u},\VEC{p})}$.  Hence
\[
\xi(\VEC{u},\VEC{p})\big((\VEC{u},\VEC{p}),
(\VEC{x},\VEC{q})\big) -
\xi(\VEC{u},\VEC{p})\big((\VEC{u},\VEC{p}),
(\VEC{x},\tilde{\VEC{q}})\big)
= \lambda\, \xi(\VEC{u},\VEC{p}) (H(\VEC{u},\VEC{x})) = 0 \ .
\]

The differential $1$-form $\omega$ is also independent of the choice of
$(\VEC{u},\VEC{p}) \in S_c$.   Any other element of $S_c$ of
the form $(\VEC{u},\VEC{z}) \in S_c$ is given by
$(\VEC{u},\VEC{z}) = \eta(\VEC{u},\VEC{p})
= (\VEC{u}, \theta.\VEC{p})$ for some
$\displaystyle \eta = e^{i\theta} \in S^1$.
It follows from (2) of Proposition~\ref{proEtaPrtps} that
$\eta_\ast$ is invertible and $\displaystyle (\eta_\ast)^{-1} = (\eta^{-1})_\ast$
for all $\displaystyle \eta \in S^1$.  Hence, given
$\big((\VEC{u},\VEC{z}),(\VEC{x},\VEC{q})\big) \in
\TS_{(\VEC{u},\VEC{z})} S_c$, there
exists $\big((\VEC{u},\VEC{p}),(\VEC{x},\tilde{\VEC{q}})\big) 
\in \TS_{(\VEC{u},\VEC{p})} S_c$ such that
$\eta_\ast(\VEC{u},\VEC{p}),(\VEC{x},\tilde{\VEC{q}})\big)
= \big((\VEC{u},\VEC{z}),(\VEC{x},\VEC{q})\big)$ (see
Remark~\ref{rmkEtaProps}). Hence
\begin{align*}
\xi(\VEC{u},\VEC{z})\big((\VEC{u},\VEC{z}),(\VEC{x},\VEC{q})\big)
&= \xi(\eta(\VEC{u},\VEC{p}))\big(
\eta_\ast\big((\VEC{u},\VEC{p}),(\VEC{x},\tilde{\VEC{q}})\big)\big)
= \eta^\ast(\xi)(\VEC{u},\VEC{p})
\big((\VEC{u},\VEC{p}),(\VEC{x},\tilde{\VEC{q}})\big) \\
&= \xi(\VEC{u},\VEC{p})
\big((\VEC{u},\VEC{p}),(\VEC{x},\tilde{\VEC{q}})\big)
= \omega(\VEC{u})(\VEC{u},\VEC{x}) \ ,
\end{align*}
where the second to last equality comes from the hypothesis that
$\displaystyle \eta^\ast(\xi) = \xi$ and
the last equality comes from the independence of the definition of
$\omega$ on $\tilde{\VEC{q}}$ proved in the previous paragraph.

To prove that $\omega$ is smooth, we consider a local chart
$(W,U,\phi)$ of $S$ and a local chart
$(\tilde{W},\tilde{U},\tilde{\phi})$ of $S_c$ as given in
Subsection~\ref{subsectTSc}.
The local representation of $\omega$ on $W$ is given by
\begin{align*}
&\tilde{\phi}^\ast(\xi)\big((\VEC{w},\psi), (\VEC{y}, t)\big)
= \xi\big(\tilde{\phi}(\VEC{w},\psi)\big)
\bigg( \tilde{\phi}(\VEC{w},\psi), \diff_{\VEC{w},\psi}
\tilde{\phi}(\VEC{w},\psi) \begin{pmatrix} \VEC{y}\\ t\end{pmatrix}\bigg) \\
&\qquad = \xi\big(\phi(\VEC{w}),\psi.E(\VEC{w})\big)
\bigg( (\phi(\VEC{w}),\psi.E(\VEC{w})),
\bigg( \diff \phi(\VEC{w}) \VEC{y},
\diff_{\VEC{w},\psi} (\psi.E(\VEC{w}))
\begin{pmatrix} \VEC{y} \\ t\end{pmatrix}\bigg)\bigg) \\
&\qquad = \omega\big(\phi(\VEC{u})\big)\big(\phi(\VEC{w}),\diff \phi(\VEC{w})
\VEC{y}\big)
= \phi^\ast(\omega)(\VEC{w})(\VEC{w},\VEC{y}) \ ,
\end{align*}
where we may assume that $\displaystyle \psi,t \in S^1$ are constant.
Thus $\displaystyle \phi^\ast(\omega)$ is a smooth differential form on
$W$ because it is the composition of smooth functions.
Note that if
$(\VEC{u},\VEC{x}) = \phi_\ast(\VEC{w}, \VEC{y}) \in \TS_{\VEC{u}} S$,
the equation above states that
$\displaystyle \omega(\VEC{u})(\VEC{u},\VEC{x})
= \xi(\VEC{u},\VEC{p})
\big( (\VEC{u},\VEC{p}),( \VEC{x}, \VEC{q})\big)$
with
$\tilde{x} = \psi.\tilde{E}(\VEC{u})$ and
$\displaystyle \VEC{q} = \diff_{\VEC{w},\psi} (\psi.E(\VEC{w}))
\begin{pmatrix} \VEC{y} \\ t\end{pmatrix}$.

By construction, we have
\begin{align*}
\pi_S^\ast(\omega)(\VEC{u},\VEC{p})
\big((\VEC{u},\VEC{p}), (\VEC{x}, \VEC{q})\big) 
&= \omega( \pi_S(\VEC{u},\VEC{p}) \big((\pi_S)_\ast
\big((\VEC{u},\VEC{p}), (\VEC{x}, \VEC{q})\big) \big)
= \omega(\VEC{u})(\VEC{u},\VEC{x}) \\
&= \xi(\VEC{u},\VEC{p})\big((\VEC{u},\VEC{p}),
(\VEC{x},\VEC{q})\big)
\end{align*}
for all $\big((\VEC{u},\VEC{p}), (\VEC{x},\VEC{q})\big) \in
\TS_{(\VEC{u},\VEC{p})}\, S_c$.
\end{proof}

\begin{cor}  \label{cor2rho2do}
Suppose that $\displaystyle \MM_c^{[i]} = \{ M^{[i]}_{(\VEC{u},\VEC{x})}
\}_{(\VEC{u},\VEC{x})\in S_c}$ for $i =1,2$ are 
two connections on $S_c$ and $\displaystyle \rho^{[i]}$ is a connection
$1$-form associated to $\displaystyle \MM_c^{[i]}$ for $i = 1,2$.
Then $\displaystyle \rho^{[1]}-\rho^{[2]} = \pi_S^\ast(\omega)$ for
some smooth differential $1$-form $\omega$ on $S$.
\end{cor}

\begin{proof}
It follows from Theorem~\ref{thmConn1form} and
Proposition~\ref{proERequR} that
$\displaystyle \rho^{[i]}(\VEC{u},\VEC{x})\big(H(\VEC{u},\VEC{x})\big) =1$
for all $(\VEC{u},\VEC{x}) \in S_c$ and 
$\displaystyle \eta^\ast\big(\rho^{[i]}\big)= \rho^{[i]}$ for all
$\displaystyle \eta \in S^1$ and $i =1,2$.  Hence, if we
set $\displaystyle \xi = \rho^{[1]} - \rho^{[2]}$, we get that
$\displaystyle \xi(\VEC{u},\VEC{x})\big(H(\VEC{u},\VEC{x})\big) = 0$
for all $(\VEC{u},\VEC{x}) \in S_c$ and $\displaystyle \eta^\ast(\xi) = \xi$
for all $\displaystyle \eta \in S^1$.  The conclusion of the corollary 
follows from Proposition~\ref{prorPiAom}.
\end{proof}

\begin{prop}  \label{proUnHorLift}
Let $S$ be a $2$-dimensional Riemannian manifold and
$\MM_c = \displaystyle \{ M_{(\VEC{u},\VEC{x})}\}_{(\VEC{u},\VEC{x}) \in S_c} \}$
be a connection on $S_c$ where $\rho$ is a connection $1$-form
associated to $\MM_c$.
Suppose that
\begin{enumerate}
\item $\displaystyle \sigma:[a,b] \to S$ is a smooth curve,
\item $\displaystyle \tilde{\sigma}^{[i]}:[a,b] \to S_c$ is smooth
curves such that $\displaystyle \pi_S \circ \tilde{\sigma}^{[i]} = \sigma$
for $i =1,2$, and
\item $\displaystyle \rho\big(\tilde{\sigma}^{[1]}(t))\big)
\big( (\tilde{\sigma}^{[1]})_\ast(t,1) \big) = 0$
for $a \leq t \leq b$.
\end{enumerate}
Then there exists a smooth function $\theta:[a,b] \to \RR$ such that
$\displaystyle \tilde{\sigma}^{[2]}(t) = \eta_t(\tilde{\sigma}^{[1]}(t))$ with
$\displaystyle \eta_t = e^{i\theta(t)} \in S^1$ and
$\displaystyle \rho\big(\tilde{\sigma}^{[2]}(t))\big)
\big( (\tilde{\sigma}^{[2]})_\ast(t,1) \big) = \theta'(t)$
for $a \leq t \leq b$.

Moreover, if
$\displaystyle \tilde{\sigma}^{[1]}(a) = \tilde{\sigma}^{[2]}(a)$, then
we can choose $\theta$ such that $\theta(a) = 0$.
\end{prop}

\begin{proof}
\stage{i} Since $\displaystyle \pi_S(\tilde{\sigma}^{[i]}(t)) = \sigma(t)$ for
$a \leq t \leq b$ and $i =1,2$, we have that
$\displaystyle \tilde{\sigma}^{[i]}(t) \in \TS_{\sigma(t)} S_c$ for
$a \leq t \leq b$ and $i =1,2$.  Thus, for each $t \in [a,b]$,
there exists $\displaystyle \eta_t \in S^1$ such that
$\displaystyle \tilde{\sigma}^{[2]}(t) = \eta_t(\tilde{\sigma}^{[1]}(t))$.
Since $\displaystyle \tilde{\sigma}^{[i]}$ for $i =1,2$ are
smooth functions, we have that $\displaystyle \eta:[a,b] \to S^1$
defined by
$\eta(t)= \eta_t$ for $a \leq t \leq b$ is also smooth.

Let $\theta:[a,b] \to \RR$ be a function such that
$\displaystyle e^{i \theta(t)} = \eta(t)$.  Since $\eta$ is smooth and
$[a,b]$ is connected, we get that $\displaystyle \eta([a,b]) \subset S^1$ is
connected.  We may therefore assume that $\theta([a,b])$ is connected
and $\theta:[a,b] \to \RR$ is smooth \footnote{For $t$ in a small
enough neighbourhood $V$ of $t_0 \in [a,b]$, we have that
$\log(\eta(t)) = \log(|\eta(t)|) + i \arg(\eta(t)) = i \theta(t)$
for $t\in V$.  We have used a branch of the logarithm function.
Since $\log$ is locally smooth and $\eta$ is smooth, we have that
$\theta$ is locally smooth.  Since $t_0 \in [a,b]$ is arbitrary, we
may concluded that $\theta:[a,b] \to \RR$ is smooth.}.  We have that
$\displaystyle (\RR,e^{i\theta})$ is a covering of $\displaystyle S^1$
(Figure~\ref{FundGr2}).

If $\displaystyle \tilde{\sigma}^{[1]}(a) = \tilde{\sigma}^{[2]}(a)$,
then we may take $\theta(a) = 0$.  In that case, $\theta$ is uniquely
defined because $\theta([a,b])$ is connected. 

\stage{ii}
Let $(W,U,\phi)$ be a local chart of $S$ and
$(\tilde{W},\tilde{U},\tilde{\phi})$ be the local
chart of $S_c$ as defined in Subsection~\ref{subsectTSc}.

Let $\mu:[c,d] \to W$ be the local representation of $\sigma$ where
$[c,d] \subset [a,b]$; namely, $\sigma = \phi \circ \mu$ on $[c,d]$.
Let $\displaystyle \mu^{[i]}:[c,d] \to \tilde{W}$ be the local
representation of $\displaystyle \tilde{\sigma}^{[i]}$ for $i =1,2$;
namely, $\displaystyle \tilde{\sigma}^{[i]} = \tilde{\phi} \circ \mu^{[i]}$  on
$[c,d]$ for $i =1,2$.  We have for $i=1,2$ that
$\displaystyle \mu^{[i]}(t) = (\mu(t), \psi^{[i]}(t))$ for some
$\displaystyle \psi^{[i]}(t) \in \RR$.
Moreover,
\[
\tilde{\sigma}^{[2]}(t) = \eta_t\big(\tilde{\sigma}^{[1]}(t)\big)
= \eta_t\big(\phi(\mu(t)), \psi^{[1]}(t).E(\mu(t)) \big)
= \big(\phi(\mu(t)), (\theta(t) + \psi^{[1]}(t)).E(\mu(t))\big)
\]
for $c \leq t \leq d$.  Thus
$\displaystyle \psi^{[2]}(t) = \theta(t) + \psi^{[1]}(t)$ 
for $c \leq t \leq d$.

We have
\begin{align*}
&\rho\big(\tilde{\sigma}^{[2]}(t))\big)
\big( (\tilde{\sigma}^{[2]})_\ast(t,1) \big)
= \rho\big((\tilde{\phi} \circ \mu^{[2]})(t)\big)
\big( (\tilde{\phi}\circ \mu^{[2]})_\ast(t,1) \big)
= \rho\big((\tilde{\phi} (\mu^{[2]}(t))\big)
\big(\tilde{\phi}_\ast(\mu^{[2]}_\ast(t,1)) \big) \\
&\qquad =\tilde{\phi}^\ast(\rho)(\mu^{[2]}(t))\big(\mu^{[2]}_\ast(t,1)\big)
= \tilde{\phi}^\ast(\rho)(\mu^{[2]}(t))\left( \mu^{[2]}(t), \left(\mu'(t),
\dydx{\psi^{[2]}}{t}(t) \right)\right) \\
&\qquad
= \tilde{\phi}^\ast(\rho)(\mu^{[2]}(t))\left( \mu^{[2]}(t), \left(\mu'(t),
\theta'(t) + \dydx{\psi^{[1]}}{t}(t) \right)\right) \\
&\qquad = \tilde{\phi}^\ast(\rho)(\mu^{[2]}(t))\left( \mu^{[2]}(t),
(0, \theta'(t) )\right)
+ \tilde{\phi}^\ast(\rho)(\mu^{[2]}(t))\left( \mu^{[2]}(t),
\left(\mu'(t),\dydx{\psi^{[1]}}{t}(t) \right)\right) \ ,
\end{align*}
where
\begin{align*}
&\tilde{\phi}^\ast(\rho)(\mu^{[2]}(t))
\left( \mu^{[2]}(t), (0, \theta'(t) )\right)
= \rho(\tilde{\sigma}^{[2]}(t))
\left( \tilde{\sigma}^{[2]}(t),
\diff_{\VEC{w},\nu}\tilde{\phi}(\VEC{w},\nu)
\big|_{(\VEC{w},\nu)=\mu^{[2]}(t)}
\begin{pmatrix} 0 \\ \theta'(t) \end{pmatrix}
\right) \\
&\qquad = \rho(\tilde{\sigma}^{[2]}(t))
\left( \tilde{\sigma}^{[2]}(t), \left( 0 , \theta'(t)
\pdfdx{\big(\nu.E(\mu(t))\big)}{\nu}\Big|_{\nu = \psi^{[2]}(t)}
\right) \right) \\
&\qquad = \rho(\tilde{\sigma}^{[2]}(t))
\left( \theta'(t) H(\tilde{\sigma}^{[2]}(t))\right)
= \theta'(t)\, \rho(\tilde{\sigma}^{[2]}(t))
\left( H(\tilde{\sigma}^{[2]}(t))\right)
  = \theta'(t)
\end{align*}
and
\[
\tilde{\phi}^\ast(\rho)(\mu^{[2]}(t))\left( \mu^{[2]}(t),
\left(\mu'(t),\dydx{\psi^{[1]}}{t}(t) \right)\right) = 0
\]
as we explain below.

For $s$ fixed, we have that
$\displaystyle (\eta_s)_\ast \left( \tilde{\sigma}^{[1]}_\ast(t,1) \right)
=\left( \eta_s(\tilde{\sigma}^{[1]})\right)_\ast(t,1)$.
Moreover, since
\[
\eta_s(\tilde{\sigma}^{[1]}(t))
= \left(\sigma(t), (\theta(s) + \psi^{[1]}(t)).\tilde{E}(\sigma(t)) \right)
= \tilde{\phi}\big(\mu(t), \theta(s) + \psi^{[1]}(t)\big) \ ,
\]
we get
\begin{align*}
&\left( \eta_s(\tilde{\sigma}^{[1]})\right)_\ast(t,1)
= \left( \eta_s(\tilde{\sigma}^{[1]}(t)) ,
\dfdx{ \eta_s(\tilde{\sigma}^{[1]}(t))}{t} \right)
= \left( \eta_s(\tilde{\sigma}^{[1]}(t)) ,
\dfdx{\tilde{\phi}\big(\mu(t), \theta(s) + \psi^{[1]}(t)\big)}{t} \right) \\
&\quad = \left(\left(\sigma(t),
(\theta(s) + \psi^{[1]}(t)).\tilde{E}(\sigma(t))\right),
\diff_{\VEC{w},\nu}
\tilde{\phi}(\VEC{w},\nu)\Big|_{(\VEC{w},\nu) = (\mu(t), \theta(s) + \psi^{[1]}(t))}
\begin{pmatrix} \mu'(t) \\ \displaystyle \dydx{\psi^{[1]}}{t}(t) \end{pmatrix}
\right) \ .
\end{align*}
Hence, at $t=s$, we get
\begin{align*}
(\eta_t)_\ast \left( \tilde{\sigma}^{[1]}_\ast(t,1) \right)
&= \left(\left(\sigma(t), \psi^{[2]}(t)).\tilde{E}(\sigma(t))\right),
\diff_{\VEC{w},\nu}
\tilde{\phi}(\VEC{w},\nu)\Big|_{(\VEC{w},\nu) = \mu^{[2]}(t)}
\begin{pmatrix} \mu'(t) \\ \displaystyle \dydx{\psi^{[1]}}{t}(t) \end{pmatrix}
\right) \\
&= \left( \tilde{\sigma}^{[2]}(t),
\diff_{\VEC{w},\nu}
\tilde{\phi}(\VEC{w},\nu)\Big|_{(\VEC{w},\nu) = \mu^{[2]}(t)}
\begin{pmatrix} \mu'(t) \\ \displaystyle \dydx{\psi^{[1]}}{t}(t) \end{pmatrix}
\right) \ .
\end{align*}
It follows that
\begin{align*}
&\tilde{\phi}^\ast(\rho)(\mu^{[2]}(t))\left( \mu^{[2]}(t),
\left(\mu'(t),\dydx{\psi^{[1]}}{t}(t) \right)\right) \\
&\qquad= \rho(\tilde{\sigma}^{[2]}(t))\left( \tilde{\sigma}^{[2]}(t),
\diff_{\VEC{w},\nu}\tilde{\phi}(\VEC{w},\nu)
\big|_{(\VEC{w},\nu)=\mu^{[2]}(t)}
\begin{pmatrix} \mu'(t) \\ \displaystyle \dydx{\psi^{[1]}}{t}(t) \end{pmatrix}
\right) \\
&\qquad= \rho(\tilde{\sigma}^{[2]}(t))\left(
(\eta_t)_\ast \left( \tilde{\sigma}^{[1]}_\ast(t,1) \right)\right)
= \rho(\eta_t(\tilde{\sigma}^{[1]}(t)))
\left((\eta_t)_\ast \left( \tilde{\sigma}^{[1]}_\ast(t,1) \right)\right) \\
&\qquad= \eta_t^\ast(\rho)(\tilde{\sigma}^{[1]}(t))\left(
\tilde{\sigma}^{[1]}_\ast(t,1) \right)
= \rho(\tilde{\sigma}^{[1]}(t))\left(
\tilde{\sigma}^{[1]}_\ast(t,1) \right) = 0
\end{align*}
where the fifth equality comes from Proposition~\ref{proERequR} and the 
last equality comes from the condition (3) in the statement of the
lemma.
\end{proof}

\begin{prop} \label{propHLexists}
Let $S$ be a $2$-dimensional Riemannian manifold.  Moreover, let
$\MM_c = \displaystyle \{ M_{(\VEC{u},\VEC{x})}\}_{(\VEC{u},\VEC{x}) \in S_c} \}$
be a connection on $S_c$ where $\rho$ is a connection $1$-form
associated to $\MM_c$.
Suppose that $\displaystyle \sigma:[a,b] \to S$ a piecewise
$\displaystyle C^\infty$ curve \footnotemark\ and that
$(\sigma(a),\VEC{x}) \in S_c$.
Then there exists a unique piecewise $\displaystyle C^\infty$
curve $\tilde{\sigma}:[a,b] \to S_c$ such that
\begin{enumerate}
\item $\tilde{\sigma}(a) = (\sigma(a),\VEC{x})$,
\item $\pi_S\circ \tilde{\sigma} = \sigma$, and
\item $\rho(\tilde{\sigma}(t))(\tilde{\sigma}_\ast(t,1)) = 0$
for $a \leq t \leq b$.
\end{enumerate}
Thus $\tilde{\sigma}$ is a lift that provides a parallel
translation of $(\sigma(a),\VEC{x})$ along $\sigma$.
The curve $\tilde{\sigma}$ is called the
{\bfseries horizontal lift}\index{Horizontal Lift} of $\sigma$
through $(\sigma(a),\VEC{x})$.
\end{prop}

\footnotetext{Recall that a piecewise $\displaystyle C^\infty$ function 
$\displaystyle f:[a,b] \to \RR^n$ is a continuous function which is of class
$\displaystyle C^\infty$ on all the closed intervals delimited by the
finite number of points in $[a,b]$ where it is not
$\displaystyle C^\infty$.}

\begin{rmk}
It follows from Proposition~\ref{thmConn1form} that
$\pi_{P_{\tilde{\sigma}(t)}}\big(\tilde{\sigma}_\ast(t,1)\big) = 0$
and thus $\tilde{\sigma}_\ast(t,1) \in M_{\tilde{\sigma}(t)}$
for $a \leq t \leq b$.  We may say that
$\tilde{\sigma}_\ast(t,1)$ is orthogonal to $P_{\tilde{\sigma}(t)}$.

Moreover, it follows from the uniqueness of $\tilde{\sigma}$
satisfying (1) and Proposition~\ref{proERequR} that the third
condition listed before the Definition~\ref{defnLiftId} is satisfied.
\end{rmk}

\begin{proof}
It is enough to prove the proposition for a smooth curve
$\sigma:[a,b]\to S_c$ because we can then use the result on each smooth
sections of a piecewise $\displaystyle C^\infty$ curve using the
end point of the horizontal lift of a smooth subsection as the
starting point on $S_c$ of the horizontal lift of the next smooth
subsection.

\stage{i} Let $(W,U,\phi)$ be a local charts of $S$ with
$\sigma(a) \in U$.  We prove that there exists a unique horizontal lift
of $\sigma$ on $U$.

We have seen in Remark~\ref{rmkUS1eScU} that
$\displaystyle G: U \times S^1 \to S_c(U) = \bigcup_{\VEC{u} \in U} S_c(\VEC{u})$
defined by
$G(\VEC{u},\eta) = \big(\VEC{u},\theta.\tilde{E}(\VEC{u})\big)$
for $\VEC{u} \in U$ and $\displaystyle \eta = e^{i\theta} \in S^1$
is an isomorphism, where
$\displaystyle \tilde{E}(\VEC{u}) = E(\phi^{-1}(\VEC{u}))$ for all
$\VEC{u} \in U$.
Let $G_1(\VEC{u}) = G(\VEC{u},1) = (\VEC{u}, 0.\tilde{E}(\VEC{u}))
= (\VEC{u}, \tilde{E}(\VEC{u}))$ for $\VEC{u} \in U$.  The map
$G_1:U \to S_c(U)$ is smooth.

We consider the connection
$\displaystyle \check{\MM}_c
= \{ \check{M}_{(\VEC{u},\VEC{x})}\}_{(\VEC{u},\VEC{x}) \in S_c(U)}$
on $S_c(U)$ with its associated connection $1$-from $\check{\rho}$
defined in Examples~\ref{eggConnPAp1} and \ref{eggConnPAp2}.
We prove that there exists an horizontal lift $\check{\sigma}$ of
$\sigma$ on $U$ such that $\check{\sigma}(a) = G_1(\sigma(a))$.  To
prove the existence, consider
$\check{\sigma}(t) = G_1(\sigma(t))$ for $a \leq t \leq c$ and
$c - a$ small enough to have $\sigma([a,c]) \subset U$.
We have that $\check{\sigma}(a) = G_1(\sigma(a))$ and
$\pi_S(\check{\sigma}(t)) = \sigma(t)$ for $a \leq t \leq c$.
Moreover, since
$\check{M}_{\eta(\VEC{u},\tilde{E}(\VEC{u}))} =(G_\eta)_\ast(\TS_{\VEC{u}}S)$
for $\VEC{u} \in U$ and $\displaystyle \eta \in S^1$ by definition, we
get that $\check{\sigma}_\ast(t,1) = (G_1)_\ast(\sigma_\ast(t,1)) \in 
\check{M}_{\tilde{\sigma}(t)}$ for $a \leq t \leq c$
because $\sigma_\ast(t,1) \in \TS_{\sigma(t)} S$.
Hence
$\pi_{P_{\check{\sigma}(t)}}\big(\check{\sigma}_\ast(t,1)\big) = 0$
for $a \leq t \leq c$.  It follows from
Proposition~\ref{thmConn1form} that
$\check{\rho}(\check{\sigma}(t))\big(\check{\sigma}_\ast(t,1)\big) = 0$
for $a \leq t \leq c$ \footnote{We can prove directly that
$\check{\rho}(\check{\sigma}(t))\big(\check{\sigma}_\ast(t,1)\big) = 0$.
Referring to Examples~\ref{eggConnPAp1} and \ref{eggConnPAp2},
if we define $\sigma_1(t) = (\sigma(t),1)$ for $a \leq t \leq c$, then
$\check{\sigma}(t) = (G\circ \sigma_1)(t))$ and
\begin{align*}
\check{\rho}(\check{\sigma}(t))\big(\check{\sigma}_\ast(t,1)\big)
&=(G^{-1})^\ast(\pi_2^\ast(\omega))\big((G\circ \sigma_1)(t)\big)
\big((G \circ \sigma_1)_\ast(t,1)\big)\\
&=\pi_2^\ast(\omega)\big(G^{-1}\big((G\circ \sigma_1)(t)\big)\big)
\big((G^{-1})_\ast\big( (G \circ \sigma_1)_\ast(t,1) \big)\big) \\
&=\pi_2^\ast(\omega)\big(\sigma_1(t)\big)\big((\sigma_1)_\ast(t,1)\big)
=\pi_2^\ast(\omega)\big((\sigma(t),1)\big)
\big((\sigma(t),1),(\sigma'(t),0)\big) \\
&=\omega\big(\pi_2(\sigma(t),1)\big)
\big((\pi_2)_\ast\big((\sigma(t),1),(\sigma'(t),0)\big)\big)
= \omega(1)(1,0) = 0 \ .
\end{align*}
It follows from Proposition~\ref{thmConn1form} that
$\pi_{P_{\check{\sigma}(t)}}\big(\check{\sigma}_\ast(t,1)\big) = 0$
and thus $\check{\sigma}_\ast(t,1) \in M_{\check{\sigma}(t)}$.}.
Thus $\check{\sigma}$ is a local horizontal lift of $\sigma$ through
$G_1(\sigma(a))$.

To prove the uniqueness of the local horizontal lift of $\sigma$
through $G_1(\sigma(a))$.
Suppose that $\check{\nu}$ is another local horizontal lift of $\sigma$
through $G_1(\sigma(a))$ defined on $[a,\check{c}[$.  By choosing the
smallest of $c$ and $\check{c}$, we may assume that $c = \check{c}$.
It follows from Proposition~\ref{proUnHorLift} that there exists
$\zeta:[a,c] \to \RR$ with $\zeta(a) = 0$ such that 
$\check{\nu}(t) = \eta_t(\check{\sigma}(t))$ with
$\displaystyle \eta_t = e^{i\zeta(t)}$
and
$\displaystyle \check{\rho}\big(\check{\nu}(t))\big)
\big(\check{\nu}_\ast(t,1) \big) = \zeta'(t)$
for $a \leq t \leq c$.
Since $\check{\nu}$ is a horizontal lift of $\sigma$, we have that
$\zeta'(t) = \check{\rho}(\check{\nu}(t))\big(\check{\nu}_\ast(t,1)\big) = 0$
for $a \leq t \leq c$.  Therefore $\zeta(t) = 0$ and so
$\check{\nu}(t) = \check{\sigma}(t)$ for $a \leq t \leq c$.

\stage{ii} According to Corollary~\ref{cor2rho2do}, there exists a
differential $1$-form $\omega$ on $U$ such that
$\displaystyle \check{\rho} - \rho = \pi_S^\ast{(\omega)}$.  Choose a curve
$\tilde{\sigma}:[a,c]\to S_c(U)$ such that
$\pi_S(\tilde{\sigma}(t)) = \sigma(t)$ for $a \leq t \leq c$.
It follows from Proposition~\ref{proUnHorLift} that there exists
a smooth function $\xi:[a,c] \to \RR$ such that
$\displaystyle \tilde{\sigma}(t) = \eta_t(\check{\sigma}(t))$ with
$\displaystyle \eta_t = e^{i\xi(t)} \in S^1$ and
$\displaystyle \check{\rho}\big(\tilde{\sigma}(t))\big)
\big( \tilde{\sigma}_\ast(t,1) \big) = \xi'(t)$
for $a \leq t \leq c$.

We have from Theorem~\ref{thmConn1form} that 
$\tilde{\sigma}$ is a horizontal lift of $\sigma$ if and only if\\
$\rho(\tilde{\sigma}(t))\big(\tilde{\sigma}_\ast(t,1)\big) = 0$
for $a \leq t \leq c$.
This relation is satisfied if and only if
\begin{equation} \label{propHLexistEqu1}
(\check{\rho} - \rho)(\tilde{\sigma}(t))\big(\tilde{\sigma}_\ast(t,1)\big)
= \check{\rho}(\tilde{\sigma}(t))\big(\tilde{\sigma}_\ast(t,1)\big) = \xi'(t)
\end{equation}
for $a \leq t \leq c$.  Moreover,
\begin{equation} \label{propHLexistEqu2}
\begin{split}
(\check{\rho} - \rho)(\tilde{\sigma}(t))\big(\tilde{\sigma}_\ast(t,1)\big)
&= \pi_S^\ast(\omega)(\tilde{\sigma}(t))\big(\tilde{\sigma}_\ast(t,1)\big)
= \omega\big(\pi_S(\tilde{\sigma}(t))\big)
\big((\pi_S)_\ast(\tilde{\sigma}_\ast(t,1))\big) \\
&= \omega(\sigma(t))
\big((\pi_S)_\ast\big((\sigma(t), *),(\sigma'(t), *)\big)\big) \\
&= \omega(\sigma(t))\big(\sigma(t),\sigma'(t)\big)
= \omega(\sigma(t))\big(\sigma_\ast(t,1)\big)
\end{split}
\end{equation}
for $a \leq t \leq c$.  It follows from (\ref{propHLexistEqu1})
and (\ref{propHLexistEqu2}) that $\tilde{\sigma}$ is a horizontal lift
of $\sigma$ if and only if 
$\xi'(t) = \omega(\sigma(t))\big(\sigma_\ast(t,1)\big)$
for $a \leq t \leq c$.  Namely, $\tilde{\sigma}$ is a horizontal
lift of $\sigma$ if and only if 
$\displaystyle \xi(t) = \int_a^t \omega(\sigma(s))\big(\sigma_\ast(s,1)\big)
\dx{s} + \xi_0$ for $a \leq t \leq c$ and $\xi_0 \in \RR$.
Hence, each horizontal lift $\tilde{\sigma}$ of $\sigma$ is of the
form $\tilde{\sigma} = \eta_{t,\xi_0}(\check{\sigma}(t))$ for
$a \leq t \leq c$ and
\[
\eta_{t,\xi_0} = e^{i\xi(t)} = e^{i\xi_0} e^{i \int_a^t
\omega(\sigma(s))(\sigma_\ast(s,1)) \dx{s}} \in S^1
\]
for some $\xi_0 \in \RR$.

Given $(\sigma(a),\VEC{x}) \in S_c(U)$, there exists a unique
$\xi_0 \in [0,2\pi[$ such that\\
$(\sigma(a),\VEC{x}) = \eta_{a,\xi_0}.(\check{\sigma}(a))$.
The unique horizontal lift of $\sigma$ that we are
looking for is given by $\tilde{\sigma}(t) =
\eta_{t,\xi_0}(\check{\sigma}(t))$ for $a \leq t \leq c$
with this value of $\xi_0$.

\stage{iii} We use a proof by contradiction to demonstrate that
$\tilde{\sigma}$ can be extended to the entire interval $[a,b]$.
Let
\[
t_f = \sup\,\{ t \in [a,b] : \sigma\big|_{[a.t]} \text{ as a unique
horizontal lift } \tilde{\sigma} \text{ with } \tilde{\sigma}(a) = (\sigma(a),
\VEC{x}). \} \ .
\]
Suppose that $t_f < b$.  There exist $\epsilon >0$  and a local
chart $(W,U,\phi)$ of $S$ such that\\
$\sigma([t_f - \epsilon,t_f +\epsilon]) \subset U$.
Using the previous part of the proof, there exists a unique local horizontal
lift $\hat{\sigma}:[t_f - \epsilon,t_f + \epsilon] \to S_c(U)$
of $\sigma$ such that $\hat{\sigma}(t_f) = (\sigma(t_f), \VEC{y})$
for $(\sigma(t_f),\VEC{y}) \in S_c$ arbitrary but fixed.

There exists $\displaystyle \eta \in S^1$ such that 
$\tilde{\sigma}(t_f - \epsilon) = \eta(\hat{\sigma}(t_f - \epsilon))$.
Using Proposition~\ref{proERequR}, it is easy to verify that 
$\eta \circ \tilde{\sigma}$ is a horizontal lift; namely, that (1), (2)
and (3) in the statement of the theorem are satisfied for $[a,b]$
replaced by $[t_f-\epsilon,t_f + \epsilon]$, and $(\sigma(a),\VEC{x})$
replaced by $\tilde{\sigma}(t_f - \epsilon)$.
By uniqueness of the horizontal lift of $\sigma$ through 
$\tilde{\sigma}(t_f - \epsilon)$, we must have that
$\tilde{\sigma}(t) = \eta(\hat{\sigma}(t))$ for $t_f - \epsilon \leq t < t_f$.
Therefore
\[
\tilde{\sigma}(t) =
\begin{cases}
\tilde{\sigma}(t) & \quad \text{if}\ a \leq t < t_f \\
\eta(\hat{\sigma}(t)) & \quad \text{if}\ t_f \leq t < t_f + \epsilon  
\end{cases}
\]
is a unique horizontal lift of $\sigma$ with
$\tilde{\sigma}(a) = (\sigma(a), \VEC{x})$.  This is a contradiction
that $t_f$ is maximal.  Note that the uniqueness
of the extended map $\tilde{\sigma}$ comes from the uniqueness of
$\tilde{\sigma}$ with $\tilde{\sigma}(a) = (\sigma(a), \VEC{x})$ and
the uniqueness of $\eta \circ \hat{\sigma}$ with
$\eta(\hat{\sigma}(t_f)) = \eta(\sigma(t_f), \VEC{y})$.
\end{proof}

\section{Structural Equations and Curvature}
\label{sectRiemannCurv}

\begin{defn} \label{defnO1O2}
Let $S$ be an oriented $2$-dimensional Riemannian manifold.
We define two differential $1$-form $\omega_1$ and $\omega_2$ on $S_c$
as it follows.
Given $(\VEC{u},\VEC{x}) \in S_c$, we set
\[
\omega_1(\VEC{u},\VEC{x})\big((\VEC{u},\VEC{x}),(\VEC{p},\VEC{q})\big)
= \ps{(\pi_S)_\ast\big((\VEC{u},\VEC{x}),(\VEC{p},\VEC{q})\big)}
{(\VEC{u},\VEC{x})}_{\VEC{u}}
= \ps{(\VEC{u},\VEC{p})}{(\VEC{u},\VEC{x})}_{\VEC{u}}
\]
and
\begin{align*}
\omega_2(\VEC{u},\VEC{x})\big((\VEC{u},\VEC{x}),(\VEC{p},\VEC{q})\big)
&= \ps{(\pi_S)_\ast\big((\VEC{u},\VEC{x}),(\VEC{p},\VEC{q})\big)}
{\upsilon(\VEC{u},\VEC{x})}_{\VEC{u}} \\
&= \ps{(\VEC{u},\VEC{p})}{\upsilon(\VEC{u},\VEC{x})}_{\VEC{u}}
\end{align*}
for all $\big((\VEC{u},\VEC{x}),(\VEC{p},\VEC{q})\big) \in
\TS_{(\VEC{u},\VEC{x})} S_c$ where
$\displaystyle \upsilon = e^{i \pi/2} = i \in S^1$.
\end{defn}

Recall from Lemma~\ref{lemmaPsast} that
$(\pi_S)_\ast( (\VEC{u},\VEC{x}), (\VEC{p},\VEC{q}) ) = (\VEC{u},\VEC{p})$
for all $((\VEC{u},\VEC{x}), (\VEC{p},\VEC{q})) \in
\TS_{(\VEC{u},\VEC{x})} (\TS S)$.
Using local charts, it is also easy to prove that
$\omega_1$ and $\omega_2$ are smooth because we assume that $S$ is a
smooth manifold and
the inner product $\ps{}{}_{\VEC{u}}$ on $\displaystyle \TS_{\VEC{u}} S$
is given by a smooth mapping
$\displaystyle \tau : S \to \bigcup_{\VEC{u}\in S} \T^2(\TS_{\VEC{u}} S)$.
We will give another proof of this fact later on.

\begin{prop}
Let $S$ be an oriented $2$-dimensional Riemannian manifold.  Suppose
that $\displaystyle \MM_c
= \{ M_{(\VEC{u},\VEC{x})}\}_{(\VEC{u},\VEC{x}) \in S_c}$
is a connection on $S_c$ and that $\rho$ is a connection $1$-form
associated to $\MM_c$.  Then
$\{\rho(\VEC{u},\VEC{x}),\omega_1(\VEC{u},\VEC{x}),\omega_2(\VEC{u},\VEC{x})\}$
is a basis of $\displaystyle \Omega^1\left(\TS_{(\VEC{u},\VEC{x})} S_c\right)$
for all $(\VEC{u},\VEC{x}) \in S_c$.
\end{prop}

\begin{proof}
Since $\displaystyle \Omega^1\left(\TS_{(\VEC{u},\VEC{x})} S_c\right)$
is the dual space of $\TS_{(\VEC{u},\VEC{x})} S_c$, we have that
$\displaystyle \dim \Omega^1\left(\TS_{(\VEC{u},\VEC{x})} S_c\right) = 3$.
It therefore suffices to prove that $\rho(\VEC{u},\VEC{x})$,
$\omega_1(\VEC{u},\VEC{x})$ and $\omega_2(\VEC{u},\VEC{x})$
are linearly independent.  To prove that, it suffices to verify that
there is no non-null element
$\big((\VEC{u},\VEC{x}),(\VEC{p},\VEC{q})\big)
\in \TS_{(\VEC{u},\VEC{x})} S_c = M_{(\VEC{u},\VEC{x})} \oplus
P_{(\VEC{u},\VEC{x})}$ such that
\begin{equation} \label{o1o2rEq1}
\rho(\VEC{u},\VEC{x})\big((\VEC{u},\VEC{x}),(\VEC{p},\VEC{q})\big)
= \omega_1(\VEC{u},\VEC{x})\big((\VEC{u},\VEC{x}),(\VEC{p},\VEC{q})\big)
= \omega_2(\VEC{u},\VEC{x})\big((\VEC{u},\VEC{x}),(\VEC{p},\VEC{q})\big)
= 0 \ .
\end{equation}
Suppose that there is $\big((\VEC{u},\VEC{x}),(\VEC{p},\VEC{q})\big)
\neq \big((\VEC{u},\VEC{x}),(\VEC{0},\VEC{0})\big)$
such that (\ref{o1o2rEq1}) is satisfied.  Then
\begin{align*}
(\VEC{u},\VEC{p})
&=(\pi_S)_\ast\big((\VEC{u},\VEC{x}),(\VEC{p},\VEC{q})\big) \\
&= \omega_1(\VEC{u},\VEC{x})\big((\VEC{u},\VEC{x}),(\VEC{p},\VEC{q})\big)
\, (\VEC{u},\VEC{x})
+ \omega_2(\VEC{u},\VEC{x})\big((\VEC{u},\VEC{x}),(\VEC{p},\VEC{q})\big)
\,\upsilon(\VEC{u},\VEC{x}) = (\VEC{u},\VEC{0})
\end{align*}
forces $\VEC{p}$ to be null.  Therefore
$\big((\VEC{u},\VEC{x}),(\VEC{p},\VEC{q})\big) = \lambda
H(\VEC{u},\VEC{x})$ for some $\lambda \in \RR$.  But then
$\lambda = \lambda \, \rho(\VEC{u},\VEC{x})\big(H(\VEC{u},\VEC{x})\big)
= \rho(\VEC{u},\VEC{x})\big(\lambda\, H(\VEC{u},\VEC{x})\big)
= \rho(\VEC{u},\VEC{x})\big((\VEC{u},\VEC{x}),(\VEC{p},\VEC{q})\big)
= 0$.  This is a contradiction that 
$\big((\VEC{u},\VEC{x}),(\VEC{p},\VEC{q})\big)$ is non-null.
\end{proof}

\begin{prop} \label{propEtaOmegai}
Let $S$ be an oriented $2$-dimensional Riemannian manifold.  Given
$\displaystyle \eta = e^{i\theta} \in S^1$, we have that
$\displaystyle
\eta^\ast(\omega_1) = \cos(\theta)\,\omega_1 + \sin(\theta)\,\omega_2$
and $\displaystyle
\eta^\ast(\omega_2)= -\sin(\theta)\,\omega_1 + \cos(\theta)\,\omega_2$.
Moreover,
$\displaystyle \eta^\ast(\omega_1 \wedge \omega_2) = \omega_1 \wedge \omega_2$.
\end{prop}

\begin{proof}
Using the results obtained in Subsection~\ref{subsectGrRot} with
$(\VEC{u},\VEC{v}_1)$ and $(\VEC{u},\VEC{v}_2)$ replaced by
$(\VEC{u},\VEC{x})$ and $\upsilon(\VEC{u},\VEC{x})$ respectively, we
get that
$\eta(\VEC{u},\VEC{x}) = \cos(\theta)\, (\VEC{u},\VEC{x}) +
\sin(\theta)\, \upsilon(\VEC{u},\VEC{x})$ and
$\upsilon(\VEC{u},\VEC{x}) = -\sin(\theta)\, (\VEC{u},\VEC{x}) +
\cos(\theta)\, \upsilon(\VEC{u},\VEC{x})$.
Moreover, it follows from Lemma~\ref{lemmaPsast} and Remark~\ref{rmkEtaProps}
that
$(\pi_S)_\ast\big(\eta_\ast
\big((\VEC{u},\VEC{x}),(\VEC{p},\VEC{q})\big)\big)
= (\VEC{u},\VEC{p})
= (\pi_S)_\ast\big((\VEC{u},\VEC{x}),(\VEC{p},\VEC{q})\big)$.
Therefore,
\begin{align*}
&\eta^\ast(\omega_1)(\VEC{u},\VEC{x})
\big((\VEC{u},\VEC{x}),(\VEC{p},\VEC{q})\big) 
= \omega_1(\eta(\VEC{u},\VEC{x}))
\big(\eta_\ast\big((\VEC{u},\VEC{x}),(\VEC{p},\VEC{q})\big)\big) \\
&\qquad = \ps{(\pi_S)_\ast\big(\eta_\ast
\big((\VEC{u},\VEC{x}),(\VEC{p},\VEC{q})\big)\big)}
{\eta(\VEC{u},\VEC{x})}_{\VEC{u}} \\
&\qquad = \ps{(\pi_S)_\ast\big((\VEC{u},\VEC{x}),(\VEC{p},\VEC{q})\big)}
{\cos(\theta) \,(\VEC{u},\VEC{x})
+ \sin(\theta)\,\upsilon(\VEC{u},\VEC{x})} \\
&\qquad
= \cos(\theta)\ps{(\pi_S)_\ast\big((\VEC{u},\VEC{x}),(\VEC{p},\VEC{q})\big)}
{(\VEC{u},\VEC{x})}_{\VEC{u}}
+\sin(\theta)\ps{(\pi_S)_\ast\big((\VEC{u},\VEC{x}),(\VEC{p},\VEC{q})\big)}
{\upsilon(\VEC{u},\VEC{x})}_{\VEC{u}} \\
&\qquad = \cos(\theta)\, \omega_1(\VEC{u},\VEC{x})
\big((\VEC{u},\VEC{x}),(\VEC{p},\VEC{q})\big)
+\sin(\theta)\, \omega_2(\VEC{u},\VEC{x})
\big((\VEC{u},\VEC{x}),(\VEC{p},\VEC{q})\big)
\end{align*}
for all
$\big((\VEC{u},\VEC{x}),(\VEC{p},\VEC{q})\big) \in
\TS_{(\VEC{u},\VEC{x})} S_c$.
Similarly, we have
\begin{align*}
&\eta^\ast(\omega_2)(\VEC{u},\VEC{x})
\big((\VEC{u},\VEC{x}),(\VEC{p},\VEC{q})\big) \\
&\qquad = -\sin(\theta)\, \omega_1(\VEC{u},\VEC{x})
\big((\VEC{u},\VEC{x}),(\VEC{p},\VEC{q})\big)
+\cos(\theta)\, \omega_2(\VEC{u},\VEC{x})
\big((\VEC{u},\VEC{x}),(\VEC{p},\VEC{q})\big)
\end{align*}
for all
$\big((\VEC{u},\VEC{x}),(\VEC{p},\VEC{q})\big) \in
\TS_{(\VEC{u},\VEC{x})} S_c$.

To prove the last statement of the proposition, we use
(3) of Proposition~\ref{propAltR} and Proposition~\ref{propfwewf} to
obtain
\begin{align*}
\eta^\ast(\omega_1 \wedge \omega_2) 
&= \eta^\ast(\omega_1) \wedge \eta^\ast(\omega_2)
= \big(\cos(\theta)\, \omega_1 + \sin(\theta)\, \omega_2\big)
\wedge \big(-\sin(\theta)\, \omega_1 + \cos(\theta)\, \omega_2\big) \\
&= \cos^2(\theta)\, \omega_1 \wedge \omega_2 - \sin^2(\theta)
\,\omega_2 \wedge \omega_1
= \big(\cos^2(\theta) + \sin^2(\theta)\big)\,\omega_1 \wedge \omega_2 \\
&= \omega_1 \wedge \omega_2 \ .  \qedhere
\end{align*}
\end{proof}

\begin{prop}
Let $S$ be an oriented $2$-dimensional Riemannian manifold.  There
exists a differential $2$-form $\omega$ on $S$ such that
$\displaystyle \omega_1 \wedge \omega_2 = (\pi_S)^\ast(\omega)$.
\end{prop}

\begin{proof}[Proof (Sketch)]
The proof is very similar to the proof of Proposition~\ref{prorPiAom}.

We have that
$(\omega_1\wedge \omega_2)(\VEC{u},\VEC{x})
\big( \big((\VEC{u},\VEC{x}), (\VEC{p}_1,\VEC{q}_1)\big),
\big((\VEC{u},\VEC{x}), (\VEC{p}_2,\VEC{q}_2)\big) \big) = 0$
if one of the\\
$\big((\VEC{u},\VEC{x}), (\VEC{p}_1,\VEC{q}_1)\big)$ or
$\big((\VEC{u},\VEC{x}), (\VEC{p}_2,\VEC{q}_2)\big)$ is a multiple of 
$H(\VEC{u},\VEC{x})$ because
$\omega_i(\VEC{u},\VEC{x}) \big( H(\VEC{u},\VEC{x})\big) = 0$ for
$i=1,2$.  Recall that
$H(\VEC{u},\VEC{x}) = \big( (\VEC{u},\VEC{x}),(\VEC{0},\VEC{q})\big)$
for some $\VEC{q}$.  If we substitute this expression in the definition of
$\omega_i$, then we get
$\omega_i(\VEC{u},\VEC{x}) \big( H(\VEC{u},\VEC{x})\big) = 0$ for
$i=1,2$.  Moreover, we have from the previous proposition that
$\displaystyle \eta^\ast(\omega_1 \wedge \omega_2) = \omega_1 \wedge \omega_2$.

The differential $2$-form $\omega$ on $S$ is defined as follows.
Choose $(\VEC{u},\VEC{p}) \in S_c$ arbitrary but fixed.
For $(\VEC{u},\VEC{x}_1),(\VEC{u},\VEC{x}_2) \in \TS_{\VEC{u}} S$, set
\[
\omega(\VEC{u})\big((\VEC{u},\VEC{x}_1),(\VEC{u},\VEC{x}_2)\big) =
(\omega_1 \wedge \omega_2)(\VEC{u},\VEC{p})
\Big( \big((\VEC{u},\VEC{p}), (\VEC{x}_1,\VEC{q}_1)\big),
\big((\VEC{u},\VEC{p}), (\VEC{x}_2,\VEC{q}_2)\big) \Big)
\]
for $\big((\VEC{u},\VEC{p}),(\VEC{x}_1,\VEC{q}_1)\big),
\big((\VEC{u},\VEC{p}),(\VEC{x}_2,\VEC{q}_2)\big) \in
\TS_{(\VEC{u},\VEC{p})} S_c$.
We can then proceed as in the proof of Proposition~\ref{prorPiAom}
to prove that the definition of $\omega$ is independent of the choice
of $\VEC{p}$ and
$\big((\VEC{u},\VEC{p}),(\VEC{x}_1,\VEC{q}_1)\big),
\big((\VEC{u},\VEC{p}),(\VEC{x}_2,\VEC{q}_2)\big) \in
\TS_{(\VEC{u},\VEC{p})} S_c$.

We also have by construction that
$\displaystyle (\pi_S)^\ast(\omega) = \omega_1 \wedge \omega_2$.
\end{proof}

\begin{defn}
Let $S$ be an oriented $2$-dimensional Riemannian manifold.
The {\bfseries volume element}\index{Volume Element} on $S$, denoted
$\dx{V}$, is the differential $2$-form on $S$ such that
$\displaystyle (\pi_S)^\ast(\dx{V}) = \omega_1 \wedge \omega_2$.
\end{defn}

To repeat the warning we have already given in Section~\ref{manifVolume},
the notation $\dx{V}$ is confusing because it designates the
differential $2$-form $\omega$ defined in the previous proposition 
which may not be the derivative of any differential $1$-form.

Let $(W,U,\phi)$ be a local chart of $S$ and
$(\tilde{W},\tilde{U},\tilde{\phi})$ be a local
chart of $S_c$ as defined in Subsection~\ref{subsectTSc}.
let $g:U \to S_c(U)$ be the function defined by
$g(\VEC{u}) = (\VEC{u}, \tilde{E}(\VEC{u}))$ for $\VEC{u} \in U$ where
$\displaystyle \tilde{E}(\VEC{u}) = E(\phi^{-1}(\VEC{u}))$ for all
$\VEC{u} \in U$.
The maps $g:U \to S_c(U)$ and $\upsilon\circ g:U \to S_c(U)$ are
smooth functions because their local representations are given by
$\displaystyle (\tilde{\phi}^{-1} \circ g \circ \phi)(\VEC{w}) =
(\VEC{w},0)$ and
$\displaystyle (\tilde{\phi}^{-1} \circ (\upsilon\circ g) \circ \phi)
(\VEC{w}) = (\VEC{w},\pi/2)$ for all $\VEC{w} \in W$ respectively.

Note that $\{ g(\VEC{u}), \upsilon(g(\VEC{u})) \}$ is a basis of
$\TS_{\VEC{u}} S$ for each $\VEC{u} \in U$.  Let
$\{ \tilde{\omega}_1(\VEC{u}), \tilde{\omega}_2(\VEC{u}) \}$ be
the dual basis associated to
$\{ g(\VEC{u}), \upsilon(g(\VEC{u})) \}$; namely,
$\tilde{\omega}_1(\VEC{u})(g(\VEC{u})) = 1$,
$\tilde{\omega}_1(\VEC{u})(\upsilon(g(\VEC{u}))) = 0$,
$\tilde{\omega}_2(\VEC{u})(g(\VEC{u})) = 0$ and
$\tilde{\omega}_2(\VEC{u})(\upsilon(g(\VEC{u}))) = 1$.
We have that $\tilde{\omega}_1$ and $\tilde{\omega}_2$ are smooth
differential $1$-form on $U$ because $g$ and
$\upsilon\circ g$ are smooth.
We get for every $(\VEC{u},\VEC{x}) \in \TS_{\VEC{u}} S$ that
\begin{align*}
g^\ast(\omega_1)(\VEC{u})\big(\VEC{u},\VEC{x}\big)
&= \omega_1(g(\VEC{u}))\big( g_\ast(\VEC{u},\VEC{x})\big)
= \ps{(\pi_S)_\ast\big(g_\ast(\VEC{u},\VEC{x})\big)}{g(\VEC{u})} \\
&= \ps{(\pi_S\circ g)_\ast(\VEC{u},\VEC{x})}{g(\VEC{u})}
= \ps{(\VEC{u},\VEC{x})}{g(\VEC{u})}
= \tilde{\omega}_1(\VEC{u},\VEC{x}) \ ,
\end{align*}
where the fourth equality comes from $\pi_S \circ g = \Id_U$.
Similarly,
$\displaystyle g^\ast(\omega_2)(\VEC{u})\big(\VEC{u},\VEC{x}\big)
= \tilde{\omega}_2(\VEC{u},\VEC{x})$
for all $(\VEC{u},\VEC{x}) \in \TS_{\VEC{u}} S$.  Thus
$\displaystyle \tilde{\omega}_i = g^\ast(\omega_i)$ for $i=1,2$.
It follows that
\begin{align}
\tilde{\omega}_1 \wedge \tilde{\omega}_2
&= g^\ast(\omega_1) \wedge g^\ast(\omega_1)
= g^\ast( \omega_1 \wedge \omega_2) = g^\ast((\pi_s)^\ast(\dx{V})) \nonumber \\
&= (\pi_S \circ g)^\ast(\dx{V}) = \dx{V}   \label{volEvolEq1}
\end{align}
on $U$ because $\pi_S \circ g = \Id_U$.

If $[g(\VEC{u}), \upsilon(g(\VEC{u}))] = \mu_{\VEC{u}}$, the orientation
on $\TS_{\VEC{u}}S$ for $\VEC{u} \in S$, then
(\ref{volEvolEq1}) shows that the definition of volume elements given
above is identical to the definition of volume elements given in
Definition~\ref{defnVolElemD1} of Section~\ref{manifVolume}.

Let $\displaystyle \rho_i = (\pi_S)^\ast(\tilde{\omega}_i)$ for
$i=1,2$.  The $\rho_i$ are smooth differential $1$-form on $S_c(U)$
because the $\tilde{\omega}_i$ are smooth differential $1$-form on $U$.
We first need to deduce some relations between the $\rho_i$ and
$\omega_i$ to be able to derive the structural equations later.
We have
\[
\rho_1 \wedge \rho_2 = (\pi_S)^\ast(\tilde{\omega}_1) \wedge
(\pi_S)^\ast(\tilde{\omega}_2)
= (\pi_S)^\ast(\tilde{\omega}_1 \wedge \tilde{\omega}_2)
= (\pi_S)^\ast(\dx{V}) = \omega_1 \wedge \omega_2 \ .
\]
In fact, we have more than this relation.
Given $\big(g(\VEC{u}),(\VEC{p},\VEC{q})\big) \in \TS_{g(\VEC{u})} S_c$,
we have
\begin{align*}
&\omega_1(g(\VEC{u}))\big(g(\VEC{u}),(\VEC{p},\VEC{q})\big)\,g(\VEC{u})
+ \omega_2(g(\VEC{u}))\big(g(\VEC{u}),(\VEC{p},\VEC{q})\big)\,
\upsilon(g(\VEC{u})) \\
&\quad
=(\VEC{u},\VEC{p}) = (\pi_S)_\ast\big(g(\VEC{u}),(\VEC{p},\VEC{q})\big) \\
&\quad = \tilde{\omega}_1(\VEC{u})
\big( (\pi_S)_\ast\big(g(\VEC{u}),(\VEC{p},\VEC{q})\big) \big)
\, g(\VEC{u})
+ \tilde{\omega}_2(\VEC{u})\big( (\pi_S)_\ast
\big(g(\VEC{u}),(\VEC{p},\VEC{q})\big)\big)
\, \upsilon(g(\VEC{u})) \\
&\quad = \tilde{\omega}_1(\pi_S(g(\VEC{u})))
\big( (\pi_S)_\ast\big(g(\VEC{u}),(\VEC{p},\VEC{q})\big) \big)
\, g(\VEC{u})
+ \tilde{\omega}_2(\pi_S(g(\VEC{u})))\big( (\pi_S)_\ast
\big(g(\VEC{u}),(\VEC{p},\VEC{q})\big)\big)
\, \upsilon(g(\VEC{u})) \\
% &\quad = (\pi_S)^\ast(\tilde{\omega}_1)(g(\VEC{u}))
% \big(g(\VEC{u}),(\VEC{p},\VEC{q})\big) \, g(\VEC{u})
% + (\pi_S)^\ast(\tilde{\omega}_2)(g(\VEC{u}))
% \big(g(\VEC{u}),(\VEC{p},\VEC{q})\big) \, \upsilon(g(\VEC{u})) \\
&\quad = \rho_1(g(\VEC{u})\big(g(\VEC{u}),(\VEC{p},\VEC{q})\big)
\, g(\VEC{u}) 
+ \rho_2(g(\VEC{u})\big(g(\VEC{u}),(\VEC{p},\VEC{q})\big)
\, \upsilon(g(\VEC{u}))
\end{align*}
where the third identity comes from the fact that
$\{ \tilde{\omega}_1(\VEC{u}), \tilde{\omega}_2(\VEC{u})) \}$ is
the dual basis of $\{ g(\VEC{u}), \upsilon(g(\VEC{u})) \}$
on $\TS_{\VEC{u}} S$ for all $\VEC{u} \in U$.  By matching the
coefficients of $g(\VEC{u})$ and $\upsilon(g(\VEC{u}))$, we get
that $\rho_i(g(\VEC{u})) = \omega_i(g(\VEC{u}))$ on
$\TS_{g(\VEC{u})} S_c$ for $i=1,2$.

Moreover, given $\displaystyle \eta = e^{i\theta} \in S^1$, we have
\[
\eta^\ast(\rho_i) = \eta^\ast(\pi_S)^\ast(\tilde{\omega}_i)
= (\pi_S \circ \eta)^\ast(\tilde{\omega}_i)
= (\pi_S)^\ast(\tilde{\omega}_i) = \rho_i
\]
for $i =1,2$.  Using Proposition~\ref{propEtaOmegai}, we get
\begin{align}
&\omega_1(\eta(g(\VEC{u})))\big(
\eta_\ast\big( g(\VEC{u}) , (\VEC{p},\VEC{q})\big)\big)
=\eta^\ast(\omega_1)(g(\VEC{u}))\big( g(\VEC{u}), (\VEC{p},\VEC{q})\big)
\nonumber \\
&\quad 
= \cos(\theta)\,\omega_1(g(\VEC{u}))\big( g(\VEC{u}) , (\VEC{p},\VEC{q})\big)
+\sin(\theta)\,\omega_2(g(\VEC{u}))\big( g(\VEC{u}) , (\VEC{p},\VEC{q})\big)
\nonumber \\
&\quad
= \cos(\theta)\,\rho_1(g(\VEC{u}))\big( g(\VEC{u}) , (\VEC{p},\VEC{q})\big)
+\sin(\theta)\,\rho_2(g(\VEC{u}))\big( g(\VEC{u}) , (\VEC{p},\VEC{q})\big)
\nonumber \\
&\quad = \cos(\theta)\,\eta^\ast(\rho_1)(g(\VEC{u}))
\big( g(\VEC{u}) , (\VEC{p},\VEC{q})\big)
+\sin(\theta)\,\eta^\ast(\rho_2)(g(\VEC{u}))
\big( g(\VEC{u}) , (\VEC{p},\VEC{q})\big) \nonumber \\
&\quad = \cos(\theta)\,\rho_1(\eta(g(\VEC{u})))
\big(\eta_\ast\big( g(\VEC{u}) , (\VEC{p},\VEC{q})\big) \big)
+\sin(\theta)\,\rho_2(\eta(g(\VEC{u})))
\big(\eta_\ast\big( g(\VEC{u}) , (\VEC{p},\VEC{q})\big)\big)
\label{O1cP1psP2Eq1}
\end{align}
for all $\big( g(\VEC{u}) , (\VEC{p},\VEC{q})\big) \in
\TS_{g(\VEC{u})} S_c$.

Let $\displaystyle \tilde{\eta}:S_c \to S^1$ be the function defined by
$\tilde{\eta}(\VEC{u},\VEC{x}) = \eta$ if
$(\VEC{u},\VEC{x}) = \eta(g(\VEC{u}))$.  Using local coordinates, it
is easy to proof that the function
$\displaystyle \tilde{\eta}: S_c \to S^1$ is a smooth function.
Let $\tilde{\theta}:S_c \to \RR$ be the function defined by
$\tilde{\theta}(\VEC{u},\VEC{x}) = \theta$ where
$\displaystyle \tilde{\eta}(\VEC{u},\VEC{x}) = e^{i\theta}$ with
$0 \leq \theta < 2\pi$.  Namely, $\theta$ is the argument of
$\tilde{\eta}(\VEC{u},\VEC{x})$ modulo $2\pi$.

Since all elements of $S_c(\VEC{u})$ are of the
form $\eta(g(\VEC{u}))$ for some $\displaystyle \eta \in S^1$ and all
elements of $\TS_{\eta(g(\VEC{u}))} S_c$ are of the form
$\eta_\ast\big( g(\VEC{u}) , (\VEC{p},\VEC{q})\big)$
for some $\big( g(\VEC{u}) , (\VEC{p},\VEC{q})\big) \in \TS_{g(\VEC{u})}$,
we get from (\ref{O1cP1psP2Eq1}) that
\begin{equation} \label{O1cP1psP2Eq2}
\omega_1(\VEC{u},\VEC{x})
= \cos\big(\tilde{\theta}(\VEC{u},\VEC{x})\big)\, \rho_1(\VEC{u},\VEC{x})
+ \sin\big(\tilde{\theta}(\VEC{u},\VEC{x})\big)\, \rho_2(\VEC{u},\VEC{x})
\end{equation}
on $\TS_{(\VEC{u},\VEC{x})} S_c$.  Similarly, we have
\begin{equation} \label{O1cP1psP2Eq3}
\omega_2(\VEC{u},\VEC{x})
= -\sin\big(\tilde{\theta}(\VEC{u},\VEC{x})\big)\, \rho_1(\VEC{u},\VEC{x})
+ \cos\big(\tilde{\theta}(\VEC{u},\VEC{x})\big)\, \rho_2(\VEC{u},\VEC{x})
\end{equation}
on $\TS_{(\VEC{u},\VEC{x})} S_c$.

Since $\rho_1$ and $\rho_2$ are smooth differential $1$-form
on $S_c(U)$, and $\tilde{\theta}$ depends smoothly on
$(\VEC{u},\VEC{x}) \in \tilde{U}$,
we get from the previous two relation that $\omega_1$ and $\omega_2$
are smooth differential $1$-form on $S_c(U)$.  This is true for
all local charts of $S$ and $S_c$.

Let $S$ be a smooth $2$-dimensional manifold and
$\displaystyle \MM_c = \{ M_{(\VEC{u},\VEC{x})}\}_{(\VEC{u},\VEC{x}) \in S_c}$
be a connection on $S_c$ where $\rho$ is a connection $1$-form
associated to $\MM_c$.
Since $\{\rho, \omega_1,\omega_2\}$ is a basis of
$\displaystyle \Omega^1\left(\TS_{(\VEC{u},\VEC{x})} S_c\right)$
for all $(\VEC{u},\VEC{x}) \in S_c$, we have that all the differential
$2$-forms on $S_c$ can be expressed uniquely as a linear combination
of $\rho\wedge \omega_1$, $\rho\wedge \omega_2$ and
$\omega_1 \wedge \omega_2$.  In particular, we may express
$\df{\rho}$, $\df{\omega_1}$ and $\df{\omega_2}$ as a linear combination
of $\rho\wedge \omega_1$, $\rho\wedge \omega_2$ and
$\omega_1 \wedge \omega_2$.  These linear combinations are called the
{\bfseries Cartan structural formulae}\index{Cartan Structural Formulae}.
The rest of this section is devoted to computing these Cartan
structural formulae. 

\subsection{First (Cartan) Structural Equation}

Let $(W,U,\phi)$ be a local charts of $S$.  Since
$\displaystyle \df{\rho_i} = \df{\big((\pi_S)^\ast(\tilde{\omega}_i)\big)}
= (\pi_S)^\ast(\df{\tilde{\omega}_i})$ and
$\df{\tilde{\omega}_i} = A_i \dx{V}$
for some smooth function $A_i:U \to \RR$ because $\df{\tilde{\omega}_i}$ is a
smooth differential $2$-form on $U$, we get
\[
\df{\rho_i} = (\pi_S)^\ast\big(A_i \dx{V}\big)
= (A_i \circ \pi_S) \, (\pi_S)^\ast(\dx{V})
= (A_i \circ \pi_S) \, \omega_1 \wedge \omega_2
\]
for $i =1,2$.
If we compute the differential of (\ref{O1cP1psP2Eq2}), we get
\begin{align}
\df{\omega_1}(\VEC{u},\VEC{x})
&= -\sin(\tilde{\theta}) \,\big(\df{\tilde{\theta}}
\wedge \rho_1\big)(\VEC{u},\VEC{x})  
+ \cos(\tilde{\theta}) \, \df{\rho_1}(\VEC{u},\VEC{x}) \nonumber \\
&\qquad \quad + \cos(\tilde{\theta})\,
\big(\df{\tilde{\theta}} \wedge \rho_2\big)(\VEC{u},\VEC{x})
+ \sin(\tilde{\theta}) \, \df{\rho_2}(\VEC{u},\VEC{x}) \nonumber \\
&= \big(\df{\tilde{\theta}} \wedge (-\sin(\tilde{\theta})\,\rho_1
+ \cos(\tilde{\theta})\, \rho_2)\big)(\VEC{u},\VEC{x})
+ \big( \cos(\tilde{\theta})\, (A_1 \circ \pi_S)(\VEC{u},\VEC{x}) \nonumber \\
&\qquad \quad 
+ \sin(\tilde{\theta}) \, (A_2 \circ \pi_S)(\VEC{u},\VEC{x}) \big)\,
\big(\omega_1 \wedge \omega_2\big)(\VEC{u},\VEC{x}) \nonumber \\
&= \big(\df{\tilde{\theta}} \wedge \omega_2\big)(\VEC{u},\VEC{x})
+ \big(\cos(\tilde{\theta}) \, (A_1 \circ \pi_S)(\VEC{u},\VEC{x}) \nonumber \\
&\qquad \quad 
+ \sin(\tilde{\theta}) \, (A_2 \circ \pi_S)(\VEC{u},\VEC{x}) \big)\,
\big(\omega_1 \wedge \omega_2\big)(\VEC{u},\VEC{x})
\label{FirstStructEq1}
\end{align}
for $(\VEC{u},\VEC{x}) \in S_c(\VEC{u})$, where we have abbreviated 
$\tilde{\theta}(\VEC{u},\VEC{x})$ to simply $\tilde{\theta}$.

Using the local chart
$(\tilde{W},\tilde{U},\tilde{\phi})$ of $S_c$ introduced in
Subsection~\ref{subsectTSc}, we get
$\displaystyle \tilde{\phi}^\ast(\tilde{\theta}) : W \times I \to \RR$
is defined by
$\displaystyle \tilde{\phi}^\ast(\tilde{\theta})(\VEC{w},\theta) = \theta$.
Hence
$\displaystyle \df{\big(\tilde{\phi}^\ast(\tilde{\theta})\big)}
(\VEC{w},\theta)\big((\VEC{w},\theta),(\VEC{r},t)\big) = t$ 
for all $\big((\VEC{w},\theta),(\VEC{r},t)\big) \in
\TS_{(\VEC{w},\theta)} (W\times I) = (W\times I) \times (\RR \times \RR)$.
Suppose that
$\big((\VEC{u},\VEC{x}),(\VEC{p},\VEC{q})\big) =
\tilde{\phi}_\ast\big((\VEC{w},\theta),(\VEC{r},t)\big) \in
\TS_{(\VEC{u},\VEC{x})} S_c$.  Using the function $G$ defined in
Example~\ref{eggConnPAp1} and the differential $1$-forms
$\omega$ and $\check{\rho}$ defined in Example~\ref{eggConnPAp2}, we get that
$\displaystyle (G^{-1}\circ \tilde{\phi})(\VEC{w},\theta) = (\VEC{u}, \eta)$
for $\displaystyle \eta = e^{i\theta}$ and
$\displaystyle (G^{-1} \circ \tilde{\phi})_\ast
\big((\VEC{w},\theta),(\VEC{r},t)\big) = 
\big((\VEC{u},\eta),(\VEC{r},t\, i \eta)\big)$.  Hence
\begin{align*}
\df{\tilde{\theta}}(\VEC{u},\VEC{x})
\big((\VEC{u},\VEC{x}),(\VEC{p},\VEC{q})\big)
&= \tilde{\phi}^\ast(\df{\tilde{\theta}})(\VEC{w},\theta)
\big((\VEC{w},\theta),(\VEC{r},t)\big)
= \df{\big(\tilde{\phi}^\ast(\tilde{\theta})}\big)(\VEC{w},\theta)
\big((\VEC{w},\theta),(\VEC{r},t)\big) \\
&= t = \omega(\eta)(\eta, t \, i\eta) = \pi_2^\ast(\omega)(\VEC{u}, \eta)
\big((\VEC{u},\eta),(\VEC{r},t\, i \eta)\big) \\
&= \pi_2^\ast(\omega)((G^{-1}\circ \tilde{\phi})(\VEC{w},\theta))
\big((G^{-1} \circ \tilde{\phi})_\ast
\big((\VEC{w},\theta),(\VEC{r},t)\big)\big) \\
& = \pi_2^\ast(\omega)(G^{-1}(\tilde{\phi}(\VEC{w},\theta)))
\big((G^{-1})_\ast\big(\tilde{\phi}_\ast
\big((\VEC{w},\theta),(\VEC{r},t)\big)\big)\big) \\
&= (G^{-1})^\ast\big(\pi_2^\ast(\omega)\big)(\tilde{\phi}(\VEC{w},\theta))
\big(\tilde{\phi}_\ast\big((\VEC{w},\theta),(\VEC{r},t)\big)\big) \\
&= (G^{-1})^\ast\big(\pi_2^\ast(\omega)\big)
(\VEC{u},\VEC{x})\big((\VEC{u},\VEC{x}),(\VEC{p},\VEC{q})\big) \\
&= \check{\rho}(\VEC{u},\VEC{x})\big((\VEC{u},\VEC{x}),(\VEC{p},\VEC{q})\big)
\end{align*}
Thus, it follows from (\ref{FirstStructEq1}) that
\begin{align*}
\df{\omega_1}(\VEC{u},\VEC{x})
&= \big(\check{\rho} \wedge \omega_2\big)(\VEC{u},\VEC{x})
+ \big(\cos(\tilde{\theta})\, (A_1 \circ \pi_S)(\VEC{u},\VEC{x}) \\
&\qquad + \sin(\tilde{\theta}) \, (A_2 \circ \pi_S)(\VEC{u},\VEC{x})
\big)\, \big(\omega_1 \wedge \omega_2\big)(\VEC{u},\VEC{x})
\end{align*}
for $(\VEC{u},\VEC{x}) \in S_c(\VEC{u})$.
We may rewrite this latest equation as
\begin{equation} \label{FirstStructEq2}
\df{\omega_1} = \check{\rho} \wedge \omega_2
+ B_1 \, \omega_1 \wedge \omega_2
\end{equation}
on $S_c(U)$, where $B_1:S_c(U) \to \RR$ is defined by
$B_1(\VEC{u},\VEC{x})
= \cos(\theta)\, (A_1 \circ \pi_S)(\VEC{u},\VEC{x})
+ \sin(\theta) \, (A_2 \circ \pi_S)(\VEC{u},\VEC{x})$
for $(\VEC{u},\VEC{x}) \in S_c(U)$.
The function $B_1$ is smooth because $A_1$, $A_2$ and $\tilde{\theta}$
are smooth.

Using (\ref{O1cP1psP2Eq3}) and proceeding as we just did above, we get
\begin{equation} \label{FirstStructEq3}
\df{\omega_2} = -\check{\rho} \wedge \omega_1
+ B_2 \, \omega_1 \wedge \omega_2
\end{equation}
on $S_c(U)$, where $B_2:S_c(U) \to \RR$ is defined by
$B_2(\VEC{u},\VEC{x}) = -\sin(\theta)\, (A_1 \circ \pi_S)(\VEC{u},\VEC{x})
+ \cos(\theta) \, (A_2 \circ \pi_S)(\VEC{u},\VEC{x})$
for $(\VEC{u},\VEC{x}) \in S_c(U)$.
As for the function $B_1$, the function $B_2$ is smooth because $A_1$,
$A_2$ and $\tilde{\theta}$ are smooth.

According to Corollary~\ref{cor2rho2do}, there exists a differential
$1$-form $\omega$ on $U$ such that
$\displaystyle \check{\rho} - \rho = \pi_S^\ast(\omega)$.  Since
$\{ \tilde{\omega}_1(\VEC{u}), \tilde{\omega}_2(\VEC{u}) \}$ is
a basis of $\displaystyle \Omega^1\big(\TS_{\VEC{u}} S\big)$
for all $\VEC{u} \in U$, we have that
$\omega = a_1 \tilde{\omega}_1 + a_2 \tilde{\omega}_2$ for some smooth
functions $a_1,a_2 : U \to \RR$.  Hence
\[
\check{\rho} - \rho = \pi_S^\ast(\omega)
= (a_1\circ \pi_S)\, \pi_S^\ast(\tilde{\omega}_1)
+ (a_2 \circ \pi_S)\, \pi_S^\ast(\tilde{\omega}_2)
= (a_1\circ \pi_S) \rho_1 + (a_2 \circ \pi_S) \rho_2
\]
on $S_c(U)$.  Since
$\{ \omega_1(\VEC{u},\VEC{x}), \omega_2(\VEC{u},\VEC{x}) \}$ is also a
basis of $\displaystyle \Omega^1\big(\TS_{(\VEC{u},\VEC{x})} S_c\big)$
for all $(\VEC{u},\VEC{x}) \in S_c(U)$, we can write
\begin{equation} \label{FirstStructEq4}
\check{\rho} - \rho = b_1 \omega_1 + b_2 \omega_2
\end{equation}
on $S_c(U)$, where $b_1,b_2 : S_c(U) \to \RR$ are two smooth
functions.  If we substitute the latest formula into
(\ref{FirstStructEq2}) and (\ref{FirstStructEq3}), we get
\begin{align*}
\df{\omega_1} &= (\rho + b_1 \omega_1 + b_2 \omega_2) \wedge \omega_2
+ B_1 \, \omega_1 \wedge \omega_2
= \rho \wedge \omega_2 + (b_1 + B_1) \, \omega_1 \wedge \omega_2 \\
&= \rho \wedge \omega_2 + c_1 \, \omega_1 \wedge \omega_2
\end{align*}
and
\begin{align*}
\df{\omega_2} &= -(\rho + b_1 \omega_1 + b_2 \omega_2) \wedge \omega_1
+ B_2 \, \omega_1 \wedge \omega_2
= -\rho  \wedge \omega_1 + (b_2 + B_2) \, \omega_1 \wedge \omega_2 \\
&= -\rho  \wedge \omega_1 + c_2 \, \omega_1 \wedge \omega_2 \ ,
\end{align*}
where $c_i = b_i + B_i:S_c(U) \to \RR$ are smooth function for $i =1,2$.

The definition of $c_1$ and $c_2$ are independent of
the local chart since $\rho$, $\omega_1$ and $\omega_2$ are
independent of the local chart.  Therefore the previous two relation
are true on $S_c$.

\begin{defn}   \label{defnFCartanEs}
The {\bfseries first (Cartan) structural equations}\index{Cartan
Structural Formulae!First Structural Equations} are the equation
\[
\df{\omega_1} = \rho \wedge \omega_2 + c_1 \, \omega_1 \wedge \omega_2 
\qquad \text{and} \qquad
\df{\omega_2} = -\rho  \wedge \omega_1 + c_2 \, \omega_1 \wedge \omega_2
\]
on $S_c$.
\end{defn}

The first structural equations can be simplified.  Obviously,
$\rho \wedge \omega_2$ in $\df{\omega_1}$ and 
$-\rho \wedge \omega_1$ in $\df{\omega_2}$ cannot be removed.
However, by selecting the appropriate connection $1$-form on $S_c$, we
may get $c_1 = c_2 = 0$ on $\TS\, S_c$.

\begin{theorem} \label{thmRconnect}
Let $S$ be an oriented $2$-dimensional Riemannian manifold.
There exists a unique connection $1$-from $\rho$ on $S_c$ such that
\[
\df{\omega_1} = \rho \wedge \omega_2
\qquad \text{and} \qquad
\df{\omega_2} = -\rho  \wedge \omega_1
\]
on $S_c$.  This connection $1$-form is called the
{\bfseries Riemannnian connection}\index{Riemannian Connection}.
\end{theorem}

\begin{proof}
Let $\eta_1$ be a connection $1$-form on $S_c$.  Note that
connection $1$-forms can be generated using
Proposition~\ref{propCIfInv} for instance.
Let $\eta_2$ be another connection $1$-form on $S_c$.  Proceeding as we did to
deduce (\ref{FirstStructEq4}), we get
\begin{equation} \label{FirstStructEq5}
  \eta_1 - \eta_2 = d_1 \omega_1 + d_2 \omega_2
\end{equation}
on $S_c(U)$ for some smooth functions $d_1,d_2 : S_c(U) \to \RR$.
If we substitute $\rho = \eta_1$ in the first structural
equations, then we get
\begin{equation} \label{FirstStructEq6}
\df{\omega_1} = (\eta_2 + d_1 \omega_1 + d_2 \omega_2 ) \wedge \omega_2
+ c_1 \, \omega_1 \wedge \omega_2
= \eta_2 \wedge \omega_2 + (c_1 + d_1)\, \omega_1 \wedge \omega_2
\end{equation}
and
\begin{equation} \label{FirstStructEq7}
\df{\omega_2} = -(\eta_2 + d_1 \omega_1 + d_2 \omega_2 ) \wedge \omega_1
+ c_2 \, \omega_1 \wedge \omega_2
= -\eta_2 \wedge \omega_1 + (c_2 + d_2) \, \omega_1 \wedge \omega_2 \ .
\end{equation}
Therefore, we get that $\df{\omega_1} = \eta_2 \wedge \omega_2$ 
and $\df{\omega_2} = -\eta_2  \wedge \omega_1$ if we select
$\eta_2 = \eta_1 - d_1 \omega_1 - d_2 \omega_2$ with
$d_1 = -c_1$ and $d_2 = -c_2$.

To prove uniqueness, suppose that $\eta_3$ is a connection $1$-form on
$S_c$ such that $\df{\omega_1} = \eta_3 \wedge \omega_2$ 
and $\df{\omega_2} = -\eta_3  \wedge \omega_1$.  Instead of
(\ref{FirstStructEq5}), we now have that
$\eta_1 - \eta_3 = \tilde{d}_1 \omega_1 + \tilde{d}_2 \omega_2$
for some smooth functions $\tilde{d}_1,\tilde{d}_2 : S_c(U) \to \RR$.
Moreover, (\ref{FirstStructEq6}) and
(\ref{FirstStructEq7}) become
\[
\df{\omega_1} = \eta_3 \wedge \omega_2
+ (c_1 + \tilde{d}_1)\, \omega_1 \wedge \omega_2
= \df{\omega_1} + (c_1 + \tilde{d}_1)\, \omega_1 \wedge \omega_2
\]
and
\[
\df{\omega_2} = -\eta_3 \wedge \omega_1
+ (c_2 + \tilde{d}_2) \, \omega_1 \wedge \omega_2
= \df{\omega_2} + (c_2 + \tilde{d}_2)\, \omega_1 \wedge \omega_2 \ .
\]
Thus $\tilde{d}_1 = - c_1$ and $\tilde{d}_2 = -c_2$.  Hence $\eta_3 = \eta_2$.
\end{proof}

\subsection{Second (Cartan) Structural Equation}

Let $(W,U,\phi)$ be a local charts of $S$.
As in the proof of Proposition~\ref{propHLexists},
we consider the connection $\displaystyle \check{\MM}_c
= \{ \check{M}_{(\VEC{u},\VEC{x})}\}_{(\VEC{u},\VEC{x}) \in S_c(U)}$
on $S_c(U)$ with its associated connection $1$-from $\check{\rho}$
defined in Examples~\ref{eggConnPAp1} and \ref{eggConnPAp2}.
According to Corollary~\ref{cor2rho2do}, there exists a
differential $1$-form $\omega$ on $U$ such that
$\displaystyle \check{\rho} - \rho = (\pi_S)^\ast(\omega)$.

Since $\df{\check{\rho}} = 0$ as we have seen at the end of
Example~\ref{eggConnPAp2}, we have
\[
\df{\rho} = \df{(\rho - \tilde{\rho})} = - \df{((\pi_S)^\ast(\omega))}
= -(\pi_S)^\ast(\df{\omega})
\]
for some differential $1$-form $\omega$ on $U$.  Since $\df{\omega}$
is a differential $2$-form on $U$, we have that
$\df{\omega} = \kappa \dx{V}$ for some smooth function $\kappa:U \to \RR$.
Hence,
\begin{equation} \label{SecondStructEq1}
\df{\rho} = - (\pi_S)^\ast(\kappa \dx{V})
= - (\pi_S)^\ast(\kappa)\, (\pi_S)^\ast(\dx{V}))
= - (\pi_S)^\ast(\kappa)\, \omega_1 \wedge \omega_2 
\end{equation}
on $S_c(U)$.  Note that the definition of $\kappa:U\to \RR$ is independent of
the local chart since $\rho$, $\omega_1$ and $\omega_2$ are
independent of the local chart.  Therefore (\ref{SecondStructEq1}) is true on
$S_c$.

\begin{defn} \label{defnSecondCSE}
The {\bfseries second (Cartan) structural equation}\index{Cartan
Structural Formulae!Second Structural Equation} is the equation
\[ %\begin{equation} \label{SecondStructEq2}
  \df{\rho} = -(\kappa \circ \pi_S)\, \omega_1 \wedge \omega_2
\] %\end{equation}
on $S_c$.  The function $\kappa$ is called the
{\bfseries curvature}\index{Curvature} associated to the connection
$1$-form $\rho$.
\end{defn}

\begin{egg}
Suppose that $S$ is an open subset of     \label{eggEuclk0}
$\displaystyle \RR^2$ with the
standard Euclidean norm and orientation.  This is an oriented
$2$-dimensional Riemannian manifold with the Riemann metric given by
$\ps{(\VEC{u},\VEC{x})}{(\VEC{u},\VEC{y})}_{\VEC{u}} = \ps{\VEC{x}}{\VEC{y}}$
for all
$\displaystyle (\VEC{u},\VEC{x}),(\VEC{u},\VEC{y})
\in \TS_{\VEC{u}} S \cong S \times \RR^2$,
where the last inner product is the standard inner product in
$\displaystyle \RR^3$.

We use the construction given in Examples~\ref{eggConnPAp1} and
\ref{eggConnPAp2}.  Note that the local construction presented in
these examples can be use globally in the present example.  Namely, we
can use the trivial local chart $(W,U,\phi)$ with
$\displaystyle W = U = S$ and $\phi=\Id$.

The map $\displaystyle G:U \times S^1 \to S_c(U)$ in
Example~\ref{eggConnPAp1} is now defined by
$G(\VEC{u},\eta) = \eta(\VEC{u}, \VEC{e}_1) = (\VEC{u}, \theta.\VEC{e}_1)$
where $\displaystyle \eta = e^{i\theta} \in S^1$.  We then get a
connection on $S_c$ defined by\\
$\displaystyle \check{\MM}_c = \{
\check{M}_{\eta(\VEC{u},\VEC{e}_1)} \}_{\eta \in S^1, \VEC{u} \in S}$
where
$\displaystyle \check{M}_{\eta(\VEC{u},\VEC{e}_1)} = (G_\eta)_\ast(\TS_{\VEC{u}} S)
= \{ \big( \eta(\VEC{u},\VEC{e}_1), (\VEC{p},\VEC{0}) \big) : \VEC{p}
\in \RR^2 \}$ for all $\displaystyle \eta \in S^1$ and $\VEC{u} \in S$.

We have shown in Example~\ref{eggConnPAp2} that the connection $1$-form
associated to $\check{\MM}_c$ is
$\displaystyle \check{\rho} = (G^{-1})^\ast(\pi_2^\ast(\omega))$ where
$\omega$ is the differential $1$-form on $\displaystyle S^1$ defined by
$\omega(\eta)(\eta,t \, i\eta) = t$ for all
$\displaystyle \eta \in S^1$ and $t\in \RR$.
Note that $\displaystyle \TS_\eta S^1 = \{ (\eta,t\, i \eta) : t \in \RR \}$.
Moreover, we have also introduced in Example~\ref{eggConnPAp2} the map
$\displaystyle \pi_2:U \times S^1 \to S^1$ defined by
$\pi_2(\VEC{u},\eta) = \eta$ for all
$\displaystyle (\VEC{u},\eta) \in U \times S^1$.  
Given $\big((\VEC{u},\VEC{x}),(\VEC{p},\VEC{q})\big) \in
\TS_{(\VEC{u},\VEC{x})} S_c$, there exists
$\displaystyle \big((\VEC{u},\eta),(\VEC{p},t\, i \eta)\big)
\in \TS_{(\VEC{u},\eta)} (U \times S^1)$ with $\displaystyle \eta = e^{i\theta}$ 
such that
\begin{align*}
\big((\VEC{u},\VEC{x}),(\VEC{p},\VEC{q})\big)
&= G_\ast\big((\VEC{u},\eta),(\VEC{p},t\, i \eta)\big)
= \Big( \eta(\VEC{u},\VEC{e}_1) , \Big(\VEC{p},
t\, \dfdx{(\theta.\VEC{e}_1)}{\theta} \Big) \Big) \\
&= \big( \eta(\VEC{u},\VEC{e}_1) , (\VEC{p}, t\,\theta.\VEC{e}_2) \big) \ .
\end{align*}
Hence
\begin{align*}
\check{\rho}(\VEC{u},\VEC{x})\big((\VEC{u},\VEC{x}),(\VEC{p},\VEC{q})\big)
&= (G^{-1})^\ast(\pi_2^\ast(\omega))(\VEC{u},\VEC{x})
\big((\VEC{u},\VEC{x}),(\VEC{p},\VEC{q})\big) \\
&= \pi_2^\ast(\omega)(\VEC{u},\eta)
\big( (\VEC{u},\eta) , (\VEC{p}, t\, i \eta \big)  
= \omega(\eta)\big( \eta , t i \eta \big) = t
\end{align*}
as we have shown in Example~\ref{eggConnPAp2}.

Consider the map $\tilde{\theta}: S_c \to \RR$ defined at the beginning 
of Section~\ref{sectRiemannCurv}.
Using the local charts $(\tilde{W},\tilde{U},\tilde{\phi})$ of $S_c$,
in particular
$\displaystyle \tilde{\phi}(\VEC{u},\theta) = e^{i\theta}(\VEC{u},\VEC{e}_1)
= (\VEC{u},\theta.\VEC{e}_1)$ because $\displaystyle W = U = S$, we
get that $(\tilde{\theta}\circ \tilde{\phi})(\VEC{u},\theta) = \theta$
for all $(\VEC{u},\theta) \in \tilde{U}$ and
\begin{align*}
\df{\tilde{\theta}}(\VEC{u},\VEC{x})
\big((\VEC{u},\VEC{x}),(\VEC{p},\VEC{q})\big)
&= \df{\tilde{\theta}}(\VEC{u},\theta.\VEC{e}_1)\big((\VEC{u},\theta.\VEC{e}_1),
(\VEC{p}, t\, \theta.\VEC{e}_2) \big) \\
&= \df{\tilde{\theta}}(\check{\phi}(\VEC{u},\theta))\big(
\check{\phi}_\ast\big( (\VEC{u},\theta), (\VEC{r}, t)\big)\big)
= \check{\phi}^\ast(\df{\tilde{\theta}})(\VEC{u},\theta)
\big( (\VEC{u},\theta), (\VEC{r}, t)\big) \\
&= \df{(\tilde{\theta}\circ \check{\phi})}(\VEC{u},\theta)
\big( (\VEC{u},\theta), (\VEC{r}, t)\big)
= \diff_{\VEC{u},\theta} (\tilde{\theta} \circ \tilde{\phi})(\VEC{u},\theta)
\begin{pmatrix} \VEC{r} & t \end{pmatrix}^\top \\
&= \begin{pmatrix}\VEC{0} & 1\end{pmatrix}
\begin{pmatrix} \VEC{r} & t \end{pmatrix}^\top = t
\end{align*}
where $\displaystyle \VEC{r} \in \RR^2$ is such that
$\VEC{p} = \diff \check{\phi}(\VEC{u}) \VEC{r}$.
Since $\big((\VEC{u},\VEC{x}),(\VEC{p},\VEC{q})\big) \in
\TS_{(\VEC{u},\VEC{x})} S_c$ is arbitrary, we therefore have that
$\df{\tilde{\theta}} = \check{\rho}$.  It is again consistent with the
result that we have seen in the previous section. 

To show that $\check{\rho}$ is the Riemann connection, we use
(\ref{O1cP1psP2Eq2}) and (\ref{O1cP1psP2Eq3}).  We have
\[
\omega_1(\VEC{u},\VEC{x}) = \cos(\tilde{\theta})\, \rho_1(\VEC{u},\VEC{x})
+ \sin(\tilde{\theta})\, \rho_2(\VEC{u},\VEC{x})
\]
and
\[
\omega_2(\VEC{u},\VEC{x}) = -\sin(\tilde{\theta})\, \rho_1(\VEC{u},\VEC{x})
+ \cos(\tilde{\theta})\, \rho_2(\VEC{u},\VEC{x})
\]
where $\tilde{\theta} = \tilde{\theta}(\VEC{u},\VEC{x})$,
$\displaystyle \rho_i = (\pi_S)^\ast(\tilde{\omega}_i)$ for $i=1,2$, and
$\{\tilde{\omega}_1,\tilde{\omega}_2\}$ is the dual basis associated to
the basis $\{ (\VEC{u}, \VEC{e}_1) , (\VEC{u} , \VEC{e}_2) \}$ of
$\TS_{\VEC{u}} S$.  We have for $i=1,2$ that
$\tilde{\omega}_i = \df{x_i}$ on $S$ because
$\tilde{\omega_i}(\VEC{u})\big(\VEC{u},\VEC{p}\big)
= \pi_i(\VEC{p}) = \df{x_i}(\VEC{u})(\VEC{u},\VEC{p})$
for all $(\VEC{u},\VEC{p}) \in \TS_{\VEC{u}} S$.  We may therefore
write
\[
\omega_1(\VEC{u},\VEC{x})
= \cos(\tilde{\theta})\, (\pi_S)^\ast(\df{x_1})(\VEC{u},\VEC{x})
+ \sin(\tilde{\theta})\, (\pi_S)^\ast(\df{x_2})(\VEC{u},\VEC{x})
\]
and
\[
\omega_2(\VEC{u},\VEC{x}) = -\sin(\tilde{\theta})\,
(\pi_S)^\ast(\df{x_1})(\VEC{u},\VEC{x})
+ \cos(\tilde{\theta})\, (\pi_S)^\ast(\df{x_2})(\VEC{u},\VEC{x}) \ .
\]
Hence
\begin{align*}
\df{\omega_1}
&= -\sin(\tilde{\theta})\, \df{\tilde{\theta}} \wedge (\pi_S)^\ast(\df{x_1})
+ \cos(\tilde{\theta})\, \df{\big((\pi_S)^\ast(\df{x_1})\big)} \\
&\qquad \quad + \cos(\tilde{\theta})\, \df{\tilde{\theta}}
\wedge (\pi_S)^\ast(\df{x_2})
+ \sin(\tilde{\theta})\, \df{\big((\pi_S)^\ast(\df{x_2})\big)} \\
&= \df(\tilde{\theta}) \wedge \big(-\sin(\tilde{\theta})\,(\pi_S)^\ast(\df{x_1})
+ \cos(\tilde{\theta})\, (\pi_S)^\ast(\df{x_2}) \big) \\
&\qquad \quad + \cos(\tilde{\theta})\, (\pi_S)^\ast(\df[2]{x_1})
+ \sin(\tilde{\theta})\, (\pi_S)^\ast(\df[2]{x_2})
= \df{\tilde{\theta}} \wedge \omega_2
\end{align*}
and similarly
\[
  \df{\omega_2} = - \df{\tilde{\theta}} \wedge \omega_1 \ .
\]
It then follows from Theorem~\ref{thmRconnect} that
$\check{\rho} = \df{\tilde{\theta}}$ is the Riemann connection on $S_c$.

We have already seen at the end of Example~\ref{eggConnPAp2} that
$\df{\check{\rho}} = 0$.  Hence, we get from the second structural
equation that the curvature $\kappa$ satisfies $\kappa = 0$ on all of
$\displaystyle S$.
\end{egg}

\begin{rmk}
Using the observation made in Remark~\ref{rmkHeptheta} and the
associated interpretation of vector field given in
Remark~\ref{rmkVFwithOper}, it is clear that
$\check{\rho} = \df{\theta}$ because
$\displaystyle \df{\theta}\Big(\pdydx{}{\theta}\Big) =
\pdydx{\theta}{\theta} = 1$ and
$\displaystyle \df{\theta}\Big(\pdydx{}{w_i}\Big) =
\pdydx{\theta}{w_i} = 0$ for $i = 0,1$ since $\theta$
does not depend on $\VEC{w}$ (i.e. on the base point $\VEC{u}$.)
\end{rmk}

\section{Geodesic} \label{sectGeodesic}

\begin{defn}
Let $S$ be a $2$-dimensional Riemannian manifold and
$\sigma:[a,b] \to S$ be a smooth curve.
Let $\sigmaU:[a,b] \to S_c$ be the smooth curve defined by
$\sigmaU(t) = (\sigma(t),\sigma'(t))$ for $a \leq t \leq b$.
We say that $\sigmaU$ is a {\bfseries geodesic}\index{Geodesic}
in $S$ if $\sigmaU$ is the horizontal lift $\tilde{\sigma}$ of
$\sigma$ with $\tilde{\sigma}(a) = (\sigma(a),\sigma'(a))$.
\end{defn}

If $\sigmaU$ is a geodesic, then
$(\sigma(t),\sigma'(t))  \in S_c(\sigma(t))$ and therefore
$\|\sigma'(t)\|_{\sigma(t)}= 1$ for all $t$, where
$\|\cdot\|_{\VEC{u}}$ for $\VEC{u} \in S$ is given by the Riemannian
metric on $S$.  Thus $\sigma$ is parameterized by arc length.

\begin{defn} \label{defnGeoCruv}
Let $S$ be a $2$-dimensional Riemannian manifold and
$\sigma:[a,b] \to S$ be a smooth curve parameterized by arc length.
Moreover, let $\rho$ be the Riemannian connection on $S_c$ and
$\sigmaU:[a,b] \to S_c$ be the smooth curve defined by
$\sigmaU(t) = (\sigma(t),\sigma'(t))$ for $a \leq t \leq b$.
The {\bfseries geodesic curvature}\index{Geodesic Curvature} of $\sigma$
at $t$, denoted $\kappa_{\sigma}(t)$, is defined by
$\kappa_{\sigma}(t) = \rho(\sigmaU(t))\big(\sigmaU_\ast(t,1)\big)$
for $a \leq t \leq b$.
\end{defn}

We say that $\sigma:[a,b] \to S$ is a ``straight line'' in $S$ if
$\kappa_\sigma(t) = 0$ for $0 \leq t \leq b$.  It then follows from
Proposition~\ref{propHLexists} that $\sigmaU$ is the horizontal
lift $\tilde{\sigma}$ of $\sigma$ with
$\tilde{\sigma}(a) = (\sigma(a),\sigma'(a))$;
namely, $\sigmaU$ is a geodesic.  In such a situation, the
parallel translation of $\sigmaU(a)$ along $\sigma$ to $\sigma(t)$ is
given by $(\sigma(t),\sigma'(t))$ itself.  In the literature, geodesic
are generally defined as curves with null geodesic curvature.

We have two definitions of curvature for a curve: one given in
(\ref{DGcurvature}) and the geodesic curvature defined above.
We prove that the curvature defined by (\ref{DGcurvature}) is the
geodesic curvature.  Without loss of generality, we may assume that
$\displaystyle \sigma:[a,b]\to S \subset \RR^3$ is a smooth curve
parameterized by arc length.

We have that $\sigmaU_\ast(t,1) = \big( (\sigma(t), \sigma'(t)),
(\sigma'(t), \sigma''(t)\big)$ for $a \leq t \leq b$.
We also have for $(\VEC{u},\VEC{x}) \in S_c(\VEC{u})$
that $\rho(\VEC{u},\VEC{x})\big( (\VEC{u},\VEC{x}), (\VEC{p},\VEC{q})\big)
= \ps{\big((\VEC{u},\VEC{x}),
(\VEC{p},\VEC{q})\big)}{H(\VEC{u},\VEC{x})}_{(\VEC{u},\VEC{x})}$ for\\
$\big( (\VEC{u},\VEC{x}), (\VEC{p},\VEC{q})\big) \in
\TS_{(\VEC{u},\VEC{x})} S_c$ where we are using the metric
on $\TS_{\VEC{u},\VEC{x}} S_c$ defined by
\[
\ps{\big((\VEC{u},\VEC{x}),(\VEC{p}_1,\VEC{q}_1)\big)}
{\big((\VEC{u},\VEC{x}) ,(\VEC{p}_2, \VEC{q}_2)\big)}_{(\VEC{u},\VEC{x})}
= \ps{\VEC{p}_1}{\VEC{p}_2} + \ps{\VEC{q}_1}{\VEC{q}_2}
= \VEC{p}_1\cdot \VEC{p}_2 + \VEC{q}_1 \cdot \VEC{q}_2
\]
for all $\big((\VEC{u},\VEC{x}),(\VEC{p}_i,\VEC{q}_i)\big)\in
\TS_{(\VEC{u},\VEC{x})} S_c$ and $i =1,2$.
If we derive $\displaystyle \|\sigma'(t)\|^2=1$ with respect to $t$, then
we get that $\sigma'(t) \cdot \sigma''(t) = 0$ for all $t$.
Thus $\sigma''(t)$ is perpendicular to $\sigma'(t)$ for all $t$.
Hence the fourth component of
$\displaystyle H(\sigma(t),\sigma'(t))$ is $\|\sigma''(t) \|^{-1} \,
\sigma''(t)$; namely, 
\[
H(\sigma(t),\sigma'(t))
= \big((\sigma(t),\sigma'(t)) , \big(0, \|\sigma''(t) \|^{-1}
\, \sigma''(t) \big) \big)
\]
(Figure~\ref{GeodFig0}).  
We find that the geodesic curvature of $\sigma$ at $t$ is
\begin{align*}
\kappa_{\sigma}(t) &= \rho(\sigmaU(t))(\sigmaU_\ast(t,1)) \\
&= \ps{\big((\sigma(t),\sigma'(t)),(\sigma'(t), \sigma''(t))\big)}
{\big((\sigma(t),\sigma'(t)) ,\big(0,\|\sigma''(t)\|^{-1}
\, \sigma''(t) \big) \Big)}\\
&= \|\sigma''(t) \|^{-1} \, \sigma''(t) \cdot \sigma''(t) = \|\sigma''(t)\|
\end{align*}
for $a \leq t \leq b$.  This is (\ref{DGcurvature}) for a smooth curve
$\sigma$ parameterized by arc length.  Note that we have made used of
the fact that $\|\sigma''(t) \times \sigma'(t)\|
= \|\sigma'(t)\|\, \|\sigma''(t)\| \sin(\pi/2) = \|\sigma''(t)\|$
because $\sigma'(t)$ is perpendicular to $\sigma''(t)$, so the angle between
them is $\pi/2$.

\pdfF{riemann_geom/geodfig0}{Representation of
$H(\sigmaU(t))$ along a smooth curve parameterized by arc length}
{Representation of $H(\sigmaU(t))$ along a smooth curve
parameterized by arc length.  The last component of
$H(\sigmaU(t)) \in \TS_{\sigmaU(t)} S_c$
is represented by the red arrow.  $C$ is a circle
of radius $1$ centred at the origin in $\TS_{\sigma(t)} S$}{GeodFig0}

\section{Gauss-Bonnet Theorem}

In this section, we use a few concepts from
Sections~\ref{sectSimplHomol} and \ref{sectSimplCohom} on simplicial
homology and cohomology. 
The goal of this section is to prove the following amazing theorem.

\begin{theorem}[Gauss-Bonnet] \label{thmGaussBonnet}
Let $(S,K,h)$ be a connected, oriented and smoothly triangulated
$2$-dimensional Riemannian manifold.
Then $\displaystyle \frac{1}{2\pi} \int_S \kappa \dx{V} = \Chi(K)$
where $\Chi(K)$ is the the Euler characteristic of the simplicial complex $K$
used to triangulate $S$.
\end{theorem}

Recall that we have from Theorem~\ref{thmEulerAlpha} that
$\Chi(K) = \alpha_0 - \alpha_1 + \alpha_2$ where $\alpha_0$ is the
number of vertices in $K$ (i.e.\ $0$-simplices), $\alpha_1$ is the number of
edges in $K$ (i.e.\ $1$-simplices), and $\alpha_2$ is the number of
faces or tirangles in $K$ (i.e. $2$-simplices).

Since, for us, simplicial complexes contain only a finite number of
elements, it follows from the definition of smoothly triangulated
manifold that the manifolds to which the Gauss-Bonnet above can be
applied to are compact manifolds like a sphere, a torus, etc.

We need to start with a fairly technical result.

\begin{prop} \label{propRotSimpl}
Let $S$ be an oriented $2$-dimensional Riemannian manifold, and
$\os{s}{}{}{}{} = \os{\VEC{v}_0}{\VEC{v}_1}{}{}{\VEC{v}_3}$
be an oriented $2$-simplex.  Suppose that
\begin{enumerate}
\item $V$ is an open set such that $\displaystyle [s] \subset V \subset \RR^2$,
\item $h:V \to S$ is a smooth, non-singular \footnotemark, and orientation
preserving function,
\item $\displaystyle \sigma:[a,b] \to S$ is a parametric representation of
the piecewise $\displaystyle C^\infty$ closed curve
$h(\partial\, [s])$ such that the orientation on $\sigma$ is the
orientation on $h(\partial\, [s])$ induced via $h$
by $\partial_2 \os{s}{}{}{}{}$, and
\item $\tilde{\sigma}:[a,b] \to S_c$ is
a horizontal lift of $\sigma$ (Figure~\ref{GBfig1}).
\end{enumerate}
Then $\eta(\tilde{\sigma}(b)) = \tilde{\sigma}(a)$ for
$\displaystyle \eta = e^{i \theta} \in S^1$ with
$\displaystyle \theta = -\int_{h([s])} \kappa \dx{V}$.
\end{prop}

\footnotetext{$h$ non-singular means that
$\diff h(\VEC{u})$ is of rank $2$ for all $\VEC{u} \in V$.}

Suppose that $\os{\VEC{x}_{i_0}}{}{}{}{\VEC{x}_{i_1}}$
is the oriented $1$-simplex associated to a $1$-simplex
$(\VEC{x}_{i_0},\VEC{x}_{i_1})$ of a simplicial complex $K$.
For convenience, we adopted the convention that
$\displaystyle \int_{\osscript{\VEC{x}_{i_0}}{}{}{}{\VEC{x}_{i_1}}} = 
\int_\alpha$ where $\alpha$ is the singular $1$-cube defined by
$\alpha(t) = \VEC{x}_{i_0} + t (\VEC{x}_{i_1} - \VEC{x}_{i_0})$ for
$0 \leq t \leq 1$ (or any equivalent parametric representation that
respect the orientation).  More generally, suppose that
$\os{\VEC{x}_{i_0}}{\VEC{x}_{i_1}}{}{}{\VEC{x}_{i_2}}$
is the oriented $2$-simplex associated to a $2$-simplex 
$(\VEC{x}_{i_0},\VEC{x}_{i_1},\VEC{x}_{i_2})$ of a simplicial complex
$K$, then $\displaystyle
\int_{\partial_s \osscript{\VEC{x}_{i_1}}{\VEC{x}_{i_2}}{}{}{\VEC{x}_{i_3}}}
= \int_\alpha$ where
\[
\sigma(t) =
\begin{cases}
\VEC{x}_{i_0} + t(\VEC{x}_{i_1} - \VEC{x}_{i_0}) & \quad \text{if}
\ 0\leq t \leq 1 \\
\VEC{x}_{i_1} + (t-1)(\VEC{x}_{i_2} - \VEC{x}_{i_1}) & \quad \text{if}
\ 1 < t \leq 2 \\
\VEC{x}_{i_2} + (t-2)(\VEC{x}_{i_0} - \VEC{x}_{i_2}) & \quad \text{if}
\ 2 < t \leq 3
\end{cases}
\]
(or any equivalent parametric representation that respect the orientation).
We have
$\displaystyle
\int_{\partial_s \osscript{\VEC{v}_{i_1}}{\VEC{v}_{i_2}}{}{}{\VEC{v}_{i_3}}}
= \int_{\osscript{\VEC{v}_{i_0}}{}{}{}{\VEC{v}_{i_1}}}
- \int_{\osscript{\VEC{v}_{i_0}}{}{}{}{\VEC{v}_{i_2}}}
+ \int_{\osscript{\VEC{v}_{i_1}}{}{}{}{\VEC{v}_{i_2}}}$.
All of these results could have come straight from
Section~\ref{stokesBDRS} if instead of using singular $k$-cubes we had
used triangles as it is sometime done in the literature.  The two
approaches are equivalent and have their own advantages and
disadvantages.

\pdfF{riemann_geom/gbfig1}{First figure associated to the proof of
Gauss-Bonnet theorem}{This is the first figure associated to the proof
of Proposition~\ref{propRotSimpl}.  The curve $\sigma$ is in red.}{GBfig1}

\begin{proof}[Proof (of Proposition~\ref{propRotSimpl})]
Let $\rho$ be the Riemannian connection on $S_c$ given by
Theorem~\ref{thmRconnect}.

Without loss of generality, we may use the parametric
representation $\displaystyle \sigma:[0,3] \to S$ of
$h(\partial [s])$ defined by
\[
\sigma(t) =
\begin{cases}
h(\VEC{v}_0 + t(\VEC{v}_1 - \VEC{v}_0)) & \quad \text{if}
\ 0\leq t \leq 1 \\
h(\VEC{v}_1 + (t-1)(\VEC{v}_2 - \VEC{v}_1)) & \quad \text{if}
\ 1 < t \leq 2 \\
h(\VEC{v}_2 + (t-2)(\VEC{v}_0 - \VEC{v}_2)) & \quad \text{if}
\ 2 < t \leq 3
\end{cases}
\]
Suppose that
\begin{enumerate}
\item $\tilde{\sigma}(i) = (\sigma(i),\VEC{x}_i) \in S_c(\sigma(i))$
for $0 \leq i \leq 2$,
\item $\tilde{\nu}_r$ for $0 \leq r \leq 1$ is the horizontal lift of
the curve $\nu_r:[0,1] \to S$ defined by
$\nu_r(t) = h\big(\VEC{v}_1 + t\big(\VEC{v}_0 + r(\VEC{v}_2 - \VEC{v}_0)
-\VEC{v}_1\big)\big)$ for $0 \leq t \leq 1$ such that
$\tilde{\nu}_r(0) = (\sigma(1),\VEC{x}_1)$, and
\item $\tilde{h}:[s] \to S_c$ is the map defined by
$\tilde{h}(\VEC{v}) = \tilde{\nu}_r(t)$ for
$\VEC{v} = \VEC{v}_1 + t\big(\VEC{v}_0 + r(\VEC{v}_2 - \VEC{v}_0)
-\VEC{v}_1\big)$ with $0 \leq r,t \leq 1$.
\end{enumerate}
All these assumptions are summarized in Figure~\ref{GBfig2}.

\pdfF{riemann_geom/gbfig2}{Second figure associated to the proof of
Gauss-Bonnet theorem}{This is the second figure associated to the proof
of Proposition~\ref{propRotSimpl}.  We provide a schematic representation of
$S_c$ which should not be literally interpreted as what $S_c$ look
like.  The curve $\tilde{\tau}$ is in red.}{GBfig2}

We have that $\pi_S\circ \tilde{h} = h$ on $[s]$.  Moreover,
$\tilde{h}$ is smooth and can be extended to a smooth map from an open
set containing $[s]$ to $S_c$.  We only sketch the proof of this claim. 
To prove that $\tilde{h}$ is a smooth function, a close look at the
beginning of the proof of Proposition~\ref{propHLexists} shows that
$\tilde{\nu}_r(t)$ is given locally by
$\tilde{\nu}_r(t) = \eta_{t,r}(\check{\sigma}_r(t))$
where $\displaystyle (t,r) \mapsto \eta_{t,r} \in S^1$ is smooth
and $\check{\sigma}_r(t) = G_1(\nu_r(t))$.  It suffices to replace $\sigma$
in that proof by $\nu_r$.  The smooth extension of $\tilde{h}$ to an open
neighbourhood of $[s]$ is illustrated in Figure~\ref{GBfig3}.

\pdfF{riemann_geom/gbfig3}{Third figure associated to the proof of
Gauss-Bonnet theorem}{This is the third figure associated to the proof
of Proposition~\ref{propRotSimpl}.  The extension of $h$ and $\tilde{h}$ are
given by the extension of the curves $\nu_r$ and the extension of
their horizontal lifts shown in red.}{GBfig3}

We have
\begin{align*}
\int_{h([s])} \kappa \dx{V}
&= \int_{[s]} h^\ast(\kappa \dx{V})
= \int_{[s]} (\pi_S \circ \tilde{h})^\ast\big(\kappa \dx{V}\big)
= \int_{[s]} \tilde{h}^\ast\big((\pi_S)^\ast(\kappa)
(\pi_S)^\ast(\dx{V})\big) \\
&= \int_{[s]} \tilde{h}^\ast\big((\kappa \circ\pi_S)
\, \omega_1 \wedge \omega_2 \big)
= -\int_{[s]} \tilde{h}^\ast(\df{\rho})
= -\int_{[s]} \df{(\tilde{h}^\ast(\rho))} \\
&= -\int_{\partial \osscript{s}{}{}{}{}} \tilde{h}^\ast(\rho)
= -\int_{\osscript{\VEC{v}_0}{}{}{}{\VEC{v}_1}} \tilde{h}^\ast(\rho)
-\int_{\osscript{\VEC{v}_1}{}{}{}{\VEC{v}_2}} \tilde{h}^\ast(\rho)
- \int_{\osscript{\VEC{v}_2}{}{}{}{\VEC{v}_0}} \tilde{h}^\ast(\rho) \ ,
\end{align*}
where we have use the second structural equation to obtain the fifth
equality and Stokes' theorem to obtain the seventh equality.

Let $\tilde{\tau}:[0,3] \to S_c$ be the curve defined by
\[
\tilde{\tau}(t) =
\begin{cases}
\tilde{\nu}_0(1-t) & \quad \text{if} \ 0\leq t \leq 1 \\
\tilde{\nu}_1(t-1) & \quad \text{if} \ 1 < t \leq 2 \\
\tilde{\nu}_{3-t}(1) & \quad \text{if} \ 2 < t \leq 3
\end{cases}
\]
(Figure~\ref{GBfig2}).  By uniqueness of the horizontal lift as we
have seen in Proposition~\ref{propHLexists}, we have
that $\tilde{\tau}(t) = \tilde{\sigma}(t)$ for $0 \leq t \leq 2$.
It suffices to compare the horizontal lifts of $\nu_0$ and $\nu_1$
with the horizontal lift of $\sigma$ with the initial condition
$(\sigma(1),\VEC{x}_1)$ at $t=0$.  We get that
$\tilde{\nu}_0(t) = \tilde{\sigma}(1-t)$ and
$\tilde{\nu}_1(t) = \tilde{\sigma}(t+1)$ for $0\leq t \leq 1$.

If $\alpha:[0,1] \to [\VEC{v}_0,\VEC{v}_1]$ is the parametric
representation given by
$\alpha(t) = \VEC{v}_0 + t(\VEC{v}_1-\VEC{v}_0)$ for $0\leq t \leq 1$, then
\begin{align*}
\int_{\osscript{\VEC{v}_0}{}{}{}{\VEC{v}_1}} \tilde{h}^\ast(\rho)
&= \int_0^1 \alpha^\ast\big(\tilde{h}^\ast(\rho)\big) \dx{t}
= \int_0^1 (\tilde{h} \circ \alpha)^\ast(\rho) \dx{t}
= \int_0^1 \tilde{\tau}^\ast(\rho) \dx{t} \\
&= \int_0^1 \rho(\tilde{\tau}(t))\big(\tilde{\tau}_\ast(t,1)\big) \dx{t} = 0
\end{align*}
because $\tilde{\tau}$ on $[0,1]$ is a horizontal lift of $\sigma$ and thus
$\displaystyle \rho(\tilde{\tau}(t)) \big(\tilde{\tau}_\ast(t,1)\big) = 0$
for $t\in [0,1]$ (see Proposition~\ref{propHLexists}).  For the
same reason,
\[
\int_{\osscript{\VEC{v}_1}{}{}{}{\VEC{v}_2}} \tilde{h}^\ast(\rho) = 0 \ .
\]

Since $\tilde{\sigma}$ and $\tilde{\tau}$ are two lifts of $\sigma$ on
$[2,3]$ such that
$\tilde{\tau}(2) = \tilde{\sigma}(2) = (\sigma(2),\VEC{x}_2)$, we have
from Proposition~\ref{proUnHorLift} that there exists a smooth function
$\theta:[2,3]\to \RR$ such that $\theta(2) = 0$,
$\tilde{\tau}(t) = \eta_t\big(\tilde{\sigma}(t)\big)$ with
$\displaystyle \eta_t = e^{i\theta(t)} \in S^1$ for $2\leq t \leq 3$, and
$\rho(\tilde{\tau}(t))\big(\tilde{\tau}_\ast(t,1)\big) = \theta'(t)$
for $2\leq t \leq 3$.  It follows that
$\eta_3(\tilde{\sigma}(3)) = \tilde{\tau}(3) = \tilde{\nu}_0(1)
= \tilde{\sigma}(0)$.

Moreover, if $\alpha:[2,3] \to [\VEC{v}_0,\VEC{v}_2]$ is the parametric
representation given by
$\alpha(t) = \VEC{v}_2 + (t-2)(\VEC{v}_0-\VEC{v}_2)$ for $]2,3[$, then
\begin{align*}
\int_{h([s])} \kappa \dx{V}
&= -\int_{\osscript{\VEC{v}_2}{}{}{}{\VEC{v}_0}} \tilde{h}^\ast(\rho)
= -\int_{[2,3]} \alpha^\ast\big(\tilde{h}^\ast(\rho)\big) \dx{t}
= -\int_{[2,3]} (\tilde{h} \circ \alpha)^\ast(\rho) \dx{t} \\
&= -\int_{[2,3]} \tilde{\tau}^\ast(\rho) \dx{t}
= -\int_{[2,3]} \rho(\tilde{\tau}(t))\big(\tilde{\tau}_\ast(t,1)\big) \dx{t}
= -\int_{[2,3]} \theta'(t) \dx{t} \\
&= -\theta(3) + \theta(2) = - \theta(3) \ .
\end{align*}
We get the conclusion of the proposition if we set $\eta = \eta_3$.
\end{proof}

\begin{rmk}
We could have chosen any other parametric representation of $\sigma$
in the previous proposition.  We chose the must convenient one.
  
From the Mean Value Theorem for integrals, we have that
$\kappa(\sigma(a))$ is the limit of\\
$\displaystyle \int_{h([s])} \kappa \dx{V} \big/ \int_{h([s])}\dx{V}$ as
$[s]$ shrink to $\VEC{v}_0$.  

There may be no rotation as stated in the previous proposition 
despite the fact that $\displaystyle \int_{h([s])} \kappa \dx{V}$
is not be null.  We may have that
$\displaystyle \int_{h([s])} \kappa \dx{V}$ is a multiple of $2\pi$.
\end{rmk}

Suppose that $S$ be a $2$-dimensional Riemannian manifold and that
$\breve{\sigma}:[0,L] \to S$ is a parametric representation by arc 
length of a piecewise $\displaystyle C^\infty$ curve $\sigma:[a,b]\to S$
of length $L$ such that $\sigma(t) \neq \VEC{0}$ for all but a finite
number of values of $t$ \footnote{If $\sigma'(t) \neq \VEC{0}$ for all but a
finite number of values of $t$, we can define the arc length function
$s:[a,b] \to [0,L]$ by defining it on each of the intervals where
$\sigma'(t) \neq \VEC{0}$ and summing them.  As usual, the piecewise
$\displaystyle C^\infty$ parametric representation by arc length
is given by
$\displaystyle \breve{\sigma} = \sigma \circ s^{-1}:[0,L]\to [a,b]$.}. 
We set
\[
\tau(\sigma) = \int_0^L \kappa_{\breve{\sigma}}(t) \dx{t} \ ,
\]
where $\kappa_{\breve{\sigma}}(t)$ is the geodesic curvature of
$\breve{\sigma}$ at $t$ defined in the previous section.
We have that $\tau$ defines a linear mapping on the space of all
piecewise $\displaystyle C^\infty$ curves on $S$.  If
\[
\sigma(t) =
\begin{cases}
\sigma_1(t) & \quad \ \text{if}\ a \leq t \leq c \\
\sigma_2(t) & \quad \ \text{if}\ c < t \leq b
\end{cases}
\]
for some $c \in\, ]a,b[$, and piecewise $\displaystyle C^\infty$ curves
$\sigma_1:[a,c] \to S$ and $\sigma_2:[c,b] \to S$, then
$\tau(\sigma) = \tau(\sigma_1) + \tau(\sigma_2)$.

Suppose that:
\begin{enumerate}
\item $h:V \to S$ is the smooth orientation preserving function
and $\os{s}{}{}{}{} = \os{\VEC{v}_0}{\VEC{v}_1}{}{}{\VEC{v}_3}$
is an oriented $2$-simplex satisfying the hypotheses of
Proposition~\ref{propRotSimpl},
\item $\displaystyle \sigma^{\osscript{s}{}{}{}{}}:[a,b] \to S$ is a parametric
representation of the piecewise $\displaystyle C^\infty$ closed curve
$h(\partial\, [s])$ such that the orientation on
$\displaystyle \sigma^{\osscript{s}{}{}{}{}}$ is the orientation
on $h(\partial\,[s])$ induces via $h$ by $\partial_2 \os{s}{}{}{}{}$, and
\item $t_0 = a < t_1 < t_2 < t_3 = b$ are such that
$\displaystyle \sigma^{\osscript{s}{}{}{}{}}$ is of class
$\displaystyle C^\infty$ on $[t_j,t_{j+1}]$ for $0 \leq j < 3$.
\end{enumerate}
Let
\begin{enumerate}
\item $\displaystyle \sigma_j^{\osscript{s}{}{}{}{}}
= \sigma^{\osscript{s}{}{}{}{}}\big|_{[t_j,t_{j+1}]}$ for $0 \leq j < 3$.
\item $\displaystyle \sigmaU_j^{\osscript{s}{}{}{}{}} :[t_j,t_{j+1}]
\to \TS\, S$ be the curve defined by
$\displaystyle \sigmaU_j^{\osscript{s}{}{}{}{}}(t)
= \big(\sigma^{\osscript{s}{}{}{}{}}(t),(\sigma^{\osscript{s}{}{}{}{}})'(t)\big)$
for $t_j \leq t \leq t_{j+1}$ and $0 \leq j < 3$, and
\item $\displaystyle \tilde{\sigma}_j^{\osscript{s}{}{}{}{}}:[t_j,t_{j+1}]$
be the horizontal lift of $\displaystyle \sigma_j^{\osscript{s}{}{}{}{}}$
with $\displaystyle \tilde{\sigma}_j^{\osscript{s}{}{}{}{}}(t_j)
= \tilde{\sigma}_{j-1}^{\osscript{s}{}{}{}{}}(t_j)$ for $0 < j < 3$,
and $\displaystyle \tilde{\sigma}_0^{\osscript{s}{}{}{}{}}(t_0)
= \sigmaU_0^{\osscript{s}{}{}{}{}}(t_0)
= \big(\sigma^{\osscript{s}{}{}{}{}}(t_0),
(\sigma^{\osscript{s}{}{}{}{}})'(t_0)\big)$.
\end{enumerate}
The internal angle $\displaystyle \mu_2^{\osscript{s}{}{}{}{}}$
and external angle $\displaystyle \nu_2^{\osscript{s}{}{}{}{}}$ at
$\displaystyle \sigma^{\osscript{s}{}{}{}{}}(t_2)$ are defined
in the following figure.
\pdfbox{riemann_geom/gbfig4}
We define the internal angle
$\displaystyle \mu_1^{\osscript{s}{}{}{}{}}$ and
the external angle $\displaystyle \nu_1^{\osscript{s}{}{}{}{}}$ at
$\displaystyle \sigma^{\osscript{s}{}{}{}{}}(t_1)$, and the internal
angle $\displaystyle \mu_3^{\osscript{s}{}{}{}{}}$ and the
external angle $\displaystyle \nu_3^{\osscript{s}{}{}{}{}}$ at
$\displaystyle \sigma^{\osscript{s}{}{}{}{}}(t_3)
= \sigma^{\osscript{s}{}{}{}{}}(t_0)$, in a similar fashion.  An attention
particular should be given to the directions of the angles at the
vertex $\displaystyle \sigma^{\osscript{s}{}{}{}{}}(t_2)$
indicated by the directed arcs.  All the blue vectors drawn from
$\displaystyle \sigma^{\osscript{s}{}{}{}{}}(t_2)$ are in fact in
$\displaystyle \TS_{\sigma^{\osscript{s}{}{}{}{}}(t_2)} S$ because all
the edges are in $S$.  As well,
$\displaystyle \tilde{\sigma}_1^{\osscript{s}{}{}{}{}}(t_2)
\in \TS_{\sigma^{\osscript{s}{}{}{}{}}(t_2)} S$ by
construction.  We have defined the positive directions of the
interior angles as going from one edge to the previous edge on the
$1$-chain $\partial_2 \os{s}{}{}{}{}$.  The orientation of the
$1$-chain is illustrated in the figure above.

\begin{lemma} \label{lemmaGB}
Let $S$ be an oriented $2$-dimensional Riemannian manifold, and
$\os{s}{}{}{}{} = \os{\VEC{v}_0}{\VEC{v}_1}{}{}{\VEC{v}_3}$
be an oriented $2$-simplex.  Suppose that
\begin{enumerate}
\item $V$ is an open set such that $\displaystyle [s] \subset V \subset \RR^2$,
\item $h:V \to S$ is a smooth, non-singular and orientation
preserving function, and
\item $\displaystyle \sigma^{\osscript{s}{}{}{}{}}:[a,b] \to S$ is a
parametric representation of the piecewise $\displaystyle C^\infty$
closed curve $h(\partial\, [s])$ such that the orientation on
$\displaystyle \sigma^{\osscript{s}{}{}{}{}}$ is the orientation on
$h(\partial\,[s])$ induced via $h$ by $\partial_2 \os{s}{}{}{}{}$.
\end{enumerate}
Then
\[
\int_{h(\osscript{s}{}{}{}{})} \kappa \dx{V} = -\pi
+ \sum_{j=1}^3 \big(\mu_j^{\osscript{s}{}{}{}{}}
- \tau(\sigma_{j-1}^{\osscript{s}{}{}{}{}}) \big)
= -\pi + \sum_{j=1}^3 \mu_j^{\osscript{s}{}{}{}{}}
- \tau(\sigma^{\osscript{s}{}{}{}{}}) \ .
\]
\end{lemma}

\begin{proof}
We may assume that $\displaystyle \sigma^{\osscript{s}{}{}{}{}}$ is a
parametric representation by arc length because
$\displaystyle \tau(\sigma_{j-1}^{\osscript{s}{}{}{}{}})$ is computed
using $\displaystyle \sigma^{\osscript{s}{}{}{}{}}$ parameterized by arc length.
Let $\displaystyle \tilde{\sigma}^{\osscript{s}{}{}{}{}}:[a,b] \to S_c$
be the horizontal lift of $\displaystyle \sigma^{\osscript{s}{}{}{}{}}$ with
$\displaystyle \tilde{\sigma}^{\osscript{s}{}{}{}{}}(a)
= \sigmaU^{\osscript{s}{}{}{}{}}(a)
= \big(\sigma^{\osscript{s}{}{}{}{}}(a),
(\sigma^{\osscript{s}{}{}{}{}})'(a)\big)$.
We have from Proposition~\ref{propRotSimpl} that
$\displaystyle \eta(\tilde{\sigma}^{\osscript{s}{}{}{}{}}(b))
= \tilde{\sigma}^{\osscript{s}{}{}{}{}}(a)$ for
$\displaystyle \eta = e^{i \theta} \in S^1$ with
$\displaystyle \theta = -\int_{h([s])} \kappa \dx{V}$.
We also have from Proposition~\ref{proUnHorLift} that
$\displaystyle \sigmaU_{j-1}^{\osscript{s}{}{}{}{}}(t_j)
= \eta_j\big(\tilde{\sigma}_{j-1}^{\osscript{s}{}{}{}{}}(t_j)\big)$ with
$\displaystyle \eta_j = e^{i\theta_{j-1}^{\osscript{s}{}{}{}{}}}$ where
\[
\theta_{j-1}^{\osscript{s}{}{}{}{}}
= \int_{t_{j-1}}^{t_j} \rho\big(\sigmaU_{j-1}^{\osscript{s}{}{}{}{}}(r)\big)
\big((\sigmaU_{j-1}^{\osscript{s}{}{}{}{}})_\ast(r,1) \big) \dx{r}
= \int_{t_{j-1}}^{t_j} \kappa_{\sigma_{j-1}^{\osscript{s}{}{}{}{}}}(r) \dx{r}
= \tau(\sigma_{j-1}^{\osscript{s}{}{}{}{}})
\]
for $t_{j-1} \leq t \leq t_i$ and $0 < j \leq 3$.
Hence
\begin{align}
\theta &= \sum_{j=1}^3 (\nu_j^{\osscript{s}{}{}{}{}} +
\theta_j^{\osscript{s}{}{}{}{}}) - 2\pi k
= \sum_{i=1}^3 \left( \pi - \mu_j^{\osscript{s}{}{}{}{}}
+ \tau(\sigma_{j-1}^{\osscript{s}{}{}{}{}}) \right) - 2 \pi k \nonumber \\
&= - \sum_{i=1}^3 \left( \mu_j^{\osscript{s}{}{}{}{}}
- \tau(\sigma_{j-1}^{\osscript{s}{}{}{}{}}) \right) + (3-2k)\pi \label{lemmaGBEq1}
\end{align}
for some $k \in \ZZ$.

To prove that $k=1$, we consider the continuous family of Riemannian
metrics on $h(V) \subset S$ defined by
$\displaystyle \eta^{[r]} = \eta^{[0]} + r (\eta - \eta^{[0]})$
where $\eta$ is the Riemannian metric on $S$ from the statement of
the lemma and $\displaystyle \eta^{[0]}$ is the Riemannian metric on
$S$ induced from $\displaystyle \RR^2$ using the function $h:V \to S$.
Basically, we treat $h(V)$ and $\sigma$ as $V$ and $\partial_2
\os{s}{}{}{}{}$ respectively.

For the Riemannian metric $\displaystyle \eta^{[0]}$,
we have that $\kappa = 0$ on
$h(V)$ because $V$ is an open subset of $\displaystyle \RR^2$
(see Example~\ref{eggEuclk0}), and
$\kappa_{\sigma_{j-1}} = 0$ for $0 < j \leq 3$
since $\partial_2 \os{s}{}{}{}{}$ is piecewise linear
\footnote{If we use (\ref{DGcurvature}) as we have justified at the end of
Section~\ref{sectGeodesic}, we get that $\kappa_{\sigma}(t)=0$ for all
$t$ in the interval.}.  Hence (\ref{lemmaGBEq1}) yields
$\displaystyle 0 = (3 - 2k)\pi  - \sum_{j=0}^3 \mu_j^{\osscript{s}{}{}{}{}}$.
Since the sum of
the internal angles for a triangle is $\pi$, we have that 
$\displaystyle \sum_{j=0}^3 \mu_j^{\osscript{s}{}{}{}{}} = \pi$.  Thus $k = 1$.
Since (\ref{lemmaGBEq1})) is true for any Riemannian metric and varies
continuously as $r$ goes from $0$ to $1$, we must have that $k$ also
varies continuously as $r$ goes from $0$ to $1$.  Note that
$\kappa$ and $\tau(\sigma_{j-1})$ are also varying continuously with
respect to $r$ \footnote{This should be explained more carefully.  It
is based on the fact that the differential $1$-forms $\omega_1$,
$\omega_2$ and $\rho$ defined in Definition~\ref{defnO1O2} and
Theorem~\ref{thmRconnect} vary continuously with respect to $r$.
We leave the details to the reader.}.
However, $k$ can only take integer values.  Thus $k= 1$ for all
$r\in [0,1]$; in particular for $r=1$.  This gives the conclusion of the
lemma.
\end{proof}

\begin{proof}[Proof (of Theorem~\ref{thmGaussBonnet})]
Let $(S,K,h)$ be the triple in the definition of a smoothly triangulated 
manifold, Definition~\ref{defnSTriangleM}.  We assume that the
orientation on $K$ is such that $h$ is an orientation preserving map.
In particular, $h: (s) \to h\big((s)\big)$ is orientation preserving 
for all $2$-simplices $(s) \in K$.

By definition of a smoothly triangulated manifold, for
every simplex
$(s) = (\VEC{x}_{i_0},\VEC{x}_{i_1},\VEC{x}_{i_2})$ of the simplicial complex
$K$, there exist an open set $U_{(s)}$ in the affine space containing
$\{\VEC{x}_{i_0}, \VEC{x}_{i_1}, \VEC{x}_{i_2}\}$ and a map
$h_{[s]}:U_{[s]} \to S$ of class $\displaystyle C^\infty$ such that
$[s] \subset U_{[s]}$ and $h_{[s]} = h$ on $[s]$.
This will justify the use of Lemma~\ref{lemmaGB} later on.

We first observe that every $1$-simplex in $K$ is the edge of exactly two
$2$-simplices in $K$.  Hence, the image of every $1$-simplex by $h$ is
the edge of the image of exactly two $2$-simplices by $h$.  In non
rigorous terms, each edge in the triangulation of $S$ is the edge of two
triangles.  We obviously have from the definition of simplicial
complexes that a $1$-simplex cannot be the edge of more than two
$2$-simplices.  We need to determine that it must be two.
Suppose that $(s)$ is a $1$-simplex in $K$ and that $\VEC{x} \in (s)$.
Then $\VEC{u} = h(\VEC{x}) \in h\big((s)\big)$.  Let $(W,U,\phi)$ be a
local chart about $\VEC{u}$.  We may assume that $U$ is connected
and small enough to have that
$U \cap h\big((t)\big) = \emptyset$ for all
$1$-simplices $(t) \in K$ such that $(s) \neq (t)$.
Therefore $U \setminus h\big((s)\big)$ is the union of two disjoint open
sets $U_1$ and $U_2$ in $U$ (Figure~\ref{GBfig5}).  These two open
sets are in the image by $h$ of two distinct $2$-simplices in $K$.
To convince ourselves that this is true, we note that since
$U_i$ is connected and $U_i \cap h\big((t)\big) = \emptyset$ for all
$1$-simplices $(t) \in K$, we have that
$U_i \subset h\big((t_i)\big)$ for some $2$-simplex $(t_i) \subset K$.
Since the image of an open set by a homeomorphism is an
open set, the sets $h\big((t)\big)$ for $(t)$ a $2$-simplex of
$k$ are distinct open sets.  Therefore, $U_i$ cannot be a subset of
more than one set $h\big((t_i)\big)$ without contradicting the fact that
$U_i$ is connected.  We must also have that $(t_1) \neq (t_2)$.  If
$(t_1) = (t_2)$, then $h\big((s)\big) \cap U \subset \partial U$
with $h\big((s)\big) \cap U \neq \emptyset$.  This is a contradiction
that $U$ is open.

\pdfF{riemann_geom/gbfig5}{Fifth figure associated to the proof of
Gauss-Bonnet theorem}{This figure is associated to the proof
of Theorem~\ref{thmGaussBonnet}.  It is used to proof that every
$1$-simplex in $K$ is the edge of exactly two $2$-simplices in $K$.
In non mathematical terms, each edge in the smoothly triangulated
manifold $S$ is the edge of two triangles.}{GBfig5}

From the previous paragraph, we deduce that
$\alpha_1 = (3/2) \alpha_2$.  Since the Euler characteristic is defined
as $\Chi(K) = \alpha_0 - \alpha_1 + \alpha_2$, we get
$\Chi(K) = \alpha_0 - (3/2) \alpha_2 + \alpha_2 = \alpha_0 - \alpha_2/2$.

It follows from Lemma~\ref{lemmaGB}
\begin{align*}
\frac{1}{2\pi} \int_S \kappa \dx{V}
&= \frac{1}{2\pi} \int_{h([K])} \kappa \dx{V}
= \frac{1}{2\pi} \sum_{\substack{(s)\in K \text{ is}\\\text{a $2$-simplex}}}
\int_{h(\osscript{s}{}{}{}{})} \kappa \dx{V} \\
&= \frac{1}{2\pi} \sum_{\substack{(s)\in K  \text{ is}\\\text{a $2$-simplex}}} 
\left( -\pi + \sum_{j=1}^3 \left( \mu_j^{\osscript{s}{}{}{}{}}
- \tau(\sigma_{j-1}^{\osscript{s}{}{}{}{}})\right) \right) \\
&=- \frac{\alpha_2}{2}
+ \frac{1}{2\pi} \sum_{\substack{(s)\in K  \text{ is}\\\text{a $2$-simplex}}}
\left(\sum_{j=1}^3 \mu_j^{\osscript{s}{}{}{}{}} \right)
- \frac{1}{2\pi} \sum_{\substack{(s)\in K \text{ is}\\\text{a $2$-simplex}}}
\tau(\sigma^{\osscript{s}{}{}{}{}}) \ .
\end{align*}
The last summation in the relation above is null because each
edge of the triangulation of $S$ is associated to exactly two
triangles in the triangulation of $S$; namely, each $1$-simplex in $K$
is the edge of exactly two $2$-simplices in $K$.
Suppose that $\displaystyle \sigma_{j_1-1}^{\osscript{s_1}{}{}{}{}}$ and
$\displaystyle \sigma_{j_2-1}^{\osscript{s_2}{}{}{}{}}$ are the parametric
representations of the common edge between the oriented $2$-simplices
$\os{s_1}{}{}{}{}$ and $\os{s_2}{}{}{}{}$.  Then
$\displaystyle \tau(\sigma_{j_2-1}^{\osscript{s_2}{}{}{}{}}) =
-\tau(\sigma_{j_2-1}^{\osscript{s_2}{}{}{}{}})$ because the two parametric
representations have opposite directions.

As for the sum in the middle, it is the sum of the interior angles of
all the triangles at each vertex of the triangulation of $S$; namely,
at each $0$-simplex in $K$.  The sum at each vertex is $2\pi$
(Figure~\ref{GBfig6}).
Therefore, the sum in the middle is equal to $2\pi \alpha_0$.

Hence we get
\[
\frac{1}{2\pi} \int_S \kappa \dx{V}
= - \frac{\alpha_2}{2}  + \alpha_0 
= \Chi(K) \ .  \qedhere
\]
\end{proof}

\pdfF{riemann_geom/gbfig6}{sixth figure associated to the proof of
Gauss-Bonnet theorem}{The internal angles of all the triangles at a
vertex $h(\VEC{x}_j)$ of the smoothly triangulated manifold $S$.
Recall that all the dashed blue lines delimiting the interior angles
are in $\TS_{h(\VEC{x}_j)} S$ because all the edges are in $S$.  By
construction, we have defined the direction of all the interior angle
as going from one edge to the previous edge on the $1$-chain
$\partial_2 \os{s}{}{}{}{}$ associated to the triangle, and this for
all oriented $2$-simplices $s \in K$.  The orientation of one of the
$1$-chain is illustrated in the figure.}{GBfig6}

\begin{egg}
We can use the Gauss-Bonnet theorem to prove that        \label{eggGBsphere}
$\displaystyle \int_{S^2} \kappa \dx{V} = 4\pi$ without having to compute
the integral.

Suppose that $\displaystyle (S^2,K,h)$ is a smoothly triangulated
manifold.  We have from Proposition~\ref{propIsoSShom} that
$H_j([K];\RR) \cong H_j(K;\RR)$ for all $j$, and from
Corollary~\ref{corIsoHkXY} that
$\displaystyle H_j(S^2;\RR) \cong H_j([K];\RR)$ because
$\displaystyle h:[K] \to S^2$ is a homeomorphism.  Therefore, we may use the
results in Example~\ref{eggHkSq} to get the Betti numbers
$\beta_j = \dim(H_j(K;\RR))$ for $0 \leq j \leq 2$ that we need to
compute the Euler characteristic
$\displaystyle \Chi(K) = \sum_{j=0}^2 (-1)^j \beta_j$ as defined in
Definition~\ref{defnBettiEulerC}.

We have from Example~\ref{eggHkSq} that
$\displaystyle H_0(S^2;\RR) \cong \RR$, 
$\displaystyle H_1(S^2;\RR) \cong 0$ and
$\displaystyle H_2(S^2;\RR) \cong \RR$.  Hence $\beta_0 = \beta_2 = 1$
and $\beta_1 = 0$.  We get that $\Chi(K) = 2$ and thus
$\displaystyle \int_{S^2} \kappa \dx{V} = 2\pi \Chi(K) = 4\pi$.
\end{egg}

\begin{egg}
We can use the Gauss-Bonnet theorem to prove that
$\displaystyle \int_{\torus{2}} \kappa \dx{V} = 0$ without having to compute
the integral.

As in the previous example, suppose that $\displaystyle (\torus{2},K,h)$ is a
smoothly triangulated manifold.  We may use
$\displaystyle H_j(\torus{2};\RR) \cong H_j(K;\RR)$ for $0 \leq j \leq 2$ to
get the Betti numbers $\beta_j = \dim(H_j(K;\RR))$ for $0 \leq j \leq 2$.
Since
$\displaystyle H_j(\torus{2};\RR) \cong H_j(S^1;\RR) \times H_j(S^1;\RR)$
for $0 \leq j \leq 2$, we get from Example~\ref{eggHkSq} that
$\displaystyle H_0(\torus{2};\RR) \cong \RR^2$, 
$\displaystyle H_1(\torus{2};\RR) \cong \RR^2$ and
$\displaystyle H_2(\torus{2};\RR) \cong 0$.  Hence $\beta_0 = \beta_1 = 2$
and $\beta_2 = 0$.  We get that $\Chi(K) = 0$ and thus
$\displaystyle \int_{S^2} \kappa \dx{V} = 2\pi \Chi(K) = 0$.
\end{egg}

We have seen in Proposition~\ref{propNNVFodd} that there is no
nowhere null vector field on $\displaystyle S^{k-1}$ if $k$ is odd.
We can extend this result using Gauss-Bonnet theorem.

\begin{cor}
Let $(S,K,h)$ be a connected, oriented and smoothly triangulated
$2$-dimensional Riemannian manifold.
If $S$ admits a nowhere null smooth vector field, then $\Chi(K) = 0$.
\end{cor}

\begin{proof}
Suppose that $F:S \to \TS\,S$ is a nowhere null smooth vector field.
Let $f:S \to S_c$ be the smooth map defined by
$\displaystyle f(\VEC{u}) = (\|F\VEC{u}\|_{\VEC{u}})^{-1} F(\VEC{u})$ where
$\|\cdot\|_{\VEC{u}}$ is the norm induced by the Riemannian metric
$\displaystyle \tau:S \to \bigcup_{\VEC{x} \in S} \T^2(\TS_{\VEC{u}} S)$.

According to the second structural equation
$\displaystyle \df{\rho} = - (\pi_S)^\ast(\kappa \dx{V})$ where $V$ is
the volume element on $S$.  Hence
\[
\df{(f^\ast \rho)} = f^\ast(\df{\rho})
= -f^\ast\big( (\pi_S)^\ast(\kappa \dx{V}) \big)
= - (\pi_S \circ f)^\ast(\kappa \dx{V})
= - \kappa \dx{V}
\]
because $\pi_S \circ f = \Id_S$.  Thus $\kappa \dx{V}$ is an exact
differential $2$-form on $S$.  It follows from Gauss-Bonnet theorem
and Stokes' theorem that
\[
  \Chi(K) = \frac{1}{2\pi} \int_S \kappa \dx{V} 
= - \frac{1}{2\pi} \int_S \df{(f^\ast \rho)}
= - \frac{1}{2\pi} \int_{\partial S} f^\ast \rho = 0
\]
because $\partial S = \emptyset$.
\end{proof}

It follows from the previous corollary that the connected, oriented
and smoothly triangulated $2$-dimensional Riemannian manifold
$(S,K,h)$ with $\Chi(K) \neq 0$ do not admit a nowhere null vector field.

\section{Lie Bracket}

As announced at the end of Subsection~\ref{secTSasDiffop}, we are now
going to make use of the definition of tangent spaces as spaces of
differential linear operators.  In particular, we will use the
definition of vector fields given in Remark~\ref{rmkVFwithOper}.

If $F$ is a smooth vector field on a manifold $S$ and
$\displaystyle g \in C^\infty(S)$, then
$F(g):\VEC{u} \mapsto F(\VEC{u})(g)$ for $\VEC{u} \in S$ defines a function
in $\displaystyle C^\infty(S)$.  Therefore, the following definition
makes sense.

\begin{defn} \label{defnLieB}
Let $F_1,F_2$ be two smooth vector fields on a manifold $S$.  The
{\bfseries Lie bracket}\index{Lie Bracket} of $F_1$ and $F_2$,
denoted $[F_1,F_2]$, is the smooth vector field on $S$ defined by
\[
[F_1,F_2](\VEC{u})(g) = F_1(\VEC{u})\big(F_2(g)\big)  
- F_2(\VEC{u})\big(F_1(g)\big)
\]
for all $\VEC{u} \in S$ and $\displaystyle g \in C^\infty(S)$.
\end{defn}

It is not hard to prove that
\begin{enumerate}
\item $[F_1,F_2] = -[F_2,F_1]$,
\item $[F_1 +F_2,F_3] = [F_1,F_3]+[F_2,F_3]$,
\item $[\lambda F_1,F_2] = \lambda [F_1,F_2]$ and
\item $[[F_1,F_2],F_3] + [[F_2,F_3],F_1] + [[F_3,F_1],F_2] = 0$
\end{enumerate}
for all vector field $F_i$ on $S$ and $\lambda \in \RR$ with
$1 \leq i \leq 3$.  The fourth identity above is called the
{\bfseries Jacobi identity}\index{Jacobi Identity}.  For those
interested in Lie algebra, the four properties above are those
required to say that the space of smooth vector fields on $S$ is a Lie
algebra.  There is a vast literature on this subject.

Note that if $\omega$ is a smooth differential $1$-form on $S$ and $F$ is
a smooth vector field on $S$, then $\omega(F):S \to \RR$ defined by
$\omega(F)(\VEC{u}) = \omega(\VEC{u})\big(F(\VEC{u})\big)$ for all
$\VEC{u}\in S$ is a function of class $\displaystyle C^\infty$.

\begin{prop} \label{propdyOmF1F2}
Suppose that $\omega$ is a smooth differential $1$-form on a smooth
$k$-dimensional manifold $S$, and that $F_1$ and $F_2$ are two smooth
vector fields on $S$.  Then
\begin{equation} \label{LieBEq1}
\df{\omega}(\VEC{u})\big(F_1(\VEC{u}),F_2(\VEC{u})\big)
= F_1(\VEC{u})(\omega(F_2)) - F_2(\VEC{u})(\omega(F_1))
- \omega(\VEC{u})([F_1,F_2](\VEC{u}))
\end{equation}
for all $\VEC{u} \in S$.
\end{prop}

\begin{proof}
As usual, we only need to prove the result for a local chart
$(W,U,\phi)$ of $S$.  We have that
$\displaystyle \phi^\ast(\omega) = \sum_{i=1}^k \omega_i \df{u_i}$
where $w_i : W \to \RR$ are smooth functions.  Hence $\omega$ is
locally the sum of terms of the form $h \df{g}$ where $h,g:U \to \RR$
are smooth functions.  Since both sides of (\ref{LieBEq1}) are linear
relative to $\omega$, we only have to prove (\ref{LieBEq1}) for
$h \df{g}$.

Using Definition~\ref{defnWwedgeP} and the definition of $\alt$ given
in Proposition~\ref{propAltDefn} and referring to
Remark~\ref{rmkVFwithOper}, we have that the left hand side of
(\ref{LieBEq1}) is
\begin{align*}
\df{\omega}(\VEC{u})\big(F_1(\VEC{u}),F_2(\VEC{u})\big)
&=(\df{h}\wedge \df{g})(\VEC{u})\big(F_1(\VEC{u}),F_2(\VEC{u})\big) \\
&= \df{h}(\VEC{u})\big(F_1(\VEC{u})\big) \,
\df{g}(\VEC{u})\big(F_2(\VEC{u})\big)
- \df{h}(\VEC{u})\big(F_2(\VEC{u})\big) \,
\df{g}(\VEC{u})\big(F_1(\VEC{u})\big) \\
&= F_1(\VEC{u})(h)\, F_2(\VEC{u}) (g)
- F_2(\VEC{u})(h)\, F_1(\VEC{u}) (g)
\end{align*}
for $\VEC{u} \in U$.

For the right hand side of (\ref{LieBEq1}), we have
\begin{align*}
&F_1(\VEC{u})(\omega(F_2)) - F_2(\VEC{u})(\omega(F_1))
- \omega(\VEC{u})([F_1,F_2](\VEC{u})) \\
&\qquad = F_1(\VEC{u})(h \df{g}(F_2)) - F_2(\VEC{u})(h \df{g} (F_1))
- h(\VEC{u}) \df{g}(\VEC{u})([F_1,F_2](\VEC{u})) \\
&\qquad = F_1(\VEC{u})(h F_2(g)) - F_2(\VEC{u})(h F_1(g))
- h(\VEC{u})\, [F_1,F_2](\VEC{u}))(g) \\
&\qquad = \big( F_1(\VEC{u})(h) F_2(\VEC{u})(g) + h(\VEC{u})
F_1(\VEC{u})(F_2(g)) \big)
-\big( F_2(\VEC{u})(h) F_1(\VEC{u})(g) + h(\VEC{u})
F_2(\VEC{u})(F_1(g)) \big) \\
&\qquad \qquad - h(\VEC{u}) \big( F_1(\VEC{u})(F_2(g)) -
F_2(\VEC{u})(F_1(g)) \big)
= F_1(\VEC{u})(h) F_2(\VEC{u})(g) - F_2(\VEC{u})(h) F_1(\VEC{u})(g)
\end{align*}
for $\VEC{u} \in U$, where the third equality comes from the product
rule applied to the operators $F_1(\VEC{u})$ and $F_2(\VEC{u})$, and
the definition of $[F_1,F_2]$.

Therefore, the left and right sides of (\ref{LieBEq1}) are equal.
\end{proof}

Let $F_1$, $F_2$ and $F_3$ be vector fields on $S_c$ such that
$\{\omega_1(\VEC{u},\VEC{x}), \omega_2(\VEC{u},\VEC{x}),
\rho(\VEC{u},\VEC{x}) \}$ is the dual basis associated to
$\{F_1(\VEC{u},\VEC{x}), F_2(\VEC{u},\VEC{x}), F_3(\VEC{u},\VEC{x})\}$
for all $(\VEC{u},\VEC{x}) \in S_c$.  Recall that $F_3 = H$ defined in
Section~\ref{subsectTSc} because $\rho$ is a connection $1$-form.
We get from (\ref{LieBEq1}) that
\begin{align*}
&\df{\omega_1}(\VEC{u},\VEC{x})
\big(F_1(\VEC{u},\VEC{x}),F_2(\VEC{u},\VEC{x})\big) \\
&\qquad = F_1(\VEC{u},\VEC{x})\big(\omega_1(F_2)\big)
- F_2(\VEC{u},\VEC{x})\big(\omega_1(F_1)\big)
-\omega_1(\VEC{u},\VEC{x})\big([F_1,F_2](\VEC{u},\VEC{x})\big) \\
&\qquad =
-\omega_1(\VEC{u},\VEC{x})\big([F_1,F_2](\VEC{u},\VEC{x})\big) \ .
\end{align*}
Moreover
\begin{align*}
&\big(\rho \wedge \omega_2 + c_1 \omega_1 \wedge \omega_2\big)(\VEC{u},\VEC{x})
\big(F_1(\VEC{u},\VEC{x}),F_2(\VEC{u},\VEC{x})\big) \\
&\qquad = \rho(\VEC{u},\VEC{x})\big(F_1(\VEC{u},\VEC{x})\big)\,
\omega_2(\VEC{u},\VEC{x})\big(F_2(\VEC{u},\VEC{x})\big)
- \rho(\VEC{u},\VEC{x})\big(F_2(\VEC{u},\VEC{x})\big)\,
\omega_2(\VEC{u},\VEC{x})\big(F_1(\VEC{u},\VEC{x})\big) \\
&\qquad \qquad
+c_1(\VEC{u},\VEC{x})\, \omega_1(\VEC{u},\VEC{x})\big(F_1(\VEC{u},\VEC{x})\big)
\, \omega_2(\VEC{u},\VEC{x})\big(F_2(\VEC{u},\VEC{x})\big) \\
&\qquad \qquad
-c_1(\VEC{u},\VEC{x})\, \omega_1(\VEC{u},\VEC{x})\big(F_2(\VEC{u},\VEC{x})\big)
\, \omega_2(\VEC{u},\VEC{x})\big(F_1(\VEC{u},\VEC{x})\big)
= c_1(\VEC{u},\VEC{x}) \ .
\end{align*}
If we substitute these two expressions in the first of the first structural
equations in Definition~\ref{defnFCartanEs}, we get
\begin{equation} \label{intCurvEq1}
\omega_1(\VEC{u},\VEC{x})\big([F_1,F_2](\VEC{u},\VEC{x})\big)
= - c_1(\VEC{u},\VEC{x}) \ .
\end{equation}

Computations similar to those above give
\[
\df{\omega_2}(\VEC{u},\VEC{x})
\big(F_1(\VEC{u},\VEC{x}),F_2(\VEC{u},\VEC{x})\big)
= -\omega_2(\VEC{u},\VEC{x})\big([F_1,F_2](\VEC{u},\VEC{x})\big)
\]
and
\[
\big(-\rho \wedge \omega_1 + c_2 \omega_1 \wedge \omega_2\big)(\VEC{u},\VEC{x})
\big(F_1(\VEC{u},\VEC{x}),F_2(\VEC{u},\VEC{x})\big)
= c_2(\VEC{u},\VEC{x}) \ .
\]
If we substitute these two expressions in the second of the first structural
equations in Definition~\ref{defnFCartanEs}, we get
\begin{equation} \label{intCurvEq2}
\omega_2(\VEC{u},\VEC{x})\big([F_1,F_2](\VEC{u},\VEC{x})\big)
= -c_2(\VEC{u},\VEC{x}) \ .
\end{equation}

We get from (\ref{LieBEq1}) that
\begin{align*}
&\df{\rho}(\VEC{u},\VEC{x})
\big(F_1(\VEC{u},\VEC{x}),F_2(\VEC{u},\VEC{x})\big) \\
&\qquad = F_1(\VEC{u},\VEC{x})\big(\rho(F_2)\big)
- F_2(\VEC{u},\VEC{x})\big(\rho(F_1)\big)
-\rho(\VEC{u},\VEC{x})\big([F_1,F_2](\VEC{u},\VEC{x})\big) \\
&\qquad = -\rho(\VEC{u},\VEC{x})\big([F_1,F_2](\VEC{u},\VEC{x})\big) \ .
\end{align*}
Moreover
\begin{align*}
&\big(-(\kappa \circ \pi_S)\, \omega_1 \wedge \omega_2\big)(\VEC{u},\VEC{x})
\big(F_1(\VEC{u},\VEC{x}),F_2(\VEC{u},\VEC{x})\big) \\
&\qquad
= -\kappa(\VEC{u}) \Big( \omega_1(\VEC{u},\VEC{x})\big(F_1(\VEC{u},\VEC{x})\big)
\, \omega_2(\VEC{u},\VEC{x})\big(F_2(\VEC{u},\VEC{x})\big) \\
&\qquad \qquad
- \omega_1(\VEC{u},\VEC{x})\big(F_2(\VEC{u},\VEC{x})\big)
\, \omega_2(\VEC{u},\VEC{x})\big(F_1(\VEC{u},\VEC{x})\big) \Big)
= - \kappa(\VEC{u})  \ .
\end{align*}
If we substitute these two expressions in the second structural
equation in Definition~\ref{defnSecondCSE}, we get
\begin{equation} \label{intCurvEq3}
\rho(\VEC{u},\VEC{x})\big([F_1,F_2](\VEC{u},\VEC{x})\big)
= \kappa(\VEC{u}) \ .
\end{equation}

Proceeding as we did above with $F_1,F_2$ replaced by $H,F_1$ and
$H,F_2$, we get the following six equations.
\begin{align} 
\omega_1(\VEC{u},\VEC{x})\big([H,F_1](\VEC{u},\VEC{x})\big) &= 0 \ ,
\label{intCurvEq4} \\
\omega_2(\VEC{u},\VEC{x})\big([H,F_1](\VEC{u},\VEC{x})\big) &= 1 \ ,
\label{intCurvEq5} \\
\rho(\VEC{u},\VEC{x})\big([H,F_1](\VEC{u},\VEC{x})\big) &= 0 \ , 
\label{intCurvEq6} \\
\omega_1(\VEC{u},\VEC{x})\big([H,F_2](\VEC{u},\VEC{x})\big) &= -1 \ ,
\label{intCurvEq7} \\
\omega_2(\VEC{u},\VEC{x})\big([H,F_2](\VEC{u},\VEC{x})\big) & = 0
\label{intCurvEq8}
\intertext{and}
\rho(\VEC{u},\VEC{x})\big([H,F_2](\VEC{u},\VEC{x})\big) & = 0 \ .
\label{intCurvEq9}
\end{align}

% We get from (\ref{LieBEq1}) that
% \begin{align*}
% &\df{\omega_1}(\VEC{u},\VEC{x})
% \big(H(\VEC{u},\VEC{x}),F_1(\VEC{u},\VEC{x})\big) \\
% &\qquad = H(\VEC{u},\VEC{x})\big(\omega_1(F_1)\big)
% - F_1(\VEC{u},\VEC{x})\big(\omega_1(H)\big)
% -\omega_1(\VEC{u},\VEC{x})\big([H,F_1](\VEC{u},\VEC{x})\big) \\
% &\qquad =
% -\omega_1(\VEC{u},\VEC{x})\big([H,F_1](\VEC{u},\VEC{x})\big) \ .
% \end{align*}
% Moreover
% \begin{align*}
% &\big(\rho \wedge \omega_2 + c_1 \omega_1 \wedge \omega_2\big)(\VEC{u},\VEC{x})
% \big(H(\VEC{u},\VEC{x}),F_1(\VEC{u},\VEC{x})\big) \\
% &\qquad = \rho(\VEC{u},\VEC{x})\big(H(\VEC{u},\VEC{x})\big)\,
% \omega_2(\VEC{u},\VEC{x})\big(F_1(\VEC{u},\VEC{x})\big)
% - \rho(\VEC{u},\VEC{x})\big(F_1(\VEC{u},\VEC{x})\big)\,
% \omega_2(\VEC{u},\VEC{x})\big(H(\VEC{u},\VEC{x})\big) \\
% &\qquad \qquad
% +c_1(\VEC{u},\VEC{x})\, \omega_1(\VEC{u},\VEC{x})\big(H(\VEC{u},\VEC{x})\big)
% \, \omega_2(\VEC{u},\VEC{x})\big(F_1(\VEC{u},\VEC{x})\big) \\
% &\qquad \qquad
% -c_1(\VEC{u},\VEC{x})\, \omega_1(\VEC{u},\VEC{x})\big(F_1(\VEC{u},\VEC{x})\big)
% \, \omega_2(\VEC{u},\VEC{x})\big(H(\VEC{u},\VEC{x})\big) \\
% &\qquad = 0 \ .
% \end{align*}
% If we substitute these two expressions in the first of the first structural
% equations, we get
% \begin{equation} \label{intCurvEq4}
% \omega_1(\VEC{u},\VEC{x})\big([H,F_1](\VEC{u},\VEC{x})\big) = 0 \ .
% \end{equation}

% Computations similar to those above give
% \[
% \df{\omega_2}(\VEC{u},\VEC{x})
% \big(H(\VEC{u},\VEC{x}),F_1(\VEC{u},\VEC{x})\big)
% = -\omega_2(\VEC{u},\VEC{x})\big([H,F_1](\VEC{u},\VEC{x})\big)
% \]
% and
% \[
% \big(-\rho \wedge \omega_1 + c_2 \omega_1 \wedge \omega_2\big)(\VEC{u},\VEC{x})
% \big(H(\VEC{u},\VEC{x}),F_1(\VEC{u},\VEC{x})\big)
% = -1 \ .
% \]
% If we substitute these two expressions in the second of the first structural
% equations, we get
% \begin{equation} \label{intCurvEq5}
% \omega_2(\VEC{u},\VEC{x})\big([H,F_1](\VEC{u},\VEC{x})\big) = 1 \ .
% \end{equation}

% We get from (\ref{LieBEq1}) that
% \begin{align*}
% &\df{\rho}(\VEC{u},\VEC{x})
% \big(H(\VEC{u},\VEC{x}),F_1(\VEC{u},\VEC{x})\big) \\
% &\qquad = H(\VEC{u},\VEC{x})\big(\rho(F_1)\big)
% - F_1(\VEC{u},\VEC{x})\big(\rho(H)\big)
% -\rho(\VEC{u},\VEC{x})\big([H,F_1](\VEC{u},\VEC{x})\big) \\
% &\qquad = -\rho(\VEC{u},\VEC{x})\big([H,F_1](\VEC{u},\VEC{x})\big) \ .
% \end{align*}
% Moreover,
% \begin{align*}
% &\big(-(\kappa \circ \pi_S)\, \omega_1 \wedge \omega_2\big)(\VEC{u},\VEC{x})
% \big(H(\VEC{u},\VEC{x}),F_1(\VEC{u},\VEC{x})\big) \\
% &\qquad
% = -\kappa(\VEC{u}) \Big( \omega_1(\VEC{u},\VEC{x})\big(H(\VEC{u},\VEC{x})\big)
% \, \omega_2(\VEC{u},\VEC{x})\big(F_1(\VEC{u},\VEC{x})\big) \\
% &\qquad \qquad
% - \omega_1(\VEC{u},\VEC{x})\big(F_1(\VEC{u},\VEC{x})\big)
% \, \omega_2(\VEC{u},\VEC{x})\big(H(\VEC{u},\VEC{x})\big) \Big)
% = 0  \ .
% \end{align*}
% If we substitute these two expressions in the second structural
% equation, we get
% \begin{equation} \label{intCurvEq6}
% \rho(\VEC{u},\VEC{x})\big([H,F_1](\VEC{u},\VEC{x})\big) = 0 \ .
% \end{equation}

% We get from (\ref{LieBEq1}) that
% \begin{align*}
% &\df{\omega_1}(\VEC{u},\VEC{x})
% \big(H(\VEC{u},\VEC{x}),F_2(\VEC{u},\VEC{x})\big) \\
% &\qquad = H(\VEC{u},\VEC{x})\big(\omega_1(F_2)\big)
% - F_2(\VEC{u},\VEC{x})\big(\omega_1(H)\big)
% -\omega_1(\VEC{u},\VEC{x})\big([H,F_2](\VEC{u},\VEC{x})\big) \\
% &\qquad =
% -\omega_1(\VEC{u},\VEC{x})\big([H,F_2](\VEC{u},\VEC{x})\big) \ .
% \end{align*}
% Moreover
% \begin{align*}
% &\big(\rho \wedge \omega_2 + c_1 \omega_1 \wedge \omega_2\big)(\VEC{u},\VEC{x})
% \big(H(\VEC{u},\VEC{x}),F_2(\VEC{u},\VEC{x})\big) \\
% &\qquad = \rho(\VEC{u},\VEC{x})\big(H(\VEC{u},\VEC{x})\big)\,
% \omega_2(\VEC{u},\VEC{x})\big(F_2(\VEC{u},\VEC{x})\big)
% - \rho(\VEC{u},\VEC{x})\big(F_2(\VEC{u},\VEC{x})\big)\,
% \omega_2(\VEC{u},\VEC{x})\big(H(\VEC{u},\VEC{x})\big) \\
% &\qquad \qquad
% +c_1(\VEC{u},\VEC{x})\, \omega_1(\VEC{u},\VEC{x})\big(H(\VEC{u},\VEC{x})\big)
% \, \omega_2(\VEC{u},\VEC{x})\big(F_2(\VEC{u},\VEC{x})\big) \\
% &\qquad \qquad
% -c_1(\VEC{u},\VEC{x})\, \omega_1(\VEC{u},\VEC{x})\big(F_2(\VEC{u},\VEC{x})\big)
% \, \omega_2(\VEC{u},\VEC{x})\big(H(\VEC{u},\VEC{x})\big) \\
% &\qquad = 1 \ .
% \end{align*}
% If we substitute these two expressions in the first of the first structural
% equations, we get
% \begin{equation} \label{intCurvEq7}
% \omega_1(\VEC{u},\VEC{x})\big([H,F_2](\VEC{u},\VEC{x})\big)
% = -1 \ .
% \end{equation}

% Computations similar to those above give
% \[
% \df{\omega_2}(\VEC{u},\VEC{x})
% \big(H(\VEC{u},\VEC{x}),F_2(\VEC{u},\VEC{x})\big)
% = -\omega_2(\VEC{u},\VEC{x})\big([H,F_2](\VEC{u},\VEC{x})\big)
% \]
% and
% \[
% \big(-\rho \wedge \omega_1 + c_2 \omega_1 \wedge \omega_2\big)(\VEC{u},\VEC{x})
% \big(H(\VEC{u},\VEC{x}),F_2(\VEC{u},\VEC{x})\big)
% = 0 \ .
% \]
% If we substitute these two expressions in the second of the first structural
% equations, we get
% \begin{equation} \label{intCurvEq8}
% \omega_2(\VEC{u},\VEC{x})\big([H,F_2](\VEC{u},\VEC{x})\big)
% = 0 \ .
% \end{equation}

% We get from (\ref{LieBEq1}) that
% \begin{align*}
% &\df{\rho}(\VEC{u},\VEC{x})
% \big(H(\VEC{u},\VEC{x}),F_2(\VEC{u},\VEC{x})\big) \\
% &\qquad = H(\VEC{u},\VEC{x})\big(\rho(F_2)\big)
% - F_2(\VEC{u},\VEC{x})\big(\rho(H)\big)
% -\rho(\VEC{u},\VEC{x})\big([H,F_2](\VEC{u},\VEC{x})\big) \\
% &\qquad = -\rho(\VEC{u},\VEC{x})\big([H,F_2](\VEC{u},\VEC{x})\big) \ .
% \end{align*}
% Moreover,
% \begin{align*}
% &\big(-(\kappa \circ \pi_S)\, \omega_1 \wedge \omega_2\big)(\VEC{u},\VEC{x})
% \big(H(\VEC{u},\VEC{x}),F_2(\VEC{u},\VEC{x})\big) \\
% &\qquad
% = -\kappa(\VEC{u}) \Big( \omega_1(\VEC{u},\VEC{x})\big(H(\VEC{u},\VEC{x})\big)
% \, \omega_2(\VEC{u},\VEC{x})\big(F_2(\VEC{u},\VEC{x})\big) \\
% &\qquad \qquad
% - \omega_1(\VEC{u},\VEC{x})\big(F_2(\VEC{u},\VEC{x})\big)
% \, \omega_2(\VEC{u},\VEC{x})\big(H(\VEC{u},\VEC{x})\big) \Big)
% = 0  \ .
% \end{align*}
% If we substitute these two expressions in the second structural
% equation, we get
% \begin{equation} \label{intCurvEq9}
% \rho(\VEC{u},\VEC{x})\big([H,F_2](\VEC{u},\VEC{x})\big)
% = 0 \ .
% \end{equation}

It follows from (\ref{intCurvEq1}) to (\ref{intCurvEq3}) that
\begin{align*}
[F_1,F_2](\VEC{u},\VEC{x})
&=\omega_1(\VEC{u},\VEC{x})\big([F_1,F_2](\VEC{u},\VEC{x})\big)
\, F_1(\VEC{u},\VEC{x})
+ \omega_2(\VEC{u},\VEC{x})\big([F_1,F_2](\VEC{u},\VEC{x})\big)
\, F_2(\VEC{u},\VEC{x}) \\
&\qquad + \rho(\VEC{u},\VEC{x})\big([F_1,F_2](\VEC{u},\VEC{x})\big)
\, H(\VEC{u},\VEC{x}) \\
&= - c_1(\VEC{u},\VEC{x})\, F_1(\VEC{u},\VEC{x})
- c_2(\VEC{u},\VEC{x})\, F_2(\VEC{u},\VEC{x})
+ \kappa(\VEC{u})\, H(\VEC{u},\VEC{x})
\end{align*}
for all $(\VEC{u},\VEC{x}) \in S_c$.  Similarly, it follows from
(\ref{intCurvEq4}) to (\ref{intCurvEq6}) and from
(\ref{intCurvEq7}) to (\ref{intCurvEq9}) that
\begin{align*}
[H,F_1](\VEC{u},\VEC{x})
&=\omega_1(\VEC{u},\VEC{x})\big([H,F_1](\VEC{u},\VEC{x})\big)
\, F_1(\VEC{u},\VEC{x})
+ \omega_2(\VEC{u},\VEC{x})\big([H,F_1](\VEC{u},\VEC{x})\big)
\, F_2(\VEC{u},\VEC{x}) \\
&\qquad + \rho(\VEC{u},\VEC{x})\big([H,F_1](\VEC{u},\VEC{x})\big)
\, H(\VEC{u},\VEC{x}) =  F_2(\VEC{u},\VEC{x})
\end{align*}
and
\begin{align*}
[H,F_2](\VEC{u},\VEC{x})
&=\omega_1(\VEC{u},\VEC{x})\big([H,F_2](\VEC{u},\VEC{x})\big)
\, F_1(\VEC{u},\VEC{x})
+ \omega_2(\VEC{u},\VEC{x})\big([H,F_2](\VEC{u},\VEC{x})\big)
\, F_2(\VEC{u},\VEC{x}) \\
&\qquad + \rho(\VEC{u},\VEC{x})\big([H,F_2](\VEC{u},\VEC{x})\big)
\, H(\VEC{u},\VEC{x}) = - F_1(\VEC{u},\VEC{x})
\end{align*}
respectively for all $(\VEC{u},\VEC{x}) \in S_c$.  Hence, the first
and second structural equations formulated in terms of vector fields
$F_1$, $F_2$ and $H$ are
\[
[F_1,F_2] = - c_1\, F_1 - c_2 F_2 + (\kappa \circ \pi_S) \, H \quad , \quad
[H,F_1] =  F_2 \quad \text{and} \quad
[H,F_2] = - F_1
\]
on $S_c$.  In the special case where $\rho$ is the Riemann connection
on $S_c$, then $[F_1,F_2] = (\kappa \circ \pi_S) \, H$.

\begin{rmk}
It follows from the definition of $\omega_1$ and $\omega_2$ in
Definition~\ref{defnO1O2} that
\[
(\pi_S)_\ast\big(F_1(\VEC{u},\VEC{x})\big)
= \omega_1((\VEC{u},\VEC{x})\big(F_1(\VEC{u},\VEC{x})\big) \, (\VEC{u},\VEC{x})
+ \omega_2((\VEC{u},\VEC{x})\big(F_1(\VEC{u},\VEC{x})\big) \,
\upsilon(\VEC{u},\VEC{x})
= (\VEC{u},\VEC{x})
\]
and, similarly, $(\pi_S)_\ast\big(F_2(\VEC{u},\VEC{x})\big)
= \upsilon(\VEC{u},\VEC{x})$ where
$\displaystyle \upsilon = e^{i \pi/2} \in S^1$.  Moreover,
$F_1(\VEC{u},\VEC{x})$ and $F_2(\VEC{u},\VEC{x})$ are orthogonal to
$H(\VEC{u},\VEC{x})$ because
$\rho(\VEC{u},\VEC{x}))\big(F_1(\VEC{u},\VEC{x})\big) =
\rho(\VEC{u},\VEC{x}))\big(F_2(\VEC{u},\VEC{x})\big) = 0$.  These
conditions determine uniquely $F_1(\VEC{u},\VEC{x})$ and
$F_2(\VEC{u},\VEC{x})$ for all $(\VEC{u},\VEC{x}) \in S_c$.
\end{rmk}

\subsection{An Equivalent Definition of the Lie
Bracket} \label{subsectEDLB}

There is another way to define the Lie bracket that is often used in
the literature.  First, we have the reformulate the definition of an
{\bfseries integral curve}\index{Integral Curve} at
$\VEC{u} \in S$ for a vector field $F:S \to \TS\,S$.
It is a function $\sigma_{\VEC{u}}:I \to S$ where $I$ is an open
neighbourhood of the origin such that $\sigma_{\VEC{u}}(0) = \VEC{u}$ and
$\displaystyle \sigma_{\VEC{u}}^\ast\left(\dydx{}{t}\Big|_t\right)
= F(\sigma_{\VEC{u}}(t))$ for all $t \in I$; namely,
\begin{equation} \label{IntCurvV2Eq1}
\sigma_{\VEC{u}}^\ast\left(\dydx{}{t}\Big|_t\right)(g)
= \dfdx{(g \circ \sigma_{\VEC{u}})}{t}(t) = F(\sigma_{\VEC{u}}(t))(g)
\end{equation}
for all $\displaystyle g \in C^\infty(S)$ and $t \in I$.  Note that the
interval of existence $I$ may vary with $\VEC{u}$.
Let $(W,U,\phi)$ be a local chart of $S$.  It follows from
(\ref{IntCurvV2Eq1}) that
$\displaystyle \dfdx{(\phi^{-1}\circ \sigma_{\VEC{u}})}{t}
= \tilde{F}(\phi^{-1}\circ \sigma_{\VEC{u}})$ where
$\displaystyle \tilde{F}(\VEC{w}) = \sum_{j=1}^k
\tilde{f}_j(\VEC{w})\pdydx{}{w_j}$ for $\VEC{w} \in W$ is the local
representation of $F$.  This is expected if (\ref{IntCurvV2Eq1}) is
equivalent to the condition in our previous definition of integral
curves in Definition~\ref{defnIntCurvV1}.

We have from the theorem of existence and uniqueness of solutions for
ordinary differential equations that, by shrinking $U$ if necessary,
there exists an open interval $I$ containing the origin such that the
integral curve $\sigma_{\VEC{u}}$ at any $\VEC{u} \in U$ exists on $I$ and
$\sigma_{\VEC{u}}(t) \in S$ for all $t \in I$.
Since we are mostly working on local charts,, we will assume
that this is always satisfied.  If we set
$X(t,\VEC{u}) = \sigma_{\VEC{u}}(t)$ for $\VEC{u} \in U$ and $t \in I$, 
then we get what is called a {\bfseries flow}\index{Flow}.  It is
proved in the theory of ordinary differential equations that
$X:I \times U \to S$ is smooth if the vector field $F$ is smooth (see
\cite{A}). 

Suppose that $F_1,F_2:S \to \TS\,S$ are two smooth vector fields.
Moreover, suppose that $X:I\times U \to S$ is the flow associated to
the vector field $F_1$ as explained above.
Then $\sigma_{\VEC{u}}(t) = X_{\VEC{u}}(t) = X(t,\VEC{u})$ for $t \in I$
satisfies
$\displaystyle \sigma_{\VEC{u}}^\ast\left(\dydx{}{t}\Big|_t\right)
= F_1(\sigma_{\VEC{u}}(t))$ for all $t\in I$ and
$\sigma_{\VEC{u}}(0) = \VEC{u}$.
The {\bfseries Lie bracket}\index{Lie Bracket} of $F_1$ and $F_2$ is
the vector field defined by
\begin{equation} \label{defnLieBv2}
[F_1,F_2](\VEC{u}) = \dfdx{\big(X_t^\ast(F_2)(\VEC{u})\big)}{t}\Big|_{t=0}
\end{equation}
where $X_t(\VEC{u}) = X(t,\VEC{u})$ for all $\VEC{u} \in S$.
We have from the theory of ordinary differential equations that
$X_t:U \to S$ maps open subsets of $U$ to open subsets of $S$; in
particular, $X_t$ maps open neighbourhoods of $\VEC{u} \in U$ to open
neighbourhoods of $X_t(\VEC{u})$ in $S$.
Note that we have use for the first time the pull-back of a vector field.

Let $(W,U,\phi)$ be a local chart of $S$.  Suppose that the local
representation of the vector field $F_i$ is defined by
$\displaystyle \tilde{F}_i(\VEC{w}) = \sum_{j=1}^k
(\tilde{f}_i)_j(\VEC{w}) \pdydx{}{w_j}\Big|_{\VEC{w}}$ for $\VEC{w} \in W$ and
$i =1,2$, where $\displaystyle (\tilde{f}_i)_j:W \to \RR^k$ for
$1 \leq j \leq k$ and $i=1,2$ are smooth functions.

The flow $\tilde{X}$ on $W$ associated to the vector field $\tilde{F}_1$ is
given by $\phi(\tilde{X}(t,\VEC{w})) = X(t,\phi(\VEC{w}))$ for all
$\VEC{w} \in W$ and $|t|$ small enough.  This follows from the fact that
$\displaystyle
X_{\VEC{u}}^\ast\left(\dydx{}{t}\Big|_t\right) = F_1\big(X_{\VEC{u}}(t)\big)$
implies that
\begin{align*}
\tilde{X}_{\VEC{w}}^\ast\left(\dfdx{}{t}\Big|_t\right)(g\circ \phi)
&= \dfdx{g(\phi(\tilde{X}_{\VEC{w}}(t)))}{t}
= \dfdx{g(X_{\phi(\VEC{w})}(t))}{t}
= X_{\phi(\VEC{w})}^\ast\left(\dydx{}{t}\Big|_t\right)(g) \\
&= F_1\big(X_{\phi(\VEC{w})}(t)\big)(g)
= \sum_{j=1}^k (\tilde{f}_i)_j\big(\phi^{-1}(X_{\phi(\VEC{w})}(t))\big)
\pdfdx{(g \circ \phi)}{w_j} \Big|_{\VEC{w} = \phi^{-1}((X_{\phi(\VEC{w})}(t)))} \\
&= \sum_{j=1}^k (\tilde{f}_i)_j\big(\tilde{X}_{\VEC{w}}(t)\big)
\pdfdx{(g \circ \phi)}{w_j} \Big|_{\VEC{w} = \tilde{X}_{\VEC{w}}(t)}
= \tilde{F_1}(\tilde{X}_{\VEC{w}})(g \circ \phi) 
\end{align*}
for all $\displaystyle \phi \in C^\infty(S)$.

We have that
$\displaystyle \dfdx{\tilde{X}(t,\VEC{w})}{t} =
\tilde{f}_1\big(\tilde{X}(t,\VEC{w})\big)$ is equivalent to
$\displaystyle
\tilde{X}_{\VEC{w}}^\ast\left( \dydx{}{t}\Big|_t \right)(\tilde{g})
= \tilde{F}_1(\tilde{X}_{\VEC{w}}(t))(\tilde{g})$
for all $\displaystyle \tilde{g} \in C^\infty(W)$
because
\[
\tilde{X}_{\VEC{w}}^\ast\left( \dydx{}{t}\Big|_t \right)(\tilde{g}) 
= \dfdx{\big( \tilde{g}(\tilde{X}(t,\VEC{w})) \big)}{t}
= \diff \tilde{g}(\tilde{X}(t,\VEC{w}))
\, \dfdx{ \tilde{X}(t,\VEC{w})}{t}
\]
and
\begin{align*}
\tilde{F}_1(\tilde{X}_{\VEC{w}}(t))(\tilde{g})
&= \tilde{F}_1(\tilde{X}(t,\VEC{w}))(\tilde{g})
= \sum_{j=1}^k (\tilde{f}_1)_j(\tilde{X}(t,\VEC{w}))
\, \pdydx{\tilde{g}}{w_i} (\tilde{X}(t,\VEC{w})) \\
&= \diff \tilde{g}(\tilde{X}(t,\VEC{w}))
\, \tilde{f}_1(\tilde{X}(t,\VEC{w}))
\end{align*}
for all $\displaystyle \tilde{g} \in C^\infty(W)$ and $|t|$ small enough.

The local representation of the vector field
$\displaystyle Q_t = X_t^\ast(F_2)$ is given by\\
$\displaystyle \tilde{Q}_t(\VEC{w}) = \sum_{j=1}^k (\tilde{q}_t)_j(\VEC{w})
\pdydx{}{w_j}$ where
$\tilde{q}_t(\VEC{w}) = \diff \tilde{X}_{-t}(\tilde{X}_t(\VEC{w}))
\, \tilde{f}_2(\tilde{X}_t(\VEC{w}))$
for $\VEC{w} \in W$ and $t$ small enough to have
$\tilde{X}_t(\VEC{w}) \in W$.  To prove this claim, it is more
convenient to use our standard definition of vector fields and the
isomorphism with the definition in terms of differential linear
operators.  We have
\begin{align*}
X_t^\ast(F_2)(\VEC{u})
&= (X_t^{-1})_\ast\big(F_2(X_t(\VEC{u}))\big)
= (X_{-t})_\ast\big(F_2(X_t(\VEC{u}))\big)
= (X_{-t})_\ast\big(\phi_\ast\big(
\tilde{F}_2(\phi^{-1}(X_t(\VEC{u}))) \big)\big) \\
&= (X_{-t} \circ \phi)_\ast\big(
\tilde{F}_2(\phi^{-1}(X_t(\VEC{u}))) \big)
= \big( \VEC{u}, \diff (X_{-t} \circ \phi) (\phi^{-1}(X_t(\VEC{u})))
\, \tilde{f}_2(\phi^{-1}(X_t(\VEC{u}))) \big) \\
&= \big( \VEC{u}, \diff \phi(\phi^{-1}(\VEC{u}))
\, \diff (\phi^{-1} \circ X_{-t} \circ \phi) (\phi^{-1}(X_t(\VEC{u})))
\, \tilde{f}_2(\phi^{-1}(X_t(\VEC{u}))) \big) \\
&= \big( \VEC{u}, \diff \phi(\phi^{-1}(\VEC{u}))
\, \diff \tilde{X}_{-t} (\phi^{-1}(X_t(\VEC{u})))
\, \tilde{f}_2(\phi^{-1}(X_t(\VEC{u}))) \big) \\
&= \big( \VEC{u}, \diff \phi(\phi^{-1}(\VEC{u}))
\, \diff \tilde{X}_{-t} (\tilde{X}_t(\phi^{-1}(\VEC{u})))
\, \tilde{f}_2(\tilde{X}_t(\phi^{-1}(\VEC{u}))) \big)
\end{align*}
for $\VEC{u} \in U$ and $t$ small enough.
If $\tilde{Q}_t(\VEC{w}) = (\VEC{w},\tilde{q}_t(\VEC{w}))$ for $\VEC{w} \in W$ 
is the local representation of $Q$, then we get from the previous
relation that
\[
\tilde{q}_t(\VEC{w}) = \diff \tilde{X}_{-t}(\tilde{X}_t(\VEC{w})) 
\, \tilde{f}_2(\tilde{X}_t(\VEC{w}))
\]
for $\VEC{w} \in W$ and $|t|$ small enough.

Since the local representation of the vector field
$\displaystyle Q_t = X_t^\ast(F_2)$ is given by\\
$\displaystyle \sum_{j=1}^k (\tilde{q}_t)_j(\VEC{w})
\pdfdx{}{w_j}\Big|_{\VEC{w}=\phi^{-1}(\VEC{u})}$,
the local representation of the vector field
$\displaystyle \dfdx{\big(X_t^\ast(F_2)\big)}{t}\Big|_{t=0}$
is given by
$\displaystyle \sum_{i=1}^k \dfdx{\big(\tilde{q}_t(\phi^{-1}(\VEC{u}))_j\big)}{t}
\Big|_{t=0}\, \pdfdx{}{w_j}\Big|_{\VEC{w}=\phi^{-1}(\VEC{u})}$.

We prove that the definition of the Lie bracket in (\ref{defnLieBv2}) 
is equivalent to the definition in Definition~\ref{defnLieB}.
To prove this claim, we first need a little technical result.

Suppose that $\displaystyle g \in C^\infty(S)$.  Then
\begin{align}
&\dfdx{X_t^\ast(g)}{t}(\VEC{u})
= \dfdx{g(X(t,\VEC{u}))}{t}
= \dfdx{\left((g\circ\phi)\big(\tilde{X}(t,\phi^{-1}(\VEC{u}))\big)\right)}{t}
\nonumber \\
&\quad = \diff (g\circ \phi)\big(\tilde{X}(t,\phi^{-1}(\VEC{u}))\big)
\, \dfdx{\tilde{X}(t,\phi^{-1}(\VEC{u}))}{t}
= \diff (g\circ \phi)\big(\tilde{X}(t,\phi^{-1}(\VEC{u}))\big)
\, \tilde{f}_1\big(\tilde{X}(t,\phi^{-1}(\VEC{u}))\big) \nonumber \\
&\quad = \sum_{j=1}^k (\tilde{f}_1)_j\big(\phi^{-1}(X(t,\VEC{u}))\big)
\, \pdfdx{(g \circ \phi)}{w_i}\big(\phi^{-1}(X(t,\VEC{u}))\big)
= F_1\big(X(t,\VEC{u})\big)(g) \label{defnLieBv2Eq1}
\end{align}
for all $\VEC{u} \in U$.

We are now ready to prove that the definition of Lie bracket given in
(\ref{defnLieBv2}) is equivalent to our Definition~\ref{defnLieB}.

Given $\VEC{u} = \phi(\VEC{w})$ and
$\displaystyle g \in C^\infty(S)$, we have
\begin{align*}
&\dfdx{\big(X_t^\ast(F_2)(\VEC{u})\big)}{t}\Big|_{t=0}(g)
= \dfdx{ \Big(\diff (g \circ \phi) (\VEC{w})
\, \diff \tilde{X}_{-t}(\tilde{X}_t(\VEC{w})) \,
\tilde{f}_2(\tilde{X}_t(\VEC{w}))\Big)}{t}\Big|_{t=0} \\
&\qquad = \dfdx{ \Big(\diff \big((g \circ X_{-t} \circ \phi)
\circ (\phi^{-1} \circ X_t\circ \phi)\big) (\VEC{w})
\, \diff \tilde{X}_{-t}(\tilde{X}_t(\VEC{w})) \,
\tilde{f}_2(\tilde{X}_t(\VEC{w}))\Big)}{t}\Big|_{t=0} \\
&\qquad = \dfdx{ \big(\diff (g \circ X_{-t} \circ \phi)
(\tilde{X}_t(\VEC{w})) \, 
\diff \tilde{X}_t (\VEC{w})
\, \diff \tilde{X}_{-t}(\tilde{X}_t(\VEC{w})) \,
\tilde{f}_2(\tilde{X}_t(\VEC{w}))\Big)}{t}\Big|_{t=0} \\
&\qquad = \dfdx{ \big(\diff (g \circ X_{-t} \circ \phi)
(\tilde{X}_t(\VEC{w})) \,
\tilde{f}_2(\tilde{X}_t(\VEC{w}))\Big)}{t}\Big|_{t=0} \\
&\qquad = \dfdx{ \big(\diff ( X_{-t}^\ast(g) \circ \phi)
\big(\phi^{-1} (X_t(\phi(\VEC{w})))\big) \,
\tilde{f}_2\big(\phi^{-1}(X_t(\phi(\VEC{w})))\big) \Big)}{t}\Big|_{t=0} \\
&\qquad = \dfdx{\Big( F_2(X_t(\VEC{u}))(X_{-t}^\ast(g)) \Big)}{t}\Big|_{t=0}
= \dfdx{\Big( X_t^\ast(F_2)(\VEC{u}) (X_{-t}^\ast(g))\Big)}{t}\Big|_{t=0} \\
&\qquad = \dfdx{\Big( X_t^\ast(F_2(g))(\VEC{u}) \Big)}{t}\Big|_{t=0}
+ F_2(\VEC{u}) \left(\dfdx{\Big( X_{-t}^\ast(g)\Big)}{t}\Big|_{t=0}\right) \\
&\qquad = F_1(\VEC{u})(F_2(g)) - F_2(\VEC{u})(F_1(g))
\end{align*}
where the fourth equality comes from
\[
\diff \tilde{X}_t (\VEC{w})
\, \diff \tilde{X}_{-t}(\tilde{X}_t(\VEC{w}))
= \diff \big( \tilde{X}_t \circ \tilde{X}_{-t}\big)
(\tilde{X}_t(\VEC{w}))
= \diff \Id_W(\tilde{X}_t(\VEC{w})) = \Id_W \ ,
\]
the second to last equality is derived from the Leibniz' rule
because\\
$\displaystyle (G_1,G_2) \mapsto G_1^\ast (F_2)(\VEC{u})\,(G_2^\ast(g))$ is a
bilinear map since $F_2(\VEC{u})$ is linear, and the last equality
comes from (\ref{defnLieBv2Eq1}).

\section{Geodesic Coordinate Systems}

Throughout this subsection, $S$ is an oriented $2$-dimensional 
Riemannian manifold.

Let $F_1$, $F_2$ and $F_3$ be the smooth vector fields on
$S_c$ such that
$\{\omega_1(\VEC{u},\VEC{x}), \omega_2(\VEC{u},\VEC{x}),
\rho(\VEC{u},\VEC{x}) \}$ is the dual basis associated to
$\{F_1(\VEC{u},\VEC{x}), F_2(\VEC{u},\VEC{x}), F_3(\VEC{u},\VEC{x})\}$
for all $(\VEC{u},\VEC{x}) \in S_c$; namely,
$\omega_i(\VEC{u},\VEC{x})(F_j(\VEC{u},\VEC{x})) = \delta_{i,j}$
for $1 \leq i \leq 2$ and $1 \leq j \leq 3$, and
$\rho(\VEC{u},\VEC{x})(F_j(\VEC{u},\VEC{x})) = \delta_{3,j}$ for
$1\leq j \leq 3$.  We have already
noted that $F_3 = H$ where $H$ is defined in Section~\ref{subsectTSc}.

Given $(\VEC{u},\VEC{x}) \in S_c$, as we stated in the previous
subsection, there exists a local smooth flow
$X:]-\epsilon_{(\VEC{u},\VEC{x})},\epsilon_{(\VEC{u},\VEC{x})}[\times
U_{(\VEC{u},\VEC{x})} \to S_c$ associated to the smooth vector field $F_1$
where $U_{(\VEC{u},\VEC{x})} \subset S_c$ is an open neighbourhood of
$(\VEC{u},\VEC{x})$.
In particular, for each
$(\tilde{\VEC{u}},\tilde{\VEC{x}}) \in U_{(\VEC{u},\VEC{x})}$,
the function $\tilde{\sigma}_{(\VEC{u},\VEC{x})}:
]-\epsilon_{(\VEC{u},\VEC{x})},\epsilon_{(\VEC{u},\VEC{x})}[ \to S_c$ defined by
$\tilde{\sigma}_{(\VEC{u},\VEC{x})}(t)
= X\big(t,(\tilde{\VEC{u}},\tilde{\VEC{x}})\big)$ 
for $t \in ]-\epsilon_{(\VEC{u},\VEC{x})},\epsilon_{(\VEC{u},\VEC{x})}[$
satisfies
$(\tilde{\sigma}_{(\VEC{u},\VEC{x})})_\ast(t,1)
= F_1\big(\tilde{\sigma}_{(\VEC{u},\VEC{x})}(t)\big)$ for all
$t \in ]-\epsilon_{(\VEC{u},\VEC{x})},\epsilon_{(\VEC{u},\VEC{x})}[$
and $\tilde{\sigma}_{(\VEC{u},\VEC{x})}(0) = (\tilde{\VEC{u}},\tilde{\VEC{x}})$
bu uniqueness of solutions.

We have that
$\displaystyle \{ U_{(\VEC{u},\VEC{x})} \}_{(\VEC{u},\VEC{x}) \in S_c(\VEC{u})}$
is an open cover of the compact set $S_c(\VEC{u})$.  Therefore, there
exists a finite subcover
$\displaystyle \{ U_{(\VEC{u},\VEC{x}_j)} \}_{j \in J}$ of
$S_c(\VEC{u})$.  Let $\displaystyle
\epsilon_{\VEC{u}} = \min_{j \in J} \epsilon_{(\VEC{u},\VEC{x}_j)}$.
We have that
$X:]-\epsilon_{\VEC{u}},\epsilon_{\VEC{u}}[\times S_c(\VEC{u}) \to S_c$
is well defined.

Let $\displaystyle Q:[0,\epsilon_{\VEC{u}}[ \times S^1 \to S_c$ be
the function defined by
$Q(t,\eta) = X\big(t ,\eta(\VEC{u},\tilde{E}(\VEC{u}))\big)$
for $t \in [0,\epsilon_{\VEC{u}}[$ and $\displaystyle \eta \in S^1$,
where $\displaystyle \tilde{E}(\VEC{u}) = E(\phi^{-1}(\VEC{u}))$
and $(W,U,\phi)$ is a local chart of $S$ about $\VEC{u}$.
The local chart $(W,U,\phi)$ of $S$ with $\VEC{u} \in U$ chosen to
define $\tilde{E}(\VEC{u})$ is irrelevant \footnote{Any 
$(\VEC{u},\VEC{x}) \in S_c(\VEC{u})$ with $\VEC{x}$ fixed 
could have been used instead of $(\VEC{u},\tilde{E}(\VEC{u}))$.}.
The map $Q$ is smooth because it is the composition of smooth maps.

For $\eta$ fixed, we consider the curves
$\sigma_\eta: [0,\epsilon_{\VEC{u}}[ \to S$
and $\tilde{\sigma}_\eta:[0,\epsilon_{\VEC{u}}[ \to S_c$ defined respectively 
by $\sigma_\eta(t) = \pi_S(Q(t,\eta))$ and
$\tilde{\sigma}_\eta(t)= Q(t,\eta)$ for $t \in [0,\epsilon_{\VEC{u}}[$.

\begin{prop}
The curve $\tilde{\sigma}_\eta;:[0,\epsilon_{\VEC{u}}[ \to S_c$ is the
horizontal lift of $\sigma_\eta:[0,\epsilon_{\VEC{u}}[ \to S$ through
$\eta(\VEC{u},\tilde{E}(\VEC{u}))$
\end{prop}

\begin{proof}
We have that
\begin{enumerate}
\item $\tilde{\sigma}_\eta(0) = Q(0,\eta) =
X\big(0, \eta(\VEC{u},\tilde{E}(\VEC{u}))\big)
= \eta(\VEC{u},\tilde{E}(\VEC{u}))$,
\item $\pi_s \circ \tilde{\sigma}_\eta = \sigma_\eta$ and
\item $\rho\big(\tilde{\sigma}_\eta(t)\big)
\big((\tilde{\sigma}_\eta)_\ast(t,1)\big)
= \rho\big(\tilde{\sigma}_\eta(t)\big)\big(F_1(\tilde{\sigma}_\eta(t))\big)
= 0$ for $0 \leq t < \epsilon_{\VEC{u}}$ by definition of $F_1$.
\end{enumerate}
Thus, all the conditions for an horizontal lift are satisfied.
\end{proof}

Moreover, we have the following result.

\begin{prop}
$\sigma_\eta:[0,\epsilon_{\VEC{u}}[ \to S$ is a geodesic.
\end{prop}

\begin{proof}
Since $\pi_S \circ \tilde{\sigma}_\eta = \sigma_\eta$, we have that
$(\pi_S)_\ast \circ (\tilde{\sigma}_\eta)_\ast
= (\pi_S \circ \tilde{\sigma}_\eta)_\ast = (\sigma_\eta)_\ast$.
Thus
\begin{align*}
(\sigma_\eta(t),\sigma_\eta'(t))
&= (\sigma_\eta)_\ast(t,1)
= (\pi_S)_\ast\big((\tilde{\sigma}_\eta)_\ast(t,1)\big) \\
&= \omega_1\big(\tilde{\sigma}_\eta(t)\big)
\big((\tilde{\sigma}_\eta)_\ast(t,1)\big) \, \tilde{\sigma}_\eta(t)
+ \omega_2\big(\tilde{\sigma}_\eta(t)\big)
\big((\tilde{\sigma}_\eta)_\ast(t,1)\big) \, \upsilon(\tilde{\sigma}_\eta(t))
\\
&= \omega_1\big(\tilde{\sigma}_\eta(t)\big)
\big(F_1(\tilde{\sigma}_\eta(t))\big) \, \tilde{\sigma}_\eta(t)
+ \omega_2\big(\tilde{\sigma}_\eta(t)\big)
\big(F_1(\tilde{\sigma}_\eta(t))\big) \, \upsilon(\tilde{\sigma}_\eta(t))
=  \tilde{\sigma}_\eta(t)
\end{align*}
for $0 \leq t < \epsilon_{\VEC{u}}$.
\end{proof}

Let
\[
D_{\epsilon_{\VEC{u}}} = \{ (\VEC{u},\VEC{x}) \in \TS_{\VEC{u}} S :
\| (\VEC{u},\VEC{x}) \|_{\VEC{u}} < \epsilon_{\VEC{u}} \} \cong
B_{\epsilon_{\VEC{u}}}(\VEC{0}) \subset \RR^2
\]
where the norm is given by the Riemannian metric on $S$.  The
isomorphism between $B_{\epsilon_{\VEC{u}}}(\VEC{0})$ and
$D_{\epsilon_{\VEC{u}}}$ can be defined by
$t ( \theta.\VEC{e}_1) \mapsto \big(\VEC{u}, t (\theta.\tilde{E}(\VEC{u}))\big)$
for $0 \leq t < \epsilon_{\VEC{u}}$ and $\theta \in \RR$.

Let $\displaystyle P:[0,\epsilon_{\VEC{u}}[\times S^1 \to
D_{\epsilon_{\VEC{u}}} \cong B_{\epsilon_{\VEC{u}}}(\VEC{0})$
be the function defined by $P(t,\eta) = t( \theta.\VEC{e}_1)$
for $\displaystyle (t,\eta) \in [0,\epsilon_{\VEC{u}}[\times S^1$ and
$\displaystyle \eta = e^{i \theta}$.  Since
$\displaystyle P\big|_{]0,\epsilon_{\VEC{u}}[\times S^1}$ is an isomorphism,
we may define
$\displaystyle R: D_{\epsilon_{\VEC{u}}} \to [0,\epsilon_{\VEC{u}}[\times S^1$
by
\[
R(\VEC{x}) =
\begin{cases}
(t, e^{i\theta}) & \quad \text{if} \ \VEC{x} = t(\theta.\VEC{e}_1) \
\text{with} \ t > 0 \\
(0,0) & \quad \text{if} \ \VEC{x} = \VEC{0}
\end{cases}
\]
We have that
$\displaystyle \big(P\big|_{]0,\epsilon_{\VEC{u}}[\times S^1}\big)^{-1}
= R\big|_{D_{\epsilon_{\VEC{u}}}\setminus \{\VEC{0}\}}$.

Let $G:D_{\epsilon_{\VEC{u}}} \to S$ be the map defined by
$G = \pi_S \circ Q \circ R$.  Note that $G$ is well defined
because $\pi_S(Q(0,\eta)) = \VEC{u}$ for all $\displaystyle \eta \in S^1$.

By taking $\epsilon_{\VEC{u}}$ smaller if necessary, we now prove that
$G:D_{\epsilon_{\VEC{u}}} \to G(D_{\epsilon_{\VEC{u}}})$
is a diffeomorphism.  This implies that $G(D_{\epsilon_{\VEC{u}}})$ is
an open neighbourhood of $\VEC{u}$ in $S$, and
$G_\ast$ maps $\TS_{\VEC{0}} D_{\epsilon_{\VEC{u}}}$ onto $\TS_{\VEC{u}} S$.
We use $\displaystyle D_{\epsilon_{\VEC{u}}} = B_{\epsilon_{\VEC{u}}}(\VEC{0})
\subset \RR^2$ and show that
$\displaystyle \diff G(\VEC{0}) : \RR^2 \to \TS_{\VEC{u}} S$
is of rank $2$.  It will then follow from the Inverse Function Theorem
that $G:D_{\epsilon_{\VEC{u}}} \to S$ is a diffeomorphism for
$\epsilon_{\VEC{u}}$ small enough.  Consider
$\displaystyle \VEC{x} = \theta.\VEC{e}_1 \in \RR^2$ with
$0 \leq \theta < 2\pi$ fixed and let
$\displaystyle \eta = e^{i\theta} \in S^1$.  We get the directional
derivative
\begin{align*}
\lim_{t \to 0} \frac{1}{t} \big(G(t\VEC{x}) - G(\VEC{0})\big)
&= \lim_{t \to 0} \frac{1}{t} \big(
\pi_S(Q(t,\eta)) - \pi_S(Q(0,\eta)) \big)
= \lim_{t \to 0} \frac{1}{t} \big( \sigma_\eta(t)
- \sigma_\eta(0) \big) \\
&= \sigma_\eta'(0) = \theta.\tilde{E}(\VEC{u})
\end{align*}
because $(\sigma_\eta(0),\sigma_\eta'(0)) = (\sigma_\eta)_\ast(0,1)
= \tilde{\sigma}_\eta(0) = \eta (\VEC{u},\tilde{E}(\VEC{u}))$.
Since $\displaystyle \eta \in S^1$ is arbitrary, all the directions
from the origin in $\TS_{\VEC{u}} S$ are included in the range of
$\diff G(\VEC{0})$.  Thus $\diff G(\VEC{0})$ is of rank $2$.

The following diagram summarizes the
construction of $G$.
\[
\xymatrix{
[0,\epsilon_{\VEC{u}}[\times S^1 \ar[r]^-{Q} & S_c \ar[d]^{\pi_S} \\
D_{\epsilon_{\VEC{u}}}  \ar[u]^{R} \ar[r]_{G} & S
}
\]

\begin{defn} \label{defnGCsyst}
The map $G:D_{\epsilon_{\VEC{u}}} \to S$ is called a {\bfseries polar
or geodesic coordinate system}\index{Geodesic Coordinate System}
\index{Polar Coordinate System|see{Geodesic Coordinate System}}
for $S$ about $\VEC{u}$ if $\epsilon_{\VEC{u}}$ is small enough to
ensure that $G$ is a diffeomorphism onto its
image. (Figure~\ref{GeodFig1}).
\end{defn}

\pdfF{riemann_geom/geodfig1}{Geodesic coordinate system}{Schematic
representation of a geodesic coordinate system.}{GeodFig1}

For convenience, the lines $t \mapsto t (\theta.\VEC{e}_1)$ in
$D_{\epsilon_{\VEC{u}}} \cong B_{\epsilon_{\VEC{u}}}(\VEC{0})$
for $0 \leq t < \epsilon_{\VEC{u}}$ and
$\displaystyle \eta = e^{i\theta} \in S^1$
fixed are called {\bfseries radial geodesic}\index{Radial Geodesic}.

The following technical results will be quite useful for our study
of geodesic coordinate systems.

In the rest of the section, we will regularly refer to 
local charts $(\check{W},\check{U},\check{\phi})$ of
$\displaystyle [0,\epsilon_{\VEC{u}}[ \times S^1$ where
$\check{W} = [0,\epsilon_{\VEC{u}}[ \times I$ for $I$ a connected and open
interval of length less than $2\pi$,
$\displaystyle \check{\phi}(r,\theta) = (r,e^{i\theta})$ for
$(r,\theta) \in \check{W}$, and $\check{U} = \check{\phi}(\check{W})$.

There are two natural vector fields $V_1$ and $V_2$ on
$\displaystyle ]0,\epsilon_{\VEC{u}}[\times S^1$ that we now define.

Let $\check{V}_1$ and $\check{V}_2$ be the vector fields
defined on $\check{W}$ by
$\check{V}_1(r,\theta) = \big( (r,\theta), (1,0)\big)$ and
$\check{V}_2(r,\theta) = \big( (r,\theta), (0,1)\big)$ respectively
for $(r,\theta) \in W$.  Are there any simpler non-trivial vector fields
on $W$?  If $(t,\eta) = \check{\phi}(r,\theta)$, then
\begin{align*}
V_1(t,\eta) &= \check{\phi}_\ast\big(\check{V}_1(\check{\phi}^{-1}(t,\eta))\big)
= \check{\phi}_\ast\big(\check{V}_1(r,\theta)\big)
= \check{\phi}_\ast\big( (r,\theta), (1,0)\big) \\
&= \big( \check{\phi}(r,\theta), (\diff \check{\phi})(r,\theta)
\begin{pmatrix} 1 & 0 \end{pmatrix}^\top \big)
= \Big( \check{\phi}(r,\theta), \pdydx{\check{\phi}}{r}(r,\theta) \Big)
= \big( (t,\eta) , (1,0) \big)
\end{align*}
and
\begin{align*}
V_2(t,\eta) &= \check{\phi}_\ast\big(\check{V}_2(\check{\phi}^{-1}(t,\eta))\big)
= \check{\phi}_\ast\big(\check{V}_2(r,\theta)\big)
= \check{\phi}_\ast\big( (r,\theta), (0,1)\big) \\
&= \big( \check{\phi}(r,\theta), (\diff \check{\phi})(r,\theta)
\begin{pmatrix} 0 & 1 \end{pmatrix}^\top \big)
= \Big( \check{\phi}(r,\theta), \pdydx{\check{\phi}}{\theta}(r,\theta) \Big) \\
&= \big( (t,\eta) , (0,i\, \eta) \big)
= \big( (t,\eta) , (0,\upsilon\, \eta) \big)
\end{align*}
where $\displaystyle \upsilon = e^{i(\pi/2)}$ as usual.

Since $(P \circ \check{\phi})(r,\theta) = r (\theta.\VEC{e}_1)$
for all $(r,\theta) \in \check{W}$. we get
\begin{align*}
P_\ast(V_1(t,\eta))
&=(P_\ast \circ \check{\phi}_\ast)\big((\check{\phi}^{-1}_\ast \circ
V_1 \circ \check{\phi})(t,\theta)\big)
=(P_\ast \circ \check{\phi}_\ast)\big(\check{V}_1(t,\theta)\big) \\
&=(P \circ \check{\phi})_\ast\big((t,\theta), (1,0)\big)
= (t(\theta.\VEC{e}_1), \theta.\VEC{e}_1)
\end{align*}
where $V_1(t,\eta)$ represents a vector field in
$D_{\epsilon_{\VEC{u}}}$ that point in the radial direction
$\displaystyle \eta = e^{i\theta}$, and
\begin{align*}
P_\ast(V_2(t,\eta))
&=(P_\ast \circ \check{\phi}_\ast)\big((\check{\phi}^{-1}_\ast \circ
V_2 \circ \check{\phi})(t,\theta)\big)
=(P_\ast \circ \check{\phi}_\ast)\big(\check{V_2}(t,\theta)\big) \\
&=(P \circ \check{\phi})_\ast\big((t,\theta), (0,1)\big)
= \big(t (\theta.\VEC{e}), t \big((\pi/2+\theta).\VEC{e}\big)\big)
\end{align*}
represents a vector field in $D_{\epsilon_{\VEC{u}}}$ that point in
the rotational direction as illustrated in Figure~\ref{GeodFig5}.

\pdfF{riemann_geom/geodfig5}{Representation of $P_\ast\circ V_i$ for
$1\leq i \leq 2$}{Representation of the vector field
$P_\ast \circ V_1$ on $D_{\epsilon_{\VEC{u}}}$ in blue and
the vector field
$P_\ast \circ V_2$ on $D_{\epsilon_{\VEC{u}}}$ in red.}{GeodFig5}

It will often be more convenient when performing some algebraic
manipulations to use the definition of vector fields given in
Remark~\ref{rmkVFwithOper}.  The isomorphism between
$\breve{\TS}_{\VEC{u}} S$ and $\TS_{\VEC{u}} S$ introduced in
Section~\ref{secTSasDiffop} shows that the vector fields $V_1$ and
$V_2$ on $\displaystyle ]0,\epsilon_{\VEC{u}}[\times S^1$ can also
be respectively represented locally as
\begin{equation} \label{defnV1V2rt}
\check{V}_1 = \pdydx{}{r}\Big|_{(r,\theta)= \check{\phi}^{-1}(t,\eta)}
\qquad \text{and} \qquad
\check{V}_2 = \pdydx{}{\theta}\Big|_{(r,\theta)= \check{\phi}^{-1}(t,\eta)}
\end{equation}
for $\displaystyle (t,\eta) \in ]0,\epsilon_{\VEC{u}}[\times S^1$.

\begin{lemma}   \label{lemmaMRs2}
\begin{enumerate}
\item $Q_\ast(V_1(t,\eta)) = F_1(\tilde{\sigma}_\eta(t))$
for $\displaystyle (t,\eta) \in ]0,\epsilon_{\VEC{u}}[\times S^1$.
\item $Q^\ast(\rho)(V_1) = Q^\ast(\omega_2)(V_1) = 0$ and
$Q^\ast(\omega_1)(V_1) = 1$ on
$\displaystyle ]0,\epsilon_{\VEC{u}}[\times S^1$.
\item $[V_1,V_2](t,\eta) = \big((t,\eta), (0,0) \big)$
for all $\displaystyle (t,\eta) \in ]0,\epsilon_{\VEC{u}}[\times S^1$.  
\item $Q_\ast(V_2(t,\eta))\big|_{t=0}
= H\big( \tilde{\sigma}_\eta(0)\big)$.
\end{enumerate}
\end{lemma}

\begin{proof}
\stage{1} Using a local chart $(\check{W},\check{U},\check{\phi})$ of
$\displaystyle ]0,\epsilon_{\VEC{u}}[\times S^1$ as defined above, we get for
$(t,\eta) = \check{\phi}(r,\theta)$ that
\begin{align*}
Q_\ast(V_1(t,\eta)) &= Q_\ast\big(\check{\phi}_\ast
\big(\check{V}_1(r,\theta)\big)\big)
= (Q \circ \check{\phi})_\ast\big(\check{V}_1(r,\theta)\big)
= (Q \circ \check{\phi})_\ast\big((t,\theta), (1,0)\big) \\
&= \Big( Q(\check{\phi}(t,\theta)),
\diff_{t,\theta} (Q \circ \check{\phi})(t,\theta)
\begin{pmatrix} 1 & 0 \end{pmatrix}^\top \Big)
= \Big( Q(t,\eta),
\pdfdx{(Q\circ \check{\phi})(t,\theta)}{t} \Big) \\
&= \Big( Q(t,\eta),
\pdfdx{Q(t,\eta)}{t} \Big)
= \big( \tilde{\sigma}_\eta(t), \tilde{\sigma}_\eta'(t)\big)
= (\tilde{\sigma}_\eta)_\ast(t,1) = F_1(\tilde{\sigma}_\eta(t))
\end{align*}
since $\tilde{\sigma}_\eta(t) = Q(t,\eta)$.

\stage{2} Since $Q_\ast(V_1(t,\eta)) = F_1(\tilde{\sigma}_\eta(t))$,
we get
\[
\big(Q^\ast(\rho)(V_1)\big)(t,\eta)
= \rho(Q(t,\eta))\big(Q_\ast(V_1(t,\eta))\big)
= \rho(\tilde{\sigma}_\eta(t))\big(F_1(\tilde{\sigma}_\eta(t))\big) = 0
\]
and
\[
\big(Q^\ast(\omega_2)(V_1)\big)(t,\eta)
= \omega_2(Q(t,\eta))\big(Q_\ast(V_1(t,\eta))
= \omega_2(\tilde{\sigma}_\eta(t))\big(F_1(\tilde{\sigma}_\eta(t))\big) = 0 \ .
\]
for all $\displaystyle (t,\eta) \in ]0,\epsilon_{\VEC{u}}[\times S^1$
by definition of $\rho$, $\omega_2$ and $F_1$.  Moreover,
\[
\big(Q^\ast(\omega_1)(V_1)\big)(t,\eta)
= \omega_1(Q(t,\eta))(Q_\ast(V_1(t,\eta)))
= \omega_1(\tilde{\sigma}_\eta(t))\big(F_1(\tilde{\sigma}_\eta(t))\big)
= 1
\]
for all $\displaystyle (t,\eta) \in ]0,\epsilon_{\VEC{u}}[\times S^1$
by definition of $\omega_1$ and $F_1$.

\stage{3} Using the local representation of $V_1$ and $V_2$ provided in
(\ref{defnV1V2rt}), we get
\begin{align*}
[V_1,V_2](t,\eta)(g) &= V_1(t,\eta)(V_2(g)) -V_2(t,\eta)(V_1(g)) \\
&= \pdfdx{\Big(\pdfdx{ (g\circ \phi)}{\theta}\Big)}{r}
\Big|_{(r,\theta)=\check{\phi}^{-1}(t,\eta)}
- \pdfdx{\Big(\pdfdx{ (g\circ \phi)}{r}\Big)}{\theta}
\Big|_{(r,\theta)=\check{\phi}^{-}(t,\eta)}
= 0
\end{align*}
for all $\displaystyle g \in C^\infty(]0,\epsilon_{\VEC{u}}[\times S^1)$.
Thus $[V_1,V_2](t,\eta) = \big((t,\eta), (0,0) \big)$

\stage{4} We proceed as in (1).  Using a local chart
$(\check{W},\check{U},\check{\phi})$ of
$\displaystyle ]0,\epsilon_{\VEC{u}}[\times S^1$ as defined above, we
get for $(t,\eta) = \check{\phi}(r,\theta)$ that
\begin{align*}
Q_\ast(V_2(t,\eta)) &= Q_\ast\big(\check{\phi}_\ast
\big(\check{V}_2(r,\theta)\big)\big)
= (Q \circ \check{\phi})_\ast\big(\check{V}_2(r,\theta)\big)
= (Q \circ \check{\phi})_\ast\big((t,\theta), (0,1)\big) \\
&= \Big( Q(\check{\phi}(t,\theta)),
\diff_{t,\theta} (Q \circ \check{\phi})(t,\theta)
\begin{pmatrix} 0 & 1 \end{pmatrix}^\top \Big)
= \Big( Q(t,\eta),
\pdfdx{(Q\circ \check{\phi})(t,\theta)}{\theta} \Big) \\
&= \Big( Q(t,\eta), \pdfdx{Q(t, e^{i\theta})}{\theta} \Big)
\end{align*}
Hence
\begin{align*}
Q_\ast(V_2(0,\eta)) 
&= \Big( Q(0,\eta),
\pdfdx{Q(0, e^{i\theta})}{\theta} \Big)
= \Big( \eta(\VEC{u}, \tilde{E}(\VEC{u})) ,
\pdfdx{ \big(\VEC{u}, \theta.\tilde{E}(\VEC{u}) \big)}{\theta} \Big) \\
&= \Big( \eta(\VEC{u}, \tilde{E}(\VEC{u})) ,
\Big( \VEC{0}, \pdfdx{(\theta.\tilde{E}(\VEC{u}))}{\theta} \Big)\Big)
= H\big(\eta(\VEC{u}, \tilde{E}(\VEC{u}))\big)
= H\big(\tilde{\sigma}_\eta(0)\big) \ .  \qedhere
\end{align*}
\end{proof}

\begin{rmk}
It is also possible to prove (3) in the previous lemma (with a lot more work)
using the equivalent formulation for the Lie bracket; namely,
\[
[V_1,V_2](t,\eta) = \dfdx{ Y_s^\ast(V_2)(t,\eta)}{s}\Big|_{s=0} 
\]
where $\displaystyle Y: ]-\epsilon,\epsilon[ \times
]0,\epsilon_{\VEC{u}}[ \times S^1 \to ]0,\epsilon_{\VEC{u}}[\times S^1$
for $\epsilon$ small enough is the flow associate to the vector field
$V_1$.

Using a local chart $(\check{W},\check{U},\check{\phi})$, we have that
the flow associated to $\check{V}_1$ is the function
$\displaystyle \check{Y}: ]-\epsilon_{(r,\theta)},\epsilon_{(r,\theta)}[
 \times \check{W} \to \check{W}$
defined by $\check{Y}(s,(r,\theta)) = (r+s,\theta)$ for
$\epsilon_{(r,\theta)}$ small enough.  Note that we effectively have
that $(\check{Y}_{(r,\theta)})_\ast(s,1) = \big( (r+s,\theta), (1,0)\big)
= \check{V}_1\big(\tilde{Y}_{(r,\theta)}(s)\big)$ for
$-\epsilon_{(r,\theta)} < s < \epsilon_{(r,\theta)}$.

We proceed as in Subsection~\ref{subsectEDLB} to get the local
representation of the vector field $\displaystyle Q_s = Y_s^\ast(V_2)$.
Let $\displaystyle \check{v}_2 :\check{W} \to \RR^2$ be the function
associated to the local representation $\check{V}_2$ of $V_2$ given in
(\ref{defnV1V2rt}); namely, $\displaystyle \check{v}_2 =
\begin{pmatrix} 0 & 1 \end{pmatrix}^\top$.

The local representation of the vector field
$\displaystyle Q_s = Y_s^\ast(V_2)$ is
$\displaystyle \tilde{Q}_s(r,\theta) = (\tilde{q}_s)_1(r,\theta)
\pdydx{}{r} + (\tilde{q}_s)_2(r,\theta) \pdydx{}{\theta}$ where
\[
\tilde{q}_s(r,\theta)
= \diff \check{Y}_{-s}(\tilde{Y}_s(r,\theta))
\, \check{v}_2(\check{Y}_s(r,\theta))
= \begin{pmatrix} 0 \\ 1 \end{pmatrix}
\]
for $(r,\theta) \in \check{W}$ and
$-\epsilon_{(r,\theta)} < s < \epsilon_{(r,\theta)}$.
All this work just to get that the local representation of
$\displaystyle Y_s^\ast(V_2)$ is
$\displaystyle \pdydx{}{\theta}\Big|_{(r,\theta) = \check{\phi}^{-1}(t,\eta)}$.

Therefore
\[
[V_1,V_2](t,\eta)(g) = \dfdx{\Big(
\pdfdx{(g\circ \check{\phi})}{\theta}\Big|_{(r,\theta) = \check{\phi}^{-1}(t,\eta)}
\Big)}{s}\Big|_{s=0}
= \dfdx{\Big(
\pdfdx{g(r,e^{i\theta})}{\theta}\Big|_{(r,\theta) = \check{\phi}^{-1}(t,\eta)}
\Big)}{s}\Big|_{s=0} = 0
\]
for all $\displaystyle g \in C^\infty(]0,\epsilon_{\VEC{u}}[\times S^1)$
since the expression between parentheses does not depend on $s$.
\end{rmk}

\begin{lemma} \label{lemmaMRs1}
$P_\ast\big( (r,\eta), (p,q \, \upsilon\,\eta)\big)
= \big( r(\theta.\VEC{e}_1), p(\theta.\VEC{e}_1) + q\,r (\theta + \pi/2).
\VEC{e}_1\big)$ for $\displaystyle \eta= e^{i\theta}$,
$\displaystyle \upsilon = e^{i(\pi/2)}$ and $p,q \in \RR$.
Note that $\displaystyle \big( (r,\eta), (p,q\,\upsilon\,\eta))\big)
\in \TS_{(t,\eta)}\, ]0,\epsilon_{\VEC{u}}[\times S^1$.
\end{lemma}

\begin{proof}
We consider a local chart $(\check{W},\check{U},\check{\phi})$ of
$\displaystyle [0,\epsilon_{\VEC{u}}[ \times S^1$.  Recall that
$\displaystyle \phi(r,\theta) = (r , e^{i\theta} )$
for $(r,\theta) \in \check{W} \subset ]0,\epsilon_{\VEC{u}}[\times \RR$.
The local representation of $P$ is given by
$\displaystyle \tilde{P}(r,\theta) = (P\circ \phi)(r,\theta)
= r (\theta.\VEC{e}_1)$ for $(r,\theta) \in \check{W}$.
We have
\[
\phi_\ast\big((r,\theta),(p,q)\big)
= \big( (r,e^{i\theta}), (p ,q e^{i(\theta+\pi/2)})\big)
= \big( (r,\eta), (p , q\, \upsilon\, \eta)\big)
\]
and
\[
\tilde{P}_\ast\big( (r,\theta) , (p,q)\big)
= \big(r (\theta.\VEC{e}_1), p(\theta.\VEC{e}_1) + q r \big((\theta +
\pi/2).\VEC{e}_1\big)\big)
\]
or all
$\displaystyle \big( (r,\theta) , (p,q)\big) \in \TS_{(r,\theta)} \check{W}$.
The conclusion of the lemma comes from the fact that
$P \circ \phi = \tilde{P}$ implies that
$P_\ast \circ \phi_\ast = \tilde{P}_\ast$.
\end{proof}

If we use complex representation of the elements of
$D_{\epsilon_{\VEC{u}}}$, then the conclusion of the previous lemma can be
restated as
$P_\ast( (r,\eta), (p,q \, \upsilon\,\eta))
= ( r \eta, p \eta + qr \, \upsilon\, \eta)$.

Before diving deeper into our study of geodesic coordinate systems, we
need to restate one of our previous results.  Using our knowledge of
the Lie derivative of a function along a vector field learned in
Remark~\ref{rmkdyVFwithOper}, we can restate
Proposition~\ref{propdyOmF1F2} in familiar terms.

\begin{cor} \label{cordyOmF1F2}
Suppose that $\omega$ is a smooth differential $1$-form on a smooth
$k$-dimensional manifold $S$, and that $F_1$ and $F_2$ are two smooth
vector fields on $S$.  Then
\begin{align*}
\df{\omega}(\VEC{u})\big( F_1(\VEC{u}), F_2(\VEC{u})\big)
&= \df{(\omega(F_2))}(\VEC{u})(F_1(\VEC{u}))
- \df{(\omega(F_1))}(\VEC{u})(F_2(\VEC{u})) \\
&\qquad\qquad - \omega(\VEC{u})([F_1,F_2](\VEC{u})) \ .
\end{align*}
for all $\VEC{u} \in S$.
\end{cor}

\begin{lemma}[Gauss]  \label{lemmaGps0}
$\ps{G_\ast\big(P_\ast(V_1(t,\eta))\big)}
{G_\ast\big(P_\ast(V_2(t,\eta))\big)}_{\pi_S(Q(t,\eta))} = 0$
for all $\displaystyle (t,\eta) \in ]0,\epsilon_{\VEC{u}}[\times S^1$.
\end{lemma}

\begin{proof}
We have
\[
G_\ast \circ P_\ast = (G \circ P)_\ast = (\pi_S \circ Q)_\ast
= (\pi_S)_\ast \circ Q_\ast \ .
\]
Moreover, since $Q_\ast(V_1(t,\eta)) = F_1(\tilde{\sigma}_\eta(t))
= (\tilde{\sigma}_\eta)_\ast(t,1)$
according to the first item of\\
Lemma~\ref{lemmaMRs2}, we have
\[
(\pi_S)_\ast\big(Q_\ast(V_1(t,\eta))\big)
= (\pi_S)_\ast\big((\tilde{\sigma}_\eta)_\ast(t,1)\big)
= \tilde{\sigma}_\eta(t) = Q(t,\eta)
\]
for all $\displaystyle (t,\eta) \in ]0,\epsilon_{\VEC{u}}[\times S^1$.
Therefore
\begin{align}
&\ps{G_\ast\big(P_\ast(V_1(t,\eta))\big)}
{G_\ast\big(P_\ast(V_2(t,\eta))\big)}_{\pi_S(Q(t,\eta))} \nonumber \\
&\qquad = \ps{(\pi_S)_\ast\big(Q_\ast(V_1(t,\eta))\big)}
{(\pi_S)_\ast\big(Q_\ast(V_2(t,\eta))\big)}_{\pi_S(Q(t,\eta))} \nonumber \\
&\qquad = \ps{Q(t,\eta)}
{(\pi_S)_\ast\big(Q_\ast(V_2(t,\eta))\big)}_{\pi_S(Q(t,\eta))} \nonumber \\
&\qquad = \omega_1(Q(t,\eta))\big(Q_\ast(V_2(t,\eta))\big)
= Q^\ast(\omega_1)(t,\eta)\big(V_2(t,\eta)\big) \label{geoCoordEq0}
\end{align}
for all $\displaystyle (t,\eta) \in ]0,\epsilon_{\VEC{u}}[\times S^1$.
We need to prove that the last expression in the relation above is null.

Using the first of the first structural equations in
Theorem~\ref{thmRconnect}, we have
\[
\df{(Q^\ast(\omega_1))} = Q^\ast(\df{\omega_1})
= Q^\ast(\rho \wedge \omega_2) = Q^\ast(\rho) \wedge Q^\ast(\omega_2) \ .
\]
Hence
\begin{align*}
&\df{(Q^\ast(\omega_1))}(t,\eta)\big( V_1(t,\eta), V_2(t,\eta)\big) \\
&\qquad = \frac{1}{2} \Big(
Q^\ast(\rho)(t,\eta)(V_1(t,\eta)) \,
Q^\ast(\omega_2)(t,\eta)(V_2(t,\eta)) \\
&\qquad \qquad - Q^\ast(\rho)(t,\eta)(V_2(t,\eta))
\, Q^\ast(\omega_2)(t,\eta)(V_1(t,\eta)) \Big) \\
&\qquad = \frac{1}{2} \Big(
\rho(Q(t,\eta))\big(Q_\ast(V_1(t,\eta))\big) \,
\omega_2(Q(t,\eta))\big(Q_\ast(V_2(t,\eta))\big) \\
&\qquad \qquad - \rho(Q(t,\eta))\big(Q_\ast(V_2(t,\eta))\big) \,
\omega_2(Q(t,\eta))\big(Q_\ast(V_1(t,\eta))\big) \Big)
= 0
\end{align*}
for all $\displaystyle (t,\eta) \in ]0,\epsilon_{\VEC{u}}[\times S^1$
because $\rho(Q(t,\eta))\big(Q_\ast(V_1(t,\eta))\big) = 0$
and $\omega_2(Q(t,\eta))\big(Q_\ast(V_1(t,\eta)) = 0$
according to second item of Lemma~\ref{lemmaMRs2}.

It follows from Corollary~\ref{cordyOmF1F2} that
\begin{equation} \label{geoCoordEq1}
\begin{split}
0 &= \df{(Q^\ast(\omega_1))}(t,\eta)\big( V_1(t,\eta), V_2(t,\eta)\big) \\
&= \df{\big(Q^\ast(\omega_1)(V_2)\big)}(t,\eta)(V_1(t,\eta))
- \df{\big(Q^\ast(\omega_1)(V_1)\big)}(t,\eta)(V_2(t,\eta)) \\
&\qquad - Q^\ast(\omega_1)(t,\eta)([V_1,V_2](t,\eta))
\end{split}
\end{equation}
for all $\displaystyle (t,\eta) \in ]0,\epsilon_{\VEC{u}}[\times S^1$.  However,
$\displaystyle \df{\big(Q^\ast(\omega_1)(V_1)\big)} = 0$ because
$\displaystyle Q^\ast(\omega_1)(V_1) = 1$ on
$\displaystyle ]0,\epsilon_{\VEC{u}}[\times S^1$
according to the second item of Lemma~\ref{lemmaMRs2}, and
$\displaystyle Q^\ast(\omega)(t,\eta)([V_1,V_2](t,\eta)) = 0$
for all $\displaystyle (t,\eta) \in ]0,\epsilon_{\VEC{u}}[\times S^1$
because $[V_1,V_2](t,\eta) = \big((t,\eta), (0,0) \big)$ according to
the third item of Lemma~\ref{lemmaMRs2}.
Hence, we get from (\ref{geoCoordEq1}) that
\begin{equation} \label{geoCoordEq2}
\df{\big(Q^\ast(\omega_1)(V_2)\big)}(t,\eta)(V_1(t,\eta)) = 0
\end{equation}
for all $\displaystyle (t,\eta) \in ]0,\epsilon_{\VEC{u}}[\times S^1$.
It is again advantageous to use the local representation of $V_1$ and $V_2$
provided in (\ref{defnV1V2rt}), and our knowledge of the Lie
derivative of a function along a vector field, to get from
(\ref{geoCoordEq2}) that
\[
V_1(t,\eta)\big(Q^\ast(\omega_1)(V_2)\big)
= \pdfdx{\Big( Q^\ast(\omega_1)(V_2) \circ \phi\Big)\Big)}{r}
\Big|_{(r,\theta) = \phi^{-1}(t,\eta)} = 0
\]
for $\displaystyle (t,\eta) \in ]0,\epsilon_{\VEC{u}}[\times S^1$.
Thus
$\displaystyle \big(Q^\ast(\omega_1)(V_2)\big)(t,\eta)
= Q^\ast(\omega_1)(t,\eta)\big(V_2(t,\eta)\big)$
is constant with respect to $t = r$.  Therefore, it follows from
(\ref{geoCoordEq0}) that
\[
\ps{G_\ast\big(P_\ast(V_1(t,\eta))\big)}
{G_\ast\big(P_\ast(V_2(t,\eta))\big)}_{\pi_S(Q(t,\eta))}
= \ps{Q(t,\eta)}{G_\ast\big(P_\ast(V_2(t,\eta))\big)}_{\pi_S(Q(t,\eta))}
\]
is a constant function with respect to $t$.  However,
$Q(t,\eta) \to \eta(\VEC{u},\tilde{E}(\VEC{u}))$ as $t \to 0$ and
\[
P_\ast(V_2(t,\eta))= P_\ast\big( (t,\eta), (0,\upsilon \eta)\big)
= (t \,\eta, t \,\upsilon \eta) \to (0,0)
\]
as $t \to 0$ where we have used Lemma~\ref{lemmaMRs1} to get the
last equality.  Thus\\
$\displaystyle 
\ps{G_\ast\big(P_\ast(V_1(t,\eta))\big)}
{G_\ast\big(P_\ast(V_2(t,\eta))\big)}_{\pi_S(Q(t,\eta))} \to 0$
as $t \to 0$.  Since this inner product is constant, it must
therefore be null for all $t$.
\end{proof}

\begin{cor}  \label{corMRs3}
$\displaystyle Q^\ast(\omega_1)(V_2) = 0$ on
$\displaystyle ]0,\epsilon_{\VEC{u}}[\times S^1$. 
\end{cor}

\begin{proof}
This follows from (\ref{geoCoordEq0}) and Gauss' lemma.
\end{proof}

It follows from Gauss' lemma that
$\displaystyle t \mapsto G_\ast\big(P_\ast(V_2(t,\eta))\big)$ is
orthogonal to \\
$\displaystyle t \mapsto G_\ast\big(P_\ast(V_1(t,\eta))\big) = Q(t,\eta) =
\tilde{\sigma}_\eta(t)$ for all $t \in [0,\epsilon_{\VEC{u}}[$; namely,
the geodesic curve $t \mapsto \sigma_\eta)t) = \pi_S(\tilde{\sigma}_\eta(t))$
and the curve $t \mapsto G(P((t,\eta)))
= \pi_S\big(G_\ast\big(P_\ast(V_2(t,\eta))\big)\big)$ have an
orthogonal intersect.

We now show that the length of $G_\ast\big(P_\ast(V_2(t,\eta))\big)$
along the geodesic curve $\sigma_\eta$ satisfies a differential equation.

We have
\begin{align*}
G_\ast\big(P_\ast(V_2(t,\eta))\big)
&= (\pi_S)_\ast\big(Q_\ast(V_2(t,\eta))\big) \\
&= \omega_1(Q(t,\eta))\big(Q_\ast(V_2(t,\eta))\big) \, Q(t,\eta)
+ \omega_2(Q(t,\eta))\big(Q_\ast(V_2(t,\eta))\big) \, \upsilon(Q(t,\eta)) \\
&= \omega_2(Q(t,\eta))\big(Q_\ast(V_2(t,\eta))\big) \, \upsilon(Q(t,\eta))
= Q^\ast(\omega_2)(t,\eta)\big(V_2(t,\eta)\big) \, \upsilon(Q(t,\eta))
\end{align*}
because
$\displaystyle \omega_1(Q(t,\eta))\big(Q_\ast(V_2(t,\eta))\big)
= Q^\ast(\omega_1)(t,\eta)\big(V_2(t,\eta)\big) = 0$ according to
Corollary~\ref{corMRs3}.  Thus
\begin{equation} \label{geoCoordEq3}
\|G_\ast\big(P_\ast(V_2(t,\eta))\big)\|_{\pi_S(Q(t,\eta))}
= \big|Q^\ast(\omega_2)(t,\eta)\big(V_2(t,\eta)\big)\big|
\end{equation}
for all
$\displaystyle (t,\eta) \in [0,\epsilon_{\VEC{u}}[ \times S^1$,
where the norm refer to the Riemannian metric on $S$.

We get from the second of the first structural equations that
\[
\df{\big(Q^\ast(\omega_2)\big)} = Q^\ast(\df{\omega_2})
= -Q^\ast(\rho \wedge \omega_1)
= -Q^\ast(\rho) \wedge Q^\ast(\omega_1) \ .
\]
Thus
\begin{equation} \label{geoCoordEq5}
\begin{split}
&\df{\big(Q^\ast(\omega_2)\big)}(t,\eta)\big(V_1(t,\eta),V_2(t,\eta)\big)
= \frac{1}{2} \Big(
Q^\ast(\rho)(t,\eta)(V_2(t,\eta)) \,
Q^\ast(\omega_1)(t,\eta)(V_1(t,\eta)) \\
&\qquad \qquad -Q^\ast(\rho)(t,\eta)(V_1(t,\eta))\,
Q^\ast(\omega_1)(t,\eta)(V_2(t,\eta)) \Big)
= \frac{1}{2} Q^\ast(\rho)(t,\eta)(V_2(t,\eta))
\end{split}
\end{equation}
for all
$\displaystyle (t,\eta) \in [0,\epsilon_{\VEC{u}}[ \times S^1$
because $\displaystyle Q^\ast(\omega_1)(t,\eta)(V_2(t,\eta)) = 0$ according to
Corollary~\ref{corMRs3} and
$\displaystyle Q^\ast(\omega_1)(t,\eta)(V_1(t,\eta)) = 1$ according to
the second item of Lemma~\ref{lemmaMRs2}.

However, using Corollary~\ref{cordyOmF1F2}, we have
\begin{align}
&\df{\big(Q^\ast(\omega_2)\big)}(t,\eta)\big(V_1(t,\eta),V_2(t,\eta)\big)
\nonumber \\
&\qquad = \df{\big(Q^\ast(\omega_2)(V_2)\big)}(t,\eta)(V_1(t,\eta))
- \df{\big(Q^\ast(\omega_2)(V_1)\big)}(t,\eta)(V_2(t,\eta)) \nonumber \\
&\qquad \qquad - Q^\ast(\omega_2)(t,\eta)([V_1,V_2](t,\eta)) \nonumber \\
&\qquad = \df{\big(Q^\ast(\omega_2)(V_2)\big)}(t,\eta)(V_1(t,\eta))
\label{geoCoordEq6}
\end{align}
because $\displaystyle Q^\ast(\omega_2)(V_1)=0$ and
$[V_1,V_2](t,\eta) = \big((t,\eta), (0,0) \big)$ for all
$\displaystyle (t,\eta) \in [0,\epsilon_{\VEC{u}}[ \times S^1$
according to the second and third items of Lemma~\ref{lemmaMRs2}.
We deduce from (\ref{geoCoordEq5}) and (\ref{geoCoordEq6}) that
\begin{equation} \label{geoCoordEq10}
\df{\big(Q^\ast(\omega_2)(V_2)\big)}(t,\eta)(V_1(t,\eta))
= \frac{1}{2} Q^\ast(\rho)(t,\eta)(V_2(t,\eta))
\end{equation}
for all $\displaystyle (t,\eta) \in [0,\epsilon_{\VEC{u}}[ \times S^1$.
 
We now almost repeat the previous discussion about
$\displaystyle Q^\ast(\omega_2)$ but this time with
$\displaystyle Q^\ast(\rho)$.  From the second structural equations, 
we have
\[
\df{\big(Q^\ast(\rho)\big)} = Q^\ast(\df{\rho})
= -Q^\ast((\kappa \circ \pi_S)\omega_1 \wedge \omega_2)
= - (\kappa \circ \pi_S \circ Q)  Q^\ast(\omega_1) \wedge Q^\ast(\omega_2) \ .
\]
Thus
\begin{align}
&\df{\big(Q^\ast(\rho)\big)}(t,\eta)\big(V_1(t,\eta),V_2(t,\eta)\big)
\nonumber \\
&\qquad = \frac{1}{2} \kappa(\pi_S(Q(t,\eta))) \Big(
Q^\ast(\omega_1)(t,\eta)(V_2(t,\eta)) \,
Q^\ast(\omega_2)(t,\eta)(V_1(t,\eta)) \nonumber \\
&\qquad \qquad -Q^\ast(\omega_1)(t,\eta)(V_1(t,\eta))\,
Q^\ast(\omega_2)(t,\eta)(V_2(t,\eta)) \Big) \nonumber \\
&\qquad
= -\frac{1}{2} \kappa(\pi_S(Q(t,\eta)))\, Q^\ast(\omega_2)(t,\eta)(V_2(t,\eta))
\label{geoCoordEq7}
\end{align}
for all $\displaystyle (t,\eta) \in [0,\epsilon_{\VEC{u}}[ \times S^1$
because $\displaystyle Q^\ast(\omega_1)(t,\eta)(V_2(t,\eta)) = 0$ 
and $\displaystyle Q^\ast(\omega_1)(t,\eta)(V_1(t,\eta)) = 1$ according to the
second item of Lemma~\ref{lemmaMRs2}.

Using Corollary~\ref{cordyOmF1F2}, we have
\begin{align}
&\df{\big(Q^\ast(\rho)\big)}(t,\eta)\big(V_1(t,\eta),V_2(t,\eta)\big)
\nonumber \\
&\qquad = \df{\big(Q^\ast(\rho)(V_2)\big)}(t,\eta)(V_1(t,\eta))
- \df{\big(Q^\ast(\rho)(V_1)\big)}(t,\eta)(V_2(t,\eta)) \nonumber \\
&\qquad \qquad - Q^\ast(\rho)(t,\eta)([V_1,V_2](t,\eta)) \nonumber \\
&\qquad = \df{\big(Q^\ast(\rho)(V_2)\big)}(t,\eta)(V_1(t,\eta))
\label{geoCoordEq8}
\end{align}
because $\displaystyle Q^\ast(\rho)(V_1)=0$ and
$[V_1,V_2](t,\eta) = \big((t,\eta), (0,0) \big)$ for all
$\displaystyle (t,\eta) \in [0,\epsilon_{\VEC{u}}[ \times S^1$
according to the second and third items of Lemma~\ref{lemmaMRs2}.

We deduce from (\ref{geoCoordEq7}) and (\ref{geoCoordEq8}) that
\begin{align}
\df{\big(Q^\ast(\rho)(V_2)\big)}(t,\eta)(V_1(t,\eta))
&= -\frac{1}{2} \kappa(\pi_S(Q(t,\eta)))\,
Q^\ast(\omega_2)(t,\eta)(V_2(t,\eta)) \nonumber \\
&= -\frac{1}{2} \kappa(G(P(t,\eta)))\, Q^\ast(\omega_2)(t,\eta)(V_2(t,\eta))
\label{geoCoordEq11}
\end{align}
for all $\displaystyle (t,\eta) \in [0,\epsilon_{\VEC{u}}[ \times S^1$.

If we rewrite (\ref{geoCoordEq10}) and
(\ref{geoCoordEq11}) using the local representation of $V_1$ and $V_2$
provided in (\ref{defnV1V2rt}), and our knowledge of
the Lie derivative of a function along a vector field, we respectively get
\begin{align}
\pdfdx{\Big(\big(Q^\ast(\omega_2)(V_2)\big) \circ \check{\phi}\Big)}{r}
\Big|_{(r,\theta) = \check{\phi}^{-1}(t,\eta)}
&= \frac{1}{2} Q^\ast(\rho)(t,\eta)(V_2(t,\eta)) \nonumber \\
&= \frac{1}{2} \Big(\big( Q^\ast(\rho)(V_2) \big)\circ \check{\phi}\Big)
\Big|_{(r,\theta) = \check{\phi}^{-1}(t,\eta)}  \label{geoCoordEq12}
\end{align}
and
\begin{align}
&\pdfdx{\Big(\big(Q^\ast(\rho)(V_2)\big) \circ \check{\phi}\Big)}{r}
\Big|_{(r,\theta) = \check{\phi}^{-1}(t,\eta)}
= -\frac{1}{2} \kappa(G(P(t,\eta)))\, Q^\ast(\omega_2)(t,\eta)(V_2(t,\eta))
\nonumber \\
&\hspace{7em} = -\frac{1}{2} \kappa(G(P(t,\eta)))\,
\Big(\big( Q^\ast(\omega_2)(V_2) \big)\circ \check{\phi}\Big)
\Big|_{(r,\theta) = \check{\phi}^{-1}(t,\eta)}  \label{geoCoordEq13}
\end{align}
for all $\displaystyle (t,\eta) \in [0,\epsilon_{\VEC{u}}[ \times S^1$.
Substituting (\ref{geoCoordEq13}) in the derivative with respect to $r$
of (\ref{geoCoordEq12}) yields
\[
\pdfdxn{\Big(\big(Q^\ast(\omega_2)(V_2)\big) \circ \check{\phi}\Big)}{r}{2}
\Big|_{(r,\theta) = \check{\phi}^{-1}(t,\eta)}
= -\frac{1}{4} \kappa(G(P(t,\eta)))\,
\Big(\big( Q^\ast(\omega_2)(V_2)\big) \circ \check{\phi}\Big)
\Big|_{(r,\theta) = \check{\phi}^{-1}(t,\eta)}
\]
for all $\displaystyle (t,\eta) \in [0,\epsilon_{\VEC{u}}[ \times S^1$.
For this second order differential equation to completely determine
$\displaystyle Q^\ast(\omega_2)(V_2) \circ \check{\phi}$ given $\eta$
fixed (i.e.\ along a geodesic), we need initial condition at $t=0$ for
$\displaystyle Q^\ast(\omega_2)(V_2)\circ \check{\phi}$ and its
derivative with respect to $r$.  We have from the fourth item of
Lemma~\ref{lemmaMRs2} that
\[
Q^\ast(\omega_2)(V_2)\circ \check{\phi}\big|_{(r,\theta)=\check{\phi}^{-1}(0,\eta)}
= \omega_2(Q(0,\eta))\Big(Q_\ast(V_2(t,\eta)) \big|_{t=0}\Big)
= \omega_2(\tilde{\sigma}_\eta(0))\big(H(\tilde{\sigma}_\eta(0))\big)
= 0
\]
by definition of $\omega_2$ and $H$.  We also get from
(\ref{geoCoordEq12}) that
\begin{align*}
&\pdfdx{\Big(\big(Q^\ast(\omega_2)(V_2)\big) \circ \check{\phi}\Big)}{r}
\Big|_{(r,\theta) = \check{\phi}^{-1}(0,\eta)}
= \frac{1}{2} Q^\ast(\rho)(t,\eta)(V_2(t,\eta))\Big|_{t=0} \\
&\qquad \qquad = \rho(Q(0,\eta))\big(Q_\ast(V_2(t,\eta)) \big|_{t=0}\big)
= \frac{1}{2} \rho(\tilde{\sigma}_\eta(t))
\big( H(\tilde{\sigma}_\eta(0)) \big)
= \frac{1}{2} \ .
\end{align*}
Since
$\displaystyle \big(Q^\ast(\omega_2)(V_2)\circ \check{\phi}\big)
\big|_{(r,\theta)=\check{\phi}^{-1}(0,\eta)} = 0$ and
$\displaystyle \pdfdx{\Big(\big(Q^\ast(\omega_2)(V_2)\big) \circ \check{\phi}
\Big)}{r}\Big|_{(r,\theta) = \check{\phi}^{-1}(0,\eta)} >0$, we have
that
$\displaystyle \big(Q^\ast(\omega_2)(V_2)\circ \check{\phi}\big)
\big|_{(r,\theta)=\check{\phi}^{-1}(t,\eta)} > 0$ for $t\geq 0$ small
enough.  It then follows from (\ref{geoCoordEq3}) that
\[
\|G_\ast\big(P_\ast(V_2(t,\eta))\big)\|_{G(P(t,\eta))}
= \|G_\ast\big(P_\ast(V_2(t,\eta))\big)\|_{\pi_S(Q(t,\eta))}
= Q^\ast(\omega_2)(t,\eta)\big(V_2(t,\eta)\big)
\]
for $t\geq 0$ small enough.  We have proved the following theorem.

\begin{theorem}
Let
\begin{enumerate}
\item $S$ be an oriented $2$-dimensional Riemannian manifold and
$\VEC{u} \in S$,
\item $G:D_{\epsilon_{\VEC{u}}} \to S$ be a geodesic coordinate system for
$S$ about $\VEC{u}$,
\item $t \mapsto \sigma_\eta(t) = G(P(t,\eta))$ for
$\displaystyle \eta = e^{i\theta} \in S^1$ fixed be a geodesic on $S$ through
$\VEC{u}$, and
\item $\displaystyle h(r) = \big(Q^\ast(\omega_2)(V_2) \circ
\check{\phi}\big)(r,\theta)
= \|G_\ast\big(P_\ast(V_2(\check{\phi}(r,\theta)))\big)
\|_{G(P(\check{\phi}(r,\theta)))}$ for $(t,\eta) = \check{\phi}(r,\theta)$.
\end{enumerate}
Then
\begin{equation} \label{defnJacobEqu}
  \dydxn{h}{r}{2}(r) + \frac{1}{4} \kappa(G(P(\check{\phi}(r,\theta))))
  h(r) = 0
\end{equation}
with $h(0) = 0$ and $h'(0) = 1/2$.  This ordinary differential
equation is called the
{\bfseries Jacobi's equation}\index{Jacobi's Equation}.  
\end{theorem}

The previous theorem states that the size of the vector field
$G_\ast(P_\ast(V_2))$ along the geodesic $\sigma_\eta$ satisfies 
an ordinary differential equation.  Note that the Jacobi's equation
(\ref{defnJacobEqu}) often stated in the literature does not
include the coefficient $1/4$ and the initial condition on the first
derivative of $h$ at the origin is $1$ instead of $1/2$.  The
difference between our formulation and the general formulation of
Jacobi's equation comes from the difference between our definition of
wedge product and the definition of wedge product used by some
authors.  They do not include the factor $1/2$ in their definition.

One of the fundamental roles that is played by geodesic curves is
given in the following theorem.

\begin{theorem}
Let
\begin{enumerate}
\item $S$ be an oriented $2$-dimensional Riemannian manifold and 
$\VEC{u} \in S$,
\item $G:D_{\epsilon_{\VEC{u}}} \to S$ be a geodesic coordinate system for
  $S$ about $\VEC{u}$,
\item $t \mapsto \sigma_\eta(t) = G(P(t,\eta))$ for
$\displaystyle \eta = e^{i\theta} \in S^1$ fixed be a geodesic on $S$ through
$\VEC{u}$, and
\item $\tilde{\VEC{u}} = \sigma_\eta(\epsilon)$ for some
$0 < \epsilon < \epsilon_{\VEC{u}}$.
\end{enumerate}
Then the length of the geodesic curve $\sigma_\eta$ between
$\VEC{u}$ and $\tilde{\VEC{u}}$ is less or equal to the length of any
piecewise $\displaystyle C^\infty$ curves in $S$ between
$\VEC{u}$ and $\tilde{\VEC{u}}$.
\end{theorem}

\begin{proof}
Recall that the length of a curve given by the piecewise
$\displaystyle C^\infty$ parametric representation $\alpha:[a,b] \to S$
is given by
\[
  L_{\alpha} = \int_a^b \|\alpha_\ast(t,1)\|_{\VEC{u}} \dx{t} \ .
\]

Let $\tilde{K} : D_{\epsilon_{\VEC{u}}} \to D_{\epsilon_{\VEC{u}}}$ be
the map defined by
\[
\tilde{K}(z)
=\begin{cases}
t \, \eta & \quad \text{if} \ z = t \mu \ \text{with} \ t > 0 \\
0 & \quad \text{if} \ z = 0 \\
\end{cases}
\]
where $\displaystyle \mu \in S^1$.  We have also used the complex
representation of the elements of $D_{\epsilon_{\VEC{u}}}$.
Let $U = G(D_{\epsilon_{\VEC{u}}}) \subset S$ and
$\displaystyle K = G \circ \tilde{K} \circ G^{-1}:U \to U$.
\pdfbox{riemann_geom/geodfig2}

\stage{i} We first prove that $K_\ast:\TS_{\VEC{v}} U \to \TS_{K(\VEC{v})} U$
satisfies
$\| K_\ast(\VEC{v},\VEC{x}) \|_{K(\VEC{v})} \leq \| (\VEC{v},\VEC{x})
\|_{\VEC{v}}$ for all $(\VEC{v},\VEC{x}) \in \TS_{\VEC{v}} U$.

We have from Gauss' lemma that
$\{ G_\ast(P_\ast(V_1(t,\xi))), G_\ast(P_\ast(V_2(t,\xi))) \}$ is an
orthogonal basis of $\TS_{\VEC{v}} U$ where $\VEC{v} = G(P(t, \xi)) =
\pi_S(Q(t,\xi))$.  Given $(\VEC{v}, \VEC{x}) \in \TS_{\VEC{v}} U$,
we may write
\[
(\VEC{v}, \VEC{x})
= a G_\ast(P_\ast(V_1(t,\xi))) + b G_\ast(P_\ast(V_2(t,\xi)))
\]
where $a$ and $b$ are the orthogonal projection of  
$(\VEC{v}, \VEC{x})$ on $G_\ast(P_\ast(V_1(t,\xi)))$ and
$G_\ast(P_\ast(V_2(t,\xi)))$ respectively.

Since $t \mapsto \sigma_\xi(t) = G(P(t,\xi))$ is a geodesic,
we have that
$(\sigma_\xi(t),\sigma_\xi'(t)) = (G\circ P)_\ast(V_1(t,\xi))$ is of norm $1$.
It follows from the orthogonality of
$G_\ast(P_\ast(V_1(t,\xi)))$ and $G_\ast(P_\ast(V_2(t,\xi)))$ that
\begin{align*}
\big\| (\VEC{v}, \VEC{x}) \big\|_{\VEC{v}}
&= |a|\, \big\| G_\ast(P_\ast(V_1(t,\xi)))\big\|_{\VEC{v}}
+ |b|\, \big\|G_\ast(P_\ast(V_2(t,\xi))) \big\|_{\VEC{v}} \\
&= |a| + |b|\, \big\|G_\ast(P_\ast(V_2(t,\xi))) \big\|_{\VEC{v}} \ .
\end{align*}

Moreover, we get from
$\displaystyle (\tilde{K} \circ P)(t,\xi) = \tilde{K}(t\xi) = t\eta$
that $\displaystyle \pdfdx{(\tilde{K} \circ P)}{t}(t,e^{i\nu})
= \pdfdx{(t e^{i\theta})}{t} = e^{i\theta} = \eta$ and
$\displaystyle \pdfdx{(\tilde{K} \circ P)}{\nu}(t,e^{i\nu})
= \pdfdx{(t e^{i\theta})}{\nu} = 0$.  Thus
\[
  \tilde{K}_\ast(P_\ast(V_1(t,\xi)))
  = (\tilde{K} \circ P)_\ast\big((t,\xi),(1,0)\big)
  = (t\eta, \eta) = P_\ast(V_1(t,\eta))
\]
and
\[
\tilde{K}_\ast(P_\ast(V_2(t,\xi)))
= (\tilde{K} \circ P)_\ast\big((t,\xi),(0,1)\big)
= (t\eta, 0) \ .
\]
where we have use Lemma~\ref{lemmaMRs1} for the last equality of the
first relation.  Since\\
$K_\ast(G_\ast(P_\ast(V_i(t,\eta))))
= G_\ast(\tilde{K}_\ast(P_\ast(V_i(t,\eta))))$ for $i=1,2$, we get
\[
K_\ast(\VEC{v}, \VEC{x}) = a K_\ast(G_\ast(P_\ast(V_1(t,\xi))))
= a G_\ast(P_\ast(V_1(t,\eta))) \ .
\]
Hence
\begin{align*}
\big\|K_\ast(\VEC{v}, \VEC{x})\big\|_{K(\VEC{v})}
&= |a| \big\|K_\ast(G_\ast(P_\ast(V_1(t,\xi))))\big\|_{K(\VEC{v})}
= |a| \big\|G_\ast(P_\ast(V_1(t,\eta)))\big\|_{K(\VEC{v})} \\
&= |a| \leq \big\|(\VEC{v}, \VEC{x}) \big\|_{\VEC{v}}
\end{align*}
where again we have used the fact that
$t \mapsto \sigma_\eta(t) = G(P(t,\eta))$ is a geodesic to get the
third equality.

\stage{ii}
Suppose that $\beta:[a,b] \to S$ is a piecewise
$\displaystyle C^\infty$ curve between $\VEC{u}$ and
$\tilde{\VEC{u}}$.  If $\beta(t) \not\in U$ for some
$a \leq t < b$, let $c = \inf \{ t \in [a,b] : \beta(t) \not\in U \}$.
We have that $c>a$ because $\beta(a) = \VEC{u} \in U$ and $U$ is an
open subset of $S$.  Let $\beta_1 = \beta\big|_{[a,c[}$.  We have that
$\beta_1 \in U$ for $a\leq t < c$.  Since
$L_{\beta_1} \leq L_{\beta}$, we will get the conclusion of the
theorem if we prove that $L_{\sigma_\eta} \leq L_{\beta_1}$.
\pdfbox{riemann_geom/geodfig3}

Let $\beta_2 = K\circ \beta_1$.  We have from (i) above that
\begin{align*}
L_{\beta_2} &= \int_a^c \big\|(\beta_2)_\ast(t,1) \big\|_{\beta_2(t)} \dx{t}
= \int_a^c \big\| K_\ast\big((\beta_1)_\ast(t,1)\big) \big\|_{\beta_2(t)} \dx{t}
\\
& \leq \int_a^c \big\| (\beta_1)_\ast(t,1) \big\|_{\beta_1(t)} \dx{t}
= L_{\beta_1} \ .
\end{align*}
We have that
$\beta_2([a,c[) = \sigma_\eta([0,\epsilon_{\VEC{u}}[)$.
Since
$\sigma_\eta:[0,\epsilon_{\VEC{u}}[ \to \sigma_\eta([0,\epsilon_{\VEC{u}}[)$
is a bijection, we may define $g:[a,c[ \to [0,\epsilon_{\VEC{u}}[$ by 
$\displaystyle g = \sigma_\eta^{-1} \circ \beta_2$.  The function is piecewise
$\displaystyle C^\infty$.
We get
\begin{align*}
L_{\beta_2} &= \int_a^c \big\|(\beta_2)_\ast(t,1) \big\|_{\beta_2(t)} \dx{t}
= \int_a^c \big\|(\sigma_\eta \circ g)_\ast(t,1) \big\|_{\beta_2(t)} \dx{t}
= \int_a^c \big\|(\sigma_\eta)_\ast(g_\ast(t,1)) \big\|_{\beta_2(t)} \dx{t} \\
&= \int_a^c \big\|(\sigma_\eta)_\ast(g(t),g'(t)) \big\|_{\beta_2(t)} \dx{t}
= \int_a^c \big\|\big( \sigma_\eta(g(t)),\sigma_\eta'(g(t)) g'(t)\big)
\big\|_{\beta_2(t)} \dx{t} \\
&= \int_a^c \big\|\big(\sigma_\eta(g(t)),\sigma_\eta'(g(t))\big)
\big\|_{\beta_2(t)} \, |g'(t)| \dx{t}
= \int_0^{\epsilon_{\VEC{u}}} \big\|(\sigma_\eta(s),\sigma_\eta'(s))
\big\|_{\sigma_\eta(s)} \dx{s} \\
&\geq \int_0^{\epsilon_{\VEC{u}}} \big\|\sigma_\eta(s),\sigma_\eta'(s) )
\big\|_{\sigma_\eta(s)} \dx{s}
= L_{\sigma_\eta} \ . \qedhere
\end{align*}
\end{proof}

\begin{theorem}
Let $S$ be a connected and oriented $2$-dimensional Riemannian manifold,
and $\tau : S \times S \to [0,\infty[$ be the map defined by
\[
\tau(\VEC{u}_1,\VEC{u}_2)
= \inf \big\{ L_{\beta} : \beta\ \text{is a piecewise
$\displaystyle C^\infty$ curve in $S$ between $\VEC{u}_1$ and
$\VEC{u}_2$} \big\}
\]
Then $\tau$ is a metric on $S$ and the topology generated by this
metric is identical to the initial topology on the manifold $S$.
\end{theorem}

\begin{proof}
We first prove that $\tau$ is a metric.
\begin{enumerate}
\item Given $\VEC{u}_1,\VEC{u}_2 \in S$, since $S$ is connected and so
path connected \footnote{This is true because $S$ is locally
homeomorphic to $\RR^2$.  In general, connected does not imply path
connected.}, there exists at list one piecewise $\displaystyle C^\infty$
curve between $\VEC{u}_1$ and $\VEC{u}_2$.  Therefore
$\tau(\VEC{u}_1,\VEC{u}_2)$ is defined.
\item It is clear that $\tau(\VEC{u}_1,\VEC{u}_2) = \tau(\VEC{u}_2,\VEC{u}_1)$
for all $\VEC{u}_1, \VEC{u}_2 \in S$ by definition of $\tau$.
\item Given $\VEC{u}_1,\VEC{u}_2,\VEC{u}_3 \in S$ and $\epsilon >0$.
There exist two piecewise $\displaystyle C^\infty$ curves $\beta_1$
and $\beta_2$ in $S$ such that $\beta_1$ is a curve between
$\VEC{u}_1$ and $\VEC{u}_2$ satisfying
$\tau(\VEC{u}_1,\VEC{u}_2) \geq L_{\beta_1} - \epsilon/2$, and
$\beta_2$ is a curve between $\VEC{u}_2$ and $\VEC{u}_3$ satisfying
$\tau(\VEC{u}_2,\VEC{u}_3) \geq L_{\beta_2} - \epsilon/2$.
We then have that $\beta_1 \beta_2$ is a curve between $\VEC{u}_1$
and $\VEC{u}_3$ such that
\[
\tau(\VEC{u}_1,\VEC{u}_3) \leq L_{\beta_1\beta_2}
\leq L_{\beta_1} + L_{\beta_2}
\leq \tau(\VEC{u}_1,\VEC{u}_2) + \tau(\VEC{u}_2,\VEC{u}_3) + \epsilon \ .
\]
Since $\epsilon >0$ is arbitrary, we get
\[
\tau(\VEC{u}_1,\VEC{u}_3) \leq \tau(\VEC{u}_1,\VEC{u}_2)
+ \tau(\VEC{u}_2,\VEC{u}_3) .
\]
\item If $\VEC{u}_1 = \VEC{u}_2$, then
$\tau(\VEC{u}_1,\VEC{u}_2) = 0$ because we can consider the path
$\beta:[0,1] \to S$ defined by $\beta(t) = \VEC{u}_1$ for
$0 \leq t \leq 1$.

Suppose that $\VEC{u}_1 \neq \VEC{u}_2$.
Let $G:D_{\epsilon_{\VEC{u}_1}} \to S$ be a geodesic coordinate system for
$S$ about $\VEC{u}_1$.  We may choose $\epsilon_{\VEC{u}_1}$ small
enough to have $\VEC{u}_2 \not\in U = G(D_{\epsilon_{\VEC{u}_1}})$.
If $\sigma_\eta(t) = G(P(t,\eta))$ for $0 \leq t < \epsilon_{\VEC{u}}$
is any geodesic curve, we have that $L_{\sigma_\eta} \leq L_{\beta}$
for all piecewise $\displaystyle C^\infty$ curves in $S$ between $\VEC{u}_1$
and $\VEC{u}_2$ because $\VEC{u}_2 \not\in U$.  Note that
$\epsilon_{\VEC{u}} = L_{\sigma_\eta}$ for all geodesics
$\sigma_\eta$ because it is parameterized by arc length.  We get that
$0 < \epsilon_{\VEC{u}} \leq \tau(\VEC{u}_1,\VEC{u}_2)$.
\end{enumerate}

It remains to prove that the topology generated by $\tau$ is identical
to the initial topology on the manifold $S$.
Let $G:D_{\epsilon_{\VEC{u}}} \to S$ be a geodesic coordinate system for
$S$ about $\VEC{u} \in S$.  Then, as we did for (4) above, we can show
that 
\[
B_\epsilon(\VEC{u}) \equiv \big\{ \VEC{v} \in S :
\tau(\VEC{u}.\VEC{v}) < \epsilon\} = G(D_{\epsilon})
\]
for all $0 < \epsilon < \epsilon_{\VEC{u}}$.  Thus, the two topology
have the same basis of open sets.
\end{proof}

\begin{theorem}
For $i =1,2$, suppose that $S_i$ is an 
oriented $2$-dimensional Riemannian manifold, and that
$G_i:D_{\epsilon} \to S_i$ is a geodesic coordinate system for
$S_i$ about $\VEC{u}_i$.  Let $U_i = G_i(D_{\epsilon})$
and $\kappa_i:S_i \to \RR$ be the curvature function on $S_i$ for
$1\leq i \leq 2$.  If
$\displaystyle \kappa_1 = \kappa_2 \circ G_2 \circ G_1^{-1}$
on $U_1 = G_1(D_{\epsilon})$, then $\displaystyle G_2 \circ G_1^{-1}$ is
an isometry.
\end{theorem}

\begin{proof}
We prove that
$\displaystyle (G_2 \circ G_1^{-1})_\ast: \TS U_1 \to \TS U_2$ is an
isometry.

We have that $\displaystyle G_2 \circ G_1^{-1}$ maps the geodesic
$\sigma_{1,\eta}(t) = (G_1 \circ P)(t,\eta)$ onto the geodesic 
$\sigma_{2,\eta}(t) = (G_2 \circ P)(t,\eta)$ for
$0 \leq t \leq \epsilon$ and $\displaystyle \eta \in S^1$.  As we have
seen at the beginning of the proof of Gauss' lemma, we have that
$(G_i \circ P)_\ast(V_1(t,\eta)) = \tilde{\sigma}_{i,\eta}(t)$
for $i =1,2$.  Hence
$\| (G_i\circ P)_\ast (V_1(t,\eta)) \|_{(G_i\circ P)(t,\eta)}
= \| \tilde{\sigma}_{i,\eta}(t) \|_{\sigma_{i,\eta}(t)} = 1$
for $i =1,2$ because the geodesics are parameterized by arc
length.  Thus $\displaystyle G_2 \circ G_1^{-1}$ preserve the length
of the tangent vectors along the geodesics.

It follow from Gauss's lemma that
$(G_i\circ P)_\ast (V_1(t,\eta))$ is orthogonal to 
$(G_i\circ P)_\ast (V_2(t,\eta))$ for $i =1,2$.  Thus
$\displaystyle G_2 \circ G_1^{-1}$ preserves the orthogonality at each
point along the geodesic.  Therefore, to prove that
$\displaystyle G_2 \circ G_1^{-1}$ preserves the
norm of the tangent vectors at each point of $U_1$, it suffices to
prove that $\displaystyle G_2 \circ G_1^{-1}$ preserves the norm of
$(G_1\circ P)_\ast (V_2(t,\eta))$.

As usual, let $(\check{W},\check{U},\check{\phi})$ be a local chart of
$\displaystyle [0,\epsilon[\times S^1$.  
Furthermore, let\\
$h_i(r) = \big\|\big((G_i\circ P)_\ast(V_2) \circ \check{\phi}\big)(r,\theta)
\big\|_{(G_i\circ P)(\check{\phi}(r,\theta))}$
where $(t,\eta) = \check{\phi}(r,\theta)$ and $i =1,2$.
For $\eta$ fixed, we have from Theorem~\ref{defnJacobEqu} that $h_i$
satisfies the Jacobi's equation
\[
\dydxn{h_i}{r}{2}(r) + \frac{1}{4} \kappa(G_i(P(\check{\phi}(r,\theta))))
  h_i(r) = 0
\]
with $h_i(0) = 0$ and $h_i'(0) = 1/2$.  Since
$\kappa_2(G_2(P(\check{\phi}(r,\theta)))) = 
\kappa_1(G_1(P(\check{\phi}(r,\theta))))$ for
$0\leq r \leq \epsilon$, these two differential equation are identical.
By uniqueness of solutions,
\[
\big\|\big((G_1\circ P)_\ast(V_2) \circ \check{\phi}\big)(r,\theta)
\big\|_{(G_1\circ P)(\check{\phi}(r,\theta))} = h_1(r) = h_2(r)
= \big\|\big( (G_2\circ P)_\ast(V_2)(r,\theta) \circ \check{\phi}\big)
\big\|_{(G_2\circ P)(\check{\phi}(r,\theta))}
\]
for $0 \leq r < \epsilon$ and $\displaystyle \eta \in S^1$ arbitrary.
\end{proof}

\begin{cor}
Suppose that $S$ is a connected and oriented $2$-dimensional 
Riemannian manifold.  Moreover, suppose that the curvature $\kappa$ on
$S$ is constant.  Given $(\VEC{u}_i,\VEC{x}_i) \in S_c(\VEC{u}_i)$
for $i=1,2$, there exist open neighbourhood $U_i$ of
$\VEC{u}_i$ for $i=1,2$ and an isometry $g:U_1 \to U_2$ such
that $g_\ast(\VEC{u}_1,\VEC{x}_1) = (\VEC{u}_2,\VEC{x}_2)$.
\end{cor}

\begin{proof}
Let $G_i:D_{\epsilon} \to S$ be a geodesic coordinate system for
$S$ about $\VEC{u}_i$ and set $U_i = G_i(D_{\epsilon})$ for $i=1,2$.

Choose $\displaystyle \nu \in S^1$ acting on $D_{\epsilon}$ such that
$\nu_\ast\big((G_1^{-1})_\ast(\VEC{u}_1,\VEC{x}_1)\big)
= (G_2^{-1})_\ast(\VEC{u}_2,\VEC{x}_2)$ and set $\tilde{G}_2 =  G_2 \circ \nu$.
We have that $\tilde{G}_2:D_\epsilon \to S$ is a geodesic coordinate
system for $S$ about $\VEC{u}_2$ because $\nu$ maps radial geodesics to
radial geodesics in $D_{\epsilon}$.

The map $g$ is defined by
$\displaystyle g = \tilde{G}_2 \circ G_1^{-1}:U_1 \to U_2$.
That $g$ is an isometry comes from the previous theorem with $S_1 =S_2 = S$.
We also have
\[
g_\ast(\VEC{u}_1,\VEC{x}_1) = (G_2 \circ \nu \circ G_1^{-1})_\ast
(\VEC{u}_1,\VEC{x}_1)
= (G_2)_\ast\big( \nu_\ast\big((G_1^{-1})_\ast(\VEC{u}_1,\VEC{x}_1)\big) \big)
= (\VEC{u}_2,\VEC{x}_2) \ .
\]
\pdfbox{riemann_geom/geodfig4}
This figure summarizes the proof.  For $i=1,2$, the angle
$\displaystyle \eta_i \in S^1$ is such that
$\tilde{\sigma}_{i,\eta_i}(0) = Q_i(0,\eta_i) = (\VEC{u}_i,\VEC{x}_i)$
as in the construction of $G_i$ for $i=1,2$.  Moreover,
$\sigma_{i,\eta_i}:[0,\epsilon[\to S$ is a geodesic through $\VEC{u}_i$.
\end{proof}

\section{Euclidean and Non-Euclidean Geometries}

The last section of these lecture notes is an introduction to the
fascinating and vast subject of non-Euclidean geometry.

\subsection{Orientation Preserving Isometries}

Suppose that $S_1$ and $S_2$ are two oriented $2$-dimensional
Riemannian manifolds, and that $f:S_1 \to S_2$ is an orientation
preserving isometry (see Definition~\ref{defnIsomS1S2}).
We list some of the properties of $f$ below.

Let $\tilde{f} = f_\ast\big|_{(S_1)_c}$.  We have that
$\tilde{f}:(S_1)_c \to (S_2)_c$ according to Proposition~\ref{profastMap}.

\stage{1} $\tilde{f}$ preserves the differential $1$-forms $\omega_1$,
$\omega_2$ and $\rho$ given in Definition~\ref{defnO1O2} and
Theorem~\ref{thmConn1form}.

More precisely, let $\displaystyle \omega_1^{[i]}$,
$\displaystyle \omega_2^{[i]}$ and $\displaystyle \rho^{[i]}$ 
be the forms on $S_i$ given in Definition~\ref{defnO1O2} and
Theorem~\ref{thmConn1form} for $i =1,2$.  We have
\begin{align*}
&\tilde{f}^\ast(\omega_1^{[2]})(\VEC{u},\VEC{x})\big( (\VEC{u},\VEC{x}),
(\VEC{p},\VEC{q})\big) 
=\omega_1^{[2]}\big(\tilde{f}(\VEC{u},\VEC{x})\big)
\big( \tilde{f}_\ast\big( (\VEC{u},\VEC{x}), (\VEC{p},\VEC{q}) \big)\big)
\\
&\qquad = \ps{ \big((\pi_{S_2})_\ast \circ
\tilde{f}_\ast\big)\big((\VEC{u},\VEC{x}), (\VEC{p},\VEC{q})\big)}
{\tilde{f}(\VEC{u},\VEC{x})}_{f(\VEC{u})} \\
&\qquad = \ps{ \big(\tilde{f} \circ (\pi_{S_1})_\ast\big)
\big( (\VEC{u},\VEC{x}), (\VEC{p},\VEC{q})\big)}
{ \tilde{f}(\VEC{u},\VEC{x})}_{f(\VEC{u})} \\
&\qquad = \ps{ \tilde{f} \big( (\pi_{S_1})_\ast
\big( (\VEC{u},\VEC{x}), (\VEC{p},\VEC{q})\big)\big)}
{ \tilde{f}(\VEC{u},\VEC{x})}_{f(\VEC{u})}
= \ps{(\pi_{S_1})_\ast\big( (\VEC{u},\VEC{x}), (\VEC{p},\VEC{q})\big)}
{(\VEC{u},\VEC{x})}_{\VEC{u}} \\
&\qquad =
(\omega_1^{[1]})(\VEC{u},\VEC{x})\big((\VEC{u},\VEC{x}),(\VEC{p},\VEC{q})\big) 
\end{align*}
for all $\big((\VEC{u},\VEC{x}),(\VEC{p},\VEC{q})\big) \in
\TS_{(\VEC{u},\VEC{x})} S_c$, where we have used the second item of
Proposition~\ref{proTSprpts} to get the third equality.

Likewise, we have
\[
\tilde{f}^\ast(\omega_2^{[2]})(\VEC{u},\VEC{x})\big( (\VEC{u},\VEC{x}),
(\VEC{p},\VEC{q})\big) 
= (\omega_2^{[1]})(\VEC{u},\VEC{x})
\big((\VEC{u},\VEC{x}),(\VEC{p},\VEC{q})\big) 
\]
for all $\big((\VEC{u},\VEC{x}),(\VEC{p},\VEC{q})\big) \in
\TS_{(\VEC{u},\VEC{x})} S_c$.

The prove that $\tilde{f}$ preserves the differential $1$-form $\rho$
is a little bit more complex.  We first prove that
$\displaystyle \tilde{f}^\ast(\rho^{[2]})$ is the Riemann connection on
$S_1$ if $\displaystyle \rho^{[2]}$ is the Riemann connection on $S_2$.

Using Proposition~\ref{proFPrpts}, we find that
\[
\tilde{f}^\ast(\rho^{[2]})(\VEC{u},\VEC{x})\big(H(\VEC{u},\VEC{x})\big)
= \rho^{[2]}\big(\tilde{f}(\VEC{u},\VEC{x})\big)
\big(\tilde{f}_\ast(H(\VEC{u},\VEC{x}))\big)
= \rho^{[2]}\big(\tilde{f}(\VEC{u},\VEC{x})\big)
\big(H(\tilde{f}(\VEC{u},\VEC{x}))\big) = 1 \ .
\]
Moreover, it follows from the third item of Proposition~\ref{proTSprpts} and
Proposition~\ref{proERequR} that
\[
\eta^\ast( \tilde{f}^\ast(\rho^{[2]}) ) = (\tilde{f} \circ \eta)^\ast(\rho^{[2]})
= (\eta \circ \tilde{f} )^\ast(\rho^{[2]})
= \tilde{f}^\ast(\eta^\ast(\rho^{[2]})) = \tilde{f}^\ast(\rho^{[2]}) \ .
\]
Hence, we get from Proposition~\ref{propCIfInv} that
$\displaystyle \tilde{f}^\ast(\rho^{[2]})$ is a connection $1$-form.

Since $\df{\omega_1^{[2]}} = \rho^{[2]} \wedge \omega_2^{[2]}$
and $\df{\omega_2^{[2]}} = -\rho^{[2]}  \wedge \omega_1^{[2]}$, we get
\[
\df{\omega_1^{[1]}} = \df{ (\tilde{f}^\ast(\omega_1^{[2]}))}
= \tilde{f}^\ast(\df{\omega_1^{[2]}})
= \tilde{f}^\ast(\rho^{[2]} \wedge \omega_2^{[2]})
=  \tilde{f}^\ast(\rho^{[2]}) \wedge \tilde{f}^\ast(\omega_2^{[2]})
=  \tilde{f}^\ast(\rho^{[2]}) \wedge \omega_2^{[1]}
\]
and similarly
\[
\df{\omega_2^{[1]}} = \df{ (\tilde{f}^\ast(\omega_2^{[2]}))}
= \tilde{f}^\ast(\df{\omega_2^{[2]}})
= -\tilde{f}^\ast(\rho^{[2]} \wedge \omega_1^{[2]})
=  -\tilde{f}^\ast(\rho^{[2]}) \wedge \tilde{f}^\ast(\omega_1^{[2]})
=  -\tilde{f}^\ast(\rho^{[2]}) \wedge \omega_1^{[1]} \ .
\]
Thus, we get that $\displaystyle \rho^{[1]} = \tilde{f}^\ast(\rho^{[2]})$ by
uniqueness of the Riemann connection as stated in
Theorem~\ref{thmRconnect}.

\stage{2} $f$ preserve the curvature.  More precisely,
$\kappa_1 = \kappa_2 \circ f$ where $\kappa_i$ is the curvature
function on $S_i$ associated to the connection $1$-form
$\displaystyle \rho^{[i]}$ for $i=1,2$.

We have that $\displaystyle \df{\rho^{[1]}}
= - (\kappa_1 \circ \pi_{S_1}) \omega_1^{[1]} \wedge \omega_2^{[1]}$
and $\displaystyle \df{\rho^{[2]}}
= - (\kappa_2 \circ \pi_{S_2}) \omega_1^{[2]} \wedge \omega_2^{[2]}$.
Hence, using the first item of Proposition~\ref{proTSprpts}, we get
\begin{align*}
- (\kappa_1 \circ \pi_{S_1}) \omega_1^{[1]} \wedge \omega_2^{[1]}
&= \df{\rho^{[1]}} = \df{(\tilde{f}^\ast(\rho^{[2]}))}
= \tilde{f}^\ast( \df{\rho^{[2]}})
= -\tilde{f}^\ast\big(
(\kappa_2 \circ \pi_{S_2}) \omega_1^{[2]} \wedge \omega_2^{[2]} \big) \\
&= - (\kappa_2 \circ \pi_{S_2} \circ \tilde{f})
\, \tilde{f}^\ast (\omega_1^{[2]}) \wedge \tilde{f}^\ast(\omega_2^{[2]})
= - (\kappa_2 \circ f \circ \pi_{S_1})
\, \omega_1^{[1]} \wedge \omega_2^{[1]} \ .
\end{align*}
Thus $\kappa_1 \circ \pi_{S_1} = \kappa_2 \circ \tilde{f} \circ \pi_{S_1}$.
Since $\pi_{S_1}$ is onto, we get that $\kappa_1 = \kappa_2 \circ f$.

\stage{3} $f$ preserves horizontal lifts.  More precisely, if
$\sigma:[a,b] \to S_1$ and $\tilde{\sigma}:[a,b] \to (S_1)_c$ is the
horizontal lift of $\sigma$ through $(\sigma(a), \VEC{x}) \in (S_1)_c$,
then $\tilde{\alpha} = \tilde{f} \circ \tilde{\sigma}:[a,b] \to (S_2)_c$
is the horizontal lift of $\alpha = f \circ \sigma:[a,b] \to S_2$ through
$\tilde{f}(\sigma(a),\VEC{x})$.  This is a consequence of
Proposition~\ref{propHLexists} because
\begin{enumerate}
\item $\tilde{\alpha}(a) = \tilde{f}(\tilde{\sigma}(a)) =
\tilde{f}(\sigma(a),\VEC{x})$,
\item $\pi_{S_2}\circ \tilde{\alpha} =
\pi_{S_2}\circ \tilde{f} \circ \tilde{\sigma} =
f \circ \pi_{S_1} \circ \tilde{\sigma} = f \circ \sigma = \alpha$
where the first item of Proposition~\ref{proTSprpts} was used to get the second
equality, and
\item
\begin{align*}
\rho^{[2]}(\tilde{\alpha}(t))(\tilde{\alpha}_\ast(t,1))
&= \rho^{[2]}\big((\tilde{f} \circ \tilde{\sigma})(t)\big)
\big(\tilde{f} \circ \tilde{\sigma})_\ast(t,1)\big)
= \rho^{[2]}\big(\tilde{f} (\tilde{\sigma}(t))\big)
\big((\tilde{f}_\ast(\tilde{\sigma}_\ast(t,1))\big) \\
&= \tilde{f}^\ast(\rho^{[2]})(\tilde{\sigma}(t))(\tilde{\sigma}_\ast(t,1))
= \rho^{[1]}(\tilde{\sigma}(t))(\tilde{\sigma}_\ast(t,1)) = 0
\end{align*}
where the last equality comes from the fact that $\tilde{\sigma}$ is the
horizontal lift of $\sigma$ through $(\sigma(a), \VEC{x})$.
\end{enumerate}

\stage{4} The function $f$ maps geodesic into geodesic.  More precisely, if
$\sigma:[a,b] \to S_1$ is a smooth curve and $\sigmaU:[a,b] \to (S_1)_c$
is the geodesic associated to $\sigma:[a,b] \to S_1$, then by definition
$\sigmaU(t) = (\sigma(t), \sigma'(t))$ is the horizontal lift
$\tilde{\sigma}$ of $\sigma$ with
$\tilde{\sigma}(a) = (\sigma(a), \sigma'(a))$.
According to (3) above, $\tilde{f} \circ \sigmaU$ is the horizontal
lift of $f \circ \sigma$ with $(\tilde{f} \circ \sigmaU)(a)
= \tilde{f}(\sigma(a),\sigma'(a))$.  But we also have that
$(\tilde{f} \circ \sigmaU)(t) = \tilde{f}(\sigma(t), \sigma'(t))
= f_\ast(\sigma(t), \sigma'(t)) = \big((f\circ \sigma)(t),
(f\circ \sigma)'(t)\big)$ for $a \leq t \leq b$.  Thus 
$\tilde{f} \circ \sigmaU:[a,b] \to (S_2)_c$
is the geodesic associated to $f \circ \sigma:[a,b] \to S_2$.

\subsection{Group of Orientation Preserving Isometries}

\begin{prop}
Let $S$ be an oriented Riemannian manifold.  The set $\E_S$ of all
orientation preserving ismetries from $S$ to itself forms a group with
respect to the composition of functions.  It is called the
{\bfseries group of orientation preserving isometries}\index{Group of
Orientation Preserving Isometries} of $S$.
\end{prop}

\begin{proof}
This proposition is a consequence of the fact that the composition of
orientation preserving isometries is still an orientation preserving
isometry.  Note that the identity map is the unit in $\E_S$.  
\end{proof}

\begin{defn}
We say that a subgroup $\displaystyle \E_S^o$ of the group $\E_S$ is
{\bfseries transitive}\index{Transitive} on $S$ if, for
any $\VEC{u}_1,\VEC{u}_2 \in S$, there exists $\displaystyle f \in \E_S^o$
such that $f(\VEC{u}_1) = \VEC{u}_2$.
\end{defn}

\begin{prop} \label{propTransKconst}
Let $S$ be an oriented Riemannian manifold.  If the group $\E_S$ is
transitive on $S$, then $S$ has a constant curvature.
\end{prop}

\begin{proof}
Choose $\VEC{u} \in S$.  Given any $\VEC{v} \in S$, there exists
$f \in \E_S$ such that $\VEC{v} = f(\VEC{u})$.  Since $f$ is an
orientation preserving isometry, it preserves curvature according to
the item (2) of the previous subsection.  Hence, we have that
$\kappa(\VEC{u}) = (\kappa \circ f)(\VEC{u}) = \kappa(f(\VEC{u}))
= \kappa(\VEC{v})$.  Since $\VEC{v} \in S$ is arbitrary, we get that
$\kappa$ is constant.
\end{proof}

\begin{prop}
Let $S$ be an oriented $2$-dimensional Riemannian manifold.
Given $\VEC{u} \in S$, let
$\E_S(\VEC{u}) = \{ f \in \E_S: f(\VEC{u}) = \VEC{u} \}$.  Then
$\E_S(\VEC{u})$ is a subgroup of $\E_S$.  It is called the
{\bfseries isotropy group}\index{Isotropy Group} of $S$ at $\VEC{u}$.
\end{prop}

\begin{proof}
If $f_1 , f_2 \in \E_S(\VEC{u})$, then $f_1 \circ f_2 \in \E_S(\VEC{u})$
because $f_1 \circ f_2 \in \E_S$ since $\E_S$ is a group and
$(f_1 \circ f_2)(\VEC{u}) = f_1(f_2(\VEC{u})) = f_1(\VEC{u}) = \VEC{u}$.
We obviously have that $\Id \in \E_S(\VEC{u})$.
\end{proof}

Given $\VEC{u} \in S$ fixed, we have from Proposition~\ref{propfisoRot} that
$f_\ast:\TS_{\VEC{u}} S \to \TS_{\VEC{u}} S$ is a rotation
for all $f \in \E_S$ such that $f(\VEC{u})= \VEC{u}$.
Hence, we may define a map $\displaystyle \Phi:\E_S(\VEC{u}) \to S^1$
that assign for each $f \in \E_S(\VEC{u})$ the angle of rotation
of $f_\ast:\TS_{\VEC{u}} S \to \TS_{\VEC{u}} S$.

\begin{prop} \label{propPhi1to1}
Let $S$ be a connected and oriented $2$-dimensional Riemannian
manifold.  Given $\VEC{u} \in S$, the map
$\displaystyle \Phi:\E_S(\VEC{u}) \to S^1$ is one-to-one.
\end{prop}

\begin{proof}
Suppose that $f \in \KE(\Phi)$.

Let $G: D_{\epsilon_{\VEC{u}}} \to S$ be a geodesic coordinate system of $S$
about $\VEC{u}$.  We assume that
$\displaystyle D_{\epsilon_{\VEC{u}}}
\cong B_{\epsilon_{\VEC{u}}}(\VEC{0}) \subset \RR^2$.
We have that $f$ maps $G(D_{\epsilon_{\VEC{u}}})$ into itself
for all $f \in \E_S(\VEC{u})$ because $f(\VEC{u}) = \VEC{u}$ and $f$
is an orientation preserving isometry.  To be more explicit, it
follows from item (4) in the previous subsection that the length
along the geodesic from $\VEC{u}$ to $f(\VEC{v})$ is equal to length
along the geodesic from $\VEC{u}$ to $\VEC{v}$ for all
$\VEC{v} \in G(D_\epsilon)$.  Therefore, the length
along the geodesic from $\VEC{u}$ to $f(\VEC{v})$ is less than
$\epsilon_{\VEC{u}}$.  We then get that $f(\VEC{v}) \in D_{\epsilon_{\VEC{u}}}$
if $\VEC{v} \in D_{\epsilon_{\VEC{u}}}$.  Recall that the geodesics are
parameterized by arc length.

Since $\displaystyle g_f = G^{-1} \circ f \circ
G:D_{\epsilon_{\VEC{u}}} \to D_{\epsilon_{\VEC{u}}}$
is an orientation preserving isometry leaving the origin fix, we get as in
Proposition~\ref{propfisoRot} that $g_f$ is a rotation.
Thus $(g_f)_\ast(\VEC{0},\VEC{y}) = (\VEC{0},g_f(\VEC{y}))$.
Since $\displaystyle f_\ast = (G \circ g_f \circ G^{-1})_\ast
= G_\ast \circ (g_f)_\ast \circ G^{-1}_\ast$, we get
\begin{align*}
f \in \KE(\Phi) \iff f_\ast &= \Id_{\TS_{\VEC{u}} S}
\iff (g_f)_\ast = \Id_{\TS_{\VEC{0}} D_\epsilon}
\iff g_f = \Id_{D_{\epsilon_{\VEC{u}}}} \\
&\iff f\big|_{G(D_{\epsilon_{\VEC{u}}})} = G \circ g_f \circ G^{-1}
= \Id_{G(D_{\epsilon_{\VEC{u}}})} \ .
\end{align*}
Therefore, we have
\[
G(D_{\epsilon_{\VEC{u}}}) \subset
V \equiv \{ \VEC{v} \in S : f(\VEC{v}) = \VEC{v} \ \text{and}
\ \Id_{\TS_{\VEC{v}} S} = f_\ast:\TS_{\VEC{v}} S \to \TS_{\VEC{v}} S \}
\ .
\]
Note that $f(\VEC{v}) = \VEC{v}$ for all $\VEC{v} \in G(D_{\epsilon_\VEC{u}})$
implies that
$\Id_{\TS_{\VEC{v}} S} = f_\ast:\TS_{\VEC{v}} S \to \TS_{\VEC{v}} S$
for all $\VEC{v} \in G(D_{\epsilon_\VEC{u}})$ because
$f_\ast$ is defined locally.  Since the previous reasoning can be
repeated with $\VEC{u}$ replaced by any $\VEC{v} \in V$, we find that
$V \subset S$ is an open set because the sets of the form
$G(D_{\epsilon_{\VEC{v}}})$ are open sets according to
Definition~\ref{defnGCsyst}.

Since $f:S \to S$ is smooth, we have that $V$ is also closed
\footnote{This follows from the fact that $f$ and $f_\ast$ are
continuous.  If $(W,U,\phi)$ is a local chart about $\VEC{u}$ and
$\VEC{u} = \phi(\VEC{w})$, then
$\displaystyle (\phi^{-1} \circ f \circ \phi)(\VEC{w})
= \lim_{i\to \infty} (\phi^{-1} \circ f \circ \phi)(\VEC{w_i})
= \lim_{i\to \infty} \VEC{w}_i = \VEC{w}$ for
$\{\VEC{w}_i\}_{i\in\NN} \subset \phi^{-1}(V \cap U)$ converging to $\VEC{w}$.
Likewise,
$\displaystyle \diff (\phi^{-1} \circ f \circ \phi)(\VEC{w}) \VEC{y}
= \lim_{i\to \infty} \diff (\phi^{-1} \circ f \circ \phi)(\VEC{w_i}) \VEC{y}
= \lim_{i\to \infty} \Id_{\RR^k} \VEC{y} = \VEC{y}$ for all
$\displaystyle \VEC{y}\in \RR^k$.}
This implies that $V = S$ because $V$ is a non-empty open and closed
subset of the connected set $S$.  Therefore $f = \Id_S$.
\end{proof}

\begin{defn}
Let $S$ be an oriented $2$-dimensional Riemannian
manifold.  Given $\VEC{u},\VEC{v} \in S$, the {\bfseries orbit}\index{Orbit}
of $\E_S(\VEC{u})$ through $\VEC{v}$ is the set
$\OO_{\E_S(\VEC{u})}(\VEC{v}) = \{ f(\VEC{v}) : f \in \E_S(\VEC{u}) \}$.
\end{defn}

\begin{theorem} \label{thmGeoPCirc}
Let $S$ be an oriented $2$-dimensional Riemannian manifold.
Suppose that the map $\displaystyle \Phi:\E_S(\VEC{u}) \to S^1$ is
onto for some $\VEC{u} \in S$.  If $G:D_{\epsilon_{\VEC{u}}} \to S$ is
a geodesic coordinate system of $S$ about $\VEC{u}$, then the geodesic through
$\VEC{u}$ are the orthogonal curves (parameterized by arc length) to
the orbits of $\E_S(\VEC{u})$ through $\VEC{v} \in G(D_{\epsilon_{\VEC{u}}})$
(Figure~\ref{GeodFig6}).
\end{theorem}

\begin{proof}
We assume that
$\displaystyle D_{\epsilon_{\VEC{u}}} \cong B_{\epsilon_{\VEC{u}}}(\VEC{0})
\subset \RR^2$.

Given $\VEC{v} \in G(D_\epsilon)$, let $\VEC{w} = G(\VEC{v})$.  As we
have seen in the proof of Proposition~\ref{propPhi1to1},
$\displaystyle G^{-1} \circ f \circ G = g_f$ is a rotation in
$D_{\epsilon_{\VEC{u}}}$.  Since
$\displaystyle \Phi:\E_S(\VEC{u}) \to S^1$ is onto, we have that
$\displaystyle C_{\VEC{v}} = \{ G^{-1} f(\VEC{v}) = g_f(\VEC{w}) :
f \in \E_S(\VEC{u}) \}$ is a
full circle in $D_{\epsilon_{\VEC{u}}}$.

Since the circles in
$D_{\epsilon_{\VEC{u}}}$ (i.e.\ $C_{\VEC{v}}$) are orthogonal to the
radial geodesics in $D_{\epsilon_{\VEC{u}}}$, we get that
the orbits
$\OO_{\E_S(\VEC{u})}(\VEC{v}) = \{ f(\VEC{v}) : f \in \E_S(\VEC{u}) \}
= G(C_{\VEC{v}})$ are orthogonal to the geodesic through
$\VEC{u}$ in $G(D_{\epsilon_{\VEC{u}}})$
according to Gauss' lemma because $P_\ast(V_1(t,\eta))$ and
$P_\ast(V_2(t,\eta))$ in the statement of the lemma refer to the
radial and rotational directions in $D_{\epsilon_{\VEC{u}}}$ respectively.
\end{proof}

\pdfF{riemann_geom/geodfig6}{Representation of orbits of
$\E_S(\VEC{u})$ through $\VEC{v}$}{Representation of the
of orbits of $\E_S(\VEC{u})$ through $\VEC{v} \in S$.}{GeodFig6}

The next theorem will be useful to determine the full group of
orientation preserving isometry of a manifold.

\begin{prop} \label{propIsoGr}
Let $S$ be an oriented $2$-dimensional Riemannian manifold.
Suppose that $\displaystyle \E_S^o$ is a subgroup of the group $\E_S$
of orientation preserving isometry of $S$.  If $\displaystyle \E_S^o$
is transitive on $S$ and
$\displaystyle \Phi:\E_S^o \cap \E_S(\VEC{u}) \to S^1$ is onto for
some $\VEC{u} \in S$, then $\displaystyle \E_S^o = \E_S$.
\end{prop}

\begin{proof}
Given $f_1 \in \E_S$, we may use the transitivity of
$\displaystyle \E_S^o$ to find $\displaystyle f_2 \in \E_S^o$ such
that $f_2(f_1(\VEC{u})) = \VEC{u}$.  Hence
$f_2 \circ f_1 \in \E_S(\VEC{u})$.  Since $\Phi$ maps
$\displaystyle \E_S^o \cap \E_S(\VEC{u})$ onto $\displaystyle S^1$,
there exists $\displaystyle f_3 \in \E_S^o \cap \E_S(\VEC{u})$ such
that $\Phi(f_3) = \Phi( f_2\circ f_1)$.
Since $\displaystyle \Phi: \E_S(\VEC{u}) \to S^1$ is one-to-one, we
get that $f_3 = f_2 \circ f_1$.  Thus
$\displaystyle f_1 = f_2^{-1} \circ f_3 \in \E_S^o$.
We have proved that $\displaystyle \E_S \subset \E_S^o$.  But
$\displaystyle \E_S^o$ is a subgroup of $\E_S$.  Therefore
$\displaystyle \E_S^o = \E_S$.
\end{proof}

\subsection{Manifolds of Constant Curvatures}

Oriented Riemannian manifolds with constant curvature can be classified into
three categories according to the type of geometry that they have.
Let $S$ be an oriented Riemannian manifolds with constant curvature
$\kappa$.
\begin{enumerate}
\item If $\kappa = 0$, then the geometry on $S$ is the Euclidean geometry.
\item If $\kappa > 0$, then the geometry on $S$ is the elliptic geometry.
\item If $\kappa < 0$, then the geometry on $S$ is the hyperbolic geometry.
\end{enumerate}

The Euclidean geometry is well known.  Its main feature is that, given
a line $\ell$ and a point $\VEC{p}$ not on $\ell$, there exists one
and only one line through $\VEC{p}$ parallel to $\ell$.  This is
equivalent to the property that the sum of the interior angles of a
triangle is equal to $\pi$. 

In elliptic geometry, the main feature is that, given a line $\ell$ and a
point $\VEC{p}$ not on $\ell$, there does not exist any line through $\VEC{p}$
parallel to $\ell$.  This is equivalent to the property that
the sum of the interior angles of a triangle is greater than $\pi$.

Finally, in hyperbolic geometry, the main feature is that, given a
line $\ell$ and a point $\VEC{p}$ not on $\ell$, there exist
infinitely many lines through $\VEC{p}$ parallel to $\ell$.  This is
equivalent to the property that the sum of the interior angles of a
triangle is less than $\pi$.

A excellent elementary introduction to the subject of Euclidean and
non-Euclidean geometries is \cite{G}.

We are now going to illustrate with some example these different
geometries on oriented $2$-dimensional Riemannian manifolds with
constant curvature.

\begin{egg}[Euclidean Geometry]
We consider $\displaystyle S = \RR^2$ with the standard Euclidean
norm and orientation.  This is an oriented $2$-dimensional
Riemannian manifold.

We have already seen in Example~\ref{eggEuclk0} that $S$ has null
curvature.

The group $G$ formed of all rotations and translation on
$\displaystyle S = \RR^2$.  It is clearly a subgroup of the group
$\E_S$.  We may wonder if there are other orientation preserving
isometries on $S$.  Proposition~\ref{propIsoGr} provided the answer to
this question.  It is clear that $G$ is transitive because it
contains all the translations.  We also have that
$\displaystyle \Phi: G \cap \E_S(\VEC{0}) \to S^1$ is onto since
$G$ contains all the rotation about the origin.  It then follows
from Proposition~\ref{propIsoGr} that $\E_S = G$ \footnote{It is also
true that the group of all isometries (no orientation preserving
required) on $\RR^2$ is equal to the Euclidean group $E(2)$ formed of
all rotations, reflections through plans containing the origin and
translation on $\displaystyle S = \RR^2$.}.

It follows from Theorem~\ref{thmGeoPCirc} that the geodesic through a
point $\VEC{u} \in S$ are traditional straight lines (suitably
parameterized) because the orbits of $\E_S(\VEC{u})$ are circles
centred at $\VEC{u}$.  This should not be surprising because it is
well know that in Euclidean geometry, the shortest path from one point
to another is the straight line.
\end{egg}

The next example is a little surprising but not extremely surprising
if we think about it.

\begin{egg}[Elliptic Geometry]
We consider $\displaystyle S = S^2$ with the Riemannian metric induced
as a submanifold of $\displaystyle \RR^3$.  The orientation on
$\displaystyle S^2$ is the orientation induced by the standard
outward unit normal to $\displaystyle S^2$ at each point of
$\displaystyle S^2$ as defined in Definition~\ref{manifNormal}.

The group $G$ of rotations in $\displaystyle \RR^3$ centred at the origin
is a transitive group of orientation preserving isometries on
$\displaystyle S^2$.  It therefore follows from
Proposition~\ref{propIsoGr} that $\E_S = G$.  Moreover, we get from
Proposition~\ref{propTransKconst} that the curvature $\kappa$ of
$\displaystyle S^2$ is constant.  It follows from
Example~\ref{eggGBsphere} that
$\displaystyle 4\pi = \int_{S^2} \kappa \dx{V} = \kappa \int_{S^2} \dx{V} 
= 4\pi \kappa$.  Thus $\kappa = 1 > 0$.

Given $\displaystyle \VEC{u} \in S^2$, the isotropy group of
$\displaystyle S^2$ at $\VEC{u}$ is the group of all rotations in
$\displaystyle \RR^3$ whose axis of rotation is the line through
$\VEC{u}$ and $\VEC{0}$.  Thus the orbits of $\E_S(\VEC{u})$ are
the parallels and the geodesic through $\VEC{u}$ are the great circle
through $\VEC{u}$.  In elliptic geometry, the ``straight lines'' are the
geodesic (i.e. the great circles) since they are associated to the
path of shortest distance between two points on $\displaystyle S^2$.
Hence, give a great circle $\ell$ and a point $\VEC{p}$ not on $\ell$,
there does not exist any great circle through $\VEC{p}$ that does not
intersect $\ell$.  Note also that the sum of the internal angles of a
triangle is greater than $\pi$ as illustrated in the figure below.
\pdfbox{riemann_geom/geodfig7}
\end{egg}

\begin{egg}[Hyperbolic Geometry]
It is much harder to provide a simple example of a oriented Riemannian
manifold of constant negative curvature.  The pseudosphere is probably
the simplest example in $\displaystyle \RR^3$.  Nevertheless, the
example that is almost always presented in the literature is the
{\bfseries Poincaré Model}\index{Poincaré Model}.  It is the imbedding
in $\displaystyle \RR^2$ of an oriented Riemannian manifold $S$ of
constant negative curvature.  This imbedding is not an isometry.  It
has been proved that $S$ can only be realized in $\displaystyle \RR^n$
with $n>3$.

The following figure illustrate how the Poincaré model is built.
\pdfbox{riemann_geom/geodfig8}
The ``straight lines'' in this model are the lines through the origin
and the arcs of circles that intersect perpendicularly the boundary
$\partial D$ of
$\displaystyle D = \{\VEC{x} \in \RR^2 : \|\VEC{x}\| < r \}$ for some $r>0$.
They are the stereographic projections limited to the lower half of
$\displaystyle S^2$ of all the circles \footnote{A circle on
$\displaystyle S^2$ is defined by the intersection of a sphere with
$\displaystyle S^2$.} that intersect perpendicularly
the equator of $\displaystyle S^2$.

The {\bfseries Poincaré metric}\index{Poincaré Metric} on
$\TS_{\VEC{u}} D$ for $\VEC{u} \in D$ is defined by
$\displaystyle \ps{(\VEC{u},\VEC{x}_1)}{(\VEC{u},\VEC{x}_2)}_{\VEC{u}}
= \frac{\ps{\VEC{x}_1}{\VEC{x}_2}}{(r^2 - \|\VEC{u}\|^2)^2}$
for all $(\VEC{u},\VEC{x}_i) \in \TS_{\VEC{u}} D$ for $i =1,2$,
where $\ps{}{}$ and $\|\cdot\|$ denote the standard inner product
and Euclidean norm on $\displaystyle \RR^2$.

If $\sigma:[0,1[\to D$ is the straight line emanating from the origin
given by $\sigma(t) = t (\cos{\theta},\sin{\theta})$ for
$0 \leq t < r$ and $\theta$ fixed, then the length of $\sigma$ is
\begin{align*}
L &= \int_0^r \|(\sigma(t),\sigma'(t))\|_{\VEC{u}} \dx{t}
= \int_0^r \big\|\big((t(\cos(\theta),\sin(\theta)),
(\cos(\theta),\sin(\theta))\big\|_{\VEC{u}} \dx{t} \\
&= \int_0^r \frac{1}{r^2-t^2} \dx{t} = \infty \ .
\end{align*}

To determine which lines are geodesics, we find the group $\E_D$ of
orientation preserving isometry on $D$.  The easiest way to find
$\E_D$ is to tread $D$ as a disk in the complex plane centred at the
origin.  In a first course on complex variables (see \cite{CBV}),
students learned about linear fractional transformations or Möbius
transformations.  These transformations are the composition of
rotations, translations, expansions or contractions, and inversions
with respect to the unit circle.  A special linear fractional
transformation is of the form
\[
f_{w,\theta}(z) = e^{i\theta}\frac{r^2(z - w)}{r^2-z\overline{w}}
\]
for $w \in \CC$ and $\theta \in \RR$.  Since they are the composition
of transformations that preserve angles, the linear fractional
transformations are isomorphism that preserve angles.

We have that $f_{w,\theta}$ maps $D$ onto $D$ for $w \in D$.  Note that
\begin{equation} \label{LFTEq1}
|f_{w,\theta}(z)| = \Big|\frac{r^2(z -w)}{r^2-z \overline{w}}\Big|
= \Big|\frac{z(r^2 -\overline{z} w)}{r^2-z \overline{w}}\Big|
= |z| \bigg|\frac{\overline{r^2 -z \overline{w}}}{r^2-z \overline{w}}\bigg|
= |z| = r
\end{equation}
for $|z|=r$.  Thus $f_{w,\theta}:\partial D \to \partial D$.
Moreover, this map is onto $\partial D$ because
$\displaystyle f^{-1}_{w,\theta} = f_{-w,\theta}$ exists.  Therefore
$f_{w,\theta}(\partial D)$ splits $\displaystyle \RR^2$ into two components:
$D$ and $\displaystyle \RR^2 \setminus \overline{D}$.  Since
$\displaystyle f_{w,\theta}(0) = - e^{i\theta} w \in D$, we get that
$f_{w,\theta}(D) \subset D$ because $f_{w,\theta}$ is a continuous
function and therefore maps components to components.  Again the fact that
$\displaystyle f_{w,\theta}^{-1}$ exists implies that $f_{w,\theta}(D) = D$.

We claim that $f_{w,\theta}$ is an isometry according to the Poincaré
metric.  To verify that this is true, we use the isomorphism between
$\CC$ and $\displaystyle \RR^2$ and first note that
\[
f_{w,\theta}'(z) = e^{-i\theta}\frac{r^2(r^2-|w|^2)}{(r^2-z\overline{w})^2}
\]
and
\begin{align*}
r^2 - |f_{w,\theta}(z)|^2
&= r^2 - \frac{r^4 (z-w)(\overline{z}-\overline{w})}{(r^2 -z \overline{w})
(r^2 - \overline{z} w)}
= \frac{r^2(r^2 -z \overline{w})(r^2 - \overline{z} w)
- r^4 (z-w)(\overline{z}-\overline{w})}{|r^2-z \overline{w}|^2} \\
% &= \frac{r^6 -r^4 (z \overline{w} + \overline{z} w) + r^2|z|^2 |w|^2
% - r^4|z|^2 + r^4(w \overline{z}+\overline{w} z) -r^4|w|^2)}
% {|r^2-z \overline{w}|^2} \\
&= \frac{r^4(r^2 - |z|^2) - r^2 |w|^2 (r^2-|z|^2)}{|r^2-z \overline{w}|^2}
= \frac{r^2(r^2 - |z|^2)(r^2 -|w|^2)}{|r^2-z \overline{w}|^2} \ .
\end{align*}
Hence
\[
\frac{|f_{w,\theta}'(z)|^2}{r^2 - |f_{w,\theta}(z)|^2}
= \frac{1}{r^2 - |z|^2} \ .
\]
We then get
\begin{align*}
\ps{(f_{w,\theta})_\ast(u,z_1)}{(f_{w,\theta})_\ast(u,z_2)}_{f_{w,\theta}(u)}
&= \ps{(f_{w,\theta}(u),f_{w,\theta}'(u)z_1)}
{(f_{w,\theta}(u),f_{w,\theta}'(u) z_2)}_{f_{w,\theta}(u)} \\
&= \frac{ |f_{w,\theta}'(u)|^2 \overline{z_1} z_2}{(r^2 - |f_{w,\theta}(u)|^2)^2}
= \frac{\overline{z_1} z_2}{(r^2 - |u|^2)^2}
= \ps{(u,z_1)}{(u,z_2)}_{u}
\end{align*}
for all $(u,z_i) \in \TS_u D$ and $i =1,2$.

Let $\displaystyle \E_D^o
= \{ f_{w,\theta} : w \in D \ \text{and} \ \theta \in [0,2\pi[ \}$.
We use Proposition~\ref{propIsoGr} to prove that $\displaystyle \E_D = \E_D^o$.
The group $\displaystyle \E_0^o$ is transitive because
$f_{w,0}$ maps $w\in D$ to $0$.  Thus
$\displaystyle f_{-w_2,0} \circ f_{w_1,0} = f^{-1}_{w_2,0} \circ f_{w_1,0}$ maps
$w_1 \in D$ to $w_2 \in D$.  Moreover
$\displaystyle \E_D^o \cap \E_D(0) = \{ f_{0,\theta} : \theta \in [0, 2\pi[ \}$.
Hence $\displaystyle \Phi:\E_D^o \cap \E_D(0) \to S^1$ is onto.
Therefore, we may effectively use Proposition~\ref{propIsoGr} to conclude
that $\displaystyle \E_D = \E_D^o$.

We can now confirm that the ``straight lines'' defined at the
beginning of this example are effectively the geodesics.
To be precise, the orbits of $\E_D(0)$ are the circles centred at the
origin.  Therefore, according to Theorem~\ref{thmGeoPCirc}, the
geodesic through the origin are the tradition straight lines (suitably
parameterized) through the origin.  We now use the fact that
isometries transform geodesics into geodesics to prove that all the
arcs of circles that intersect perpendicularly $\partial D$ are also
geodesics.  To be precise, we prove that all the arcs of circles
that intersect perpendicularly $\partial D$ are the images of
diameters of $D$.  We recall from basic complex analysis that linear
fractional transformations preserve angles and maps circles to circles
(assuming that straight lines are circles of infinite radii).  Suppose
that $\ell$ is a diameter of $D$.  Then $\Game = f_{w,\theta}(\ell)$ is an arc
of a circle in $D$ that may have an infinite radius.  Since
$f_{w,\theta}:\partial D \to \partial D$, the endpoints of $\Game$ are on
$\partial D$.  Since $\ell$ intersects $\partial D$ perpendicularly, we
must also have that $\Game$ intersect $\partial D$ perpendicularly.  Since
$f_{w,\theta}(z) = 0$ implies that $z=w$, we have that $\Game$ does not
contain the origin if $w$ is not on $\ell$.  Therefore, $\Game$ cannot be a
diameter and must be the arc of a circle that intersect $\partial D$
perpendicularly.  Since every arc of circle $\Game$ that intersects
perpendicularly $\partial D$ is determined by three points, the two
points of intersection with $\partial D$ and any other point on the
arc, it is a property of the special linear fractional transformations
that it is possible to determine $w$ and $\theta$ such that
$f_{w,\theta}$ maps a diagonal $\ell$ to $\Game$.  More precisely, suppose
$q_1$ and $q_3$ are the two endpoints of $\Game$ with $q_2$ another point
on $\Game$.  We have to determine $\theta$, $w$,
the endpoints $p_1$ and $p_3=-p_1$ of the diagonal $\ell$, and another
point $p_2$ on $\ell$ such that $f_{w,\theta}(p_i) = q_i$ for $1\leq i \leq 3$.
As illustrated in the following figure, $p_2 \in \ell$ and $\theta$
should satisfy $\displaystyle e^{i\theta}p_2 = q_2$ to get
$f_{w,\theta}(p_2) = q_2$ because
$|f_{w,\theta}(p_2)| = |p_2| = |q_2|$ according to (\ref{LFTEq1}).
\pdfbox{riemann_geom/geodfig10}
We get from $f_{w,\theta}(p_1) = q_1$ and $f_{w,\theta}(p_3) = q_3$
that
\[
e^{i\theta} \frac{r^2(p_1 - w)}{r^2 - p_1 \overline{w}} = q_1
\quad \text{and} \quad
e^{i\theta} \frac{r^2(-p_1 - w)}{r^2 + p_1 \overline{w}} = q_3 \ .
\]
Hence
\[
r^2p_1 + e^{-i\theta} q_1 p_1 \overline{w}  - r^2w = r^2 e^{-i\theta} q_1 
\quad \text{and} \quad
r^2p_1 + e^{-i\theta} q_3 p_1 \overline{w}  + r^2w = - r^2 e^{-i\theta} q_3  \ .
\]
Since $q_1$ and $q_3$ are on $\partial D$, we have that
$\displaystyle q_3 = e^{i\nu} q_1$ for some $\nu \in \RR$.  Thus, the
previous equations become
\begin{equation} \label{LFTEq2}
r^2p_1 + e^{-i\theta} q_1 p_1 \overline{w}  - r^2w = r^2 e^{-i\theta} q_1
\end{equation}
and
\begin{equation} \label{LFTEq3} 
r^2p_1 + e^{-i(\theta-\nu)} q_1 p_1 \overline{w}  + r^2w
= - r^2 e^{-i(\theta-\nu)} q_1  \ .
\end{equation}
If we subtract $\displaystyle e^{i\nu}$ times (\ref{LFTEq2}) from
(\ref{LFTEq3}), then we get
\[
r^2(1 - e^{i\nu}) p_1 + r^2(1+e^{i\nu}) w = -2 r^2 e^{i(\theta-\nu)}q_1 \ .
\]
We can always solve for $p_1$ or $w$ because at least one of
$\displaystyle 1 - e^{i\nu}$ or $1 + e^{i\nu}$ is non-null.  We can
substitute $p_1$ or $w$ into (\ref{LFTEq2}) to get a system of linear
equations (the real and imaginary parts) for $w$ or $p_1$
respectively.  Once we have $w$ and $p_1$ in terms of $\theta$, we can
use the equation
$\displaystyle q_2 = e^{i\theta}p_2 = \pm e^{i\theta} |q_2| p_1$ to determine
$\theta$ and thus $p_2$.

We cannot complete this example without illustrating two of the major
differences between the hyperbolic geometry and the other geometries.
The figure on the left below shows a line $\ell$ and a point $p$ not on
$\ell$ with infinitely many tangent lines through $p$ parallel to $\ell$.
The figure on the right show a triangle where the sum of the
internal angles is less than $\pi$.
\pdfbox{riemann_geom/geodfig9}
\end{egg}

The three examples of geometries that we have given are the three
possible models of geometry on simply connected, complete and oriented
$2$-dimensional Riemannian manifolds with constant curvature
because all such manifolds are isomorphic to one of the three models
represented by our three examples (see \cite{LJM}).

%%% Local Variables:
%%% mode: latex
%%% TeX-master: "notes"
%%% End:
