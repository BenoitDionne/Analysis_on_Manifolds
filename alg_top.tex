\chapter{Algebraic Topology}

It would be a mistake to talk about cohomology theory without
introducing homology theory.  We have seen that it is quite challenging
to compute the cohomology module of even some simple manifolds.
Homology theory can greatly simplify the task of computing cohomology
modules of some manifolds.  In this chapter, we explain how the
``homology module'' of a manifold $S$ can give us the
cohomology module of $S$.  We also provide a brief introduction to the
major tolls to find ``homology modules.''

This chapter should not be treated as a full introduction to algebraic
topology.  We present only the important results
about homology and cohomology theory to reach the goal mentioned in the
previous paragraph.  There are fundamentally two approaches to these
two topics: ``simplicial homology and cohomology''\ and
``singular homology and cohomology''.  The first approach is more
geometric and is mostly associated to algebraic geometry while the
second approach is purely algebraic and is associated to algebraic
topology.  We will briefly explore both approaches.

We present a very succinct introduction to simplicial homology and
cohomology in Sections~\ref{sectSimplHomol} and \ref{sectSimplCohom}.
We provide a little more complete introduction to singular homology
and cohomology in Sections~\ref{sectSingHom} and \ref{sectSingCohom}.
The importance of these sections is due to De Rham's theorem,
Theorem~\ref{deRhamThmSing}, in Subsection~\ref{ssectdeRhamSing} that
provides a relation between singular cohomology and de Rham cohomology
as presented in the previous chapter.

There is also a version of de Rham theorem, Theorem~\ref{DeRhamThm},
relating de Rham cohomology and simplicial cohomology.   We only
sketch this proof.  We have chosen to give a full proof of de Rham
theorem in the context of singular cohomology only.  In fact, as we
show in Section~\ref{sectRelSandS}, there is an equivalence between
the simplicial and singular theory.  Therefore, from a practical point
of view, there is no need to study both approaches.  From a
pedagogical point of view, it is nice to study both.
The readers whose main interested is the study of de Rham cohomology
may choose to only browse through Sections~\ref{sectSimplHomol} and
\ref{sectSimplCohom} on simplicial homology and cohomology, and
Section~\ref{sectRelSandS}.  These three sections are more interesting
for those who plan to study algebraic topology later.

Readers interested in learning more about the fascinating
subject of algebraic geometry should consult \cite{ST} for a good
basic introduction.   As for algebraic topology, \cite{GH} is
excellent references.  The content of this chapter is mainly based on
these two references and on \cite{LJM,MUat}.

This chapter requires some basic knowledge of abstract algebra; in
particular, of group theory.  There are many good books in algebra.
Two of them are \cite{He,Hu}.

\section{Fundamental Groups}  \label{sectFundGr}

All introductions to algebraic topology start with a study of
fundamental groups associated to a topological spaces even if this is
not the end goal.  It is a very visual introduction to the subject of
algebraic topology.  We provide only a brief overview of this subject,
leaving many of the stated results without proofs.
As we said in the introduction to this chapter, the reader may find
proofs of all the results that we state without proofs in \cite{GH,MUat,ST}
xofor instance.

The fundamental concept in this section is the notion of homotopy
between continuous functions.  We have already defined the concept of
homotopic functions in Definition~\ref{defnSmoothHomot} in the context
of maps between manifolds.  We generalize this concept to any
topological spaces.

\begin{focus}{Note}
From now on, when we talk about a topological space, we assume that it
is a Hausdorff space.
\end{focus}

\begin{defn}
Let $X$ and $Y$ be two topological spaces and
$f,g:X \to Y$ be two continuous functions.  We say that $f$ is
{\bfseries homotopic}\index{Homotopic Functions} to $g$ if there exists a
continuous map $H:X\times[0,1] \to Y$ such that
$H(x,0) = f(x)$ and $H(x,1) = g(x)$ for all $x \in X$.  We write
$f \sim g$.  The function $H$ is called an
{\bfseries homotopy}\index{Homotopy} from $f$ to $g$.
\end{defn}

This definition of homotopic functions differs from the definition
given in Definition~\ref{defnSmoothHomot} because we now require only that
functions be continuous.   Moreover, manifolds are replaced by any
topological spaces.   The concept of homotopy between continuous
functions defines an equivalence relation. Namely, if $f,g,h:X\to Y$
are continuous functions between two topological spaces $X$ and $Y$, then
\begin{enumerate}
\item $f \sim f$,
\item $f\sim g$ if and only if $g \sim f$, and
\item $f\sim g$ and $g \sim h$ imply that $f \sim h$.
\end{enumerate}

\begin{defn}
A topological space $X$ is {\bfseries contractible}\index{Contractible} if
there exists $x_0 \in X$ such that the identity function
$\Id_X:X \to X$ is homotopic to the constant function $f:X \to X$
defined by $f(x) = x_0$ for all $x \in X$. 
\end{defn}

As for the previous definition for homotopy between functions, we only
require a continuous homotopy for the definition of contractible
spaces instead of a smooth homotopy as in definition~\ref{defnCtoP}.

We also slightly modify our definition of path between two points
given in Definition~\ref{defnPathV1} for the context where we require
only that functions be continuous.

\begin{defn}
Let $X$ be a topological space.  A {\bfseries path from $x_0 \in X$ to
$x_1 \in X$}\index{Path} is a continuous function
$\alpha:[0,1] \to X$ such that $\alpha(0) = x_0$ and $\alpha(1) = x_1$.
\end{defn}

\begin{defn}
Let $X$ be a topological space.
If $\alpha$ is a path from $x_0\in X$ to $x_1 \in X$ and
$\beta$ is a path from $x_1 \in X$ to $x_2 \in X$, the {\bfseries
product}\index{Product of Paths} of $\alpha$ and $\beta$, denoted
$\alpha \beta$, is the path from $x_0$ to $x_2$ defined by
\[
(\alpha \beta) (t) = \begin{cases}
\alpha(2t) & \quad \text{if} \ 0 \leq t \leq 1/2 \\
\beta(2t-1) & \quad \text{if} \ 1/2 < t \leq 1
\end{cases}
\]
for $0 \leq t \leq 1$.
Also, $\displaystyle \alpha^{-1}$ is the path from $x_1$ to $x_0$ defined by
$\displaystyle \alpha^{-1}(t) = \alpha(1-t)$ for $0 \leq t \leq 1$.
\end{defn}

The product $\alpha \beta$ is what we informally called the union of the paths
$\alpha$ and $\beta$ in Chapter~\ref{chaptCVC}.  In this section, we
are really strict in preserving the interval $[0,1]$ of definition of
a path.  We are not just interested in the image of $\alpha$ and
$\beta$ as in Chapter~\ref{chaptCVC}. 

\begin{defn} \label{defnHomPaths}
Let $X$ be a topological space.  Suppose that $\alpha$ and $\beta$ are
two paths from $x_0 \in X$ to $x_1 \in X$.  We say that $\alpha$ is
{\bfseries homotopic}\index{Homotopic} to $\beta$ if there exists a
continuous map $H:[0,1] \times [0,1] \to X$ such that
$H(t,0) = \alpha(t)$ and $H(t,1) = \beta(t)$ for $0 \leq t \leq 1$,
and $H(0,s) = x_0$ and $H(1,s) = x_1$ for $0 \leq s \leq 1$.
We write $\alpha \dotsim[X] \beta$.  If there is no risk of confusion
for the range $X$ of $H$, we may simply write $\alpha \dotsim \beta$. 
\end{defn}

The path $\alpha$ is deformed to the path $\beta$ without changing the
end points (Figure \ref{FundGr1}).

\pdfF{alg_top/fundgr1}{Homotopic paths}{The path $\alpha$ from
$x_0$ to $x_1$ is homotopic to the path $\beta$ from $x_0$ to $x_1$.
We have $0 < s_1, s_2 < 1$ in the figure.}{FundGr1}

The homotopy of paths between two fixed points $x_0$ and $x_1$ of a
topological space $X$ defines an equivalence relation.
Suppose that $\alpha$, $\beta$ and $\gamma$ are paths from $x_0$ to $x_1$.
Then
\begin{enumerate}
\item $\alpha \dotsim \alpha$,
\item $\alpha \dotsim \beta$ if and only if $\beta \dotsim \alpha$, and
\item $\alpha \dotsim \beta$ and $\beta \dotsim \gamma$ imply
that $\alpha \dotsim \gamma$.
\end{enumerate}
We prove that last condition and left the proof of the two others to
the reader.  Since $\alpha \dotsim \beta$, there exists a continuous
function $H_1:[0,1] \times [0,1] \to X$ such that
$H_1(t,0) = \alpha(t)$ and $H_1(t,1) = \beta(t)$ for $0 \leq t \leq 1$,
and $H_1(0,s) = x_0$ and $H_1(1,s) = x_1$ for $0 \leq s \leq 1$.
Since $\beta \dotsim \gamma$, there exists a continuous function
$H_2:[0,1] \times [0,1] \to X$ such that
$H_2(t,0) = \beta(t)$ and $H_2(t,1) = \gamma(t)$ for $0 \leq t \leq 1$,
and $H_2(0,s) = x_0$ and $H_2(1,s) = x_1$ for $0 \leq s \leq 1$.
Let
\[
H(t,s) = \begin{cases}
H_1(t,2s) & \quad \text{if} \ 0 \leq s < 1/2 \ \text{and} \ 0\leq t \leq 1 \\
H_1(t,2s-1) & \quad \text{if} \ 1/2 \leq s \leq 1
\ \text{and} \ 0\leq t \leq 1
\end{cases}
\]
Then $H:[0,1] \times [0,1] \to X$ is a continuous function such that
$H(t,0) = \alpha(t)$ and $H(t,1) = \gamma(t)$ for $0 \leq t \leq 1$,
and $H(0,s) = x_0$ and $H(1,s) = x_1$ for $0 \leq s \leq 1$.  Thus
$\alpha \dotsim \gamma$.

It is not hard to prove that if $\alpha_1,\alpha_2$ are two paths from
$x_0\in X$ to $x_1 \in X$ such that $\alpha_1 \dotsim \alpha_2$, and
$\beta_1,\beta_2$ are two paths from $x_1 \in X$ to $x_2 \in X$ such
that $\beta_1 \dotsim \beta_2$, then $\alpha_1 \beta_1$ and
$\alpha_2\beta_2$ are two paths from $x_0$ to $x_2$ such that
$\alpha_1 \beta_1 \dotsim \alpha_2 \beta_2$.  Moreover,
$\displaystyle \alpha_1^{-1},\alpha_2^{-1}$ are two paths from
$x_1\in X$ to $x_0 \in X$ such that
$\displaystyle \alpha_1^{-1} \dotsim \alpha_2^{-1}$.

\begin{defn}
Let $X$ be a topological space and $x_0 \in X$.  A closed path in $X$ from
$x_0$ to $x_0$ is called a {\bfseries loop}\index{Loop} at $x_0$.
\end{defn}

Let $X$ be a topological space and $x_0 \in X$.  The set of
equivalence classes $[\alpha]$ of homotopic loops $\alpha$ at $x_0$ is
denoted $\pi_1(X,x_0)$.  It is a group where the product is defined by
$[\alpha]\,[\beta] = [\alpha\,\beta]$ for all loops $\alpha$ and
$\beta$ at $x_0$, the inverse is defined by 
$\displaystyle [\alpha]^{-1} = [\alpha^{-1}]$ for all loops $\alpha$
at $x_0$, and the identity element is defined by $[e_{x_0}]$ for the
path $e_{x_0}$ given by $e_{x_0}(t) = x_0$ for $0 \leq t \leq 1$. 
All these algebraic operations are well defined according to what we
said in the paragraph preceding the definition above.

\begin{defn} \label{defnFGpione}
The group $\pi_1(X,x_0)$ is called the
{\bfseries fundamental group}\index{Fundamental Group} or {\bfseries
$\displaystyle \mathbf{1^{st}}$ homotopy
group}\index{$\displaystyle 1^{st}$ Homotopy Group|see{Fundamental Group}} of
the pair $(X,x_0)$.
\end{defn}

\begin{prop}
Suppose that $f:X \to Y$ is a continuous function between two
path-connected topological spaces and that $x_0 \in X$.  Then the map
$f_\ast:\pi_1(X,x_0) \to \pi_1(Y,f(x_0))$ defined by
$f_\ast([\alpha]) = [f\circ \alpha]$ for all
$[\alpha] \in \pi_1(X,x_0)$ is an homomorphism.
\end{prop}

\begin{proof}
\stage{i} The map $f_\ast$ is well defined.  If $\sigma_0$ and
$\sigma_1$ are two loops in $X$ at $x_0$ such that
$\sigma_0 \dotsim \sigma_1$, then there exists
$H:[0,1]\times[0,1] \to X$ such that
$H(0,s) = H(1,s) = x_0$ for $0\leq s \leq 1$, and
$H(t,0) = \sigma_0(t)$ and $H(t,1) = \sigma_1(t)$ for $0\leq t \leq 1$.

We have that $f\circ \sigma_0$ and $f\circ \sigma_1$ are two loops in
$Y$ at $f(x_0)$.  Let $\tilde{H} = f \circ H$.  Then 
$\tilde{H}:[0,1]\times[0,1] \to X$ is such that
$\tilde{H}(0,s) = f(H(0,s)) = f(x_0)$ and
$\tilde{H}(1,s) = f(H(1,s)) = f(x_0)$ for $0\leq s \leq 1$, and
$\tilde{H}(t,0) = f(H(t,0)) = (f\circ \sigma_0)(t)$ and
$\tilde{H}(t,1) = f(H(t,1)) = (f\circ \sigma_1)(t)$ for $0\leq t \leq 1$.
Thus $f\circ \sigma_0 \dotsim f\circ \sigma_1$.

Hence $f_\ast([\sigma])$ is
independent of the representative path of $[\sigma]$ selected.

\stage{ii} The map $f_\ast$ is an homomorphism because
\begin{align*}
f_\ast([\alpha_0][\alpha_1]) &= f_\ast([\alpha_0\,\alpha_1])
= [f\circ (\alpha_0\,\alpha_1)] = [(f\circ \alpha_0)(f\circ\alpha_1)] \\
&= [f\circ \alpha_0][f\circ\alpha_1]
= f_\ast([\alpha_0]) f_\ast([\alpha_1])
\end{align*}
for all loops $\sigma_0$ and $\sigma_1$ in $X$ at $x_0$.
\end{proof}

Suppose that $X$, $Y$ and $Z$ are three path-connected topological
spaces, that $f:X \to Y$ and $g:Y\to Z$ are continuous functions, and
that $x_0 \in X$.  Then it is easy to prove that
$g_\ast \circ f_\ast = (g\circ f)_\ast: \pi_1(X,x_0) \to \pi_1(Z,g(f(x_0)))$.

\begin{prop}
Suppose that $f:X \to Y$ is a homeomorphism between two
path-connected topological spaces and that $x_0 \in X$.  Then the map
$f_\ast:\pi_1(X,x_0) \to \pi_1(Y,f(x_0))$ is an isomorphism.
\end{prop}

\begin{proof}
It follows from $\displaystyle f^{-1} \circ f = \Id_X$ that
$\displaystyle (f^{-1})_\ast \circ f_\ast = (f^{-1} \circ f)_\ast = (\Id_X)_\ast
= \Id_{\pi_1(X,x_0)}$ and
from $\displaystyle f \circ f^{-1} = \Id_Y$ that
$\displaystyle f_\ast \circ (f^{-1})_\ast = (f \circ f^{-1})_\ast =
(\Id_Y)_\ast = \Id_{\pi_1(Y,f(x_0))}$.  Thus $f_\ast$ has an inverse given by
$\displaystyle (f_\ast)^{-1} = (f^{-1})_\ast$.
\end{proof}

\begin{prop}
Suppose that $X$ is a path-connected topological space and that
$x_0 ,x_1 \in X$.  Then $\pi_1(X,x_0) \cong \pi_1(X,x_1)$.
\end{prop}

\begin{proof}
Since $X$ is path-connected, there exits a path $\beta$ in $X$ from
$x_0$ to $x_1$.
Consider $\beta_\sharp : \pi_1(X,x_0) \to \pi_1(X,x_1)$ defined by
$\displaystyle \beta_\sharp([\sigma]) = [\beta^{-1} \sigma \beta]$ for all
$\sigma \in \pi_1(X,x_0)$.

\stage{i}  We first note that $\beta_\sharp$ is well defined.
If $\sigma_0$ and $\sigma_0$ are two loops in $X$ at $x_0$ such that
$\sigma_0 \dotsim \sigma_1$, then there exists
$H:[0,1]\times[0,1] \to X$ such that $H(0,s) = H(1,s) = x_0$
for $0\leq s \leq 1$, and $H(t,0) = \sigma_0(t)$ and
$H(t,1) = \sigma_1(t)$ for $0\leq t \leq 1$.

We have that $\displaystyle \beta^{-1} \sigma_0 \beta$ and
$\displaystyle \beta^{-1}\sigma_1 \beta$ are two loops in $X$ at $x_1$.
Let $\tilde{H}:[0,1]\times[0,1] \to X$ be the function defined
by
\[
\tilde{H}(t,s) = \begin{cases}
\beta^{-1}(t) & \quad \text{if} \ 0 \leq t \leq 1/4
\ \text{and} \ 0\leq s \leq 1 \\
H(4t -1,s) & \quad \text{if} \ 1/4 < t < 1/2
\ \text{and} \ 0\leq s \leq 1 \\
\beta(t) & \quad \text{if} \ 1/2 \leq t \leq 1
\ \text{and} \ 0\leq s \leq 1
\end{cases}
\]
Then $\displaystyle \tilde{H}(0,s) = \beta^{-1}(0) = x_1$ and
$\tilde{H}(1,s) = \beta(1) = x_1$ for $0\leq s \leq 1$, and
\begin{align*}
\tilde{H}(t,0) &= \begin{cases}
\beta^{-1}(t) & \quad 0 \leq t \leq 1/4 \\
\sigma_0(4t -1) & \quad \text{if} \ 1/4 < t < 1/2 \\
\beta(t) & \quad \text{if} \ 1/2 \leq t \leq 1
\end{cases} \\
&= (\beta^{-1} \sigma_0\beta)(t)
\end{align*}
and
\begin{align*}
\tilde{H}(t,1) &= \begin{cases}
\beta^{-1}(t) & \quad 0 \leq t \leq 1/4 \\
\sigma_1(4t -1) & \quad \text{if} \ 1/4 < t < 1/2 \\
\beta(t) & \quad \text{if} \ 1/2 \leq t \leq 1
\end{cases} \\
&= (\beta^{-1} \sigma_1\beta)(t)
\end{align*}
for $0\leq t \leq 1$.
Thus
$\displaystyle \beta^{-1} \sigma_0 \beta \dotsim \beta^{-1}\sigma_1 \beta$.

\stage{ii} The map $\beta_\sharp$ is a homomorphism because
\begin{align*}
\beta_\sharp([\sigma_0][\sigma_1])
&= \beta_\sharp([\sigma_0\sigma_1])
= [\beta^{-1} \sigma_0\, \sigma_1 \beta]
= [\beta^{-1} \sigma_0 \, \underbrace{\beta \beta^{-1}}_{\dotsim e_{x_0}} \,
\sigma_1 \beta] \\
&= [\beta^{-1} \sigma_0 \beta]\,[\beta^{-1}\sigma_1 \beta]
= \beta_\sharp([\sigma_0])\beta_\sharp([\sigma_1])
\end{align*}
for all loops $\sigma_0$ and $\sigma_1$ in $X$ at $x_0$.

\stage{iii} The map $\beta_\sharp$ is an isomorphism.  In fact, it is
easy to verify that $\displaystyle \beta_\sharp^{-1} = (\beta^{-1})_\sharp$.
\end{proof}

If $X$ is a path-connected topological space, it follows from the
previous proposition that the fundamental group of a pair $(X,x_0)$ is,
up to isomorphism, the same group independently of the base point
$x_0 \in X$ chosen.  Hence, the following definition is justified.

\begin{defn}
Let $X$ be a path-connected topological space.  The
{\bfseries fundamental group}\index{Fundamental Group}
of $X$, denoted $\pi_1(X)$, is the group $\pi_1(X,x_0)$ for any
base point $x_0 \in X$.
\end{defn}

\subsection{Covering Spaces}

In this section, we prove only the results that will be used later.
The missing proofs and more information about covering spaces can be
found in \cite{GH,ST}.

\begin{defn}
A topological space $X$ is locally path-connected if for every
$x \in X$ and open neighbourhood $V \subset X$ of $x$, there exists an
open neighbourhood $U \subset V$ such that $U$ is path-connected.
\end{defn}

We insist on the fact that $U$ is path-connected in the
previous definition.  Namely, given any $x_1, x_2 \in X$, there exists
a path $\alpha$ in $U$ from $x_1$ to $x_2$.  The fact that
$\alpha(t) \in U$ for $0\leq t \leq 1$ is crucial.  A topological
space could be path-connected without being locally path connected as
can be seen in the following figure.
\pdfbox{alg_top/lpathc}
The set $X$ is a curve in $\displaystyle \RR^2$ with the induced
topology from $\displaystyle \RR^2$.  There is no open neighbourhood
of $\VEC{x}$ that is path-connected. 

The notion of locally path-connected may be new to the reader but it
is going to be essential for the next important concept.

\begin{defn}  \label{defnCovering}
Let $X$ and $Q$ be two path-connected and locally
path-connected topological spaces.  Suppose that $p:Q \to X$
is a continuous and surjective function such that, for all $x \in X$,
there exists an open neighbourhood $U \subset X$ of $x$ satisfying the
following conditions:
\begin{enumerate}
\item $\displaystyle p^{-1}(U) = \bigcup_{\tau \in T} V_\tau$ with
$V_\tau\subset Q$ open sets and $V_{\tau_1} \cap V_{\tau_2} = \emptyset$
for $\tau_1 \neq \tau_2$, and
\item $p\big|_{V_\tau}:V_\tau \to U$ is a homeomorphism for all $\tau$.
\end{enumerate}
Then $(Q,p)$ is called a {\bfseries covering}\index{Covering}
of $X$ (Figure \ref{FundGr2}).
\end{defn}

\pdfF{alg_top/fundgr2}{Covering of $S^1$}{Two illustrations of the covering
of $\displaystyle S^1$.  In the lower illustration, if we assume that
$\displaystyle S^1$ is represented by a circle of radius $1$ centred
at the origin, $p$ is given by $\displaystyle p(t) = e^{2\pi t i}$ for
$t \in \RR$.}{FundGr2}

\begin{prop}
If $(Q,p)$ is a covering of a topological space $X$, then $p:Q \to X$
is an open mapping.
\end{prop}

\begin{proof}
Let $W$ be an open subset of $Q$.  Given $x = p(W)$, there exists an
open neighbourhood $U \subset X$ of $x$ satisfying
Definition~\ref{defnCovering}.  Choose $y \in W$ such that $p(y) = x$.
Then $y \in V_\tau$ for some $\tau$.  Since $W \cap V_\tau$ is an open subset
of $V_\tau$ and $p\big|_{V_\tau}:V_\tau \to U$ is a homeomorphism, we have that
$B = p(W\cap V_\tau)$ is an open neighbourhood of $x$ in $U$ and also in $X$
because $U$ is open.  Thus $B$ is an open neighbourhood of $x$ with
$B \subset p(W)$.  Since $x \in p(W)$ is arbitrary, this proves that
$p(W)$ is an open subset of $X$.
\end{proof}

\begin{prop} \label{propUnLift}
Let $(Q,p)$ be a covering of a topological space $X$.
If $\alpha, \beta : Y \to Q$ are two continuous functions defined
on a connected space $Y$ such that
$p \circ \alpha = p \circ \beta$ on $Y$ and $\alpha(\tilde{y})
= \beta(\tilde{y})$ for a point $\tilde{y} \in Y$, then $\alpha = \beta$.  
\end{prop}

\begin{proof}
Let $h:Y \to Q \times Q$ be the function defined by
$h(y) =(\alpha(y),\beta(y))$ for $y \in Y$.  We have that $h$ is a
continuous function.

Let $D = \{(q,q) : q \in Q\}$.  Since $D$ is a closed set
\footnote{In fact $D$ is closed if and only if $Q$ is hausdorff.  One
part of the proof consists in proving that $Q$ Hausdorff implies that
$(Q\times Q) \setminus D$ is open.  The proof of the other direction
is even more direct.  See \cite{Du}.}, we have that
$\displaystyle Z = \{ \VEC{y} \in Y : \alpha(y)
= \beta(y) \} = h^{-1}(D)$ is closed.
We also have that $Z \neq \emptyset$ because $\tilde{y} \in Z$.  Since
$Y$ is connected, it suffices to prove that $Z$ is also open to
conclude that $Z=Y$.

To prove that $Z$ is open, choose $\breve{y} \in Z$ and let
$x = p(\alpha(\breve{y})) = p(\beta(\breve{y}))$.  There exists an
open neighbourhood $U \subset X$ of $x$ satisfying
Definition~\ref{defnCovering}.
Then $\alpha(\breve{y}) = \beta(\breve{y}) \in V_\tau$ for some $\tau$.
Since $\alpha$ is continuous, there exists an open neighbourhood
$\displaystyle W_1 \subset \alpha^{-1}(V_\tau)$ of $\breve{y}$.
Similarly, since $\beta$ is continuous, there exists an open neighbourhood
$\displaystyle W_2 \subset \beta^{-1}(V_\tau)$ of $\breve{y}$.
Since $p\big|_{V_\tau}:V_\tau \to U$ is one-to-one and
$p \circ \alpha = p \circ \beta$ on $W_1 \cap W_2$, we get that
$\alpha = \beta$ on the open set $W_1 \cap W_2$.  Thus
$W_1 \cap W_2 \subset Z$.
Since $\breve{y} \in Z$ is arbitrary, this proves that $Z$ is open.
\end{proof}

The following theorem has deep consequences.

\begin{theorem}[Covering Homotopy Theorem]
Suppose that $(Q,p)$ is a covering of a topological space $X$
and that $Y$ is a compact and connected space.
Moreover, suppose that $f,g:Y \to Q$ are two continuous
functions and that $F:Y \times [0,1] \to X$ is a homotopy between 
$p\circ f$ and $p \circ g$.  Then there exists a homotopy
$G:Y \times [0,1]\to Q$ between $f$ and $g$ satisfying the
following conditions.
\begin{enumerate}
\item $p \circ G = F$ on $Y \times [0,1]$.
\item If there exists $y \in Y$ and an interval $E \subset [0,1]$
such that $F(y,s)$ is constant for $s \in E$, then
$G(y,s)$ is constant for $s \in E$.
\end{enumerate}
\end{theorem}

\begin{proof}
\stage{i} The set $Y \times [0,1]$ is compact because $Y$ and $[0,1]$
are compact.  For each $x \in X$, let $\displaystyle U_x \subset X$
be an open neighbourhood of $x$ satisfying Definition~\ref{defnCovering}.
Since the collection $\displaystyle \{ U_x \}_{x\in X}$
is an open cover of $F(Y\times[0,1])$, we have that the collection
$\displaystyle \big\{ F^{-1}(U_x) \big\}_{x\in X}$
is an open cover of the compact set $Y\times[0,1]$.  Hence, there
exists a finite subcover
$\displaystyle \big\{ F^{-1}(U_{x_j}) \big\}_{1 \leq j \leq J}$
of $Y\times [0,1]$.

We can cover $Y \times [0,1]$ with open sets of the form
$W_\alpha \times I_\alpha$ where $W_\alpha \subset Y$ is a connected
open set, $I_\alpha$ is an open interval, and
$\displaystyle W_\alpha \times I_\alpha \subset F^{-1}(U_{x_j})$ for some
$j \in \{1,2,\ldots, J\}$.  Again, because $Y \times [0,1]$ is
compact, there exists a finite subcover
$\displaystyle \{ W_{\alpha_k} \times I_{\alpha_k} \}_{1\leq k \leq K}$
of $Y \times [0,1]$.  By splitting the intervals $I_{\alpha_k}$, we
get a finite subcover of $Y \times [0,1]$ of the form
$\displaystyle \{ W_{\alpha_k} \times [s_{i-1}, s_i]
\}_{1\leq k \leq K,1\leq i \leq I}$ where
$s_0 =0 < s_1 < s_2 < \ldots < s_I = 1$.  We still have that
$\displaystyle F\big(W_{\alpha_k} \times [s_{i-1}, s_i]\big) \subset
U_{x_{j(k,i)}}$ for some $j(k,i) \in \{1,2,\ldots, J\}$.

\stage{ii}  We define continuous function
$G_i:Y \times [s_{i-1},s_i] \to Q$ such that
\[
p \circ G_i = F\Big|_{Y \times [s_{i-1},s_i]}\quad  \text{and} \quad
G_i\big|_{Y \times \{s_i\}} = G_{i-1}\big|_{Y \times \{s_i\}}
\]
for $1 \leq i \leq I$.
Because of the second condition, we will have that
$G:Y \times [0,1] \to Q$ defined by
$G(y,s) = G_i(y,s)$ for $y\in Y$ and $t \in [s_{i-1},s_i]$ is a well
defined continuous function satisfying (1) in the statement of the
theorem.

\stage{iii}
Assume that $1 \leq i \leq I$.  Let
\[
G_i(y,s_{i-1}) =
\begin{cases}
f(y) & \quad \text{if}\ i = 1 \\
G_{i-1}(y,s_{i-1}) & \quad \text{if} \ 1 < i \leq I
\end{cases}
\]
for $y \in Y$.  Since $W_{\alpha_k}$ is connected, we have that
\[
G_i(W_{\alpha_k},s_{i-1})
= \begin{cases}
f(W_{\alpha_k}) & \quad \text{if}\ i = 1 \\
G_{i-1}(W_{\alpha_k},s_{i-1}) & \quad \text{if} \ 1 < i \leq I
\end{cases}
\]
is connected.  Since
$F(W_{\alpha_k},[s_{i-1},s_i]) \subset U_{x_{j(k,i)}}$ for some
$j(k,i) \in \{1,2,\ldots,J\}$ and
\begin{align*}
p \circ G_i\big|_{W_{\alpha_k} \times \{s_{i-1}\}} &=
\begin{cases}
p \circ f\big|_{W_{\alpha_k}} & \quad \text{if} \ i = 0 \\
p \circ G_{i-1}\big|_{W_{\alpha_k} \times \{s_{i-1}\}} &
\quad \text{if} \ 1 < i \leq I
\end{cases} \\
&= F\big|_{W_{\alpha_k} \times\{s_{i-1}\}} \ ,
\end{align*}
we have that
$G_i(W_{\alpha_k},s_{i-1}) \subset V_{x_{j(k,i)},\tau}$ for some
$\tau \in T$, where
$\displaystyle p^{-1}(U_{x_{j(k,i)}})
= \bigcup_{\tau \in T} V_{x_{j(k,i)},\tau}$ with
$V_{x_{j(k,i)},\tau_1} \cap V_{x_{j(k,i)},\tau_2} = \emptyset$ if
$\tau_1 \neq \tau_2$
as given in Definition~\ref{defnCovering}.  More precisely,
$G_i(W_{\alpha_k},s_{i-1})$ is a connected subset
of $\displaystyle p^{-1}(U_{x_{j(k,i)}})
= \bigcup_{\tau \in T} V_{x_{j(k,i)},\tau}$
where the $V_{x_{j(k,i)},\tau}$ are distinct open sets.  Therefore
$G_i(W_{\alpha_k},s_{i-1})$ is a subset of only one of them.

Since $p\big|_{V_{x_{j(k,i)},\tau}} :V_{x_{j(k,i)},\tau} \to U_{x_{j(k,i)}}$ is a
homeomorphism, we may set\\
$\displaystyle G_{i,k}(y,s) = p^{-1}(F(y,s))$ for
$(y,s) \in W_{\alpha_k} \times [s_{i-1},s_i]$.
We define $G_i:Y \times [s_{i-1},s_i] \to Q$ by
$G_i(y,s) = G_{i,k}(y,s)$ if
$(y,s) \in W_{\alpha_k} \times [s_{i-1},s_i]$.  To prove that $G_i$
is well defined and continuous on $Y \times [s_{i-1},s_i]$, we
prove in the next step that
$G_{i,k_1}(y,s) = G_{i,k_2}(y,s)$ for
$(y,s) \in (W_{\alpha_{k_1}} \cap W_{\alpha_{k_2}}) \times [s_{i-1},s_i]$.

\stage{iv}  Suppose that
$W_{\alpha_{k_1}} \cap W_{\alpha_{k_2}} \neq \emptyset$.  We have by
definition of $G_{i,k}$ that
\[
G_{i,k_m}(W_{\alpha_{k_m}},[s_{i-1},s_i]) \subset
V_{x_{j(k_m,i)},\tau_m}
\]
for some $j(k_m,i) \in \{1,2,\ldots,J\}$ and $\tau_m \in T$ with
$1\leq m \leq 2$.  We also have
\[
G_{i,k_1}(y,s_{i-1}) = G_{i,k_2}(y,s_{i-1})
= \begin{cases}
f(y) & \quad \text{if} \ i = 1 \\
G_{i-1}(y,s_{i-1}) & \quad \text{if} \ 1 < i \leq I
\end{cases}
\]
for all $(y,s) \in (W_{\alpha_{k_1}} \cap W_{\alpha_{k_2}})
\times [s_{i-1},s_i]$.  Thus
$G_{i,k_1}(y,s_{i-1}) = G_{i,k_2}(y,s_{i-1}) \in 
V_{x_{j(k_1,i)},\tau_1} \cap V_{x_{j(k_2,i)},\tau_2}$ for all
$(y,s) \in (W_{\alpha_{k_1}} \cap W_{\alpha_{k_2}}) \times [s_{i-1},s_i]$.

Consider $(y,s) \in (W_{\alpha_{k_1}} \cap W_{\alpha_{k_2}})
\times [s_{i-1},s_i]$ and the two functions
\begin{align*}
\sigma_1:[0,1] &\to Q \\
t & \mapsto G_{i,k_1}(y,t s + (1-t) s_{i-1})
\end{align*}
and
\begin{align*}
\sigma_2:[0,1] &\to Q \\
t & \mapsto G_{i,k_2}(y,t s + (1-t) s_{i-1})
\end{align*}
We have that $\sigma_1(0) = \sigma_2(0)$ and
\begin{align*}
(p\circ \sigma_1)(t) &= (p \circ G_{i,k_1}(y,t s + (1-t) s_{i-1})
= F(y, t s + (1-t) s_{i-1}) \\
&= (p \circ G_{i,k_2}(y, t s + (1-t) s_{i-1})
= (p\circ \sigma_2)(t)
\end{align*}
for $0 \leq t \leq 1$.  It follows from Proposition~\ref{propUnLift}
that $\sigma_1(t) = \sigma_2(t)$ for $0\leq t \leq 1$.  Thus
$G_{i,k_1}(y,s) = \sigma_1(1) = \sigma_2(1) = G_{i,k_2}(y,s)$.

\stage{v} Repeating recursively (iii) and (iv) from $i=1$ to $i=I$
yields the functions $G_i$ in (ii).

\stage{vi} Recall that $G_{i,k}$ is defined by
$\displaystyle G_{i,k}(y,s) = p^{-1}(F(y,s))$ for
$(y,s) \in W_{\alpha_k} \times [s_{i-1},s_i]$ where
$p\big|_{V_{x_{j(k,i)},\tau}} :V_{x_{j(k,i)},\tau} \to U_{x_{j(k,i)}}$ is a
homeomorphism.  Therefore $G_{i,k}(y,s)$ is constant with respect to
$s$ if $F(y,s)$ is constant with respect to $s$ because
$p\big|_{V_{x_{j(k,i)},\tau}} :V_{x_{j(k,i)},\tau} \to U_{x_{j(k,i)}}$
is one-to-one.
\end{proof}

It follows from Prop~\ref{propUnLift} that if $(Q,p)$ is a covering of
a topological space $X$ and $q \in Q$, then
$p_\ast: \pi_1(Q,q) \to \pi_1(X, p(q))$ is one-to-one.  If $\alpha$ is
a path in $X$ such that $\alpha(0) = p(q)$, then there exists a unique
path $\beta$ in $Q$ such that $p \circ \beta = \alpha$ and $\beta(0) = q$.

There are several interesting and important corollaries to this
theorem.  For instance, if $x \in X$ and $x = p(q)$, there exists a
one-to-one map from the quotient group\\
$\displaystyle \pi_1(X,x) / p_\ast(\pi_1(Q,q))$
onto $\displaystyle p^{-1}(\{x\})$.  To prove this result, we first
construct a map $M$ from $\displaystyle \pi_1(X,x)$ onto
$\displaystyle p^{-1}(\{x\})$ as it follows.  Given
$[\alpha] \in \pi_1(X,x)$, we have from the result stated in the previous 
paragraph that there exists a unique path $\beta$ in $Q$ such that
$p \circ \beta = \alpha$ with $\beta(0) = q$.  We set
$M([\alpha]) = \beta(1)$.  The rest of the proof consists in proving
that the definition of $M$ is independent of the representative $\alpha$
chosen, that $M$ is onto and that the kernel of $M$ is
$p_\ast(\pi_1(Q,q))$.

\begin{defn}
Let $X$ be a path-connected and locally path-connected topological space.
We say that $X$ is {\bfseries simply connected}\index{Simply Connected}
if $\pi_1(X)$, the fundamental group of $X$, is trivial.
\end{defn}

It follows that if $X$ is simply connected and $x \in X$, then
$\alpha \dotsim e_x$ for all loops $\alpha$ in $X$ at $x$, where $e_x$ is the
path defined by $e_x(t) = x$ for $0 \leq t \leq 1$.

\begin{defn}
A topological space $X$ is
{\bfseries locally simply connected}\index{Locally Simply Connected}
if for every $x \in X$ there exists an open neighbourhood $U$ of $x$ such
$\alpha \dotsim[U] e_x$ for all loops $\alpha$ in $U$ at $x$.
\end{defn}

We insist on the fact that the range of the homotopy in the previous
definition is only $U$.

\begin{defn}
Let $(Q,p)$ be a covering of a topological space $X$.  We say
that $(Q,p)$ is a {\bfseries universal covering}\index{Universal Covering}
of $X$ if $Q$ is simply connected
\end{defn}

\begin{theorem}  \label{thCoveringEx}
Let $X$ be a path-connected, locally path-connected and locally simply
connected topological space.  Given $x \in X$, let $H$ be a subgroup
of $\pi_1(X,x)$.  Then there exists a covering $(Q,p)$ of $X$
such that $p_\ast(\pi_1(Q,q)) = H$ for $q \in Q$ such that $p(q) = x$.
\end{theorem}

In particular, if $H = \{[e_x]\}$ for $x \in X$, we get a universal
covering $(Q,p)$ of $X$ because, as we stated before,
$p_\ast:\pi_1(Q,q) \to \pi_1(X,p(q))$ is one-to-one.

It can be proved that if $(Q,p)$ is a covering of
a topological space $X$ and $x \in X$, then
$p_\ast(\pi_1(Q,q_1))$ and $p_\ast(\pi_1(Q,q_2))$ with
$p(q_1)= p(q_2) = x$ are two conjugate subgroups in
$\pi_1(X,x)$; namely, there exists $[\beta] \in \pi_1(X,x)$ such that
$\displaystyle [\beta]\, p_\ast(p_1(Q,q_1))\, [\beta]^{-1}= p_\ast(p_1(Q,q_2))$.
In fact, if $\gamma$ is a path in $Q$ from $q_1$ to $q_2$,
then $\beta = p \circ \gamma$.

\begin{defn}
Two coverings $(Q_1,p_1)$ and $(Q_1,p_1)$ of a
topological space $X$ are {\bfseries isomorphic}\index{Isomorphic Coverings}
if there exists a homeomorphism $h:Q_1 \to Q_2$ such that $p_2 \circ h = p_1$. 
\end{defn}

Another result that can relatively easily be proved states that
if $(Q_1,p_1)$ and $(Q_2,p_2)$ are two coverings of a locally simply
connected space $X$ such that $p_\ast(\pi_1(Q_1,q_1)) = p_\ast(\pi_1(Q_2,q_2))$
for any $q_1 \in Q_1$ and $q_2 \in Q_2$ with
$p(q_1) = p(q_2)$, then $(Q_1,p_1)$ and $(Q_2,p_2)$ are isomorphic coverings.

\begin{defn}
Let $(Q,p)$ be a covering of a topological space $X$.
A {\bfseries covering transformation}\index{Covering Transformation}
of $(Q,p)$ is a homeomorphism $h:Q \to Q$ such that $p \circ h = p$.
\end{defn}

The set of covering transformations of $(Q,p)$ form a group
under composition of functions.  This group is denoted
${\cal G}(Q,p)$.

If $U$ is in open set as in the definition of covering,
Definition~\ref{defnCovering}, then a covering transformation $h$
permutes the open sets $V_j$.

\begin{defn}
A {\bfseries regular covering}\index{Regular Covering} of a
topological space $X$ is a covering $(Q,p)$ of $X$ such that
$p_\ast(\pi_1(Q,q))$ is a normal subgroup of $\pi_1(X,p(q))$ for
$q \in Q$; namely,\\
$\displaystyle [\beta]\, p_\ast(\pi_1(Q,q))\, [\beta]^{-1} =
p_\ast(\pi_1(Q,q))$ for all $[\beta] \in \pi_1(X,p(q))$.
\end{defn}

It follows from the paragraph following Theorem~\ref{thCoveringEx}
that the definition of regularity is independent of the point
$q \in Q$ selected.

The next theorem is the fundamental theorem of this subsection.

\begin{theorem}
Suppose that $(Q,p)$ is a regular covering of a locally simply
connected space $X$ and that $q \in Q$.
Then ${\cal G}(Q,p)$ is group isomorphic
to $\displaystyle \pi_1(X,p(q)) / p_\ast (\pi_1(Q,q))$.
In particular, if the covering is also universal, then
${\cal G}(Q,p)$ is group isomorphic to $\pi_1(X,p(q))$.
\end{theorem}

\begin{prop} \label{propFGSone}
$\displaystyle \pi_1(S^1,1) \cong \ZZ$.  In fact, since
$\displaystyle S^1$ is path-connected, we have that
$\displaystyle \pi_1(S^1) \cong \ZZ$.
\end{prop}

\begin{proof}
In the statement of the theorem, we assume the complex representation
of $\displaystyle S^1$; namely,
$\displaystyle S^1 = \{z \in \CC : |z| = 1 \}$.

Suppose that $(Q,p)$ is the covering described in
Figure~\ref{FundGr2}.  We have that $Q = \RR$ and $(Q,p)$ is an
universal and regular covering of $\displaystyle S^1$.
If $h:\RR \to \RR$ is a covering transformation, 
then $p \circ h = p$ implies that
\[
e^{2\pi h(q) i} = e^{2\pi q i} \iff e^{2\pi (h(q)-q) i} = 1
\]
for all $q \in \RR$.  Thus $h(q)-q$ is a continuous function defined
on $\RR$, a connected set, whose range is $\ZZ$.  This is possible
only if there exists $n\in \ZZ$ such that $h(q) - q = n$ for all
$q \in \RR$.  This proves that the covering transformations are all
given by integer translations.  Hence ${\cal G}(Q,p) \cong \ZZ$ and
from the previous theorem
$\displaystyle \pi_1(S^1) \cong {\cal G}(Q,p) \cong \ZZ$.
\end{proof}

\begin{prop}
Suppose that $X$ and $Y$ are path-connected topological spaces, and that
$x_0 \in X$ and $y_0 \in Y$.  Then
$\pi_1(X\times Y, (x_0,y_0)) \cong \pi_1(X,x_0) \times \pi_1(Y,y_0)$
\end{prop}

\begin{proof}
It is easy to verify that the two projections
$P:X\times Y \to X$ and $Q:X\times Y \to Y$ induces two homomorphism
$P_\ast: \pi_1(X\times Y, (x_0,y_0)) \to \pi_1(X,x_0)$
defined by $P_\ast([\sigma]) = [P \circ \sigma]$
and $Q_\ast:\pi_1(X\times Y, (x_0,y_0)) \to \pi_1(Y,y_0)$
defined by $Q_\ast([\sigma]) = [Q \circ \sigma]$
for $[\sigma] \in \pi_1(X\times Y, (x_0,y_0))$.
Thus $(P_\ast,Q_\ast): \pi_1(X\times Y, (x_0,y_0)) \to
\pi_1(X,x_0) \times \pi_1(Y,y_0)$ defined by
$(P_\ast,Q_\ast)([\sigma]) = (P_\ast([\sigma]),Q_\ast([\sigma]))$
for $[\sigma] \in \pi_1(X\times Y, (x_0,y_0))$ is an homomorphism.

The map $(P_\ast,Q_\ast)$ is also an isomorphism because it has an
inverse.  If $\sigma_X$ is a loop in $X$ at $x_0$ and
$\sigma_Y$ is a loop in $Y$ at $y_0$, then $(\sigma_X,\sigma_Y)$
is a loop in $X \times Y$ at $(x_0,y_0)$.
Let $R: \pi_1(X,x_0) \times \pi_1(Y,y_0) \to \pi_1(X\times Y, (x_0,y_0))$
be the map defined by
$\displaystyle R([\sigma_X],[\sigma_Y]) = [(\sigma_X,\sigma_Y)]$ 
for $[\sigma_X] \in \pi_1(X,x_0)$ and
$[\sigma_Y] \in \pi_1(Y,y_0)$.
The map $R$ is well defined because
$\sigma_X \dotsim \tilde{\sigma}_X$ and
$\sigma_Y \dotsim \tilde{\sigma}_Y$ imply
$(\sigma_X,\sigma_Y) \dotsim (\tilde{\sigma}_X,\tilde{\sigma}_Y)$
\footnote{We leave it to the reader to construction the map
$H:[0,1]\times[0,1] \to X \times Y$ needed to prove that
$(\sigma_X,\sigma_Y) \dotsim (\tilde{\sigma}_X,\tilde{\sigma}_Y)$.}.
Since all loops $\sigma$ in $X \times Y$ at $(x_0,y_0)$ are of the form
$\sigma = (\sigma_X,\sigma_Y)$ where $\sigma_X = P\circ \sigma$
is a loop in $X$ at $x_0$ and $\sigma_Y = Q\circ \sigma$ is a loop
in $Y$ at $y_0$,  we have that $R$ is onto.  The map $R$ is also
one-to-one because 
$\sigma \dotsim \tilde{\sigma}$ in $X \times Y$ implies that
$P\circ \sigma \dotsim P\circ \tilde{\sigma}$ in $X$ and
$Q\circ \sigma \dotsim Q\circ \tilde{\sigma}_Y$ in $Y$
\footnote{Again, we leave it to the reader to construction the map
$H_P:[0,1]\times[0,1] \to X$ and $H_Q:[0,1]\times[0,1] \to Y$
needed to prove that
$P\circ \sigma \dotsim P\circ \tilde{\sigma}$ in $X$ and
$Q\circ \sigma \dotsim Q\circ \tilde{\sigma}_Y$ in $Y$.}.

It is now easy to verify that
$\displaystyle R\circ (P_\ast,Q_\ast) = \Id_{\pi_1(X\times Y, (x_0,y_0))}$ and \\
$\displaystyle (P_\ast,Q_\ast) \circ R = \Id_{\pi_1(X, x_0) \times \pi_1(Y,y_0)}$.
\end{proof}

\begin{egg}
The torus $\displaystyle \torus{2} \subset \RR^3$         \label{eggFGtorus}
is isomorphic to $\displaystyle S^1 \times S^1$.
This can be seen from the parametric representation of the torus.
Hence, it follows from the previous proposition that
$\displaystyle \pi_1(\torus{2}) \cong \pi_1(S^1) \times \pi_1(S^1) \cong
\ZZ \times \ZZ$ since $\torus{2}$ is path-connected.
\end{egg}

To conclude this section, we plan to prove a limited version of the
famous Brouwer fixed point theorem.  Namely, a version limited to
$\displaystyle \RR^2$.  We first need a new definition and a lemma.

\begin{defn} \label{defnRetract}
Let $Y$ be a subset of a topological space $X$.
A {\bfseries retraction}\index{Retraction}
of $X$ onto $Y$ is a continuous function $r:X \to Y$ such that $r(x) = x$
for all $x \in Y$ or, equivalently, $r \circ \iota = \Id_Y$ for the
inclusion map $\iota :Y \to X$.  We then say that $Y$ is a
{\bfseries retract}\index{Retract} of $X$.
\end{defn}

\begin{lemma}  \label{lenNoRetract}
Let $\displaystyle D = \{ \VEC{x} \in \RR^2 : \|\VEC{x}\| \leq 1 \}$.
There does not exist any retraction of $D$ onto
$\displaystyle S^1 = \{\VEC{x} \in \RR^2 : \|\VEC{x}\|=1 \}$.
\end{lemma}

\begin{proof}
Suppose that there is a retraction $\displaystyle g:D \to S^1$.
Let $\displaystyle \iota:S^1 \to D$ be the inclusion map.  
Then $g \circ \iota = \Id_{S^1}$ and
\begin{equation} \label{lemBrouwerEq1}
g_\ast \circ \iota_\ast = (g \circ \iota)_\ast
= \Id_{\pi_1(S^1,\VEC{e}_1)} \ .
\end{equation}
However, $\pi_1(D,\VEC{e}_1) = \{[e_{\VEC{e}_1}]\}$ where
$e_{\VEC{e}_1}(t) = \VEC{e}_1$ for $0 \leq t \leq 1$
because $D$ is contractible.  Thus
$\displaystyle \iota_\ast(\pi_1(S^1,\VEC{e}_1)) \subset \pi_1(D,\VEC{e}_1) =
\{[e_{\VEC{e}_1}]\}$ implies that
$\displaystyle (g_\ast \circ \iota_\ast)(\pi_1(S^1,\VEC{e}_1))
= \{[e_{\VEC{e}_1}]\}$.  This is a contradiction of
Proposition~\ref{propFGSone} because, according to
(\ref{lemBrouwerEq1}), the image of $g_\ast \circ \iota_\ast$ should
be $\displaystyle \pi_1(S^1,\VEC{e}_1) \cong \ZZ \not\cong
\{[e_{\VEC{e}_1}]\}$.
\end{proof}

\begin{theorem}[Brouwer Fixed Point Theorem]
Let $\displaystyle D = \{ \VEC{x} \in \RR^2 : \|\VEC{x}\| \leq 1 \}$.
Suppose that $f:D \to D$ is a continuous function.  Then there exists
$\VEC{y} \in D$ such that $f(\VEC{y}) = \VEC{y}$; namely, $f$ has
a {\bfseries fixed point}\index{Fixed Point} in $D$.
\end{theorem}

\begin{proof}
Suppose that there does not exist $\VEC{y} \in D$ such that
$f(\VEC{y}) = \VEC{y}$.
Let $\displaystyle g:D \to S^1$ be the function defined by
$\displaystyle g(\VEC{x})$ is the intersection with the circle
$\displaystyle S^1$ of the line through $\VEC{x}$ originating from
$f(\VEC{x})$.  Such a line is well defined because
$f(\VEC{x}) \neq \VEC{x}$ for all $\VEC{x} \in D$.  It
is also a continuous function because $f$ is continuous.
Moreover $g(\VEC{x}) = \VEC{x}$ for all $\displaystyle \VEC{x} \in S^1$.
Thus, $g$ is a retraction of $D$ onto $\displaystyle S^1$.
This is a contradiction of Lemma~\ref{lenNoRetract}.
\end{proof}

The proof of the Brouwer fixed point theorem is standard and can be
generalized to $\displaystyle \RR^{n+1}$ with $n>1$ using a group
structure on $\displaystyle S^n$ other than the fundamental group of
$\displaystyle S^n$ because
$\displaystyle \pi_1(S^n,\VEC{e}_1) = \{[ e_{\VEC{e}_1}]\}$ for $n>1$.

\begin{rmk}
It is possible to define higher order homotopy groups as it is
explained in \cite{GH}.  We however follow the tradition of instead
introducing singular homology as we will do in
Section~\ref{sectSingHom}.   As we will see in
Subsection~\ref{subsectP1EquH1} later, it is a natural way to
generalize the concept of fundamental group.
\end{rmk}

%%% Local Variables:
%%% mode: latex
%%% TeX-master: "notes"
%%% End:


\section{Simplicial Homology} \label{sectSimplHomol}

As mentioned in the introduction of this chapter, readers may choose
to only browse through this and the next sections if they are not
very interested in learning about simplicial homology and cohomology.
There are however some concepts and definitions in this two sections
that the reader should learn because they are used in the next
chapter.  In particular, we are talking about the definitions of
simplex and simplicial complex in Section~\ref{subsOShomo}, oriented
simplex in Section~\ref{ssectSimpCo}, and smoothly triangulated
manifold in Section~\ref{ssectSCandCM}.  Sections~\ref{sectSingHom} and
\ref{sectSingCohom} on singular homology and cohomology are the
important sections in this chapter.  In particular,
Subsection~\ref{ssectdeRhamSing} contains the statement and the proof 
of de Rham theorem, Theorem~\ref{deRhamThmSing}. which is the
motivation for this chapter.

We present and sketch the proof of a version of de Rham theorem for
simplicial cohomology in Section~\ref{ssectSCandCM} for those who
may only be interested in a brief study of simplicial homology and
cohomology.

Our presentation of the simplicial homology and cohomology is more for
a pedagogical purpose, to prepare the reader to the study of singular
homology and cohomology.  There is a lot of similarity between the
results and techniques of proof used in simplicial theory and those used in
singularity theory.  This is not surprising as it is explained in
Section~\ref{sectRelSandS}.  Nevertheless, there are really
important and interesting results that can only be deduced from
simplicial homology.  Theorem~\ref{thmEulerAlpha} is one of them.
Both simplicial and singular theories are important.  It is not a
question of which one is better than the other.

\subsection{Simplicial Complexes} \label{ssectSimpCo}

\begin{defn}
A subset $\{\VEC{x}_0,\VEC{x}_1,\VEC{x}_2,\ldots, \VEC{x}_k\}$ of
$\displaystyle \RR^n$ is
{\bfseries convex independent}\index{Convex Independent} if
$\{\VEC{x}_1-\VEC{x}_0,\VEC{x}_2-\VEC{x}_0,\ldots, \VEC{x}_k-\VEC{x}_0\}$
are linearly independent.
\end{defn}

\begin{defn}
Let $s = \{\VEC{x}_0,\VEC{x}_1,\VEC{x}_2,\ldots, \VEC{x}_k\}$ be a
convex independent subset of $\displaystyle \RR^n$.   The set
$\displaystyle \left\{ \sum_{j=0}^k a_j\VEC{x}_j : a_j \geq 0 \ \text{and} \
\sum_{j=0}^k a_k = 1 \right\}$, denoted $[s]$ or
$[\VEC{x}_0,\VEC{x}_1,\VEC{x}_2,\ldots,\VEC{x}_k]$, is called a
{\bfseries closed $\mathbf{k}$-simplex}\index{Closed $k$-Simplex}.
The {\bfseries vertices}\index{Vertex} of $[s]$ are the
points $\VEC{x}_j$ for $0\leq j \leq k$. The
{\bfseries closed faces}\index{Closed Face} of $[s]$ are the
closed $\mathbf{m}$-simplices
$[\VEC{x}_{j_0},\VEC{x}_{j_1},\VEC{x}_{j_2},\ldots, \VEC{x}_{j_m}]$
for $\{ j_0, j_1, \ldots, j_m\} \subset \{0,1,2,\ldots, k\}$
and $0\leq m \leq k$.
The {\bfseries dimension}\index{Dimension} of $[s]$, denoted $\dim \, [s]$,
is $k$.
\end{defn}

The reader should prove that $[s]$ is the smallest
convex set containing $s$.

\begin{defn}
The coefficients $a_j$ in the expression
$\displaystyle \VEC{x} = \sum_{j=0}^k a_j\VEC{x}_j$ with $a_j \geq 0$ and
$\displaystyle \sum_{j=0}^k a_k = 1$ are called
{\bfseries barycentric coordinates}\index{Barycentric Coordinates}
of $x$.
\end{defn}

\begin{defn}
Let $s = \{\VEC{x}_0,\VEC{x}_1,\VEC{x}_2,\ldots, \VEC{x}_k\}$ be a
convex independent subset of $\displaystyle \RR^n$.
The set $\displaystyle \left\{ \sum_{j=0}^k a_j\VEC{x}_j : a_j > 0
\ \text{and} \ \sum_{j=0}^k a_k = 1 \right\}$, denoted $(s)$ or
$(\VEC{x}_0,\VEC{x}_1,\VEC{x}_2, \ldots, \VEC{x}_k)$,
is called an {\bfseries open $\mathbf{k}$-simplex}\index{Open $k$-Simplex}.
The {\bfseries dimension}\index{Dimension} of $(s)$, denoted $\dim\, (s)$,
is $k$.
\end{defn}

\begin{defn}
The {\bfseries open faces}\index{Open Face} of the closed $k$-simplex
$[\VEC{x}_0,\VEC{x}_1,\ldots,\VEC{x}_k]$ are the
open $m$-simplices
$(\VEC{x}_{j_0},\VEC{x}_{j_1},\VEC{x}_{j_2},\ldots, \VEC{x}_{j_m})$
for $\{ j_0, j_1, \ldots, j_m\} \subset \{0,1,2,\ldots, k\}$
and $0\leq m \leq k$.
\end{defn}

A closed $k$-simplex $[\VEC{x}_0,\VEC{x}_1,\ldots,\VEC{x}_k]$ is the
union of all its open faces (Figure~\ref{SimpCpx1}).  The reader must
realize that the closed $0$-complex $[\VEC{x}_j]$ is equal to the open
$0$-complex $(\VEC{x}_j)$.  If $(s_1)$ and $(s_2)$ are two open faces
of a closed $k$-simplex, then either $(s_1) = (s_2)$ or
$(s_1) \cap (s_2) = \emptyset$.

\pdfF{alg_top/simpcpx1}{Closed $3$-simplex}{Illustration of a closed
$3$-simplex
$\displaystyle [\VEC{x}_0,\VEC{x}_1,\VEC{x}_2,\VEC{x}_3] \subset \RR^3$.
This closed simplex is the union of the open faces
$(\VEC{x}_0,\VEC{x}_1,\VEC{x}_2,\VEC{x}_3)$, $(\VEC{x}_0,\VEC{x}_1,\VEC{x}_2)$,
$(\VEC{x}_0,\VEC{x}_1,\VEC{x}_3)$,
$(\VEC{x}_0,\VEC{x}_2,\VEC{x}_3)$, $(\VEC{x}_1,\VEC{x}_2,\VEC{x}_3)$,
$(\VEC{x}_0,\VEC{x}_1)$, $(\VEC{x}_0,\VEC{x}_2)$, $(\VEC{x}_0,\VEC{x}_3)$,
$(\VEC{x}_1,\VEC{x}_2)$, $(\VEC{x}_1,\VEC{x}_3)$, $(\VEC{x}_2,\VEC{x}_3)$,
$(\VEC{x}_0)$, $(\VEC{x}_1)$, $(\VEC{x}_2)$ and $(\VEC{x}_3)$.}{SimpCpx1}

\begin{defn}
A {\bfseries simplicial complex}\index{Simplicial Complex} $K$ in
$\displaystyle \RR^n$ is a finite set of open simplices such that:
\begin{enumerate}
\item If $(s)$ is an open simplices in $K$, then all the open faces of
the closed simplex $[s]$ are also in $K$.
\item If $(s_1)$ and $(s_2)$ are two open simplices in $K$, then
either $(s_1) = (s_2)$ or $(s_1) \cap (s_2) = \emptyset$.
\end{enumerate}
The {\bfseries dimension}\index{Dimension} of the simplicial complex
$K$, denoted $\dim K$, is defined as the dimension of the largest open
simplex in $K$.  The union of the open simplices in $K$ is the subset
of $\displaystyle \RR^n$ denoted $[K]$.
The {\bfseries vertices}\index{Vertices} of $K$ are the elements
$\VEC{x} \in [K]$ such that $(\VEC{x}) \in K$; namely, $(\VEC{x})$ is a
$0$-simplex in $K$.
\end{defn}

We have that $\displaystyle [K] = \bigcup_{(s) \in K} (s)
= \bigcup_{(s) \in K} [s]$ and $[K]$ is a compact subsets of
$\displaystyle \RR^n$ because it is a finite union of the compact sets
$[s]$ for $(s) \in K$.

\pdfF{alg_top/simpcpx2}{Simplicial complex}{The simplicial complex
$K$ represented by the set $\displaystyle [K] \subset \RR^3$ in the
figure above is the union of the set of all the open faces of the
closed $3$-simplex $[\VEC{x}_0,\VEC{x}_1,\VEC{x}_2,\VEC{x}_3]$ listed in
Figure~\ref{SimpCpx1} with the set
$\displaystyle \left\{ (\VEC{x}_0,\VEC{x}_4,\VEC{x}_5),
(\VEC{x}_0,\VEC{x}_4), (\VEC{x}_0,\VEC{x}_5), (\VEC{x}_4,\VEC{x}_5),
(\VEC{x}_4), (\VEC{x}_5) \right\}$.  The dimension of $K$ is $3$.}{SimpCpx2}

All the following notions will be very useful to study simplicial
complexes.

\begin{defn}
A subset $L$ of a simplicial complex $K$ is called a
{\bfseries subcomplex}\index{Subcomplex} of $K$ if $L$ is itself a
simplicial complex.
\end{defn}

\begin{defn}
Let $K$ be a simplicial complex of dimension $k$.  Given
$0 \leq r \leq k$, the {\bfseries $\mathbf{r}$-skeleton}\index{$r$-Skeleton} 
of $K$, denoted $\displaystyle K^r$, is the set
$\displaystyle K^r = \left\{ (s) : (s) \in K \ \text{and}\ \dim\, (s)
\leq r \right\}$.
\end{defn}

Since a closed $k$-simplex $[s]$ is the union of all its open faces,
we may treat $s$ as the set of all the open faces of $[s]$; namely,
we may treat $s$ as a simplicial complex.  Note that all the open
faces of $[s]$ are open $m$-simplices with $m \leq \dim\, [s]$.
By extension, the {\bfseries $r$-skeleton}\index{$r$-Skeleton} of $s$ for
$0 \leq r \leq k$, denoted $\displaystyle s^r$, is the set of all open
faces of $[s]$ corresponding to open $m$-simplex with $m \leq r$.
Moreover, $\displaystyle [s^r]$ will denote the union
of all the open simplices in $\displaystyle s^r$.

\begin{defn}
Given $\displaystyle \VEC{x} \in \RR^n$ and
$\displaystyle A \subset \RR^n$, the pair $(\VEC{x},A)$ is in
{\bfseries general position}\index{General Position} if
$\VEC{x} \not\in A$ and, for any two distinct points
$\VEC{a}_1$ and $\VEC{a}_2$ in $A$, we have that
$[\VEC{x},\VEC{a}_1] \cap [\VEC{x},\VEC{a}_2] = \{ \VEC{x}\}$.
If $(\VEC{x},A)$ is in general position, then the {\bfseries cone}\index{Cone}
with base $A$ and vertex $\VEC{x}$, denoted $\VEC{x} \ast A$, is the set
$\displaystyle \VEC{x} \ast A = \bigcup_{\VEC{a} \in A} [\VEC{x},\VEC{a}]$.

Given a simplicial complex $K$, the {\bfseries cone}\index{Cone}
$\VEC{x} \ast K$ is the simplicial
complex consisting of all the open simplices of the form
$(\VEC{x},\VEC{x}_0, \VEC{x}_1, \ldots , \VEC{x}_k)$ with
all their faces for all
$(\VEC{x}_0, \VEC{x}_1, \ldots , \VEC{x}_m) \in K$ with $m \leq \dim K$.
\end{defn}

The closed $k$-simplex
$[s] = [\VEC{x}_0,\VEC{x}_1, \VEC{x}_2, \ldots, \VEC{x}_k]$ can be
expressed as $\displaystyle [s] = \VEC{x} \ast [s^{k-1}]$ for any
$\VEC{x} \in (s)$.

\subsection{Barycentric Subdivisions} \label{ssectBarSubd}

\begin{defn}
The {\bfseries barycentre}\index{Barycentre} of an open $k$-simplex
$(s) = (\VEC{x}_0, \VEC{x}_1, \ldots, \VEC{x}_k)$ is the point
$\VEC{b}_{(s)} \in (s)$ defined by
$\displaystyle \VEC{b}_{(s)} = (k+1)^{-1} \sum_{j=0}^k \VEC{x}_j$.
\end{defn}

\begin{defn}
A {\bfseries subdivision}\index{Subdivision} of a simplicial complex
$K$ is another simplicial complex $L$ such that $[L] = [K]$ and, if
$(s) \in L$, then $(s)$ is a subset of an open simplex in $K$.
\end{defn}

\pdfF{alg_top/simpcpx3}{Subdivision of a simplicial complex}{
(b) and (c) illustrate two possible subdivisions of the simplicial
complex illustrated in (a).  (b) is not a subdivision of the
simplicial complex illustrated in (c) and vice-versa.}{SimpCpx3}

\begin{defn}
A {\bfseries partial order}\index{Partial order} on a simplicial
complex $K$ is defined as follows.  Given $(s_1)$ and $(s_2)$ in $K$,
we write that $(s_1) \leq (s_2)$ if $(s_1)$ is an open face of
$[s_2]$.  If in addition $(s_1) \neq (s_2)$, then we write that
$(s_1) < (s_2)$.
\end{defn}

The following result will play a fundamental role in the ``simplicial
approximation'' of ``simplicial function'' in the next subsection.
Its proof is not too complicated but it is long (see \cite{ST}).  It
uses the concept of general position introduced above.

\begin{theorem}
Suppose that $K$ is a simplicial complex in $\displaystyle \RR^n$.  Let
$\displaystyle K^{[1]}$ be the simplicial complex defined by
\begin{align*}
K^{[1]} & = \left\{ (\VEC{b}_{(s_0)}, \VEC{b}_{(s_1)},\ldots, \VEC{b}_{(s_m)}) :
(s_1) < (s_2) < \ldots < (s_m) \ \text{and} \right. \\
&\hspace{13em} \left.  \ (s_j) \in K \ \text{for}
\ 0 \leq j \leq m \leq \dim K  \right\} \ .
\end{align*}
Then $\displaystyle K^{[1]}$ is a subdivision of $K$
(Figure~\ref{SimpCpx4}).  Moreover, we have that \\
$(\VEC{b}_{(s_0)}, \VEC{b}_{(s_1)},\ldots, \VEC{b}_{(s_m)})$ is a
subset of $(s_m)$ for all sequences
$(s_1) < (s_2) < \ldots < (s_m)$ of open simplices in $K$.
\end{theorem}

\pdfF{alg_top/simpcpx4}{First barycentric subdivision}{We have
illustrated some of the open simplices in the simplicial complex
$\displaystyle s^{[1]}$ where $s$ is the simplicial complex
associated to the closed $2$-simplex $[s] = [x_0,x_1,x_2]$.  We have
drawn in blue the open $2$-simplex
$(\VEC{b}_{(s_1)}, \VEC{b}_{(s_2)}, \VEC{b}_{(s_3)})$ for $(s_1) = (\VEC{x}_1)$,
$(s_2) = (\VEC{x}_1,\VEC{x}_2)$ and $(s_3) = (s)$.  We effectively
have that $(s_1) < (s_2) < (s_3)$.  We have drawn in red the
open $1$-simplex $(\VEC{b}_{(s_4)}, \VEC{b}_{(s_3)})$ for
$(s_4) = (\VEC{x}_2)$ and $(s_3)$.  Again, we have that $(s_4) < (s_3)$.
Note that the open $0$-simplex $(\VEC{b}_{(t)})$ is associated to a
$0$-simplex $(t)$ is $(t)$ itself.}{SimpCpx4}

It follows from the definition of $\displaystyle K^{[1]}$ that
$\displaystyle \dim K^{[1]} = \dim K$.

\begin{defn} \label{defnBarySubd}
Let $K$ be a simplicial complex in $\displaystyle \RR^n$.
The simplicial complex $\displaystyle K^{[1]}$ is called the
{\bfseries first barycentric subdivision}\index{Barycentric
Subdivision!First Barycentric Subdivision} of $K$.  Inductively, we
may define the {\bfseries $\displaystyle \mathbf{m^{th}}$ barycentric
subdivision}\index{Barycentric Subdivision!$m^{th}$
Barycentric Subdivision} of $K$ as the simplicial complex
$\displaystyle K^{[m]} = (K^{[m-1]})^{[1]}$ for $m>1$.
\end{defn}

Recall that the diameter of a set $\displaystyle S \subset \RR^n$ is defined
by $\displaystyle \diam S = \inf_{\VEC{x},\VEC{y} \in S}
\|\VEC{x}- \VEC{y}\|$ where $\|\cdot\|$ is a norm on
$\displaystyle \RR^n$, usually the Euclidean norm. 

\begin{lemma} \label{lemDiams}
If $[s] = [\VEC{x}_0,\VEC{x}_1,\VEC{x}_2,\ldots, \VEC{x}_k]$ is a
closed $k$-simplex, then $\diam\, [s] = \|\VEC{x}_{j_1}  - \VEC{x}_{j_2}\|$
for some $j_1,j_2 \in \{0,1,2,\ldots,k\}$.
\end{lemma}

\begin{proof}
Since $[s]$ is a compact set, there exist $\VEC{y}_1$ and $\VEC{y}_2$
in $[s]$ such that $\diam\, [s] = \|\VEC{y}_1 - \VEC{y}_2\|$.
Without loss of generality, suppose that $\VEC{y}_2$ is not a vertex.
Then we may express $\VEC{y}_2$ as
$\VEC{y}_2 = p \VEC{w}_1 + (1-p) \VEC{w}_2$ for some $0 < p < 1$ and
$\VEC{w}_1, \VEC{w}_2 \in [s]$.  Let
$g(t) = \|\VEC{y}_1 - t \VEC{w}_1 - (1-t) \VEC{w}_2\|$ for
$0 \leq t \leq 1$.  We have that $g$ is a convex function; namely,
$g(tq_1 + (1-t)q_2) \leq t g(q_1) + (1-t) g(q_2)$ for $0\leq q_1,q_2 \leq 1$
and $0 \leq t \leq 1$.  Here is a possible graph of $g$.
\pdfbox{alg_top/convex}
Thus $g$ reaches its maximum at one of the end points.  This implies
that $\diam\, [s] = g(p) < \max\{ g(0),g(1)\}
= \max \{ \|\VEC{y}_1 - \VEC{w}_1\|, \|\VEC{y}_1 -\VEC{w}_2\| \}$. 
This is a contradiction.
\end{proof}

\begin{defn}
Let $K$ be a simplicial complex.  The {\bfseries mesh}\index{Mesh}
of $K$ is defined as $\displaystyle \mesh K = \max_{(s) \in K} \diam\, [s]$.
\end{defn}

\begin{prop} \label{propMeshMBS}
Suppose that $K$ is a simplicial complex of dimension $k$.  Then
$\displaystyle \mesh K^{[1]} \leq (k/(k+1)) \mesh K$.
\end{prop}

\begin{proof}
By definitions of $\displaystyle K^{[1]}$ and the mesh of a simplicial
complex, there exists $(s_1) < (s_2) < \ldots < (s_m)$ in $K$ with $m \leq k$
such that
$\displaystyle \mesh K^{[1]}
= \diam\, [\VEC{b}_{(s_0)}, \VEC{b}_{(s_1)},\ldots, \VEC{b}_{(s_m)}]$.
It follows from Lemma~\ref{lemDiams} that
$\displaystyle \mesh^{[1]} K = \|\VEC{b}_{(s_p)} - \VEC{b}_{(s_q)}\|$ for some
$p,q \in \{0,1,\ldots,m\}$.  Without loss of generality, we may
re-index the vertices of $K$ and assume
that $p < q$, $(s_p) = (\VEC{x}_0,\VEC{x}_1,\ldots, \VEC{x}_p)$ and
$(s_q) = (\VEC{x}_0,\VEC{x}_1,\ldots, \VEC{x}_q)$.  Hence
\begin{align*}
&\mesh K^{[1]} = \|\VEC{b}_{(s_p)} - \VEC{b}_{(s_q)}\|
= \left\| \frac{1}{p+1} \sum_{j=0}^p \VEC{x}_j - 
\frac{1}{q+1} \sum_{i=0}^q \VEC{x}_i \right\| \\
&= \left\| \left(\frac{1}{p+1}- \frac{1}{q+1}\right) \sum_{j=0}^p \VEC{x}_j - 
\frac{1}{q+1} \sum_{i=p+1}^q \VEC{x}_i \right\|
= \left\| \frac{q-p}{(p+1)(q+1)}\sum_{j=0}^p \VEC{x}_j - 
\frac{1}{q+1} \sum_{i=p+1}^q \VEC{x}_k \right\| \\
&= \frac{1}{q+1}\left\| \frac{q-p}{p+1}\sum_{j=0}^p \VEC{x}_j - 
\sum_{i=p+1}^q \VEC{x}_i \right\|
= \frac{1}{q+1}\left\| \frac{q+1}{p+1}\sum_{j=0}^p \VEC{x}_j - 
\sum_{i=0}^q \VEC{x}_i \right\| \\
&= \frac{1}{q+1}\left\| \sum_{i=0}^q \left( \frac{1}{p+1}\sum_{j=0}^p
\VEC{x}_j - \VEC{x}_i \right) \right\|
= \frac{1}{(q+1)(p+1)} \left\| \sum_{i=0}^q \left( \sum_{j=0}^p
(\VEC{x}_j - \VEC{x}_i) \right) \right\| \\
&\leq \frac{1}{(q+1)(p+1)} \sum_{i=0}^q \sum_{j=0}^p
\left\|  \VEC{x}_j - \VEC{x}_i\right\| \ ,
\end{align*}
where we have used $(q-p) = (q+1) - (p+1)$ to get the fifth equality.
There are $(p+1)(q+1) - (p+1) = q(p+1)$ terms in the last sum because
$(p+1)$ of the terms are null, those for $i = j$.  Moreover
$\left\|  \VEC{x}_j - \VEC{x}_i\right\| \leq \mesh K$ for all $i$ and $j$.
thus
\[
\mesh K^{[1]} \leq \frac{q(p+1)}{(q+1)(p+1)} \mesh K 
= \frac{q}{q+1} \mesh K \leq \frac{k}{k+1} \mesh K
\]
because $q \leq k$.
\end{proof}

It follows from the previous proposition that
$\displaystyle \mesh K^{[m]} \leq \big( (k/(k+1)\big)^m \mesh K \to 0$
as $m \to \infty$.

\subsection{Simplicial Approximation} \label{ssectSA}

\begin{defn} \label{defnsimplMaps}
Let $K$ and $L$ be two simplicial complexes.  A
{\bfseries simplicial map}\index{Simplicial Map} between $K$ and $L$ is a
function $\phi :[K] \to [L]$ that satisfies the following conditions.
\begin{enumerate}
\item $\phi(\VEC{x})$ is a vertex of $L$ if $\VEC{x}$ is a vertex of $K$.
\item If $(s) = (\VEC{x}_0,\VEC{x}_1, \ldots, \VEC{x}_k)\in K$, then
there exists $(t) \in L$ such that $\phi(\VEC{x}_i) \in [t]$ for
$0 \leq i \leq k$. 
\item If $\VEC{x} \in (s) = (\VEC{x}_0,\VEC{x}_1, \ldots, \VEC{x}_k)$
and $\displaystyle \VEC{x} = \sum_{j=0}^k a_j \VEC{x}_j$ with $a_j >0$
and $\displaystyle \sum_{j=0}^k a_j = 1$, then 
$\displaystyle \phi(\VEC{x}) = \sum_{j=0}^k a_j \phi(\VEC{x}_j)$.
\end{enumerate}
\end{defn}

The set $\displaystyle \bigg\{ \phi(\VEC{x}) :
\displaystyle \VEC{x} = \sum_{j=0}^k a_j \VEC{x}_j \ \text{with}\ a_j >0
\ \text{and} \ \sum_{j=0}^k a_j = 1\bigg\}$ in (3) represents an
open face of $[t]$ mentioned in (2).  It may not be $(t)$ itself because
$\phi(\VEC{x}_i)$ for $0\leq i \leq k$ may not be all
the vertices of $[t]$.

\begin{defn} \label{defnStarV}
Let $K$ be a simplicial complex and $\VEC{x}$ be a vertex of
$K$.  The {\bfseries star}\index{Star} of $\VEC{x}$, denoted $\St(\VEC{x})$,
is the set defined by $\displaystyle
\St(\VEC{x}) = \bigcup_{\substack{(s) \in K\\\VEC{x} \in [s]}} (s)$
(Figure~\ref{Star1}).
\end{defn}

\pdfF{alg_top/star1}{Star of a vertex}{We have illustrated on the
left a simplicial complex $K$ where $[K]$ is the union of the
closed simplices $[\VEC{x}_0,\VEC{x}_1,\VEC{x}_6]$,
$[\VEC{x}_1,\VEC{x}_5,\VEC{x}_6]$, $[\VEC{x}_3,\VEC{x}_4,\VEC{x}_5]$, 
$[\VEC{x}_1,\VEC{x}_2]$ and $[\VEC{x}_2,\VEC{x}_3]$.  We have
illustrated on the right in blue the star of $\VEC{x}_1$.}{Star1}

Given a vertex $\VEC{x}$ of a simplicial complex $K$, we have that
$\St(\VEC{x})$ is open in the induced topology on $[K]$ from
$\displaystyle \RR^n$.  Moreover, $\VEC{x}$ is the only vertex of $K$ in
$\St(\VEC{x})$ and
$\displaystyle \{\St(\VEC{x}) : \VEC{x} \ \text{is a vertex of} \ K\}$
is an open cover of $[K]$ with respect to the induced topology
on $[K]$ from $\displaystyle \RR^n$.

\begin{defn}
Let $K$ and $L$ be two simplicial complexes.  A
simplicial map $\phi:[K]\to [L]$ is a {\bfseries simplicial
approximation}\index{Simplicial Approximation} of a continuous
function $f:[K] \to [L]$ if
$f( \St(\VEC{x}) ) \subset \St(\phi(\VEC{x}))$ for all vertices
$\VEC{x}$ of $K$.
\end{defn}

\begin{prop} \label{propNfmg}
Suppose that $K$ and $L$ are two simplicial complexes.
If $\phi:[K]\to [L]$ is a simplicial approximation of a continuous
function $f:[K] \to [L]$ and $\VEC{x} \in [K]$, then
$f(\VEC{x})$ and $\phi(\VEC{x})$ belongs to the same closed simple
$[t]$ with $(t) \in L$.
\end{prop}

\begin{proof}
Given $\VEC{x} \in [K]$, there exists an open simplex
$(s) = (\VEC{x}_0,\VEC{x}_1,\ldots, \VEC{x}_m) \in K$ such that
$\VEC{x} \in (s)$.  Since
$\VEC{x} \in (s) \subset \St(\VEC{x}_j)$ for all $0\leq j \leq m$, we
get from the definition of simplicial approximation that
$f(\VEC{x}) \in f(\St(\VEC{x}_j)) \subset \St(\phi(\VEC{x}_j))$
for $0 \leq j \leq m$.

There exists an open simplex $(t) \in L$ such that $f(\VEC{x}) \in (t)$.
Since $f(\VEC{x}) \in (t) \cap \St(\phi(\VEC{x}_j))$ for
$0 \leq j \leq m$, we get that
$(t) \cap \St(\phi(\VEC{x}_j)) \neq \emptyset$
for $0 \leq j \leq m$.  By definition of simplicial complex, we
have that $(t) \subset \St(\phi(\VEC{x}_j))$
for $0 \leq j \leq m$.  By definition of the star of a vertex, we
have that $\phi(\VEC{x}_j)$ for $0 \leq j \leq m$ are vertices of 
$[t]$.  Note that they may not be all the vertices of $[t]$.

Since $\VEC{x} \in (s)$, we have that
$\displaystyle \VEC{x} = \sum_{j=0}^m a_j \VEC{x}_j$ for some
$a_j> 0$ such that $\displaystyle \sum_{j=0}^m a_j = 1$.  Hence, we
get from the definition of simplicial map that
$\displaystyle \phi(\VEC{x}) = \sum_{j=0}^m a_j \phi(\VEC{x}_j)$ with
$a_j> 0$ such that $\displaystyle \sum_{j=0}^m a_j = 1$.  Thus
$\phi(\VEC{x}) \in \big(\phi(\VEC{x}_0), \phi(\VEC{x}_1), \ldots,
\phi(\VEC{x}_k)\big) \subset [t]$.  Note that we may not have that
$\big(\phi(\VEC{x}_0), \phi(\VEC{x}_1), \ldots, \phi(\VEC{x}_k)\big) = (t)$
because, as we said above, $\phi(\VEC{x}_j)$ for $0 \leq j \leq m$ may
not be all the vertices of $[t]$.  Therefore
$\big(\phi(\VEC{x}_0), \phi(\VEC{x}_1), \ldots, \phi(\VEC{x}_k)\big)$
may be only an open face of $[t]$ not equal to $(t)$ itself..
\end{proof}

This proposition implies that
$\displaystyle \|f - \phi\|_{[K]} = \max_{\VEC{x} \in [K]}
\|f(\VEC{x}) - \phi(\VEC{x}\| \leq \mesh L$.

\begin{prop}
Suppose that $K$ and $L$ are two simplicial complexes.
If $\phi:[K]\to [L]$ is a simplicial approximation of a simplicial map
$f:[K] \to [L]$, then $\phi = f$.
\end{prop}

\begin{proof}
For each vertex $\VEC{x} \in K$, we have that
$f(\VEC{x}) \in f(\St(\VEC{x})) \subset \St(\phi(\VEC{x}))$.
Since $f$ is a simplicial map, $f(\VEC{x})$ is a vertex of $L$.
Since $\St(\phi(\VEC{x}))$ contains only one vertex of $L$, namely
$\phi(\VEC{x})$, we get that $f(\VEC{x}) = \phi(\VEC{x})$.

Since $f$ and $\phi$ are simplicial maps that take the same value at
all the vertices of $K$, then it follows from the third condition of the
definition of simplicial maps, Definition~\ref{defnsimplMaps},
that $f = \phi$.
\end{proof}

The following theorem provides a necessary and sufficient condition to
create a simplicial approximation to a continuous function between two
simplicial complexes.

\begin{prop} \label{propExtSApprox}
Suppose that $K$ and $L$ are two simplicial complexes and that
$f:[K]\to [L]$ is a continuous function.  If $\phi$ is a map defined
on the set of vertices of $K$ with values in the set of vertices of
$L$, then $\phi$ can be extended to a simplicial approximation of
$f$ if and only if $f(\St(\VEC{x})) \subset \St(\phi(\VEC{x}))$ for
all vertices $\VEC{x}$ of $K$.
\end{prop}

\begin{proof}
Obviously, if $\phi$ can be extended to a simplicial approximation of
$f$, then we have from the definition of simplicial approximation that
$f(\St(\VEC{x})) \subset \St(\phi(\VEC{x}))$ for
all vertices $\VEC{x}$ of $K$.

Suppose that $f(\St(\VEC{x})) \subset \St(\phi(\VEC{x}))$ for
all vertices $\VEC{x}$ of $K$.  Consider a simplex
$(s) = (\VEC{x}_0,\VEC{x}_1,\ldots,\VEC{x}_m) \in K$.
Since $(s) \subset \St(\VEC{x}_j)$ for $0 \leq j \leq m$, we have
that $f\big( (s)\big) \subset f(\St(\VEC{x}_j)) \subset \St(\phi(\VEC{x}_j))$
for $0 \leq j \leq m$.  Thus $\displaystyle
\bigcap_{j=0}^m \St(\phi(\VEC{x}_j)) \neq \emptyset$.  Therefore,
there exists an open simplex $(t) \subset \St(\phi(\VEC{x}_j))$ for
$0 \leq j \leq m$.  We get that $\phi(\VEC{x}_j)$ for $0 \leq j \leq m$
are vertices of $[t]$.  Thus, we may define $\phi\big|_{[s]}$ by
$\displaystyle \phi\big|_{[s]}(\VEC{x}) = \sum_{j=0}^m a_j \phi(\VEC{x}_j)$ for
$\displaystyle \VEC{x} = \sum_{j=0}^k a_j \VEC{x}_j$ with $a_j >0$
and $\displaystyle \sum_{j=0}^k a_j = 1$.  In particular,
$\phi\big((s)\big) \subset [t]$.

Since $(s_1) = (s_2)$ or $(s_1) \cap (s_2) = \emptyset$ for all
simplices $(s_1)$ and $(s_2)$ of $K$, and $\phi\big|_{[s_1]}(\VEC{x})
= \phi\big|_{[s_2]}(\VEC{x})$ for $\VEC{x} \in [s_1]\cap [s_2]$,
we have that $\phi$ is a well defined simplicial map between $K$ and
$L$.
\end{proof}

\begin{theorem} \label{thESAf}
Suppose that $K$ and $L$ are two simplicial complexes and that
$f:[K]\to [L]$ is a continuous function.  Moreover, suppose that
$\{K_m\}_{m\in \NN}$ is a collection of subdivision of $K$ such that
$\mesh K_m \to 0$ as $m \to \infty$.  Then there exists $M>0$ such that
there exists a simplicial approximation $\phi_m:[K_n] \to [L]$ of $f$
for each $m \geq M$.
\end{theorem}

\begin{proof}
Since $\displaystyle \{\St(\VEC{y}) : \VEC{y} \ \text{is a vertex of} \ L\}$
is an open cover of $[L]$ with respect to the induced topology on
$[L]$ and $f:[K]\to [L]$ is continuous, we have that
$\displaystyle \{f^{-1}(\St(\VEC{y})) :
\VEC{y} \ \text{is a vertex of} \ L\}$ is an open cover of $[K]$ with
respect to the induce topology om $[K]$.  Since $[K]$ is compact,
there exists a number $\epsilon>0$, called the Lebesgue number, such
that any open ball in $[K]$ of radius less than or equal to $\epsilon$
is a subset $\displaystyle f^{-1}(\St(\VEC{y}))$ for some vertex
$\VEC{y}$ of $L$ \footnote{Since the number of vertices of $L$ is
finite, we could have proved the existence of $\epsilon$ without
making use of the theory of compact sets.}.

Choose $M$ such that $\mesh K_m < \epsilon/2$ for $m \geq M$.
Assume that $m \geq M$.  If ${s} \in K_m$, then
$\diam [s] \leq \mesh K_m < \epsilon/2$.  Hence, for each vertex
$\VEC{x}$ of $K_m$, we have that
$\displaystyle \St(\VEC{x}) \subset B_\epsilon(\VEC{x}) \subset 
f^{-1}(\St(\VEC{y}))$ for some vertex $\VEC{y}$ of $L$; namely,
\begin{equation} \label{scapproEq1}
  f(\St(\VEC{x})) \subset \St(\VEC{y})
\end{equation}
for some vertex $\VEC{y}$ of $L$.  If there are more then one vertex
$\VEC{y}$ of $L$ satisfying the previous statement, select one of
them.  Let $\phi_m(\VEC{x}) = \VEC{y}$.  This define a map from the set
of vertices of $K_m$ to the set of vertices of $L$.  We also have from
(\ref{scapproEq1}) that
$\displaystyle f(\St(\VEC{x})) \subset \St(\phi(\VEC{x}))$.
We can then use Proposition~\ref{propExtSApprox} to get a simplicial
approximation $\phi_m:[K_m] \to [L]$ of $f$.
\end{proof}

The previous theorem ensures that we can always find a simplicial
approximation as closed as we want to a continuous function between
two simplicial complexes.

\begin{cor}
Suppose that $K$ and $L$ are two simplicial complexes and that
$f:[K]\to [L]$ is a continuous function.  Given $\epsilon >0$, there
exist subdivisions $\tilde{K}$ and $\tilde{L}$ of $K$ and $L$
respectively, and a simplicial approximation
$\phi:[\tilde{K}]\to [\tilde{L}]$ of $f$, such that
$\|f - \phi\|_{[K]} < \epsilon$.
\end{cor}

\begin{proof}
As remarked after Proposition~\ref{propMeshMBS}, if
$\displaystyle L_j = L^{[j]}$ for $\displaystyle j\in \NNp$, then we
get a sequence of barycentric subdivisions of $L$ such that
$\mesh L_j \to 0$ as $j \to \infty$.  Thus there exists $j > 0$ such that
$\mesh L_j < \epsilon$.

Similarly, if $\displaystyle K_m = K^{[m]}$ for
$\displaystyle m\in \NNp$, then we get a sequence of barycentric
subdivisions of $M$ such that $\mesh K_m \to 0$ as
$m \to \infty$.  If we apply Theorem~\ref{thESAf} to
$f:[K] \to [L_j] = [L]$ using the collection of subdivisions
$\{K_m\}_{m\in \NNp}$ of $K$, then we find that there exists $m > 0$ and
a simplicial approximation $\phi:[K_m]\to [L_j]$ of $f$.
As remarked after Proposition~\ref{propNfmg}, we then have that
$\|f - \phi\|_{[K]} \leq \mesh L_j < \epsilon$.
We take $\tilde{K} = K_m$ and $\tilde{L} = L_j$.
\end{proof}

\subsection{Oriented Simplicial Homology} \label{subsOShomo}

From now on, when we refer to a simplicial simplex $K$, we assume that
a fixed indexing of the vertices of $K$ has been selected.  More
specifically, we assume that $\VEC{x}_0$, $\VEC{x}_1$, $\VEC{x}_2$,
\ldots are the vertices of $K$.

\begin{defn}
Let $[s]= [\VEC{x}_{j_0}, \VEC{x}_{j_1}, \ldots, \VEC{x}_{j_k}]$ be a
closed $k$-simplex.
An {\bfseries oriented $\mathbf{k}$-simplex}\index{Oriented $k$-simplex}
associated to $[s]$ is the pair $([s],\sigma)$ where
$\sigma: \{j_0,j_1,j_2,\ldots,j_k\} \to \{j_0,j_1,j_2,\ldots,j_k\}$
is a permutation.
\end{defn}

\begin{defn} \label{defnECordks}
Let $[s]= [\VEC{x}_{j_0}, \VEC{x}_{j_1}, \ldots, \VEC{x}_{j_k}]$
be a closed $k$-simplex.  We say that
two oriented $k$-simplices $([s],\sigma_1)$ and $([s],\sigma_2)$ are
{\bfseries equivalent}\index{Equivalent} if there exists an even
permutation $\beta:\{j_0,j_1,j_2,\ldots,j_k\} \to \{j_0,j_1,j_2,\ldots,j_k\}$
such that $\beta\circ \sigma_1 = \sigma_2$; namely,
$\beta(\sigma_1(j_i)) = \sigma_2(j_i)$ for $0 \leq i \leq k$.
\end{defn}

The notion of equivalence defined in the previous definition is an
equivalence relation that partitions the oriented $k$-simplices
associated to a closed $k$-simplex into two equivalence classes.

\begin{defn}
Let $[s] = [\VEC{x}_{j_0},\VEC{x}_{j_1}, \ldots, \VEC{x}_{j_k}]$
be a closed $k$-simplex.  We use the notations
$\os{\VEC{x}_{\sigma(j_0)}}{}{\VEC{x}_{\sigma(j_1)}}{}{\VEC{x}_{\sigma(j_k)}}$
or simply $\os{s_\sigma}{}{}{}{}$ to denote the equivalence class
associated to $([s],\sigma)$.
\end{defn}

If $[s]=[\VEC{x}_{j_0}, \VEC{x}_{j_1}, \ldots, \VEC{x}_{j_k}]$, then the
two equivalence classes of oriented $k$-simplices associate to $[s]$ are
$\os{\VEC{x}_{j_0}}{\VEC{x}_{j_1}}{\VEC{x}_{j_2}}{}{\VEC{x}_{j_k}}$ and
$\os{\VEC{x}_{j_1}}{\VEC{x}_{j_0}}{\VEC{x}_{j_2}}{}{\VEC{x}_{j_k}}$.
Note that
$\os{\VEC{x}_{j_0}}{\VEC{x}_{j_1}}{\VEC{x}_{j_2}}{}{\VEC{x}_{j_k}}$
is the equivalence class associated to $([s],\sigma_1)$ with
$\sigma_1 = \Id$ and
$\os{\VEC{x}_{j_1}}{\VEC{x}_{j_0}}{\VEC{x}_{j_2}}{}{\VEC{x}_{j_k}}$
is the equivalence class associated to $([s],\sigma_2)$ with
$\sigma_2$ is defined by
\[
\sigma_2(j_m) = \begin{cases}
j_1 & \quad \text{if} \ m = 0 \\
j_0 & \quad \text{if} \ m = 1 \\
j_m & \quad \text{otherwise}
\end{cases}
\]

\begin{defn}
Let $K$ be a simplicial complex and $R$ be an integral domain.
The {\bfseries module of $\mathbf{k}$-chains}\index{Module of $k$-Chains}
of $K$, denoted $C_k(K;R)$, is the free abelian group generated by the
equivalence classes of oriented $k$-simplices $\os{s_\sigma}{}{}{}{}$ for all
$(s) \in K$ where, given $(s) = (\VEC{x}_{j_0},\VEC{x}_{j_1},
\ldots, \VEC{x}_{j_k})$, we set
$\os{s_{\sigma_1}}{}{}{}{} = - \os{s_{\sigma_2}}{}{}{}{}$ if
$\beta\, \circ\, \sigma_1 = \sigma_2$ for an odd permutation
$\beta:\{j_0,j_1,j_2,\ldots,j_k\} \to \{j_0,j_1,j_2,\ldots,j_k\}$.
The elements of
$C_k(K;R)$ are called {\bfseries $\mathbf{k}$-chains}\index{$k$-Chain}
of $K$.
\end{defn}

For us, the integral domain $R$ will either be $\ZZ$ or $\RR$.

The elements $c \in C_k(K;R)$ are finite sums of the form
$\displaystyle c = \sum_{\substack{(s) \in K\\\dim(s)=k}} a_{(s)}
\os{s}{}{}{}{}{}$ with $a_{(s)} \in R$, or more explicitly
\[
  c = \sum_{(\VEC{x}_{j_0},\VEC{x}_{j_1},\ldots,\VEC{x}_{j_k}) \in K}
a_{(\VEC{x}_{j_0},\VEC{x}_{j_1},\ldots,\VEC{x}_{j_k})}
\os{\VEC{x}_{j_0}}{}{\VEC{x}_{j_1}}{}{\VEC{x}_{j_k}}
\]
for $a_{(\VEC{x}_{j_0},\VEC{x}_{j_1},\ldots,\VEC{x}_{j_k})}\in R$.

\begin{defn}
Let $K$ be a simplicial complex and $R$ be an integral domain.
If $(s) = (\VEC{x}_{j_0},\VEC{x}_{j_1},\ldots, \VEC{x}_{j_k}) \in K$
with $k>0$, then
the {\bfseries boundary}\index{Boundary} of $\os{s}{}{}{}{} \in C_k(K;R)$,
denoted $\partial_k \os{s}{}{}{}{}$, is defined by
\[
\partial_k \os{s}{}{}{}{} = \sum_{i=0}^k (-1)^i
\os{\VEC{x}_{j_0}}{}{\VEC{x}_{j_1}}{\widehat{\VEC{x}_{j_i}}}{\VEC{x}_{j_k}}
\in C_{k-1}(K;R) \ .
\]
(Figure~\ref{SimpHm1}).  If $(s) = (\VEC{x}_j) \in K$, then
$\partial_0 \os{\VEC{x}_j}{}{}{}{} = 0$.
\end{defn}

\pdfF{alg_top/simphm1}{Boundary of $k$-chains}{Examples of the
boundary of $k$-chains for the simplest $k$-chains.  For the $1$-chain
$\os{\VEC{x}_0}{}{}{}{\VEC{x}_1}$, we have that
$\partial_1 \os{\VEC{x}_0}{}{}{}{\VEC{x}_1} =
\os{\VEC{x}_0}{}{}{}{} - \os{\VEC{x}_1}{}{}{}{}$.
For the $2$-chain $\os{\VEC{x}_0}{\VEC{x}_1}{}{}{\VEC{x}_3}$, we have that
$\partial_2 \os{\VEC{x}_0}{\VEC{x}_1}{}{}{\VEC{x}_3} =
\os{\VEC{x}_1}{}{}{}{\VEC{x}_2} - \os{\VEC{x}_0}{}{}{}{\VEC{x}_2}
+ \os{\VEC{x}_0}{}{}{}{\VEC{x}_1} = 
\os{\VEC{x}_0}{}{}{}{\VEC{x}_1} + \os{\VEC{x}_1}{}{}{}{\VEC{x}_2}
+ \os{\VEC{x}_2}{}{}{}{\VEC{x}_0}$
as we have illustrated in the figure above.}{SimpHm1}

\begin{defn}
Let $K$ be a simplicial complex and $R$ an integral domain.
The {\bfseries boundary map}\index{Boundary Map}
$\partial_k: C_k(K;R) \to C_{k-1}(K;R)$ is define by
$\displaystyle \partial_k \bigg( \sum_{\substack{(s) \in K\\ \dim(s) = k}}
a_{(s)}\os{s}{}{}{}{} \bigg)
= \sum_{\substack{(s) \in K\\ \dim(s) =k}}  a_{(s)} \partial_k  \os{s}{}{}{}{}$
or more explicitly
\begin{align*}
&\partial_k \bigg( \sum_{(\VEC{x}_{j_0},\VEC{x}_{j_1},\ldots,\VEC{x}_{j_k}) \in K}
  a_{(\VEC{x}_{j_0},\VEC{x}_{j_1},\ldots,\VEC{x}_{j_k})}
  \os{\VEC{x}_{j_0}}{}{\VEC{x}_{j_1}}{}{\VEC{x}_{j_k}} \bigg) \\
&\hspace{7em} = \sum_{(\VEC{x}_{j_0},\VEC{x}_{j_1},\ldots,\VEC{x}_{j_k}) \in K}
  a_{(\VEC{x}_{j_0},\VEC{x}_{j_1},\ldots,\VEC{x}_{j_k})}
\partial_k  \os{\VEC{x}_{j_0}}{}{\VEC{x}_{j_1}}{}{\VEC{x}_{j_k}} \ .
\end{align*}
\end{defn}

A simple computation yields $\partial_{k-1} (\partial_k(c)= 0$ for all
$k$-chains $c$.  Note that $\partial_{k-1} (\partial_k(c)$ is the sum
of expressions of the form
\begin{align*}
&(-1)^{j_s+j_r}a_{\osscript{\VEC{x}_0}{}{\VEC{x}_1}{}{\VEC{x}_k}}
\os{\VEC{x}_{j_0}}{\VEC{x}_{j_1}}
{\ldots,\widehat{\VEC{x}_{j_r}}}{\widehat{\VEC{x}_{j_s}}}{\VEC{x}_{j_k}} \\
&\qquad\qquad + (-1)^{j_s+j_r-1}a_{\osscript{\VEC{x}_0}{}{\VEC{x}_1}{}{\VEC{x}_k}}
\os{\VEC{x}_{j_0}}{\VEC{x}_{j_1}}
{\ldots,\widehat{\VEC{x}_{j_r}}}{\widehat{\VEC{x}_{j_s}}}{\VEC{x}_{j_k}}
\end{align*}
that cancel each other.

\begin{defn}
Let $K$ be a simplicial complex and $R$ an integral domain,\\
$Z_k(K;R) = \{ c \in C_k(K;R) : \partial_k(c) = 0 \}$
and $B_k(K;R) = \{ \partial_{k+1}(c) : c \in C_{k+1}(K;R) \}$.
The {\bfseries (simplicial) $\displaystyle \mathbf{k^{th}}$ homology
module}\index{Homology Module!Simplicial $k^{th}$ Homology Module} of $K$ is the
free abelian quotient group $H_k(K;R) = Z_k(K;R) / B_k(K;R)$.  The
elements of $Z_k(K;R)$ are called {\bfseries cycles}\index{Cycle} and those of
$B_k(K;R)$ are called {\bfseries boundaries}\index{Boundary}.
The equivalence class of $H_k(K;R)$ associated to $c \in Z_k(K;R)$ is
denoted $[c]_K$.
\end{defn}

Suppose that $K$ and $L$ are two simplicial complexes and that
$f:[K] \to [L]$ is a homeomorphism that maps open simplices in $K$ to
open simplices in $L$.  The map $C(f): C_k(K;R) \to C_k(L;R)$ defined by
$\displaystyle C(f)\Big( \sum_{\substack{(s) \in K\\ \dim(s) =k}}
a_{(s)}\os{s}{}{}{}{} \Big)
= \sum_{\substack{(s) \in K\\ \dim(s) =k}} a_{(s)} \os{f((s))}{}{}{}{}$
is an isomorphism.
Since $C(f)$ commutes with $\partial_k$ for all $k$, we may use $C(f)$
to induce a map $H(f): H_k(K;R) \to H_k(L,R)$ because
$C(f)(Z_k(K;R)) \subset Z_k(L;R)$ and 
$C(f)(B_k(K;R)) \subset B_k(L;R)$.

\begin{prop} \label{propK0KZ}
Let $K$ be a simplicial complex and $R$ be an integral domain.  The
dimension of $H_0(K;R)$ is equal to the number of (connected) components
of $[K]$.  Namely, the number of independent generators of $H_0(K;R)$
is equal to the number of (connected) components of $[K]$.
\end{prop}

\begin{proof}[Proof (Sketch)]
Since $Z_0(K;R) = C_0(K;R)$ and all elements of $C_0(K;R)$
are of the form $\displaystyle c = \sum_{(\VEC{x}_j) \in K} a_{(\VEC{x}_j)}
\os{\VEC{x}_j}{}{}{}{}$ with $a_{(\VEC{x}_j)} \in R$,  it therefore
suffices to consider the linear independence of the equivalence
classes in $H_0(K;R)$ associated to
$\os{\VEC{x}_j}{}{}{}{} \in C_0(K;R)$ for $(\VEC{x}_j) \in K$ to
determine the dimension of $H_0(K;R)$.

If $(\VEC{x}_{j_0},\VEC{x}_{j_1}) \in K$, then
$\partial_1 \os{\VEC{x}_{j_0}}{}{}{}{\VEC{x}_{j_1}} = \os{\VEC{x}_{j_1}}{}{}{}{}
- \os{\VEC{x}_{j_0}}{}{}{}{}$.  Hence
$\os{\VEC{x}_{j_0}}{}{}{}{}$ and $\os{\VEC{x}_{j_1}}{}{}{}{}$ are
in the same equivalence classes of $H_0(K;R)$.
Inductively, we find that two equivalence classes of
$H_0(K;R)$, one associated to $\os{\VEC{x}_{j_0}}{}{}{}{}$ and one associated
$\os{\VEC{x}_{j_1}}{}{}{}{}$, are distinct if there is no sequence
$(x_{j_0},x_{k_1})$, $(x_{k_1},x_{k_2})$, $(x_{k_2},x_{k_3})$, \ldots,
$(x_{k_{m-1}},x_{k_m})$, $(x_{k_m},x_{j_1})$ of open $1$-simplices in
$K$.  This is only possible if $\VEC{x}_{j_0}$ and $\VEC{x}_{j_1}$
belongs to two distinct connected components of $[K]$.
\end{proof}

\begin{egg}
We consider the simplicial complex
\[
K = \{ (\VEC{x}_0), (\VEC{x}_1),
(\VEC{x}_2), (\VEC{x}_0,\VEC{x}_1), (\VEC{x}_1,\VEC{x}_2),
(\VEC{x}_0,\VEC{x}_2), (\VEC{x}_0,\VEC{x}_1,\VEC{x}_3) \} \ .
\]

\stage{a} It follows from Proposition~\ref{propK0KZ} that
$\displaystyle H_0(K^1;\ZZ) \cong \ZZ$.

\stage{b} We have
\begin{align*}
C_1(K^1;\ZZ) &= \left\{ a_{(\VEC{x}_0,\VEC{x}_1)}
\os{\VEC{x}_0}{}{}{}{\VEC{x}_1}
+ a_{(\VEC{x}_1,\VEC{x}_2)} \os{\VEC{x}_1}{}{}{}{\VEC{x}_2}
+ a_{(\VEC{x}_0,\VEC{x}_2)} \os{\VEC{x}_0}{}{}{}{\VEC{x}_2} : \right. \\
&\hspace{7em} \left. a_{(\VEC{x}_0,\VEC{x}_1)}, a_{(\VEC{x}_1,\VEC{x}_2)},
a_{(\VEC{x}_0,\VEC{x}_2)} \in \ZZ \right\}
\cong \ZZ^3
\end{align*}
and
\begin{align*}
&\partial_1 \left(a_{(\VEC{x}_0,\VEC{x}_1)} \os{\VEC{x}_0}{}{}{}{\VEC{x}_1}
+ a_{(\VEC{x}_1,\VEC{x}_2)} \os{\VEC{x}_1}{}{}{}{\VEC{x}_2}
+ a_{(\VEC{x}_0,\VEC{x}_2)} \os{\VEC{x}_0}{}{}{}{\VEC{x}_2} \right) \\
&\quad = a_{(\VEC{x}_0,\VEC{x}_1)}
\left(\os{\VEC{x}_1}{}{}{}{} - \os{\VEC{x}_0}{}{}{}{} \right)
+ a_{(\VEC{x}_1,\VEC{x}_2)}
\left(\os{\VEC{x}_2}{}{}{}{} - \os{\VEC{x}_1}{}{}{}{} \right)
+ a_{(\VEC{x}_0,\VEC{x}_2)}
\left(\os{\VEC{x}_2}{}{}{}{} - \os{\VEC{x}_0}{}{}{}{} \right) \\
&\quad = \left( a_{(\VEC{x}_0,\VEC{x}_1)} - a_{(\VEC{x}_1,\VEC{x}_2)} \right)
\os{\VEC{x}_1}{}{}{}{}
- \left( a_{(\VEC{x}_0,\VEC{x}_2)} + a_{(\VEC{x}_0,\VEC{x}_1)} \right)
\os{\VEC{x}_0}{}{}{}{}
+ \left(a_{(\VEC{x}_1,\VEC{x}_2)} + a_{(\VEC{x}_0,\VEC{x}_2)} \right)
\os{\VEC{x}_2}{}{}{}{} = 0
\end{align*}
if and only if
$a_{(\VEC{x}_0,\VEC{x}_1)} = a_{(\VEC{x}_1,\VEC{x}_2)} 
= - a_{(\VEC{x}_0,\VEC{x}_2)}$.  Hence
\[
Z_1(K^1;\ZZ)
= \left\{ a \left( \os{\VEC{x}_0}{}{}{}{\VEC{x}_1}
+ \os{\VEC{x}_1}{}{}{}{\VEC{x}_2}
- \os{\VEC{x}_0}{}{}{}{\VEC{x}_2} \right) : a \in \ZZ \right\}
\cong \ZZ \ .
\]
We also have that $\displaystyle B_1(K^1;\ZZ) = \{ 0 \}$ because
$\displaystyle C_2(K^1;\ZZ) = \{ 0 \}$.  Therefore
$\displaystyle H_1(K^1;\ZZ) = Z_1(K^1;\ZZ) / B_1(K^1;\ZZ) \cong \ZZ$.

\stage{c} It follows from Proposition~\ref{propK0KZ} that
$H_0(K;\ZZ) \cong \ZZ$. 

\stage{d} As before, we have
\[
Z_1(K;\ZZ)
= \left\{ a \left( \os{\VEC{x}_0}{}{}{}{\VEC{x}_1}
+ \os{\VEC{x}_1}{}{}{}{\VEC{x}_2}
- \os{\VEC{x}_0}{}{}{}{\VEC{x}_2} \right) : a \in \ZZ \right\}
\cong \ZZ \ .
\]
We have
\[
C_2(K;\ZZ) = \left\{ a \os{\VEC{x}_0}{\VEC{x}_2}{}{}{\VEC{x}_3}
: a \in \ZZ \right\} \cong \ZZ \ .
\]
Since
\begin{equation} \label{eggHkKZ2}
\partial_2 \left(a \os{\VEC{x}_0}{\VEC{x}_1}{}{}{\VEC{x}_2} \right)
= a \left(\os{\VEC{x}_1}{}{}{}{\VEC{x}_2} - \os{\VEC{x}_0}{}{}{}{\VEC{x}_2}
+ \os{\VEC{x}_0}{}{}{}{\VEC{x}_1} \right)
\end{equation}
for $a \in \ZZ$, we get
\[
B_1(K;\ZZ) = \left\{
a \left(\os{\VEC{x}_1}{}{}{}{\VEC{x}_2} - \os{\VEC{x}_0}{}{}{}{\VEC{x}_2}
+ \os{\VEC{x}_0}{}{}{}{\VEC{x}_1} \right) : a \in \ZZ \right\}
= Z_1(K^2;\ZZ) \cong \ZZ \ .
\]
Thus $H_1(K;\ZZ) = Z_1(K;\ZZ) / B_1(K;\ZZ) = 0$.

\stage{e} Finally, it follows from (\ref{eggHkKZ2}) that
$\partial_3 \left( a \os{\VEC{x}_1}{\VEC{x}_2}{}{}{\VEC{x}_3} \right) = 0$
if and only if $a = 0$.  So $Z_2(K;\ZZ) = \{ 0 \}$ and therefore
$H_2(K;\ZZ) = 0$.
\end{egg}

If $R = \RR$, then $C_k(K,\RR)$, $B_k(K,\RR)$ can be treated as
vector spaces over $\RR$.

We now define the Euler characteristic of a simplicial complex and
deduce some very interesting properties associated to
the Euler characteristic of a simplicial complex.

\begin{defn}   \label{defnBettiEulerC}
Let $K$ be a simplicial complex.
The {\bfseries $\displaystyle \mathbf{j^{th}}$ Betti
number}\index{$\displaystyle j^{th}$ Betti Number}
of $K$, denoted $\beta_j$, is the non-negative integer defined by
$\beta_j = \dim(H_j(K;\RR))$.
The {\bfseries Euler characteristic}\index{Euler Characteristic} of $K$,
denote $\Chi(K)$, is the integer defined by
$\displaystyle \Chi(K) = \sum_{j=0}^{\dim(K)} (-1)^j \beta_j$.
\end{defn}

Some of the readers may have recognized the famous Euler
characteristic formula introduced in undergraduate geometry courses
\cite{C}. 
Suppose that $K$ is the triangulation of the unit sphere
$\displaystyle S^2 \subset \RR^3$.   The Euler characteristic of this
triangulation $K$ of $\displaystyle S^2$ is given by $V - E + F$ where
$V$ is the number of vertices, $E$ is the number of edges, and $F$ is
the number of faces (i.e. triangles) of $K$.  It is a 
value independent of the triangulation of $\displaystyle S^2$.

\begin{theorem} \label{thmEulerAlpha}
Let $K$ be a simplicial complex and $\alpha_j$ be the number of
open $j$-simplices in $K$.  Then
$\displaystyle \Chi(K) = \sum_{j=0}^{\dim(K)} (-1)^j \alpha_j$.
\end{theorem}

\begin{proof}
Let $C_{-1}(K;\RR) = \{ 0 \}$.
Since $\partial_0(\os{\VEC{x}_j}{}{}{}{}) = 0$ for all vertices $\VEC{x}_j$
of $K$, we may also set $B_{-1}(K;\RR) = \{ 0 \}$.  Since
$\partial_i: C_i(K,\RR) \to C_{i-1}(K;\RR)$ is a linear operator for
$0 \leq i \leq \dim(K)$, we have
\[
\alpha_i = \dim(C_i(K;\RR)) = \dim(\KE(\partial_i)) + \dim(\IMG(\partial_i))
= \dim(Z_i(K;\RR)) + \dim(B_{i-1}(K;\RR))
\]
for $0\leq i \leq \dim(K)$.  We also have
\[
\beta_i = \dim(H_i(K;\RR)) = \dim\left(Z_i(K;\RR)/B_i(K;\RR)\right)
= \dim(Z_i(K;\RR)) - \dim(B_i(K;\RR))
\]
for $0\leq i \leq \dim(K)$.  Hence
\begin{align*}
\Chi(K) &= \sum_{i=0}^{\dim(K)} (-1)^i \beta_i
= \sum_{i=0}^{\dim(K)} (-1)^i \left( \dim(Z_i(K;\RR)) -
\dim(B_i(K;\RR)) \right) \\
&= \sum_{i=0}^{\dim(K)} (-1)^i \dim(Z_i(K;\RR))
+ \sum_{i=0}^{\dim(K)} (-1)^{i+1} \dim(B_i(K;\RR)) \\
&= \sum_{i=0}^{\dim(K)} (-1)^i \dim(Z_i(K;\RR))
+ \sum_{i=1}^{1+\dim(K)} (-1)^i \dim(B_{i-1}(K;\RR)) \\
&= \big( \dim(Z_0(K;\RR)) + \underbrace{\dim(B_{-1}(K;\RR))}_{=0} \big) \\
&\qquad
+ \sum_{i=1}^{\dim(K)} (-1)^i \left( \dim(Z_i(K;\RR)) + \dim(B_{i-1}(K;\RR))
\right) \\
&\qquad + (-1)^{1+\dim(K)} \underbrace{\dim(B_{\dim(K)}(K;\RR))}_{=0} \\
&= \sum_{i=0}^{\dim(K)} (-1)^i \left( \dim(Z_i(K;\RR)) + \dim(B_{i-1}(K;\RR))
\right)
= \sum_{i=0}^{\dim(K)} (-1)^i \alpha_i \ .
\end{align*}
Note that
$\dim(B_{\dim(K)}(K;\RR)) = 0$ because $C_{1+\dim(K)}(K;\RR) = \{ 0 \}$.
\end{proof}

\subsection{Reduced and Relative Homology} \label{ssectSimplRDh}

Let $K$ be a simplicial complex and $R$ be an integral domain.
We have by definition that $\partial_0(c) = 0$ for all 
$c \in C_0(K;R)$.  Namely, we are assuming that the
image of $\partial_0$ is in the space $C_{-1}(K;R) = \{ 0 \}$.

We replace $C_{-1}(K;R) = \{0\}$ and the boundary operator
$\partial_0 : C_0(K;R) \to C_{-1}(K;R)$ 
to obtain a slightly new collection of homology modules.
Let
\[
C_k^\sharp(K;R) = \begin{cases}
R & \quad \text{if} \ k = -1 \\
C_k(K;R) & \quad \text{if} \ k \neq -1
\end{cases}
\]
Also, let $\displaystyle \partial_k^\sharp = \partial_k$ for $k \neq 0$ and
\[
\partial_0^\sharp \Big( \sum_{(\VEC{x}_j)\in K}
a_{(\VEC{x}_j)}\os{\VEC{x}_j}{}{}{}{} \Big)
= \sum_{(\VEC{x}_j) \in K} a_{(\VEC{x}_j)}
\]
for $a_{(\VEC{x}_j)} \in R$.
It is not difficult to prove that
$\displaystyle \partial_0^\sharp \circ \partial_1^\sharp= 0$.
We can then proceed as we did before to define
$\displaystyle Z_k^\sharp(X;R)$, $\displaystyle B_k^\sharp(X;R)$
and $\displaystyle H_k^\sharp(X;R)$.

\begin{defn}
Let $K$ be a simplicial complex and $R$ be an integral domain.
The quotient $\displaystyle H_k^\sharp(K;R) =
Z_k^\sharp(X;R) / B_k^\sharp(X;R)$ is called the
{\bfseries reduced (simplicial) $\displaystyle \mathbf{k^{th}}$ homology
module}\index{Homology Module!Reduced (Simplicial) $k^{th}$ Homology Module}
of $X$.
\end{defn}

Again, let $K$ be a simplicial complex and $R$ be an integral domain.
Suppose that $L$ is a simplicial subcomplex of $K$.  Recall that this
means that $L \subset K$ and $L$ is itself a simplicial complex.
Assuming that we keep the same indexing of the vertices of $L$ as the
one used for $K$, then $C_k(L;R)$ is a submodule of $C_k(K;R)$.

\begin{defn}
Let $L$ be a simplicial subcomplex of a simplicial complex $K$ and $R$
be an integral domain.
The {\bfseries module of relative chains}\index{Module of Relative Chains}
of $K \mod L$, denoted $C_k(K,L;R)$, is the free abelian group defined as
$\displaystyle C_k(K,L;R) = C_k(K;R)/C_k(L;R)$.
\end{defn}

Since $\partial_k:C_k(L;R) \to C_{k-1}(L;R)$, we may define 
the operator $\overline{\partial}_k:C_k(K,L;R) \to C_{k-1}(K,L;R)$
as $\overline{\partial}_k(\relC[K,L]{c}) = \relC[K,L]{\partial_k c}$ for all
$\relC[K,L]{c} \in C_k(K,L;R)$ where $\relC[K,L]{c}$ denotes the equivalence
class in $C_k(K,L;R)$ associated to $c \in C_k(K;R)$.
We get the following commutative diagram.
\[
\xymatrix@C+2ex{
C_k(K;R) \ar[d]^{\partial_k} \ar[r]^-{\pi_k} & C_k(K,L;R)
\ar[d]^{\overline{\partial}_k} \\
C_{k-1}(K;R) \ar[r]^-{\pi_{k-1}} & C_{k-1}(K,L;R)
}
\]
where $\pi_k:C_k(K;R) \to C_k(K,L;R)$ is the projection defined
by $\pi_k(c) = \relC[K,L]{c}$ for $c \in C_k(K;R)$.

\begin{defn}
Let $L$ be a simplicial subcomplex of a simplicial complex $K$ and $R$
be an integral domain.
The {\bfseries relative (simplicial) $\displaystyle \mathbf{k^{th}}$ homology
module}\index{Homology Module!Relative (Simplicial) $k^{th}$ Homology Module} of
$K \mod L$ is the free abelian group
$\displaystyle H_k(K,L;R) = \KE(\overline{\partial}_k)
/\IMG(\overline{\partial}_{k-1})$.
The equivalence class in $H_k(K,L;R)$ associated to
$\relC[K,L]{c} \in Z_k(K,L;R) = \KE(\overline{\partial}_k)$ is denoted
$[c]_{K,L}$.
\end{defn}

It is possible to develop the theory of reduced and relative homology
as it is done in \cite{MUat} but we will only do so in the context of
singular homology later on.
Since singular homology seems to be more accessible and
user friendly than simplicial homology (though the latest is more
visual), we will only study reduced and relative homology
in the context of singular homology in Sections~\ref{sectSingHom}.

\section{Simplicial Cohomology} \label{sectSimplCohom}

We only provide a very brief introduction to simplicial cohomology.
We give just enough information to reach our goal of providing a
relation between the de Rham cohomology on a manifold $S$ and the
simplicial cohomology on a simplicial complex $K$ associated (in some
way to be defined) to $S$.

\subsection{Cohomology Modules}

\begin{defn}
Let $K$ be a simplicial complex.  For $0 \leq k \leq \dim(K)$, let
$\displaystyle C^k(K;\RR)$ denote the dual space of $C_k(K;\RR)$.
Namely, $\displaystyle C^k(K;\RR)$ is the vector space of all linear
functionals on $C_k(K;\RR)$.  The elements of $\displaystyle C^k(K;\RR)$
are called {\bfseries cochaines}\index{Cochaines}. 
\end{defn}

\begin{defn}
Let $K$ be a simplicial complex.  The adjoint of the boundary map
$\partial_k: C_k(K;\RR) \to C_{k-1}(K;\RR)$ is the map
$\displaystyle \dfC_{k-1}: C^{k-1}(K;\RR) \to C^k(K;\RR)$ defined
by
\begin{equation} \label{simplCoOpEq1}
\big(\dfC_{k-1}(\phi)\big)(c) = \phi\big(\partial_k(c)\big)
\end{equation}
for all $c \in C_k(K;\RR)$ and $\displaystyle \phi \in C^{k-1}(K;\RR)$.
The map $\displaystyle \dfC_k$ is called
the {\bfseries coboundary operator}\index{Coboundary Operator}.
\end{defn}

It is clear that $\displaystyle \dfC_k$ is well and uniquely
defined by (\ref{simplCoOpEq1}).  It is easy to prove using
$\partial_{k+1} \circ \partial_{k+2} = 0$ that
$\displaystyle \dfC_{k+1} \circ \dfC_k = 0$.

\begin{defn}
Let $K$ be a simplicial complex,
$\displaystyle Z^k(K;\RR) = \{ \phi \in C^k(K;\RR) :
\dfC_k \phi = 0 \}$
and $\displaystyle B^k(K;\RR) = \{ \dfC_{k-1} \phi : \phi \in
C^{k-1}(K;\RR) \}$.
The {\bfseries $\displaystyle \mathbf{k^{th}}$ cohomology
module}\index{Cohomology Module!$k^{th}$ Cohomology Module}
of $K$ is the
quotient $\displaystyle H^k(K;\RR) = Z^k(K;\RR) / B^k(K;\RR)$.
The elements of $\displaystyle Z^k(K;\RR)$ are called
{\bfseries cocycles}\index{Cocycle} and those of
$\displaystyle B^k(K;\RR)$ are called
{\bfseries coboundaries}\index{Coboundary}.
The equivalence class in $\displaystyle H^k(K;R)$ associated to
$\displaystyle \phi \in Z^k(K;R)$ is denoted $[\phi]_K$.
\end{defn}

\begin{theorem} \label{thmHupeHdk}
Let $K$ be a simplicial complex.  Then $\displaystyle H^k(K;\RR)$ is
isomorphic to $\displaystyle (H_k(K;\RR))^\ast$.
\end{theorem}

\begin{proof}
The map $\displaystyle h:H^k(K;\RR) \to (H_k(K;\RR))^\ast$ defined by
$h([\phi]_K) = \phi$ for $\displaystyle \phi \in Z^k(K;\RR)$ is well
defined because
\begin{align*}
\phi \in Z^k(K;\RR) & \Rightarrow \dfC_k(\phi) = 0
\Rightarrow \phi(\partial_{k+1}(c)) = (\dfC_k \phi)(c) =  0
\ \text{for all} \ c \in C_{k+1}(K;\RR) \\
&\Rightarrow \phi = 0 \ \text{on} \ B_k(K;\RR)
\Rightarrow \phi \in (H_k(K;\RR))^\ast \ .
\end{align*}

We have that $h$ is one-to-one.  Suppose that $[\phi_1]_K = [\phi_2]_K$.
Then $\displaystyle \phi_1 - \phi_2 = \psi \in B^k(K;\RR)$.  Hence
$h([\phi_1]_K) = h([\phi_2]_K) + h([\psi]_K) = h([\phi_2]_K)$
because
\begin{align*}
\psi \in B^k(K;\RR)
&\Rightarrow \psi = \dfC_{k-1} \eta \ \text{for some}
\ \eta \in C^{k-1}(K;\RR) \\
&\Rightarrow \psi(c) = (\dfC_{k-1} \eta)(c)
= \eta(\partial_k(c)) = 0 \ \text{for all} \ c \in Z_k(K;\RR) \\
&\Rightarrow \psi = 0 \ \text{on}\ Z_k(K;\RR)
\Rightarrow h([\psi]_K) = 0 \ .
\end{align*}

We have that $h$ is onto.  Given $\displaystyle \phi \in (H_k(K;\RR))^\ast$,
then $\displaystyle [\phi]_K \in H^k(K;\RR)$ because
\begin{align*}
\phi(c) = 0 \ \text{for all}\ c \in B_k(K;\RR)
&\Rightarrow (\dfC_k \phi)(c) = \phi(\partial_{k+1}(c)) = 0
\ \text{for all} \ c \in C_{k+1}(K;\RR) \ \text{since} \\
&\qquad \ \partial_{k+1}(c) \in B_k(K;\RR) \ \text{for all}
\ c \in C_{k+1}(K;\RR) \\
&\Rightarrow \dfC_k \phi = 0 \Rightarrow \phi \in Z^k(K;\RR) \ .
\end{align*}
Thus $h([\phi]_K) = \phi$.

We leave it to the reader to proof that $h$ is a linear mapping.

Hence, the map $h$ is an isomorphism between $\displaystyle H^k(K;\RR)$ and
$\displaystyle (H_k(K;\RR))^\ast$.
\end{proof}

It is possible to describe explicitly the action of the coboundary
operator $\displaystyle \dfC_k$.  Suppose that $K$ is a
simplicial complex.   Given an open $k$-simplex $(s)$ in $K$, we defined
$\displaystyle \phi_{\osscript{s}{}{}{}{}} \in C^k(K;\RR)$ by
\begin{equation} \label{dfnBCkKR}
\phi_{\osscript{s}{}{}{}{}}(\os{t}{}{}{}{}) =
\begin{cases}
1 & \quad \text{if} \ \os{t}{}{}{}{} = \os{s}{}{}{}{} \\    
-1 & \quad \text{if} \ \os{t}{}{}{}{} = -\os{s}{}{}{}{} \\    
0 & \quad \text{otherwise}
\end{cases}
\end{equation}
for all open $k$-simplices $(t)$ in $K$.
If $\big\{ \os{s_1}{}{}{}{}, \os{s_2}{}{}{}{}, \ldots, \os{s_m}{}{}{}{} \big\}$
is a basis of $C_k(K;\RR)$, then
$\big\{ \phi_{\osscript{s_1}{}{}{}{}}, \phi_{\osscript{s_2}{}{}{}{}}, \ldots,
\phi_{\osscript{s_m}{}{}{}{}} \big\}$ is a basis of $\displaystyle C^k(K;\RR)$.

\begin{prop}
We have for all open $k$-simplices
$(s) = (\VEC{x}_0,\VEC{x}_1,\ldots,\VEC{x}_k) \in K$ that
\begin{equation} \label{pakpoEq1}
\dfC_k (\phi_{\osscript{\VEC{x}_0}{}{\VEC{x}_1}{}{\VEC{x}_k}})
= \sum_{\substack{(\VEC{y}) \in K\\
(\VEC{y},\VEC{x}_0,\VEC{x}_1,\ldots, \VEC{x}_k) \in K}}
\phi_{\osscript{\VEC{y}}{\VEC{x}_0}{\VEC{x}_1}{}{\VEC{x}_k}} \ .
\end{equation}
\end{prop}

\begin{proof}
Since $C_{k+1}(K;\RR)$ is generated by oriented $(k+1)$-simplices, we
only have to prove (\ref{pakpoEq1}) for an arbitrary ordered
$(k+1)$-simplex $\os{\VEC{z}_0}{}{\VEC{z}_1}{}{\VEC{z}_{k+1}}$.

By definition,
\begin{align*}
\big(\dfC_k (\phi_{\osscript{\VEC{x}_0}{}{\VEC{x}_1}{}{\VEC{x}_k}})
\big)(\os{\VEC{z}_0}{}{\VEC{z}_1}{}{\VEC{z}_{k+1}})
&= \phi_{\osscript{\VEC{x}_0}{}{\VEC{x}_1}{}{\VEC{x}_k}}
\big(\partial_{k+1}(\os{\VEC{z}_0}{}{\VEC{z}_1}{}{\VEC{z}_{k+1}})\big) \\
&= \phi_{\osscript{\VEC{x}_0}{}{\VEC{x}_1}{}{\VEC{x}_k}}
\bigg(\sum_{j=0}^{k+1} (-1)^j
\os{\VEC{z}_0}{}{\VEC{z}_1}{\widehat{\VEC{z}_j}}{\VEC{z}_{k+1}}\bigg) \\
&= \sum_{j=0}^{k+1} (-1)^j
\phi_{\osscript{\VEC{x}_0}{}{\VEC{x}_1}{}{\VEC{x}_k}}
\big(\os{\VEC{z}_0}{}{\VEC{z}_1}{\widehat{\VEC{z}_j}}{\VEC{z}_{k+1}}\big) \ .
\end{align*}
However
\begin{align*}
&\phi_{\osscript{\VEC{x}_0}{}{\VEC{x}_1}{}{\VEC{x}_k}}
\big(\os{\VEC{z}_0}{}{\VEC{z}_1}{\widehat{\VEC{z}_j}}{\VEC{z}_{k+1}}\big) \\
& \qquad = \begin{cases}
1 & \quad \text{if}
\ \os{\VEC{z}_0}{}{\VEC{z}_1}{\widehat{\VEC{z}_j}}{\VEC{z}_{k+1}}
= \os{\VEC{x}_0}{}{\VEC{x}_1}{}{\VEC{x}_k} \\
-1 & \quad \text{if}
\ \os{\VEC{z}_0}{}{\VEC{z}_1}{\widehat{\VEC{z}_j}}{\VEC{z}_{k+1}}
= - \os{\VEC{x}_0}{}{\VEC{x}_1}{}{\VEC{x}_k} \\
0 & \quad \text{otherwise}
\end{cases} \\
& \qquad = \begin{cases}
1 & \quad \text{if}
\ \os{\VEC{z}_0}{}{\VEC{z}_1}{}{\VEC{z}_{k+1}}
= (-1)^j \os{\VEC{z}_j}{\VEC{x}_0}{\VEC{x}_1}{}{\VEC{x}_k} \\
-1 & \quad \text{if}
\ \os{\VEC{z}_0}{}{\VEC{z}_1}{}{\VEC{z}_{k+1}}
= (-1)^{j+1} \os{\VEC{z}_j}{\VEC{x}_0}{\VEC{x}_1}{}{\VEC{x}_k} \\
0 & \quad \text{otherwise}
\end{cases} \\
& \qquad = \begin{cases}
(-1)^j \phi_{\osscript{\VEC{z}_j}{\VEC{x}_0}{\VEC{x}_1}{}{\VEC{x}_k}}
\big(\os{\VEC{z}_0}{}{\VEC{z}_1}{}{\VEC{z}_{k+1}}\big)
& \quad \text{if}
\ \os{\VEC{z}_j}{\VEC{x}_0}{\VEC{x}_1}{}{\VEC{x}_k} \\
& \quad = \pm \os{\VEC{z}_0}{}{\VEC{z}_1}{}{\VEC{z}_{k+1}} \\
0 & \quad \text{otherwise}
\end{cases}
\end{align*}
Thus (\ref{pakpoEq1}) is true for 
$\os{\VEC{z}_0}{}{\VEC{z}_1}{}{\VEC{z}_{k+1}}$.
\end{proof}

\subsection{Relative Cohomology}

It is also natural to define relative $k$-cohomology modules.  This is
another topic that we only briefly introduce but that could be greatly
developed.  We will not do so in the context of simplicial theory but
will treat it a little bit more seriously in
Subsection~\ref{ssectRCsingT} in the context of singular theory.

\begin{defn}
Let $L$ be a simplicial subcomplex of a simplicial complex $K$, and $R$
be an integral domain.  The adjoint of the boundary operator 
$\overline{\partial}_k : C_k(K,L;R) \to C_{k-1}(K,L;R)$ is the operator
$\displaystyle \overline{\dfC}_{k-1} : \big(C_{k-1}(K,L;R)\big)^\ast
\to \big(C_k(K,L;R)\big)^\ast$ defined
\begin{equation} \label{relsimplCoOpEq1}
\big(\overline{\dfC}_{k-1}(\phi)\big)([c])
= \phi\big(\overline{\partial}_k([c])\big)
\end{equation}
for all $\relC[K,L]{c} \in C_k(K,L;R)$ and
$\displaystyle \phi \in \big(C_{k-1}(K,L;R)\big)^\ast$.
\end{defn}

It is clear that $\displaystyle \overline{\dfC}_k$ is well
and uniquely defined by (\ref{relsimplCoOpEq1}).

\begin{defn}
Let $L$ be a simplicial subcomplex of a simplicial complex $K$, and $R$
be an integral domain.
The {\bfseries relative $\displaystyle \mathbf{k^{th}}$ cohomology
module}\index{Cohomology Module!Relative $k^{th}$ Cohomology Module} of
$K \mod L$ is the
quotient space $\displaystyle H^k(K,L;R) = \KE(\overline{\dfC}_k)/
\IMG(\overline{\dfC}_{k-1})$.
\end{defn}

\subsection{Cup Product}

There is a feature of cohomology theory that is equivalent to the
wedge product of differential forms on manifolds.  It is called the
cup product.

\begin{defn}
Let $K$ be a simplicial complex and $R$ be an integral domain.
The {\bfseries cup product}\index{Cup Product} of
$\displaystyle \phi_1 \in C^{k_1}(K;R)$ with
$\displaystyle \phi_2 \in C^{k_2}(K;R)$ is the linear functional
$\displaystyle \phi_1 \cup \phi_2 \in C^{k_1+k_2}(K;R)$ defined first by
$(\phi_1 \cup \phi_2)(\os{\VEC{x}_0}{}{\VEC{x}_1}{}{\VEC{x}_{k_1+k_2}})
= \phi_1(\os{\VEC{x}_0}{}{\VEC{x}_1}{}{\VEC{x}_{k_1}})
\,\phi_2(\os{\VEC{x}_{k_1}}{}{\VEC{x}_{K_1+1}}{}{\VEC{x}_{k_1+k_2}})$
for all $(\VEC{x}_0,\VEC{x}_1, \ldots,\VEC{x}_{k_1+k_2}) \in K$ and
after extended linearly to $\displaystyle C^{k_1+k_2}(K;R)$.
\end{defn}

It is not hard to prove that the cup product is a bilinear map, that
it is associative and that the identity element is the linear functional
$\displaystyle \delta \in C^0(K;R)$ defined by
$\delta(\os{\VEC{x}_i}{}{}{}{}) = 1$ for all vertices $\VEC{x}_i$ of $K$.
We define the cup product of
$\displaystyle [\phi_1]_K \in H^{k_1}(K;R)$ and
$\displaystyle [\phi_2]_K \in H^{k_2}(K;R)$ as
$\displaystyle [\phi_1]_K\cup[\phi_2]_K = [\phi_1 \cup \phi_2]_K$.

We will not justify these definitions and deduce their properties.  We
will prove later that simplicial homology and simplicial cohomology on
the simplex $K$ are respectively isomorphic to singular homology and
singular cohomology on $[K]$.  It is in the context of singular
theory that we will prove that the cup product is well defined, and
present other results about the cup product. There is effectively a
lot more that can be said about the cup product.

\subsection{Relation Between Simplicial Cohomology and
de Rham Cohomology}  \label{ssectSCandCM}

\begin{defn}  \label{defnSTriangleM}
A triple $(S,K,h)$ is a {\bfseries smoothly triangulated
manifold}\index{Smoothly Triangulated Manifold} if
\begin{enumerate}
\item $S$ is a manifold of class $\displaystyle C^\infty$,
\item $K$ is a simplicial complex,
\item $h:[K] \to S$ is a homeomorphism,
\item there exist, for every
$(s) = (\VEC{x}_{j_0},\VEC{x}_{j_1}, \ldots, \VEC{x}_{j_p}) \in K$, an
open subset $U_{[s]}$ of the $k$-dimensional affine space
$\displaystyle A_{[s]} =
\{\VEC{x}_{j_0} + \sum_{i=1}^p y_i (\VEC{x}_{j_i}-\VEC{x}_{j_0}) :
y_i \in \RR \ \text{for} \ 1 \leq i \leq p \}$
and a map $h_{[s]}:U_{[s]} \to S$ of class $\displaystyle C^\infty$
such that $[s] \subset U_{[s]}$ and $h_{[s]}(\VEC{x}) = h(\VEC{x})$
for all $\VEC{x} \in [s]$ (Figure~\ref{SimpHm2}), and
\item $h_{[s]}(U_{[s]})$ is a submanifold of $S$ of class
$\displaystyle C^\infty$.
\end{enumerate}
\end{defn}

\pdfF{alg_top/simphm2}{Smoothly triangulated manifolds}{We consider
the simplicial complex\\ $K = \{(\VEC{x}_0),(\VEC{x}_1),(\VEC{x}_2),
(\VEC{x}_0,\VEC{x}_1),(\VEC{x}_0,\VEC{x}_2),(\VEC{x}_1,\VEC{x}_2)\}$
and the circle $\displaystyle S^1$ inscribed in $[K]$.  The maps $h$
is represented by the black arrows.  A possible open set
$U_{[\VEC{x}_0,\VEC{x}_1]}$ and map $h_{[\VEC{x}_0,\VEC{x}_1]}$ is
provided with $h_{[\VEC{x}_0,\VEC{x}_1]}$ on
$U_{[\VEC{x}_0,\VEC{x}_1]} \setminus [\VEC{x}_0,\VEC{x}_1]$ 
represented by the blue arrows.}{SimpHm2}

\begin{defn} \label{defnjthBC}
Let $K$ be a simplicial complex and $\VEC{x}_j$ for $0 \leq j \leq q$
be the vertices of $K$.  The
{\bfseries $\displaystyle \mathbf{j^{th}}$ barycentric
coordinates}\index{Barycentric Coordinates!$j^{th}$ Barycentric Coordinates} of
$\VEC{x} \in [K]$, denoted $b_j(\VEC{x})$, is defined by
\[
b_j(\VEC{x}) = \begin{cases}
0 & \hspace{1.5em} \text{if} \ x \not\in \St(\VEC{x}_j) \\
\begin{array}{l}
\text{The $j^{th}$ barycentric} \\[-0.3em]
\text{coordinate of} \ \VEC{x} \in (s)
\end{array} & \quad
\text{if}\ \VEC{x} \in (s) \subset \St(\VEC{x}_j)
\end{cases}
\]
\end{defn}

\begin{rmk}
The $\displaystyle j^{th}$ barycentric coordinates given in
Definition~\ref{defnjthBC} have several interesting properties.
\begin{enumerate}
\item  We have that $b_j:[K]\to \RR$ is continuous.  For instance, suppose
that $K$ is the simplicial complex consisting of all the
faces of $[\VEC{x}_2,\VEC{x}_3,\VEC{x}_5]$ and the open simplices
$(\VEC{x}_0)$, $(\VEC{x}_1)$, $(\VEC{x}_0,\VEC{x}_1)$ and
$(\VEC{x}_1,\VEC{x}_2)$ as represented in the following figure.
\pdfbox{alg_top/simphm3}
$\St(\VEC{x}_2)$ is represented by the continuous lines and the
shaded area.  Given\\
$\VEC{x} \in (\VEC{x}_2,\VEC{x}_3,\VEC{x}_4)$, we have that
$\displaystyle \VEC{x} = \sum_{j=2}^4 a_j \VEC{x}_j$
with $\displaystyle \sum_{j=2}^4 a_j = 1$.
The function $b_2$ is continuous with respect to the $a_j$ in the open set
$(\VEC{x}_2,\VEC{x}_3,\VEC{x}_4)$.  Therefore, if
$\VEC{x}$ converges to $\tilde{\VEC{x}} \in [\VEC{x}_3,\VEC{x}_4]$, then
$a_3$ and $a_4$ converge to $\tilde{a}_3$ and $\tilde{a}_4$
respectively where
$\displaystyle \tilde{\VEC{x}} = \sum_{j=3}^4 \tilde{a}_j \VEC{x}_j$
with $\displaystyle \sum_{j=3}^4 \tilde{a}_j = 1$.
Thus $a_3 + a_4 \to 1$ and so $b_2(\VEC{x}) = a_2$ converges to $0$.
Hence $b_2$ is continuous along $[\VEC{x}_3,\VEC{x}_4]$.  Likewise,
if $\VEC{x}$ converges to $\breve{\VEC{x}} \in [\VEC{x}_2,\VEC{x}_3]$, then
$a_2$ and $a_3$ converge to $\breve{a}_2$ and $\breve{a}_3$
respectively where
$\displaystyle \breve{\VEC{x}} = \sum_{j=2}^3 \breve{a}_j \VEC{x}_j$
with $\displaystyle \sum_{j=2}^3 \breve{a}_j = 1$.
Thus $b_2(\VEC{x}) = a_2$ converges to $\breve{a}_2 = b_2(\breve{\VEC{x}})$.
Hence $b_2$ is continuous along $[\VEC{x}_2,\VEC{x}_3]$.  Proceeding
in a similar way, we can prove that $b_2$ is continuous on all of
$[K]$.  Likewise, $b_1$ and $b_3$ are continuous on all of
$[K]$.
\item For all $\VEC{x} \in [K]$, we have that
$\displaystyle \VEC{x} = \sum_{j=0}^q b_j(\VEC{x}) \VEC{x}_j$ with
$b_j(\VEC{x}) \geq 0$ and $\displaystyle \sum_{j=0}^q b_j(\VEC{x}) = 1$.
More precisely, since
$\VEC{x} \in (\VEC{x}_{j_0},\VEC{x}_{j_1}, \ldots, \VEC{x}_{j_p}) \in K$
for some $j_i \in \{0,1,2,\ldots,q\}$ and $0 \leq i \leq p$, we get
that $\displaystyle \VEC{x} = \sum_{i=0}^p b_{j_i}(\VEC{x}) \VEC{x}_{j_i}$
with $b_{j_i}(\VEC{x}) > 0$ and
$\displaystyle \sum_{i=0}^p b_{j_i}(\VEC{x}) = 1$.  Moreover,
$b_j(\VEC{x}) = 0$ if $j \not\in \{ j_0,j_1,\ldots,j_p\}$.
\end{enumerate}
\end{rmk}

\begin{defn} \label{defnStarOS}
Let $K$ be a simplicial complex and $(s)$ be an open simplex in $K$.
The {\bfseries star}\index{Star} of $(s)$, denoted $\St\big((s)\big)$, is
defined by
$\displaystyle \St\big((s)\big)
= \bigcup_{\substack{(t)\, \in K \text{\ such that}
\\(s) \text{ is a face of } [t]}} (t)$.
\end{defn}

Recall that, by definition of a simplicial complex, if $(s) \in K$,
then all the faces of $[s]$ are also in $K$; in particular,
$(s) \subset \St\big((s)\big)$ because $(s)$ is a face of $[s]$.

\begin{rmk}
Below are some of the properties of the star of an open simplex as
given in Definition~\ref{defnStarOS}.  Some these properties are easy
to prove and proving them is left to the reader.
\begin{enumerate}
\item For a $0$-simplex $(\VEC{x}) \in K$,
we have that $\St\big((\VEC{x})\big) = \St(\VEC{x})$, the star of the
vertex $\VEC{x}$ of $K$ as given in Definition~\ref{defnStarV}.
\item If $(s)$ is an open simplex of $K$, then $\St\big((s)\big)$ is
an open set in $[K]$.  This follows from the fact that $\St\big((s)\big)$
is the union of open simplices of $[K]$.
\item If $(s) = (\VEC{x}_{j_0},\VEC{x}_{j_1},\ldots,\VEC{x}_{j_p}) \in K$
where the $\VEC{x}_{j_i}$ for $0 \leq i \leq p$ are vertices of $K$, then
$\VEC{x} \in \St\big((s)\big)$ if and only if
$b_{j_i}(\VEC{x}) \neq 0$ for $0\leq i \leq p$.
To verify this claim, it suffices to note that if
$\VEC{x} \in \St\big((s)\big)$ then $\VEC{x} \in (t)$ such that $(s)$
is a face of $[t]$.  Thus
$(t) = (\VEC{x}_{j_0},\VEC{x}_{j_1}, \ldots, \VEC{x}_{j_q})$ with
$q\geq p$ and $j_i \not\in \{j_0,j_1, \ldots, j_p\}$ for $p < i \leq q$.
Hence $\displaystyle \VEC{x} = \sum_{i=0}^q b_{j_i}(\VEC{x}) \VEC{x}_{j_1}$
with $b_{j_i}(\VEC{x}) > 0$ for $0 \leq i \leq q$
and $\displaystyle \sum_{i=0}^q b_{j_i}(\VEC{x}) = 1$.  
\item It follows from (3) that
$\displaystyle [K]\setminus \St\big((s)\big) =
\big\{ \VEC{x} \in K : b_{j_i}(\VEC{x}) = 0 \ \text{for some} \ i\in
\{0,1,\ldots,p\} \big\}$.
\item If $(s_1)$ and $(s_2)$ are two open $q$-simplices of a simplicial
complex $K$, and $(s_1) \neq (s_2)$, then
$[s_2] \subset [K]\setminus \St\big((s_1)\big)$.  This follows from
the fact that all faces of a closed $q$-simplex, other than the open
$q$-simplex itself, are open $j$-simplices with $j<q$.  So $(s_1)$
cannot be a face of $[s_2]$.
\end{enumerate}
\end{rmk}

Suppose that $(S,K,h)$ is a smoothly triangulated manifold.  We would
like to find a relation between the $\displaystyle k^{th}$ cohomology
module of $S$ defined in Definition~\ref{defnqdeRhamCM} and the
$\displaystyle k^{th}$ cohomology group $\displaystyle H^k(K;\RR)$ of
$K$.  For that, we will define a map
$\displaystyle F_k : C^\infty\big(S,\Omega^k(S)\big) \to C^k(K,\RR)$
such that the following diagram commutes.
\[
\xymatrix{
C^\infty\big(S,\Omega^k(S)\big) \ar[r]^-{\df{}}\ar[d]_{F_k} &
C^\infty\big(S,\Omega^{k+1}(S)\big) \ar[d]^{F_{k+1}} \\
C^k(K;\RR) \ar[r]_-{\dfC_k} & C^{k+1}(K;\RR)
}
\]
Suppose that $\omega$ is a differential $k$-form on $S$.  To define
$\displaystyle F_k(\omega) \in C^k(K;\RR)$, it is enough to define
$F_k(\omega)$ on the equivalence classes of oriented
$k$-simplices of $K$ and use the linearity of $F_k(\omega)$ to extend
its definition to all of $\displaystyle C_k(K;\RR)$.
Given $\os{s}{}{}{}{} \in C_k(K;\RR)$ with
$(s) = (\VEC{x}_{j_0},\VEC{x}_{j_1},\ldots,\VEC{x}_{j_k}) \in K$, there
exists an open subset $U_{[s]}$ and a smooth mapping $h_{[s]}:U_{[s]} \to S$
that satisfy Definition~\ref{defnSTriangleM}.  We set
\[
  F_k(\omega)(\os{s}{}{}{}{}) = \int_{[s]} h_{[s]}^\ast(\omega) \ .
\]
We may consider that $[s]$ is a subset of the $k$-dimensional manifold
$U_{[s]}$.  This manifold can be described by only one local chart;
namely, $(W_{[s]},U_{[s]},\phi_{[s]})$ where
\begin{align*}
\phi_{[s]}:\RR^k &\to \{\VEC{x}_{j_0}
+ \sum_{i=1}^k y_i (\VEC{x}_{j_i}-\VEC{x}_{j_0}) :
y_i \in \RR \ \text{for} \ 1 \leq i \leq k \} \\
\VEC{y} & \mapsto \VEC{x}_{j_0} + \sum_{i=1}^k y_i (\VEC{x}_{j_i}-\VEC{x}_{j_0})
\end{align*}
and $\displaystyle W_{[s]} = \phi_{[s]}^{-1}(U_{[s]})$.
Hence, the local representation of $\displaystyle h_{[s]}^\ast(\omega)$
is given by $\displaystyle \phi_{[s]}^\ast(h_{[s]}^\ast(\omega)) =
g_{[s]} \df{y_1}\wedge \df{y_2} \wedge \ldots \wedge \df{y_q}$ where
$g_{[s]}:W_{[s]} \to \RR$ is of class $\displaystyle C^\infty$ and
\[
F_k(\omega)(\os{s}{}{}{}{}) = \int_{W_{[s]}} g_{[s]}
\df{y_1}\wedge \df{y_2} \wedge \ldots \wedge \df{y_k} \ .
\]

\begin{prop}
We have that $\displaystyle \dfC_k \circ F_k = F_{k+1} \circ \df{}$.
\end{prop}

\begin{proof}
It suffices to use Stokes' theorem.  Given a differential $k$-form on
$S$ and $\os{s}{}{}{}{} \in C_k(K,\RR)$, we have
\begin{align*}
F_{k+1}(\df{\omega})(\os{s}{}{}{}{})
&= \int_{[s]} h_{[s]}^\ast(\df{\omega})
= \int_{[s]} \df{\left(h_{[s]}^\ast(\omega)\right)}
= \int_{\partial [s]} h_{[s]}^\ast(\omega) \\
&= F_k(\omega)(\partial_k \os{s}{}{}{}{})
= \dfC_k(F_k(\omega))(\os{s}{}{}{}{}) \ .
\end{align*}
Stokes' theorem, Theorem~\ref{stokesStokes}, was used for the third
equality.  Moreover, $\partial [s]$ must be interpreted according to
Stokes' theorem. Hence the orientation on $\partial [s]$ is the
induced orientation from the orientation on $[s]$.  It is in fact the
orientation given by $\partial_k \os{s}{}{}{}{}$.
\end{proof}

The next theorem is the main result of this section.

\begin{theorem}[De Rham's Theorem] \label{DeRhamThm}
Let $(S,K,h)$ be a smoothly triangulated manifold and
$\displaystyle \tilde{F}_k:H^k(S) \to H^k(K;\RR)$ be the map defined by
$\displaystyle \tilde{F}_k([\omega])([c]) = [F_k(\omega)(c)]_K$ for the
closed differential $k$-form $\omega$ on $S$ and $k$-chains $c$ in $Z_k(K;\RR)$.
Then $\tilde{F}_k$ is an isomorphism for $0 \leq k \leq \dim(S)$.
\end{theorem}

The proof of this theorem is very long but it is not hard. We only
sketch the proof of this famous theorem.  The reader can find the
detailed proof in \cite{ST}.

\begin{proof}[Proof (Sketch)]
\stage{Claim 1} There exist linear maps
$\displaystyle G_k: C^k(K;\RR) \to C^\infty\big(S,\Omega^k(S)\big)$
for $0 \leq k \leq \dim(S)$ such that:
\begin{enumerate}
\item $\displaystyle \df{}\circ G_k = G_{k+1}\circ \dfC_k$ for
$0 \leq k < \dim(S)$.
\item $F_k \circ G_k = \Id$.
\item If $\displaystyle \phi \in C^0(K,\RR)$ satisfies
$\phi(\os{\VEC{x}}{}{}{}{}) = 1$
for all $0$-simplices $(\VEC{x}) \in K$, then $G_0(\phi)$ is the
differential $0$-form (i.e.\ a function on $S$) defined by
$G_0(\phi)(\VEC{x}) = 1$ for all $\VEC{x} \in S$.
\item If $\os{s}{}{}{}{}$ is an oriented $k$-simplex of $K$, then
$G_k(\phi_{\osscript{s}{}{}{}{}})$ is null in an open set containing $[K]
\setminus \St\big((s)\big)$, where $\phi_{\osscript{s}{}{}{}{}}$ is
defined in (\ref{dfnBCkKR}).
\end{enumerate}

\stage{i} Suppose that $K$ is a simplicial complex of dimension $q$
and that $\VEC{x}_0$, $\VEC{x}_1$, \ldots, $\VEC{x}_p$ are the
vertices of $K$.  We define a partition of unity subordinated to the open
cover $\big\{ \St(\VEC{x}_j) \big\}_{0\leq j \leq p}$
of $[K]$.

Let
$A_j = \big\{ \VEC{x} \in [K] : b_j(\VEC{x}) \geq 1/(q+1) \big\}$
and
$B_j = \big\{ \VEC{x} \in [K] : b_j(\VEC{x}) \leq 1/(q+2) \big\}$
for $0 \leq j \leq p$, where $b_j$ is the barycentric coordinate
function defined in Definition~\ref{defnjthBC}.

Since $A_j$ is a closed subset of the compact set $[K]$, we have
that $A_j$ is compact for $0 \leq j \leq p$.

We have that $[K]\setminus \St(\VEC{x}_j) \subset B_j$
and $A_j \subset C_j = [K] \setminus B_j \subset \St(\VEC{x}_j)$
for $0 \leq j \leq p$ as can be seen in the figure below.
\pdfbox{alg_top/derham1}
In the figure above, the dotted area represents $\St(\VEC{x}_j)$.
The collection $\{ A_j\}_{0\leq j \leq p}$ is a closed cover of
$[K]$.  Given $\VEC{x} \in [K]$, we have that $\VEC{x}$ is an element
of an open $m$-simplex
$(s) = (\VEC{x}_{j_1},\VEC{x}_{j_2}, \ldots, \VEC{x}_{j_m})$
of $K$ with $m \leq q$.  By definition of the barycentric coordinates,
we have that $b_{j_i}(\VEC{x}) > 0$ for $0\leq i \leq m$ and
$\displaystyle \sum_{i=0}^m b_{j_i}(\VEC{x}) = 1$.  Therefore, we must
have that $b_{j_i}(\VEC{x}) \geq 1/(m+1) \geq 1/(q+1)$ for some
$0 \leq i \leq m$.  Thus $\VEC{x} \in A_{j_i}$.

Given $0 \leq j \leq p$, there exists according to
Proposition~\ref{topolProp2} an open set
$\displaystyle V_j \subset \RR^n$ such that
$\displaystyle  A_j \subset V_j \subset \overline{V}_j
\subset \RR^n \setminus B_j$.
Using partition of unity for instance, we can select a function
$\displaystyle f_j \in C^\infty(\RR^n)$ such that
$0 \leq f_j(\VEC{x}) \leq 1$ for all $\displaystyle \VEC{x} \in \RR^n$,
$f_j(\VEC{x}) = 1$ for all $\VEC{x} \in A_j$ and $f_j(\VEC{x}) = 0$
for all $\displaystyle \VEC{x} \not\in V_j$.
It follows from the previous paragraph that, given any $\VEC{x} \in [K]$,
there exists $0 \leq j \leq p$ such that $f_j(\VEC{x}) = 1$.
Let $\displaystyle g = \sum_{j=0}^pf_j$ on $\displaystyle \RR^n$.  We
have that $g(\VEC{x}) \geq 1$ for all $\VEC{x} \in [K]$.  Thus
$g(\VEC{x}) \geq 1/2 > 0$ for all $\VEC{x}$ in an open set $W \supset [K]$. 
Let $g_j = f_j/g$  on $W$ for $0 \leq j \leq p$.  Hence
$\displaystyle g_j\in C^\infty(W)$ and
$\displaystyle \sum_{j=0}^pg_j = 1$ on $[K]$.  Moreover,
$\supp g_j\Big|_{[K]} \subset \supp f_j\Big|_{[K]}
\subset C_j \subset \St(\VEC{x}_j)$ for $0 \leq j \leq p$.

The collection $\{g_j\}_{0\leq j \leq p}$ is the partition of unity
subordinated to the open cover\\
$\big\{ \St(\VEC{x}_j) \big\}_{0\leq j \leq p}$
of $[K]$ that we are looking for.

\stage{ii} Since $G_k$ is linear, we only have to define $G_k$ for
$\phi_{\osscript{s}{}{}{}{}}$ where $\os{s}{}{}{}{}$ is an
oriented $k$-simplices of $K$.  Suppose that
$\os{s}{}{}{}{}= \os{\VEC{x}_{j_0}}{}{\VEC{x}_{j_1}}{}{\VEC{x}_{j_k}}$.
We set
\[
G_k(\phi_{\osscript{s}{}{}{}{}})
= k! \sum_{i=0}^k (-i)^i g_{j_i} \df{g_{j_0}} \wedge \df{g_{j_1}}
\wedge \ldots \wedge \widehat{\df{g_{j_i}}} \wedge \ldots \wedge \df{g_{j_k}}
\ .
\]
Note that $G_k(\phi_{\osscript{s}{}{}{}{}})$ defines
a differential $k$-form on $W \supset [K]$.  We can use the maps
$h_{[s]]}$ provided by the smoothly triangulated manifold $(S,K,h)$ to
transport $G_k(\phi_{\osscript{s}{}{}{}{}})$ to $S$ \footnote{In
\cite{ST}, they simplify the writing of the proof by assuming that
$[K] = S$.  In other words, they project $K$ on $S$ using the maps $h_s$.}. 

The proof given in \cite{ST} that $G_k$ satisfies all four conditions
listed above requires, among several results, Stokes' theorem on chains.

\stage{Claim 2}
If $\omega$ is a closed differential $k$-form on $S$ such that
$\displaystyle F_k(\omega) = \dfC_k(\phi)$ for some
$\displaystyle \phi \in C^{k-1}(K;R)$,
then there exists a differential $(k-1)$-form $\rho$ on $S$ such that
$F_{k-1}(\rho) = \phi$ and $\df{\rho} = \omega$.

The proof of this claim is in two parts.

\stage{First Part}
Given an open $q$-simplex $(s) \in K$, let
$\displaystyle B_{(s)}
= \bigcup_{\substack{(t) \text{ an open face of }[s]\\\text{with } \dim (t) < q}}
(t)$.

We prove the following statements.
\begin{enumerate}
\item Assuming $q>0$, if $\omega$ is a closed differential $k$-form defined
in an open neighbourhood of $B_{(s)}$ in $\displaystyle \RR^n$
where $(s) \in K$ is an open $q$-simplex. then there exists a closed
differential $k$-form $\rho$ defined in an open neighbourhood of $[s]$
such that $\rho = \omega$ in the open neighbourhood of $B_{(s)}$.  If
$q=k+1$, then we must also require that
$\displaystyle \int_{\partial [s]} \omega =0$.
\item Assuming $q,k>0$, if $\omega$ is a closed differential $k$-form
defined in an open neighbourhood of $[s]$ in $\displaystyle \RR^n$ where
$(s) \in K$ is an open $q$-simplex, and $\rho$ is a differential $(k-1)$-form
defined in an open neighbourhood of $B_{(s)}$ and such
that $\df{\rho} = \omega$ in that neighbourhood of $B_{(s)}$, then
there exists a differential $(k-1)$-form $\nu$ defined in an open
neighbourhood of $[s]$ such that $\nu = \rho$ in an open neighbourhood
of $B_{(s)}$ and $\df{\nu} = \omega$ in a neighbourhood of $[s]$.
If $q=k$, then we must also
require that $\displaystyle \int_{\partial [s]} \rho = \int_{[s]} \omega$.
\end{enumerate}

Note that the two extra conditions in the two previous statement are
required because we get from Stokes' theorem that
\[
\int_{\partial [s]} \omega = \int_{\partial [s]} \rho
= \int_{[s]} \df{\rho} = \int_{[s]} 0 = 0
\]
if $\rho$ exists in the first statement and $q = k+1$, and
\[
\int_{[s]} \omega = \int_{[s]} \df{\nu}
= \int_{\partial [s]} \nu = \int_{\partial [s]} \rho
\]
if $\nu$ exists in the second statement and $q=k$.

The proof of the two statements above is an iterative proof.  We first
prove that (1) is true for $k=0$.  Next, we prove that if (1) is true
for $k=i$, then (2) is true for $k=i+1$.  Finally, we prove that
if (2) is true for $k=i+1$, then (1) is true for $k=i+1$.  Hence,
iterating from $i=0$ up to $i = \dim(S)$ shows that (1) and (2) are
true for all values of $k$

To prove (1) for $k=0$, we note that $\omega$ is a function on an open
neighbourhood of $B_{(s)}$ such that $\displaystyle \pdydx{\omega}{x_i} = 0$ in
this open neighbourhood for all $i$.  Thus $\omega$ is constant on the
components of $B_{(s)}$.  If $q>1$, then $B_{(s)}$ is
connected and so $\omega(\VEC{x}) = c$, a constant, for all
$\VEC{x} \in B_{(s)}$.  Therefore, we may set $\rho = c$ in an opne
neighbourhood of $[s]$.  If $q=1$, then
$[s] = [\VEC{x}_{j_1},\VEC{x}_{j_2}]$ for some vertices 
$\VEC{x}_{j_1}$ and $\VEC{x}_{j_2}$ of $K$, and
$B_{(s)} = \{\VEC{x}_{j_1},\VEC{x}_{j_2}\}$.  But since $q = k+1$, we
have that
$\displaystyle 0 = \int_{\partial [s]} \omega = \omega(\VEC{x}_{j_1}) - 
\omega(\VEC{x}_{j_0})$.  Thus $\omega(\VEC{x}_{j_1}) =
\omega(\VEC{x}_{j_0}) = c$, a constant.  Therefore, we may again set
$\rho =c$ in an open neighbourhood of $[s]$.

To prove that (1) for $k=i$ implies (2) for $k=i+1$, we may assume that
$\omega$ is a closed differential $k$-form in a connected open set
$U \supset [s]$ because $[s]$ is connected.  It then follows from 
Corollary~\ref{corHqg0e0} that $\omega = \df{\mu_1}$ for some
differential $(k-1)$-form $\mu_1$ on $U$ \footnote{We are using the fact that
$\displaystyle H^k(U) = H^k(\RR^n) = 0$ for $k>0$ because $U$ is
diffeomorphic to $\displaystyle \RR^n$.}.  Since $\mu_1$ may not be
equal to $\rho$ in a neighbourhood of $B_{(s)}$, we consider
$\mu_2 = \mu_1 - \rho$.  We have that
$\df{\mu_2} = \df{\mu_1} - \df{\rho} = \omega - \omega = 0$ in a
neighbourhood of $B_{(s)}$ and, if $q = k$, we have from Stokes'
theorem that
\[
\int_{\partial [s]} \mu_2
= \int_{\partial [s]} \mu_1 - \int_{\partial [s]} \rho
= \int_{[s]} \df{\mu_1} - \int_{\partial [s]} \rho
= \int_{[s]} \omega - \int_{\partial [s]} \rho
= 0
\]
by assumption.  Therefore, we may use (1) with $\mu_2$ to get a closed
differential $k$-form $\mu_3$ defined in an open neighbourhood of
$[s]$ such that $\mu_3 = \mu_2$ in a neighbourhood of $B_{(s)}$.
Then $\nu = \mu_1 - \mu_3$ because $\nu = \rho$ and
$\df{\nu} = \df{\mu_1} - \df{\mu_3} = \omega$ in a neighbourhood of 
$B_{(s)}$.

To prove that (2) for $k=i+1$ implies (1) for $k=i+1$ is a little bit
trickier.  Suppose that
$(s) = (\VEC{x}_{j_0},\VEC{x}_{j_1},\ldots,\VEC{x}_{j_q})$.  Let
$(t) = = (\VEC{x}_{j_1},\ldots,\VEC{x}_{j_q})$ and
$F_{(s)} = B_{(s)}\setminus (t)$.  Since $F_{(s)}$ is
star-shaped, we may assume that the open neighbourhood of $F_{(s)}$
where $\omega$ is defined is also star-shaped.  Hence, we may use
Poincaré Lemma to get a differential $(k-1)$-form $\mu$ defined in
this neighbourhood such that $\df{\mu} = \omega$.

If $q>1$, then we can show that (2) can be used with $\rho$ replaced by
$\mu$ and $(s)$ by $(t)$ to get a differential $(k-1)$-form $\nu$
defined in an open neighbourhood of $[t]$ such that
$\nu = \mu$ in an open neighbourhood of $B_{(t)}$ and $\df{\nu} = \omega$
in a neighbourhood of $[t]$.  Consider the differential $k$-form
defined by
\[
\tau = \begin{cases}
\mu & \quad \text{on the neighbourhood of}\ F_{(s)} \\
\nu & \quad \text{on the neighbourhood of}\ [t]
\end{cases}
\]
where the neighbourhoods may have to be shrunk a little if necessary.
We have that $\df{\tau} = \omega$ in a neighbourhood of $B_{(s)}$.
If $q=1$, then we also have that there is a differential $k-1$-form $\tau$
defined in $B_{(s)}$ such that $\df{\tau} = \omega$.  In this case, we
use the fact that the neighbourhood of $B_{(s)}$ may be chosen to be
the union of two disconnect open sets; one about each of the two
vertices in $B_{(s)}$. 

To complete the proof, we select a real valued function
$\displaystyle f \in C^\infty(\RR^n)$ such that $f=1$ in an open
neighbourhood of $B_{(s)}$ and $f=0$ outside the open neighbourhood
where $\tau$ is defined.  Then $\rho = \df{(f \tau)}$ satisfies (1)
because it is obviously exact and 
\[
\df{\rho} = \df{f} \wedge \tau + f \wedge \df{\tau} = \df{\tau} = \omega
\]
in the open neighbourhood where $f =1$.

\stage{Second Part}
The second part of the proof of Claim 2 used the second statement
of the previous part to construct a sequence of differential $(k-1)$-forms
$\rho_0$, $\rho_1$, $\rho_2$, \ldots , $\rho_{\dim(S)}$ such that
$\rho_j$ is defined in an open neighbourhood of the $j$-skeleton
$\displaystyle [K^j]$ of $K$,
$\df{\rho_j} = \omega$ in this open neighbourhood of $\displaystyle [K^j]$,
$\rho_j = \rho_{j-1}$ in an open neighbourhood of $\displaystyle [K^{j-1}]$
for $0 < j \leq \dim(S)$, and $F_j(\rho_j) = \phi$ if $j = k-1$.
We have that $\rho$ in Claim 2 is given by $\rho = \rho_{\dim(S)}$.

The construction of the $\rho_j$ is done by induction on the order $j$
of the skeletons of $K$.

For $j=0$, the skeleton $\displaystyle K^0$ is a
set of discrete points.  We may assume that the open neighbourhood of
$\displaystyle [K^0]$ is the union of disjoint open balls $B_j$ with
$\VEC{x}_j$ being the only vertex of $K$ in $B_j$.  Using Poincaré
Lemma on each open ball, we get a differential $(k-1)$-form
$\tilde{\rho}_0$ on the open neighbourhood of $\displaystyle [K^0]$ such that
$\df{\tilde{\rho}_0} = \omega$.
If $k-1 > 0$, then we may take $\rho_0 = \tilde{\rho}_0$.
If $k-1 = 0$, then we have to make sure that
$F_0(\rho_0) = \phi$ on $\displaystyle [K^0]$.
For each vertex $\displaystyle \VEC{x}_j$ of $K$, set $\displaystyle
a_j = \phi\big( (\VEC{x}_j)\big) - \int_{(\VEC{x}_j)}  \tilde{\rho}_0
= \phi\big( (\VEC{x}_j)\big) - \tilde{\rho}_0(\VEC{x}_j)$.
Then $\rho_0$ defined in the open neighbourhood of
$\displaystyle [K^0]$ by $\rho_0\big|_{B_j} = \tilde{\rho}_0 + a_j$ for all $j$
satisfies all the required conditions.  In particular,
$\displaystyle F_0(\rho_0)\big((\VEC{x}_j)\big) = \int_{(\VEC{x}_j)}
\big( \tilde{\rho}_0 + a_j \big) = \phi\big( (\VEC{x}_j)\big)$ for all
$\VEC{x}_j$.

We assume that we have constructed $\rho_{j-1}$ according to the
requirements.

We have that $\omega$ is a closed differential $k$-form defined in
an open neighbourhood of a closed $j$-simplex $[s]$ and $\rho_{j-1}$ is a
differential $(j-1)$-form defined in an open neighbourhood of
$B_{(s)}$ such that $\df{\rho_{j-1}} = \omega$ in this open
neighbourhood of $B_{(s)}$.  If $k = j$, then
\[
\int_{[s]} \omega
= F_k(\omega)\big(\os{s}{}{}{}{}\big)
= \dfC_k(\phi)\big(\os{s}{}{}{}{}\big)
= \phi\big( \partial_k \os{s}{}{}{}{} \big)
= F_{j-1}(\rho_{j-1})\big( \partial_k \os{s}{}{}{}{}\big)
= \int_{\partial_k [s]} \rho_{j-1}
\]
where the second equality comes from the assumption in Claim 2 and the
fourth equality from our hypothesis of induction.
Hence the second statement of the first part can be used to get a
differential $(k-1)$-form $\tilde{\rho}_{(s)}$ in an open neighbourhood
$U_{(s)}$ of $[s]$ such that $\tilde{\rho}_{[s]} = \rho_{j-1}$ in an open
neighbourhood of $B_{(s)}$ and $\df{\tilde{\rho}_{(s)}} = \omega$ in
$U_{(s)}$.   We define $\tilde{\rho}_j$ in a neighbourhood
of $\displaystyle [K^j]$ by
$\tilde{\rho}_j\big|_{U_{(s)}} = \tilde{\rho}_{(s)}$ for
all open $j$-simplices $(s)$ in $K$.   The differential $(k-1)$-form
$\tilde{\rho}_j$ is well defined by because
$\tilde{\rho}_{(s_1)} = \tilde{\rho}_{(s_2)} = \rho_{j-1}$ in an open
neighbourhood of the intersection of $[s_1]$ and $[s_2]$ for all
open $j$-simplices $(s_1)$ and $(s_2)$ in $K$.

If $k-1 \neq j$, then we may define $\rho_j$ in a neighbourhood
of $\displaystyle [K^j]$ as $\rho_j = \tilde{\rho}_j$.

If $k-1 = j$, then we use a trick similar to the one that we have used in
the case $j=0$ to ensure that $F_j(\rho_j) = \phi$ on $\displaystyle C^j(K;R)$.
Let $\displaystyle \alpha_j = \phi - F_{k-1}(\tilde{\rho}_{k-1}) \in C^j(K;R)$.
We define $\rho_j$ in a neighbourhood $U$ of $\displaystyle [K^j]$ as
$\rho_j = \tilde{\rho}_j\big|_U + G_{k-1}(\alpha_j)\big|_U$.

We have from Claim 1 that $G_q(\phi_{\osscript{t}{}{}{}{}}) = 0$
in an open neighbourhood of $K \setminus \St\big( (t) \big)$ for all
open $q$-simplices $(t)$ in $K$.  Thus
$G_q(\phi_{\osscript{t}{}{}{}{}}) = 0$ in a neighbourhood of
$\displaystyle [K^{q-1}]$ for all open $q$-simplices $(t)$ in $K$.
Since all elements of $\displaystyle C^q(K;R)$ are linear combinations
of elements of the form $\phi_{\osscript{t}{}{}{}{}}$ for $(t)$ an
open $q$-simplex in $K$, we get that $G_q(\phi) = 0$ in a
neighbourhood of $\displaystyle [K^{q-1}]$ for all
$\displaystyle \phi \in C^q(K;R)$.  Thus, for $q=k-1 = j$, we
get that
\[
\df{\rho_j} = \df{\tilde{\rho}_j} + \df{\big( G_{k-1}(\alpha_j)\big)}
= \df{\tilde{\rho}_j} + G_k\big(\dfC_{k-1}(\alpha_j)\big)
= \df{\tilde{\rho}_j} = \omega
\]
in a neighbourhood of $\displaystyle [K^{j}]$, and
\[
\rho_j = \tilde{\rho}_j + G_{k-1}(\alpha_j)
= \tilde{\rho}_j = \rho_{j-1}
\]
in a neighbourhood of $\displaystyle [K^{j-1}]$.  Moreover,
\[
F_j(\rho_j)(\os{s}{}{}{}{})
= F_j(\tilde{\rho}_j)(\os{s}{}{}{}{})
+ F_j\big( G_{k-1}(\alpha_j)(\os{s}{}{}{}{}) \big)
= \phi(\os{s}{}{}{}{}) - \alpha_j(\os{s}{}{}{}{})
+ \alpha_j(\os{s}{}{}{}{})
= \phi(\os{s}{}{}{}{})
\]
for all $j$-simplices $(\os{s}{}{}{}{})$, where we have use (2) of
Claim 1 to get the second equality.

\stage{Conclusion}
It follows from (1) of Claim 1 that
$\displaystyle G_k(Z^k(K;\RR)) \subset Z^k(S)$ where
$\displaystyle Z^k(S)$ is the set of closed differential $k$-forms, and
$\displaystyle G_k(B^k(K;\RR)) \subset B^k(S)$ where
$\displaystyle B^k(S)$ is the set of exact differential $k$-forms.
Thus $G_k$ defines a map, that we denote $\tilde{G}_k$, from
$\displaystyle H^k(K;\RR)$ to $\displaystyle H^k(S)$.  Hence, it
follows from (2) that $\tilde{G}_k$ is a right inverse for
$\tilde{F}_k$ and so $\tilde{F}_k$ is onto.

It follows from Claim 2 that, if $\displaystyle \omega \in Z^k(S)$ and
$\displaystyle F_k(\omega) \in B^k(K;\RR)$, then
$\displaystyle \omega \in B^k(S)$.  Since $F_k$ is a linear map, this
proves that $\tilde{F}_k([\omega]) = [0]_K$ implies that $[\omega] = [0]$.
Thus $F_k$ is one-to-one.  This completes the proof that $\tilde{F}_k$ is an
isomorphism.
\end{proof}

%%% Local Variables:
%%% mode: latex
%%% TeX-master: "notes"
%%% End:


\section{Singular Homology}  \label{sectSingHom}

We present the singular approach to homology theory.  We generally
follows the approach of \cite{GH} with a few exceptions.

\subsection{Homotopy Modules}

\begin{defn}
For $k \geq 0$, the
{\bfseries standard $\mathbf{k}$-simplex}\index{Standard $k$-simplex},
denoted $\Delta_k$, is defined by\\
$\displaystyle \Delta_k = \left\{ \sum_{j=0}^k a_j \VEC{e}_j :
a_j \geq 0 \ \text{and} \ \sum_{j=0}^k a_j = 1 \right\}$
where $\displaystyle \VEC{e}_0 = \VEC{0} \in \RR^\infty$ and
$\displaystyle \VEC{e}_j \in \RR^\infty$ for
$1\leq j \leq k$ is the unit vector with $1$ for the
$\displaystyle j^{th}$ component and $0$ everywhere else.
\end{defn}

Using the notation of the section on simplicial homology, we have that
$\Delta_k$ is the closed $k$-simplex
$[\VEC{e}_0,\VEC{e}_1,\VEC{e}_2,\ldots, \VEC{e}_k]$.

\begin{defn}
For $k\geq 0$, a
{\bfseries singular $\mathbf{k}$-simplex}\index{Singular $k$-Simplex}
in a topological space $X$ is a continuous map\\
$\sigma:\Delta_k \to X$.  Given an integral domain $R$, 
the free module over $R$ generated by all the
singular $k$-simplices in $X$ is denoted $S_k(X;R)$.
The elements of $S_k(X;R)$ are called
{\bfseries singular $\mathbf{k}$-chain}\index{Singular $k$-Chain}.
The $R$-module $S_k(X;R)$ is simply denoted $S_k(X)$ when the choice
of the integral domain is clear.  It is also convenient to set
$S_k(X;R) = \{0\}$ for $k<0$.
\end{defn}

The two integral domains that we will use in all our applications
are $R= \RR$ or $R = \ZZ$.

By definition of a module over an integral domain generated by a set
of elements, a singular $k$-chain $c \in S_k(X;R)$ can be written as
$\displaystyle c = \sum_{j \in J} a_j \sigma_j$
where $J \subset \NN$ is a finite set, $a_j \in R$ and
$\sigma_j$ is a singular $k$-simplex for $j \in J$.  The sum of two
singular $k$-chains
$\displaystyle c_1 = \sum_{j \in J} a_j \sigma_j$ and
$\displaystyle c_2 = \sum_{j \in J} b_j \sigma_j$ is therefore the
singular $k$-chain given by
$\displaystyle c_1 + c_2 = \sum_{j \in J} (a_j + b_j) \sigma_j$.
We may always assume that the two singular $k$-chains $c_1$ and $c_2$
are sums over the same set $J$ by adding terms of the form $0
\sigma_j$ to the sum where they are not present.  The product of a
singular $k$-chain $\displaystyle c = \sum_{j \in J} a_j \sigma_j$ by
$r \in R$ is the singular $k$-chain given by 
$\displaystyle r c = \sum_{j \in J} (r a_j) \sigma_j$.

The next known definition is important to review since it is going to
play a very important role in this section.

\begin{defn}
An {\bfseries affine space}\index{Affine Space} is a subset $X$
of $\displaystyle \RR^n$ such that $t \VEC{x}_1 + (1-t)\VEC{x}_2 \in X$
for all $\VEC{x}_i,\VEC{x}_2 \in X$ and $0 \leq t \leq 1$.
A function $\displaystyle f:X\to Y$ between two affine spaces $X$ and $Y$ is
{\bfseries affine}\index{Affine Function} if
$f(t\VEC{x}_1 + (1-t)\VEC{x}_2) = tf(\VEC{x}_1) + (1-t)f(\VEC{x}_2)$ for all
$\displaystyle \VEC{x}_1,\VEC{x}_2 \in X$ and $0 \leq t \leq 1$.
\end{defn}

There is a more general definition of affine space than the one that
we have given.  We will only use affine spaces that are subsets
$\displaystyle \RR^n$ for some $n>0$.  More precisely, we will basically only
consider the affine space $\Delta_k$ for $k\in \NN$. 

Suppose that $X$ and $Y$ are two affine spaces.
It follows from the definition of an affine function that if
$\displaystyle f:X \to Y$ is an affine function, then
$\displaystyle f\Big(\sum_{j\in J} a_j \VEC{x}_j\Big) =
\sum_{j\in J} a_j f(\VEC{x}_j)$ for all $\VEC{x}_j \in X$ 
and $a_j \in \RR$ such that $\displaystyle \sum_{j\in J} a_j = 1$
with $a_j \geq 0$.

\begin{prop}
If $\displaystyle f:\RR^n\to \RR^m$ is affine, then there exist a
linear mapping $\displaystyle L:\RR^n \to \RR^m$ and
$\displaystyle \tilde{\VEC{x}} \in \RR^m$ such that $f(\VEC{x}) = L(\VEC{x})
+ \tilde{\VEC{x}}$ for all $\displaystyle \VEC{x} \in \RR^n$.
\end{prop}

\begin{proof}
Let $\tilde{\VEC{x}} = f(\VEC{0})$ and
$L(\VEC{x}) = f(\VEC{x}) - \tilde{\VEC{x}}$ for
$\displaystyle \VEC{x} \in \RR^n$.  It
suffices to prove that $\displaystyle L:\RR^n \to \RR^m$ is a linear
map.  We have
\begin{align}
L(a\VEC{x}) &= f(a\VEC{x}) - \tilde{\VEC{x}}
= f(a \VEC{x} +(1-a)\,\VEC{0}) - \tilde{\VEC{x}}
= a f(\VEC{x}) +(1-a)f(\VEC{0}) - \tilde{\VEC{x}} \nonumber \\
&= a f(\VEC{x}) +(1-a) \tilde{\VEC{x}} - \tilde{\VEC{x}} 
= a (f(\VEC{x}) - \tilde{\VEC{x}}) = a L(\VEC{x}) \label{singHEq1}
\end{align}
for all $a\in \RR$ and $\VEC{x} \in \RR^n$.  Moreover
\begin{align*}
L(\VEC{x}_1+\VEC{x}_2) &=
L\left(2\left( \frac{1}{2}\VEC{x}_1+ \frac{1}{2}\VEC{x}_2\right)\right)
= 2 L\left(\frac{1}{2}\VEC{x}_1+ \frac{1}{2}\VEC{x}_2\right)
= 2 \left( f\left( \frac{1}{2}\VEC{x}_1+ \frac{1}{2}\VEC{x}_2\right)
- \tilde{\VEC{x}} \right) \\
&= 2 \left( \frac{1}{2} f(\VEC{x}_1)+ \frac{1}{2}f(\VEC{x}_2)
- \tilde{\VEC{x}} \right)
= (f(\VEC{x}_1) - \tilde{\VEC{x}}) + (f(\VEC{x}_2) - \tilde{\VEC{x}})
= L(\VEC{x}_1) + L(\VEC{x}_2)
\end{align*}
for all $\VEC{x}_1,\VEC{x}_2 \in \RR^n$, where we have used
(\ref{singHEq1}) to get the second equality.
\end{proof}

It follows from the previous proposition that an affine function
$\displaystyle f:\RR^n \to \RR^m$ is completely determined by it values at
$\VEC{v}_j$ for $0 \leq j \leq n$ if
$\displaystyle \RR^n = \Span \{ \VEC{v}_j: 0\leq j \leq n\}$.
Moreover, we have that $\displaystyle f\left(\sum_{j=0}^n a_j \VEC{v}_j\right)
= \sum_{j=0}^n a_j f(\VEC{v}_j)$ if $\displaystyle \sum_{j=0}^n a_j = 1$
for $a_j \in \RR$.

For $k>0$ and $0 \leq j \leq k$, let $F_k^j:\Delta_{k-1} \to \Delta_k$ be the
affine function defined by
\[
F_k^j(\VEC{e}_i) = \begin{cases}
\VEC{e}_i & \quad \text{if} \ i < j \\
\VEC{e}_{i+1} & \quad \text{if} \ i \geq j
\end{cases}
\]
for $0 \leq i \leq k-1$.

\begin{defn}
Let $X$ be a topological space and $R$ be an integral domain.
Given a singular $k$-simplex $\sigma:\Delta_k \to X$ in $X$ with
$k>0$.  The {\bfseries $\mathbf{j}$-face}\index{$j$-face} of
$\sigma$ is the singular $(k-1)$-simplex defined by
$\displaystyle \sigma^{(j)} = \sigma \circ F_k^j:\Delta_{k-1} \to X$.
The {\bfseries boundary}\index{Boundary} of $\sigma$ is defined by
$\displaystyle \partial_k (\sigma) = \sum_{j=0}^k (-1)^j \sigma^{(j)}$.

The {\bfseries boundary operator}\index{Boundary Operator}
$\partial_k : S_k(X;R) \to S_{k-1}(X;R)$ for $k>0$ is
defined by linearity; namely, given a singular $k$-chain
$\displaystyle c = \sum_{j \in J} a_j \sigma_j$, we set
$\displaystyle \partial_k c = \sum_{j \in J} a_j \partial_k(\sigma_j)$.
It is convenient to define $\partial_k = 0$ for $k\leq 0$.
\end{defn}

\begin{prop} \label{propdkmqdk0}
Let $X$ be a topological space.  Then
$\partial_{k-1} \circ \partial_k = 0$ for all $k\in \ZZ$.
\end{prop}

\begin{proof}
The result is obviously true for $k \leq 1$.  Thus, we may assume that
$k>1$.

\stage{i} We first note that
$\displaystyle F_k^i \circ F_{k-1}^j = F_k^j \circ F_{k-1}^{i-1}$
for $0 \leq j < i \leq k$.  To prove this
statement, it suffices to prove it for all $\VEC{e}_q$ for
$1 \leq q \leq k-1$ because we are working with affine functions.  We
have from
\[
F_{k-1}^j(\VEC{e}_q) =
\begin{cases}
\VEC{e}_q & \quad \text{if} \ q < j \\
\VEC{e}_{q+1} & \quad \text{if} \ j \leq q
\end{cases}
\]
that
\[
F_k^i(F_{k-1}^j(\VEC{e}_q)) =
\begin{cases}
\VEC{e}_q & \quad \text{if} \ q < j \\
\VEC{e}_{q+1} & \quad \text{if} \ j \leq q < i -1 \\
\VEC{e}_{q+2} & \quad \text{if} \ i-1 \leq q
\end{cases}
\]
Moreover, we have from
\[
F_{k-1}^{i-1}(\VEC{e}_q) =
\begin{cases}
\VEC{e}_q & \quad \text{if} \ q < i-1 \\
\VEC{e}_{q+1} & \quad \text{if} \ i-1 \leq q
\end{cases}
\]
that
\[
F_k^j(F_{k-1}^{i-1}(\VEC{e}_q)) =
\begin{cases}
\VEC{e}_q & \quad \text{if} \ q < j \\
\VEC{e}_{q+1} & \quad \text{if} \ j \leq q < i -1 \\
\VEC{e}_{q+2} & \quad \text{if} \ i-1 \leq q  
\end{cases}
\]
Thus $\displaystyle F_k^i(F_{k-1}^j(\VEC{e}_q))
= F_k^j(F_{k-1}^{i-1}(\VEC{e}_q))$.

\stage{ii} By linearity, it is enough to proof
$\partial_{k-1} \circ \partial_k = 0$ for singular $k$-simplices.
Given a singular $k$-simplex $\sigma$, we have that
\begin{align*}
\partial_{k-1}(\partial_k(\sigma))
&= \partial_{k-1}\left( \sum_{i=0}^k (-1)^i \sigma \circ F_k^i \right)
= \sum_{j=0}^{k-1} \sum_{i=0}^k (-1)^j (-1)^i
\sigma \circ F_k^i \circ F_{k-1}^j \\
&= \sum_{i=1}^k \sum_{j=0}^{i-1}  (-1)^{j+i}
\sigma \circ F_k^i \circ F_{k-1}^j
+ \sum_{i=0}^{k-1} \sum_{j=i}^{k-1}  (-1)^{j+i}
\sigma \circ F_k^i \circ F_{k-1}^j \\
&= \sum_{i=1}^k \sum_{j=0}^{i-1}  (-1)^{j+i}
\sigma \circ F_k^j \circ F_{k-1}^{i-1}
+ \sum_{j=0}^{k-1} \sum_{i=j+1}^k  (-1)^{j+i-1}
\sigma \circ F_k^j \circ F_{k-1}^{i-1} = 0
\end{align*}
where, for the fourth equality, we have use (i) to get the first sum
and the substitution $(i,j) \mapsto (j, i-1)$ to get the second sum.
Note that each term in the first sum appears with the opposite sign
in the second sum, and we have $0 \leq j < i \leq k$ in both sums.
\end{proof}

\begin{defn}
Let $X$ be a topological space.  Two $k$-cycles $c_1,c_2 \in S_k(X;R)$ are
{\bfseries homologous}\index{Homologous} if there exists
$\tilde{c} \in S_{k+1}(X;R)$ such that $c_1 - c_2 = \partial_{k+1} \tilde{c}$.
We write $c_1 \hsim c_2$.
\end{defn}

The reader should verify that the homologous relation defines an
equivalence relation on $S_k(X;R)$.

\begin{defn}
Let $X$ be a topological space.
A {\bfseries $\mathbf{k}$-cycle}\index{$k$-Cycle}
in $S_k(X;R)$ is a singular $k$-chain $c$ such that $\partial_k c = 0$.
The set of all $k$-cycles in $S_k(X;R)$ is a $R$-module denoted
$Z_k(X;R)$.

A {\bfseries $\mathbf{k}$-boundary}\index{$k$-Boundary}
in $S_k(X;R)$ is a singular $k$-chain $c$ with the property that there
exists a singular $(k+1)$-chain $\tilde{c}$ such that
$\partial_{k+1} \tilde{c} = c$. 
The set of all $k$-boundaries in $S_k(X;R)$ is a $R$-module denoted
$B_k(X;R)$.

The {\bfseries singular $\displaystyle \mathbf{k^{th}}$ homology
module}\index{Homology Module!Singular $k^{th}$ Homology Module} of
$X$ is the $R$-module defined as
$H_k(X;R) = Z_k(X;R) / B_k(X;R)$.
The equivalence class of $H_k(X;R)$ associated to $c \in Z_k(X;R)$ is
denoted $[c]_X$.
\end{defn}

\begin{egg}
The first example is a simple but important example.   \label{eggHk1}
Let $X = \{x\}$, a set with a single element.  A singular $k$-simplex
$\sigma:\Delta_k \to X$ is therefore a constant function defined by
$\sigma(\VEC{x}) = x$ for all $\VEC{x} \in \Delta_k$.  Hence
\[
\partial_k \sigma = \sum_{j=0}^k (-1)^j \sigma^{(i)} =
\begin{cases}
0 & \quad \text{if} \ k \ \text{is odd} \\
\sigma^{(0)} & \quad \text{if} \ k \ \text{is even}
\end{cases}
\]
because $\sigma^{(j)}: \Delta_{k-1} \to X$ is the constant function
defined by $\sigma^{(j)}(\VEC{x}) = x$ for all $\VEC{x} \in \Delta_{k-1}$
for $0 \leq j \leq k$.

It follows that
\[
Z_k(X;R) = \begin{cases}
S_k(X;R) & \quad \text{if} \ k \ \text{is odd or} \ k = 0 \\
\{0\} & \quad  \text{if} \ k \ \text{is even}
\end{cases}
\]
and
\[
B_k(X;R) = \begin{cases}
S_k(X;R) & \quad \text{if} \ k \ \text{is odd} \\
\{0\} & \quad  \text{if} \ k \ \text{is even or} \ k = 0
\end{cases}
\]
Therefore
\[
H_k(X;R) = \begin{cases}
S_0(X;R) & \quad \text{if} \ k = 0 \\
\{[0]_X\} & \quad \text{if} \ \ k > 0
\end{cases}
\]
From now on, we adopt the traditional convention of writing
$H_k(X;R) = 0$ instead of $H_k(X;R) = \{[0]_X\}$.
\end{egg}

\begin{prop} \label{propHkeHkJ}
If the topological space $X$ is the union of distinct (connected)
components $X_j$ for $j \in J$, then
$\displaystyle H_k(X;R) = \bigoplus_{j\in J}H_k(X_j;R)$.
\end{prop}

\begin{proof}[Proof (Sketch)]
If $\sigma \in S_k(X;R)$, then $\sigma(\Delta_k) \subset X_j$ for some
$j\in J$ because $\sigma(\Delta_k)$ is connected.  Therefore
$\displaystyle S_k(X;R) = \bigoplus_{j\in J} S(X_j;R)$.  Moreover
$\partial_q :S_q(X_j;R) \to S_{q-1}(X_j;R)$ for all $q \in \ZZ$ and all
$j \in J$.  Thus $\displaystyle Z_k(X;R) = \bigoplus_{j\in J} Z(X_j;R)$
and $\displaystyle B_k(X;R) = \bigoplus_{j\in J} B(X_j;R)$.  The
conclusion of the proof follows from a standard result about the
quotient space of direct sums of modules in algebra.
\end{proof}

\begin{prop} \label{propH0XR}
Let $X$ be a path-connected topological space.  Then $H_0(X;R) \cong R$.
\end{prop}

\begin{proof}
Let $e_x:\Delta_0 \to X$ be the singular $0$-simplex defined by
$e_x(\VEC{e}_0) = x$ for $x \in X$.

\stage{i} We define a map $h: S_0(X;R) \to R$ as it follows.
Given $c \in S_0(X;R)$, we have that
$\displaystyle c = \sum_{j \in J} a_j e_{x_j}$ where $J \subset \NN$
is a finite set, $a_j \in R$ and $x_j \in X$ for all $j \in J$.
We set $\displaystyle \epsilon(c) = \sum_{j \in J} a_j$.
We prove that if $c_1$ and $c_2$ are two homologous singular $0$-chains,
then $\epsilon(c_1) = \epsilon(c_2)$.  If $c_1$ and $c_2$ are two homologous
singular $0$-chains, then $c_1 - c_2 = \partial_1 c$ for some
$c \in S_1(X;R)$.  Hence
$\epsilon(c_1) - \epsilon(c_2) = \epsilon(c_1-c_2) = \epsilon(\partial_1 c)$.
To prove that $\epsilon(\partial_1 c) = 0$, it suffices to prove that
$\epsilon(\partial_1 \sigma) = 0$ for any singular $1$-simplex in $X$.
We effectively have that
$\displaystyle \epsilon(\partial_1 \sigma)
= \epsilon(\sigma \circ F^0_1 - \sigma \circ F^1_1)
= \epsilon(e_{\sigma(\VEC{e}_1)} - e_{\sigma(\VEC{e}_0)}) = 1 - 1 = 0$.

We may therefore define a map $\tilde{\epsilon}: H_0(X;R) \to R$ as it follows.
Given a singular $0$-chain $\displaystyle c = \sum_{j \in J} a_j e_{x_j}$
where $J \subset \NN$ is a finite set, $a_j \in R$ and $x_j \in X$ for
all $j \in J$. we set
$\displaystyle \tilde{\epsilon}([c]_X) = \sum_{j \in J} a_j$.

\stage{ii} Let $c$ be a singular $0$-chain.  We have that
$\displaystyle c = \sum_{j \in J} a_j e_{x_j}$ where $J \subset \NN$
is finite, $a_j \in R$ and $x_j \in X$ for all $j \in J$.
Choose $\tilde{x} \in X$.  Given any $x \in X$, there exists a path
$\tau_x:[0,1] \to X$ such that $\tau_x(0) = \tilde{x}$ and $\tau_x(1) = x$
because $X$ is path-connected.
Let $\hat{\tau}_x:\Delta_1 \to X$ be the singular $1$-simplex defined
by $\hat{\tau}_x((1-t)\VEC{e}_0 + t \VEC{e}_1) = \tau_x(t)$ for $t \in [0,1]$.
Since $\partial_1 \hat{\tau}_x = e_x - e_{\tilde{x}}$,
we have that $e_x \hsim e_{\tilde{x}}$.
Therefore $\displaystyle c \hsim \Big( \sum_{j \in J} a_j \Big) e_{\tilde{x}}$.

\stage{iii} The map $\tilde{\epsilon}$ is a module homomorphism.  If
$r \in R$ and $\displaystyle c = \sum_{j \in J} a_j e_{x_j}$
for some $a_j \in R$ and $x_j \in X$, then
$\displaystyle rc = \sum_{j \in J} ra_j e_{x_j}$.
Hence $\displaystyle \tilde{\epsilon}([rc]_X) = \sum_{j \in J} ra_j =
r \sum_{j \in J} a_j = r \tilde{\epsilon}([c]_X)$.
If $\displaystyle c_1 = \sum_{j \in J} a_j e_{x_j}$
and
$\displaystyle c_2 = \sum_{j \in J} b_j e_{x_j}$ for some $a_j, b_j \in R$
and $x_j \in X$, then
$\displaystyle c_1 + c_ 2 = \sum_{j \in J} (a_j+b_j) e_{x_j}$.
Therefore
$\displaystyle \tilde{\epsilon}([c_1+c_2]_X) = \sum_{j \in J} (a_j+b_j)
= \sum_{j \in J} a_j+ \sum_{j \in J}b_j = \tilde{\epsilon}([c_1]_X) +
\tilde{\epsilon}([c_2]_X)$.

The map $\tilde{\epsilon}$ is obviously onto.  It is also injective.
Suppose that
$\displaystyle c_1 = \sum_{j \in J} a_j e_{x_j}$ and
$\displaystyle c_2 = \sum_{j \in J} b_j e_{x_j}$ for some $a_j, b_j \in R$
and $x_j \in X$.  If $\tilde{\epsilon}([c_1]_X) = \tilde{\epsilon}([c_2]_X)$,
then $\displaystyle \sum_{j \in J} a_j = \sum_{j \in J} b_j$.
Hence, we get from (ii) that
$\displaystyle c_1 \hsim \sum_{j \in J} a_j e_{\tilde{x}}
= \sum_{j \in J} b_j e_{\tilde{x}} \hsim c_2$.  Thus $[c_1]_X = [c_2]_X$.
\end{proof}

The proof of the next proposition is left to the reader.

\begin{prop}
Let $X$ and $Y$ be two topological spaces, and $f : X \to Y$ be a
continuous function.  The map
$S_k(f): S_k(X;R) \to S_k(Y;R)$ defined by
$\displaystyle S_k(f)\Big( \sum_{j \in J} a_j \sigma_j \Big) =
\sum_{j \in J} a_j\, f\circ \sigma$ for
is a module homomorphism.
\end{prop}

\begin{prop} \label{propSkPropr}
Let $X$, $Y$ and $Z$ be three topological spaces.  If $f : X \to Y$
and $g:Y \to Z$ are two continuous functions, and $k \in \ZZ$, then
\begin{enumerate}
\item $S_k(g\circ f) = S_k(g) \circ S_k(f)$ and
\item $\partial_k \big( S_k(f) c \big) = S_{k-1}(f) (\partial_k c)$ for all
$c \in S_k(X;R)$.
\end{enumerate}
\end{prop}

\begin{proof}
The proof of (1) is left to the reader.  As usual, we only need to
prove (2) for singular $k$-simplices.  Let $\sigma$ be a singular
$k$-simplex, then
\begin{align*}
S_{k-1}(f) (\partial_k(\sigma))
&= S_{k-1}(f) \left( \sum_{i=0}^k (-1)^i \sigma \circ F_k^i \right)
= \sum_{i=0}^k (-1)^i f \circ \sigma \circ F_k^i \\
&= \partial_k ( f\circ \sigma)
= \partial_k ( S_k(f) (\sigma)) \ .  \qedhere
\end{align*}
\end{proof}

\begin{cor} \label{corDefHkf}
Let $X$ and $Y$ be two topological spaces, and $f : X \to Y$ be a
continuous function.  Then
$S_k(f) (Z_q(X;R)) \subset Z_k(Y;R)$ and
$S_k(f) (B_q(X;R)) \subset B_k(Y;R)$.  Moreover,
$H_k(f) : H_k(X;R) \to H_k(Y;R)$ defined by
$H_k(f)([c]_X) = [S_k(f)(c)]_Y$ for all $c \in H_k(X;R)$ is a module
homomorphism.
\end{cor}

\begin{proof}
Since $\partial_k \big( S_k(f) c \big) = S_{k-1}(f) (\partial_k c)$ for all
$c \in S_k(X;R)$, we have that $S_k(f) (Z_q(X;R)) \subset Z_k(Y;R)$ and
$S_k(f) (B_q(X;R)) \subset B_k(Y;R)$.

To prove that $H_k(f)([c]_X)$ is well defined, we need to show that
$[S_k(f)(c)]_Y$ is independent of the representative $c$ of the
equivalence class $[c]_X$ that is used.
Suppose that $\tilde{c} \hsim c$.  Then $\tilde{c} = c + \partial_{k+1} b$
for some $b \in S_{k+1}(X;R)$.  Hence
$S_k(f)(\tilde{c}) = S_k(f)(c) + S_k(f)(\partial_{k+1} b)
= S_k(f)(c) + \partial_{k+1} (S_k(f)(b))$ with
$S_k(f)(b) \in S_{k+1}(Y;R)$.  Thus
$[S_k(f)(\tilde{c})]_Y = [S_k(f)(c)]_Y$.

It is easy to verify that
$H_k(f)([c_1]_X +[c_2]_X) = H_k(f)([c_1]_X) + H_k(f)([c_2]_X)$ and \\
$H_k(f)(\alpha[c]_X) = \alpha H_k(f)([c]_X)$ for
$[c]_X, [c_1]_X,[c_2]_X \in H_k(X;R)$
and $\alpha \in R$.  This proves that $H_k(f)$ is a module
homomorphism.
\end{proof}

The following results follow from Proposition~\ref{propSkPropr}.

\begin{cor} \label{corCompHgHf}
If $X$, $Y$ and $Z$ are topological spaces and 
$f:X\to Y$ and $g:Y\to Z$ are two continuous functions, then
$H_k(g \circ f) = H_k(g) \circ H_k(f)$.
\end{cor}

\begin{cor}  \label{corIsoHkXY}
Suppose that $X$ and $Y$ are two topological spaces.  If $X$ and $Y$ are
homeomorphic, then $H_k(X;R) \cong H_k(Y;R)$.
\end{cor}

\subsection{Exact Sequences}

\begin{defn}
Let $X_j$ be modules over an integral domain $R$ and
$f_j:X_j\to X_{j+1}$ be continuous functions.  We say that the
sequence
\[
\xymatrix{ \ar[r]^(0.4){f_{j-1}} & X_j \ar[r]^{f_j}
& X_{j+1} \ar[r]^{f_{j+1}} & X_{j+2} \ar[r]^-{f_{j+2}} & }
\]
is a {\bfseries (long) exact sequence}\index{Exact Sequence}\index{Long
Exact Sequence|see{Exact Sequence}}
if the kernel of a map in the sequence equal the image of the
previous map in the sequence.
\end{defn}

We often have to deal with the following type of sequences.

\begin{defn} \label{defnSESequ}
Let $X$, $Y$ and $Z$ be modules over an integral domain $R$.  Moreover
let $f:X\to Y$ and $g:Y\to Z$ be continuous functions.  We say that the
sequence
\[
\xymatrix{ 0 \ar[r] & X \ar[r]^{f} & Y \ar[r]^{g} & Z \ar[r] & 0 }
\]
is a {\bfseries short exact sequence}\index{Short Exact Sequence} if
the kernel of a map in the sequence equal the image of the
previous map (if any) in the sequence.
\end{defn}

By tradition, the expression $0 \rightarrow X$ denotes the map
from the set $\{0\}$ to $X$ that maps $0$ to the null element in $X$.
Similarly, the expression $Z \rightarrow 0$ denotes the map from
$Z$ to $\{0\}$ that map all elements in $Z$ to $0$.

\begin{defn}
Let $X$, $Y$ and $Z$ be modules over an integral domain $R$.
We say that a short exact sequence
\[
\xymatrix{ 0 \ar[r] & X \ar[r]^{f} & Y \ar[r]^{g} & Z \ar[r] & 0 }
\]
is {\bfseries split}\index{Short Exact Sequence!Split} if there
exists a function $h:Y\to X$ such that $h \circ f = \Id_X$.
\end{defn}

The next result is kind of surprising.

\begin{prop} \label{propSplit2Cond}
Let $X$, $Y$ and $Z$ be modules over an integral domain $R$.
A short exact sequence
\[
\xymatrix{ 0 \ar[r] & X \ar[r]^{f} & Y \ar[r]^{g} & Z \ar[r] & 0 }
\]
is split if and only if there exists a function $k:Z \to Y$ such that
$g \circ k = \Id_Z$.
\end{prop}

\begin{proof}
\stage{$\mathbf{\Rightarrow}$} Since the short exact sequence is
split, there exists $h:Y\to X$ such that $h \circ f = \Id_X$.
To define $k:Z \to Y$, consider $z \in Z$.  Since $g$ is onto, there
exists $y \in Y$ such that $g(y) = z$.  Let $k(z) = y - f(h(y))$.

We first show that $k$ is well defined.  Suppose that $\tilde{y} \in Y$
is such that $g(\tilde{y}) = z$.  Then
$g(y - \tilde{y}) = g(y) - g(\tilde{y}) = 0$.  So
$y - \tilde{y} \in \KE(g) = \IMG(f)$.  Thus there exists $x\in X$ such
that $f(x) = y - \tilde{y}$.  We get
\begin{align*}
\big(y - f(h(y))\big) - \big(\tilde{y} - f(h(\tilde{y}))\big)
&= y - \tilde{y} - f(h(y - \tilde{y})) \\
&= f(x) - f(h(f(x))) = f(x) - f(x) = 0
\end{align*}
because $h \circ f = \Id_X$.

We also have that
$g(k(z)) = g(y - f(h(y))) = g(y) - g(f(h(y))) = g(y) = z$
for all $z \in Z$ because $\KE(g) = \IMG(f)$, where $y\in Y$ is the $y$
in the definition of $k$ at $z \in Z$.  Thus $g \circ k = \Id_Z$.

We must not forget to verify that $k$ is a homomorphism of modules.
Given $z, \tilde{z} \in Z$.  Suppose that $y,\tilde{y}$ are such that
$g(y) = z$ and $g(\tilde{y}) = \tilde{z}$.  Then
$g(y + \tilde{y}) = z + \tilde{z}$ and
$k(z + \tilde{z}) = y + \tilde{y} + f(h(y + \tilde{y}))
= \big(y + f(h(y))\big) + \big(\tilde{y} + f(h(\tilde{y}))\big)
= k(z) + k(\tilde{z})$.
Proceeding similarly, we also get that
$k(rz) = rk(z)$ for all $z\in Z$ and $r \in R$.

\stage{$\mathbf{\Leftarrow}$} We assume that there exists $k:Z\to Y$
such that $k \circ g = \Id_Z$.  To define $h:Y\to X$, consider $y\in Y$.
Since $g(y - k(g(y))) = g(y) - g(k(g(y))) = g(y) - g(y) = 0$ because
$k \circ g = \Id_z$, we have that $y-k(g(y)) \in \KE(g)$.  Since
$\IMG(f) = \KE(g)$, there exists $x \in X$ such that $f(x) = y -k(g(y))$.
We set $h(y) = x$.  The function $h$ is well defined because $f$ is
one-to-one.  We also have that $h(f(x)) = x$ for all $x\in X$ because
$f(x) - k(g(f(x))) = f(x)$ since $\KE(g)= \IMG(f)$ and $f$ is one-to-one.
Thus $h \circ f = \Id_X$.  Finally, it is easy to show that $h$ is an
homomorphism of modules because $f$ is one.
\end{proof}

\begin{rmk}
It should be pointed out that $h$ and $k$ in the previous proposition
yields the following short exact sequence
\[
\xymatrix{ 0 & X \ar[l] & Y \ar[l]_{h} & Z \ar[l]_{k} & 0 \ar[l] }
\]
We have that $h \circ f = \Id_X$ ensures that $h$ is onto and
$g \circ k = \Id_Z$ ensures that $k$ is one-to-one.
We also have that $\IMG(k) = \KE(h)$.  To prove that
$\IMG(k) \subset \KE(h)$, consider $\tilde{y} = k(z)$ for $z \in Z$.
Then $\tilde{y} = y - f(h(y))$ for some $y \in Y$.  Hence
$h(\tilde{y}) = h(y - f(h(y))) = h(y) - h(f(h(y))) = h(y) - h(y) = 0$
because $h\circ f = \Id_X$.  Thus $\tilde{y} \in \KE(h)$.
To prove that $\IMG(k) \supset \KE(h)$, suppose that $y \in \KE(h)$.
Then $k(g(y)) = y - f(h(y)) = y$ because $h(y) = 0$.  Thus $y \in \IMG(k)$.
\end{rmk}

\begin{prop} \label{propSplitOplus}
A short exact sequence
\begin{equation} \label{SplitOplusEq1}
\xymatrix{ 0 \ar[r] & X \ar[r]^{f} & Y \ar[r]^{g} & Z \ar[r] & 0 }
\end{equation}
is split if and only if $Y \cong X \oplus Z$.
\end{prop}

\begin{proof}
\stage{i} Suppose that (\ref{SplitOplusEq1}) is split.  We then have
that
\[
\xymatrix{ 0 \ar[r] & X \ar@/^/[r]^{f} & Y \ar@/^/[l]^{h}\ar@/^/[r]^{g}
& Z \ar@/^/[l]^{k}\ar[r] & 0 }
\]
is exact in both direction.

Given $y \in f(X) \cap k(Z)$, then
$y = f(x) = k(z)$ for some $x\in X$ and $z \in Z$.  Thus
$0 = g(f(x)) = g(y) = g(k(z)) = z$ because $\KE(g) = \IMG(f)$.
Hence $y = k(z) = k(0) = 0$.
We get that $f(X) \cap k(Z) = \{0\}$.

Given $y \in Y$, it follows from
$g(y - k(g(y)) = g(y) - g(k(g(y))) = g(y) - g(y) = 0$ that
$y - k(g(y)) = f(x)$ for some $x \in X$ because $\KE(g) = \IMG(f)$.
Thus $y = f(x) + k(g(y)) \in f(X) + k(Z)$.
We get that $Y \subset f(X) + k(Z)$.

Therefore $Y = f(X) \oplus k(Z)$.  Since $X \cong f(X)$ and
$Z \cong k(Z)$ because $f$ and $k$ are isomorphism, we have that
$Y \cong X \oplus Z$.

\stage{ii} Suppose that $Y \cong X \oplus Z$.  So there exists an
isomorphism $\Phi:Y \to X \oplus Z$.  Let\\
$\tilde{h}:X \oplus Z \to X$ be the function defined by
$\tilde{h}(x,z) = x$ for all $(x,z) \in X \oplus Z$,\\
$\tilde{g}:X \oplus Z \to Z$ be the function defined by
$\tilde{g}(x,z) = z$ for all $(x,z) \in X \oplus Z$,\\
$\tilde{f}:X \to X \oplus Z$ be the function defined by
$\tilde{f}(x) = (x,0)$ for all $x \in X$ and\\
$\tilde{k}:Z \to X \oplus Z$ be the function defined by
$\tilde{k}(z) = (0,z)$ for all $z \in Z$.
Then
\[
\xymatrix@C+2ex{ 0 \ar[r] & X \ar@/^/[r]^-{\tilde{f}}
& X \oplus Z \ar@/^/[l]^-{\tilde{h}}\ar@/^/[r]^-{\tilde{g}}
& Z \ar@/^/[l]^-{\tilde{k}}\ar[r] & 0 }
\]
is exact in both direction.  Let
$\displaystyle f = \Phi^{-1}\circ \tilde{f}$, $g = \tilde{g} \circ \Phi$ and
$h = \tilde{h} \circ \Phi$.  We get that
\[
\xymatrix{ 0 \ar[r] & X \ar[r]^{f} & Y \ar[r]^{g} & Z \ar[r] & 0 }
\]
is a short exact sequence and it is split because
$h \circ f = (\tilde{h} \circ \Phi) \circ (\Phi^{-1}\circ \tilde{f})
= \tilde{h} \circ \tilde{f} = \Id_X$.
\end{proof}

\subsection{Chain Complexes}

\begin{defn}
Let $R$ be an integral domain.  A
{\bfseries chain complex}\index{Chain Complex} is a sequence
$\C = \{ (C_k,\partial_k) \}_{k\in \ZZ}$ where the sets $C_k$ are
$R$-modules and the maps $\partial_k:C_k \to C_{k-1}$ are homomorphism
such that $\partial_k \circ \partial_{k+1} = 0$ for $k \in \ZZ$.
\end{defn}

A diagram representing a chain complex is given below.
\[
\xymatrix{
\ar[r]^-{\partial_{k+2}} & C_{k+1} \ar[r]^{\partial_{k+1}}
& C_k \ar[r]^{\partial_k} & C_{k-1} \ar[r]^-{\partial_{k-1}} &
}
\]

\begin{rmk}
Unless otherwise stated, we always assume that $C_k = \{0\}$ for
$k<0$.  This is sufficient for our applications.
\end{rmk}

We have already worked with chain complexes before.  One such example is
given by the sequence of pairs
$(C_k,\partial_k) = (S_k(X;R), \partial_k)$ for $k \in \ZZ$
defined at the beginning of the previous section.

\begin{defn}
A {\bfseries chain map}\index{Chain Map} between two chain complexes
given by $\C = \{(C_k,\partial_k)\}_{k\in \ZZ}$ and
$\tilde{\C} = \{ (\tilde{C}_k,\tilde{\partial}_k)\}_{k\in \ZZ}$
is a sequence $\F = \{f_k\}_{k\in \ZZ}$ where the maps
$f_k:C_k \to \tilde{C}_k$ are homomorphism such that
$\tilde{\partial}_k \circ f_k = f_{k-1} \circ \partial_k$ for $k \in \ZZ$.
\end{defn}

The previous definition yields the following commutative diagram.
\[
\xymatrix{
\ar[r]^-{\partial_{k+2}} & C_{k+1}  \ar[r]^-{\partial_{k+1}} \ar[d]^-{f_{k+1}}
& C_k \ar[r]^-{\partial_k} \ar[d]^-{f_k} & C_{k-1} \ar[r]^-{\partial_{k-1}} 
\ar[d]^-{f_{k-1}} & \\
\ar[r]_-{\tilde{\partial}_{k+2}} & \tilde{C}_{k+1}
\ar[r]_-{\tilde{\partial}_{k+1}} & \tilde{C}_k \ar[r]_-{\tilde{\partial}_k}
& \tilde{C}_{k-1} \ar[r]_-{\tilde{\partial}_{k-1}} &
}
\]

As we did before, we may define $k$-cycles, $k$-boundaries and
$k$-singular homology modules.

\begin{defn}
Consider a chain complex given by $\C = \{(C_k,\partial_k)\}_{k\in \ZZ}$.
A {\bfseries $\mathbf{k}$-cycle}\index{$k$-Cycle}
in $\C$ is an element $c \in C_k$ such that $\partial_k(c) = 0$.
The set of all $k$-cycles in $C_k$ is a $R$-module denoted $Z_k(\C)$.

A {\bfseries $\mathbf{k}$-boundary}\index{$k$-Boundary}
in $\C$ is an element $c \in C_k$ with the property that there exists an
element $\tilde{c} \in C_{k+1}$ such that $\partial_{k+1}(\tilde{c}) = c$.
The set of all $k$-boundaries in $C_k$ is a $R$-module denoted
$B_k(\C)$.

The {\bfseries singular $\displaystyle \mathbf{k^{th}}$ homology
module}\index{Homology Module!Singular $k^{th}$ Homology Module} of
$\C$ is the quotient space $H_k(\C) = Z_k(\C) / B_k(\C)$.  The
equivalence class of $H_k(\C)$ associated to $c \in Z_k(\C)$ is denoted $[c]$.
\end{defn}

It follows from the definition that a chain map
$\F = \{f_k\}_{k\in\ZZ}$ between two chain complexes
$\C = \{(C_k,\partial_k)\}_{k\in \ZZ}$ and
$\tilde{\C} = \{ (\tilde{C}_k,\tilde{\partial}_k)\}_{k\in \ZZ}$
sends $k$-cycles in $\C$ to $k$-cycles in $\tilde{\C}$, and
$k$-boundaries in $\C$ to $k$-boundaries in $\tilde{\C}$.
Hence, as we did in Corollary~\ref{corDefHkf}, we may define
$H_k(\F): H_k(\C) \to H_k(\tilde{\C})$ by
$H_k(\F)([c]) = [f_k(c)]$ for all $c \in Z_k(\C)$.

The next result is traditionally called a lemma but it really should
be a theorem because it is used in very many proofs in algebraic
topology.

\begin{lemma}[Zig-Zag Lemma]  \label{lemZigZag}
Suppose that $\F = \{f_k\}_{k\in\ZZ}$ is a    \index{Zig-Zag Lemma}
chain map between the chain
complexes $\displaystyle \A = \{(A_k,\partial_k^{\A})\}_{k\in \ZZ}$ and
$\displaystyle \BB = \{(B_k,\partial_k^{\BB})\}_{k\in \ZZ}$, and
$\GG = \{g_k\}_{k\in\ZZ}$  is a chain map between the chain complexes
$\displaystyle \BB = \{(B_k,\partial_k^{\BB})\}_{k\in \ZZ}$ and
$\displaystyle \C = \{(C_k,\partial_k^{\C})\}_{k\in \ZZ}$.
If
\[
\xymatrix{
0 \ar[r] & A_k \ar[r]^{f_k} & B_k \ar[r]^{g_k} & C_k \ar[r] & 0
}
\]
is an exact sequence for all $k$, then there exists a long exact
sequence
\[
\xymatrix@C+2ex{
\ar[r]^-{\tilde{\partial}_{\,k+1}} & H_k(\A) \ar[r]^{H_k(\F)}
& H_k(\BB) \ar[r]^{H_k(\GG)} & H_k(\C) \ar[r]^-{\tilde{\partial}_{\,k}}
& H_{k-1}(\A) \ar[r]^-{H_{k-1}(\F)} &
}
\]
where $\tilde{\partial}_{\,k}$ is induced by
$\displaystyle \partial_k^{\BB}$.
\end{lemma}

\begin{proof}
The proof is based on the following commutative diagram.
\[
\xymatrix{
0 \ar[r] & A_{k+1} \ar[r]^{f_{k+1}}\ar[d]^{\partial_{k+1}^{\A}}
& B_{k+1} \ar[r]^{g_{k+1}}\ar[d]^{\partial_{k+1}^{\BB}}
& C_{k+1} \ar[r]\ar[d]^{\partial_{k+1}^{\C}} & 0 \\
0 \ar[r] & A_k \ar[r]^{f_k}\ar[d]^{\partial_k^{\A}}
& B_k \ar[r]^{g_k}\ar[d]^{\partial_k^{\BB}}
& C_k \ar[r]\ar[d]^{\partial_k^{\C}} & 0 \\
0 \ar[r] & A_{k-1} \ar[r]^{f_{k-1}} & B_{k-1} \ar[r]^{g_{k-1}}
& C_{k-1} \ar[r] & 0
}
\]

\stage{i} Our first task is to define $\tilde{\partial}_k$ on $H_k(\C)$.
Given $[c] \in H_k(\C)$, there exists $b \in B_k$ such that
$g_k(b) = c$ because $g_k$ is onto.  We have that
$\displaystyle \partial_k^{\BB}(b) \in B_{k-1} \cap \IMG(f_{k-1})$ because
$\displaystyle \partial_k^{\BB}(b) \in \KE(g_{k-1}) = \IMG(f_{k-1})$ since
$\displaystyle g_{k-1}(\partial_k^{\BB}(b)) = \partial_k^{\C}(g_k(b))  
= \partial_k^{\C}(c) = 0$ for $[c] \in H_k(\C)$.  There exists
$a \in A_{k-1}$ such that $\displaystyle f_{k-1}(a) = \partial_k^{\BB}(b)$.  In
fact, $a$ is unique because $f_{k-1}$ is one-to-one.

We set $\tilde{\partial}_k([c]) = [a] \in H_{k-1}(\A)$.

We have to show that $\tilde{\partial}_k([c])$ is well defined;
namely, that it is independent of the representative of the
equivalence class $[c] \in H_k(\C)$ that we choose to find $a \in A_{k-1}$.
Suppose that $\tilde{c} \in Z_k(\C)$ is such that $[\tilde{c}] = [c]$.
As before, there exists $\tilde{b} \in B_k$ such that
$g_k(\tilde{b}) = \tilde{c}$, and
$\tilde{a} \in A_{k-1}$ such that $\displaystyle f_{k-1}(\tilde{a})
= \partial_k^{\BB}(\tilde{b})$.

Since $[c]=[\tilde{c}]$, we have that
$\displaystyle c - \tilde{c} = \partial_{k+1}^{\C}(\breve{c})$ for some
$\breve{c} \in C_{k+1}$.  Choose $\breve{b} \in B_{k+1}$ such that
$g_{k+1}(\breve{b}) = \breve{c}$.  This is possible because $g_{k+1}$
maps $B_{k+1}$ onto $C_{k+1}$.   Then
\[
g_k\big( b - \tilde{b} - \partial_{k+1}^{\BB}(\breve{b})\big)
= g_k(b) - g_k(\tilde{b}) - \partial_{k+1}^{\C}(g_{k+1}(\breve{b}))\big)
= c - \tilde{c} - \partial_{k+1}^{\C}(\breve{c}) = 0 \ .
\]
Since $\displaystyle
b - \tilde{b} - \partial_{k+1}^{\BB}(\breve{b}) \in \KE(g_k) = \IMG(f_k)$,
there exists $\breve{a} \in A_k$ such that
$\displaystyle f_k(\breve{a}) = b - \tilde{b} - \partial_{k+1}^{\BB}(\breve{b})$.
Moreover, since
\begin{align*}
f_{k-1}( \partial_k^{\A}(\breve{a})) 
&= \partial_k^{\BB}(f_k(\breve{a}))
= \partial_k^{\BB}\big( b - \tilde{b} -\partial_{k+1}^{\BB}(\breve{b}) \big)
= \partial_k^{\BB}(b) - \partial_k^{\BB}(\tilde{b}) \\
&=  f_{k-1}(a) - f_{k-1}(\tilde{a}) = f_{k-1}(a - \tilde{a})
\end{align*}
and $f_{k-1}$ is one-to-one, we get that
$\displaystyle a - \tilde{a} = \partial_k^{\A}(\breve{a})$ with
$\breve{a} \in A_k$.  Therefore $[a]=[\tilde{a}]$.

To prove that $\tilde{\partial}_k$ is an homomorphism, suppose that
$[c], [\tilde{c}] \in H_k(\C)$.  We have by definition of
$\tilde{\partial}_k$ that $\tilde{\partial}_k([c]) = [a]$ and
$\tilde{\partial}_k([\tilde{c}]) = [\tilde{a}]$ where
$\displaystyle f_{k-1}(a) = \partial_k^{\BB}(b)$
with $g_k(b) = c$ for some $b \in B_k$ and
$\displaystyle f_{k-1}(\tilde{a}) = \partial_k^{\BB}(\tilde{b})$ 
with $g_k(\tilde{b}) = \tilde{c}$ for some $\tilde{b} \in B_k$.
Since
$\displaystyle f_{k-1}(a + \tilde{a}) = \partial_k^{\BB}(b + \tilde{b})$ 
with $g_k(b+ \breve{b}) = c + \breve{c}$, we get
\[
\tilde{\partial}_k([c]+[\tilde{c}])
= \tilde{\partial}_k([c+\tilde{c}]) = [ a + \tilde{a}]
= [a] + [\tilde{a}] =
\tilde{\partial}_k([c]) + \tilde{\partial}_k([\tilde{c}]) \ .
\]
Similarly, $\tilde{\partial}_k(r[c]) = r \tilde{\partial}_k([c])$
for all $r \in R$.

\stage{ii} We prove that $\IMG(H_k(\F)) = \KE(H_k(\GG))$.
Since $g_k \circ f_k = 0$ on $A_k$, we get that
$\big(H_k(\GG) \circ H_k(\F)\big)([c])
= [(g_k \circ f_k)(c)] = [0]$ for all $[c] \in H_k(\A)$.  Thus
$\IMG(H_k(\F)) \subset \KE(H_k(\GG))$.

Suppose that $[b] \in \KE(H_k(\GG))$.  We therefore have that
$H_k(\GG)([b]) = [g_k(b)] = [0]$.  Thus
$\displaystyle g_k(b) = \partial_{k+1}^{\C}(c)$ for some $c \in C_{k+1}$.
Since $g_{k+1}$ is onto, there exists $\tilde{b} \in B_{k+1}$ such
that $g_{k+1}(\tilde{b}) = c$.  We get
\[
g_k(b - \partial_{k+1}^{\BB}(\tilde{b}))
= g_k(b) - \partial_{k+1}^{\C}(g_{k+1}(\tilde{b}))
= g_k(b) - \partial_{k+1}^{\C}(c) = 0 \ .
\]
Since $\IMG(f_k) = \KE(g_k)$, there exists $a \in A_k$ such that
$f_k(a) = b - \partial_{k+1}^{\BB}(\tilde{b})$.

We have that $a \in Z_k(\A)$ because $b \in Z_k(\BB)$ implies that
\[
f_{k-1}(\partial_k^{\A}(a)) = \partial_k^{\BB}(f_k(a))
= \partial_k^{\BB}\big( b - \partial_{k+1}^{\BB}(\tilde{b}) \big)
= \partial_k^{\BB}(b) = 0 
\]
and $f_{k-1}$ being one-to-one implies that
$\displaystyle \partial_k^{\A}(a) = 0$.  Thus $[a] \in H_k(\A)$.
Moreover,\\
$\displaystyle H_k(\F)([a]) = [f_k(a)]
= [b - \partial_{k+1}^{\BB}(\tilde{b})] = [b]$.
Thus $[b] \in \IMG(H_k(\F)$.  This proves that
$\KE(H_k(\GG)) \subset \IMG(H_k(\F))$.

\stage{iii} We prove that $\IMG(H_k(\GG)) = \KE(\tilde{\partial}_k)$.
Given $[c] \in H_k(\C)$, we have that $\tilde{\partial}_k([c]) = [a]$ where
$\displaystyle f_{k-1}(a) = \partial_k^{\BB}(b)$ with $g_k(b) = c$ for
some $b \in B_k$.

If $[c] \in \IMG(H_k(\GG))$, then $[c] = H_k(\GG)([b]) = [g_k(b)]$
for some $[b] \in H_k(\BB)$.  In particular, $b \in Z_k(\BB)$.
Thus $\displaystyle f_{k-1}(a) = \partial_k^{\BB}(b) = 0$ implies that
$a=0$ because $f_{k-1}$ is one-to-one.  Thus $\tilde{\partial}_k([c]) = [0]$;
namely, $[c] \in \KE(\tilde{\partial}_k)$.  We have proved that
$\IMG(H_k(\GG) \subset \KE(\tilde{\partial}_k)$.

Suppose that $[c] \in \KE(\tilde{\partial}_k)$.
Then $\displaystyle a = \partial_k^{\A}(\tilde{a})$ for some
$\tilde{a} \in A_k$.  We have that $b - f_k(\tilde{a}) \in Z_k(\BB)$ because
\[
\partial_k^{\BB}(b - f_k(\tilde{a}))
= \partial_k^{\BB}(b) - f_{k-1}(\partial_k^{\A}(\tilde{a}))
= \partial_k^{\BB}(b) - f_{k-1}(a) = 0
\]
and $[c] = H_k(\GG)\big([b-f_k(\tilde{a})]\big)$ because
\[
g_k\big(b - f_k(\tilde{a})\big)
= g_k(b) - g_k(f_k(\tilde{a})) = g_k(b) - 0 = c \ .
\]
Thus $[c] \in \IMG(H_k(\GG))$.  This proves that
$\KE(\tilde{\partial}_k) \subset \IMG(H_k(\GG)$.

\stage{iv} We prove that $\IMG(\tilde{\partial}_k) = \KE(H_{k-1}(\F))$.
If $[a] \in \IMG(\tilde{\partial}_k)$, then
$\displaystyle f_{k-1}(a) = \partial_k^{\BB}(b)$ for some $b \in B_k$.
Hence $\displaystyle 
H_{k-1}(\F)([a]) = [f_{k-1}(a)] = [ \partial_k^{\BB}(b) ] = [0]$.
Thus $[a] \in \KE(H_{k-1}(\F))$.  Hence
$\IMG(\tilde{\partial}_k) \subset \KE(H_{k-1}(\F))$.

Suppose that $[a] \in \KE(H_{k-1}(\F))$.  Since $H_{k-1}(\F)([a])
= [f_{k-1}(a)] = [0]$, we get that
$\displaystyle f_{k-1}(a) = \partial_k^{\BB}(b)$
for some $b \in B_k$.  Let $c = g_k(b)$.  We have that $c \in Z_k(C)$
because
\[
\partial_k^{\C}(c) = \partial_k^{\C}(g_k(b)) = 
g_{k-1}\big(\partial_k^{\BB}(b)\big) = g_{k-1}(f_{k-1}(a)) = 0 \ .
\]
We also have by definition of $\tilde{\partial}_k$ that
$\tilde{\partial}_k([c]) = [a]$.  Thus $[a] \in \IMG(\tilde{\partial}_k)$.
This proves that $\KE(H_{k-1}(\F)) \subset \IMG(\tilde{\partial}_k)$.
\end{proof}

\begin{defn} \label{defnChHom}
Let $\F = \{f_k\}_{k\in \ZZ}$ and $\GG = \{g_k\}_{k\in \ZZ}$ be two
chain maps between the chain complexes
$\C = \{(C_k,\partial_k)\}_{k\in \ZZ}$ and
$\tilde{\C} = \{ (\tilde{C}_k,\tilde{\partial}_k)\}_{k\in \ZZ}$.
A {\bfseries chain homotopy}\index{Chain Homotopy} between $\F$ and
$\GG$ is a sequence
$\D = \{D_k\}_{k\in \ZZ}$ where the maps $D_k:C_k \to \tilde{C}_{k+1}$
are homomorphism such that 
$\tilde{\partial}_{k+1} \circ D_k + D_{k-1} \circ \partial_k = f_k - g_k$
for $k \in \ZZ$.
\end{defn}

A commutative diagram representing a chain homotopy is given below.
\[
\xymatrix@R+2em@C+2em{
\quad \ar[r]^-{\partial_{k+2}}
&  C_{k+1}  \ar[r]^{\partial_{k+1}} \ar@/^/[d]^{f_{k+1}} \ar@/_/[d]_{g_{k+1}}
\ar[ld]^(0.6){D_{k+1}}
&  C_k  \ar[r]^{\partial_k} \ar@/^/[d]^{f_k} \ar@/_/[d]_{g_k}  \ar[ld]^(0.6){D_k}
&  C_{k-1}  \ar[r]^-{\partial_{k-1}} \ar@/^/[d]^{f_{k-1}} \ar@/_/[d]_{g_{k-1}}
\ar[ld]^(0.6){D_{k-1}}
& \rule{0em}{1em} \rule{1.3em}{0em} \ar[ld]^(0.6){D_{k-2}}\\
\rule{0em}{1em} \rule{1.3em}{0em}
\ar[r]_-{\tilde{\partial}_{k+2}} &  \tilde{C}_{k+1}
\ar[r]_{\tilde{\partial}_{k+1}} &  \tilde{C}_k
\ar[r]_{\tilde{\partial}_k} &  \tilde{C}_{k-1}
\ar[r]_-{\tilde{\partial}_{k-1}} &
}
\]

\begin{prop} \label{propCHiFeG}
If we have a chain homotopy as in Definition~\ref{defnChHom}, then
$H_k(\F) = H_k(\GG)$ for all $k$.
\end{prop}

\begin{proof}
Given $[c] \in H_k(C)$.  Then $\partial_k(c) = 0$.  It follows from
$\tilde{\partial}_{k+1}(D_k(c)) + D_{k-1}(\partial_k(c)) = f_k(c) - g_k(c)$
that $f_k(c) - g_k(c) = \tilde{\partial}_{k+1}(D_k(c))$ with
$D_k(c) \in \tilde{C}_{k+1}$.  Thus $H_k(\F)([c]) = [f_k(c)]= [g_k(c)]
= H_k(\GG)([c])$.
\end{proof}

\begin{defn}
Let $\displaystyle \C = \{ (C_k, \partial_k) \}_{k \in \ZZ}$ be a chain complex
A surjective linear operator $\epsilon : C_0 \to R$ such that
$\epsilon \circ \partial_1 = 0$ is called an
{\bfseries augmentation}\index{Augmentation}
of the chain complex $\C$.
\end{defn}

Suppose that $\epsilon$ is an augmentation of the
$\displaystyle \C = \{ (C_k, \partial_k) \}_{k \in \ZZ}$.  If we set
\[
C^\sharp_k =
\begin{cases}
C_k & \quad \text{if} \ k \neq -1 \\
R & \quad \text{if} \ k = -1
\end{cases}
\qquad \text{and} \qquad
\partial^\sharp_k =
\begin{cases}
\partial_k & \quad \text{if} \ k \neq 0 \\
\epsilon & \quad \text{if} \ k = 0  
\end{cases}
\]
we get a new chain complex
$\displaystyle \C^\sharp = \{ (C_k^\sharp,\partial_k^\sharp) \}_{k\in \ZZ}$.

\begin{defn} \label{defnRedCC}
The chain complex
$\displaystyle \C^\sharp = \{ (C_k^\sharp,\partial_k^\sharp) \}_{k\in \ZZ}$
defined above is called the
{\bfseries reduced chain complex}\index{Reduced Chain Complex}
of the chain complex $\C$ with augmentation $\epsilon$ \footnotemark.
\end{defn}

\footnotetext{We have not found any reason why an ``augmentation'' of
a chain complex leads to a ``reduced'' chain complex.  It is perhaps
due to Proposition~\ref{propHkEquHskpR}.  The choice of terminology in
mathematics is not always obvious or even logic.}

Suppose that
$\displaystyle \tilde{\C} = \{ (\tilde{C}_k,\tilde{\partial}_k) \}_{k\in \ZZ}$
is another chain complex and
$\displaystyle \tilde{\C}^\sharp
= \{ (\tilde{C}_k^\sharp,\tilde{\partial}_k^\sharp) \}_{k\in \ZZ}$ is
a reduced chain complex of $\tilde{\C}$ with augmentation $\tilde{\epsilon}$.
Moreover, suppose that $\F = \{f_k\}_{k\in \ZZ}$ is a chain map
between $\C$ and $\tilde{\C}$.

\begin{defn}
We say that the chain map $\F$ defined above is
{\bfseries augmentation-preserving}\index{Augmentation-Preserving} 
if $\tilde{\epsilon} \circ f_0 = \epsilon$.
\end{defn}

If we set $f_{-1} = \Id_R$, then $\F = \{f_k\}_{k\in \ZZ}$ becomes a chain map
between $\displaystyle \C^\sharp$ and $\displaystyle \tilde{\C}^\sharp$.

\begin{prop} \label{propCsCtfauHsH}
Let $\displaystyle \C^\sharp = \{ (C_k^\sharp, \partial_k^\sharp) \}_{k \in \ZZ}$
be a reduced chain complex of the chain complex
$\C = \{ (C_k, \partial_k) \}_{k \in \ZZ}$ with augmentation $\epsilon$, and
$\displaystyle \tilde{\C}^\sharp
= \{ (\tilde{C}_k^\sharp,\tilde{\partial}_k^\sharp) \}_{k\in \ZZ}$ be a
reduced chain complex of the chain complex
$\displaystyle \tilde{\C} = \{ (\tilde{C}_k,\tilde{\partial}_k) \}_{k\in \ZZ}$
with augmentation $\tilde{\epsilon}$.
Suppose that $\F = \{ f_k\}_{k\in \ZZ}$ is an augmentation-preserving
chain map between $\displaystyle \C$ and $\displaystyle \tilde{\C}$.
Then $\displaystyle H_k(\F): H_k(\C^\sharp) \to H_k(\tilde{\C}^\sharp)$
is an isomorphism if and only if
$H_k(\F) : H_k(\C) \to H_k(\tilde{\C})$ is an isomorphism.
\end{prop}

\begin{proof}
The result is obvious for $k \neq 0$ because
$\displaystyle H_k(\C^\sharp) = H_k(\C)$ and 
$\displaystyle H_k(\tilde{\C}^\sharp) = H_k(\tilde{\C})$ for $k\neq 0$.

For $k=0$, we have the following commutative diagram
\begin{equation} \label{CsCtfauHsHEq1}
\xymatrix{ 0 \ar[r] & \KE(\epsilon) \ar[r]^{\iota} \ar[d]^{f_0}
& C_0 \ar@/^/[r]^{\epsilon} \ar[d]^{f_0}
& R \ar[r] \ar@/^/[l]^{g} \ar[d]^{\Id_R} & 0 \\
0 \ar[r] & \KE(\tilde{\epsilon}) \ar[r]^{\iota}
& \tilde{C}_0 \ar@/^/[r]^{\tilde{\epsilon}}
& R \ar[r] \ar@/^/[l]^{\tilde{g}} & 0
}
\end{equation}
where the two horizontal sequence are exact.  The map
$\iota$ is the inclusion map and $g:R \to C_0$ is a map such that
$\epsilon \circ g = \Id_R$.  A map $g$ exists because $\epsilon$ is onto.
Let $\tilde{g} = f_0 \circ g \circ \Id_R$.

It follows from Proposition~\ref{propSplit2Cond} that both horizontal
sequence are split.  Hence, we get from Proposition~\ref{propSplitOplus} that
$C_0 = \KE(\epsilon) \oplus \IMG(g)$ and
$\tilde{C}_0 = \KE(\tilde{\epsilon}) \oplus \IMG(\tilde{g})$.

Since $C_0 = \KE(\partial_0)$ because we assume that $C_{-1} = \{0\}$,
we have
\[
H_0(\C) = C_0/\IMG(\partial_1) = \big(\KE(\epsilon) \oplus
\IMG(g)\big) / \IMG(\partial_1)
\cong \big(\KE(\epsilon) / \IMG(\partial_1)\big) \oplus \IMG(g)
\]
because $\IMG(\partial_1) \subset \KE(\epsilon)$ and
$\KE(\epsilon) \cap \IMG(g) = \{0\}$.  More precisely, we have that
$\IMG(g) / \IMG(\partial_1) \cong \IMG(g)$.

Similarly, since $\tilde{C}_0 = \KE(\tilde{\partial}_0)$ because we assume
that $\tilde{C}_{-1} = \{0\}$, we have
\[
H_0(\tilde{\C}) = \tilde{C}_0/\IMG(\tilde{\partial}_1)
= \big(\KE(\tilde{\epsilon}) \oplus \IMG(\tilde{g})\big)
/ \IMG(\tilde{\partial}_1)
\cong \big(\KE(\tilde{\epsilon}) / \IMG(\tilde{\partial}_1)\big)
\oplus \IMG(\tilde{g})
\]
because $\IMG(\tilde{\partial}_1) \subset \KE(\tilde{\epsilon})$ and
$\KE(\tilde{\epsilon}) \cap \IMG(\tilde{g}) = \{0\}$.  Again,
we have that
$\IMG(\tilde{g}) / \IMG(\tilde{\partial}_1) \cong \IMG(\tilde{g})$.

Using the notation of the reduced chain complexes, we get
\begin{equation} \label{HFHCHCtEq1}
H_0(\C) \cong \big(\KE(\partial_0^\sharp) / \IMG(\partial_1^\sharp)\big)
\oplus \IMG(g) = H_0(\C^\sharp) \oplus \IMG(g)
\end{equation}
and
\begin{equation} \label{HFHCHCtEq2}
H_0(\tilde{\C}) \cong \big(\KE(\tilde{\partial}_0^\sharp) /
\IMG(\tilde{\partial}_1^\sharp)\big) \oplus \IMG(\tilde{g})
\cong H_0(\tilde{\C}^\sharp) \oplus \IMG(\tilde{g}) \ .
\end{equation}
Since $\IMG(g) \cong R$ and $\IMG(\tilde{g}) \cong R$ because the
two horizontal sequences in (\ref{CsCtfauHsHEq1}) are split, we
get that $\displaystyle H_0(\C^\sharp) \cong H_0(\tilde{\C}^\sharp)$ if
and only if $H_0(\C) \cong H_0(\tilde{\C})$.

We have that
$f_0 : \IMG(g) \to \IMG(\tilde{g})$ is an isomorphism.
Suppose that $f_0(a) = f_0(b)$ with $a,b \in \IMG(g)$.  Then
$\epsilon(a) = \tilde{\epsilon}(f_0(a)) = \tilde{\epsilon}(f_0(b)) = 
\epsilon(b)$.  Thus $a-b \in \KE(\epsilon) \cap \IMG(g) = \{0\}$.
Hence $a = b$.  This proves that $f_0: \IMG(g) \to \IMG(\tilde{g})$ is
one-to-one.  To prove that it is onto, consider $b \in \IMG(\tilde{g})$.
Therefore there exists $r \in R$ such that
$b = \tilde{g}(r) = f_0(g(r))$ and $g(r) \in \IMG(g)$.

It then follows from (\ref{HFHCHCtEq1}) and
(\ref{HFHCHCtEq2}) that $H_0(\F):H_0(\C) \to H_0(\tilde{\C})$ is an
isomorphism if and only
$\displaystyle H_0(\F):H_0(\C^\sharp) \to H_0(\tilde{\C}^\sharp)$
is an isomorphism because $H_0(\F): \IMG(g) / \IMG(\partial_1) \cong
\IMG(g) \to \IMG(\tilde{g}) / \IMG(\tilde{\partial}_1) \cong
\IMG(\tilde{g})$ is an isomorphism.
\end{proof}

\begin{defn}
A chain complex $\{(C_k,\partial_k)\}_{k\in \ZZ}$ is
{\bfseries acyclic}\index{Acyclic} if $H_k(\C) = 0$ for all $k$.
\end{defn}

\begin{defn}
A topological space $X$ is {\bfseries aspherical}\index{Aspherical Space}
if every continuous map $\displaystyle f: S^k\to X$ can be extended to
a continuous map $\displaystyle F: B_{k+1} \to X$ where
$\displaystyle B_{k+1} = \overline{B_1(\VEC{0})} \subset \RR^{k+1}$ is the
closed unit ball centred at the origin.
\end{defn}

\begin{rmkList} \label{rmkSasph}
\begin{enumerate}
\item Convex subsets $\displaystyle X \subset \RR^n$ are aspherical.
Suppose that $\displaystyle f: S^k \to X$ is a continuous map.
Choose $y \in X$.  Since
$\displaystyle \overline{B_1(\VEC{0})} \cong \big(S^k \times [0,1]\big) /
\big(S^k \times \{0\}\big)$, we define
$F: \overline{B_1(\VEC{0})} \to X$ as it follows.
$F([x,t]) = (1-t)y +tf(x)$ for all $[x,t] \in \overline{B_1(\VEC{0})}$.
Note that $F([x,t] \in X$ for all $x\in X$ and $t \in [0,1]$ because
$X$ is convex.  We also have that $F([x,1]) = f(x)$ for all
$\displaystyle x\in S^k$.  For the readers who may be wondering why we
had to quotient $\displaystyle S^k \times [0,1]$ by
$\displaystyle S^k \times \{0\}$, they should remember that
we always have this singularity at the origin when using polar
coordinates.  The quotient by $\displaystyle S^k \times \{0\}$
eliminates this problem.
\item Since $\Delta_k \times [0,1]$ is convex, it is aspherical.
\end{enumerate}
\end{rmkList}

\begin{theorem}
Suppose that $X$ is an aspherical space and that $R$ is an integral
domain.  Then a reduced chain complex
$\displaystyle \C^\sharp = \big\{ \big( S_k^\sharp(X;R),
\partial_k^\sharp\big)\big\}_{k\in \ZZ}$ of the chain complex
$\displaystyle \C = \{ ( S_k(X;R), \partial_k)\}_{k\in \ZZ}$
with augmentation $\epsilon$ is acyclic.
\end{theorem}

\begin{proof}
The proof is in two steps.

\stage{I}
Define the linear map $\eta: R \to S_0(X;R)$ by $\eta(s) = s\,e_x$ for
some $x \in X$ such that $\epsilon(e_x) = 1$.  Recall that
$e_x:\Delta_0 \to X$ is the singular $0$-simplex defined by
$e_x(\VEC{e}_0) = x$.  Hence, $\eta$ is a right inverse of $\epsilon$;
namely, $\epsilon \circ \eta = \Id_R$.

Let $\F = \{ f_k\}_{k\in \ZZ}$ be the chain map between $\C$ and
itself defined by $f_0 = \eta \circ \epsilon$ and $f_k = 0$ otherwise.
Moreover, Let $\GG = \{ g_k\}_{k\in \ZZ}$ be another chain map between $\C$ and
itself defined by $g_k = \Id_{S_k(X;R)}$ for $k \geq 0$ and $g_k = 0$ otherwise.

We prove that if there is a chain homotopy $\D = \{D_k\}_{k\in \ZZ}$
between $\F$ and $\GG$, then $\displaystyle \C^\sharp$ is acyclic.
We have that $D_k:S_k(X;R) \to S_{k+1}(X;R)$ are homomorphism such that 
\begin{equation} \label{pHkDI0Eq1}
\partial_{k+1} \circ D_k + D_{k-1} \circ \partial_k = g_k - f_k
\end{equation}
for $k \in \ZZ$.

Suppose that $\sigma \in Z_k(S;R)$.  It follows from (\ref{pHkDI0Eq1}) that
\[
\partial_{k+1}(D_k(\sigma))
= \begin{cases}
\sigma & \quad \text{if} \ k \neq 0 \\
\sigma - \eta(\epsilon(\sigma)) & \quad \text{if} \ k = 0
\end{cases}
\]
We have that $\sigma \in B_k(S;R)$ for $k>0$.  Thus $[\sigma] = 0$.
Since $\sigma \in Z_k(S;R)$ is arbitrary, we get that
$\displaystyle H_k^\sharp(S;R) = H_k(S;R) = 0$ for $k>0$.
If $k=0$, then we have
\begin{equation} \label{pHkDI0Eq2}
\sigma - \eta(\epsilon(\sigma))
= \partial_1(D_0(\sigma)) \ .
\end{equation}
By definition of the augmentation $\epsilon$, we have that
$\displaystyle \KE(\epsilon) = Z^\sharp_0(S;R)$.  We need to prove
that $\displaystyle \KE(\epsilon) \subset \IMG(\partial_1) = B_0^\sharp(S;R)$
to obtain that $\displaystyle H_0^\sharp(S;R) = 0$.  Consider
$\sigma \in \KE(\epsilon)$. 
It follows from (\ref{pHkDI0Eq2}) that
$\sigma = \partial_1(D_0(\sigma))$.  Thus
$\displaystyle \sigma \in \IMG(\partial_1) = B_0^\sharp(S;R)$.
Hence $\displaystyle [\sigma] = 0 \in H_0^\sharp(S;R)$.
Since $\sigma \in \KE(\epsilon)$ is arbitrary, we get that
$\displaystyle H_0^\sharp(S;R) = 0$.

\stage{II}  We prove that there is a chain homotopy
$\D = \{D_k\}_{k\in \ZZ}$ between the chain maps $\F$ and
$\GG$ defined in (i).

Let $\displaystyle \dot{\Delta}_{k+1} = \bigcup_{i=0}^{k+1}
F_{k+1}^i(\Delta_k)$.  It is the topological boundary of $\Delta_{k+1}$. 

Given a singular $k$-simplex, we proceed by induction
to define\\
$D_k :S_k(X;R) \to S_{k+1}(X;R)$.  Obviously, we may set
$D_k = 0$ for $k<0$.

\stage{II.i} Given a singular $0$-form $\sigma$ in $X$, we define
$\beta_\sigma:\dot{\Delta}_1 \to X$ such that
$\displaystyle \beta_\sigma \circ F^0_1 = \sigma$ and
$\displaystyle \beta_\sigma \circ F^1_1 = (\eta \circ \epsilon)(\sigma)$.
Since $\dot{\Delta}_1 = \{\VEC{e}_0,\VEC{e}_1\}$, two isolated points,
we have that $\beta_\sigma$ is obviously continuous on
$\dot{\Delta}_0$.  Thus, the map $\beta_\sigma$ can be extended to
$\Delta_1$ because $X$ is aspherical to give $D_0(\sigma):\Delta_1 \to X$.
Note that $\displaystyle B_k$ is homeomorphic to $\Delta_k$ for all $k\geq 0$.
Now that $D_0$ is defined for arbitrary singular $0$-forms in $X$, we
may extend $D_0$ by linearity to $S_0(X;R)$.

\stage{II.ii} Given a singular $1$-simplex $\sigma$ in $X$, we define
$\beta_\sigma:\dot{\Delta}_2 \to X$ such that
$\displaystyle \beta_\sigma\circ F^0_2 = \sigma$ and
$\displaystyle \beta_\sigma \circ F^i_2 =
D_0(\sigma\circ F^{i-1}_1)$ for $i =1,2$.
To prove that $\beta_\sigma$ is continuous on $\dot{\Delta}_2$, we
have to prove that $\beta_\sigma$ is continuous on the intersection of
the distinct faces of $\Delta_2$.  Let
$\displaystyle K_{i,j} = F^i_2(\Delta_1) \cap F^j_2(\Delta_1)$ for
$0 \leq j < i \leq 2$.

We have proved in the context of the proof of
Proposition~\ref{propdkmqdk0} that
$\displaystyle F_2^i \circ F_1^j = F_2^j \circ F_1^{i-1}$
for $0 \leq j < i \leq 2$.  Thus
$\displaystyle K_{i,j} = (F^i_2\circ F^j_1)(\Delta_0)
= (F_2^j \circ F_1^{i-1})(\Delta_0)$ for $0 \leq j < i \leq 2$.
Hence
$\displaystyle K_{0,1} = \{\VEC{e}_2\} = \{(F^1_2\circ F^0_1)(\VEC{e}_0) \}$,
$\displaystyle K_{0,2} = \{\VEC{e}_1\} = \{ (F^2_2\circ F^0_1)(\VEC{e}_0) \}$ and
$\displaystyle K_{1,2} = \{\VEC{e}_0\} = \{ (F_2^2 \circ F_1^1)(\VEC{e}_0) \}$.
Since the intersection of the faces is only one point, we obviously
have that $\beta_\sigma$ is continuous on $\dot{\Delta}_1$.
Therefore, the map $\beta_\sigma$ can be extended to $\Delta_2$ because $X$ is
aspherical to give $D_1(\sigma):\Delta_2 \to X$.
Now that $D_1$ is defined for arbitrary singular $1$-forms in $X$, we
may extend $D_1$ by linearity to $S_1(X;R)$.

\stage{II.iii} We assume that $D_{k-1} : S_{k-1}(X;R) \to S_k(X;R)$ has
been defined.  Given a singular $k$-simplex $\sigma$ in $X$, 
we define $\beta_\sigma:\dot{\Delta}_{k+1} \to X$ such that
$\displaystyle \beta_\sigma\circ F^0_{k+1} = \sigma$ and
$\displaystyle \beta_\sigma \circ F^i_{k+1} =
D_{k-1}(\sigma\circ F^{i-1}_k)$ for $1 \leq i \leq k+1$.
To prove that $\beta_\sigma$ is continuous on $\dot{\Delta}_{k+1}$, we
have to prove that $\beta_\sigma$ is continuous on the intersection of
the distinct faces of $\Delta_{k+1}$.

Let $\displaystyle K_{i,j} = F^i_{k+1}(\Delta_k) \cap F^j_{k+1}(\Delta_k)$ for
$0 \leq j < i \leq k+1$.  As we mentioned before, we have proved in
the context of the proof of Proposition~\ref{propdkmqdk0} that
$F_{k+1}^i \circ F_k^j = F_{k+1}^j \circ F_k^{i-1}$
for $0 \leq j < i \leq k+1$.  Thus
$K_{i,j} = (F^i_{k+1}\circ F^j_k)(\Delta_{k-1})
= (F_{k+1}^j \circ F_k^{i-1})(\Delta_{k-1})$ for
$0 \leq j < i \leq k+1$.  Since $i > j \geq 0$, we get from the
definition of $D_{k-1}$ that
\begin{align}
\beta_\sigma \circ F^i_{k+1} \circ F^j_k
&= (\beta_\sigma \circ F^i_{k+1}) \circ F^j_k
= D_{k-1}(\sigma \circ F_k^{i-1}) \circ F^j_k \nonumber \\
&= \begin{cases}
\sigma \circ F_k^{i-1} & \quad \text{if}\ j = 0 \\
D_{k-2}(\sigma \circ F_k^{i-1} \circ F^{j-1}_{k-1}) & \quad \text{if}
\ 0 < j < i
\end{cases}  \label{pHkDI0Eq3}
\end{align}
To get the last equality, wee use the fact that
$\displaystyle D_{k-1}(\sigma \circ F_k^{i-1})\Big|_{\dot{\Delta}_k}
= \beta_{\sigma\circ F_k^{i-1}}:\dot{\Delta}_k \to X$ for the
definition of $D_{k-1}$.

Moreover, we get from the definition of $\beta_\sigma$ and $D_{k-1}$ that
\begin{align}
\beta_\sigma \circ F^j_{k+1} \circ F^{i-1}_k
&= (\beta_\sigma \circ F^j_{k+1}) \circ F^{i-1}_k
= \begin{cases}
\sigma \circ F_k^{i-1} & \quad \text{if}\ j = 0 \\
D_{k-1}(\sigma \circ F_k^{j-1}) \circ F^{i-1}_k & \quad \text{if}
\ 0 < j < i
\end{cases} \nonumber \\
&= \begin{cases}
\sigma \circ F_k^{i-1} & \quad \text{if}\ j = 0 \\
D_{k-2}(\sigma \circ F_k^{j-1} \circ F^{i-2}_{k-1}) & \quad \text{if}
\ 0 < j < i
\end{cases}  \label{pHkDI0Eq4}
\end{align}
To get the last equality, we again use the fact that
$\displaystyle D_{k-1}(\sigma \circ F_k^{i-1})\Big|_{\dot{\Delta}_k}
= \beta_{\sigma\circ F_k^{i-1}}:\dot{\Delta}_k \to X$ for the
definition of $D_{k-1}$.

Since $\displaystyle F_k^{i-1} \circ F_{k-1}^{j-1}
= F_k^{j-1} \circ F_{k-1}^{i-2}$
for $1 \leq j < i \leq k+1$, we get from (\ref{pHkDI0Eq3}) and
(\ref{pHkDI0Eq4}) that
$\displaystyle \beta_\sigma \circ F^i_{k+1} \circ F^j_k
= \beta_\sigma \circ F^j_{k+1} \circ F^{i-1}_k$; namely,
$\beta_\sigma$ on the $\displaystyle j^{th}$ face of the 
$\displaystyle i^{th}$ face of $\dot{\Delta}_{k+1}$ is equal to
$\beta_\sigma$ on the $\displaystyle (i-1)^{th}$ face of the 
$\displaystyle j^{th}$ face of $\dot{\Delta}_{k+1}$.
Thus $\beta_\sigma: \dot{\Delta}_k \to X$ is continuous.
Therefore, the map $\beta_\sigma$ can be extended to $\Delta_{k+1}$
because $X$ is aspherical to give $D_k(\sigma):\Delta_{k+1} \to X$.
Now that $D_k$ is defined for arbitrary singular $k$-forms in $X$, we
may extend $D_k$ by linearity to $S_k(X;R)$.

\stage{II.iv} It is left to prove that $\D = \{D_k\}_{k\in \ZZ}$ is a
chain homotopy between the chain maps $\F$ and $\GG$ defined in (i);
namely, that
\begin{equation} \label{pHkDI0Eq5}
\partial_{k+1} \circ D_k + D_{k-1} \circ \partial_k = g_k - f_k
\end{equation}
for $k \in \ZZ$.  The equation is trivially true for $k<0$.
For $k=0$, we have from the definition of $D_0$ that
\[
\partial_1(D_0(\sigma)) = D_0(\sigma) \circ F^0_1 -
D_0(\sigma) \circ F^1_1 = \sigma - (\eta \circ \epsilon)(\sigma)
= g_0(\sigma) - f_0(\sigma)
\]
for all singular $0$-simplices in $X$.  By linearity,
(\ref{pHkDI0Eq5}) is true on $S_0(X;R)$.

For $k>0$, we have
\begin{align*}
&\partial_{k+1} (D_k(\sigma)) + D_{k-1}(\partial_k(\sigma))
= \sum_{i=0}^{k+1} (-1)^i D_k(\sigma) \circ F^i_{k+1}
+ D_{k-1}\left( \sum_{i=0}^k (-1)^i \sigma \circ F^i_k \right) \\
&\qquad = (D_k(\sigma)) \circ F^0_{k+1}
+ \sum_{i=1}^{k+1} (-1)^i D_k(\sigma) \circ F^i_{k+1}
+ \sum_{i=0}^k (-1)^i  D_{k-1}(\sigma \circ F^i_k ) \\
&\qquad = \underbrace{D_k(\sigma) \circ F^0_{k+1}}_{=\sigma}
+ \sum_{i=0}^k (-1)^{i+1}
\underbrace{D_k(\sigma) \circ F^{i+1}_{k+1}}_{= D_{k-1}(\sigma\circ F_k^i)}
+ \sum_{i=0}^k (-1)^i  D_{k-1}(\sigma \circ F^i_k ) \\
&\qquad = \sigma = g_k(\sigma) - f_k(\sigma)
\end{align*}
for all singular $k$-simplices in $X$.
By linearity, (\ref{pHkDI0Eq5}) is true on $S_k(X;R)$ for $k>0$.
\end{proof}

\begin{prop} \label{propHkDI0}
Let $R$ be an integral domain.  Then
$H_k(\Delta_k \times [0,1];R) = 0$ for $k>0$.  
\end{prop}

\begin{proof}
According to the Remark~\ref{rmkSasph}, $\Delta_k \times [0,1]$ is aspherical.  
It follows from the previous theorem that the reduced chain complex
$\displaystyle \C^\sharp = \big\{ \big( S_k^\sharp(\Delta_k \times [0,1];R),
\partial_k^\sharp\big)\big\}_{k\in \ZZ}$ of the chain complex
$\displaystyle \C = \{ ( S_k(\Delta_k \times [0,1];R), \partial_k)\}_{k\in \ZZ}$
with augmentation $\epsilon$ is acyclic.  Therefore
$\displaystyle H_k(\Delta_k \times [0,1];R)
= H_k^\sharp(\Delta_k \times [0,1];R) = 0$ for $k>0$.  
\end{proof}

The proof of the next theorem is based on the previous proposition.

\begin{theorem} \label{thmExistCH}
Let $X$ and $Y$ be two topological spaces, and $f,g:X \to Y$ be two
homotopic functions.  Consider chain maps
$\F = \{f_k\}_{k\in \ZZ}$ and $\GG = \{g_k\}_{k\in \ZZ}$
between the chain complexes
$\C = \{(S_k(X;R),\partial_k)\}_{k\in \ZZ}$ and
$\tilde{\C} = \{ (S_k(Y;R) ,\partial_k)\}_{k\in \ZZ}$ defined by
$f_k = S_k(f)$ and $g_k = S_k(g)$ for all $k$.  There exists a chain homotopy
$\D = \{ D_k\}_{k\in Z}$ between $\F$ and $\GG$
\end{theorem}

\begin{proof}
\stage{i} Let $\iota_y: X \to X \times [0,1]$ be the functions defined by
$\iota_y(x) = (x,y)$ for $x \in X$ and $y \in [0,1]$.  Moreover, let
$\SS_0 = \{S_k(\iota_0)\}_{k\in \ZZ}$ and $\SS_1 = \{S_k(\iota_1)\}_{k\in \ZZ}$
be chain maps between the chain complexes
$\C = \{(S_k(X;R),\partial_k)\}_{k\in \ZZ}$ and
$\breve{\C} = \{ (S_k(X\times [0,1];R) ,\partial_k)\}_{k\in \ZZ}$.  We prove that
there exists a chain homotopy $\tilde{\D} = \{ \tilde{D}_k\}_{k\in Z}$ between
$\SS_0$ and $\SS_1$.   To be more precise, we prove by induction the
following statement.

\noindent{$\mathbf{P_k}$}:
There exists a homomorphism $\tilde{D}_k:S_k(X;R) \to S_{k+1}(X\times [0,1];R)$
such that:
\begin{enumerate}
\item $\partial_{k+1} \circ \tilde{D}_k + \tilde{D}_{k-1} \circ \partial_k
= S_k(\iota_1) - S_k(\iota_0)$.
\item $\tilde{D}_k \circ S_k(h) = S_{k+1}(h \times \Id_{[0,1]})
\circ \tilde{D}_k$ for all continuous map $h:X \to Y$
\footnote{This property is called
{\bfseries naturality}\index{Naturality} and is really important in
category theory.  In particular, we can show that the definition of
$\tilde{D}_k$ is independent of the topological space homeomorphic to
$X$.  We leave this subject to those in pure algebra and logic.}.
\item $\partial_{k+1} \circ \big( S_{k+1}(\iota_1) - S_{k+1}(\iota_0)
  - \tilde{D}_k \circ \partial_{k+1} \big) = 0$.
\end{enumerate}
Note that $h \times \Id_{[0,1]}:X\times [0,1] \to \tilde{X} \times [0,1]$ is
defined by $(h \times \Id_{[0,1]})(x,t) = (h(x),t)$ for all
$(x,t) \in X \times [0,1]$.

We may set $\tilde{D}_k = 0$ for $k<0$.  Hence, we only have to consider
$k \geq 0$.

If $\sigma$ is a singular $k$-simplex in $X$, we also
note that the singular $k$-simplex $S_k(\iota_y)(\sigma)$ in
$X \times [0,1]$ is defined by
$(S_k(\iota_y)(\sigma))(\VEC{x}) = (\sigma, e_y)(\VEC{x})
= (\sigma(\VEC{x}),y)$
for all $\VEC{x} \in \Delta_k$, where the singular $k$-simplex
$e_y :\Delta_k \to [0,1]$ is defined by
$e_y(\VEC{x}) = y$ for all $\VEC{x} \in \Delta_k$.

We proof that $P_0$ is true.
Given a singular $0$-simplex $\sigma$ in $X$, we define
$\tilde{D}_0(\sigma) \in S_1(X\times [0,1];R)$ as
$\tilde{D}_0(\sigma)(\VEC{x})
= \big(\rho_{\sigma}(\VEC{x}),\rho_I(\VEC{x})\big)$
for all $\VEC{x} \in \Delta_1$, where $\rho_{\sigma}:\Delta_1 \to X$ and
$\rho_I:\Delta_1 \to [0,1]$ are defined by
$\rho_{\sigma}(\VEC{x}) = \sigma(\VEC{e}_0)$ and
$\rho_I(\VEC{x}) = \VEC{x} \cdot \VEC{e}_1$ for all $\VEC{x} \in \Delta_1$.
We extend $\tilde{D}_0$ to $S_1(X\times [0,1];R)$ by linearity.
Moreover, because of the linearity, it is enough to verify the three
properties of $P_0$ for arbitrary singular simplices in $X$.

Suppose that $\sigma$ is an arbitrary singular $0$-simplex in $X$.
Since $\tilde{D}_{-1} = 0$, we have
\[
(\partial_1 \circ \tilde{D}_0)(\sigma)
+ (\tilde{D}_{-1} \circ \partial_0)(\sigma)
= (\partial_1 \circ \tilde{D}_0)(\sigma)
= S_0(\iota_1)(\sigma) - S_0(\iota_0)(\sigma)
\]
because
\begin{align*}
\big(\partial_1(\tilde{D}_0(\sigma))\big)(\VEC{e}_0)
&= (\tilde{D}_0(\sigma) \circ F^0_1)(\VEC{e}_0) -
(\tilde{D}_0(\sigma) \circ F^1_1)(\VEC{e}_0)
= \tilde{D}_0(\sigma) (\VEC{e}_1) - \tilde{D}_0(\sigma)(\VEC{e}_0) \\
&= (\sigma(\VEC{e}_0),1) - (\sigma(\VEC{e}_0),0)
= S_0(\iota_1)(\sigma)(\VEC{x}_0) - S_0(\iota_0)(\sigma)(\VEC{x}_0) \ .
\end{align*}
So (1) of $P_0$ is satisfied.

Since
\begin{align*}
\tilde{D}_0(S_0(h)(\sigma))
&= \tilde{D}_0(h \circ \sigma) = (\rho_{h\circ \sigma}, \rho_I)
= (h\circ \rho_{\sigma} ,\rho_I)
= (h \times \Id_{[0,1]})(\rho_{\sigma}, \rho_I) \\
&= S_1(h \times \Id_{[0,1]})(\tilde{D}_0(\sigma)) \ ,
\end{align*}
we get that (2) of $P_0$ is satisfied.

Suppose that $\sigma$ is an arbitrary singular $1$-simplex in $X$.
To prove (3) of $P_0$, namely
$\partial_1 \circ \big( S_1(\iota_1) - S_1(\iota_0)
- \tilde{D}_0 \circ \partial_1 \big)\big) = 0$, we note that
\begin{align*}
&\big(\partial_1(\tilde{D}_0(\partial_1(\sigma)))\big)(\VEC{e}_0)
= \big( \big(\tilde{D}_0 (\partial_1(\sigma))\big) \circ F^0_1 -
\big(\tilde{D}_0 (\partial_1(\sigma))\big) \circ F^1_1\big)(\VEC{e}_0) \\
&\qquad = \big(\tilde{D}_0 (\partial_1(\sigma))\big) (\VEC{e}_1) -
\big(\tilde{D}_0 (\partial_1(\sigma))\big) (\VEC{e}_0)
= \big( \partial_1(\sigma)(\VEC{e}_0),1 \big)
-\big( \partial_1(\sigma)(\VEC{e}_0),0 \big) \\
&\qquad
= \big( (\sigma\circ F^0_1)(\VEC{e}_0),1 \big)
- \big( (\sigma\circ F^1_1)(\VEC{e}_0),1 \big)
- \big( (\sigma\circ F^0_1)(\VEC{e}_0),0 \big)
+ \big( (\sigma\circ F^1_1)(\VEC{e}_0),0 \big) \\
&\qquad = (\sigma(\VEC{e}_1),1) - (\sigma(\VEC{e}_0),1)
- (\sigma(\VEC{e}_1),0) + (\sigma(\VEC{e}_0),0) \\
&\qquad= S_1(\iota_1)(\sigma)(\VEC{e}_1)  - S_1(\iota_1)(\sigma)(\VEC{e}_0)
-S_1(\iota_0)(\sigma)(\VEC{e}_1) + S_1(\iota_0)(\sigma)(\VEC{e}_0) \\
&\qquad= \big(S_1(\iota_1)(\sigma) \circ F^0_1
- S_1(\iota_1)(\sigma) \circ F^1_1\big)(\VEC{e}_0)
- \big( S_1(\iota_0)(\sigma) \circ F^0_1
- S_1(\iota_0)(\sigma) \circ F^1_1\big)(\VEC{e}_0) \\
&\qquad = \partial_1 (S_1(\iota_1)(\sigma)) (\VEC{e}_0)
- \partial_1 (S_1(\iota_0)(\sigma)) (\VEC{e}_0)
\end{align*}

We assume that $P_{k-1}$ is true and prove that $P_k$ is true.

It follows from (3) of $P_{k-1}$ with $X$ replaced by $\Delta_k$ that
\[
\partial_k \circ \big( (S_k(\iota_1) - S_k(\iota_0)
- \tilde{D}_{k-1} \circ \partial_k )(\delta)\big) = 0
\]
where $\delta:\Delta_k \to \Delta_k$ is defined by $\delta(\VEC{x}) = \VEC{x}$
for all $\VEC{x} \in \Delta_k$.  Note that everything that we
have said so far is true for any topological space $X$.  This is the
reason why we may use $X = \Delta_k$.

Thus $\big( S_k(\iota_1) - S_k(\iota_0)
- \tilde{D}_{k-1} \circ \partial_k \big)(\delta)
\in Z_k(\Delta_k \times [0,1];R)$.  Since
$H_k(\Delta_k \times [0,1];R) = 0$ for $k>0$ according to
Proposition~\ref{propHkDI0},
there exists $\tilde{\delta} \in S_{k+1}(\Delta_k \times [0,1];R)$ such
that
\begin{equation} \label{singCHEq1}
\partial_{k+1} \tilde{\delta} = \big( S_k(\iota_1) - S_k(\iota_0)
- \tilde{D}_{k-1} \circ \partial_k \big)(\delta) \ .
\end{equation}
We set $\tilde{D}_{k}(\delta) = \tilde{\delta}$.

Given a singular $k$-simplex $\sigma$ in $X$, we set
$\tilde{D}_k(\sigma) = S_{k+1}(\sigma \times \Id_{[0,1]})(\tilde{\delta}) \in
S_{k+1}(X \times [0,1];R)$.  Recall that
$(\sigma \times \Id_{[0,1]}):\Delta_k \times [0,1] \to \Delta_k \times [0,1]$
is defined by $(\sigma \times \Id_{[0,1]})(\VEC{x},t)
= (\sigma(\VEC{x}),t)$ for all $(\VEC{x},t) \in \Delta_k \times [0,1]$.
We extend $\tilde{D}_k$ to $S_k(X;R)$ by linearity.

As for $P_0$, because of the linearity, it is enough to verify the
three properties of $P_k$ for arbitrary singular simplices in $X$.

Suppose that $\sigma$ is an arbitrary singular $k$-simplex in $X$.
(2) of $P_k$ follows from
\begin{align*}
\tilde{D}_k\big(S_k(h)(\sigma)\big) &= \tilde{D}_k( h \circ \sigma)
= S_{k+1}\big((h\circ \sigma) \times \Id_{[0,1]}\big)(\tilde{\delta})
= S_{k+1}(h\times \Id)\big(S_{k+1}(\sigma \times
\Id)(\tilde{\delta})\big) \\
&= S_{k+1}(h\times \Id_{[0,1]})( \tilde{D}_k(\sigma)) \ .
\end{align*}

To prove (1) of $P_k$, we get from (\ref{singCHEq1}) that
\begin{equation} \label{singCHEq2}
\partial_{k+1} (\tilde{D}_k(\delta)) + \tilde{D}_{k-1}(\partial_k(\delta))
= \partial_{k+1} (\tilde{\delta}) + \tilde{D}_{k-1}(\partial_k(\delta))
= S_k(\iota_1)(\delta) - S(\iota_0)(\delta)
\end{equation}
and note that $\sigma = S_k(\sigma)(\delta)$.  Hence
\begin{align*}
&\partial_{k+1}(\tilde{D}_k(\sigma)) + \tilde{D}_{k-1}(\partial_k(\sigma))
= \partial_{k+1}\big(S_{k+1}(\sigma \times
\Id_{[0,1]})(\tilde{D}_k(\delta))\big) +
\tilde{D}_{k-1}(\partial_k(S_k(\sigma)(\delta))) \\
&= S_k(\sigma \times \Id_{[0,1]})(\partial_{k+1}(\tilde{D}_k(\delta))) +
\tilde{D}_{k-1}(S_{k-1}(\sigma)(\partial_k(\delta))) \\
&= S_k(\sigma \times \Id_{[0,1]})(\partial_{k+1}(\tilde{D}_k(\delta))) +
S_k(\sigma \times \Id_{[0,1]})(\tilde{D}_{k-1}(\partial_k(\delta))) \\
&= S_k(\sigma \times \Id_{[0,1]})\big( \partial_{k+1}(\tilde{D}_k(\delta))
+ \tilde{D}_{k-1}(\partial_k(\delta))\big)
= S_k(\sigma \times \Id_{[0,1]})\big( S_k(\iota_1)(\delta) -
S(\iota_0)(\delta)\big) \\
&= S_k(\sigma \times \Id_{[0,1]})\big( (\delta,e_1) - (\delta,e_0) \big)
= (\sigma,e_1) - (\sigma,e_0)
= S_k(\iota_1)(\sigma) - S_k(\iota_0)(\sigma)
\end{align*}
where the second equality comes from (2) of
Proposition~\ref{propSkPropr}, the third equality comes from (2) for
$P_{k-1}$ and the fifth equality comes from (\ref{singCHEq2}).

Finally, (3) of $(P_k)$ follows from (1) because
\begin{align*}
\partial_{k+1}\circ \big( S_{k+1}(\iota_1) - S_{k+1}(\iota_0) \big)
&= \big( S_k(\iota_1) - S_k(\iota_0) \big) \circ \partial_{k+1}
= \big( \partial_{k+1} \circ \tilde{D}_k + \tilde{D}_{k+1} \circ
\partial_k \big) \circ \partial_{k+1}\\
&= \partial_{k+1} \circ \tilde{D}_k \circ \partial_{k+1}
\end{align*}
where again we have use (2) of Proposition~\ref{propSkPropr} to obtain
the first equality.  This completes the proof by induction.

\stage{ii} Since $f \sim g$, there exists a continuous function
$H:X \times [0,1] \to X$ such that $H(x,0) = f(x)$ and $H(x,1) = g(x)$
for all $x \in X$.  Thus $f = H\circ \iota_0$ and $g = H\circ \iota_1$.

It follows from (i) that
$\partial_{k+1} \circ \tilde{D}_k+ \tilde{D}_{k-1} \circ \partial_k
= S_k(i_1) - S_k(i_0)$
for all $k$.  Hence,
\begin{align*}
& \partial_{k+1} \circ (S_{k+1}(H) \circ \tilde{D}_k) +
(S_k(H) \circ \tilde{D}_{k-1}) \circ \partial_k
= S_k(H) \circ ( \partial_{k+1} \circ \tilde{D}_k) +
S_k(H)\circ (\tilde{D}_{k-1} \circ\partial_k) \\
&\qquad
= S_k(H) \big(  \partial_{k+1} \circ \tilde{D}_k
+ \tilde{D}_{k-1} \circ\partial_k\big)
= S_k(H)\circ (S_k(i_1) - S_k(i_0)) \\
&\qquad = S_k(H \circ i_1) - S_k(H\circ i_0)
= S_k(g) - S_k(f)
\end{align*}
where the first and fourth equality come from Proposition~\ref{propSkPropr},
and the third equality comes from (i).
This complete the proof of the theorem if we take $D_k = S_{k+1}(H)
\circ \tilde{D}_k$ for all $k$.
\end{proof}

The next corollary is a consequence of Theorem~\ref{thmExistCH} and
Proposition~\ref{propCHiFeG}.

\begin{cor} \label{corXCHom}
Suppose that $X$ and $Y$ are two topological spaces.  If $f,h:X \to Y$ are
homotopic maps, then $H_k(f) = H_k(g)$ for all $k$.
\end{cor}

\begin{cor} \label{corContrHk}
If $X$ is a contractible topological space and $R$ is an integral
domain, then $H_k(X;R) = 0$ for $k>0$ and $H_0(X;R) \cong R$.
\end{cor}

\begin{proof}
It follows from the definition of contractible spaces that there
exists $x_0 \in X$ such that the identity function $\Id_X:X \to X$ is
homotopic to the constant function $f:X\to X$ defined by $f(x) = x_0$
for all $x\in X$.  From the previous corollary, we have that
$H_k(\Id_X) = H_k(f)$.  It follows from Example~\ref{eggHk1} that
\[
H_k(X;R) = H_k(\Id_X)(H_k(X;R)) = H_k(f)(H_k(X;R))
\cong \begin{cases}
0 & \quad \text{if} \ k > 0 \\
R & \quad \text{if} \ k = 0
\end{cases}
\]
because $H_k(f)(H_k(X;R)) = H_k(\{x_0\};R)$.
\end{proof}

\begin{defn}
Let $X$ and $Y$ be two topological spaces.  We say that a continuous
map $f:X \to Y$ is a {\bfseries homotopic
equivalence}\index{Homotopic Equivalence} if there exists a continuous
map $h:Y \to X$ such that $h \circ f \sim \Id_X$ and
$f \circ h \sim \Id_Y$.  In that situation, we say that $X$ and $Y$ are
{\bfseries homotopic equivalent}\index{Homotopic Equivalent}.
\end{defn}

\begin{prop}
Suppose that $X$ and $Y$ are two homotopic equivalent topological
spaces.  Then $H_k(X;R) \cong H_k(Y;R)$.
\end{prop}

\begin{proof}
The previous proposition follows from Corollaries~\ref{corCompHgHf} and
\ref{corXCHom}.

Suppose that $f:X \to Y$ and $h:Y \to X$ are two continuous maps
such that $h \circ f \sim \Id_X$ and $f \circ h \sim \Id_Y$.  Then
$H_k(h) \circ H_k(f) = H_k(h \circ f) = H_k(\Id_X) = \Id_{H_k(X;R)}$
and
$H_k(f) \circ H_k(h) = H_k(f \circ h) = H_k(\Id_Y) = \Id_{H_k(Y;R)}$.
Thus $H_k(h)$ is the inverse of $H_k(f)$.  Hence
$H_k(f)$ is an isomorphism between $H_k(X;R)$ and $H_k(Y;R)$.
\end{proof}

\subsection{Reduced Homology}

Let $X$ be a topological space and $R$ be an integral domain.
This section presents an example of an augmentation for the
chain complex $\displaystyle \C = \{ (S_k(X;R), \partial_k) \}_{k \in \ZZ}$.
This example plays a crucial role later.

We consider the reduced chain complex
$\displaystyle \C^\sharp = \{ (S_k^\sharp(X;R),\partial_k^\sharp) \}_{k\in \ZZ}$
of the chain complex $\C$ with augmentation
\[
\partial_0^\sharp \left( \sum_{j\in J} a_j \sigma_j \right)  
= \sum_{j\in J} a_j \ ,
\]
where $J \subset \NN$ is a finite set, $a_j \in R$ and $\sigma_j$ is a
singular $0$-simplex for $j \in J$.  According to the definition of a
reduced chain complex, Definition~\ref{defnRedCC}, we have
\[
S_k^\sharp(X;R) = \begin{cases}
R & \quad \text{if} \ k = -1 \\
S_k(X;R) & \quad \text{if} \ k \neq -1
\end{cases}
\]
and $\displaystyle \partial_k^\sharp = \partial_k$ for $k \neq 0$.
We have already seen this augmentation in the proof of
Proposition~\ref{propH0XR} where $\displaystyle \partial_0^\sharp$ was
denoted $\epsilon$.  We also proved that
$\displaystyle \partial_0^\sharp \circ \partial_1^\sharp= 0$.
We can then proceed as we did before to define
$\displaystyle Z_k^\sharp(X;R)$, 
$\displaystyle B_k^\sharp(X;R)$ and $\displaystyle H_k^\sharp(X;R)$.

\begin{defn}
Let $X$ be a topological space $X$ and $R$ be an integral domain.
The $R$-module $\displaystyle H_k^\sharp(X;R)$ is called the
{\bfseries reduced (singular) $\mathbf{k^{th}}$ homology
module}\index{Homology Module!Reduced (Singular) $k^{th}$ Homology Module}
of $X$.
\end{defn}

In fact $\displaystyle H_k^\sharp(X;R) \cong H_k(X;R)$ for $k\neq 0$.
Only $\displaystyle H_0^\sharp(X;R)$ and $\displaystyle H_0(X;R)$ may
be different.  For instance, we saw that in Example~\ref{eggHk1} that
$H_0(X,R) = R$ if $X = \{x\}$.  However, we have that
$\displaystyle H_0^\sharp(X,R) = 0$ if $X = \{x\}$.  In general, we have the
following result.

\begin{prop} \label{propHkEquHskpR}
Let $X$ be a topological space $X$ and $R$ be an integral domain.
Then $\displaystyle H_0(X;R) \cong H_0^\sharp(X;R) \oplus R$.
\end{prop}

\begin{proof}
We first note that $Z_0(X;R) = S_0(X;R)$.
Moreover, $\displaystyle B_0(X;R) = B_0^\sharp(X;R)$ because
$\partial_1^\sharp = \partial_1$.
Since $\displaystyle \partial_0^\sharp \circ \partial_1 = 0$, we also
have that $\displaystyle B_0(X;R) \subset \KE(\partial_0^\sharp)
= Z_0^\sharp(X;R)$.

Given $b \in S_0(X;R)$ such that
$\displaystyle \partial_0^\sharp(b) = 1$, we prove that
\[
Z_0(X;R) = Z_0^\sharp(X;R) \oplus \{ r b : r \in R\} \ .
\]
Suppose that
$\displaystyle c \in Z_0^\sharp(X;R) \cap \{ r b : r \in R\}$.
Then $c = q b$ for some $q \in R$.  But
$\displaystyle q = q \partial_0^\sharp(b) = \partial_0^\sharp(q b)
= \partial_0^\sharp(c) = 0$.
Thus $c = 0$ and so
$\displaystyle Z_0^\sharp(X;R) \cap \{ r b : r \in R\} = \{0\}$.
Given $c \in S_0(S;R)$, suppose that $\displaystyle \partial_0^\sharp(c) = q$.
Then $c = (c - q b) + qb$ with
$\displaystyle c - q b \in \KE(\partial_0^\sharp) = Z_0^\sharp(X;R)$ and
$q b \in \{ r b : r \in R\}$.  Thus
$\displaystyle S_0(X;R) = Z_0(X;R) = Z_0^\sharp(X;R) + \{ r b : r \in R\}$.  

Hence
\begin{align*}
H_0(X;R) &= Z_0(X;R) / B_0(X;R)
= \big( Z_0^\sharp(X;R) \oplus \{ r b : r \in R\} \big)/  B_0^\sharp(X;R) \\
&\cong \big( Z_0^\sharp(X;R) / B_0^\sharp(X;R) \big) \oplus \{ r b : r \in R\}
\cong H_0^\sharp(X;R) \oplus R \ .
\end{align*}
Note that $\displaystyle b \not\in B_0^\sharp(X;R)$ because
$\displaystyle \partial_0^\sharp(b) \neq 0$.
\end{proof}

In general, the results about singular homology and reduced
singular homology differ only at the level of $S_0(X;R)$.

\subsection{Relation Between $\pi_1(X,x_0)$ and $H_1(X;R)$.}
\label{subsectP1EquH1}

We take a short pause in our study of singular homology to present the
relation between the concept of fundamental groups introduced in
Section~\ref{sectFundGr} and the singular $1^{st}$ homology module.

\begin{theorem}
Let $X$ be a topological space and $x_0 \in X$.  There exists a
natural group homomorphism $\Chi$ from $\pi_1(X,x_0)$ to $H_1(X;\ZZ)$.
Moreover, if $X$ is path connected, then $\Chi$ is onto and the kernel
of $\Chi$, denoted $\KE(\Chi)$, is the commutator subgroup of
$\pi_1(X,x_0)$.
\end{theorem}

\begin{proof}
The map $\Chi : \pi_1(X,x_0) \to H_1(X;\ZZ)$ is define by
$\Chi([\alpha]) = [\breve{\alpha}]$ where
$\breve{\alpha}$ is the singular $1$-simplex defined by
$\breve{\alpha}\big((1-t)\VEC{e}_0 + t\VEC{e}_1\big)
= \alpha(t)$ for $0\leq t \leq 1$.

\stage{i} We first show that $\Chi$ is well defined.
Let $\alpha_0$ and $\alpha_1$ be two loops at $x_0$
such that $\alpha_0 \dotsim \alpha_1$.  Therefore, according to the
definition of homotopic paths, Definition~\ref{defnHomPaths}, 
there exists a continuous function $H:[0,1]\times[0,1] \to X$ such
that $H(t,0) = \alpha_0(t)$ and $H(t,1) = \alpha_1(t)$
for $0 \leq t \leq 1$ and $H(0,s) = H(1,s) = x_0$ for $0 \leq s \leq 1$.

In the context of this proof, we will associate to each path
$\alpha:[0,1]\to X$ the singular $1$-simplex
$\breve{\alpha}:\Delta_1 \to X$ defined by
$\breve{\alpha}\big( (1-t)\VEC{e}_0 +t\VEC{e}_1\big) = \alpha(t)$
for $0 \leq t \leq 1$, and vice-versa.  In particular, singular
simplices and singular chains are identified with the symbol $\breve{}$\ . 

We define a singular $2$-simplex $\beta:\Delta_2 \to X$ in $X$ as
\[
\breve{\beta}( t(s\VEC{e}_2 + (1-s)\VEC{x}_1) + (1-t) \VEC{e}_0) = H(t,s)
\]
for $t,s \in [0,1]$.  The only issue with this definition may be at
$\VEC{e}_0$ because $\VEC{e}_0 = 0 (s\VEC{e}_2 + (1-s)\VEC{x}_1) + \VEC{e}_0$
for $0\leq s \leq 1$.  However, this is not an issue because
$H(0,s) = x_0$ for $0 \leq s \leq 1$.  We also have that $\breve{\beta}$
is continuous.

We have
\begin{equation} \label{thpi1equH1Eq1}
\partial_2 \breve{\beta} = \beta \circ F^0_2
- \breve{\beta} \circ F^1_2 + \breve{\beta} \circ F^2_2
\end{equation}
where
\begin{align*}
\breve{\beta}(F^2_2((1-t)\VEC{e}_0 + t\VEC{e}_1)
&= \breve{\beta}((1-t)\VEC{e}_0 + t\VEC{e}_1) = \breve{\beta}( t(0 \VEC{e}_2 +
\VEC{e}_1) + (1-t) \VEC{e}_0) = H(t,0) \\
& = \alpha_0(t) = \breve{\alpha}_0\big( (1-t)\VEC{e}_0 +t\VEC{e}_1\big)
\end{align*}
for $0\leq t \leq 1$,
\begin{align*}
\breve{\beta}(F^1_2((1-t)\VEC{e}_0 + t\VEC{e}_1)
&= \breve{\beta}((1-t)\VEC{e}_0 + t\VEC{e}_2) = \breve{\beta}( t( \VEC{e}_2 +
0\VEC{e}_1) + (1-t) \VEC{e}_0) = H(t,1) \\
& = \alpha_1(t) = \breve{\alpha}_1\big((1-t)\VEC{e}_0 + t\VEC{e}_1\big)
\end{align*}
for $0\leq t \leq 1$, and
\begin{align*}
\breve{\beta} \circ F^0_2((1-s) \VEC{e}_0 + s \VEC{e}_1) 
&= \breve{\beta}((1-s)\VEC{e}_1+s\,\VEC{e}_2)
= \breve{\beta}( (s \VEC{e}_2 + (1-s) \VEC{e}_1) + 0\,\VEC{e}_0) = H(0,s) \\
&= x_0 = \breve{e}_{x_0}\big((1-s)\VEC{e}_0 + s\VEC{e}_1\big)
\end{align*}
for $0\leq s \leq 1$.  In the previous expression,
$\breve{e}_{x_0}:\Delta_1 \to X$ is defined by
$\breve{e}_{x_0}(\VEC{x}) = x_0$ for $\VEC{x} \in \Delta_1$.
Since $\displaystyle \breve{\beta} \circ F^0_2 = \breve{e}_{x_0} =
\partial_2(\breve{\gamma})$ where the singular $2$-simplex
$\breve{\gamma}$ in $X$ is defined by $\breve{\gamma}(\VEC{x}) = x_0$
for all $\VEC{x} \in \Delta_2$,
we get from (\ref{thpi1equH1Eq1}) that
$\displaystyle \breve{\alpha}_0 - \breve{\alpha}_1 = \breve{\beta} \circ F^2_2 
- \breve{\beta} \circ F^1_2 \in B_1(X;\ZZ)$.  Thus
$\Chi([\alpha_0]) = [\breve{\alpha}_0] = [\breve{\alpha}_1]
= \Chi([\alpha_1])$ in $H_1(X;\ZZ)$.

\stage{ii} We show that $\Chi$ is an homomorphism.
Let $\alpha_0$ and $\alpha_1$ be two loops at $x_0$.
We need to show that $\Chi([\alpha_0]\,[\alpha_1])
= \Chi([\alpha_0]) + \Chi([\alpha_1])$.
Since $[\alpha_0]\,[\alpha_1] = [\alpha_0 \alpha_1]$, we need to show
that $[\breve{\alpha_0 \alpha_1}]
= [\breve{\alpha}_0] + [\breve{\alpha}_1]$ in $H_1(X;\ZZ)$.

Let $\VEC{y}_s = (1-s)\VEC{e}_1 + (s/2)\VEC{e}_2$ for $0 \leq s \leq 1$
and $\breve{\beta}$ be the singular $2$-simplex in $X$ defined by
\[
\breve{\beta}(\VEC{x})
= \begin{cases}
\displaystyle \breve{\alpha}_0\big( (1 - 2t(2-s)^{-1})\VEC{e}_0 +
2t(2-s)^{-1} \VEC{e}_1\big)
 & \displaystyle \ \ \text{if} \ \VEC{x} =
\big( 1 - 2t(2-s)^{-1}\big)\VEC{e}_0 \\
&\qquad + 2t(2-s)^{-1} \VEC{y}_s \ \text{with} \\
&\displaystyle \quad 
0 \leq s \leq 1 \ \text{and} \ 0 \leq t \leq (2-s)/2 \\[0.8em]
\displaystyle \breve{\alpha}_1\big( 2(1-t)(2-s)^{-1} \VEC{e}_0
& \displaystyle \ \ \text{if} \ \VEC{x} =
\big( 1 - 2(1-t)(2-s)^{-1}\big)\VEC{e}_2\big) \\
\quad + (1 - 2(1-t)(2-s)^{-1})\VEC{e}_1\big)
&\qquad + 2(1-t)(2-s)^{-1} \VEC{y}_s \ \text{with} \\
&\displaystyle \quad 0 \leq s \leq 1 \ \text{and} \ s/2 < t \leq 1
\end{cases}
\]
Note that $\breve{\beta}$ is well defined because
$\breve{\beta}(\VEC{x}) =
\breve{\alpha}_0(\VEC{e}_1) = \breve{\alpha}_1(\VEC{e}_0) = x_0$ for
$\displaystyle \VEC{x} = \VEC{y}_s$ with $0 \leq s \leq 1$.  The
following figure illustrates the singular $2$-simplex $\breve{\beta}$.
\pdfbox{alg_top/singhm1}

We have
\begin{equation} \label{thpi1equH1Eq2}
\partial \breve{\beta} = \breve{\beta} \circ F^0_2 - \breve{\beta} \circ F^1_2
+ \breve{\beta} \circ F^2_2
\end{equation}
where
\[
(\breve{\beta} \circ F^2_2)((1-t)\VEC{e}_0 + t\VEC{e}_1)
= \breve{\beta}((1-t)\VEC{e}_0 + t\VEC{e}_1)
= \breve{\beta}((1-t)\VEC{e}_0 + t\VEC{y}_0)
= \breve{\alpha}_0((1 - t)\VEC{e}_0 + t \VEC{e}_1)
\]
for $0 \leq t \leq 1$,
\[
(\breve{\beta} \circ F^0_2)((1-t)\VEC{e}_0 + t\VEC{e}_1)
= \breve{\beta}((1-t)\VEC{e}_1 + t\VEC{e}_2)
= \breve{\beta}((1-t)\VEC{y}_0 + t\VEC{e}_2)
= \breve{\alpha}_1((1 - t)\VEC{e}_0 + t \VEC{e}_1)
\]
for $0 \leq t \leq 1$, and
\begin{align*}
&(\breve{\beta} \circ F^1_2)((1-t)\VEC{e}_0 + t\VEC{e}_1)
= \breve{\beta}((1-t)\VEC{e}_0 + t\VEC{e}_2) \\
&\qquad = \begin{cases}
\breve{\beta}((1 - 2t)\VEC{e}_0 + 2t \VEC{y}_1) & \quad \text{if} \ 0
\leq t \leq 1/2 \\
\breve{\beta}(1-2(1-t))\VEC{e}_2 + 2(1-t) \VEC{y}_1) & \quad \text{if} \ 1/2
\leq t \leq 1 \\
\end{cases} \\
&\qquad = \begin{cases}
\breve{\alpha}_0((1 - 2t)\VEC{e}_0 + 2t \VEC{e}_1) & \quad \text{if} \ 0
\leq t \leq 1/2 \\
\breve{\alpha}_1(2(1-t)\VEC{e}_0 + (1-2(1-t)) \VEC{e}_1) & \quad \text{if} \ 1/2
\leq t \leq 1 \\
\end{cases} \\
&\qquad = \breve{\alpha_0 \alpha_1}((1-t)\VEC{e}_0 + t \VEC{e}_1))
\end{align*}
for $0 \leq t \leq 1$.  It then follows from (\ref{thpi1equH1Eq2})
that
$\partial \breve{\beta} = \breve{\alpha}_1 - \breve{\alpha_0\alpha_1}
+ \breve{\alpha}_0$.  Thus $\breve{\alpha}_1 - \breve{\alpha_0\alpha_1}
+ \breve{\alpha}_0 \in B_1(X;\ZZ)$.  Hence
$\Chi([\alpha_0\alpha_1]) = [\breve{\alpha_0 \alpha_1}]
= [\breve{\alpha}_0] + [\breve{\alpha}_1]
= \Chi([\alpha_0]) + \Chi([\alpha_1])$ in $H_1(X;\ZZ)$.

\stage{iii} We assume that $X$ is path-connected and show that $\Chi$
is onto.  Suppose that $\breve{c} \in Z_1(X;\ZZ)$.  We can write $\breve{c}$ as
$\displaystyle \breve{c} = \sum_{j=1}^J a_j \breve{\alpha}_j$ where
$a_j\in \ZZ$ and $\breve{\alpha}_j$ is singular $1$-simplex for
$1 \leq j \leq J$.  Hence
\begin{equation} \label{thpi1equH1Eq3}
\partial_1(\breve{c}) =
\sum_{j=1}^J a_j \big( \breve{\alpha}_j \circ F^0_1
- \breve{\alpha}_j\circ F^1_1 \big) = 0 \ .
\end{equation}
Since $X$ is path-connected, we can choose for each $\breve{\alpha}_j$ in
the summation above, a path $\eta_{j,1}$ in $X$ from $x_0$ to
$\displaystyle (\breve{\alpha}_j \circ F^0_1)(\VEC{e}_0) =
\breve{\alpha}_j(\VEC{e}_1)$ and a path $\eta_{j,0}$ in $X$ from $x_0$ to
$\displaystyle (\breve{\alpha}_j \circ F^1_1)(\VEC{e}_0) =
\breve{\alpha}_j(\VEC{e}_0)$ that satisfy the following conditions:
\begin{enumerate}
\item If $\displaystyle (\breve{\alpha}_{j_1} \circ F^0_1)(\VEC{e}_0) =
(\breve{\alpha}_{j_2} \circ F^0_1)(\VEC{e}_0)$ for some $j_1 \neq j_2$,
then $\eta_{j_1,1} = \eta_{j_2,1}$.
\item If $\displaystyle (\breve{\alpha}_{j_1} \circ F^0_1)(\VEC{e}_0) =
(\breve{\alpha}_{j_2} \circ F^1_1)(\VEC{e}_0)$ for some $j_1 \neq j_2$,
then $\eta_{j_1,1} = \eta_{j_2,0}$.
\item If $\displaystyle (\breve{\alpha}_{j_1} \circ F^1_1)(\VEC{e}_0) =
(\breve{\alpha}_{j_2} \circ F^1_1)(\VEC{e}_0)$ for some $j_1 \neq j_2$,
then $\eta_{j_1,0} = \eta_{j_2,0}$.
\item If $\displaystyle (\breve{\alpha}_j \circ F^1_1)(\VEC{e}_0) = x_0$,
then $\eta_{j_1,0} = e_{x_0}$ where $e_{x_0}(t) = x_0$ for
$0 \leq t \leq 1$.
\item If $\displaystyle (\breve{\alpha}_j \circ F^0_1)(\VEC{e}_0) = x_0$,
then $\eta_{j_1,1} = e_{x_0}$.
\end{enumerate}
\pdfbox{alg_top/singhm4}
It follows from (\ref{thpi1equH1Eq3}) that
\begin{equation} \label{thpi1equH1Eq4}
  \sum_{j=1}^J a_j( \breve{\eta}_{j,1} - \breve{\eta}_{j,0} ) = 0
\end{equation}
where $\breve{\eta}_{j,i}\big((1-t)\VEC{e}_0 + t\VEC{e}_1\big)
= \eta_{j,i}(t)$ for $0 \leq t \leq 1$
because
\[
\sum_{j=1}^J a_j( \breve{\eta}_{j,1}(\VEC{e}_1) - \breve{\eta}_{j,0}(\VEC{e}_1) )
= 
\sum_{j=1}^J a_j \left( (\breve{\alpha}_j \circ F^0_1)(\VEC{e}_0)  -
(\breve{\alpha}_j \circ F^1_1)(\VEC{e}_0) \right)) = 0
\]
and
$\breve{\eta}_{j_1,1} \neq \breve{\eta}_{j_2,1}$ if and only if
$\breve{\eta}_{j_1,1}(\VEC{e}_1) \neq \breve{\eta}_{j_2,1}(\VEC{e}_1)$,
$\breve{\eta}_{j_1,1} \neq \breve{\eta}_{j_2,0}$ if and only if
$\breve{\eta}_{j_1,1}(\VEC{e}_1) \neq \breve{\eta}_{j_2,0}(\VEC{e}_1)$, and
$\breve{\eta}_{j_1,0} \neq \breve{\eta}_{j_2,0}$ if and only if
$\breve{\eta}_{j_1,0}(\VEC{e}_1) \neq \breve{\eta}_{j_2,0}(\VEC{e}_1)$
by construction.

Consider the loops at $x_0$ given by
$\displaystyle \beta_j = \eta_{j,0} \alpha_j \eta_{j,1}^{-1}$ for
all $j$.  We have from (ii) that
\begin{equation} \label{thpi1equH1Eq5}
\begin{split}
\Chi([\beta_j]) &= \Chi([\eta_{j,0}]\,[\alpha_j]\,[\eta_{j,1}^{-1}])
= \Chi([\eta_{j,0}]) + \Chi([\alpha_j]) - \Chi([\eta_{j,1}]) \\
&= [\breve{\eta}_{j,0}] + [\breve{\alpha}_j] - [\breve{\eta}_{j,1}]
\end{split}
\end{equation}
for all $j$.  In the previous equality, we have use the fact that
$\displaystyle \eta_{j,i}\eta_{j,i}^{-1} \dotsim e_{x_0}$ for $0 \leq i \leq 1$
and all $j$ implies that
\[
\Chi([\eta_{j,i}]) + \Chi([\eta_{j,i}^{-1}])
= \Chi([\eta_{j,i}]\,[\eta_{j,i}^{-1}]) = \Chi([\eta_{j,i}\eta_{j,i}^{-1}])
= \Chi([e_{x_0}]) = [\breve{e}_{x_0}] = 0
\]
in $H_1(X;\ZZ)$ for $0 \leq i \leq 1$ and all $j$.  Recall that
$e_{x_0}:[0,1] \to X$ is the loop at $x_0$ defined by
$e_{x_0}(t) = x_0$ for $0 \leq t \leq 1$.  Thus
$\displaystyle \Chi([\eta_{j,i}^{-1}]) = -\Chi([\eta_{j_i}])$
for $0 \leq i \leq 1$ and all $j$.

Consider the loop at $x_0$ given by
$\displaystyle \gamma = \prod_{j=1}^J \beta_j^{a_j}$.  It follows from
(\ref{thpi1equH1Eq4}) and (\ref{thpi1equH1Eq5}) that
\begin{align*}
\Chi([\gamma]) &= \Chi\Big( \prod_{j=1}^J [\beta_j^{a_j}]\Big)
= \sum_{j=1}^J \Chi([\beta_j^{a_j}]) = \sum_{j=1}^J \Chi([\beta_j]^{a_j})
= \sum_{j=1}^J a_j \Chi([\beta_j]) \\
&= \sum_{j=1}^J a_j \left( [\breve{\eta}_{j,0}] + [\breve{\alpha}_j] -
[\breve{\eta}_{j,1}] \right)
= \sum_{j=1}^J a_j [\breve{\alpha}_j] + \sum_{j=1}^J a_j
\left( [\breve{\eta}_{j,0}] - [\breve{\eta}_{j,1}] \right)
= \sum_{j=1}^J a_j [\breve{\alpha}_j] = [\breve{c}] \ .
\end{align*}

\stage{iv}  All elements in the commutator subgroup $G$ of $\pi_1(X.x_0)$
are of the form $\displaystyle \alpha \beta \alpha^{-1}\beta^{-1}$
where $\alpha$ and $\beta$ are two loops in $X$ at $x_0$.  For
these elements, we have
\[
\Chi\big([\alpha \beta \alpha^{-1}\beta^{-1}]\big)
= \Chi\big([\alpha]\, [\beta]\,[\alpha^{-1}]\,[\beta^{-1}]\big)
= \Chi([\alpha]) + \Chi([\beta]) - \Chi([\alpha]) - \Chi([\beta]) = 0 \ .
\]
Thus $G \subset \KE(\Chi)$.

To prove that $G = \KE(\Chi)$, we need to introduce a little tool.
Given a loop $\gamma$ in $X$ at $x_0$, suppose that
$\displaystyle \gamma = \prod_{j=1}^J \alpha_j^{\epsilon_j}$ where
the $\alpha_j$ are paths in $X$ (they may not be loops) and the
$\epsilon_j$ are either $1$ or $-1$.  We set
$\displaystyle \exp_{\gamma}(\alpha_i) = \sum_{\alpha_j=\alpha_i} \epsilon_j$
for each path $\alpha_i$ in $\gamma$.

\noindent{\bfseries Claim}: If $\exp_\gamma(\alpha_i) = 0$ for all
$\alpha_i$ in $\gamma$, then $[\gamma] \in G$.

To prove this claim, we choose the paths $\eta_{j,i}$ for
$0\leq i \leq 1$ and $1 \leq j \leq J$ as we have done in (iii).
\pdfbox{alg_top/singhm5}
Hence $\displaystyle \gamma \dotsim
\prod_{j=1}^J \big(\eta_{j,0} \alpha_j\eta_{j,1}^{-1}\big)^{\epsilon_j}$ because
$\displaystyle \eta_{j,1}^{-1}\eta_{j+1,0} \dotsim e_{x_0}$ for
$1 \leq j < J$,
$\eta_{1,0} = e_{x_0}$ and $\eta_{J,1} = e_{x_0}$.
If we set $\displaystyle \beta_j = \eta_{j,0} \alpha_j\eta_{j,1}^{-1}$ for all
$j$, we have that
$\displaystyle \gamma \dotsim \prod_{j=1}^J \beta_j^{\epsilon_j}$ where the
$\beta_j$ are loops at $x_0$.

We consider the quotient group
$\pi_1(X,x_0)/G$.  Let $[\gamma]_G$ and $[\beta_j]_G$
denote the equivalence classes of $[\gamma]$ and $[\beta_j]$ 
in $\pi_1(X,x_0)/G$.  We note that $\pi_1(X,x_0)/G$ is a commutative
group because
$\big(\sigma_1 \sigma_2\big) \big(\sigma_2 \sigma_1\big)^{-1}
= \big(\sigma_1\sigma_2\big) \big(\sigma_1^{-1} \sigma_2^{-1}\big) \in G$
for any loops $\sigma_1,\sigma_2$ in $X$ at $x_0$.
Thus
$[\sigma_1]_G\,[\sigma_2]_G
= [\sigma_1\sigma_2]_G = [\sigma_2\sigma_1]_G = 
[\sigma_2]_G\,[\sigma_1]_G$.

Hence, we may reorder the factors in the product for $[\gamma]_G$ to get
\[
[\gamma]_G = \prod_{i=1}^I [\beta_{j_i}]_G^{exp_{\gamma}(\alpha_i)}
= \prod_{i=1}^I [\beta_{j_i}]_G^0 = [e_{x_0}]_G 
\]
in $\pi_1(X,x_0)/G$ where each $[\beta_{j_i}]_G$ in the product above
represent a distinct element of $\pi_1(X,x_0)/G$.  Namely, each
$\beta_{j_i}$ in the sum is a representative of all the $\beta_j$
which are homotopic to $\beta_{j_i}$, or equivalently, $\alpha_{j_i}$
is a representative of all the $\alpha_j$ which are homotopic to
$\alpha_{j_i}$.  We therefore have that $[\gamma] \in G$.

Now that this little claim is proved, we go back to the proof that
$\KE(\Chi) = G$.

If $\Chi([\gamma]) = 0$, then $\breve{\gamma} \in B_1(X;\RR)$.  Thus
\begin{equation} \label{thpi1equH1Eq6}
\begin{split}
\breve{\gamma} &= \partial_2\left(
\sum_{j=1}^J b_j \breve{\sigma}_j \right)
= \sum_{j=1}^J b_j \bigg(
\underbrace{\breve{\alpha}_j\circ F^0_2}_{=\breve{\alpha}_j^{(0)}}
- \underbrace{\breve{\alpha}_j \circ F^1_2}_{=\breve{\alpha}_j^{(1)}} 
+ \underbrace{\breve{\alpha}_j\circ F^2_2}_{=\breve{\alpha}_j^{(2)}} \bigg) \\
&= \sum_{j=1}^J b_j \left( \breve{\alpha}_j^{(0)}
- \breve{\alpha}_j^{(1)} + \breve{\alpha}_j^{(2)} \right)
\end{split}
\end{equation}
for some singular $2$-simplices $\breve{\alpha}_j$ and $b_j \in \ZZ$.
It follows that the sum of the coefficients of $\breve{\gamma}$ in the
last sum in (\ref{thpi1equH1Eq6}) is $1$ and the sum of the coefficients
for the other singular $1$-simplices in the last sum in
(\ref{thpi1equH1Eq6}) is $0$.

We choose the paths $\eta_{j,k,i}$ for
$0\leq i \leq 1$, $0 \leq k \leq 2$ and $1\leq j \leq J$ as we have
done in (iii); namely,
$\eta_{j,k,1}$ is a path in $X$ from $x_0$ to
$\displaystyle (\breve{\alpha}_j^{(k)} \circ F^0_1)(\VEC{e}_0)$ and
$\eta_{j,k,0}$ is a path in $X$ from $x_0$ to
$\displaystyle (\breve{\alpha}_j^{(k)} \circ F^1_1)(\VEC{e}_0)$
for $0 \leq k \leq 2$ and $1\leq j \leq J$.  Moreover, paths
$\eta_{j_1,k_1,i_1}$ and $\eta_{j_2,k_2,i_2}$ that end at the same
point are the same path, and paths $\eta_{j,k,i}$ that end at $x_0$
are in fact equal to the constant path $e_{x_0}$.
Note that
\begin{align*}
\eta_{j,0,1}(1) &= \big(\breve{\alpha}_j^{(0)}\circ F^0_1\big)(\VEC{e}_0)
= \breve{\alpha}_j^{(0)}(\VEC{e}_1)
= \big(\breve{\alpha}_j\circ F^0_2\big)(\VEC{e}_1)
= \breve{\alpha}_j(\VEC{e}_2) \ , \\
\eta_{j,0,0}(1) &= \big(\breve{\alpha}_j^{(0)}\circ F^1_1\big)(\VEC{e}_0)
= \breve{\alpha}_j^{(0)}(\VEC{e}_0)
= \big(\breve{\alpha}_j\circ F^0_2\big)(\VEC{e}_0)
= \breve{\alpha}_j(\VEC{e}_1) \ , \\
\eta_{j,1,1}(1) &= \big(\breve{\alpha}_j^{(1)}\circ F^0_1\big)(\VEC{e}_0)
= \breve{\alpha}_j^{(1)}(\VEC{e}_1)
= \big(\breve{\alpha}_j\circ F^1_2\big)(\VEC{e}_1)
= \breve{\alpha}_j(\VEC{e}_2) \ , \\
\eta_{j,1,0}(1) &= \big(\breve{\alpha}_j^{(1)}\circ F^1_1\big)(\VEC{e}_0)
= \breve{\alpha}_j^{(1)}(\VEC{e}_0)
= \big(\breve{\alpha}_j\circ F^1_2\big)(\VEC{e}_0)
= \breve{\alpha}_j(\VEC{e}_0) \ , \\
\eta_{j,2,1}(1) &= \big(\breve{\alpha}_j^{(2)}\circ F^0_1\big)(\VEC{e}_0)
= \breve{\alpha}_j^{(2)}(\VEC{e}_1)
= \big(\breve{\alpha}_j\circ F^2_2\big)(\VEC{e}_1)
= \breve{\alpha}_j(\VEC{e}_1) \ \text{and} \\
\eta_{j,2,0}(1) &= \big(\breve{\alpha}_j^{(2)}\circ F^1_1\big)(\VEC{e}_0)
= \breve{\alpha}_j^{(2)}(\VEC{e}_0)
= \big(\breve{\alpha}_j\circ F^2_2\big)(\VEC{e}_0)
= \breve{\alpha}_j(\VEC{e}_0) \ .
\end{align*}
All this information is summarized in the following figure.
\pdfbox{alg_top/singhm2}

Let
$\displaystyle \beta_{j,k} = \eta_{j,k,0}\,\alpha_j^{(k)} \eta_{j,k,1}^{-1}$
for $0 \leq k \leq 2$ and $1 \leq j \leq J$, and
$\displaystyle \beta_j = \beta_{j,0}\beta_{j,1}^{-1}\beta_{j,2}$ for
$1 \leq j \leq J$.  We have
\begin{align}
\beta_j &= \beta_{j,0}\beta_{j,1}^{-1}\beta_{j,2}
= \big(\eta_{j,0,0} \alpha_j^{(0)} \eta_{j,0,1}^{-1}\big)
\big(\eta_{j,1,0} \alpha_j^{(1)} \eta_{j,1,1}^{-1}\big)^{-1}
\big(\eta_{j,2,0} \alpha_j^{(2)} \eta_{j,2,1}^{-1}\big) \nonumber \\
&= \eta_{j,0,0} \alpha_j^{(0)} \big(\eta_{j,0,1}^{-1} \eta_{j,1,1}\big)
\big(\alpha_j^{(1)}\big)^{-1} \big(\eta_{j,1,0}^{-1} \eta_{j,2,0} \big)
\alpha_j^{(2)} \eta_{j,2,1}^{-1} \nonumber \\
&\dotsim \eta_{j,0,0} \alpha_j^{(0)} \big(\alpha_j^{(1)}\big)^{-1}
\alpha_j^{(2)} \eta_{j,2,1}^{-1}
\dotsim e_{x_0}  \label{thpi1equH1Eq7}
\end{align}
because $\displaystyle \eta_{j,0,1}^{-1} \eta_{j,1,1} \dotsim e_{x_0}$ and
$\displaystyle \eta_{j,1,0}^{-1} \eta_{j,2,0} \dotsim e_{x_0}$.
We also use the fact that $\hat{\alpha}_j(\Delta_2)$ is contractible
to $\breve{\alpha}_j(\VEC{e}_1)$, because $\Delta_2$ is contractible to
$\VEC{e}_1$, to obtain the second homotopy in (\ref{thpi1equH1Eq7}).

Hence $\displaystyle \xi = \prod_{j=1}^J \beta_j^{b_j} \dotsim e_{x_0}$.
It follows that $\displaystyle \gamma\,\xi^{-1} \dotsim \gamma$ and therefore
$\displaystyle [\gamma]\,[\xi^{-1}] = [\gamma]$ in $\pi_1(X,x_0)$.
However, if $\alpha$ is one of the factors of $\gamma$, we have that
$\exp_{\xi}(\alpha) = 1$ is $\alpha = \gamma$ and
$\exp_{\xi}(\alpha) = 0$ otherwise.  This follows from the fact
to the sum of the coefficients of $\breve{\alpha}$ in the last sum in
(\ref{thpi1equH1Eq6}) is equal to $\exp_{\xi}(\alpha)$ because
$\Chi:\pi_1(X,x_0) \to H_1(X;\ZZ)$ is a group homomorphism, and
the paragraph following (\ref{thpi1equH1Eq6}).
Moreover $\exp_{\xi}(\alpha) = 0$ if $\alpha$ is one of the
$\eta_{j,k,i}$ because $\displaystyle \eta_{j,k,i}^{-1}$ is present in $\xi$ if
$\eta_{j,k,i}$ is present in $\xi$.
Therefore $\exp_{\gamma\xi^{-1}}(\sigma) = 0$
for all factors $\alpha$ in $\displaystyle \gamma \xi^{-1}$.  It
follows from the claim that
$\displaystyle [\gamma] = [\gamma \xi^{-1}]$ is in the commutator
subgroup $G$ of $\pi_1(X,x_0)$.
\end{proof}

\begin{cor}
Suppose that $X$ is a path-connected topological space and $x_0 \in X$,
then the map $\Chi$ defined in the previous theorem is a group
isomorphism between $\pi_1(X)$ and $H_1(X;\ZZ)$ if and only if
$\pi_1(X)$ is an abelian group.
\end{cor}

This corollary follows from the fact that a group is commutative if
its commutator subgroup is trivial; namely, it is composed of the unit
or identity element of the group only.

It follows from the previous corollary and Proposition~\ref{propFGSone}
that $\displaystyle H_1(S^1;\ZZ) \cong \pi_1(S^1) \cong \ZZ$.
Moreover, we get from Example~\ref{eggFGtorus} that
$\displaystyle H_1(\torus{2};\ZZ) \cong \pi_1(\torus{2}) \cong \ZZ^2$
where, as usual, $\torus{2}$ denotes the torus in $\displaystyle \RR^3$.

\subsection{Relative Homology} \label{subsectRelHom}

Though our goal is not to provide a full introduction to algebraic
topology, we feel that we cannot at least mention some of the
important concepts and results of this subject.  The readers should
consider this subsection as an invitation to pursue their study of
algebraic topology.

Let $X$ be a topological space and $R$ be an integral domain.  Moreover,
suppose that $A$ is a subset of $X$.  We may therefore consider
the $R$-module $S_k(A;R)$.  It is the submodule of $S_k(X;R)$
consisting of all the singular $k$-chains that are linear combinations of
singular $k$-simplices with images in $A$.

Since the boundary operator $\partial_k:S_k(X;R) \to S_{k-1}(X;R)$ maps
$S_k(A;R)$ into $S_{k-1}(A;R)$, we can define the operator
$\overline{\partial}_k:S_k(X;R)/S_k(A;R) \to S_{k-1}(X;R)/S_{k-1}(A;R)$
as\\
$\overline{\partial}_k(\relC[X,A]{c}) = \relC[X,A]{\partial_k c}$ for all
$\relC[X,A]{c} \in S_k(X;R)/S_k(A;R)$, where $\relC[X,A]{c}$ denotes the
equivalence class in $S_k(X;R)/S_k(A;R)$ associated to $c \in S_k(X;R)$.
We get the following commutative diagram.
\[
\xymatrix{
S_k(X;R) \ar[d]^{\partial_k} \ar[r]^-{\pi_k} & S_k(X;R)/S_k(A;R)
\ar[d]^{\overline{\partial}_k} \\
S_{k-1}(X;R) \ar[r]^-{\pi_{k-1}} & S_{k-1}(X;R)/S_{k-1}(A;R)
}
\]
where $\pi_q:S_q(X;R) \to S_q(X;R)/S_q(A;R)$ is the projection defined
by $\pi_q(c) = \relC[X,A]{c}$ for $c \in S_q(X;R)$.

It follows from $\partial_{k-1}\partial_k = 0$ that
$\overline{\partial}_{k-1}\overline{\partial}_k = 0$.

\begin{defn}
Let $A$ be a subset of a topological space $X$ and $R$ be an integral
domain.  The {\bfseries relative (singular)
$\displaystyle \mathbf{k^{th}}$ homology module}\index{Homology
Module!Relative (Singular) $k^{th}$ Homology Module} of
$X \mod A$ is the $R$-module defined as
$H_k(X,A;R) = \KE\big(\overline{\partial}_k\big) /
\IMG\big(\overline{\partial}_{k+1}\big)$.  The equivalence class in
$H_k(X,A;R)$ associated to $c \in S_k(X;R)$ is denoted
$[c]_{X,A}$.
\end{defn}

It is possible to describe more explicitly the elements of
$H_k(X,A;R)$.  For that, we need a couple of definitions.

\begin{defn}
Let $A$ be a subset of a topological space $X$ and $R$ be an integral
domain.  Given $c_1,c_2 \in S_k(X;R)$, we say that $c_1$ and $c_2$
are {\bfseries homologous in $\bm{X \mod A}$}\index{Homologous} if
there exists $\tilde{c} \in S_{k+1}(X;R)$ such that
$c_1- c_2 - \partial_{k+1}(\tilde{c}) \in S_k(A;R)$.  We then write
that $c_1 \hsim c_2$ in $X \mod A$.
\end{defn}

\begin{defn}
Elements of
$Z_k(X,A;R) = \{ c \in S_k(X;R) : \partial_k(c) \in S_{k-1}(A;R) \}$
are called {\bfseries relative $\mathbf{k}$-cycle}\index{Relative $k$-Cycle}
of $X \mod A$ and elements of
$B_k(X,A;R) = \{ c \in S_k(X;R) : c \hsim 0 \text{ in } X \mod A\}$
are called
{\bfseries relative $\mathbf{k}$-boundary}\index{Relative $k$-Boundary}
of $X \mod A$.
\end{defn}

We have that $Z_k(X,A;R) \supset Z_k(X;R)$ and
$B_k(X,A;R) \supset B_k(X;R) \cup S_k(A;R)$.
The reader may be wondering why the boundary operator is not used in
the definition of\\
$B_k(X,A;R)$.  In fact, it is used when referring to
homologous equivalence in $X \mod A$ as it is illustrated in the proof of the
following proposition.

\begin{prop}
Let $A$ be a subset of a topological space $X$ and $R$ be an integral
domain.  Then $H_k(X,A;R) \cong Z_k(X,A;R) / B_k(X,A;R)$.
\end{prop}

\begin{proof}
We have
\begin{align*}
\KE\big(\overline{\partial}_k\big)
&= \big\{ \relC[X,A]{c} : \overline{\partial}_k(\relC[X,A]{c}) 
= \relC[X,A]{ \partial_k c} = \relC[X,A]{0} \big\} \\
&= \big\{ \relC[X,A]{c} : c \in S_k(X,R) \ \text{and}
\ \partial_k(c) \in S_k(A;R) \big\} = Z_k(X,A;R) / S_k(A;R)
\end{align*}
and
\begin{align*}
\IMG\big(\overline{\partial}_{k+1}\big)
&= \big\{ \relC[X,A]{c} : \relC[X,A]{c}
= \overline{\partial}_{k+1}(\relC[X,A]{b})
\ \text{for some} \ \relC[X,A]{b} \in S_{k+1}(X;R)/S_{k+1}(A;R) \big\} \\
&= \big\{ \relC[X,A]{c} : \relC[X,A]{c - \partial_{k+1}(b)} = \relC[X,A]{0}
\ \text{for some}\ \relC[X,A]{b} \in S_{k+1}(X;R)/S_{k+1}(A;R) \big\} \\
&= \big\{ \relC[X,A]{c} : c \in S_k(X;R) \ \text{and}
\ c - \partial_{k+1}(b) \in S_k(A;R) \ \text{for some}
\ b \in S_{k+1}(X;R) \big\} \\
&= \big\{ \relC[X,A]{c} : c \in S_k(X;R)\ \text{and}
\ c \hsim 0 \text{ in } X \mod A \big\} = B_k(X,A;R)/S_k(A;R) \ .
\end{align*}
Hence
\begin{align*}
\KE\big(\overline{\partial}_k\big) / \IMG\big(\overline{\partial}_k\big) &=
\big( Z_k(X,A;R) / S_k(A;R)\big) / \big(B_k(X,A;R) / S_k(A;R) \big) \\
&\cong Z_k(X,A;R) / B_k(X,A;R) \ . \qedhere
\end{align*}
\end{proof}

\begin{defn}
Let $A$ be a subset of a topological space $X$ and $B$ be a subset of
a topological space $Y$.  If $f:X \to Y$ is a function such that
$f(A) \subset B$, we write $f:(X,A) \to (Y,B)$.
\end{defn}

Suppose that $f:(X,A) \to (Y,B)$ is a continuous function and $R$ is
an integral domain.  Then $S_k(f):S_k(X;R) \to S_k(X;R)$ and
$S_k(f):S_k(A;R) \to S_k(B;R)$.  This implies that
$\displaystyle \tilde{S}_k(f) : S_k(X;R)/S_k(A;R) \to S_k(X;R)/S_k(B;R)$
defined by $\tilde{S}_k(f)(\relC[X,A]{c}) = \relC[Y,B]{c}$ for
$c \in S_k(X;R)$ is well defined.  The following proposition can then
be easily proved by the reader.

\begin{prop} \label{propHkPropS}
Suppose that $A$ is a subset of a topological space $X$, $B$ is a subset of a
topological space $Y$ and $R$ is an integral domain.  If $f:(X,A)\to (Y,B)$
is continuous function, then $H_k(f): H_k(X,A;R) \to H_k(Y,B;R)$ defined by
$H_k(f)([c]_{X,A}) = [f\circ c]_{Y,B}$ for all
$[c]_{X,A} \in H_k(X,A;R)$ is a homomorphism that satisfies the
following statements.
\begin{enumerate}
\item $H_{k-1}(f) \circ \overline{\partial}_k
= \overline{\partial}_k \circ H_k(f)$.
\item If $C$ is a subset of a topological space $Z$ and
$g:(Y,B)\to (Z,C)$ is a continuous function, then
$H_k(g) \circ H_k(f) = H_k(g\circ f)$.
\end{enumerate}
\end{prop}

Suppose that $A$ and $B$ are two subsets of a topological space $X$
such that $A \subset B$, and $R$ is an integral domain.  The function
$\Id_X:X \to X$ is a continuous function such that
$\Id_X:(X,\emptyset) \to (X,A)$.  Hence
$H_k(\Id_X) : H_k(X,A;R) \to H_k(X,B;R)$ is well
defined by $H_k(\Id_X)([c]_{X,A}) = [\Id_X \circ c]_{X,B} = [c]_{X,B}$ for all
$[c]_{X,A} \in H_k(X.A;R)$.
If $A = \emptyset$, we get that
$H_k(\Id_X) :  H_k(X;R) = H_k(X,\emptyset;R) \to H_k(X,B;R)$.
Another function that will play a significant role
in this section is the inclusion map $\iota:A \to X$ for $A \subset X$.
We have the following result.

\begin{prop} \label{propAXXAexact}
Suppose that $A$ is a subset of a topological space $X$ and $R$ is an
integral domain.  Then the sequence.
\[
\xymatrix@C+2em{
H_k(A;R) \ar[r]^{H_k(\iota)} & H_k(X;R) \ar[r]^{H_k(\Id_X)} & H_k(X,A;R)
}
\]
is exact.
\end{prop}

\begin{proof}
Since $\Id_X \circ\, \iota:A \to X$ is a continuous function such
that $(\Id_X \circ\, \iota)(\emptyset) = \emptyset$, we have
that $H_k(\Id_X \circ\, \iota):H_k(A;R) = H_k(A,\emptyset;R) \to H_k(X,A;R)$
is well defined.  Moreover,
$H_k(\Id_X \circ\, \iota)([c]_A) = [0]_{X,A} \in H_k(X,A;R)$
for all $[c]_A \in H_k(A;R)$ because
\[
c \in S_k(A;R) \Rightarrow c \hsim 0 \ \text{in} \ X \mod A
\Rightarrow c \in B_k(X,A;R) \Rightarrow
\relC[X,A]{c} \in \IMG\big(\overline{\partial}_{k+1}\big) \ .
\]
Thus $H_k(\Id_X) \circ H_k(\iota) = H_k(\Id_X \circ\, \iota) = 0$.
Hence $\IMG( H_k(\iota)) \subset \KE( H_k(\Id_X))$.

To prove the reverse inclusion, consider $[c]_X \in \KE(H_k(\Id_X))$.
We get from $H_k(\Id_X)([c]_X) = [c]_{X,A} = [0]_{X,A} \in H(X,A;R)$
that $\relC[X,A]{c} \in \IMG\big(\overline{\partial}_{k+1}\big)$.  Thus
$\relC[X,A]{c} = \overline{\partial_{k+1}} (\relC[X,A]{b})
= \relC[X,A]{\partial_{k+1} b}$ for some $b \in S_{k+1}(X;R)$.
Hence $c - \partial_{k+1}(b) \in S_k(A;R)$ for some
$b \in S_{k+1}(X;R)$.  Since $\partial_k(c) = 0$
because $[c]_X \in H_k(X;R)$, we have that
$c - \partial_{k+1}(b)$ is a $k$-cycle in $S_k(A;R)$.
Thus $[c - \partial_{k+1}(b)]_A \in H_k(A;R)$ and
$H_k(\iota)([c - \partial_{k+1}(b)]_A)
= [c - \partial_{k+1}(b)]_X = [c]_X \in H_k(X;R)$ proving that
$[c]_X \in \IMG( H_k(\iota))$.
\end{proof}

Many of the results about singular homology modules have matching
results for regular homology modules.  For instance,
Proposition~\ref{propHkeHkJ} becomes the following proposition.

\begin{prop} \label{propXAopXjAj}
Suppose that $A$ is a subset of a topological space $X$.  If the
topological space $X$ is the union of distinct path-connected 
components $X_j$ for $j \in J$ and $A_j = A \cap X_j$ for all $j \in J$, then
$\displaystyle H_k(X,A;R) = \bigoplus_{j\in J} H_k(X_j,A_j;R)$.
\end{prop}

However, there are some differences.  For instance,
Proposition~\ref{propH0XR} is replaced by the following proposition.

\begin{prop} \label{propH0XA0}
Suppose that $A$ is a non-empty subset of a path-connected topological
space $X$.  Then $H_0(X,A;R) = 0$.
\end{prop}

\begin{proof}
Choose $x_0 \in A$ and, for each $x \in X$, choose a path $\tilde{\sigma}_x$
from $x_0$ to $x$.  For $x \in X$, let $\sigma_x$ be the singular
$1$-simplex defined by $\sigma_x((1-t)\VEC{e}_0+ t\VEC{e}_1)
= \tilde{\sigma}_x(t)$ for $0 \leq t \leq 1$.   As usual, we define
the singular $0$-simplex $e_x$ by $e_x(\VEC{e}_0) = x$.

Suppose that $c$ is a relative $0$-cycle in $X \mod A$.  We have that
$\displaystyle c =  \sum_{j\in J} a_je_{x_j}$ for some $a_j \in R$ and
$x_j \in X$, where $J$ is a finite subset of $\displaystyle \NNp$.
Since
\[
\partial_1\Big( \sum_{j\in J} a_j\sigma_{x_j} \Big)
= \sum_{j\in J} a_j \big( \sigma_{x_j}\circ F^0_1 - 
\sigma_{x_j}\circ F^1_1 \big)
= \sum_{j\in J} a_j \big( e_{x_j} - e_{x_0} \big)
= c - \sum_{j\in J} a_j e_{x_0} \ ,
\]
we have that
$\displaystyle c \hsim \sum_{j\in J} a_j e_{x_0} \in S_0(A;R)$ in $X \mod A$.
Thus $c$ is a relative $k$-boundary in $X \mod A$.
Therefore $[c]_{X,A} = [0]_{X,A} \in H_k(X,A;R)$.
\end{proof}

We extend the definition of homotopic maps between two topological
spaces to the context of relative homology modules.

\begin{defn} \label{defnRelHF}
Let $A$ be a subset of a topological space $X$, $B$ be a subset of a
topological space $Y$ and $R$ be an integral domain.  Suppose that
$f:(X,A)\to (Y,B)$ and $g:(X,A)\to (Y,B)$ are two continuous
functions.  We say that $f$ is
{\bfseries homotopic}\index{Homotopic Functions} to $g$
if there exists a continuous mapping $H:X\times[0,1] \to Y$ such that
$H(x,0) = f(x)$ and $H(x,1) = g(x)$ for all $x \in X$, and
$H(A,t) \subset B$ for $0\leq t \leq 1$.  We write $f\rsim g$.
\end{defn}

The next proposition could be proved using the same technique used to
prove Corollary~\ref{corXCHom}.  This obviously requires the use of
chain complexes for relative homology modules.

\begin{prop} \label{propfgHHkfgE}
Let $A$ be a subset of a topological space $X$, $B$ be a subset of a
topological space $Y$ and $R$ be an integral domain.  Suppose that
$f:(X,A)\to (Y,B)$ is homotopic to $g:(X,A)\to (Y,B)$.  Then
$H_k(f) = H_k(g)$.  
\end{prop}

There are many interesting results about relative homology modules.
We are going to present some of these results.  The reader should
consult \cite{GH,MUat} for more information.

The map $C_k: H_k(X,A;R) \to H_{k-1}(A;R)$ defined by
$C_k([c]_{X,A}) = [\partial_k(c)]_A$ is well defined because
\[
[c]_{X,A} \in H_k(X,A;R)
\Rightarrow \relC[X,A]{c} \in \KE\big(\overline{\partial}_k\big)
\Rightarrow \partial_k(c) \in S_{k-1}(A;R) \ .
\]
Thus $\partial_k(c) \in Z_{k-1}(A;R)$ because
$\partial_k(c) \in S_{k-1}(A;R)$ and $\partial_{k-1}(\partial_k(c)) = 0$.
Hence, $[\partial_k(c)]_A \in H_{k-1}(A;R)$.

\begin{defn}
The map $C_k$ defined above is called a
{\bfseries connecting homomorphism}\index{Connection Homomorphis}
\end{defn}

This definition is justified by the following result.

\begin{prop} \label{propConnectingH}
Let $A$ be a subset of a topological space $X$ and $R$ be an integral
domain.  Then the sequence
\begin{equation} \label{relHomEq2}
\xymatrix@C+2.5ex{
\ar[r]^-{C_{k+1}} & H_k(A;R) \ar[r]^{H_k(\iota)}
& H_k(X;R) \ar[r]^-{H_k(\Id_X)}  & H_k(X,A;R) \\
\ar[r]^-{C_k} & H_{k-1}(A;R) \ar[r]^{H_{k-1}(\iota)}
& H_{k-1}(X;R) \ar[r]^-{H_{k-1}(\Id_X)} &
}
\end{equation}
is a long exact sequence.
\end{prop}

\begin{proof}
We have already from Proposition~\ref{propAXXAexact} that
$\displaystyle \IMG(H_k(\iota)) = \KE(H_k(\Id_X))$.  The only thinks
that we have to prove is that
$\displaystyle \IMG(C_k) = \KE(H_{k-1}(\iota))$
and
$\displaystyle \IMG(H_k(\Id_X)) = \KE(C_k)$.

\stage{i} $\displaystyle \IMG(H_k(\Id_X)) \subset \KE(C_k)$.
Consider $[c]_X \in H_k(X;R)$.  We have that $\partial_k(c) = 0$.
Thus $H_k(\Id_X)([c]_X) = [c]_{X,A}$ with $\partial_k(c) = 0$.
Hence $C_k([c]_{X,A}) = [\partial_k(c)]_A = [0]_A \in H_{k-1}(A;R)$.

\stage{ii} $\displaystyle \IMG(H_k(\Id_X)) \supset \KE(C_k)$.
Suppose that $[c]_{X,A} \in H_k(X,A;R)$ is such that
$C_k([c]_{X,A}) = [\partial_k(c)]_A = [0]_A \in H_{k-1}(A;R)$.
Therefore $\partial_k(c) \in B_{k-1}(A;R)$; namely, $\partial_k(c)$ is a
$(k-1)$-boundary in $S_{k-1}(A;R)$.  Hence, there exists
$b \in S_k(A;R)$ such that $\partial_k(b) = \partial_k(c)$.
Since $c - b \in S_k(X;R)$ with
$\partial_k(c-b) = \partial_k(c) - \partial_k(b) = 0$, we have that
$c- b$ is a $k$-cycle in $S_k(X;R)$ and so $[c - b]_X \in  H_k(X;R)$.
We finally have that $H_k(\Id_X)([c-b]_X) = [c-b]_{X,A} = [c]_{X,A}$
because $b \in S_k(A;R)$.

\stage{iii} $\displaystyle \IMG(C_k) \subset \KE(H_{k-1}(\iota))$.
Consider $[c]_{X,A} \in H_k(X,A;R)$.  Then
$C_k([c]_{X,A}) = [\partial_k(c)]_A \in H_{k-1}(A;R)$.  Hence
\[
H_{k-1}(\iota)([\partial_k(c)]_A) = [S_{k-1}(i)(\partial_k(c))]_X
= [\partial_k(S_k(i)(c))]_X = [0]_X \in H_{k-1}(X;R)
\]
because
$\partial_k(S_k(i)(c)) \in B_{k-1}(X;R)$ since $S_k(i)(c) \in S_k(X;R)$.

\stage{iv} $\displaystyle \IMG(C_k) \supset \KE(H_{k-1}(\iota))$.
Suppose that $[c]_A \in H_{k-1}(A;R)$ is such that
$H_{k-1}(\iota)([c]_A) = [S_{k-1}(\iota)(c)]_X = [0]_X \in H_{k-1}(X;R)$.
Since $S_{k-1}(\iota)(c) = c \in B_{k-1}(X;R)$, there exists
$b \in S_k(X;R)$ such that $\partial_k(b) = c$.  Since
$c \in S_{k-1}(A;R)$, we have that $\partial_k(b) \in S_{k-1}(A;R)$.
Thus $b \in Z_k(X,A;R)$.  Finally, we get that
$C_k([b]_{X,A}) = [\partial_k(b)]_A = [c]_A$.
\end{proof}

Note that there is a much shorter proof of the previous proposition
using the Zig-Zag Lemma.  The reader can find it in (ii) the proof of
Lemma~\ref{lemEquDofET} given later.  However, we choose to give this
stand alone proof since it illustrates nicely the basic techniques used in
algebraic topology.

\begin{defn}
The sequence in (\ref{relHomEq2}) is called a {\bfseries homology
sequence}\index{Homology Sequence} for the pair $(X,A)$.
\end{defn}

If $B$ is a subset of a topological space $Y$ and
$f:(X,A) \to (Y,B)$ is a continuous function, then the following
diagram commutes.
\[
\xymatrix@C+2em{
\ar[r]^-{C_{k+1}} & H_k(A;R) \ar[r]^{H_k(\iota)} \ar[d]^{H_k(f)}
& H_k(X;R) \ar[r]^-{H_k(\Id_X)} \ar[d]^{H_k(f)} & H_k(X,A;R)
\ar[r]^-{C_k} \ar[d]^{H_k(f)} & \\
\ar[r]_-{C_{k+1}} & H_k(B;R) \ar[r]_{H_k(\iota)}
& H_k(Y;R) \ar[r]_-{H_k(\Id_X)} & H_k(Y,B;R) \ar[r]_-{C_k} &
}
\]

\begin{prop}  \label{propAretractSES}
Suppose that $A$ is a retract of a topological space $X$ and $R$ is an
integral domain.  Then the sequence.
\[
\xymatrix@C+2.5ex{
0 \ar[r] & H_k(A;R) \ar[r]^-{H_k(\iota)} & H_k(X;R) \ar[r]^-{H_k(\Id_X)}
& H_k(X,A;R) \ar[r] & 0
}
\]
is a short exact sequence.
\end{prop}

\begin{proof}
We have already from Proposition~\ref{propAXXAexact} that
$\displaystyle \IMG(H_k(\iota)) = \KE(H_k(\Id_X))$.

Since $A$ is a retract of $X$, there exists a map $r:X \to A$ such that
$r \circ \iota = \Id_{A}$ where $\iota: A \to X$ is the inclusion map.
It follows that $H_k(r) \circ H_k(\iota) = H_k(\Id_{A}) = \Id_{H_k(A;R)}$.
Thus $\KE(H_k(\iota)) = \{[0]_A\}$.  It is therefore justify to
introduce $0 \rightarrow H_k(A;R)$.

If we consider the long exact sequence of Proposition~\ref{propConnectingH}
and apply the previous reasoning to $k-1$, we get that
$\IMG(C_k) = \KE(H_{k-1}(\iota)) = \{[0]\}$.  Thus, we may also
introduce $H_k(X,A;R) \rightarrow 0$.
\end{proof}

The fact that $\IMG(H_k(\iota)) = \KE( H_k(\Id_X))$ in the previous
proposition is not unique to $A$ a retract of $X$.  We have already
proved it in Proposition~\ref{propAXXAexact} in the general case of
$A$ a subset of $X$.

We have shown in the proof of Proposition~\ref{propAretractSES} that
\[
\xymatrix@C+2.5ex{
0 \ar[r] & H_k(A;R) \ar[r]^-{H_k(\iota)} & H_k(X;R) \ar[r]^-{H_k(\Id_X)}
& H_k(X,A;R) \ar[r] & 0
}
\]
is in fact a split short exact sequence if $A$ is a retract of $X$ because
$H_k(r) \circ H_k(\iota) = \Id_{H_k(A;R)}$, where $r:X \to A$ is the
map such that $r \circ \iota = \Id_A$ by definition of a retract.
Therefore $H_k(X;R) \cong H_k(A;R) \oplus H_k(X,A;R)$.

\begin{prop}  \label{propConPaConn}
Suppose that $A$ is a contractible subset a path-connected topological
space $X$ and $R$ is an integral domain.  Then the sequence.
\[
\xymatrix@C+2.5ex{
0 \ar[r] & H_k(X;R) \ar[r]^-{H_k(\Id_X)} & H_k(X,A;R) \ar[r] & 0
}
\]
is an exact sequence for $k>0$.  In particular, $H_k(X;R) \cong H_k(X,A;R)$
for $k>0$.
\end{prop}

\begin{proof}
\stage{$k>1$}
Since $\displaystyle A$ is contractible, we get from
Corollary~\ref{corContrHk} that $\displaystyle H_k(A;R) = 0$.
It follows from Proposition~\ref{propConnectingH} that
\[
\xymatrix@C+2.5ex{
0 \ar[r]^-{H_k(\iota)} & H_k(X;R) \ar[r]^-{H_k(\Id_X)}
& H_k(X,A;R) \ar[r]^-{C_k} & 0
}
\]
is exact.  Hence $\KE(H_k(\Id_X)) = \IMG(H_k(\iota)) = \{[0]_X\}$ and
$\IMG(H_k(\Id_X)) = \KE(C_k) = H_k(X,A;R)$.  Thus $H_k(X;R) \cong H_k(X,A;R)$.

\stage{$k=1$} We still have that $\displaystyle H_1(A;R) = 0$
because $A$ is contractible.  We also have from
Proposition~\ref{propH0XA0} that $\displaystyle H_0(X,A;R) = 0$
because $X$ is path-connected.  Hence, it follows from
Proposition~\ref{propConnectingH} that
\[
\xymatrix@C+2ex{
0 \ar[r]^-{H_1(\iota)}
& H_1(X;R) \ar[r]^-{H_1(\Id_X)}  & H_1(X,A;R) \ar[r]^-{C_1}
& H_0(A;R) \ar[r]^-{H_0(\iota)} & H_0(X;R) \ar[r]^-{H_0(\Id_X)} & 0
}
\]
Since $\KE(H_1(\Id_{X})) = \IMG(H_1(\iota)) = \{[0]_X\}$, we
have that $H_1(\Id_{X})$ is one-to-one.
We have that $\displaystyle H_0(A;R) \cong R$ according to
Corollary~\ref{corContrHk} and
$\displaystyle H_0(X;R) \cong R$ according to
Propositions~\ref{propH0XR}.
Therefore, since $\displaystyle \IMG(H_0(\iota)) =
\KE(H_0(\Id_X)) = H_0(X;R) \cong R$, we get that
$\displaystyle \KE(H_0(\iota)) = \{[0]_A\}$.
It follows that $\IMG(C_1) = \KE(H_0(\iota)) = \{[0]_A\}$.
Hence, we get that
\[
\xymatrix@C+2ex{
0 \ar[r]^-{H_1(\iota)}
& H_1(X;R) \ar[r]^-{H_k(\Id_X)}  & H_1(X,A;R) \ar[r]^-{C_1} & 0
}
\]
is exact.  As in (i), this implies that
$H_1(X;R) \cong H_1(X,A;R)$.
\end{proof}

The previous result is also true for $k=0$ if we consider reduced
homology.

\begin{prop}  \label{propContrXeXA}
Suppose that $A$ is a contractible subset of a topological space $X$ and
$R$ is an integral domain.  Then the sequence. 
\[
\xymatrix@C+2ex{
0 \ar[r] & H_k^\sharp(X;R) \ar[r]^-{H_k(\Id_X)} & H_k^\sharp(X,A;R) \ar[r] & 0
}
\]
is an exact sequence for $k \geq 0$.
In particular, $\displaystyle H_k^\sharp(X;R) \cong H_k^\sharp(X,A;R)$
for $k \geq 0$.
\end{prop}

\begin{proof}
Since $A$ is contractible, we have that from
Corollary~\ref{corContrHk} that $\displaystyle H_k^\sharp(A;R)
= H_k(A;R) = 0$ for $k >0$.  In the reduced homology, we
also have that $\displaystyle H_0^\sharp(A;R) = 0$.

It follows from Proposition~\ref{propConnectingH} (which is also true
for reduced homology) that
\[
\xymatrix@C+2ex{
0 \ar[r]^-{H_k(\iota)} & H_k^\sharp(X;R) \ar[r]^-{H_k(\Id_X)}
& H_k^\sharp(X,A;R) \ar[r]^(0.7){C_k} & 0
}
\]
is exact for all $k$.  Hence $\KE(H_k(\Id_X)) = \IMG(H_k(\iota)) = \{[0]_X\}$
and $\displaystyle \IMG(H_k(\Id_X)) = \KE(C_k) = H_k^\sharp(X,A;R)$ for all $k$.
Thus $\displaystyle H_k^\sharp(X;R) \cong H_k^\sharp(X,A;R)$.
\end{proof}

\subsection{Excision} \label{subsectExcis}

As for the previous subsection, the readers should consider this
subsection as an invitation to pursue their study of algebraic
topology.

The following definition has very profound applications.

\begin{defn}
Let $A$ be a subset of a topological space $X$, $U$ be a subset of $A$
and $R$ be an integral domain.  We say that the inclusion map
$\iota:(X\setminus U,A\setminus U) \to (X,A)$ is an
{\bfseries excision}\index{Excision} if
$H_k(\iota):H_k(X\setminus U,A\setminus U;R) \to H_k(X,A;R)$
is an isomorphism for all $k$.  We then say that $U$ can be
{\bfseries excised}\index{Excised} from $(X,A)$.
\end{defn}

The following theorem is one of the fundamental theorems concerning
excisions.

\begin{theorem}[Excision] \label{thmExcis}
Let $A$ be a subset of a topological space $X$, $U$ be a subset of $A$
and $R$ be an integral domain.  If $\displaystyle \overline{U} \subset A^\circ$,
then $U$ can be excised from $(X,A)$.
\end{theorem}

Recall that $\overline{U}$ denotes the closure of the set $U$ and
$\displaystyle A^\circ$ denotes the interior of the set $A$.

It is interesting to note that the proof of Theorem~\ref{thmExcis}
requires a singular version of the theory of simplicial approximation
introduced in Section~\ref{ssectSA}.  We briefly review this theory
and then prove Theorem~\ref{thmExcis}.
The first concept to generalize is the concept of barycentric
subdivision.

\begin{defn}
Let $X$ be an affine space and $\delta:\Delta_k \to X$ be an
affine function.  Given $b \in X$, the
{\bfseries join}\index{Join} $J_{k,b}(\delta)$ is the affine
function from $\Delta_{k+1}$ to $X$ defined by
\[
J_{k,b}(\delta)\left(\sum_{i=0}^{k+1} a_i \VEC{e}_i\right)
= a_0 b + \sum_{i=1}^{k+1} a_i \delta(\VEC{e}_{i-1})
\]
for $\displaystyle \sum_{i=0}^{k+1} a_i = 1$ with $a_i \geq 0$ for all
$i$ (Figure~\ref{singApprF1}).
Given a linearly combination $\displaystyle c = \sum_{j\in J} a_j \delta_j$
where $J \subset \NN$ is a finite set, $\delta_j:\Delta_k \to X$ is a
affine function and $a_j \in \RR$ for all $j \in J$, we define
$J_{k,b}(c)$ as $\displaystyle J_{k,b}(c) = \sum_{j\in J} a_j J_{k,b}(\delta_j)$.
\end{defn}

\pdfF{alg_top/singapprF1}{Graphical representation of $J_{1,b}(\delta)$}
{Graphical representation of $J_{1,b}(\delta)$ for an affine function
singular $\delta:\Delta_1 \to X$.}{singApprF1}

If $\delta:\Delta_0 \to X$, then
\begin{align*}
\big(\partial_1 (J_{0,b}(\delta))\big) (\VEC{e}_0)
&= \big(J_{0,b}(\delta) \circ F^0_1
- J_{0,b}(\delta) \circ F^1_1\big)(\VEC{e}_0)
= J_{0,b}(\delta) (\VEC{e}_1) - J_{0,b}(\delta) (\VEC{e}_0) \\
&= \delta(\VEC{e}_0) - b \ .
\end{align*}
Thus
\begin{equation} \label{singApproP1J0}
  \partial_1 (J_{0,b}(\delta)) = \delta - e_{b} \ ,
\end{equation}
where $e_{b}:\Delta_0 \to X$ is defined by $e_{b}(\VEC{e}_0) = b$.  Hence, if
$\displaystyle c = \sum_{j\in J} a_j \delta_j$ where
$\delta_j:\Delta_0 \to X$ is an affine function and $a_j \in \RR$ for all
$j\in J$,  then $\displaystyle \partial_1(J_{0,b}(c))
= \sum_{j\in J} a_j \partial_1(J_{0,b}(\delta_j))
= \sum_{j\in J} a_j (\delta_j - e_{b})
= c - \Big(\sum_{j\in J} a_j\Big) e_{b}$.

If $\delta:\Delta_k \to X$ for $k>0$ is an affine function, then
\begin{align*}
&\big(\partial_{k+1} (J_{k,b}(\delta))\big)
\left( \sum_{i=0}^k a_i\VEC{e}_i\right)
= \sum_{j=0}^{k+1} (-1)^j \big(J_{k,b}(\delta) \circ
F^j_{k+1}\big) \left( \sum_{i=0}^k a_i\VEC{e}_i\right) \\
&\qquad = \sum_{j=0}^{k+1} (-1)^j J_{k,b}(\delta)\left(
\sum_{i=0}^{j-1} a_i\VEC{e}_i + \sum_{i=j}^k a_i\VEC{e}_{i+1}\right) \\
&\qquad = \sum_{i=0}^k a_i\delta(\VEC{e}_i)
+ \sum_{j=1}^{k+1} (-1)^j \left( a_0 b +
\sum_{i=1}^{j-1} a_i\delta(\VEC{e}_{i-1})
+ \sum_{i=j}^k a_i\delta(\VEC{e}_i) \right) \\
&\qquad = \delta\left(\sum_{i=0}^k a_i\VEC{e}_i\right)
+ \sum_{j=1}^{k+1} (-1)^j \left( a_0 b +
\sum_{i=1}^{j-1} a_i\delta(\VEC{e}_{i-1})
+ \sum_{i=j}^k a_i\delta(\VEC{e}_i) \right)
\end{align*}
and
\begin{align*}
&J_{k-1,b}(\partial_k(\delta))
\left( \sum_{i=0}^k a_i\VEC{e}_i\right)
= J_{k-1,b}\left(\sum_{j=0}^k (-1)^j(\delta \circ F^j_k)\right)
\left( \sum_{i=0}^k a_i\VEC{e}_i\right) \\
&\qquad
% = \left(\sum_{j=0}^k (-1)^j J_{k-1,b}(\delta \circ F^j_k)\right)
% \left( \sum_{i=0}^k a_i\VEC{e}_i\right)
= \sum_{j=0}^k (-1)^j J_{k-1,b}(\delta \circ F^j_k)
\left( \sum_{i=0}^k a_i\VEC{e}_i\right)
= \sum_{j=0}^k (-1)^j \left( a_0 b
+ \sum_{i=1}^ka_i (\delta \circ F^j_k)(\VEC{e}_{i-1})\right) \\
&\qquad = \sum_{j=0}^k (-1)^j \left( a_0 b
+ \sum_{i=1}^ja_i \delta(\VEC{e}_{i-1})
+ \sum_{i=j+1}^ka_i \delta(\VEC{e}_i) \right) \\
&\qquad = -\sum_{j=1}^{k+1} (-1)^j \left( a_0 b
+ \sum_{i=1}^{j-1}a_i \delta(\VEC{e}_{i-1})
+ \sum_{i=j}^ka_i \delta(\VEC{e}_i) \right) \ .
\end{align*}
Thus 
\begin{equation} \label{singApproPkp1Jk}
\partial_{k+1} (J_{k,b}(\delta))
= \delta - J_{k-1,b}(\partial_k(\delta))
\end{equation}
for $k>0$.  Hence, if
$\displaystyle c = \sum_{j\in J} a_j \delta_j$ where $J \subset \NN$
is a finite set, $\delta_j:\Delta_k \to X$ is an affine function and
$a_j \in \RR$ for all $j\in J$, then $\displaystyle \partial_{k+1}(J_{k,b}(c))
= \sum_{j\in J} a_j \partial_{k+1}(J_{k,b}(\delta_j))
= \sum_{j\in J} a_j \big(\delta_j - J_{k-1,b}(\partial_k(\delta_j))\big)
= c - J_{k-1,b}(\partial_k(c))$.

If the reader has read Subsection~\ref{subsOShomo} about oriented
simplicial homology, then there is a nice way to simplify all the
computations that we have done above.  We can use the following
associative diagram.
\[
\xymatrix@R-2ex{
\text{affine functions} \ar@{}[d] & \text{oriented simplices} \ar@{}[d] \\
\gamma:\Delta_k \to X \ar@{<->}[r] \ar[d] &
\os{\gamma(\VEC{e}_0)}{\gamma(\VEC{e}_1)}{}{}{\gamma(\VEC{e}_k)}
\ar[d] \\
J_{k,b}(\gamma) \ar@{<->}[r] \ar[d] &
\os{b}{\gamma(\VEC{e}_0)}{\gamma(\VEC{e}_1)}{}{\gamma(\VEC{e}_k)}
\ar[d] \\
\partial_{k+1}(J_{k,b}(\gamma)) \ar@{<->}[r] &
\partial_{k+1}
\os{b}{\gamma(\VEC{e}_0)}{\gamma(\VEC{e}_1)}{}{\gamma(\VEC{e}_k)}
}
\]
The relations
$\partial_1 (J_{0,b}(\delta)) = \delta - e_{b}$ and
$\displaystyle \partial_{k+1} (J_{k,b}(\delta))
= \delta - J_{k-1,b}(\partial_k(\delta))$ for $k>0$ then follow
easily from
\begin{align*}
& \partial_{k+1} (J_{k,b}(\delta)) = \partial_{k+1}
\os{b}{\gamma(\VEC{e}_0)}{\gamma(\VEC{e}_1)}{}{\gamma(\VEC{e}_k)} \\
&\qquad = \os{\gamma(\VEC{e}_0)}{}{\gamma(\VEC{e}_1)}{}{\gamma(\VEC{e}_k)}
+ \sum_{j=1}^{k+1} (-1)^j
\os{b}{\gamma(\VEC{e}_0)}{\gamma(\VEC{e}_1)}
{\widehat{\gamma(\VEC{e}_{j-1})}}{\gamma(\VEC{e}_k)} \\
&\qquad
= \os{\gamma(\VEC{e}_0)}{}{\gamma(\VEC{e}_1)}{}{\gamma(\VEC{e}_k)}
- \sum_{j=0}^k (-1)^j
\os{b}{\gamma(\VEC{e}_0)}{\gamma(\VEC{e}_1)}
  {\widehat{\gamma(\VEC{e}_j)}}{\gamma(\VEC{e}_k)} \\
&\qquad
= \os{\gamma(\VEC{e}_0)}{}{\gamma(\VEC{e}_1)}{}{\gamma(\VEC{e}_k)}
- J_{k-1,b} \big(\partial_k\os{\gamma(\VEC{e}_0)}{}{\gamma(\VEC{e}_1)}{}
{\gamma(\VEC{e}_k)} \big)
= \delta - J_{k-1,b}(\partial_k(\delta)) \ .
\end{align*}

The concept of barycentric subdivision introduced in
Definition~\ref{defnBarySubd} is replaced by the following operator.

\begin{defn}
The {\bfseries barycentric subdivision operator}\index{Barycentric
Subdivision Operator} on the space of affine functions from $\Delta_k$ to
itself, denoted $\Sd_k$, is the operator defined by
\[
\Sd_k(\delta) =
\begin{cases}
\delta & \quad \text{if} \ k = 0 \\
J_{k-1,\delta(\VEC{b}_{(k)})}
(\Sd_{k-1}(\partial_k(\delta))) & \quad \text{if} \ k > 0  
\end{cases}
\]
for all affine functions $\delta:\Delta_k \to \Delta_k$
where $\displaystyle \VEC{b}_{(k)} = (1+k)^{-1} \sum_{j=0}^k \VEC{e}_k$ is
the barycentre of $\Delta_k$.
If $\displaystyle c = \sum_{j\in J} a_j \delta_j$ where
$J \subset \NN$ is a finite set, 
$\delta_j:\Delta_k \to \Delta_k$ is an affine function and $a_j \in \RR$
for all $j \in J$, then $\Sd_k(c)$ is defined as
$\displaystyle \Sd_k(c) = \sum_{j\in J} a_j \Sd_k(\delta_j)$.
\end{defn}

\begin{egg}
We have sketched in Figure~\ref{SingBSO} the action of the Barycentric
subdivision operators $\Sd_1(\delta)_1$ and $\Sd_2(\delta_2)$ where
$\delta_i$ is the identity map on $\Delta_i$ for $i=1,2$.
In the first case
\[
\Sd_1(\delta_1) = J_{0,\delta_1(\VEC{b}_{(1)})}\big(\Sd_0(\delta_1\circ F^0_1)
- \Sd_0(\delta_1 \circ F^1_1)\big)
= J_{0,\VEC{b}_{(1)}}(F^0_1) - J_{0,\VEC{b}_{(1)}}(F^1_1)
\]
In the second case
\begin{align*}
\Sd_2(\delta_2)
&= J_{1,\delta_2(\VEC{b}_{(2)})}\big(\Sd_1(\delta_2 \circ F^0_2)\big)
-J_{1,\delta_2(\VEC{b}_{(2)})}\big(\Sd_1(\delta_2 \circ F^1_2)\big)
+ J_{1,\delta_2(\VEC{b}_{(2)})}\big(\Sd_1(\delta_2\circ F^2_2)\big) \\
&= J_{1,\VEC{b}_{(2)}}(\Sd_1(F^0_2))
-J_{1,\VEC{b}_{(2)}}(\Sd_1(F^1_2)) + J_{1,\VEC{b}_{(2)}}(\Sd_1(F^2_2)) \\
&= J_{1,\VEC{b}_{(2)}}\big(J_{0,F^0_2(\VEC{b}_{(1)})}(\Sd_0(
\partial_1 F^0_2))\big)
-J_{1,\VEC{b}_{(2)}}\big(J_{0,F^1_2(\VEC{b}_{(1)})}(\Sd_0(\partial_1 F^1_2))\big)
\\
&\qquad
+J_{1,\VEC{b}_{(2)}}\big(J_{0,F^2_2(\VEC{b}_{(1)})}(\Sd_0(\partial_1 F^2_2))\big)
\\
&= J_{1,\VEC{b}_{(2)}}(J_{0,F^0_2(\VEC{b}_{(1)})}(F^0_2\circ F^0_1))
- J_{1,\VEC{b}_{(2)}}(J_{0,F^0_2(\VEC{b}_{(1)})}(F^0_2\circ F^1_1)) \\
&\qquad -J_{1,\VEC{b}_{(2)}}(J_{0,F^1_2(\VEC{b}_{(1)})}(F^1_2\circ F^0_1))
+J_{1,\VEC{b}_{(2)}}(J_{0,F^1_2(\VEC{b}_{(1)})}(F^1_2\circ F^1_1)) \\
&\qquad + J_{1,\VEC{b}_{(2)}}(J_{0,F^2_2(\VEC{b}_{(1)})}(F^2_2 \circ F^0_1))
- J_{1,\VEC{b}_{(2)}}(J_{0,F^2_2(\VEC{b}_{(1)})}(F^2_2 \circ F^1_1))\ .
\end{align*}
\end{egg}

\pdfF{alg_top/singBSO}{Barycentric subdivision operators}
{We have only sketched the action of three of the terms
for $\Sd_2(\delta_2)$.}{SingBSO}

We may extend the barycentric subdivision operator to $S_k(X;R)$ where
$X$ is a topological space and $R$ is an integral domain.

\begin{defn}
Let $X$ be a topological space and $R$ be an integral domain.
The {\bfseries barycentric subdivision operator}\index{Barycentric
Subdivision Operator} on $S_k(X;R)$, denoted $\Sd_k$, is the linear
operator from $S_k(X;R)$ into itself defined by
$\Sd_k(\sigma) = S_k(\sigma)(\Sd_k(\delta_k))$
for all singular $k$-simplices, where $\delta_k:\Delta_k \to \Delta_k$
is the identity map.
If $\displaystyle c = \sum_{j\in J} a_j \sigma_j$ where $J \subset \NN$
is a finite set, $\sigma_j$ is a singular $k$-simplex and $a_j \in R$
for all $j \in J$, then $\Sd_k(c)$ is defined as
$\displaystyle \Sd_k(c) = \sum_{j\in J} a_j \Sd_k(\sigma_j)$.
\end{defn}

Suppose that $X$ and $Y$ are two topological spaces and $R$ is an
integral domain.  If $f:X\to Y$ is a continuous function, then we have
the following commutative diagram.
\begin{equation} \label{SfCommDiagr}
\xymatrix@C+2em{
S_k(X;R) \ar[r]^{S_k(f)} \ar[d]^{\Sd_k} & S_k(Y;R) \ar[d]^{\Sd_k} \\
S_k(X;R) \ar[r]^{S_k(f)} & S_k(Y;R)
}
\end{equation}
It is effectively easy to verify this property.  If $\sigma$ is a
singular $k$-simplex in $X$, then
\begin{align*}
S_k(f)(\Sd_k(\sigma)) &= S_k(f)\big(S_k(\sigma) (\Sd_k(\delta_k))\big)
= (S_k(f)\circ S_k(\sigma))(\Sd_k(\delta_k)) \\
&= S_k(f\circ \sigma)(\Sd_k(\delta_k))
= \Sd_k(f\circ \sigma) = \Sd_k( S_k(f) (\sigma)) \ .
\end{align*}

The following result is a little harder to proof.

\begin{prop} \label{propSkm1PkpkSk}
Let $X$ be a topological space and $R$ be an integral domain.
Then $\Sd_{k-1}(\partial_k(c)) = \partial_k(\Sd_k(c))$ for all
$c \in S_k(X;R)$ and $k>0$.
\end{prop}

\begin{proof}
\stage{i} We first prove by induction that
$\Sd_{k-1}(\partial_k(\delta)) = \partial_k(\Sd_k(\delta))$
is true for affine functions $\delta:\Delta_k \to \Delta_k$.

For $k=1$, we have
\begin{align*}
\partial_1(\Sd_1(\delta))
&= \partial_1\big(J_{0,\delta(\VEC{b}_{(1)})}(\Sd_0(\partial_1(\delta)))\big)
= \partial_1\big(J_{0,\delta(\VEC{b}_{(1)})}(\Sd_0(\delta\circ F^0_1 - \delta\circ
F^1_1))\big) \\
&= \partial_1\big(J_{0,\delta(\VEC{b}_{(1)})}
(\delta\circ F^0_1 - \delta\circ F^1_1)\big)
= \big(\delta\circ F^0_1 - e_{\delta(\VEC{b}_{(1)})}\big)
- \big(\delta\circ F^1_1 - e_{\delta(\VEC{b}_{(1)})}\big) \\
&= \delta\circ F^0_1 - \delta\circ F^1_1
= \partial_1(\delta) = \Sd_0(\partial_1(\delta))
\end{align*}
where the third equality comes from the definition of $\Sd_0$ and the
fourth equality comes from (\ref{singApproP1J0}).

Suppose that the result is true for $k \geq 1$ and 
consider affine functions $\delta:\Delta_{k+1} \to \Delta_{k+1}$.  Then
\begin{align*}
\partial_{k+1}(\Sd_{k+1}(\delta))
&= \partial_{k+1}\big(J_{k,\delta(\VEC{b}_{(k+1)})}
(\Sd_k(\partial_{k+1}(\delta)))\big) \\
& = \Sd_k(\partial_{k+1}(\delta))
- J_{k-1,\delta(\VEC{b}_{(k+1)})}
\big(\partial_k(\Sd_k(\partial_{k+1}(\delta)))\big) \\
&= \Sd_k(\partial_{k+1}(\delta)) - J_{k-1,\delta(\VEC{b}_{(k+1)})}
\big(\Sd_{k-1}(\partial_k(\partial_{k+1}(\delta)))\big) \\
&= \Sd_k(\partial_{k+1}(\delta))
- J_{k-1,\delta(\VEC{b}_{(k+1)})}\big(\Sd_{k-1}(0)\big)
= \Sd_k(\partial_{k+1}(\delta))
\end{align*}
where we have used (\ref{singApproPkp1Jk}) to get the second equality
and our hypothesis of induction to get the third equality.

\stage{ii} We can now prove that the statement of the proposition is
true for a singular $k$-simplex $\sigma$.
\begin{align*}
&\partial_k(\Sd_k(\sigma))
= \partial_k\big(S_k(\sigma)(\Sd_k(\delta_k))\big)
= S_{k-1}(\sigma)\big(\partial_k(\Sd_k(\delta_k))\big)
= S_{k-1}(\sigma)\big(\Sd_{k-1}(\partial_k(\delta_k))\big) \\
&\quad = S_{k-1}(\sigma)\left(\Sd_{k-1}
\left(\sum_{j=0}^k(-1)^j\delta_k\circ F^j_k\right)\right)
= S_{k-1}(\sigma)\left(\sum_{j=0}^k(-1)^j\Sd_{k-1}(F^j_k)\right) \\
&\quad = S_{k-1}(\sigma)\left(\sum_{j=0}^k(-1)^j
S_{k-1}(F^j_k)(\Sd_{k-1}(\delta_{k-1}))\right)
= \sum_{j=0}^k(-1)^j S_{k-1}(\sigma \circ F^j_k)(\Sd_{k-1}(\delta_{k-1})) \\
&\quad = \sum_{j=0}^k(-1)^j \Sd_{k-1}(\sigma \circ F^j_k)
= \Sd_{k-1}\left(\sum_{j=0}^k(-1)^j \sigma \circ F^j_k\right)
= \Sd_{k-1}(\partial_k \sigma)
\end{align*}
where we have used (i) to get the third equality and the fact that
$\delta_k:\Delta_k \to \Delta_k$ is the identity map to get the fifth
equality.
\end{proof}

Let $X$ be a topological space and $R$ be an integral domain.
To obtain a theory of singular approximation, we need to introduce
a map $T_k:S_k(X;R) \to S_{k+1}(X;R)$.

\begin{defn}
We define a linear function, denoted $T_k$, that maps an affine function
$\delta:\Delta_k \to \Delta_k$ to an affine function
$T_k(\delta):\Delta_{k+1} \to \Delta_{k+1}$ as it follows.
\[
T_k(\delta) = \begin{cases}
0 & \quad \text{if} \ k = 0 \\
J_{k,\delta(\VEC{b}_{(k)})}\big(\delta - \Sd_k(\delta)
- T_{k-1}(\partial_k(\delta))\big)
& \quad \text{if} \ k > 0
\end{cases}
\]
for all affine functions $\delta:\Delta_k \to \Delta_k$,
where $\displaystyle \VEC{b}_{(k)} = (1+k)^{-1} \sum_{j=0}^k \VEC{e}_k$ is
the barycentre of $\Delta_k$.
If $\displaystyle c = \sum_{j\in J} a_j \delta_j$ where $J \subset \NN$
is a finite set, $\delta_j:\Delta_k \to \Delta_k$ is an affine
function and $a_j \in \RR$ for $j\in J$, then $T_k(c)$ is defined as
$\displaystyle T_k(c) = \sum_{j\in J} a_j T_k(\delta_j)$.
\end{defn}

We extend the function $T_k$ to $S_k(X;R)$ where
$X$ is a topological space and $R$ is an integral domain.

\begin{defn}
Let $X$ be a topological space and $R$ be an integral domain.
We define a linear function $T_k: S_k(X;R) \to S_{k+1}(X;R)$ by
$T_k(\sigma) = S_{k+1}(\sigma)(T_k(\delta_k))$
for all singular $k$-simplices, where $\delta_k:\Delta_k \to \Delta_k$
is the identity map.
If $\displaystyle c = \sum_{j\in J} a_j \sigma_j$ where $J \subset
\NN$ is a finite set, $\sigma_j$ is a singular $k$-simplex and
$a_j \in R$ for all $j\in J$, then $T_k(c)$ is 
defined as $\displaystyle T_k(c) = \sum_{j\in J} a_j T_k(\sigma_j)$.
\end{defn}

We note that $S_{k+1}(\sigma)(T_k(\delta_k))$ is well defined because
$T_k(\delta_k)(\Delta_{k+1}) \subset \Delta_k$ and
$\sigma:\Delta_k \to X$.  In fact, we may extend $\sigma$ to
$\Delta_{k+1}$ by setting $\sigma(\VEC{e}_{k+1}) = 0$.

Suppose that $X$ and $Y$ are two topological spaces and $R$ is an
integral domain.  If $f:X\to Y$ is a continuous function, then the
reader can proceed as we did for $\Sd_k$ to verify that we have the
following commutative diagram.
\begin{equation} \label{TfCommDiagr}
\xymatrix@C+2em{
S_k(X;R) \ar[r]^{S_k(f)} \ar[d]^{T_k} & S_k(Y;R) \ar[d]^{T_k} \\
S_{k+1}(X;R) \ar[r]^{S_{k+1}(f)} & S_{k+1}(Y;R)
}
\end{equation}

\begin{prop} \label{proppTTpISd}
Let $X$ be a topological space and $R$ be an integral domain.  Then
$\partial_{k+1}(T_k(c)) + T_{k-1}(\partial_k(c)) = c - \Sd_k(c)$ for all
$c \in S_k(X;R)$ and $k>0$.
\end{prop}

\begin{proof}
\stage{i} We first prove by induction the
$\partial_{k+1}(T_k(\delta)) +T_{k-1}(\partial_k(\delta)) = \delta -
\Sd_k(\delta)$ is true for affine functions $\delta:\Delta_k \to \Delta_k$.

For $k=1$, we have
\begin{align*}
&\partial_2(T_1(\delta))
= \partial_2\left( J_{1,\delta(\VEC{b}_{(k)})}\big(\delta - \Sd_1(\delta)
- T_0(\partial_1(\delta))\big)\right)
= \partial_2\left( J_{1,\delta(\VEC{b}_{(k)})}\big(\delta
- \Sd_1(\delta)\big)\right) \\
&\qquad = \delta - \Sd_1(\delta)
- J_{0,\delta(\VEC{b}_{(k)})}\big(\partial_1(\delta - \Sd_1(\delta))\big)
= \delta - \Sd_1(\delta) - J_{0,\delta(\VEC{b}_{(k)})}\big(\partial_1(\delta) -
\partial_1(\Sd_1(\delta))\big) \\
&\qquad
= \delta - \Sd_1(\delta) - J_{0,\delta(\VEC{b}_{(k)})}\big(\partial_1(\delta)
- \Sd_0(\partial_1(\delta))\big)
= \delta - \Sd_1(\delta) - J_{0,\delta(\VEC{b}_{(k)})}\big(\partial_1(\delta)
- \partial_1(\delta)\big) \\
&\qquad
= \delta - \Sd_1(\delta) = \delta - \Sd_1(\delta) - T_0(\partial_1(\delta))
\end{align*}
for all affine functions $\delta:\Delta_1 \to \Delta_1$, where we have
used (\ref{singApproPkp1Jk}) to get the third equality and
Proposition~\ref{propSkm1PkpkSk} to get the fifth equality.

Suppose that the result is true for $k > 0$ and 
consider affine functions $\delta:\Delta_{k+1} \to \Delta_{k+1}$.  Then
\begin{align*}
&\partial_{k+2}(T_{k+1}(\delta)))
= \partial_{k+2} \left( J_{k+1,\delta(\VEC{b}_{(k+1)})}
\big( \delta - \Sd_{k+1}(\delta)
- T_k(\partial_{k+1}(\delta))\big)\right) \\
&\qquad = \big( \delta - \Sd_{k+1}(\delta) - T_k(\partial_{k+1}(\delta))\big)
- J_{k,\delta(\VEC{b}_{(k+1)})} \big(\partial_{k+1}\big( \delta - \Sd_{k+1}(\delta)
- T_k(\partial_{k+1}(\delta))\big)\big) \\
&\qquad = \big( \delta - \Sd_{k+1}(\delta) - T_k(\partial_{k+1}(\delta))\big)
- J_{k,\delta(\VEC{b}_{(k+1)})}\big(\partial_{k+1}(\delta)
- \partial_{k+1}(\Sd_{k+1}(\delta)) \\
&\qquad\qquad - \partial_{k+1}(T_k(\partial_{k+1}(\delta)))\big) \\
&\qquad = \big( \delta - \Sd_{k+1}(\delta) - T_k(\partial_{k+1}(\delta))\big)
- J_{k,\delta(\VEC{b}_{(k+1)})}\big(\partial_{k+1}(\delta)
- \Sd_k(\partial_{k+1}(\delta)) \\
&\qquad \qquad
- \big(\partial_{k+1}(\delta) - \Sd_k(\partial_{k+1}(\delta))
  - T_{k-1}(\partial_k(\partial_{k+1}(\delta))) \big) \big) \\
&\qquad = \delta - \Sd_{k+1}(\delta) - T_k(\partial_{k+1}(\delta))
\end{align*}
where we have used (\ref{singApproPkp1Jk}) to get the second equality,
and Proposition~\ref{propSkm1PkpkSk} and the hypothesis of induction to get the
fourth equation.

\stage{ii} We can now prove that the statement of the proposition is
true for a singular $k$-simplex $\sigma$.
\begin{align*}
&\partial_{k+1}(T_k(\sigma))
= \partial_{k+1}(S_{k+1}(\sigma)(T_k(\delta_k)))
= S_k(\sigma)(\partial_{k+1}(T_k(\delta_k))) \\
& = S_k(\sigma)\big(\delta_k - \Sd_k(\delta_k)
-T_{k-1}(\partial_k(\delta_k))\big)
= S_k(\sigma)\bigg(\delta_k - \Sd_k(\delta_k)
-T_{k-1}\bigg(\sum_{j=0}^k (-1)^j \delta_k\circ F^j_k\bigg)\bigg) \\
&= S_k(\sigma)\bigg(\delta_k - \Sd_k(\delta_k)
- \sum_{j=0}^k (-1)^j T_{k-1}(F^j_k)\bigg) \\
&= S_k(\sigma)\bigg(\delta_k - \Sd_k(\delta_k)
- \sum_{j=0}^k (-1)^j S_k(F^j_k) (T_{k-1}(\delta_{k-1}))\bigg) \\
&= S_k(\sigma)(\delta_k) - S_k(\sigma)(\Sd_k(\delta_k))
- S_k\bigg(\sum_{j=0}^k (-1)^j \sigma \circ F^j_k\bigg)(T_{k-1}(\delta_{k-1})) \\
&= \sigma - \Sd_k(\sigma) - S_k(\partial_k \sigma) (T_{k-1}(\delta_{k-1}))
= \sigma - \Sd_k(\sigma) -T_{k-1}(\partial_k(\sigma))
\end{align*}
where we have used (i) to get the third equality.  We have also used
several times the fact that $\delta_q$ is the identity map on
$\Delta_q$ for $q=k$ and $k-1$.
\end{proof}

Since the topological spaces $X$ that we are considering for the singular 
$k$-chains may not have a metric, we have to formulate the distance
between singular $k$-chains in terms of open sets.

\begin{defn}
Let $\U = \{U_i\}_{i\in I}$ be an open cover of a topological space $X$.
We say that a singular $k$-simplex $\sigma$ in $X$ is {\bfseries small
of order $\mathbf{\U}$}\index{Small of Order} if there is an $i \in I$
such that $\sigma(\delta_k) \subset U_i$.
\end{defn}

We associate to every linear combination
$\displaystyle c = \sum_{j\in J} a_j \delta_j$ where $J \subset \NN$
is a finite set, $a_j \in \RR$ and $\delta_j:\Delta_k \to \Delta_k$ is
an affine function for all $j \in J$, the value
$\displaystyle \Dm(c) = \max_{j\in J}\, \diam(\delta_j(\Delta_k))$.
We have a result similar to Proposition~\ref{propMeshMBS}.  The proof
is very similar.

\begin{prop}
Let $\delta:\Delta_k \to \Delta_k$ be an affine function.  Then
$\displaystyle \Dm\big(\Sd_k(\delta)\big) \leq \frac{k}{k+1} \Dm(\delta)$.
\end{prop}

\begin{proof}[Proof (Sketch)]
There is a result equivalent to Lemma~\ref{lemDiams} for affine
functions $\delta:\Delta_k \to \Delta_k$.  Namely, 
$\diam(\delta(\Delta_k)) =
\|\delta(\VEC{e}_{i_1})  - \delta(\VEC{e}_{i_2})\|$
for some $i_1,i_2 \in \{0,1,2,\ldots,k\}$.  The proof is basically
identical because $\delta(\VEC{x}) = t\delta(\VEC{x}) + (1-t)
\delta(\VEC{x})$ for all $\VEC{x} \in \Delta_k$.

By definitions of $\displaystyle \Sd_k(\delta)$ and the previous
paragraph, it follows that
$\Dm(c) = \|\delta(\VEC{b}_{(p)}) - \delta(\VEC{b}_{(q)}\|$
for some $p,q \in \{0,1,\ldots,k\}$ where $\VEC{b}_{(p)}$ and
$\VEC{b}_{(q)}$ are vertices on the image of the same function
\[
\Sd_k(\delta_k) =
J_{k-1,\VEC{b}_{(k)}}\big(J_{k-2, F^{s_k}_k(\VEC{b}_{(k-1)})}\big( \ldots
\big(J_{0,F^{s_k}_k \circ \cdots \circ F^{s_2}_2(\VEC{b}_{(1)})}
\big(F_k^{s_k} \circ F_{k-1}^{s_{k-1}} \circ
\ldots \circ F_1^{s_1}\big)\big)\big)\big) : \Delta_k \to \Delta_k
\]
for some $0\leq s_j \leq j$ with $1 \leq j \leq k$, where $\delta_k$ is
the identity map.  We may assume that $p < q$,
$\displaystyle \VEC{b}_{(p)} = \frac{1}{1+p} \sum_{j=0} \VEC{e}_{i_j}$
and
$\displaystyle \VEC{b}_{(q)} = \frac{1}{1+q} \sum_{j=0} \VEC{e}_{i_j}$.
Note that the image of $\Sd_k(\delta_k)$ is nothing else than
$\displaystyle K^{[1]}$ defined in Subsection~\ref{ssectBarSubd} where
$K$ is the $k$-simplex $[\VEC{e}_0,\VEC{e}_1,\ldots,\VEC{e}_k]$.
We can then proceed as in the proof of Proposition~\ref{propMeshMBS}
to show that
\begin{align*}
&\Dm(c) = \|\delta(\VEC{b}_{(p)}) - \delta(\VEC{b}_{(q)})\|
= \left\| \frac{1}{p+1} \sum_{j=0}^p \delta(\VEC{e}_{i_j}) 
- \frac{1}{q+1} \sum_{j=0}^q \delta(\VEC{e}_{i_j}) \right\| \\
&\qquad \leq \frac{1}{(q+1)(p+1)} \sum_{j=0}^q \sum_{j=0}^p
  \left\|  \delta(\VEC{e}_{i_j}) - \delta(\VEC{e}_{i_j})\right\|
\leq \frac{q(p+1)}{(q+1)(p+1)} \Dm(\delta) \\
&\qquad = \frac{q}{q+1} \Dm(\delta) \leq \frac{k}{k+1} \Dm(\delta)
\end{align*}
because
$\left\|  \delta(\VEC{e}_{i_j}) - \delta(\VEC{e}_{i_j})\right\|
\leq \Dm(\delta)$ for all $i$ and $j$, and $q \leq k$.
\end{proof}

Given an affine function $\delta:\Delta_k \to \Delta_k$, it follows
from the previous proposition that
$\displaystyle \Dm\big((\Sd_k)^q(\delta)\big)
\leq \left(\frac{k}{k+1}\right)^q \Dm(\delta) \to 0$
as $q \to \infty$, where
\[
(\Sd_k)^q(\delta)
= \underbrace{\Sd_k(\Sd_k(\ldots(\Sd_k(\delta))))}_{q\text{ times}} \ .
\]
for $q \in \NN$.

The theorem in singular homology that corresponds to
Theorem~\ref{thESAf} is the following theorem.

\begin{theorem}  \label{thSdkqSmall}
Let $\U = \{U_i\}_{i\in I}$ be an open cover of a topological space
$X$ and $R$ be an integral domain.  Given a singular $k$-simplex
$\sigma$ in $X$, there exists $q \in \NN$ such that
$\displaystyle (\Sd_k)^q(\sigma)$ is a linear combination of singular
$k$-simplices small of order $\U$.
\end{theorem}

\begin{proof}
Since $\displaystyle \{\sigma^{-1}(U_i)\}_{i\in I}$ is an open cover
of the compact set $\Delta_k$, there exists a Lebesgue number
$\epsilon$ such that, for every $\VEC{x} \in \Delta_k$, we have that
$\displaystyle B_\epsilon(\VEC{x}) \cap \Delta_k
\subset \sigma^{-1}(U_{i_\VEC{x}})$ for some $i_{\VEC{x}} \in I$;
namely,
$\displaystyle \sigma(B_\epsilon(\VEC{x}) \cap \Delta_k)
\subset U_{i_\VEC{x}}$ for some $i_{\VEC{x}} \in I$.

Choose $q$ large enough to have
$\displaystyle \Dm\big((\Sd_k)^q(\delta_k)\big) < \epsilon$ where
$\delta_k$ is the identity map on $\Delta_k$.
It follows from the commutative diagram in (\ref{SfCommDiagr}) that
$\displaystyle (\Sd_k)^q(\sigma) = S_k(\sigma)\big((\Sd_k)^q(\delta_k)\big)$.
So $\displaystyle (\Sd_k)^q(\sigma)$ is the linear combinations of
singular $k$-simplices of the form $\sigma \circ \delta$, where
$\delta:\Delta_k \to \Delta_k$ is an affine function such that
$\diam(\delta(\Delta_k)) < \epsilon$.  Since
$\delta(\Delta_k) \subset B_\epsilon(\VEC{x}) \cap \Delta_k$ for some
$\VEC{x} \in \Delta_k$, we have that
$(\sigma \circ \delta)(\Delta_k) \subset U_{i_{\VEC{x}}}$ for some
$i_{\VEC{x}} \in I$.
\end{proof}

\begin{prop} \label{propHkXAsmall}
Let $A$ be a subset of a topological space $X$ and $R$ be an integral
domain.  Moreover, let $\U = \{U_i\}_{i\in I}$ be an open cover of $X$.
Every homology class in $H_k(X,A;R)$ can be represented by a relative
$k$-cycle of $X \mod A$ which is a linear combination of singular
$k$-simplices small of order $\U$.
\end{prop}

\begin{proof}
Given $[c]_{X,A} \in H_k(X,A;R)$, we have that
$c$ is a relative $k$-cycle of $X \mod A$.  Recall that this means that
$\partial_k(c) \in S_{k-1}(A;R)$.

It follows from Proposition~\ref{proppTTpISd} that
\begin{equation} \label{propHkRCsmallEq1}
\partial_{k+1}(T_k(c)) + T_{k-1}(\partial_k(c)) = c - \Sd_k(c) \ .
\end{equation}
Since $\partial_k(c) \in S_{k-1}(A;R)$, we have that
$T_{k-1}(\partial_k(c)) \in S_k(A;R)$ follows from the
commutative diagram in (\ref{TfCommDiagr}) with $f$ replaced by the
inclusion map $\iota:A \to X$.  Moreover,
$\Sd_k(c)$ is a relative $k$-cycle of $X \mod A$
because $\partial_k(\Sd_k(c)) = \Sd_{k-1}(\partial_k(c))$
according to Proposition~\ref{propSkm1PkpkSk} and
$\partial_k(c) \in S_{k-1}(A;R)$ implies
that $\Sd_{k-1}(\partial_k(c)) \in S_{k-1}(A;R)$ due to 
the commutative diagram in (\ref{SfCommDiagr}) with $f$ replaced by the
inclusion map $\iota:A \to X$.
It therefore follows from (\ref{propHkRCsmallEq1}) and the definition
of homologous in $X \mod A$ that
$c \hsim \Sd_k(c)$ in $X \mod A$.   We get that
$[c]_{X,A} = [\Sd_k(c)]_{X,A}$ in $H_k(X,A;R)$.

It follows by induction that
$\displaystyle [c]_{X,A} = [(\Sd_k)^q(c)]_{X,A}$ for all $q \in \NN$.
If we apply Theorem~\ref{thSdkqSmall} to each singular $k$-simplex of
the linear combination of $c$, then we can find $q$ large enough to
have that $\displaystyle (\Sd_k)^q(c)$ a linear combination of 
singular $k$-simplices small of order $\U$.
\end{proof}

We can finally prove the excision theorem.

\begin{proof}[Proof (of Theorem~\ref{thmExcis})]
Let $\iota:A \to X$ be the inclusion map.  We prove that
$H_k(\iota):H_k(X\setminus U,A\setminus U;R) \to H_k(X,A;R)$ is an
isomorphism.

\stage{i} We first prove that $H_k(\iota)$ is onto.
Suppose that $[c]_{X,A} \in H_k(X,A;R)$.  It follows from
Proposition~\ref{propHkXAsmall} that we may assume that
$\displaystyle c = \sum_{j\in J} a_j \sigma_j$ is a relative $k$-cycle
where each $\sigma_j$ is small of order
$\displaystyle \U = \{ X \setminus \overline{U}, A^\circ\}$.

If $\sigma_j(\Delta_k) \not\subset X \setminus U$, then
we must have that $\displaystyle \sigma_j(\Delta_k) \subset A^\circ$ because
$X \setminus \overline{U} \subset X \setminus U$.
It follows that $\sigma_j \in S_k(A;R)$.  In other words, we have that
$\sigma_j \in B_k(X,A;R)$ and so $[\sigma_j]_{X,A} = [0]_{X,A}$.  We can
then remove all this terms from the summation and still get a
summation homologous in $X \mod A$ to $c$.    We may therefore assume
that $\sigma_j(\Delta_k) \subset X \setminus U$ for all $j$.

Since $c$ is a relative $k$-cycle and $c \in S_k(X \setminus U;R)$, we
get that \\
$\partial_k(c) \in S_k(A;R) \cap S_k(X \setminus U;R)$.
Thus $\partial_k(c) \in S_k(A\setminus U;R)$.  Hence $c$ is a
relative $k$-cycle in $X\setminus U \mod A \setminus U$.
We have that $[c]_{X\setminus U,A\setminus U}
\in H_k(X\setminus U;A\setminus U;R)$ and
$H_k(\iota)([c]_{X\setminus U,A\setminus U}) = [c]_{X,A} \in H_k(X,A;R)$.

\stage{ii} We now prove that $H_k(\iota)$ is one-to-one.
Suppose that $[c]_{X\setminus U,A\setminus U}
\in H_k(X \setminus U,A\setminus U;R)$
satisfies $H_k(\iota)([c]_{X\setminus U,A\setminus U})
= [c]_{X,A} = [0]_{X,A} \in H_k(X,A;R)$.
Since $\relC[X,A]{c} \in \IMG\big(\overline{\partial}_k\big)
= B_k(X,A;R)/S_k(A;R)$, we have that $c$ is a relative $k$-boundary in
$X \mod A$.  Therefore, there exists $b \in S_{k+1}(X;R)$ such that
$\tilde{c} = c + \partial_{k+1}(b) \in S_k(A;R)$.

We get from Proposition~\ref{propSkm1PkpkSk} that
\begin{equation} \label{thmExcisEq1}
(\Sd_k)^q(\tilde{c}) = (\Sd_k)^q(c) +
\partial_{k+1}\left((\Sd_{k+1})^q(b)\right)
\end{equation}
for $q \geq 0$.
According to Proposition~\ref{propHkXAsmall}, we may choose $q$ large
enough such that $\displaystyle (\Sd_k)^q(b)$ is small of order
$\displaystyle \U = \{X\setminus \overline{U}, A^\circ\}$.  Thus, we may write
$\displaystyle (\Sd_k)^q(b) = b_1 + b_2$ for some $b_1 \in
S_{k+1}(X\setminus \overline{U};R) \subset S_{k+1}(X\setminus U;R)$
and $\displaystyle b_2 \in S_{k+1}(A^\circ;R) \subset S_{k+1}(A;R)$.
We get from (\ref{thmExcisEq1}) that
\[
(\Sd_k)^q(\tilde{c}) - \partial_{k+1}(b_2) = (\Sd_k)^q(c) +
\partial_{k+1}(b_1) \ ,
\]
where the right hand side is in $S_k(X\setminus U;R)$ and the left
hand side is in $S_k(A;R)$.  Therefore, both sides of the previous
equality must be in $S_k(A\setminus U;R)$.  We conclude that
$\displaystyle (\Sd_k)^q(c) \hsim
(\Sd_k)^q(\tilde{c}) - \partial_{k+1}(b_2) \in S_k(A\setminus U;R)$.
Thus $\displaystyle (\Sd_k)^q(c)$ is a relative $k$-boundary in
$X\setminus U \mod A \setminus U$ because
$\displaystyle (\Sd_k)^q(c) \in S_k(X\setminus U;R)$
since $c \in S_k(X\setminus U;R)$.  It follows that
$\displaystyle [c]_{X\setminus U,A\setminus U} =
[(\Sd_k)^q(c)]_{X\setminus U,A\setminus U}
= [0]_{X\setminus U,A \setminus U} \in H_k(X\setminus U,A\setminus U;R)$.
Note that $\displaystyle [c]_{X\setminus U,A\setminus U} =
[(\Sd_k)^q(c)]_{X\setminus U,A\setminus U}$ as we saw in the context
of the proof of Proposition~\ref{propHkXAsmall}.
\end{proof}

Before closing this subsection, we should mention some results to
illustrate the power of the technique of excision.  First, we have to
strengthen the concept of retraction introduced in
definition~\ref{defnRetract} to the context of relative homology.

\begin{defn}
Let $A$ and $Y$ be two subsets of a topological space $X$ and let
$B = Y \cap A$ (Figure~\ref{DefRetr1}).  We say that $(Y,B)$ is a
{\bfseries deformation retract}\index{Deformation Retract} of $(X,A)$
if there exists a function $r:X \to Y$ such that
$r(A) \subset B$, $r \circ \iota = \Id_Y$ where $\iota:Y \to X$ is the
inclusion map (i.e. $Y$ is a retract of $X$), and
$\Id_X \rsim \iota \circ r$.
The homotopy $H:X \times [0,1] \to H$ from $\Id_X$ to $\iota \circ r$
is called {\bfseries deformation retraction}\index{Deformation Retraction}.
\end{defn}

\pdfF{alg_top/defretr1}{Deformation retraction}{$(Y,B)$ is a
deformation retract of $(X,A)$.}{DefRetr1}

In the previous definition, to apply the definition of homotopic
functions given in Definition~\ref{defnRelHF}, we consider
$\Id_X:(X,A) \to (X,A)$ and $\iota \circ r:(X,A) \to (X,A)$.  To get
$(\iota \circ r)(A) \subset B$, we have to require that $r(A) \subset B$.

\begin{prop} \label{propVUAX}
Let $X$ be a topological space and $R$ be an integral domain.  Assume that
$V \subset U \subset A \subset X$ (Figure~\ref{Excision1}).  If $V$
can be excised from $(X,A)$ and $(X\setminus U,A \setminus U)$ is
a deformation retract of $(X \setminus V, A \setminus V)$, then $U$
can be excised from $(X,A)$.
\end{prop}

\begin{proof}
We have that $X \setminus U \subset X \setminus V$,
$A \setminus U \subset A \setminus V$ and
$(A \setminus V) \cap (X \setminus U) = A \setminus U$
(Figure~\ref{Excision1}).  As usual, let $\iota:Y \to X$ be the
inclusion map.  Let $r:X\setminus V \to X \setminus U$ be the
function satisfying $r(x) = x$ for all $x \in X\setminus U$,
$r(A\setminus V) \subset A \setminus U$, and
$\Id_{X \setminus V} \rsim \iota \circ r$.

It follows from $r \circ \iota = \Id_{X\setminus U}$,
$\Id_{X \setminus V} \rsim \iota \circ r$, and
Propositions~\ref{propHkPropS} and \ref{propfgHHkfgE}, that
$H_k(r) \circ H_k(\iota) = H_k(\Id_{X\setminus U}) =
\Id_{H_k(X\setminus U,A\setminus U;R)}$ and
$H_k(\iota) \circ H_k(r) = H_k(\Id_{X\setminus V}) =
\Id_{H_k(X\setminus V,A\setminus V;R)}$.  Thus
$H_k(\iota):H_k(X\setminus U,A\setminus U;R) \to 
H_k(X\setminus V,A\setminus V;R)$ is an isomorphism.

Since $V$ can be excised from $(X,A)$, we have that
$H_k(\iota):H_k(X\setminus V,A\setminus V;R) \to H_k(X,A;R)$ is an
isomorphism.  It follows that
$H_k(\iota):H_k(X\setminus U,A\setminus U;R) \cong H_k(X,A;R)$.
\end{proof}

\pdfF{alg_top/excision1}{Excision theorem}{Representation of the sets
involved in the statement of Proposition~\ref{propVUAX}.
$X\setminus V$ is the union of the pale and darker blue regions.
$A \setminus V$ is the darker blue region.  $X\setminus U$ is the
union of the regions with horizon dashed lines and the region with
horizontal and vertical dashed lines.  $A\setminus U$ is the region with
horizontal and vertical dashed lines.}{Excision1}

\begin{egg}
Let                      \label{eggSpnSmn}
$\displaystyle S^n_+ = \{ \VEC{x} \in S^n : x_{n+1} \geq 0 \}$
and
$\displaystyle S^n_- = \{ \VEC{x} \in S^n : x_{n+1} \leq 0 \}$.
We use the previous proposition to prove below that 
$\displaystyle (S^n_-)^\circ = \{ \VEC{x} \in S^n : x_{n+1} < 0 \}$ 
can be excised from $\displaystyle (S^n,S^n_-)$.  Hence
$\displaystyle H_k(S^n_+,S^{n-1};R) \cong H_k(S^n,S^n_-;R)$ because
$\displaystyle S^n \setminus (S^n_-)^\circ = S^n_+$ and
$\displaystyle S^n_- \setminus (S^n_-)^\circ = S^{n-1}$.

We cannot directly use the excision theorem, Theorem~\ref{thmExcis},
because we are in the situation where
$\displaystyle U = (S^n_-)^\circ$ and $\displaystyle A = S^n_-$ in the 
statement of the theorem.  Therefore $\overline{U} \not\subset A^\circ$.
We have to use the previous proposition with
$\displaystyle V = \{ \VEC{x} \in S^n : x_{n+1} \leq -\epsilon \}$ and
$0 \leq \epsilon < 1$ (Figure~\ref{Excision2}).  We have from
Theorem~\ref{thmExcis} that $V$ can be excised from
$\displaystyle (S^n,S^n_-)$ because
$\displaystyle \overline{V} \subset (S^n_-)^\circ$.  Moreover,
$\displaystyle (S^n_+,S^{n-1})
= \big(S^n \setminus (S^n_-)^\circ, S^n_- \setminus (S^n_-)^\circ\big)$
is a deformation retract of
$\displaystyle (S^n \setminus V,S^n_- \setminus V)$.
A possible function $\displaystyle r: S^n \setminus V \to S^n_+$ for
the deformation retract is represented by the arrows along the great circles in
Figure~\ref{Excision2}.  Only the $\displaystyle \VEC{x} \in (S_-^n)^\circ$ are
moving along the great circles toward $\displaystyle S^{n-1}$.
\end{egg}

\pdfF{alg_top/excision2}{Example of excision theorem}{Representation
of the set $V$ involved in Example~\ref{eggSpnSmn}.  The arrows
illustrate the map $r$ use to show that
$\displaystyle (S^n \setminus (S^n_-)^\circ,S^n_- \setminus (S^n_-)^\circ)$
is a deformation retract of $\displaystyle (S^n \setminus V,S^n_- \setminus V)$.
Only the $\displaystyle \VEC{x} \in (S_-^n)^\circ$ are moving along
the great circles toward $\displaystyle S^{n-1}$.}{Excision2}

\begin{egg}
We may use the tools that we have introduced     \label{eggHkSq}
so far to compute $\displaystyle H_k(S^n)$ for $k,n>0$.

\stage{i}
Let $B^n = \overline{B_1(\VEC{0})} \subset \RR^n$.  So $\partial B^n = S^{n-1}$.
We get from Proposition~\ref{propConnectingH} that
\begin{equation} \label{eggHkSn}
\xymatrix@C+2.5ex{
\ar[r]^-{C_{k+1}} & H_k(S^{n-1};R) \ar[r]^{H_k(\iota)}
& H_k(B^n;R) \ar[r]^-{H_k(\Id_{B^n})} & H_k(B^n,S^{n-1};R) \\
\ar[r]^-{C_k} & H_{k-1}(S^{n-1};R) \ar[r]^-{H_{k-1}(\iota)}
& H_{k-1}(B^n;R) \ar[r]^-{H_{k-1}(\Id_{B^n})} &
}
\end{equation}

\stage{$\mathbf{k>1}$}
Since $\displaystyle B^n$ is contractible, we get from
Corollary~\ref{corContrHk} that $\displaystyle H_k(B^n;R) = 0$
and $\displaystyle H_{k-1}(B^n;R) = 0$.  We deduce from
(\ref{eggHkSn}) that
\[
\xymatrix@C+2.5ex{
0  \ar[r]^-{H_k(\Id_{B^n})} & H_k(B^n,S^{n-1};R) \ar[r]^-{C_k} &
H_{k-1}(S^{n-1};R) \ar[r]^-{H_{k-1}(\iota)} & 0
}
\]
Since $\KE(C_k) = \IMG(H_k(\Id_{B^n})) = \{[0]_{B^n,S^{n-1}}\}$
and $\IMG(C_k) = \KE(H_{k-1}(\iota)) = H_{k-1}(S^{n-1};R)$, we find that $C_k$ is
a isomorphism between $\displaystyle H_k(B^n,S^{n-1};R)$ and
$\displaystyle H_{k-1}(S^{n-1};R)$.

\stage{$\mathbf{k=1}$} We consider two cases: $n>1$ and $n=1$.

For $n>1$, we still have that $\displaystyle H_1(B^n;R) = 0$
because $\displaystyle B^n$ is contractible.  Moreover, we have from
Proposition~\ref{propH0XA0} that
$\displaystyle H_0(B^n,S^{n-1};R) = 0$ because
$\displaystyle B^n$ is path-connected.  Hence, we get from
(\ref{eggHkSn}) that
\[
\xymatrix@C+3ex{
0 \ar[r]^-{H_1(\Id_{B^n})} & H_1(B^n,S^{n-1};R)
\ar[r]^-{C_1} & H_0(S^{n-1};R) \ar[r]^{H_0(\iota)}
& H_0(B^n;R) \ar[r]^-{H_0(\Id_{B^n})} & 0
}
\]
Since $\KE(C_1) = \IMG(H_1(\Id_{B^n})) = \{[0]_{B^n,S^{n-1}}\}$, we
still have that $C_1$ is one-to-one.
We now have that $\displaystyle H_0(B^n;R) \cong R$ and
$\displaystyle H_0(S^{n-1};R) \cong R$ according to
Proposition~\ref{propH0XR} because $\displaystyle B^n$ and
$\displaystyle S^{n-1}$ are path-connected.  Hence, since
$\displaystyle \IMG(H_0(\iota)) = \KE(H_0(\Id_{B^n})) = H_0(B^n;R) \cong R$,
we get
that $\displaystyle \KE(H_0(\iota)) = \{[0]\} \in H_0(S^{n-1};R) \cong R$.
It follows that $\IMG(C_1) = \KE(H_0(\iota)) = \{[0]_{S^{n-1}}\}$.  Since $C_1$
is one-to-one, we must have that
$\displaystyle H_1(B^n,S^{n-1};R) = 0$.

For $n=1$, we still have that $\displaystyle H_1(B^1;R) = 0$
because $\displaystyle B^1 = [-1,1]$ is contractible and
$\displaystyle H_0(B^1,S^0;R) = 0$ because $B^1$ is
path-connected.  Hence, we get from (\ref{eggHkSn}) that
\[
\xymatrix@C+3ex{
0 \ar[r]^-{H_1(\Id_{B^1})} & H_1(B^1,S^0;R)
\ar[r]^-{C_1} & H_0(S^0;R) \ar[r]^{H_0(\iota)}
& H_0(B^1;R) \ar[r]^-{H_0(\Id_{B^1})} & 0
}
\]
Since $\KE(C_1) = \IMG(H_1(\Id_{B^1})) = \{[0]_{B^1,S^0}\}$, we
still have that $C_1$ is one-to-one.
We now have that $\displaystyle H_0(B^1;R) \cong R$ and
$\displaystyle H_0(S^0;R) \cong R^2$ according to
Propositions~\ref{propH0XR} and \ref{propHkeHkJ}.  Hence, since
$\displaystyle \IMG(H_0(\iota)) = \KE(H_0(\Id_{B^1})) = H_0(B^1;R) \cong R$,
we get that $\displaystyle \KE(H_0(\iota)) \cong R$.
It follows that $\IMG(C_1) = \KE(H_0(\iota)) \cong R$.  Since $C_1$
is one-to-one, we have that $\displaystyle H_1(B^1,S^0;R) \cong R$.

In summary, we have found so far that
$\displaystyle H_k(B^n,S^{n-1};R) \cong H_{k-1}(S^{n-1};R)$ for
$k>1$ and $n>0$, $\displaystyle H_1(B^n,S^{n-1};R) = 0$ for
$n>1$ and $\displaystyle H_1(B^1,S^0;R) \cong R$.

\stage{ii}
The projection $\displaystyle \pi:\RR^{n+1} \to \RR^n$ defined by
$\pi(\VEC{x}) = (x_1,x_2,\ldots,x_n)$ for
$\displaystyle \VEC{x} \in \RR^{n+1}$ and $n>0$ provides a homeomorphism
$\displaystyle \pi:(S^n_+,S^{n-1}) \to (B^n,S^{n-1})$.  Hence
$\displaystyle H_k(S^n_+,S^{n-1};R) \cong H_k(B^n,S^{n-1};R)$ for $n>0$.

Moreover, we get from Example~\ref{eggSpnSmn} that
$\displaystyle H_k(S^n_+,S^{n-1};R) \cong H_k(S^n,S^n_-;R)$ for $n>0$.
Thus $\displaystyle H_k(B^n,S^{n-1};R) \cong H_k(S^n,S^n_-;R)$ for $n>0$.

It follows from Proposition~\ref{propConPaConn} that
$\displaystyle H_k(S^n,S^n_-;R) \cong H_k(S^n;R)$ for $k>0$ and $n>0$ because
$\displaystyle S^n$ is path connected and $\displaystyle S^n_-$
is contractible.  Therefore
$\displaystyle H_k(B^n,S^{n-1};R) \cong H_k(S^n;R)$ for $k>0$

In conclusion, we have found that
\begin{enumerate}
\item $\displaystyle H_0(S^0;R) \cong R^2$ comes from
Propositions~\ref{propH0XR} and \ref{propHkeHkJ} because
$\displaystyle S^0 = \{-1,1\}$ has two components,
\item $\displaystyle H_0(S^n;R) \cong R$ for $n>0$ comes
from Propositions~\ref{propH0XR} because $\displaystyle S^n$ is
path-connected,
\item $\displaystyle H_1(S^0;R) = 0$ comes from
Example~\ref{eggHk1} and Proposition~\ref{propHkeHkJ}
because $\displaystyle S^0 = \{-1,1\}$,
\item $\displaystyle H_1(S^1;R) \cong H_1(B^1,S^0;R) \cong R$,
\item $\displaystyle H_1(S^n;R) \cong H_1(B^n,S^{n-1};R) = 0$ for $n>1$, and
\item $\displaystyle H_k(S^n;R) \cong H_k(B^n,S^{n-1};R) \cong
H_{k-1}(S^{n-1};R)$ for $k>1$ and $n>0$.
\end{enumerate}
\end{egg}

\subsection{Mayer-Vietoris Sequence}

The Mayer-Vietoris sequence alluded to in
Subsection~\ref{subsectMVarg} is an import topic.  To prove
Mayer-Vietoris theorem below, we need along the way a very famous
result.  The proof of this result is amusing.

\begin{theorem}[Five Lemma]
Let $R$ be an integral domain.            \index{Five Lemma}
Suppose that we have the following commutative diagram
\[
\xymatrix{
A_1 \ar[r]^{f_1} \ar[d]^{h_1} & A_2 \ar[r]^{f_2} \ar[d]^{h_2}
& A_3 \ar[r]^{f_3} \ar[d]^{h_3} & A_4 \ar[r]^{f_4} \ar[d]^{h_4} &
A_5 \ar[d]^{h_5} \\
B_1 \ar[r]_{g_1} & B_2 \ar[r]_{g_2} & B_3 \ar[r]_{g_3}
& B_4 \ar[r]_{g_4} & B_5
}
\]
where the $A_j$ and $B_j$ are $R$-module, and the $f_j$, $g_j$ and
$h_j$ are $R$-module homomorphism.
Moreover, suppose that each row represents an exact sequence and that
the $h_j$ for $j\neq 3$ are isomorphism.  Then $h_3$ is also an
isomorphism.
\end{theorem}

\begin{proof}
\stage{i} $h_3$ is one-to-one.

Suppose that $h_3(a_3) = 0$.  Using the diagram above, we get that
$g_3(h(a_3)) = 0 = h_4(f_3(a_3))$.  Thus
$f_3(a_3) = 0$ because $h_4$ is an isomorphism.
Therefore, there exists $a_2 \in A_2$ such that $f_2(a_2) = a_3$ because
$\IMG(f_2) = \KE(f_3)$.  Using the diagram above, we get that
$h_3(a_3) = 0 = h_3(f_2(a_2)) = g_2(h_2(a_2))$.  Since
$\IMG(g_1) = \KE(g_2)$, there exists $b_1 \in B_1$ such that
$g_1(b_1) = h_2(a_2)$.  Since $h_1$ is an isomorphism, there exists
$a_1 \in A_1$ such that $h_1(a_1)=b_1$.  Hence, using the diagram
above, we get that $h_2(a_2) = g_1(h_1(a_1)) = h_2(f_1(a_1))$.
Since $h_2$ is an isomorphism, we get that $a_2 = f_1(a_1)$.
Finally, $a_3 = f_2(a_2) = f_2(f_1(a_1)) = 0$ because
$\IMG(f_1) = \KE(f_2)$.  This proves that $h_3$ is one-to-one.

\stage{ii} $h_3$ is onto.

Given $b_3 \in B_3$, there exists $a_4 \in A_4$ such that
$h_4(a_4) = g_3(b_3)$ because $h_4$ is an isomorphism.
Using the diagram above and $\IMG(g_3) = \KE(g_4)$, we then get that
$h_5(f_4(a_4)) = g_4(h_4(a_4)) = g_4(g_3(b_3)) = 0$.
Since $h_5$ is an isomorphism, we get that $f_4(a_4) = 0$.
Since $\IMG(f_3) = \KE(f_4)$, there exists $a_3 \in A_3$ such that
$f_3(a_3) = a_4$.  Hence, using the diagram above, we get that
$g_3(h_3(a_3)) = h_4(f_3(a_3)) = h_4(a_4) = g_3(b_3)$.  Thus
$h_3(a_3) - b_3 \in \KE(g_3)$.  Since $\IMG(g_2) = \KE(g_2)$, there
exists $b_2 \in B_2$ such that $g_2(b_2) = h_3(a_3) - b_3$.
Since $h_2$ is an isomorphism, there
exists $a_2 \in A_2$ such that $h_2(a_2) = b_2$.  Using the diagram
above for the last time, we get that
$h_3(f_2(a_2)) = g_2(h_2(a_2)) = g_2(b_2) = h_3(a_3) - b_3$.  Therefore
$h_3( a_3 - f_2(a_2)) = b_3$.  Since $b_3$ is arbitrary, we have shown
that $\IMG(h_3) = B_3$.
\end{proof}

\begin{defn}
Suppose that $X_1$ and $X_2$ are two topological spaces.  The triad
$(X_1 \cup X_2,X_1,X_2)$ is said to be {\bfseries Exact}\index{Exact Triad} if
$X_1 \setminus X_2$ can be excised from $(X_1 \cup X_2,X_1)$
and $X_2 \setminus X_1$ can be excised from $(X_1 \cup X_2,X_2)$.
\end{defn}

In other words, The triad $(X_1 \cup X_2,X_1,X_2)$ is exact if the
inclusion maps $\iota_2 : (X_2, X_1 \cap X_2) \to (X_1 \cup X_2, X_1)$
and $\iota_1 : (X_1, X_1 \cap X_2) \to (X_1 \cup X_2, X_2)$ are excisions.
For instance, using the notation introduced in
Example~\ref{eggSpnSmn}, we have that
$\displaystyle (S^n,S_n^+,S_n^-)$ is an exact triad.

\begin{lemma} \label{lemEquDofET}
Suppose that $X_1$ and $X_2$ are two topological spaces.
Let
\[
\tilde{H}_k = \KE\big(\partial_k\big|_{S_k(X_1;R) + S_k(X_2;R)}\big)
\big/
\KE\big(\partial_{k+1}\big|_{S_{k+1}(X_1;R) + S_{k+1}(X_2;R)}\big) \ .
\]
Then $(X_1 \cup X_2,X_1,X_2)$ is an exact triad if and only if
$h_k:\tilde{H}_k \to H_k(X_1 \cup X_2;R)$ defined by
$h_k([c_1+c_2]) = [c_1 +c_2]_X$ for all $c_1 \in S_k(X_1;R)$ and
$c_2 \in S_k(X_2;R)$ is an isomorphism.
\end{lemma}

\begin{proof}
Let $X = X_1 \cup X_2$ and $B = X_1 \cap X_2$.

We first note that $h_k$ is well defined.  Suppose that
$[c_1+c_2]=[\tilde{c}_1+\tilde{c}_2]$ with $c_1, \tilde{c}_1 \in S_k(X_1;R)$ and
$c_2, \tilde{c}_2 \in S_k(X_2;R)$.  Then
$(c_1 + c_2) - (\tilde{c}_1 + \tilde{c}_2) = \partial_{k+1}(b_1 + b_2)$ 
for some $b_1 \in S_{k+1}(X_1;R)$ and $b_2 \in S_{k+1}(X_2;R)$.  Since
$b_1 + b_2 \in S_{k+1}(X;R)$, we get that
$[c_1+c_2]_X=[\tilde{c}_1+\tilde{c}_2]_X$.

\stage{i} We prove that the sequence
\begin{equation} \label{triadEqEq1}
\xymatrix{
0 \ar[r] & S_k(X_2;R) \ar[r]^-{\iota}
& S_k(X_1;R) + S_k(X_2;R) \ar[r]^-{\pi_1}
& S_k(X_1;R)/S_k(B;R) \ar[r] & 0
}
\end{equation}
is exact where $\iota(c) = c$ for all $c \in S_k(X_2;R)$ and
$\pi_1(c_1 + c_2) = \relC[X_1,B]{c_1}$ for all
$c_1 \in S_k(X_1;R)$ and $c_2 \in S_k(X_2;R)$.

We first have to prove that $\pi_1$ is well defined.  Suppose
$c_1, \tilde{c}_1\in S_k(X_1;R)$ and $c_2, \tilde{c}_2 \in S_k(X_2;R)$
are such that $c_1 + c_2 = \tilde{c}_1 + \tilde{c}_2$.  Then
$c_1 - \tilde{c}_1 = \tilde{c}_2 - c_2$ where the right side is in
$S_k(X_1;R)$ and the left side in $S_k(X_2;R)$.  Hence
$c_1 - \tilde{c}_1 \in S_k(B;R)$.
Thus $\relC[X_1.B]{c_1} = \relC[X_1.B]{\tilde{c}_1}$.

The function $\iota$ is clearly one-to-one and the function $\pi_1$ is
clearly onto.  We also have that $\IMG(\iota) = S_q(X_2;R) = \KE(\pi_1)$.
Thus the sequence (\ref{triadEqEq1}) is exact.

\stage{ii} Using the Zig-Zag lemma, Lemma~\ref{lemZigZag}, we get from
(\ref{triadEqEq1}) the long exact sequence
\begin{equation} \label{triadEqEq2}
\xymatrix@C+2ex{
\ar[r]^-{\tilde{\partial}_{\,k+1}} & H_k(X_2;R) \ar[r]^-{H_k(\iota)}
& \tilde{H}_k \ar[r]^-{H_k(\pi_1)}
& H_k(X_1,B;R) \ar[r]^-{\tilde{\partial}_k}
& H_{k-1}(X_2;R) \ar[r]^-{H_{k-1}(\iota)} &
}
\end{equation}

Moreover, using again the Zig-Zag lemma with the exact sequence
\[
\xymatrix{
0 \ar[r] & S_k(X_2;R) \ar[r]^-{\iota}
& S_k(X;R) \ar[r]^-{\pi_2} & S_k(X;R)/S_k(X_2;R) \ar[r] & 0
}
\]
we get the long exact sequence
\begin{equation} \label{triadEqEq3}
\xymatrix@C+1em{
\ar[r]^-{\tilde{\partial}_{\,k+1}} & H_k(X_2;R) \ar[r]^-{H_k(\iota)}
& H_k(X;R) \ar[r]^-{H_k(\pi_2)}
& H_k(X,X_2;R) \ar[r]^-{\tilde{\partial}_k} & \\
& H_{k-1}(X_2;R) \ar[r]^-{H_{k-1}(\iota)} & & &
}
\end{equation}
Note that we could also have referred to
Proposition~\ref{propConnectingH} to get (\ref{triadEqEq3}).
In fact, this is an equivalent proof of Proposition~\ref{propConnectingH}.

\stage{iii}
Suppose that $(X,X_1,X_2)$ is an exact triad.  Thus
$H_k(\iota_1):H_k(X_1,B;R) \to H_k(X,X_2;R)$ is an isomorphism.
We join the long exact sequences in (\ref{triadEqEq2}) and
(\ref{triadEqEq3}) to get the following commutative diagram.
\[
\xymatrix@C+2ex{
\ar[r]^-{H_{k+1}(\pi_1)}
& H_{k+1}(X_1,B;R) \ar[r]^-{\tilde{\partial}_{\,k+1}}\ar[d]^{H_{k+1}(\iota_1)}
& H_k(X_2;R) \ar[r]^-{H_k(\iota)}\ar[d]^{\Id}
& \tilde{H}_k \ar[r]^-{H_k(\pi_1)}\ar[d]^{h_k}
& H_k(X_1,B;R) \ar[r]^-{\tilde{\partial}_k}\ar[d]^{H_k(\iota_1)} & \\
\ar[r]^-{H_{k+1}(\pi_2)}
&H_{k+1}(X,X_2;R) \ar[r]^-{\tilde{\partial}_{\,k+1}}
& H_k(X_2;R) \ar[r]^-{H_k(\iota)}
& H_k(X;R) \ar[r]^-{H_k(\pi_2)}
& H_k(X,X_2;R) \ar[r]^-{\tilde{\partial}_k} & \\
& H_{k-1}(X_2;R) \ar[r]^-{H_{k-1}(\iota)}\ar[d]^{\Id} & & & & \\
& H_{k-1}(X_2;R) \ar[r]^-{H_{k-1}(\iota)} & & & &
}
\]
where $\iota_1 : (X_1,B) \to (X,X_2)$ is the inclusion map.
The only condition that may require a little bit attention in order
to prove that we really have a commutative diagram is
$H_k(\iota_1) \circ H_k(\pi_1) = H_k(\pi_2) \circ h_k$.
Given $[c_1 + c_2] \in \tilde{H}_k$, we have that
$H_k(\pi_2)(h_k([c_1+c_2])) = H_k(\pi_2)([c_1+c_2]_X) 
= [c_1 + c_2]_{X,X_2} = [c_1]_{X,X_2}$ because $c_2 \in S_k(X_2;R)$.
Moreover, $H_k(\iota_1)(H_k(\pi_1)([c_1 + c_2]) = H_k(\iota_1)([c_1]_{X_1,B})
= [c_1]_{X,X_2}$.

Since $H_k(\iota_1)$, $H_{k+1}(\iota_1)$ and all the identity maps are
isomorphism, we get from the Five Lemma that $h_k$ is an isomorphism.

\stage{iv} Suppose $h_k:\tilde{H}_k \to H_k(X_1 \cup X_2;R)$ defined by
$h_k([c_1+c_2]) = [c_1 +c_2]_X$ for all $c_1 \in S_k(X_1;R)$ and
$c_2 \in S_k(X_2;R)$ is an isomorphism.
We join the long exact sequences (\ref{triadEqEq2}) and
(\ref{triadEqEq3}) to get the following commutative diagram.
\[
\xymatrix@C+2ex{
\ar[r]^-{\tilde{\partial}_{\,k+1}}
& H_k(X_2;R) \ar[r]^-{H_k(\iota)}\ar[d]^{\Id}
& \tilde{H}_k \ar[r]^-{H_k(\pi_1)}\ar[d]^{h_k}
& H_k(X_1,B;R) \ar[r]^-{\tilde{\partial}_k}\ar[d]^{H_k(\iota_1)}
& H_{k-1}(X_2;R) \ar[r]^-{H_{k-1}(\iota)}\ar[d]^{\Id} & \\
\ar[r]^-{\tilde{\partial}_{\,k+1}}
& H_k(X_2;R) \ar[r]^-{H_k(\iota)}
& H_k(X;R) \ar[r]^-{H_k(\pi_2)}
& H_k(X,X_2;R) \ar[r]^-{\tilde{\partial}_k}
& H_{k-1}(X_2;R) \ar[r]^-{H_{k-1}(\iota)} & \\
& \tilde{H}_{k-1} \ar[r]^-{H_{k-1}(\pi_1)}\ar[d]^{h_{k-1}} & & & & \\
& \tilde{H}_{k-1} \ar[r]^-{H_{k-1}(\pi_1)} & & & & 
}
\]

Since $h_k$, $h_{k-1}$ and all the identity mappings are
isomorphism, we get from the Five Lemma that $H_k(\iota_1)$ is an isomorphism.
\end{proof}

\begin{prop}[Mayer-Vietoris]  \label{propMayVistHom}
Suppose that $(X_1 \cup X_2,X_1,X_2)$ is an exact triad.  Then
the sequence
\[
\xymatrix{
\ar[r]^-{H_{k+1}} & H_k(X_1 \cap X_2;R) \ar[r]^-{F_k}
& H_k(X_1;R) \oplus H_k(X_2;R) \ar[r]^-{G_k} & \\
& H_k(X_1\cup X_2;R) \ar[r]^-{H_k} & H_{k-1}(X_1\cap X_2;R) \ar[r]^-{F_{k-1}} & 
}
\]
is a long exact sequence where
$F_k([c]_{X_1\cap X_2}) = ([c]_{X_2},-[c]_{X_1})$,
$G_k([c_2]_{X_2},[c_1]_{X_1}) = [c_1]_{X_1\cup X_2} + [c_2]_{X_1 \cup X_2}$ and
$H_k:H_k(X_1 \cup X_2;R) \to H_k(X_1 \cap X_2;R)$ is defined in the
proof below.  This sequence is called a
{\bfseries Mayer-Vietoris sequence}\index{Mayer-Vietoris Sequence}
\end{prop}

\begin{proof}
Let $B= X_1\cap X_2$ and $X = X_1 \cup X_2$.

We consider the following three chain complexes.
$\displaystyle \A = \{(S_k(B;R),\partial_k)\}_{k\in \ZZ}$,\\
$\displaystyle \BB = \{(S_k(X_1;R) \oplus S_k(X_2;R),\partial_k)\}_{k\in \ZZ}$
and
$\displaystyle \C = \{(S_k(X_1;R) +S_k(X_2;R),\partial_k\}_{k\in \ZZ}$.
Let $\F = \{f_k\}_{k\in\ZZ}$ be the chain map between $\A$ and $\BB$
defined by $f_k(c) = (c,-c)$ for all $c \in S_k(B;R)$, and
$\GG = \{g_k\}_{k\in\ZZ}$ is a chain map between $\BB$ and $\C$
defined by $g_k(c_1,c_2) = c_1+c_2$ for all $c_1 \in S_k(X_1;R)$
and $c_2 \in S_k(X_2;R)$.
Note that $\partial_k$ is defined on $S_k(X_1;R) \oplus S_k(X_2;R)$
by $\partial_k(c_1,c_2) = (\partial_k(c_1),\partial_k(c_2))$ for all
for all $c_1 \in S_k(X_1;R)$ and $c_2 \in S_k(X_2;R)$.

The sequence
\[
\xymatrix{
0 \ar[r] & S_k(B;R) \ar[r]^-{f_k} & S_k(X_1;R) \oplus S_k(X_2;R) \ar[r]^-{g_k}
& S_k(X_1;R) + S_k(X_2;R) \ar[r] & 0
}
\]
is an exact sequence for all $k$ because $f_k$ is one-to-one, $g_k$ is
onto and $\IMG(f_k) = \KE(g_k) = \{ (c,-c) : c \in S_k(B;R) \}$.
It follows from the Zig-Zag lemma, Lemma~\ref{lemZigZag}, that
\[
\xymatrix{
\ar[r]^-{H_{k+1}} & H_k(\A) \ar[r]^-{F_k}
& H_k(\BB) \ar[r]^-{G_k}
& H_k(\C) \ar[r]^-{H_k} & H_{k-1}(\A) \ar[r]^-{F_{k-1}} & 
}
\]
is a long exact sequence where $H_k(\A) = H_k(B;R)$,
$H_k(\C) \cong H_k(X;R)$ according to the previous lemma, and
\begin{align*}
H_k(\BB)
&= \KE\big(\partial_k\big|_{S_k(X_1;R) \oplus S_k(X_2;R)}\big) \big/
\IMG\big(\partial_{k+1}\big|_{S_{k+1}(X_1;R) \oplus S_{k+1}(X_2;R)}\big) \\
& \cong \KE\big(\partial_k\big|_{S_k(X_1;R)}\big) \big/
\IMG\big(\partial_{k+1}\big|_{S_{k+1}(X_1;R)}\big)
\oplus
\KE\big(\partial_k\big|_{S_k(X_2;R)}\big) \big/
\IMG\big(\partial_{k+1}\big|_{S_{k+1}(X_2;R)}\big) \\
&\cong H_k(X_1;R) \oplus H_k(X_2;R)
\end{align*}
because $\KE\big(\partial_k\big|_{S_k(X_1;R) \oplus S_k(X_2;R)}\big)
= \KE\big(\partial_k\big|_{S_k(X_1;R)}\big) \oplus
\KE\big(\partial_k\big|_{S_k(X_2;R)}\big)$ and\\
$\IMG\big(\partial_{k+1}\big|_{S_{k+1}(X_1;R) \oplus S_{k+1}(X_2;R)}\big)
= \IMG\big(\partial_{k+1}\big|_{S_{k+1}(X_1;R)}\big) \oplus
\IMG\big(\partial_{k+1}\big|_{S_{k+1}(X_2;R)}\big)$.
Note that $F_k = H_k(f_k)$ and $G_k = H_k(g_k)$ in the notation of the
Zig-Zag lemma.

It follows from the Zig-Zag lemma that
$H_k:H_k(X_1 \cup X_2;R) \to H_{k-1}(X_1 \cap X_2;R)$ is defined by
$H_k([c]_{X_1 \cup X_2}) = [a]_{X_1 \cap X_2}$ where
$a \in Z_{k-1}(X_1 \cap X_2;R)$ is such that
$f_{k-1}(a) = (a,-a) = (\partial_k (b_1), \partial_k(b_2))$
for $b_1 \in Z_k(X_1;R)$ and $b_2 \in Z_k(X_2;R)$ with
$g_k(b_1,b_2) = b_1 + b_2 = c$.
\end{proof}

\begin{egg}
The torus $\displaystyle \torus{2} \subset \RR^3$ is the union of two annulus,
$X_1$ and $X_2$, as in Figure~\ref{Torus}.  We have that
$X_1 \cap X_2$ is the union of two disjoint circles.

We prove that $(\torus{2},X_1,X_2)$ is an exact triad.  We first prove that
$\displaystyle U = X_2^\circ = X_2 \setminus X_1$ can be excised from
$(\torus{2},X_2)$.  We cannot use Theorem~\ref{thmExcis} to justify this
operation because $\displaystyle \overline{U} \not\subset X_2^\circ$.
To justify the excision of $U$ from $(\torus{2},X_2)$, we have to use
Proposition~\ref{propVUAX} with $V$ as in Figure~\ref{Torus}.  We have
that $\displaystyle \overline{V} \subset X_2^\circ$.  Therefore, 
$V$ can be excised from $(\torus{2},X_2)$ according to Theorem~\ref{thmExcis}.
Moreover, we have that
$(\torus{2}\setminus U,X_2\setminus U) = (X_1,\partial X_1)$
is a deformation retract of $(\torus{2}\setminus V, X_2 \setminus V)$.  It then
follows from Proposition~\ref{propVUAX} that $\displaystyle X_2^\circ$
can be excised from $(\torus{2},X_2)$.  Similarly,
$\displaystyle X_1^\circ = X_1 \setminus X_2$
can be excised from $(\torus{2},X_1)$.
This proves that $(\torus{2},X_1,X_2)$ is a triad.

\pdfF{alg_top/torus}{Torus}{The torus is the union of two annulus,
$X_1$ and $X_2$, whose intersection is the union of two transversal
sections of the torus homeomorphic to a circle.}{Torus}

We get the following Mayer-Vietoris sequence from the Mayer-Vietoris
theorem.
\[
\xymatrix{
\ar[r]^-{H_{k+1}} & H_k(\partial X_1;R) \ar[r]^-{F_k}
& H_k(X_2;R) \oplus H_k(X_1;R) \ar[r]^-{G_k}
& H_k(\torus{2};R) \ar[r]^-{H_k} & H_{k-1}(\partial X_1;R) \ar[r]^-{F_{k-1}}  &
}
\]
In particular, for $k=1$ and using reduced homology, we get the exact
sequence
\[
\xymatrix{
0 \ar[r]^-{G_2} & H_2(\torus{2};R)  \ar[r]^-{H_2}
& H_1(\partial X_1;R) \ar[r]^-{F_1}
& H_1(X_2;R) \oplus H_1(X_1;R) \ar[r]^-{G_1} & \\
& H_1(\torus{2};R) \ar[r]^-{H_1} &
H_0^\sharp(\partial X_1;R) \ar[r]^-{F_0}  & 0 &
}
\]
because $\displaystyle H_0^\sharp(X_2;R) \cong H_0^\sharp(S^1;R) = 0$,
$\displaystyle H_0^\sharp(X_1;R) \cong H_0^\sharp(S^1;R) = 0$,
$\displaystyle H_2(X_2;R) \cong H_2(S^1;R) = 0$ and
$\displaystyle H_2(X_1;R) \cong H_2(S^1;R) = 0$ according to
the information collected at the end of Example~\ref{eggHkSq}.
We also have that $\displaystyle H_1(X_2;R) \cong H_1(S^1;R) \cong R$ and
$\displaystyle H_1(X_1;R) \cong H_1(S^1;R) \cong R$.  Moreover, we get from
Proposition~\ref{propHkeHkJ} that
$\displaystyle H_1(\partial X_1;R) \cong H_1(S^1;R) \oplus H_1(S^1;R)
\cong R \oplus R$.  Finally, we have that 
$\displaystyle H_0^\sharp(\partial X_1;R) \cong R$ by direct computation.

In summary, the sequence above is equivalent to
\begin{equation} \label{sequTorus}
\xymatrix{
0 \ar[r]^-{G_2} & H_2(\torus{2};R)  \ar[r]^-{H_2}
& R \oplus R \ar[r]^-{F_1} & R \oplus R \ar[r]^-{G_1}
& H_1(\torus{2};R) \ar[r]^-{H_1} & R \ar[r]^-{F_0}  & 0
}
\end{equation}
If we determine the kernel and range of $F_1$, then we will be able to
determine the singular $\displaystyle 1^{st}$ and $\displaystyle 2^{nd}$
homology module of the torus $\torus{2}$.  Recall that
$F_1 = (H_1(\iota) \times H_1(\iota)) \circ \eta_1$ where
the first inclusion map is from $\partial X_1$ into $X_2$, the
second inclusion map is from $\partial X_1$ into $X_1$, and
$\eta_1:H_1(\partial X_1;R) \to H_1(\partial X_1;R) \times
H_1(\partial X_1;R)$ is defined by
$\eta_1([c]_{\partial X_1}) = \big([c]_{\partial X_1},[c]_{\partial X_1}\big)$
for all $[c]_{\partial X_1} \in H_1(\partial X_1;R)$.
Since $\displaystyle \partial X_1 = S^1$, we have from
Example~\ref{eggHkSq} that $H_1(\partial X_1;R) \cong R$.
Thu $\IMG(F_1) \cong R$.  This implies that $\KE(F_1) \cong R$.

We may now combine this information with the sequence in
(\ref{sequTorus}).  Since $H_2$ is one-to-one and
$\IMG(H_2) = \KE(F_1) \cong \RR$, we get that $H_2(\torus{2};R) \cong R$. 
Since $\KE(G_1) = \IMG(F_1) \cong R$, we have that
$\IMG(G_1) \cong R$.  Since $\KE(H_1) = \IMG(G_1) \cong R$ and
$\IMG(H_1) \cong R$, we get that $H_1(\torus{2};R) \cong R \oplus R$. 
\end{egg}

\begin{egg}
To give another example of how Mayer-Vietoris     \label{eggFlower}
sequences can be used,
suppose that $S$ and $T$ are two topological spaces, and that
$s_0 \in S$ and $t_0 \in T$.  Let $X_1 = \{(s,t_0) : s \in S\}$ and
$X_2 = \{(s_0,t) : t \in T\}$.  They are both subsets of $S \times T$.
If $S_0$ is an open neighbourhood of $s_0$ such that
$S \setminus S_0$ is a deformation retract of
$S \setminus \{s_0\}$ and $T_0$ is an open neighbourhood of $t_0$
such that $T \setminus T_0$ is a deformation retract of
$T \setminus \{t_0\}$, then it follows from Proposition~\ref{propVUAX}
that $(X_1 \cup X_2, X_1,X_2)$ is an exact triad.
To be more precise, it follows from Proposition~\ref{propVUAX} with
$A = X_2$, $X = X_1 \cup X_2$,
$U = X_2 \setminus \{(s_0,t_0)\} = X \setminus X_1$ and
$V= X_2 \setminus (\{s_0\}\times T_0)$ that the inclusion map
$\iota_1: (X\setminus U, A \setminus U) = (X_1, X_1 \cap X_2) \to
(X,A) = (X_1 \cup X_2,X_2)$ is an excision.  Note that
$(X \setminus U, A \setminus U) = (X_1, \{(s_0,t_0)\})$ is a
deformation retract of $(X \setminus V,A \setminus V) =
(X_1 \cup (\{s_0\} \times T_0), \{s_0\} \times T_0)$ because
$T \setminus T_0$ is a deformation retract of
$T \setminus \{t_0\}$.  Similarly, we have that
the inclusion map
$\iota_2: (X_2, X_1 \cap X_2) \to (X_1 \cup X_2,X_1)$ is an excision. 

Since $(X_1 \cup X_2, X_1,X_2)$ is an exact triad, we get the
Mayer-Vietoris sequence
\[
\xymatrix{
\ar[r]^-{H_{k+1}} & H_k(\{(s_0,t_0)\};R) \ar[r]^-{F_k}
& H_k(X_2;R) \oplus H_k(X_1;R) \ar[r]^-{G_k} & \\
& H_k(X_1 \cup X_2;R) \ar[r]^-{H_k}
& H_{k-1}(\{(s_0,t_0)\};R) \ar[r]^-{F_{k-1}}  &
}
\]
because $X_1 \cap X_2 = \{(s_0,t_0)\}$.
All the previous theory is also valid for reduced homology.
Since $\displaystyle H_k^\sharp(\{(s_0,t_0)\};R) = 0$ for
all $k \geq 0$, we get the exact sequence
\[
\xymatrix{
0 \ar[r]^-{F_k} & H_k^\sharp(X_2;R) \oplus H_k^\sharp(X_1;R) \ar[r]^-{G_k}
& H_k^\sharp(X_1 \cup X_2;R) \ar[r]^-{H_k} & 0
}
\]
Thus $\displaystyle H_k^\sharp(X_1 \cup X_2;R) \cong
H_k^\sharp(X_2;R) \oplus H_k^\sharp(X_1;R)$.

Suppose that $S$ and $T$ are two subsets of $\displaystyle \RR^n$ that
are homeomorphic to $\displaystyle S^1$ and intersect at a single
point $\VEC{x}_0$ that we may assume to be associated to $(s_0,t_0)$.
The set $G_2 = X_1 \cup X_2$ is represented in Figure~\ref{Flower}.
We get from the previous discussion that
$\displaystyle H_k^\sharp(G_2;R) \cong
H_k^\sharp(S^1;R) \oplus H_k^\sharp(S^1;R)$.

We can generalize this construction by induction.  Suppose that
$G_{q+1}$ is the union of $G_q$ with another ``leave'' $L$ which is
homeomorphic to $\displaystyle S^1$ and such that
$L \cap G_q = \{\VEC{x}_0\}$ as in Figure~\ref{Flower} for $q=4$.
We can show that $(G_{q+1},G_q,L)$ is an exact triad and use the 
Mayer-Vietoris sequence to show that
$\displaystyle H_k^\sharp(G_{k+1};R) = H_k^\sharp(S^1;R)
\oplus H_k^\sharp(G_q;R)$.

Using the information collected at the end of Example~\ref{eggHkSq},
we conclude that
\[
H_k(G_q) \cong 
\begin{cases}
0 & \quad \text{if} \ k > 1 \\
\displaystyle  R^q & \quad \text{if} \ k = 1 
\end{cases}
\]
\end{egg}

\pdfF{alg_top/flower}{Flowers}{The topological spaces $G_2$ and $G_4$
referred to in Example~\ref{eggFlower}}{Flower}

\subsection{Further Studies}

There are a lot more results about singular homology that could be
presented in a first course.  We list a few of them below.

\subsubsection{Vector Fields}

Singular homology theory can be used to give another proof of
Proposition~\ref{propNNVFodd} stating that there is not a
nowhere null vector field on $\displaystyle S^{k-1}$ if $k$ is odd.

\subsubsection{Jordan-Brouwer Separation Theorem}

The ``obvious'' but non-trivial Jordan-Brouwer separation theorem can be
proved using singular homology.  Let
$\displaystyle s_{n-1} \subset S^n$ be a set homeomorphic to
$\displaystyle S^{n-1}$.  The Jordan-Brouwer separation theorem states that
$\displaystyle S^n \setminus s_{n-1}$ is the union of two components
having $s_{n-1}$ as boundary.  In particular, suppose that
$\displaystyle \sigma:[0,1] \to \RR^2$ is a closed path such that
$\sigma(t_1) \neq \sigma(t_2)$ for $t_1 \neq t_2$.  Since the sphere
$\displaystyle S^2$ is homeomorphic to the extended space
$\displaystyle \RR^2$ through stereographic projection
(Figure~\ref{Stereo}), we get that $\sigma([0,1])$ splits
$\displaystyle \RR^2$ into two components whose boundary is
$\sigma([0,1])$ because $\sigma([0,1])$ lifts to a closed path $s_1$ on
$\displaystyle S^2$.  In fact, the result is also true 
for a curve $\displaystyle \sigma:]0,1[ \to \RR^2$ that does not
intersect itself and such that $\sigma(t) \to \infty$ as $t\to 0$ and
$t\to 1$ because this curve can also be lifted to a closed path on
$\displaystyle S^2$.

\pdfF{alg_top/stereo}{Jordan-Brouwer separation theorem}{
The stereographic projection of $\displaystyle S^2$ on
$\displaystyle \RR^2$.}{Stereo}

\section{Singular Cohomology} \label{sectSingCohom}

\subsection{Cohomotopy Modules}

\begin{defn}
Let $X$ be a topological space and $R$ be an integral domain.  The dual
space $\displaystyle S_k(X;R)^\ast$ of $S_k(X;R)$ is denoted
$\displaystyle S^k(X;R)$.  Namely, $\displaystyle S^k(X;R)$ is the
$R$-module of all $R$-linear functionals on $S_k(X;R)$.  The elements
of $\displaystyle S^k(X,R)$ are called
{\bfseries cochaines}\index{Cochaines}. 
\end{defn}

\begin{defn}
Let $X$ be a topological space and $R$ be an integral domain.  The
adjoint of the boundary map $\partial_k: S_k(X;R) \to S_{k-1}(X;R)$ is
the map $\displaystyle \dfC_{k-1}: S^{k-1}(X;R) \to S^k(X;R)$ defined
by
\begin{equation} \label{singCoOpEq1}
\big(\dfC_{k-1}(\phi)\big)(c) = \phi\big(\partial_k(c)\big)
\end{equation}
for all $c \in S_k(X;R)$ and $\displaystyle \phi \in S^{k-1}(X;R)$.
The map $\displaystyle \dfC_k$ is called
the {\bfseries coboundary operator}\index{Coboundary Operator}.
\end{defn}

It is clear that $\displaystyle \dfC_k$ is well and uniquely
defined by (\ref{singCoOpEq1}).  Since
$\partial_{k+1} \circ \partial_{k+2} = 0$, we have that
$\displaystyle \dfC_{k+1} \circ \dfC_k = 0$.

Let $X$ and $Y$ be two topological spaces and $R$ be an integral domain.
Given $f:X\to Y$, we define an homomorphism.
$\displaystyle S^k(f):S^k(Y;R) \to S^k(X;R)$ by
$\displaystyle \big(S^k(f)(\phi)\big)(c) = \phi\big(S(f)(c)\big)$ for all
$\displaystyle c \in S^k(X;R)$ and $\displaystyle \phi \in S^k(Y;R)$.
It is easy to verify that
$\displaystyle \dfC_k \circ S^k(f) = S^{k+1}(f) \circ \dfC_k$.

\begin{defn}
Let $X$ be a topological space and $R$ be an integral domain.  The
$R$-modules of {\bfseries cocycles}\index{Cocycle} and
{\bfseries coboundaries}\index{Coboundary} are respectively defined by
$\displaystyle Z^k(X;R) = \{ \phi \in S^k(X;R) : \dfC_k(\phi) = 0 \}$
and $\displaystyle B^k(X;\RR) = \{ \dfC_{k-1}(\phi) : \phi \in
S^{k-1}(X,R)\}$.
The {\bfseries $\displaystyle \mathbf{k^{th}}$ cohomology
module}\index{Cohomology Module!$k^{th}$ Cohomology Module}
of $X$ is the quotient space $\displaystyle H^k(X;R) = Z^k(X;R) / B^k(X;R)$.
The equivalence class of $\displaystyle H^k(X;R)$ associated to
$\displaystyle \phi \in Z^k(X;R)$ is denoted $[\phi]_X$.
\end{defn}

\begin{rmk}
Let $X$ be a topological space and $R$ be an integral domain.
Consider $\displaystyle [\phi] \in H^k(X;R)$.  If
$\displaystyle \tilde{\phi} \in Z^k(X;R)$ is
another member of the equivalence class $\displaystyle [\phi] \in H^k(X;R)$,
then there exists $\displaystyle \psi \in S^{k-1}(X;R)$ such that
$\displaystyle \phi = \tilde{\phi} + \dfC_{k-1}(\psi)$.
Despite the fact that $\phi$ and $\tilde{\phi}$ may not be equal on
$S_k(X;R)$, we have that
$\displaystyle \phi(c) = \tilde{\phi}(c)
+ \big(\dfC_{k-1}(\psi)\big)(c)
= \tilde{\phi}(c) + \psi(\partial_k(c)) = \tilde{\phi}(c)$ for all
$c \in Z_k(X;R)$.
\end{rmk}

Many of the results in singular homology have equivalent results in 
singular cohomology.  For instance Proposition~\ref{propHkeHkJ}
becomes the following proposition.  The proof of this proposition is
almost identical to the proof of Proposition~\ref{propHkeHkJ}.

\begin{prop} \label{propHkeHkJCoho}
If the topological space $X$ is the union of distinct (connected)
components $X_j$ for $j \in J$, then
$\displaystyle H^k(X;R) = \bigoplus_{j\in J}H^k(X_j;R)$.
\end{prop}

We have a result similar to Theorem~\ref{thmHupeHdk} for simplicial
cohomology.

\begin{theorem} \label{thmHupeHdkS}
Let $X$ be a topological space and $R$ be an integral domain.
Then $\displaystyle H^k(X;R)$ is isomorphic to
$\displaystyle (H_k(X;R))^\ast$.
\end{theorem}

\begin{proof}
The proof is basically identical to the proof of Theorem~\ref{thmHupeHdk}.
We repeat it here for the sake of those who skipped the sections on
simplicial homology and cohomology.
 
The map $\displaystyle h:H^k(X;R) \to (H_k(X;R))^\ast$ defined by
$h([\phi]_X) = \phi$ for $\displaystyle \phi \in Z^k(X;R)$ is well
defined because
\begin{align*}
\phi \in Z^k(X;R) & \Rightarrow \dfC_k(\phi) = 0
\Rightarrow \phi(\partial_{k+1}(c)) = (\dfC_k \phi)(c) =  0
\ \text{for all} \ c \in S_{k+1}(X;R) \\
&\Rightarrow \phi = 0 \ \text{on} \ B_k(X;R)
\Rightarrow \phi \in (H_k(X;R))^\ast \ .
\end{align*}

We have that $h$ is one-to-one.  Suppose that $[\phi_1]_X = [\phi_2]_X$.
Then $\displaystyle \phi_1 - \phi_2 = \psi \in B^k(X;R)$.  Hence
$h([\phi_1]_X) = h([\phi_2]_X) + h([\psi]_X) = h([\phi_2]_X)$
because
\begin{align*}
\psi \in B^k(X;R)
&\Rightarrow \psi = \dfC_{k-1} \eta \ \text{for some}
\ \eta \in S^{k-1}(X;R) \\
&\Rightarrow \psi(c) = (\dfC_{k-1} \eta)(c)
= \eta(\partial_k(c)) = 0 \ \text{for all} \ c \in Z_k(X;R) \\
&\Rightarrow \psi = 0 \ \text{on}\ Z_k(X;R)
\Rightarrow h([\psi]_X) = 0 \ .
\end{align*}

We have that $h$ is onto.  Given $\displaystyle \phi \in (H_k(X;R))^\ast$,
then $\displaystyle [\phi]_X \in H^k(X;R)$ because
\begin{align*}
\phi(c) = 0 \ \text{for all}\ c \in B_k(X;R)
&\Rightarrow (\dfC_k \phi)(c) = \phi(\partial_{k+1}(c)) = 0
\ \text{for all} \ c \in S_{k+1}(X;R) \ \text{since} \\
&\qquad \ \partial_{k+1}(c) \in B_k(X;R) \ \text{for all}
\ c \in S_{k+1}(X;R) \\
&\Rightarrow \dfC_k \phi = 0 \Rightarrow \phi \in Z^k(K;R) \ .
\end{align*}
Thus $h([\phi]_K) = \phi$.

Hence, the map $h$ is an isomorphism between $\displaystyle H^k(X;R)$ and
$\displaystyle (H_k(X;R))^\ast$.
\end{proof}

\begin{cor} \label{corHupeHdkS}
Let $X$ be a topological space and $R$ be an integral domain.
Then $\displaystyle H^k(X;R)$ is isomorphic to
$\displaystyle H_k(X;R)$.
\end{cor}

\begin{proof}
This follows from the previous theorem and the fact that
$\displaystyle H_k(X;R) \cong (H_k(X;R))^\ast$.
\end{proof}

Let $X$ and $Y$ be two topological spaces and $R$ be an integral domain.
Given $f:X\to Y$, we define an homomorphism.
$\displaystyle H^k(f):H^k(Y;R) \to H^k(X;R)$ by
$\displaystyle H^k(f)([\phi]_Y) = [S^k(f)(\phi)]_X$ for all
$\displaystyle [\phi]_Y \in H^k(Y;R)$.  This map is well defined because
$\displaystyle S^k(f)$ maps $\displaystyle Z^k(Y;R)$ to
$\displaystyle Z^k(X;R)$ and
$\displaystyle B^k(Z;R)$ to $\displaystyle B^k(X;R)$.

\subsection{Cochain Complexes}

It should not be a surprise that there is a version of the Zig-Zag
lemma for cohomology.  But before stating it, we need a couple of
definitions.

\begin{defn}
Let $R$ be an integral domain.  A
{\bfseries cochain complex}\index{Cochain Complex} is a sequence
$\displaystyle \C = \{ (C^k,\dfC_k) \}_{k\in \ZZ}$ where the sets $C^k$ are
$R$-modules and the maps $\dfC_k:C_k \to C_{k+1}$ are homomorphism
such that $\dfC_{k+1} \circ \dfC_k = 0$ for $k \in \ZZ$.
\end{defn}

\begin{defn}
A {\bfseries cochain map}\index{Cochain Map} between two chain complexes
given by $\displaystyle \C = \{(C^k,\dfC_k)\}_{k\in \ZZ}$ and
$\displaystyle \tilde{\C} = \{ (\tilde{C}^k,\tilde{\dfC}_k)\}_{k\in \ZZ}$
is a sequence $\F = \{f_k\}_{k\in \ZZ}$ where the maps
$\displaystyle f_k:C^k \to \tilde{C}^k$ are homomorphism such that
$\tilde{\dfC}_k \circ f_k = f_{k+1} \circ \dfC_k$ for $k \in \ZZ$.
\end{defn}

\begin{defn}
Consider a cochain complex given by
$\displaystyle \C = \{(C^k,\dfC_k)\}_{k\in \ZZ}$.
A {\bfseries $\mathbf{k}$-cocycle}\index{$k$-Cocycle}
in $\C$ is an element $\displaystyle c \in C^k$ such that $\dfC_k(c) = 0$.
The set of all $k$-cocycles in $\displaystyle C^k$ is a $R$-module
denoted $\displaystyle Z^k(\C)$.

A {\bfseries $\mathbf{k}$-coboundary}\index{$k$-Coboundary}
in $\C$ is an element $\displaystyle c \in C^k$ with the property that
there exists an element $\displaystyle \tilde{c} \in C^{k-1}$ such that
$\dfC_{k-1}(\tilde{c}) = c$.
The set of all $k$-coboundaries in $\displaystyle C^k$ is a $R$-module denoted
$\displaystyle B^k(\C)$.

The {\bfseries singular $\displaystyle \mathbf{k^{th}}$ cohomology
module}\index{Cohomology Module!Singular $k^{th}$ Cohomology Module}
of $\C$ is the $R$-module
defined as $\displaystyle H^k(\C) = Z^k(\C) / B^k(\C)$.  The
equivalence class of $\displaystyle H^k(\C)$ associated to
$\displaystyle c \in Z^k(\C)$ is denoted $[c]$.
\end{defn}

It follows from the definition that a cochain map
$\F = \{f_k\}_{k\in\ZZ}$ between two chain complexes
$\displaystyle \C = \{(C^k,\dfC_k)\}_{k\in \ZZ}$ and
$\displaystyle \tilde{\C} = \{ (\tilde{C}^k,\tilde{\dfC}_k)\}_{k\in \ZZ}$
sends $k$-cocycles in $\C$ to $k$-cocycles in $\tilde{\C}$, and
$k$-coboundaries in $\C$ to $k$-coboundaries in $\tilde{\C}$.
Hence, we may define $\displaystyle H^k(\F): H^k(\C) \to H^k(\tilde{\C})$ by
$\displaystyle H^k(\F)([c]) = [f_k(c)]$ for all
$\displaystyle c \in Z^k(\C)$.

We can know state the version of the Zig-Zag lemma for cohomology.

\begin{lemma}[Zig-Zag Lemma]  \label{lemZigZagCohom}
Suppose that $\F = \{f_k\}_{k\in\ZZ}$ is a    \index{Zig-Zag Lemma}
cochain map between the cochain
complexes $\displaystyle \A = \{(A^k,\dfC_k^{\A})\}_{k\in \ZZ}$ and
$\displaystyle \BB = \{(B^k,\dfC_k^{\BB})\}_{k\in \ZZ}$, and
$\GG = \{g_k\}_{k\in\ZZ}$ is a cochain map between the cochain complexes
$\displaystyle \BB = \{(B^k,\dfC_k^{\BB})\}_{k\in \ZZ}$ and
$\displaystyle \C = \{(C^k,\dfC_k^{\C})\}_{k\in \ZZ}$.
If
\[
\xymatrix{
0 \ar[r] & A^k \ar[r]^{f_k} & B^k \ar[r]^{g_k} & C^k \ar[r] & 0
}
\]
is an exact sequence for all $k$, then there exists a long exact
sequence
\[
\xymatrix@C+2ex{
& \ar[r]^-{\tilde{\dfC}_{\,k-1}} & H^k(\A) \ar[r]^{H^k(\F)}
& H^k(\BB) \ar[r]^{H^k(\GG} & H^k(\C) \ar[r]^-{\tilde{\dfC}_{\,k}}
& H^{k+1}(\A) \ar[r]^-{H^k(\F)} &
}
\]
where $\tilde{\dfC}_k$ is induced by $\displaystyle \dfC_k^{\BB}$.
\end{lemma}

The proof of this lemma is very similar (dual) to the proof of
Lemma~\ref{lemZigZag} and is left as exercise to the reader.

There is also a dual to Proposition~\ref{propMayVistHom} which could
be proved in a similar fashion or using a duality argument (see
\cite{LJM,MU}) \footnote{Duality and category theories are necessary
tools in the study of cohomology and would have simplified the proofs
of some results.  Since we are not planning to systematically study
cohomology theory, the inclusion of these subjects in these lecture
notes would have added too much extra material to these lecture
notes; in particular, to this chapter which is already longer than
initially intended.}.

\begin{prop}[Mayer-Vietoris]  \label{propMayVistCoHom}
Suppose that $(X_1 \cup X_2,X_1,X_2)$ is an exact triad.  Then 
the sequence
\[
\xymatrix{
\ar[r]^-{H_{k-1}^\ast} & H^k(X_1 \cup X_2;R) \ar[r]^-{F_k^\ast}
& H^k(X_2;R) \oplus H^k(X_1;R) \ar[r]^-{G_k^\ast} & \\
& H^k(X_1\cap X_2;R) \ar[r]^-{H_k^\ast}
& H^{k+1}(X_1\cup X_2;R) \ar[r]^-{F_{k+1}^\ast} & 
}
\]
is an exact sequence where
$\displaystyle F_k^\ast([\phi]_{X_1\cup X_2}) = ([\phi]_{X_2},[\phi]_{X_1})$,
$G_k([\phi_2]_{X_2},[\phi_1]_{X_1}) = [\phi_1]_{X_1\cap X_2}
- [\phi_2]_{X_1 \cap X_2}$ and
$\displaystyle H_k^\ast: H^k(X_1 \cap X_2;R) \to H^{k}(X_1 \cup X_2;R)$
is the dual of
$H_k:H_k(X_1 \cup X_2;R) \to H_k(X_1 \cap X_2;R)$; namely,\\
$\displaystyle H_k^\ast([\phi]_{X_1 \cap X_2})([c]_{X_1\cup X_2})
= [\phi]_{X_1 \cap X_2}\big(H_{k+1}([c]_{X_1\cup X_2})\big)$ for all
$[c]_{X_1 \cup X_2} \in H_{k+1}(X_1 \cup X_2;R)$ and
$\displaystyle [\phi]_{X_1\cap X_2} \in H^k(X_1 \cap X_2;R)$.
\end{prop}

\subsection{Relative Cohomology} \label{ssectRCsingT}

It is also possible to define relative $\displaystyle k^{th}$ cohomology
modules.

\begin{defn}
Let $A$ be a subset of a topological space $X$ and $R$ be an integral
domain.  The adjoint of the boundary operator 
$\overline{\partial}_k : S_k(X;R)/S_k(A;R) \to
S_{k-1}(X;R)/S_{k-1}(A;R)$ is the operator
$\overline{\dfC}_{k-1} : \big(S_{k-1}(X;R)/S_{k-1}(A;R)\big)^\ast
\to \big(S_k(X;R)/S_k(A;R)\big)^\ast$ defined
\begin{equation} \label{relsingCoOpEq1}
\big(\overline{\dfC}_{k-1}(\phi)\big)(\relC[X,A]{c})
= \phi\big(\overline{\partial}_k(\relC[X,A]{c})\big)
\end{equation}
for all $\relC[X,A]{c} \in S_k(X;R)/S_k(A;R)$ and
$\displaystyle \phi \in \big(S_{k-1}(X;R)/S_{k-1}(A;R)\big)^\ast$.
\end{defn}

It is clear that $\displaystyle \overline{\dfC}_k$ is well
and uniquely defined by (\ref{relsingCoOpEq1}).

\begin{defn}
Let $A$ be a subset of a topological space $X$ and $R$ be an integral
domain.  The {\bfseries relative $\displaystyle \mathbf{k^{th}}$ cohomology
module}\index{Cohomology Module!Relative $k^{th}$ Cohomology Module}
of $X \mod A$ is the
quotient space $\displaystyle H^k(X,A;R) = \KE(\overline{\dfC}_k)/
\IMG(\overline{\dfC}_{k-1})$.
The equivalence class of $H^k(X,A;R)$ associated to
$\phi \in \KE(\overline{\dfC}_k)$ is denoted $[\phi]_{X,A}$.
\end{defn}

\begin{prop} \label{propCohoSES}
Let $A$ be a subset of a topological space $X$ and $R$ be an integral
domain.  The short exact sequence
\[
\xymatrix{
0 \ar[r] & S_k(A;R) \ar[r]^-{S_k(\iota)} & S_k(X;R) \ar[r]^-{\pi_k}
& S_k(X;R)/S_k(A;R) \ar[r] & 0
}
\]
yields the short exact sequence
\[
\xymatrix{
0 \ar[r] & (S_k(X;R)/S_k(A;R))^\ast \ar[r]^-{\pi_k^\ast}
& S^k(X;R) \ar[r]^-{S_k(\iota)^\ast} & S^k(A;R) \ar[r] & 0
}
\]
where $\displaystyle S_k(\iota)^\ast : S^k(X;R) \to S^k(A;R)$ is defined by
$\big(S_k(\iota)^\ast(\phi)\big)(c) = \phi\big(S_k(\iota)(c)\big)$
for all $c \in S_k(A;R)$ and $\displaystyle \phi \in S^k(X;R)$, and
$\displaystyle \pi_k^\ast : \big(S_k(X;R)/S_k(A;R)\big)^\ast
\to S^k(X;R)$ is defined by
$\big(\pi_k^\ast(\phi)\big)(c) = \phi(\relC[X,A]{c})$ for all
$c \in S_k(X;R)$ and $\displaystyle \phi \in (S_k(X;R)/S_k(A;R))^\ast$.
\end{prop}

\begin{proof}
\stage{i} $\displaystyle S_k(\iota)^\ast$ is onto.  If
$\displaystyle \phi \in S^k(A;R)$, then we can extend it to a linear
functional $\tilde{\phi}$ on $S_k(X;R)$ by setting
$\tilde{\phi}(\sigma) = \phi(\sigma)$ for all singular $k$-simplices
with image in $A$ and $\tilde{\phi}(\sigma) = 0$ for all singular
$k$-simplices whose image is not in $A$.  Hence,
$\displaystyle \big(S_k(\iota)^\ast(\tilde{\phi})\big)(c)
= \tilde{\phi}( S_k(\iota)(c)) = \phi(c)$ for all
$c \in S_k(A;R)$.

\stage{ii} $\displaystyle \pi_k^\ast$ is obviously one-to-one because
$\displaystyle \big(\pi_k^\ast(\phi)\big)(c) = 0$ for all $c \in S_k(X;R)$
means that $\phi(\relC[X,A]{c}) = 0$ for all
$\relC[X,A]{c} \in S_k(X;R)/S_k(A;R)$.  Thus $\phi = 0$.

\stage{iii} If $\displaystyle \phi \in \KE(S_k(\iota)^\ast)$, then
$\displaystyle 0 = \big(S_k(\iota)^\ast(\phi)\big)(c) =
\phi\big(S_k(\iota)(c)\big)$ for all $c \in S_k(A;R)$.  Therefore
$\phi(c) = 0$ for all $c \in S_k(A;R)$.  Hence, if we set
$\tilde{\phi}(\relC[X,A]{c}) = \phi(c)$ for all
$\relC[X,A]{c} \in S_k(X;R)/S_k(A;R)$, we get a well defined linear
functional on $S_k(X;R)/S_k(A;R)$.  Since
$\displaystyle \big(\pi_k^\ast(\tilde{\phi})\big)(c) =
\tilde{\phi}(\relC[X,A]{c}) = \phi(c)$ for all $c \in S_k(X;R)$, we get that
$\displaystyle \big(\pi_k^\ast(\tilde{\phi})\big) = \phi$ and so
$\displaystyle \phi \in \IMG(\pi_k^\ast)$.

We have shown that
$\displaystyle \KE(S_k(\iota)^\ast) \subset \IMG(\pi_k^\ast)$.  To
prove the reverse inequality, suppose that
$\displaystyle \phi = \pi_k^\ast(\tilde{\phi})$ for
$\displaystyle \tilde{\phi} \in (S_k(X;R)/S_k(A;R))^\ast$.
Then
\begin{align*}
\big(\big(S_k(\iota)^\ast\big)(\phi)\big)(c)
&= \big(\big(S_k(\iota)^\ast\big)\big(\pi_k^\ast(\tilde{\phi})\big)\big)(c)
= \big(\pi_k^\ast(\tilde{\phi})\big)\big(S_k(\iota)(c)\big)
= \tilde{\phi}\big(\pi_k(S_k(\iota)(c))\big) \\
&= \tilde{\phi}(\relC[X,A]{c}) = 0
\end{align*}
for all $c \in S_k(A;R)$.
Thus $\displaystyle \phi \in \KE(S_k(\iota)^\ast)$.  Hence
$\displaystyle \IMG(\pi_k^\ast) \subset \KE(S_k(\iota)^\ast)$.
\end{proof}

It follows from the previous proposition that
$\displaystyle \pi_k^\ast$ is an isomorphism between \\
$\displaystyle (S_k(X;R)/S_k(A;R))^\ast$ and the set
$\displaystyle \{ \phi \in S^k(X;R) : \phi(c) = 0 \ \text{for all}
\ c \in S_k(A;R)\}$.

\begin{defn}
The set $\displaystyle S^k(X,A;R) =
\{ \phi \in S^k(X;R) : \phi(c) = 0 \ \text{for all} \ c \in S_k(A;R)\}$ is the
{\bfseries annihilator}\index{Annihilator} of $S_k(A;R)$.
The set of {\bfseries relative cocycles}\index{Relative Cocycle}
is the set $\displaystyle Z^k(X,A;R) = \{ \phi \in S^k(X,A;R) :
\phi(c) = 0 \ \text{for all} \ c \in B_k(X,A;R) \}$ and
the set of {\bfseries relative coboundaries}\index{Relative Coboundary}
is the set $\displaystyle B^k(X,A;R) = \{ \dfC_{k-1}(\phi) :
\phi \in S^{k-1}(X,A;R) \}$.
\end{defn}

As for the relative homology, we have that
$\displaystyle H^k(X,A;R) \cong  Z^k(X,A;R)/B^k(X,A;R)$.

We can use $\displaystyle \pi_k^\ast: (S_k(X;R)/S_k(A;R))^\ast\to S^k(X;R)$
in Proposition~\ref{propCohoSES} to obtain an isomorphism
between $\displaystyle (S_k(X;R)/S_k(A;R))^\ast$ and
$\displaystyle S^k(X,A;R)$.  Using this isomorphism, then
$\displaystyle \overline{\dfC}_k$ is a restriction 
of $\displaystyle \dfC_k$.

We have that $\displaystyle B^k(X,A;R) \subset Z^k(X,A;R)$ because
$\displaystyle \big(\dfC_{k-1}(\phi)\big)(c)
= \phi(\partial_k(c)) = 0$ for all $c \in B_k(X,A;R)$
if $\displaystyle \phi \in B^k(X,A;R)$.  To be more precise,
$c \in B_k(X,A;R)$ implies that $c = b + \partial_{k+1}(a)$ with
$b \in S_k(B;R)$ and $a \in S_{k+1}(X;R)$.  Thus
$\phi(\partial_k(c)) = \phi(\partial_k(b)) = 0$ because
$\phi(g) = 0$ for all $g \in S_{k-1}(A;R)$.

Unfortunately, we do not have a result as strong as in
Theorem~\ref{thmHupeHdkS} for relative homology.

\begin{theorem}
Let $A$ be a subset of a topological space $X$ and $R$ be a principal
ideal domain.  Then $\displaystyle {\cal H}: H^k(X,A;R) \to (H_k(X,A;R))^\ast$
defined by ${\cal H}([\phi]_{X,A})([c]_{X,A}) = \phi(c)$ for all
$\displaystyle [c]_{X,A} \in H_k(X,A;R)$ and
$\displaystyle [\phi]_{X,A} \in H^k(X,A;R)$ is an epimorphism; namely,
a surjective homomorphism.
\end{theorem}

Instead of giving the definition of principal ideal domain, let us
just say that $\ZZ$ and $\RR$ are principal ideal domain.  The proof
of the previous theorem can be found in (\cite{GH}).

It is probably clear to the reader that we could go on an develop a
full theory of cohomology and relative cohomology as we have done for
singular homology and relative singular homology.  We will not do so
and refer the interested reader to \cite{GH,MUat}.  We have already
gone much further than our initial intention at the beginning of the
chapter to provide a short survey of algebraic topology.

\subsection{Cup Product}

There is however another topic that needs to be mentioned.  There is a
feature of singular cohomology theory that does not have any equivalent in
singular homology theory.  This feature is present in our study of differential
forms on manifolds.  It is the wedge product that combines a
differential $k_1$-form with a differential $k_2$-form to produce a
differential $(k_1+k_2)$-form.  In cohomology theory, this new feature
is called the ``cup product.''\  We have already seen it in our study
of simplicial cohomology.

Let $\delta_1:\Delta_{k_1} \to \Delta_{k_1+k_2}$ and
$\delta_2:\Delta_{k_2} \to \Delta_{k_1+k_2}$ be the affine maps
defined by $\delta_1(\VEC{e}_j) = \VEC{e}_j$ for $0 \leq j \leq k_1$
and $\delta_2(\VEC{e}_j) = \VEC{e}_{k_1+j}$ for $0 \leq j \leq k_2$.

\begin{defn}
Let $X$ be a topological space and $R$ an integral domain, and
let $\displaystyle S^\ast(X;R) = \bigoplus_{k\geq 0} S^k(X;R)$.
The {\bfseries cup product}\index{Cup Product} of
$\displaystyle \phi_1 \in S^{k_1}(X;R)$ with
$\displaystyle \phi_2 \in S^{k_2}(X;R)$ is the linear functional
$\displaystyle \phi_1 \cup \phi_2 \in S^{k_1+k_2}(X;R)$ defined by
$(\phi_1 \cup \phi_2)(\sigma) = \phi_1(\sigma\circ \delta_1) \, 
\phi_2(\sigma\circ \delta_2)$ for all singular $(k_1+k_2)$-simplices
$\sigma$ and extended linearly to $\displaystyle S_{k_1+k_2}(S;R)$.
\end{defn}

Given $\displaystyle \phi = \sum_{j\in \NN} a_j\phi_j$ and 
$\displaystyle \psi = \sum_{j\in \NN} b_j\psi_j$, where
all but a finite number of $a_j\in \RR$ and $b_j\in \RR$ are null,
and $\displaystyle \phi_j,\psi_j \in S^j(X;R)$ for all $j$.  We have
by definition that
\[
\phi \cup \psi = \sum_{i,j\in \NN} a_jb_i \, \phi_j \cup \psi_i \ .
\]
It is not hard to prove that the cup product on $\displaystyle S^\ast(X;R)$
is bilinear, associative and that the identity element in
$\displaystyle S^\ast(X;R)$ is the linear functional
$\displaystyle \phi \in S^0(X;R)$
defined by $\phi(\sigma) = 1$ for all singular $0$-simplices
$\sigma$.  The cup product also satisfies the relation
\begin{equation} \label{partial1U2}
\dfC_{k_1+k_1}(\phi_1 \cup \phi_2) =
\dfC_{k_1}(\phi_1) \cup \phi_2 + (-1)^{k_1} \phi_1 \cup
\dfC_{k_2}(\phi_2)
\end{equation}
for all $\displaystyle \phi_i \in S^{k_1}(X;R)$ and
$\displaystyle \phi_2 \in S^{k_2}(X;R)$.  This is similar to item (2)
of Theorem~\ref{stokesDF} about differential forms.

Let $\displaystyle Z^\ast(X;R) = \bigoplus_{k\geq 0} Z^k(X;R)$ and
$\displaystyle B^\ast(X;R) = \bigoplus_{k\geq 0} B^k(X;R)$.  We have
that $\displaystyle B^\ast(X;R)$ is a (two-sided) ideal of
$\displaystyle Z^\ast(X;R)$.  Note that the cup product of two
elements in $\displaystyle Z^\ast(X;R)$ yields an element in
$\displaystyle Z^\ast(X;R)$.  Hence
$\displaystyle H^\ast(X;R) = \bigoplus_{k\geq 0} H^k(X;R)$
is a graded $R$-algebra.  It is graded because it is the direct sums of
$R$-modules and it is an algebra because of the cup product.

Let $X$ and $Y$ be two topological spaces and $R$ be an integral domain.
Given $f:X\to Y$, we define an homomorphism
$\displaystyle S^\ast(f):S^\ast(Y;R) \to S^\ast(X;R)$ as it follows.  If
$\displaystyle \phi = \sum_{j\in \NN} a_j\phi_j$ where
all but a finite number of $a_j\in \RR$ are null and
$\displaystyle \phi_j\in S^j(X;R)$ for all $j$, then we set
$\displaystyle S^\ast(f)(\phi) = \sum_{j\in \NN} a_j S^j(f)(\phi_j)$.
We may use $\displaystyle S^\ast(f)$ to define
$\displaystyle H^\ast(f):H^\ast(Y;R) \to H^\ast(X;R)$.  Namely, we set
$\displaystyle H^\ast(f)([\phi]_Y) = [ S^\ast(f)(\phi)]_X$ for all
$\displaystyle [\phi]_Y \in H^\ast(Y;R)$.

We define the cup product of
$\displaystyle [\phi_1]_X,[\phi_2]_X \in H^\ast(X;R)$ as
$\displaystyle [\phi_1]_X\cup[\phi_2]_X = [\phi_1 \cup \phi_2]_X$.  To prove
that this cup product is well defined, it suffices to consider
$\displaystyle [\phi_i]_X \in H^{k_i}(X;R)$ with $k_i\in \NN$ for
$i =1,2$.  Suppose that $\phi_i = \psi_i + \dfC_k(\tau_i)$ for some
$\displaystyle \psi_i \in Z^{k_i}(X;R)$ and
$\displaystyle \tau_i \in S^{k_i-1}(S;R)$ for
$i =1,2$.  It follows from (\ref{partial1U2}) that
\begin{align*}
\big(\psi_1 + \dfC_{k_1}(\tau_1)\big) \cup \phi_2
&= \psi_1 \cup \phi_2 + \dfC_{k_1}(\tau_1) \cup \phi_2 \\
&= \psi_1 \cup \phi_2 + \left( \dfC_{k_1+k_1}(\tau_1 \cup \phi_2)
+ (-1)^{k_1+1} \tau_1 \cup \dfC_{k_2}(\phi_2) \right) \\
&= \psi_1 \cup \phi_2 + \dfC_{k_1+k_1}(\tau_1 \cup \phi_2)
\end{align*}
because $\displaystyle \phi_2 \in Z^{k_2}(X;R)$.  Similarly,
\begin{align*}
\psi_1 \cup \big(\psi_2 + \dfC_{k_2}(\tau_2)\big)
&= \psi_1 \cup \psi_2 + \psi_1 \cup \dfC_{k_2}(\tau_2) \\
&= \psi_1 \cup \psi_2 + (-1)^{k_2}\left( \dfC_{k_1+k_1}(\psi_1 \cup \tau_2)
- \dfC_{k_1}(\psi_1) \cup \tau_2 \right) \\
&= \psi_1 \cup \psi_2 + (-1)^{k_2}\dfC_{k_1+k_1}(\psi_1 \cup \tau_2)
\end{align*}
because $\displaystyle \psi_1 \in Z^{k_1}(X;R)$.  Hence, by
transitivity, we have
\[
[\phi_1 \cup \phi_2]_X
= \big[\big(\psi_1 + \dfC_{k_1}(\tau_1)\big) \cup \phi_2\big]_X
= [ \psi_1 \cup \phi_2 ]_X
= \big[\psi_1 \cup \big(\psi_2 + \dfC_{k_2}(\tau_2)\big)\big]_X
= [\psi_1 \cup \psi_2]_X \ .
\]

The relation in singular cohomology that is equivalent to item (4) in
Proposition~\ref{propWedgeR} is
\[
  [\phi_1]_X\cup[\phi_2]_X = (-1)^{k_2} [\phi_2]_X \cup [\phi_1]_X
\]
for $\displaystyle [\phi_i]_X \in H^{k_i}(X;R)$ with $k_i\in \NN$.  To
quote \cite{GH}, ``The proof of this theorem is surpringly
complicated.''\ The proof can be found in \cite{GH}.

\subsection{Relation Between Singular Cohomology and
de Rham Cohomology}  \label{ssectdeRhamSing}

For this section, we follow the approach in \cite{LJM}.

Suppose that $\displaystyle S \subset \RR^n$ is a $k$-dimensional
smooth manifold.  We are going to prove that $\displaystyle H^q(S)$ is
(linearly) isomorphic to $\displaystyle H^q(S;\RR)$ for all $q \in \NN$.
From now on, $\displaystyle H^q(S)$ without any
mention to the integral domain $\RR$ will always refer to the
$\displaystyle q^{th}$ (de Rham) Cohomology Module of $S$ defined in
Definition~\ref{defnqdeRhamCM}.

We define a linear map $\displaystyle \JJ : H^q(S) \to H^q(S;\RR)$
for $q \in \NN$ as it follows.  Given
$\displaystyle [\omega] \in H^q(S)$, we set
\begin{equation} \label{defnCalJ}
  \JJ([\omega])([c]_S) = \int_c \omega
\end{equation}
for all $\displaystyle [c]_S \in H_q(S;\RR)$.  This map is well defined
because it is independent of the representative of the equivalence
class $[c]_S \in H_q(S;\RR)$ and the representative of the equivalent
class $\displaystyle [\omega] \in H^q(S)$ used to define the integral.
Suppose that $\tilde{c} \hsim c$.
Then $\tilde{c} = c + \partial_{q+1}b$ for some
$b \in S_{q+1}(S;\RR)$.  Using Stokes' theorem for chains, we get
\[
\int_{\tilde{c}} \omega = \int_c \omega + \int_{\partial_{q-1} b} \omega
= \int_c \omega + \int_b \df{\omega} = \int_c \omega
\]
because $\df{\omega} = 0$ since $\displaystyle [\omega] \in H^q(S)$.
Suppose that $\tilde{\omega} \sim \omega$.  Then $\tilde{\omega} =
\omega + \df{\eta}$ for some differential $(q-1)$-form $\eta$ on $S$.
Again, using Stokes' theorem for chains, we get
\[
\int_c \tilde{\omega} = \int_c \omega + \int_c \df{\eta}
= \int_c \omega + \int_{\partial_q c} \eta = \int_c \omega
\]
because $\partial_q c = 0$ since $[c]_S \in H_q(S;\RR)$.  
It is easy to verify that $\displaystyle \JJ(\omega) \in H^q(S;\RR)$.

There is a version of Proposition~\ref{propMayVistCoHom} for
cohomology on manifolds.
Suppose that $S$ is a $k$-dimensional smooth manifold.
To reformulate our definition of the cohomology module
$\displaystyle H^q(S)$ given
in Definition~\ref{defnqdeRhamCM} of Chapter~\ref{chaptCohom}, we set
$\displaystyle Z^k(\Omega^{\,q}(S),\df{}) = \{ \omega \in \Omega^{\,q}(S)
:\df{\omega} = 0 \}$ and
$\displaystyle B^k(\Omega^{\,q}(S),\df{}) = \{ \df{\omega}
: \omega \in \Omega^{\,{q-1}}(S) \}$.  Then
$\displaystyle H^q(S) = Z^q(S) / B^q(S)$.

\begin{prop}[Mayer-Vietoris] \label{propMayVistdeR}
Suppose that $S$ is a $k$-dimensional smooth manifold and that
$S = U \cup V$ where $U$ and $V$ are two open subsets of $S$.  Then
the sequence
\[
\xymatrix{
\ar[r]^-{\tilde{H}_{q-1}^\ast} & H^q(S) \ar[r]^-{\tilde{F}_q^\ast}
& H^q(U) \oplus H^q(V) \ar[r]^-{\tilde{G}_q^\ast}
& H^q(U \cap V) \ar[r]^-{\tilde{H}_q^\ast}
& H^{q+1}(S) \ar[r]^-{\tilde{F}_{q+1}^\ast} & 
}
\]
is an exact sequence where
$\displaystyle \tilde{F}_q^\ast([\omega]) = ([\omega\big|_U],[\omega\big|_V])$,
$\displaystyle \tilde{G}_q([\omega_1],[\omega_2]) = [\omega_1\big|_{U\cap V}]
- [\omega_2\big|_{U\cap V}]$ and
$\displaystyle \tilde{H}_q^\ast: H^q(U \cap V) \to H^{q+1}(S)$ is
defined in proof below.
\end{prop}

\begin{proof}
Let $B= U\cap V$.
We consider the following three cochain complexes.
$\displaystyle \A = \{(\Omega^{\,q}(S),\df{})\}_{k\in \ZZ}$,
$\displaystyle \BB = \{(\Omega^{\,q}(U) \oplus \Omega^{\,q}(V),
\df{})\}_{k\in \ZZ}$
and
$\displaystyle \C = \{(\Omega^{\,q}(B),\df{})\}_{k\in \ZZ}$.
Let $\F = \{\tilde{f}_k\}_{k\in\ZZ}$ be the cochain map between $\A$ and $\BB$
defined by $\tilde{f}_k(\eta) = \big(\eta\big|_U,\eta\big|_V\big)$ for all
$\displaystyle \eta \in \Omega^{\,q}(S)$, and
$\GG = \{\tilde{g}_k\}_{k\in\ZZ}$ be the cochain map between $\BB$ and $\C$
defined by
$\tilde{g}_k(\eta_1,\eta_2) = \eta_1\big|_{U\cap V}-\eta_2\big|_{U\cap V}$
for all $\displaystyle \eta_1 \in \Omega^{\,q}(U_1)$ and
$\displaystyle \eta_2 \in \Omega^{\,q}(U_2)$.
Note that the differential operator $\df{}$ is defined on
$\displaystyle \Omega^{\,q}(U) \oplus \Omega^{\,q}(V)$
by $\df{(\eta_1,\eta_2)} = (\df(\eta_1),\df(\eta_2))$ for all
$\displaystyle \eta_1 \in \Omega^{\,q}(U_1)$ and
$\displaystyle \eta_2 \in \Omega^{\,q}(U_2)$.

Our goal is to prove that the sequence
\[
\xymatrix{
0 \ar[r]
& \Omega^{\,q}(S) \ar[r]^-{\tilde{f}_q}
& \Omega^{\,q}(U) \oplus \Omega^q(V) \ar[r]^-{\tilde{g}_q}
& \Omega^{\,q}(B) \ar[r] & 0
}
\]
is an exact sequence.  The conclusion of the proposition will then
follow from the Zig-Zag lemma, Lemma~\ref{lemZigZagCohom}.

It follows from the Zig-Zag lemma that
$\displaystyle \tilde{H}_q^\ast:H^q(U \cap V) \to H^{q+1}(S)$ is defined by
$\displaystyle \tilde{H}_q^\ast([\gamma]) = [\alpha]$ where
$\displaystyle \alpha \in Z^{q+1}(S)$ is such that
$\tilde{f}_{k+1}(\alpha) = (\alpha\big|_U, \alpha\big|_V) =
(\df{\beta_1}, \df{\beta_2})$
for $\displaystyle \beta_1 \in Z^q(U)$ and
$\displaystyle \beta_2 \in Z^q(V)$ with
$\tilde{g}_k(\beta_1,\beta_2) = \beta_1\big|_{U\cap V} - \beta_2\big|_{U\cap V}
= \gamma$.

\stage{i} We have that $\tilde{f}_q$ is one-to-one because
$\tilde{f}_q(\eta) = \big(\eta\big|_U,\eta\big|_V\big) = (0,0)$ implies that
$\eta\big|_S = 0$ since $S = U \cup V$.

\stage{ii} To prove that $\IMG(\tilde{f}_q) = \KE(\tilde{g}_k)$, we note that
\[
  \tilde{g}_q(\tilde{f}_q(\eta)) = \tilde{g}_q\big(\eta\big|_U ,\eta\big|_V\big)
= \eta\big|_B - \eta\big|_B = 0
\]
for all $\displaystyle \eta \in \Omega^{\,q}(S)$.  Thus
$\IMG(\tilde{f}_q) \subset \KE(\tilde{g}_k)$.

Suppose that $(\eta_1,\eta_2) \in \KE(\tilde{g}_k)$.  Then
$\eta_1\big|_B - \eta_2\big|_B = 0\big|_B$.
Let
\[
\eta = \begin{cases}
\eta_1 & \quad \text{on}\ U \\
\eta_2 & \quad \text{on}\ U\setminus V  
\end{cases}
\]
We have that $\displaystyle \eta \in \Omega^{\,q}$ because
$\eta_1 = \eta_2$ on $B = U\cap V$.  Since
$\tilde{f}_k(\eta) = \big(\eta\big|_U,\eta\big|_V\big) = (\eta_1,\eta_2)$,
we have that $(\eta_1,\eta_2) \in \IMG(\tilde{f}_k)$.  This proves that
$\KE(\tilde{g}_k) \subset \IMG(\tilde{f}_q)$.

\stage{iii} We now prove that $\tilde{g}_q$ is onto.
Given $\displaystyle \eta \in \Omega^{\,q}(B)$, we need to find
$\displaystyle \eta_1 \in \Omega^{\,q}(U)$ and
$\displaystyle \eta_2 \in \Omega^{\,q}(V)$ such that
$\tilde{g}_q(\eta_1,\eta_2) = \eta_1\big|_B - \eta_2\big|_B = \eta$.

Since $\{U,V\}$ is an open cover of $S$, we may use Theorem~\ref{cov1}
to get a partition of unity $\displaystyle \{\psi_j\}_{j\in \NNp}$
subordinate to the open cover $\{U,V\}$ of $\supp \eta$.
Let $\displaystyle \phi_1 = \sum_{\supp \psi_j \subset U} \psi_j$
and $\displaystyle \phi_2 = \sum_{\supp \psi_j \subset V} \psi_j$.
The function $\phi_1$ and $\phi_2$ are well defined and of class
$\displaystyle C^\infty$ because the sums are finite on compact sets
according to Proposition~\ref{cov4}.
We have that $\supp \phi_1 \subset U$ and $\supp \phi_2 \subset V$.
Let
\[
\eta_1 = \begin{cases}
\phi_2 \eta & \quad \text{on} \ U \cap \supp \phi_2 \\
0 & \quad \text{on} \ U \setminus \supp \phi_2
\end{cases}
\qquad \text{and} \qquad
\eta_2 = \begin{cases}
-\phi_1 \eta & \quad \text{on} \ V \cap \supp \phi_1 \\
0 & \quad \text{on} \ V \setminus \supp \phi_1  
\end{cases}
\]
Since $U \cap \supp \phi_2 \subset U \cap V$ and
$V \cap \supp \phi_1 \subset V \cap U$ where $\eta$ is defined,
we get that $\eta_1$ and $\eta_2$ are well defined,
$\displaystyle \eta_1 \in \Omega^{\,q}(U)$ and
$\displaystyle \eta_2 \in \Omega^{\,q}(V)$.
We also have that
$\tilde{g}_k(\eta_1,\eta_2) = \eta_1\big|_B - \eta_2\big|_B
= \phi_2 \eta + \phi_1\eta =(\phi_2+\phi_1)\eta = \eta$ on $B$.
\end{proof}

The operator $\JJ$ has some natural properties.

\begin{prop} \label{propFqJcomm}
Suppose that $S_1$ and $S_2$ are $k$-dimensional smooth manifolds.
If $f:S_1 \to S_2$ is a smooth function, then
\[
\xymatrix@C+2em{
H^q(S_2) \ar[r]^-{f^\sharp}\ar[d]^{\JJ} & H_k^q(S_1) \ar[d]^-{\JJ} \\
H^q(S_2;\RR) \ar[r]^-{H^q(f)} & H^q(S_1;\RR)
}
\]
is a commutative diagram for all $q \in \NN$, where
$\displaystyle f^\sharp$ is defined in Subsection~\ref{ssubCMGcase}.
\end{prop}

\begin{proof}
If $\displaystyle [\omega] \in H^q(S_2)$ and $\sigma$ is a $q$-cycle
in $S_1$, then $\displaystyle f^\sharp([\omega]) = [f^\ast(\omega)]$ and
\begin{align*}
\JJ(f^\sharp([\omega]))([\sigma]_{S_1})
&= \int_\sigma f^\ast(\omega)
= \int_{\Delta_q} \sigma^\ast(f^\ast(\omega)
= \int_{\Delta_q} (f\circ \sigma)^\ast(\omega)
= \int_{f \circ \sigma} \omega \\
&= \JJ([\omega])([f\circ \sigma]_{S_2})
= \JJ([\omega])\big( [S_q(f)(\sigma)]_{S_2}\big)
= H^q(f)\big(\JJ([\omega])\big)([\sigma]_{S_1}) \ .
\end{align*}
Hence, by linearity,
$\displaystyle \JJ(f^\sharp([\omega]))([c]_{S_1})
= H^q(f)\big(\JJ([\omega])\big)([c]_{S_1})$ for all
$[c]_{S_1} \in H_q(S_1;\RR)$.
\end{proof}

\begin{lemma} \label{lemSUVET}
Suppose that $S$ is a $k$-dimensional smooth manifold and that
$S = U \cap V$ where $U$ and $V$ are open subsets of $S$.
Then $(S,U,V)$ is an exact triad.
\end{lemma}

\begin{proof}
We have that $S\setminus U = V \setminus U$ is a closed subset
(in the topology induced on $S$) of the open set $V$.  It follows
from Excision theorem, Theorem~\ref{thmExcis}, that $V \setminus U$
can be excised from $(S,V)$; namely,
$H_k(\iota):H_k(U, U \cap V;\RR) \to H_k(S,V;\RR)$ is
an isomorphism.  Similarly, $U \setminus V$ can be excised from
$(S,U)$; namely, $H_k(\iota):H_k(V, U \cap V;\RR) \to H_y(S,V;\RR)$.
\end{proof}

\begin{prop} \label{propHqJcomm}
Suppose that $S$ is $k$-dimensional smooth manifold and that
$S = U \cup V$ where $U$ and $V$ are open subsets of $S$.  Then 
\[
\xymatrix@C+2em{
H^{q-1}(U\cap V) \ar[r]^-{\tilde{H}_{q-1}^\ast}\ar[d]^{\JJ}
& H^q(S) \ar[d]^-{\JJ} \\
H^{q-1}(U\cap V;R) \ar[r]^-{H_{q-1}^\ast} & H^q(S;\RR)
}
\]
is a commutative diagram for all $q \in \NN$ where
$\displaystyle H_q^\ast$ and $\displaystyle \tilde{H}_q^\ast$ are defined in
Propositions~\ref{propMayVistCoHom} and \ref{propMayVistdeR}
respectively.
\end{prop}

\begin{proof}
Note that we can refer to Propositions~\ref{propMayVistCoHom} in the
statement of the proposition because $(S,U,V)$ is an exact triad
according to Lemma~\ref{lemSUVET}.

The diagram states that
\begin{equation} \label{propHqJEq1}
\JJ\big(\tilde{H}_{q-1}^\ast([\omega])\big)([c]_S)
= H_{q-1}^\ast\big(\JJ([\omega])\big)([c]_S)
\end{equation}
for all $\displaystyle [c]_S \in H_q(S;\RR)$ and
$\displaystyle [\omega] \in H^{q-1}(U\cap V)$.  From the duality
mentioned in Proposition~\ref{propMayVistCoHom}, we get that
(\ref{propHqJEq1}) is equivalent to
\begin{equation} \label{propHqJEq2}
\JJ\big(\tilde{H}_{q-1}^\ast([\omega])\big)([c]_S)
= \JJ([\omega])(H_q([c]_S))
\end{equation}
for all $\displaystyle [c]_S \in H_q(S;\RR)$ and
$\displaystyle [\omega] \in H^{q-1}(U\cap V)$.
We can restate (\ref{propHqJEq2}) as
\begin{equation} \label{propHqJEq3}
\int_c \eta = \int_a \omega
\end{equation}
where $\displaystyle [\eta] = \tilde{H}_{q-1}^\ast([\omega]) \in H^q(S)$
and $\displaystyle [a]_{U\cap V} = H_q([c]_S) \in H_q(U\cap V)$
for all $\displaystyle [c]_S \in H_q(S;\RR)$ and
$\displaystyle [\omega] \in H^{q-1}(U\cap V)$.
We prove (\ref{propHqJEq3}).

If we refer to the definition of $H_q$ given in the proof of
Proposition~\ref{propMayVistHom}, we have that
$[a]_{U\cap V} = H_q([c]_S)$ is given by
$a \in Z_{q-1}(U\cap V;\RR)$ such that
$f_{k-1}(a) = (a,-a) = (\partial_k (b_1), \partial_k(b_2))$ for
$b_1 \in Z_k(U;R)$ and $b_2 \in Z_k(V;R)$ with
$g_k(b_1,b_2) = b_1 + b_2 = c$ in $S$.
In particular, we have that $a = \partial_k(b_1) = \partial_k(b_2)$
with $\partial_k(b_1)(\Delta_k) \subset U \cap V$ and
$\partial_k(b_2)(\Delta_k) \subset U \cap V$. 

If we now refer to the definition of
$\displaystyle \tilde{H}_{q-1}^\ast$ given in
the proof of Proposition~\ref{propMayVistHom}, we have that
$\displaystyle [\eta] = \tilde{H}_{q-1}^\ast([\omega])$ is given by
$\displaystyle \eta \in Z^{q+1}(S)$ is such that
$\tilde{f}_{k+1}(\eta) = (\eta\big|_U, \eta\big|_V) =
(\df{\beta_1}, \df{\beta_2})$
for $\displaystyle \beta_1 \in Z^q(U)$ and
$\displaystyle \beta_2 \in Z^q(V)$ with
$\tilde{g}_k(\beta_1,\beta_2) = \beta_1\big|_{U\cap V} - \beta_2\big|_{U\cap V}
= \omega$.  In particular, we have that
$\eta\big|_U = \df{\beta_1}$ and $\eta\big|_V = \df{\beta_2}$.

Since $\partial_q(b_1) + \partial_q(b_2) = \partial_q(c) = 0$ because
$c \in Z_q(S;\RR)$ and $\df{\beta_1}\big|_{U\cap V}
- \df{\beta_2}\big|_{U\cap V} = \df{\omega} = 0$
because $\displaystyle \omega \in Z^{q-1}(U\cap V)$, we get from
Stokes' theorem that
\begin{align*}
\int_a \omega = \int_{\partial_k(b_1)} \omega
&= \int_{\partial_k(b_1)} (\beta_1 - \beta_2)
= \int_{\partial_k(b_1)} \beta_1 - \int_{\partial_k(b_1)} \beta_2
= \int_{\partial_k(b_1)} \beta_1 + \int_{\partial_k(b_2)} \beta_2 \\
&= \int_{b_1} \df{\beta_1} + \int_{b_2} \df{\beta_2}
= \int_{b_1} \eta + \int_{b_2} \eta = \int_c \eta \ .
\end{align*}
This proves (\ref{propHqJEq3}).
\end{proof}

\begin{theorem}[de Rham]  \label{deRhamThmSing}
Let $\displaystyle S \subset \RR^n$ be a $k$-dimensional smooth
manifold and $q \in \NN$.  The map
$\displaystyle \JJ : H^q(S) \to H^q(S;\RR)$ defined in  
(\ref{defnCalJ}) is an isomorphism.
\end{theorem}

\begin{proof}
For convenience, we will say that a smooth manifold $M$ is a
{\bfseries de Rham} manifold if
$\displaystyle \JJ : H^q(M) \to H^q(M;\RR)$ is an isomorphism.

\stage{Claim 1} If $U$ is a convex open subset of $\displaystyle \RR^k$,
then $U$ is a de Rham manifold.

It follows from Corollary~\ref{corHqg0e0} (see comments after the
corollary) and Corollary~\ref{corContrHk} that $\displaystyle H^q(U) = 0$
and $\displaystyle H^q(U;\RR) \cong H_q(U;\RR) = 0$ for $q>0$,
where Corollary~\ref{corHupeHdkS} was used for the last relation.  The
condition for being a de Rahm manifold is therefore trivially satisfied.

If $q=0$, we get from Corollaries~\ref{corHqe0e1} and
\ref{corContrHk} that $\displaystyle H^0(U) \cong \RR$
and $\displaystyle H^0(U;\RR) \cong H_0(U;\RR) \cong \RR$.
Hence $\displaystyle H^0(U)$ is the set of constant functions on $U$
and $\displaystyle H^0(U;\RR)$ is the set of constant functions on
$\{ e_{\VEC{u}} : \VEC{u} \in U\}$ where $e_{\VEC{u}}:\Delta_0 \to U$ is
defined by $e_{\VEC{u}}(\VEC{e}_0) = \VEC{u}$ for all $\VEC{u} \in U$.
If we identify $e_{\VEC{u}}$ to $\VEC{u}$, we have the set of constant
functions $f$ on $U$ in both cases.  Hence
\[
\JJ(f)(\sigma) = \int_\sigma f = \int_{\Delta_0} \sigma^\ast(f)
= f(\sigma(\VEC{e}_0)) = f(\VEC{u}) \neq 0
\]
for all singular $0$-simplices $\sigma = e_{\VEC{u}}$ in $U$ unless $f = 0$.
Thus $\displaystyle \JJ : H^0(U) \to H^0(U;\RR)$ is an isomorphism.

\stage{Claim 2} If $\displaystyle \{S_j\}_{j=0}^\infty$ is a
collections of disjoint de Rahm manifolds where each $S_j$ is
connected, then $\displaystyle \bigcup_{j=0}^\infty S_j$ is a de Rham
manifold.

This follows from the fact that the inclusion
$\displaystyle \iota : S_i \to \bigcup_{j=0}^\infty S_j$ induces an
isomorphism $\displaystyle \tilde{\iota} : \bigoplus_{j=0}^\infty S_j
\to \bigcup_{j=0}^\infty S_j$.
Hence, using Propositions~\ref{propHkeHkJ} and \ref{propHkeHkJCoho},
we get that\\
$\displaystyle H_q(\tilde{\iota}) : \bigoplus_{j=0}^\infty H_q(S_j;\RR)
\to H_q\Big(\bigcup_{j=0}^\infty S_j;\RR \Big)$ and
$\displaystyle \tilde{\iota}^{\, \sharp}:\bigoplus_{j=0}^\infty H^q(S_j)
\to H^q\Big(\bigcup_{j=0}^\infty S_j\Big)$ are isomorphism
(see comment after Proposition~\ref{propFsharpP}).
We have from Proposition~\ref{propFqJcomm} that
\[
\xymatrix@C+2em{
H^q\Big(\bigcup_{j=0}^\infty S_j\Big)
\ar[r]^{\tilde{\iota}^{\,\sharp}}\ar[d]^{\JJ} & \bigoplus_{j=0}^\infty H^q(S_j)
\ar[d]^-{\JJ} \\
H^q\Big(\bigcup_{j=0}^\infty S_j;\RR\Big) \ar[r]^{H_q(\tilde{\iota})} &
\bigoplus_{j=0}^\infty H^q(S_j;\RR)
}
\]
is a commutative diagram.  Since $H_q(\tilde{\iota})$ and
$\displaystyle \tilde{\iota}^{\,\sharp}$ are isomorphism and
$\displaystyle \JJ : H^q(S_j) \to H^q(S_j;\RR)$ 
is an isomorphism for all $j$, we get that
$\displaystyle \JJ : H^q\Big(\bigcup_{j=0}^\infty S_j\Big)
\to H^q\Big(\bigcup_{j=0}^\infty S_j;\RR\Big)$ is an isomorphism.

\stage{Claim 3} If $\displaystyle U = U_1 \cup U_2$ where $U_1$ and
$U_2$ are two open subsets of $S$, and the sets $U_1$, $U_2$ and
$U_1 \cap U_2$ are de Rham manifolds, then $U$ is a de Rham manifold.

According to Lemma~\ref{lemSUVET}, $(U_1 \cup U_2,U_1,U_2)$ is an exact triad.
Hence, we get from Proposition~\ref{propMayVistCoHom} and
Proposition~\ref{propMayVistdeR} that each row of the following
diagram is exact.
\[
\xymatrix{
\ar[r]^-{\tilde{F}_{q-1}^\ast}
& H^{q-1}(U_1) \oplus H^{q-1}(U_2) \ar[r]^-{\tilde{G}_{q-1}^\ast}
\ar[d]^{\JJ}
& H^{q-1}(U_1 \cap U_2) \ar[r]^-{\tilde{H}_{q-1}^\ast}\ar[d]^{\JJ}
& H^q(U) \ar[r]^-{\tilde{F}_q^\ast}\ar[d]^{\JJ} & \\
\ar[r]^-{F_{q-1}^\ast}
& H^{q-1}(U_2;R) \oplus H^{q-1}(U_1;R) \ar[r]^-{G_{q-1}^\ast}
& H^{q-1}(U_1\cap U_2;R) \ar[r]^-{H_{k-1}^\ast}
& H^k(U;R) \ar[r]^-{F_k^\ast} & \\
& H^q(U_1) \oplus H^q(U_2) \ar[r]^-{\tilde{G}_q^\ast}\ar[d]^{\JJ}
& H^q(U_1 \cap U_2) \ar[r]^-{\tilde{H}_q^\ast}\ar[d]^{\JJ}
& & \\
& H^k(U_2;R) \oplus H^k(U_1;R) \ar[r]^-{G_k^\ast}
& H^k(U_1\cap U_2;R) \ar[r]^-{H_k^\ast}
& &
}
\]
It follows from Propositions~\ref{propFqJcomm} and \ref{propHqJcomm}
that this diagram is commutative.  Since the first, second, fourth
and fifth maps $\JJ$ are isomorphism by assumption, we get from
the Five Lemma that $\displaystyle \JJ: H^k(U) \to H^k(U;R)$ is also an
isomorphism.

\stage{Claim 4} If $\displaystyle U = \bigcup_{j=1}^J U_j$
where each $U_j$ is an open subset of $S$ and a de Rham manifold, and
any finite intersection of $U_j$ with $1\leq j \leq J$ is a de Rham
manifold, then $U$ is a de Rham manifold.

The proof is by induction on $J$.  The result is trivial if $J=1$.
Suppose that the result is true for $J$ and that
$\displaystyle U = \bigcup_{j=1}^{J+1} U_j$
where each $U_j$ is an open subset of $S$ and a de Rham manifold, and
any finite intersection of $U_j$ for $1 \leq j \leq J+1$ is a de Rham
manifold. 

By induction, we have that $\displaystyle \tilde{U} = \bigcup_{j=1}^J U_j$
is a de Rham manifold.  If we apply Claim 3 to $\tilde{U}$ and $U_{J+1}$,
we get that $U = \tilde{U} \cup U_{J+1}$ is a de Rham manifold.
Note that $\displaystyle \tilde{U} \cap U_{J+1}
= \bigcup_{j=1}^J (U_j \cap U_{J+1})$ is a de Rham manifold because
each $U_j \cap U_{j+1}$ is a de Rahm manifold by assumption and
any finite intersection of $U_j \cap U_{J+1}$ is a finite intersection
of $U_j$ with $1 \leq j \leq J+1$ which is also a de Rham manifold by
assumption.  Thus, by induction hypothesis with $U_j \cap U_{J+1}$
with $1 \leq j \leq J$ instead of $U_j$ with $1\leq j \leq J$, we get
that $U = \tilde{U} \cup U_{J+1}$ is a de Rham manifold.

\stage{Claim 5} If $S$ has a basis of open sets $\{ U_i\}_{i \in I}$
such that each $U_i$ is a connected, bounded and open set
which is also a de Rham manifold, and finite intersections of sets
from $\{ U_i\}_{i \in I}$ are de Rham manifolds, then $S$ is a de Rham
manifold.

Let $\displaystyle \{\phi_j\}_{j \in \NNp}$ be a partition of unity
subordinated to $\{ U_i\}_{i \in I}$.  Consider the function
$\displaystyle g = \sum_{j=1}^\infty j \phi_j$ on $S$.  Note that the
sum $\displaystyle \sum_{j=1}^\infty j \phi_j$ is finite on compact
subsets of $S$ according to Proposition~\ref{cov4}.
Thus $g$ is well defined.  We also have that
$\displaystyle g(\VEC{u}) = \sum_{j=1}^\infty j \phi_j(\VEC{u})
\geq \sum_{j=1}^\infty \phi_j(\VEC{u}) = 1$ for all $\VEC{u} \in S$.

Moreover
$\displaystyle g^{-1}([0,a]) \subset \bigcup_{j=1}^{J_a} \overline{U}_{i_j}$
for some $i_j \in I$ such that $\supp \phi_j \subset U_{i_j}$
and $J_a$ is a positive integer greater or equal to $a$.
To prove this statement, we prove that, if
$\displaystyle \VEC{u} \not\in \bigcup_{j=1}^{J_a} \overline{U}_{i_j}$, then
$d(\VEC{u}) > 1$.  Since $\phi_j(\VEC{u}) = 0$ for $1 \leq j \leq j_a$
because $\VEC{u} \not\in U_{i_j} \supset \supp \phi_j$ for $1 \leq j \leq j_a$,
we get that
$\displaystyle g(\VEC{u}) = \sum_{j=1}^\infty j \phi_j(\VEC{u})
= \sum_{j=J_a+1}^\infty j \phi_j(\VEC{u})
\geq (J_a +1) \sum_{j=J_a+1}^\infty \phi_j(\VEC{u})
= J_a +1 > c$.

Note that $\displaystyle g^{-1}([0,a])$ is a closed subset of the
compact set $\displaystyle \bigcup_{j=1}^{J_a} \overline{U}_{i_j}$.
Thus $\displaystyle g^{-1}([0,a])$ is a compact subset of $S$ for all
$a >0$.

Let $\displaystyle V_m = g^{-1}([m,m+1])$ and
$\displaystyle W_m = g^{-1}(]m-1/2,m+3/2[)$ for $m \in \NN$.
Since $W_m$ is an open subset of $S$, it is the union of some open sets
from $\{ U_i\}_{i \in I}$.  Since $V_m \subset W_m$, these subsets
form an open cover of the compact set $V_m$.  Thus, there exists a
finite subcover
$\displaystyle \{ U_{i_{m,r}}\}_{r=1}^{R_m}$ of $V_m$.
We have from Claim 4
that $\displaystyle G_m = \bigcup_{r=1}^{R_m} U_{i_{m,r}}$ is a de
Rham manifold.  Note that $G_m$ is the union of more than one distinct
connected de Rham manifolds if
$\displaystyle \bigcap_{r=1}^{R_m} U_{i_{m,r}} = \emptyset$.

By construction, we have that $V_m \subset G_m \subset W_m$.
Thus $G_{m_1} \cap G_{m_2} = \emptyset$ for $|m_2 - m_1| > 1$.
Hence $\displaystyle G_e = \bigcup_{m\ \text{even}} G_m$ and
$\displaystyle G_o = \bigcup_{m\ \text{odd}} G_m$ are both the union
of distinct connected de Rham manifolds.  It follows from Claim 2 that
$G_e$ and $G_o$ are de Rham manifolds.

Since $\displaystyle G_m \cap G_{m+1}
= \bigcup_{r=1}^{R_m} \bigcup_{\tilde{r}=1}^{R_{m+1}} U_{i_{m,r}}
\cap U_{i_{m+1,\tilde{r}}}$, we get that $G_m \cap G_{m+1}$ is the finite
union of open and connected de Rham manifolds because any finite
intersection of $U_i$ is de Rham manifolds by assumption.
Thus $G_m \cap G_{m+1}$ is the disjoint union of open and connected de
Rham Manifolds according to Claim 4.
Since $(G_{m_1} \cap G_{m_1+1}) \cap
(G_{m_2} \cap G_{m_2+1}) = \emptyset$ for $m_1 \neq m_2$, it follows
that $\displaystyle G_e \cap G_o = \bigcup_{m=1}^\infty (G_m \cap G_{m+1})$
is the disjoint union of open and connected de Rham Manifolds.  We get
from Claim 2 that $G_e \cap G_o$ is a de Rham manifold.
Finally,  it follows from Claim 3 that $S = G_e \cup G_o$ is a de Rham
manifold.
\pdfbox{alg_top/derham}

\stage{Claim 6} If $\displaystyle U \subset \RR^k$ is an open set,
then $U$ is a de Rham manifold.

We have that $\displaystyle U = \bigcup_{i\in I} U_i$ where each $U_i$
is an open ball in $\displaystyle \RR^k$.  It follows from Claim 1
that the $U_i$ are de Rham manifold since open balls are convex.
The $U_i$ are also connected and bounded.  Since the finite
intersection of open balls is also a convex set, we have that finite
intersections of sets from $\{ U_i\}_{i \in I}$ are de Rham manifolds.
Therefore the collection $\{ U_i\}_{i \in I}$ meets the requirements
of Claim 5.  Hence $U$ is de Rham.

\stage{Claim 7} Finally, if $\displaystyle S \subset \RR^n$ is a
$k$-dimensional smooth manifold, then $S$ is a de Rham manifold.

The manifold $S$ has a basis of open set $\{ U_i\}_{i \in I}$ where
each $U_i$ is associated to a local chart $(W_i,U_i,\phi_i)$ of $S$.
We may select the $U_i$ to be bounded and connected \footnote{For
instance, for each $\VEC{u}$, we may use $U_{\VEC{u}} = S \cap B_r(\VEC{u})$
for $r$ small enough, and use $I = S$ as set of indices.}.
Since each $U_i$ is diffeomorphic to the open set
$\displaystyle W_i \subset \RR^k$ and $W_i$ is a de Rham manifold
according to Claim 6, we get from Proposition~\ref{propFqJcomm} that
$U_i$ is a de Rham manifold.   This is also true for the finite
intersection of set from $\{ U_i\}_{i \in I}$ be definition of local
charts.  It follows from Claim 5 that $S$ is a de Rham manifold.
\end{proof}

%%% Local Variables:
%%% mode: latex
%%% TeX-master: "notes"
%%% End:


\section{Relation Between Simplicial and Singular Theory}
\label{sectRelSandS}

This section is not needed for the next chapter and, in a sense, is
not related to the goal of these lecture notes.  However, after having
introduced both the simplicial and singular homology theory, we feel
that it is important to know how they are related and thus how they
may interact.  After having read the sections about the simplicial
and singular theories, the reader has probably realized that there are
a lot of similarity between the theoretical results and techniques
used in both theories.  As we demonstrate in this section, this in not
really surprising.

The style of this section is a little lest detailed than in the rest
of these lecture notes.

\subsection{Relation Between Simplicial Homology and Singular
Homology}

We follow the presentation in \cite{MUat}.

Through out this section, $L$ is a subcomplex of a simplicial complex
$K$ and $R$ is an integral domain.  We assume that
an indexing of the vertices of $K$ has been selected.  More
specifically, we assume that $\VEC{x}_0$, $\VEC{x}_1$, $\VEC{x}_2$,
\ldots are the vertices of $K$.  In particular, we assume that there
could be countably infinitely many open simplices in $K$  This
is more general than in Sections~\ref{sectSimplHomol} and
\ref{sectSimplCohom} where we assumed that there were only a finite
number of open simplices in $K$.   Most of what we said in
Sections~\ref{sectSimplHomol} and \ref{sectSimplCohom} is true
in the case where $K$ contains countably infinitely many open
simplices.  However, there are a few exceptions.  Barycentric
subdivisions and simplicial approximation are two concepts that need
to be generalized to address the issue that $K$ may contain countably
infinitely many open simplices.  The reader may find more information
about this subject in \cite{MUat}.

Adding the possibility that the simplicial complex $K$ could contain
countably infinitely many open simplices does not really increase the
level of complexity in this section.  The only extra work is in part
(II) of the proof of Proposition~\ref{propHkEHkv1}.

\subsubsection{Ordered Simplicial Homology}

We need to enlarge $C_k(K;R)$ to be able to define a mapping
onto $S_k([K];R)$.  For instance, we need a set large enough to be
able to map an element of this set to a singular $k$-simplex
$\sigma:\Delta_k \to [K]$ representing a constant function into $[K]$.

In addition to the $k$-simplices $(s) \in K$, we
also consider elements of the form\\
$(\VEC{y}_0,\VEC{y}_1, \ldots, \VEC{y}_k)$
where $\VEC{y}_0$, $\VEC{y}_1$, \ldots, $\VEC{y}_k$ are vertices of a
$k$-simplex $(s) \in K$, and the $\VEC{y}_j$ may not be all distinct.

\begin{defn}
Elements of the form $(\VEC{y}_0,\VEC{y}_1, \ldots, \VEC{y}_k)$ as
described above are called
{\bfseries ordered $\mathbf{k}$-simplices}\index{Ordered $k$-Simplex}
of $K$.
\end{defn}

\begin{defn}
Let $\tilde{C}_k(K;R)$ be the free abelian group generated by the
elements $(\VEC{y}_0,\VEC{y}_1, \ldots, \VEC{y}_k)$ as described
above.  Elements of $\tilde{C}_k(K;R)$ are called
{\bfseries ordered $\mathbf{k}$-chains}\index{Ordered $k$-Chain}
of $K$.
\end{defn}

We define the boundary operator $\tilde{\partial}_k$ on $\tilde{C}_K(K;R)$ as
we did for the boundary operator $\partial_k$ on $C_K(K;R)$; namely,
\[
\tilde{\partial}_k (\VEC{y}_0,\VEC{y}_1, \ldots, \VEC{y}_k)
= \sum_{i=0}^k (-1)^i
(\VEC{y}_0,\VEC{y}_1,\ldots,\widehat{\VEC{y}_i}, \ldots,
\VEC{y}_k) \ .
\]

In analogy to $H_k(K;R) = \KE(\partial_k) / \IMG(\partial_{k-1})$,
we can define $\tilde{H}_k(K;R) = \KE(\tilde{\partial}_k)
/ \IMG(\tilde{\partial}_{k-1})$.  Likewise, we define the
quotient space
$\displaystyle \tilde{C}_k(K,L;R) = \tilde{C}_k(K;R)/\tilde{C}_k(L;R)$.
Since $\tilde{\partial}_k:\tilde{C}_k(L;R) \to \tilde{C}_{k-1}(L;R)$,
we may define the operator
$\overline{\tilde{\partial}}_k:\tilde{C}_k(K,L;R) \to \tilde{C}_{k-1}(K,L;R)$
by $\overline{\tilde{\partial}}_k(\relC[K,L]{c}) =
\relC[K,L]{\tilde{\partial}_k(c)}$ for all $c \in \tilde{C}_k(K;R)$, where
$\relC[K,L]{\cdot}$ represents an equivalence class in
$\tilde{C}_k(K,L;R)$.   We also define
$\displaystyle \tilde{H}_k(K,L;R) = \KE\big(\overline{\tilde{\partial}}_k\big) /
\IMG\big(\overline{\tilde{\partial}}_{k-1}\big)$.

\begin{rmk}
We will use the same notation $[\cdot]_K$ to denote
equivalence classes in $H_k(K;R)$, $\tilde{H}_k(K;R)$ and their reduced
form.  Similarly, we will use the same notation $[\cdot]_{K,L}$ to denote
equivalence classes in $H_k(K,L;R)$, $\tilde{H}_k(K,L;R)$ and there reduced
form.  This will simplify the notation.  The context is generally
sufficient to determine in which space the equivalence class belongs to.
\end{rmk}

\begin{defn}
Let $K_1$ and $K_2$ be two simplicial complexes, $R$ be an integral
domain, and $f$ be a simplicial map between $K_1$ and $K_2$.  The
function $\tilde{C}_k(f): \tilde{C}_k(K_1;R) \to \tilde{C}_k(K_2;R)$
is the linear function defined by 
$\tilde{C}_k(f)\big((\VEC{y}_0,\VEC{y}_1,\ldots,\VEC{y}_k)\big)
= (f(\VEC{y}_0),f(\VEC{y}_1),\ldots,f(\VEC{y}_k)\big)$
for all ordered $k$-simplices
$(\VEC{y}_0,\VEC{y}_1,\ldots,\VEC{y}_k)$ and
extended linearly to all of $\tilde{C}_k(K_1;R)$.
\end{defn}

Since $\displaystyle \tilde{C}_k(f)$ commutes with
$\displaystyle \tilde{\partial}_k$, we get the following proposition.

\begin{prop}
In the context of the previous definition, the map
$\displaystyle \tilde{H}_k(f): \tilde{H}_k(K_1;R) \to \tilde{H}_k(K_2,R)$
defined by
$\displaystyle \tilde{H}_k(f)([c]_{K_1}) = [ \tilde{C}_k(f)(c)]_{K_2}$ for all
$\displaystyle [c]_{K_1} \in \tilde{H}_k(K_1;R)$ is an homomorphism.
\end{prop}

The following result is easy to proof.

\begin{prop}
Suppose that $L_i$ is a subcomplex of a simplicial complex
$K_i$ for $i =1,2$, and that $R$ is an integral domain.
If $f:(K_1,L_1) \to (K_2,L_2)$, then the map
$\displaystyle \tilde{H}_k(f): \tilde{H}_k(K_1,L_1;R) \to
\tilde{H}_k(K_2,L_2,R)$ defined by
$\displaystyle \tilde{H}_k(f)(\relC[K_1,L_1]{c}) =
\relC[K_2,L_2]{\tilde{C}_k(f)(c)}$ for all
$\displaystyle \relC[K_1,L_1]{c} \in \tilde{H}_k(K_1,L_1;R)$ is an
homomorphism.
\end{prop}

The theory of reduced and relative simplicial homology presented
in Section~\ref{ssectSimplRDh} can be replicated to obtain the same
theoretical results for reduced and relative ordered simplicial
homology.

The free abelian group $\tilde{C}_k(K;R)$ is a useful tool to proof the
relation between simplicial and singular homology but it is not useful
for computations because it is much bigger than $C_k(K;R)$.

We first construct an augmentation-preserving chain map
$\F = \{ f_k\}_{k\in \NN}$ between the chain complex
$\tilde{\C} = \{ \tilde{C}_k(K;R), \tilde{\partial}_k \}_{k\in \NN}$ and the
chain complex $\SS = \{ S_k([K];R), \partial_k \}_{k\in \NN}$.
Recall that $\displaystyle [K] = \bigcup_{(s) \in K} (s)$.
Given an ordered $k$-simplex
$(s) = (\VEC{y}_0,\VEC{y}_1, \ldots,\VEC{y}_k)$,
we set $F\big((\VEC{y}_0,\VEC{y}_1, \ldots,\VEC{y}_k)\big)
= \delta_{(s)}$, where $\delta_{(s)}:\Delta_k \to [K]$
is the affine map defined by the conditions 
$\delta_{(s)}(\VEC{e}_i) = \VEC{y}_i$ for
$0 \leq i \leq k$.  We extend $F$ to $\tilde{C}_k(K;R)$ by linearity;
namely,
\[
F\Big(\sum_{\substack{(s) \text{ an ordered }\\
k\text{-simplex}}} a_{(s)} (s)\Big) = \sum_{\substack{(s) \text{ an ordered }\\
k\text{-simplex}}} a_{(s)} F\big((s)\big) \ ,
\]
where $a_{(s)} \in R$ and the sum is finite.  The chain map
$\F = \{ f_k\}_{k\in \ZZ}$ is defined by $f_k = F$ for all $k$.
We also have that $\F = \{ f_k\}_{k\in \NN}$ is a chain map
between the chain complex
$\{ \tilde{C}_k(L;R), \tilde{\partial}_k \}_{k\in \NN}$ and the
chain complex $\{ S_k([L];R), \partial_k \}_{k\in \NN}$.
Since the chain map $\F$ is augmentation preserving, $\F$ is also a
chain map between the augmented chain complex
$\displaystyle \tilde{\C}^\sharp
= \{ \tilde{C}_k^\sharp (K;R), \tilde{\partial}_k^\sharp \}_{k\in \NN}$
and the augmented chain complex
$\displaystyle \SS^\sharp
= \{ S_k^\sharp([K];R), \partial_k^\sharp \}_{k\in \NN}$.

Using $F$, we may define the following homomorphisms:
{
%\renewcommand{\labelitemi}{$\circ$}
\begin{itemize}
\item
$\tilde{H}_k(F) : \tilde{H}_k(K;R) \to H_k([K];R)$ by
$\tilde{H}_k(F)([c]_K) = [F(c)]_{[K]}$ for all $[c]_K \in
\tilde{H}_k(K;R)$,
\item $\tilde{H}_k(F) : \tilde{H}_k(K,L;R) \to H_k([K],[L];R)$ by
$\tilde{H}_k(F)([c]_{K,L}) = [F(c)]_{[K],[L]}$ for all
$[c]_{K,L} \in \tilde{H}_k(K,L;R)$, and
\item $\displaystyle \tilde{H}_k^\sharp(F) : \tilde{H}_k^\sharp(K;R) \to
H_k^\sharp([K];R)$ by
$\displaystyle \tilde{H}_k^\sharp(F)([c]_K) = [F(c)]_{[K]}$ for all
$\displaystyle [c]_K \in \tilde{H}_k^\sharp(K;R)$.
\end{itemize}
}

\begin{prop} \label{propHkEHkv1}
Let $L$ be a subcomplex of a simplicial complex $K$ and $R$ be an
integral domain. Then
\begin{enumerate}
\item $\tilde{H}_k(F): \tilde{H}_k(K;R) \to H_k([K];R)$ is an isomorphism,
\item $\tilde{H}_k^\sharp(F) : \tilde{H}_k^\sharp(K;R) \to
H_k^\sharp([K];R)$ is an isomorphism, and
\item $\tilde{H}_k(F) : \tilde{H}_k(K,L;R) \to H_k([K],[L];R)$ is an
isomorphism.
\end{enumerate}
\end{prop}

\begin{proof}
It follows from Proposition~\ref{propCsCtfauHsH} that (1) is
equivalent to (2).  So, we only need to prove (2) and (3).

\stage{I} We first assume that $K$ is a finite set.  The proof is by
induction on the number of elements in $K$.

\stage{I.i} If $K$ has only one element, then $K = \{ (\VEC{x}_i) \}$ for some
$\VEC{x}_i\in \RR^n$.  We have that $\tilde{C}_k(K;R)$ is generated by the
ordered $k$-simplex $(s) = (\VEC{x}_i,\VEC{x}_i, \ldots, \VEC{x}_i)$ and
$\delta_{(s)}:\Delta_k \to [K] = \{\VEC{x}_i\}$ is the affine map
defined by $\delta_{(s)}(\VEC{x}) = \VEC{x}_i$ for all
$\VEC{x} \in \Delta_k$.  Therefore $F:\tilde{C}_k(K;R) \to S_k([K];R)$ is
given by $F(r (s)) = r \delta_{s}$ for all $r \in R$.  Hence $F$ is
an isomorphism.  It follows that (2) is true.  The statement (3) is
obviously true because the only non-trivial relative homology modules
are $\tilde{H}_k(K,\emptyset;R) = \tilde{H}_k(K;R)$ and
$H_k([K],\emptyset;R) = H_k([K];R)$.

\stage{I.ii} We assume that (2) and (3) are true if $K$ has less
than $N$ elements and prove that it is also true if $K$ has $N$ elements.

It suffices to prove that
$\displaystyle \tilde{H}_k^\sharp(F): \tilde{H}_k^\sharp(K;R) \to
H_k^\sharp([K];R)$ is an isomorphism.
Suppose that this is true.  We get the following commuting diagram from
Proposition~\ref{propConnectingH} \footnote{There are versions of
Proposition~\ref{propConnectingH} for the reduced homology and the
ordered simplicial homology.}
\[
\xymatrix@C+2em{
\ar[r]^-{C_{k+1}} & \tilde{H}_k(L;R) \ar[r]^{\tilde{H}_k(\iota)}
\ar[d]^{\tilde{H}_k(F)} & \tilde{H}_k(K;R)
\ar[r]^{\tilde{H}_k(\Id_X)}  \ar[d]^{\tilde{H}_k(F)}
& \tilde{H}_k(K,L;R) \ar[d]^{\tilde{H}_k(F)} \\
\ar[r]^-{C_{k+1}} & H_k([L];R) \ar[r]^{H_k(\iota)}
& H_k([K];R) \ar[r]^{H_k(\Id_X)}  & H_k([K],[L];R) \\
\ar[r]^-{C_k} & \tilde{H}_{k-1}(L;R)
\ar[r]^{\tilde{H}_{k-1}(\iota)} \ar[d]^{\tilde{H}_{k-1}(F)}
& \tilde{H}_{k-1}(K;R) \ar[r]^-{\tilde{H}_{k-1}(\Id_X)}
\ar[d]^{\tilde{H}_{k-1}(F)} & \\
\ar[r]^-{C_k} & H_{k-1}([L];R) \ar[r]^{H_{k-1}(\iota)}
& H_{k-1}([K];R) \ar[r]^-{H_{k-1}(\Id_X)} &
}
\]
where each row is an exact sequence.
Since $\displaystyle \tilde{H}_k^\sharp(F): \tilde{H}_k^\sharp(L;R) \to
H_k^\sharp([L];R)$ is an isomorphism because $L$ has less than $N$
elements, and
$\displaystyle \tilde{H}_k^\sharp(F): \tilde{H}_k^\sharp(K;R) \to
H_k^\sharp([K];R)$ is an isomorphism by assumption,
we get from Proposition~\ref{propCsCtfauHsH} that
$\tilde{H}_k(F): \tilde{H}_k(L;R) \to H_k([L];R)$ and
$\tilde{H}_k(F): \tilde{H}_k(K;R) \to H_k([K];R)$
are isomorphism.  Hence, we get from the Five Lemma that
$\tilde{H}_k(F) : \tilde{H}_k(K,L;R) \to H_k([K],[L];R)$ 
is an isomorphism.

We now prove that
$\displaystyle \tilde{H}_k^\sharp(F): \tilde{H}_k^\sharp(K;R) \to
H_k^\sharp([K];R)$ is an isomorphism.
Suppose that $(s) \in K$ is a simplex which is not
the face of any other simplex but itself.  Then
$K_1 = K \setminus \{s\}$ is simplicial subcomplex of $K$ and $K_1$
has $N-1$ elements.  Let $K_2$ be the simplicial complex which is the
collection of $(s)$ and all its faces, and $K_3 = K_2 \setminus \{(s)\}$
(Figure~\ref{HkEHkfig}).

\pdfF{alg_top/hkehk}{Isomorphism between singular and simplicial homology}
{Figure associated to the proof of Proposition~\ref{propHkEHkv1}.
The set $[K_1]$ is the region in grey including the continuous black lines.
$[K_2]$ is the region in blue including the dashed blue lines forming
the boundary.  $[K_3]$ is the union of the dashed blue lines.  $V_1$
is the region in pale green excluding the dashed green lines forming
the boundary.  Finally, $V$ is the region in darker green excluding
the dashed green lines forming the boundary.  The arrows in black
describe the retraction $r$ of $[K_2] \setminus V$ onto $[K_3]$.}
{HkEHkfig}

We get the following commutative diagram.
\[
\xymatrix@C+2em{
\tilde{H}_k^\sharp(K;R) \ar[r]^{\tilde{H}_k^\sharp(F)}
\ar[d]_{\tilde{H}_k(\Id)} & H_k^\sharp([K];R) \ar[d]^{H_k(\Id)} \\
\tilde{H}_k(K,K_2;R) \ar[r]^-{\tilde{H}_k(F)} & H_k([K],[K_2];R) \\
\tilde{H}_k(K_1;K_3;R) \ar[r]^-{\tilde{H}_k(F)} \ar[u]^{\tilde{H}_k(\iota)}
& H_k([K_1],[K_3];R) \ar[u]_{H_k(\iota)}
}
\]
where $\iota$ denotes an inclusion map.  We have by
induction that
$\tilde{H}_k(F): \tilde{H}_k(K_1;K_3;R) \to H_k([K_1],[K_3];R)$
is an isomorphism.  To prove that
$\tilde{H}_k^\sharp(F): \tilde{H}_k^\sharp(K;R) \to H_k^\sharp([K];R)$
is an isomorphism, we prove
that all the maps $\tilde{H}_k(\iota)$, $H_k(\iota)$,
$\tilde{H}_k(\Id)$ and $H_k(\Id)$ are isomorphism.

Since $[K_2] = [s]$ is contractible, we get from
Proposition~\ref{propContrXeXA} that
$\displaystyle H_k(\Id):H_k^\sharp([K];R) \to H_k^\sharp([K],[K_2];R)$
is an isomorphism for $k \geq 0$.  Since 
$\displaystyle H_k^\sharp([K],[K_2];R) = H_k([K],[K_2];R)$ for $k>0$,
we get that
$\displaystyle H_k(\Id):H_k^\sharp([K];R) \to H_k([K],[K_2];R)$
is an isomorphism for $k>0$.  For $k=0$, we assume that
$[K]$ is the disjoint union of path-connected components 
$X_j$ for $j \in J$.  There exists $j_0 \in J$ such that
$(s) \subset X_{j_0}$.  We have from Propositions~\ref{propXAopXjAj} 
and \ref{propH0XA0} that
\begin{align*}
H_0([K],[K_2];R)
&= \Big( \bigoplus_{j \in J \setminus \{j_0\}} H_0(X_j;R) \Big) \oplus 
H_0(X_{j_0},[K_2];R) \\
&= \Big( \bigoplus_{j \in J \setminus \{j_0\}} H_0(X_j;R) \Big) \oplus 0 \ .
\end{align*}
We get from Propositions~\ref{propHkeHkJ} and \ref{propHkEquHskpR}
that
\[
\Big( \bigoplus_{j \in J \setminus \{j_0\}} H_0(X_j;R) \Big) \oplus 
H_0(X_{j_0};R) = H_0([K];R) \cong H_0^\sharp([K];R) \oplus R \ .
\]
Since $H_0(X_j;R) \cong R$ for all $j$ according to
Corollary~\ref{corContrHk}, we get that
$\displaystyle H_0([K],[K_2];R)
= \bigoplus_{j \in J \setminus \{j_0\}} H_0(X_j;R)
\cong H_0^\sharp([K];R)$.  Since the previous isomorphism is deduced
from the identity map, we again have that
$\displaystyle H_0(\Id):H_0^\sharp([K];R) \to H_0([K],[K_2];R)$
is an isomorphism.  The same reasoning shows that
$\displaystyle \tilde{H}_k(\Id):\tilde{H}_k^\sharp(K;R) \to
\tilde{H}_k(K,K_2;R)$ is an isomorphism for $k \geq 0$
\footnote{The courageous reader is invited to repeat for
relative and reduced simplicial homology what we have done for relative
and reduced singular homology.}.

Consider the inclusion map
$\iota : \tilde{C}_k(K_1;R)/\tilde{C}_k(K_3;R) \to
\tilde{C}_k(K;R)/\tilde{C}_k(K_2;R)$.  The map $\iota$ is an isomorphism
because $K_1 \setminus K_3 = K \setminus K_2$.  More
precisely, if\\
$\relC[K,K_2]{c} \in \tilde{C}_k(K;R)/\tilde{C}_k(K_2;R)$, then
$c - b \in \tilde{C}_k(K_2;R)$ for some $b \in \tilde{C}_k(K_1;R)$ because
$(s) \in K_2$.  Hence $\iota(\relC[K_1,K_3]{b}) = \relC[K,K_2]{c}$ and $\iota$
is onto.  If $\relC[K,K_2]{c_1} = \iota(\relC[K_1,K_3]{c_1})
= \iota(\relC[K_1,K_3]{c_2})= \relC[K,K_2]{c_2}$ with
$c_1,c_2 \in \tilde{C}_k(K_1;R)$, then
$c_1 - c_2 = b \in \tilde{C}_k(K_2;R) \cap \tilde{C}_k(K_1;R)$.
Thus $c_1 - c_2 = b \in \tilde{C}_k(K_3;R)$ and therefore
$\relC[K_1,K_3]{c_1}= \relC[K_1,K_3]{c_2}$.
This implies that
$\tilde{H}_k(\iota) : \tilde{H}_k(K_1,K_3;R) \to \tilde{H}_k(K,K_2;R)$
is also an isomorphism.

The prove that $H_k(\iota) : H_k([K_1],[K_3];R) \to H_k([K],[K_2];R)$
is an isomorphism is not as simple as the proof that
$\tilde{H}_k(\iota)$ is an isomorphism given above.  We plan to use excision to
prove that $H_k(\iota) : H_k([K_1],[K_3];R) \to H_k([K],[K_2];R)$
is an isomorphism.  Unfortunately, the excision theorem,
Theorem~\ref{thmExcis}, cannot be used directly because we have the
situation where the set that we want to excise, namely $(s)$, is such
that $\overline{(s)} = [s]$ is not a subset of
$\displaystyle [K_2]^\circ = (s)$,
where closure and interior of sets are relative to the induce topology
on $[K]$ from $\displaystyle \RR^n$.
We therefore need to use Proposition~\ref{propVUAX}.

Suppose that $(s)$ is a $k$-simplex and that $(s_0)$, $(s_1)$, \ldots,
$(s_k)$ are the faces of $(s)$ which are $(k-1)$-simplices.
Let $V_1$ be the $k$-simplex
$(\VEC{b}_{(s_0)},\VEC{b}_{(s_1)}, \ldots, \VEC{b}_{(s_k)})$ where
$\VEC{b}_{(s_j)}$ is the barycentre of $(s_j)$ for $0 \leq j \leq k$.
Let $\VEC{p}$ be the barycentre of $V_1$. 
Referring to the statement of Proposition~\ref{propVUAX}, the set
$V \subset (s)$ that we consider is given by
\[
V = \VEC{p} + \epsilon (V_1 - \VEC{p}) =
\{ \VEC{p} + \epsilon (\VEC{x} - \VEC{p}) : \VEC{x} \in V_1 \}
\]
for $0 < \epsilon <1$ fixed (Figure~\ref{HkEHkfig}).

We have that $\overline{V} \subset (s)^\circ = (s)$.
There is a retraction $r$ of $[K_2] \setminus V$ onto $[K_3]$
(Figure~\ref{HkEHkfig}).  By defining $r$ to be the identity on
$[K_1]\setminus [K_2]$, we get a deformation retraction of
$([K] \setminus V, [K_2] \setminus V)$ onto
$([K_1],[K_3]) = ([K]\setminus (s),[K_2] \setminus (s))$.  We may
therefore use Proposition~\ref{propVUAX} to conclude that
$H_k(\iota):H_k([K_1],[K_3];R) \to H_k([K] \setminus V, [K_2] \setminus V;R)$
is an isomorphism.

Moreover, since $\overline{V} \subset (s)^\circ = (s)$, we may use the
excision theorem, Theorem~\ref{thmExcis}, to conclude that
$H_k(\iota):H_k([K] \setminus V, [K_2] \setminus V;R) \to H_k([K], [K_2];R)$.
is an isomorphism.  It follows from the previous paragraph that
$H_k(\iota) : H_k([K_1],[K_3];R) \to H_k([K],[K_2];R)$
is an isomorphism.

\stage{II} The fact that we have proved the proposition when $K$ has
any finite number of elements does not automatically implies that the
proposition is true when $K$ has infinitely many elements.  We do not
have a concept of limit for the spaces $C_k(K;R)$, $S_k(K;R)$, and so on,
as the size of $K$ increases.  We now assume that $K$ has countably
infinitely many elements.

It still follows from Proposition~\ref{propCsCtfauHsH} that (1) is
equivalent to (2).  Moreover, if we prove that
$\displaystyle \tilde{H}_k^\sharp(F): \tilde{H}_k^\sharp(K;R) \to
H_k^\sharp([K];R)$ is an isomorphism for all $k$, then we can use long
exact sequences and the Five Lemma exactly as we have done in (I.ii) to
show that this implies that (3) is true for all $k$.

Thus, we only have to prove that
$\displaystyle \tilde{H}_k^\sharp(F): \tilde{H}_k^\sharp(K;R) \to
H_k^\sharp([K];R)$ where $K$ has infinitely many elements.

\stage{II.i}  Suppose that $c \in S_k([K];R)$.  We may express $c$ as a
finite sum $\displaystyle c = \sum_{j\in J} a_j \sigma_j$, where
$a_j \in R$ and $\sigma_j$ is a singular $k$-simplex in $[K]$ for all $j$.
We prove that there exists a simplicial
subcomplex $L_c$ of $K$ such that $L_c$ has a finite number of elements and
$\displaystyle Q = \bigcup_{j\in J} \sigma_j(\Delta_k) \subset [L_c]$.
Since the $\sigma_j$ are continuous functions and $\Delta_k$ is
compact, we have that $\sigma_j(\Delta_k)$ is a compact set.
Thus $Q$ is a compact set because it is the finite union of compact
sets.  Suppose that there is no finite number of simplices
in $K$ that cover $Q$.  Then there exists an
infinite collection $\displaystyle \{ (s_i) \}_{i\in \NN}$ of distinct
simplices in $K$ such that $(s_i) \cap Q \neq \emptyset$
for all $i \in \NN$ \footnote{These simplices are obtained inductively using
the fact that $\displaystyle Q \subset [K] = \bigcup_{(s) \in K} (s)$.}.
Choose $\VEC{z}_i \in (s_i) \cap Q$ for each $i \in \NN$.
Let $\displaystyle U_m = Q \setminus \bigcup_{i\neq m} \{ \VEC{z}_i \}$
for all $m \in \NN$.  The sets $U_m$ are open sets (see Remark~\ref{rmkUmOpen}
below) and $\displaystyle Q = \bigcup_{m \in \NN} U_m$.  Thus
$\{ U_m\}_{m\in \NN}$ is an open cover of the compact set
$Q$.  Therefore, there exists a finite subcover
$\{ U_{m_t}\}_{0\leq t \leq T}$ of $Q$.  But
$\displaystyle Q \varsupsetneqq \bigcup_{0\leq t\leq T} U_{m_t}$ because
$\VEC{z}_i \not\in \bigcup_{0\leq t \leq T} U_{m_t}$ if $i \neq m_t$ for
$0 \leq t\leq T$.  This is a contradiction.   Hence, there are a
finite number of simplices in $K$ whose union cover $Q$.
If we add all the faces of these simplices, we get a finite simplicial
subcomplex $L_c$ of $K$ with $Q \subset [L_c]$.

\stage{II.ii}  We first prove that
$\displaystyle \tilde{H}_k^\sharp(F): \tilde{H}_k^\sharp(K;R) \to
H_k^\sharp([K];R)$ is onto.  Suppose that
$\displaystyle [c]_K \in H_k^\sharp([K];R)$ and that
$\displaystyle c = \sum_{j\in J} a_j \sigma_j$, where
$a_j \in R$ and $\sigma_j$ is a singular $k$-simplex in $[K]$ for all
$j \in J$.
Using (II.i), we get a finite simplicial subcomplex $L_c$ of $K$ such that
$\displaystyle c \in S_k^\sharp([L_c];R)$ and thus
$\displaystyle [c]_{L_c} \in H_k^\sharp([L_c];R)$.
we consider the commutative diagram
\[
\xymatrix@C+2em{
\tilde{H}_k^\sharp(L_c;R) \ar[r]^-{\tilde{H}_k^\sharp(F)}
\ar[d]_{\tilde{H}_k(\iota)}
& H_k^\sharp([L_c];R) \ar[d]^{H_k(\iota)} \\
\tilde{H}_k^\sharp(K;R) \ar[r]^-{\tilde{H}_k^\sharp(F)}
& H_k^\sharp([K];R)
}
\]
We have from (I) that
$\displaystyle \tilde{H}_k^\sharp(F) : \tilde{H}_k^\sharp(L_c;R) \to
H_k^\sharp([L_c];R)$ is an isomorphism.  Thus, there
exists $\displaystyle [b]_{L_c} \in \tilde{H}_k^\sharp(L_c;R)$ such that
$\displaystyle \tilde{H}_k^\sharp(F)([b]_{L_c}) = [c]_{[L_c]}$.  we
therefore get that\\
$\displaystyle \tilde{H}_k^\sharp(F)\big(\tilde{H}_k(\iota)
([b]_{L_c})\big) = H_k(\iota)\big( \tilde{H}_k^\sharp(F)([b]_{L_c}) \big)
= [c]_{[K]}$.  Hence $[c]_{[K]}$ is in the image of
$\displaystyle \tilde{H}_k^\sharp(F): \tilde{H}_k^\sharp(K;R) \to
H_k^\sharp([K];R)$.

To prove that $\displaystyle \tilde{H}_k^\sharp(F): \tilde{H}_k^\sharp(K;R) \to
H_k^\sharp([K];R)$ is one-to-one, suppose that
$\displaystyle \tilde{H}_k^\sharp(F)([c]_K) = [F(c)]_{[K]} = [0]_{[K]}$ for some
$\displaystyle [c]_K \in \tilde{H}_k^\sharp(K;R)$.  This means that
$\displaystyle F(c) = \partial^\sharp(b)$ for some $b \in S_{k+1}([K];R)$.
Using (II.i), we get a finite simplicial subcomplex $L_c$ of $K$ such that
$\displaystyle F(c) \in S_k^\sharp([L_c];R)$ and a finite simplicial
subcomplex $L_b$ of $K$ such that
$\displaystyle b \in S_{k+1}^\sharp([L_b];R)$.
Let $L = L_c \cup L_b$.  We have that $L$ is a finite simplicial
subcomplex of $K$ with $\displaystyle F(c) \in S_k^\sharp([L];R)$ and
$\displaystyle b \in S_{k+1}([L];R)$.
We also have that $\displaystyle c \in \tilde{C}_k^\sharp(L;R)$
because $c$ is a finite linear combinations of ordered $k$-simplices
which are included in $L$ by construction \footnote{It follows from
the definition of $F$ that If $(s)$ is one of the ordered
$k$-simplices involved in the linear combination for $c$, then $(s)$
must comes from one of the initial simplices that cover
$Q$ in (II.i).  Recall that if two open simplices
intersect, then they must be equal.}.
We consider the commutative diagram
\[
\xymatrix@C+2em{
\tilde{H}_k^\sharp(L;R) \ar[r]^-{\tilde{H}_k^\sharp(F)}
\ar[d]_{\tilde{H}_k(\iota)}
& H_k^\sharp([L];R) \ar[d]^{H_k(\iota)} \\
\tilde{H}_k^\sharp(K;R) \ar[r]^-{\tilde{H}_k^\sharp(F)}
& H_k^\sharp([K];R)
}
\]
Since $\displaystyle F(c) = \partial^\sharp(b) \in S_k^\sharp([L];R)$
with $\displaystyle b \in S_{k+1}^\sharp([L];R)$, we have that
$\displaystyle [F(c)]_{[L]}= [0]_{[L]} \in \tilde{H}_k^\sharp([L];R)$.
Since $\displaystyle \tilde{H}_k^\sharp(F) : \tilde{H}_k^\sharp(L;R) \to
H_k^\sharp([L];R)$ is an isomorphism according to (I), we have that
$\displaystyle [c]_L = [0]_L \in \tilde{H}_k^\sharp(L;R)$.  Thus
$\displaystyle [c]_K = \tilde{H}(\iota)([c]_L) = [0]_K \in 
\tilde{H}_k^\sharp(K;R)$.
\end{proof}

HERE

\begin{rmk}
We prove that the sets $U_m$ in part (II.i) of          \label{rmkUmOpen}
the previous proof are open sets.  By adding the missing faces if
needed to $\displaystyle \{ (s_i) \}_{i\in \NN}$, we get a
a simplicial subcomplex $L$ of $K$.  Consider
$\VEC{z} \in U_m$.  If $\VEC{z} \in U_m \setminus [L]$, 
then $V = Q \setminus [L]$ can be used as an open neighbourhood of
$\VEC{z}$ such that $V \subset U_m$.  If
$\displaystyle \VEC{z} \in [L] = \bigcup_{(s) \in L} (s)$, then
there exists a unique $(s) \in L$ such that $\VEC{z} \in (s)$.
The finite union $\displaystyle W =
\bigcup_{\substack{(t) \in K\\ [t]\cap [s] \neq \emptyset}} (t)$ 
is an open neighbourhood of $\VEC{z}$ in the induce topology on $[K]$.
Moreover, $W$ contains only a finite number of $\VEC{z}_i$ with
$i\neq m$ because the open simplices are distinct and we have only one
$\VEC{z}_i$ per $(s_i)$; We are referring to
the $\VEC{z}_i$ with $i \neq m$ that may be in one of the $(t)$ listed
in the union.  We do not want to reject $\VEC{z}_m$ if it is in $W$;
in particular if $\VEC{z} = \VEC{z}_m$.
If $\VEC{z}_{m_s}$ for $m_s \neq m$ and $0\leq s \leq S$ are the
$\VEC{z}_i \in W$, then $\displaystyle Q \cap \big( W \setminus
\{ \VEC{z}_{m_s} : 0\leq s \leq S\}\big) \subset U_m$ is an open
neighbourhood of $\VEC{z}$.  
\end{rmk}

\subsubsection{Definition of the Isomorphism}

We really want a version of the previous proposition for oriented
simplicial homology.  Instead of ordered simplices, we want to work
with oriented simplices.  Instead of the map $F$ defined previously,
we would like to use the following map.  Given an oriented $k$-simplex
$\os{s}{}{}{}{} = \os{\VEC{x}_{i_0}}{}{\VEC{x}_{i_1}}{}{\VEC{x}_{i_k}}$,
we set $G(\os{s}{}{}{}{}) = \delta_{\osscript{s}{}{}{}{}}$ where as before
$\delta_{\osscript{s}{}{}{}{}}:\Delta_k \to [K]$
is the affine map defined by
$\delta_{\osscript{s}{}{}{}{}}(\VEC{e}_j) = \VEC{x}_{i_j}$ for
$0 \leq j \leq k$.  We extend $G$ to $C_k(K;R)$ by linearity;
namely,
$\displaystyle G\Big(\sum_{\substack{(s) \in K\\\dim(s)=k}} a_{(s)}
\os{s}{}{}{}{}\Big) = \sum_{\substack{(s) \in K\\\dim(s)=k}} a_{(s)}
G(\os{s}{}{}{}{})$ where $a_{(s)} \in R$ and the sum is finite.
The rest of this subsection is devoted to justify how we may
``replace'' $F$ by $G$.

Let $\displaystyle \phi:C_k(K;R) \to \tilde{C}_k(K;R)$
be the function defined by \\
$\phi(\os{\VEC{x}_{j_0}}{}{\VEC{x}_{j_1}}{}{\VEC{x}_{j_k}})
= (\VEC{x}_{j_0},\VEC{x}_{j_1}, \ldots, \VEC{x}_{j_k})$ for all 
oriented $k$-simplices with $j_0 < j_1 < \ldots < j_k$ and
extended to $\displaystyle C_k(K;R)$ by linearly,
and $\displaystyle \psi:\tilde{C}_k(K;R) \to C_k(K;R)$
be the function defined by
\[
\psi\big((\VEC{y}_0,\VEC{y}_1,\ldots,\VEC{y}_k)\big)
= \begin{cases}
\os{\VEC{y}_0}{}{\VEC{y}_1}{}{\VEC{y}_k}
& \quad \text{if} \ (\VEC{y}_0,\VEC{y}_1,\ldots,\VEC{y}_k) \in K \\
0 & \quad \text{otherwise}
\end{cases}
\]
for all ordered $k$-simplices and extended to
$\displaystyle \tilde{C}_k(K;R)$ by linearly.
We note that $\psi((s)) = 0$ when $(s)$ has at least two equal vertices.
It is clear that $\tilde{\partial}_k \circ \phi = \phi \circ \partial_k$.
It is also true that
$\psi \circ \tilde{\partial}_k = \partial_k \circ \psi$ but the
proof is not completely obvious.  If $(s) \in \tilde{C}_k(K;R)$ does
not have any equal vertices, then the result is obvious.  If
$(s) = \big(\VEC{y}_0,\VEC{y}_1,\ldots,\VEC{y}_k\big)$ has two equal
vertices, say $\VEC{y}_m = \VEC{y}_{m+1}$ for some $0 \leq m < k$,
then $\partial_k \big(\psi\big((s)\big)\big) = \partial_K(0) = 0$ and
\begin{align*}
\psi\big(\tilde{\partial}_k(s)\big)
&= \sum_{i=0}^k (-1)^i \psi\big(
(\VEC{y}_0,\VEC{y}_1,\ldots,\widehat{\VEC{y}_i}, \ldots,
\VEC{y}_k)\big) \\
&= (-1)^m \psi\big(
(\VEC{y}_0,\VEC{y}_1,\ldots,\widehat{\VEC{y}_m}, \ldots, \VEC{y}_k)\big)
+ (-1)^{m+1} \psi\big(
(\VEC{y}_0,\VEC{y}_1,\ldots,\widehat{\VEC{y}_{m+1}}, \ldots, \VEC{y}_k)\big)
= 0
\end{align*}
because $(\VEC{y}_0,\VEC{y}_1,\ldots,\widehat{\VEC{y}_m}, \ldots, \VEC{y}_k)
= (\VEC{y}_0,\VEC{y}_1,\ldots,\widehat{\VEC{y}_{m+1}}, \ldots, \VEC{y}_k)$.
If $(s) = \big(\VEC{y}_0,\VEC{y}_1,\ldots,\VEC{y}_k\big)$ has more
than two equal vertices, then
$\partial_k \big(\psi\big((s)\big)\big) = \partial_K(0) = 0$ and
\[
\psi\big(\tilde{\partial}_k(s)\big)
= \sum_{i=0}^k (-1)^i \psi\big(
(\VEC{y}_0,\VEC{y}_1,\ldots,\widehat{\VEC{y}_i}, \ldots,
\VEC{y}_k)\big)
= 0
\]
because each $(\VEC{y}_0,\VEC{y}_1,\ldots,\widehat{\VEC{y}_i}, \ldots,
\VEC{y}_k)$ has at last two equal vertices.

We have the following commutative diagram
\[
\xymatrix{
\ar[r]^(0.2){\partial_{k+2}}
& C_{k+1}(K;R)  \ar[r]^-{\partial_{k+1}} \ar@/^/[d]^-{\phi}
& C_k(K;R) \ar[r]^-{\partial_k} \ar@/^/[d]^-{\phi}
& C_{k-1}(K;R) \ar[r]^(0.7){\partial_{k-1}} \ar@/^/[d]^-{\phi} & \\
\ar[r]_(0.2){\tilde{\partial}_{k+2}}
& \tilde{C}_{k+1}(K;R) \ar[r]_-{\tilde{\partial}_{k+1}}
\ar@/^/[u]^-{\psi}
& \tilde{C}_k(K;R) \ar[r]_-{\tilde{\partial}_k} \ar@/^/[u]^-{\psi}
& \tilde{C}_{k-1}(K;R) \ar[r]_(0.7){\tilde{\partial}_{k-1}}
\ar@/^/[u]^-{\psi} &
}
\]
Namely, $\{\phi_k\}_{k\in \ZZ}$ with $\phi_k = \phi$ for all $k$ is a
chain map from the chain complex
$\displaystyle \C = \{(C_k(K;R),\partial_k)\}_{k\in \ZZ}$
to the chain complex
$\displaystyle \tilde{\C}
= \{(\tilde{C}_k(K;R),\tilde{\partial}_k)\}_{k\in \ZZ}$ and
$\{\psi_k\}_{k\in \ZZ}$ with $\psi_k = \psi$ for all $k$ is a
chain map from $\displaystyle \tilde{\C}$ to $\displaystyle \C$.

Since
$\displaystyle \phi: \KE(\partial_k) \to \KE(\tilde{\partial}_k)$
and
$\displaystyle \phi: \IMG(\partial_{k-1}) \to
\IMG(\tilde{\partial}_{k-1})$, we can
define the homomorphism
$\displaystyle \tilde{H}_k(\phi):H_k(K;R) \to \tilde{H}_k(K;R)$
by
$\displaystyle \tilde{H}_k(\phi)([c]_K) = [\phi(c)]_K$ for all
$\displaystyle c \in C_k(K;R)$,
where as usual $[\cdot]_K$ represents equivalence classes in both
$\displaystyle H_k(K;R)$ and
$\displaystyle \tilde{H}_k(K;R)$.
Similarly, since
$\displaystyle \psi:\KE(\tilde{\partial}_k) \to \KE(\partial_k)$
and
$\displaystyle \psi:\IMG(\tilde{\partial}_{k-1}) \to
\IMG(\partial_{k-1})$, we may
define the homomorphism
$\displaystyle \tilde{H}_k(\psi):\tilde{H}_k(K;R) \to
H_k(K;R)$ by
$\displaystyle \tilde{H}_k(\psi)([c]_K) = [\psi(c)]_K$ for all
$\displaystyle c \in \tilde{C}_k(K;R)$.

We can repeat all the previous discussion for the reduced
homology (i.e.\ with $\partial_k$ replaced by
$\displaystyle \partial_k^\sharp$ and $\tilde{\partial}_k$
replaced by $\displaystyle \tilde{\partial}_k^\sharp$) and
define
$\displaystyle \tilde{H}_k^\sharp(\phi):H_k^\sharp(K;R) \to
\tilde{H}_k^\sharp(K;R)$ and
$\displaystyle \tilde{H}_k^\sharp(\psi):\tilde{H}_k^\sharp(K;R) \to
H_k^\sharp(K;R)$.  We assume that
$\displaystyle \tilde{C}_{-1}^\sharp(K;R) = R$ and
$\displaystyle C_{-1}^\sharp(K;R) = R$, and that
$\displaystyle \phi:C_{-1}^\sharp(K;R) \to \tilde{C}_{-1}^\sharp(K;R)$
and
$\displaystyle \psi:\tilde{C}_{-1}^\sharp(K;R) \to C_{-1}^\sharp(K;R)$
are the identity maps.

According to Proposition~\ref{propCsCtfauHsH}, if we prove that
$\displaystyle \tilde{H}_k^\sharp(\phi)$ is an isomorphism, then 
we will have that $\displaystyle \tilde{H}_k(\phi)$ is an isomorphism.
To prove that $\displaystyle \tilde{H}_k^\sharp(\phi)$ is an
isomorphism, we prove that $\displaystyle \tilde{H}_k^\sharp(\psi)$ is
the inverse of $\displaystyle \tilde{H}_k^\sharp(\phi)$.

It is clear that $\psi \circ \phi = \Id$ on
$\displaystyle C_k^\sharp(K;R)$.  Thus
$\displaystyle \tilde{H}_k^\sharp(\psi) \circ \tilde{H}_k^\sharp(\phi) = \Id$
on $\displaystyle H_k^\sharp(K;R)$.
To prove that
$\displaystyle \tilde{H}_k^\sharp(\phi) \circ \tilde{H}_k^\sharp(\psi) = \Id$
requires a little bit more work because $\phi \circ \psi$ is not the identity.
We need to prove that there exists a chain homotopy between
$\F = \{ f_k\}_{k\in \ZZ}$ with
$\displaystyle f_k = \Id:\tilde{C}_k^\sharp(K;R) \to \tilde{C}_k^\sharp(K;R)$
and
$\GG = \{ g_k\}_{k\in \ZZ}$ with
$\displaystyle g_k = \phi\circ \psi :\tilde{C}_k^\sharp(K;R) \to
\tilde{C}_k^\sharp(K;R)$.
Namely, we need to find $\DD = \{ D_k \}_{k\in \ZZ}$ with
$\displaystyle D_k : \tilde{C}_k^\sharp(K;R) \to \tilde{C}_{k+1}^\sharp(K;R)$
such that
$\displaystyle \tilde{\partial}_{k+1}^\sharp \circ D_k +
D_{k-1} \circ \tilde{\partial}_k^\sharp =
\phi\circ \psi - \Id$ on $\tilde{C}_k^\sharp(K;R)$.
After this is done, we will be able to use
Proposition~\ref{propCHiFeG} \footnote{To be precise, a version of
this proposition for ordered simplicial homology.}
to conclude that
$\displaystyle \tilde{H}_k^\sharp(\Id) = \tilde{H}_k^\sharp(\phi\circ \psi)$
for all $k$; namely,
$\displaystyle \tilde{H}_k^\sharp(\phi) \circ \tilde{H}_k^\sharp(\psi)
= \Id:\tilde{H}_k^\sharp(K;R) \to \tilde{H}_k^\sharp(K;R)$.

\begin{prop}  \label{propOOhomCM}
There exists $\DD = \{ D_k \}_{k\in \ZZ}$ with
$\displaystyle D_k : \tilde{C}_k^\sharp(K;R) \to \tilde{C}_{k+1}^\sharp(K;R)$
such that
$\displaystyle \tilde{\partial}_{k+1}^\sharp \circ D_k
+ D_{k-1} \circ \tilde{\partial}_k^\sharp =
\phi\circ \psi - \Id$ on $\displaystyle \tilde{C}_k^\sharp(K;R)$.
\end{prop}

\begin{proof}
\stage{i} Suppose that $(\VEC{p},[K])$ is an general position.  We
consider the cone $\VEC{p} \ast K$.  Recall that it is a simplicial
complex consisting of all the simplices of the form
$(\VEC{p},\VEC{x}_{i_0}, \VEC{x}_{i_1}, \ldots , \VEC{x}_{i_k})$ and
all their faces where
$(\VEC{x}_{i_0}, \VEC{x}_{i_1}, \ldots , \VEC{x}_{i_k}) \in K$.
Hence $K \subset \VEC{p} \ast K$.

We prove that the chain complex
$\displaystyle \{(\tilde{C}_k^\sharp(\VEC{p}\ast K;R),
\tilde{\partial}_k^\sharp)\}_{k\in \ZZ}$ is
acyclic.  Let
$\displaystyle Q_k: \tilde{C}_k^\sharp(\VEC{p}\ast K;R) \to
\tilde{C}_{k+1}^\sharp(\VEC{p}\ast K;R)$ be
the mapping define by 
$Q_k\big( (\VEC{y}_0, \VEC{y}_1, \ldots, \VEC{y}_k) \big)
=(\VEC{p}, \VEC{y}_0, \VEC{y}_1, \ldots, \VEC{y}_k)$
for all $(\VEC{y}_0, \VEC{y}_1, \ldots, \VEC{y}_k) \in \VEC{p} \ast K$, and 
extended by linearity to $\displaystyle \tilde{C}_k^\sharp(\VEC{p}\ast K;R)$.

Proceeding as we did in Section~\ref{subsectExcis}, we find that
$\displaystyle \tilde{\partial}_1^\sharp(Q_0(c))
= c - \tilde{\partial}_0^\sharp(c) (\VEC{p})$ for
all $\displaystyle c \in \tilde{C}_0^\sharp(\VEC{p} \ast K;R)$ and
$\displaystyle \tilde{\partial}_{k+1}^\sharp(Q_k(c))
= c - Q_{k-1}(\tilde{\partial}_k^\sharp(c))$
for all $\displaystyle c \in \tilde{C}_k^\sharp(\VEC{p} \ast K;R)$ if $k>0$.

If $\displaystyle c \in \tilde{Z}_0^\sharp(\VEC{p} \ast K;R)$, then
$\displaystyle \tilde{\partial}_0^\sharp(c) =0$.
Therefore $\displaystyle c = \tilde{\partial}_1^\sharp(Q_0(c))$.
Hence $\displaystyle c \in \tilde{B}_0^\sharp(\VEC{p} \ast K;R)$.  Thus
$[c]_{\VEC{p}\ast K} = [0]_{\VEC{p}\ast K}$.  We get that
$\displaystyle \tilde{H}_0^\sharp(\VEC{p} \ast K;R) = 0$.

If $\displaystyle c \in \tilde{Z}_k^\sharp(\VEC{p} \ast K;R)$ with $k>0$, then
$\displaystyle \tilde{\partial}_k^\sharp(c) = 0$.  Therefore
$\displaystyle c = \tilde{\partial}_{k+1}^\sharp(Q_k(c))$.
Hence $\displaystyle c \in \tilde{B}_k^\sharp(\VEC{p} \ast K;R)$.
Thus $[c]_{\VEC{p}\ast K} = [0]_{\VEC{p}\ast K}$.  We get that
$\displaystyle \tilde{H}_k^\sharp(\VEC{p} \ast K;R) = 0$.

\stage{ii}  Let $\eta = \phi\circ \psi$.
Given $\sigma = (\VEC{y}_0,\VEC{y}_1,\ldots,\VEC{y}_k) \in \VEC{p} \ast K$,
Let $L_\sigma$ be the smallest subcomplex of $\VEC{p}\ast K$ containing
the vertices $\VEC{y}_0$, $\VEC{y}_1$, \ldots, $\VEC{y}_k$,
$\eta(\VEC{y}_0)$, $\eta(\VEC{y}_1)$, \ldots,$\eta(\VEC{y}_{k-1}$ and
$\eta(\VEC{y}_k)$.
We have that $\displaystyle \tilde{\partial}_k^\sharp(\sigma) \in
\tilde{C}_k^\sharp(L_\sigma;R)$.
Moreover, if $\displaystyle \sigma \in \tilde{C}_k^\sharp(K;R)$,
then $L_\sigma \subset K$ because
$\displaystyle \eta: \tilde{C}_k^\sharp(K;R) \to \tilde{C}_k^\sharp(K;R)$.

We use a proof by induction to construct
$\displaystyle D_k : \tilde{C}_k^\sharp(\VEC{p} \ast K;R) \to
\tilde{C}_{k+1}^\sharp(\VEC{p} \ast K;R)$
such that
$\displaystyle \tilde{\partial}_{k+1}^\sharp \circ D_k
= D_{k-1} \circ \tilde{\partial}_k^\sharp =
\eta - \Id$ on $\displaystyle \tilde{C}_k^\sharp(\VEC{p} \ast K;R)$
and such that
$\displaystyle D_k \big(\tilde{C}_k^\sharp(Q;R)\big) \subset
\tilde{C}_{k+1}^\sharp(Q;R)$ for all simplicial subcomplexes of
$\VEC{p} \ast K$; in particular, for $Q = K$.

\stage{$\mathbf{k=-1}$}
We define $D_{-1} : \tilde{C}_{-1}^\sharp(\VEC{p} \ast K;R) = R \to
\tilde{C}_0^\sharp(\VEC{p} \ast K;R)$ by $D_{-1}(r) = 0$ for $r \in R$.

\stage{$\mathbf{k=0}$}
Consider
$\displaystyle \sigma = (\VEC{y}) \in \tilde{C}_1^\sharp(\VEC{p} \ast K;R)$.
Thus $\VEC{y}$ is a vertex of $\VEC{p} \ast K$.
Since
\[
\partial_0^\sharp \big( \eta(\sigma) - \Id(\sigma) \big)
= \eta(\partial_0^\sharp(\sigma)) - \Id(\partial_0^\sharp(\sigma))
= \eta(1) - \Id(1) = 1 -1 = 0 \ ,
\]
we get that $\displaystyle \eta(\sigma) - \Id(\sigma) \in
\tilde{Z}_0^\sharp(\VEC{p} \ast K;R)$.
Since $\displaystyle \tilde{H}_0^\sharp(\VEC{p} \ast K;R) = 0$
by (i), we have that $\displaystyle \eta(\sigma)
- \Id(\sigma) \in \tilde{B}_0^\sharp(\VEC{p} \ast K;R)$.  Thus,
there exists $\displaystyle b_\sigma \in \tilde{C}_1^\sharp(\VEC{p}\ast K;R)$
such that
$\displaystyle \tilde{\partial}_1^\sharp(b_\sigma) = \eta(\sigma) - \Id(\sigma)$.
We may assume that
$\displaystyle b_\sigma \in \tilde{C}_1^\sharp(L_\sigma;R)$ because
$\displaystyle \eta(\sigma) - \Id(\sigma) \in \tilde{C}_0^\sharp(L_\sigma;R)$.

We set $D_0(\sigma) = b_\sigma$ for all ordered $0$-simplices $\sigma$
of $\VEC{p} \ast K$, and extend $D_0$ by linearity to
$\displaystyle \tilde{C}_0^\sharp(\VEC{p} \ast K;R)$.

By linearity, we have that
\[
\tilde{\partial}_1^\sharp(D_0(c)) + D_{-1}(\tilde{\partial}_0^\sharp(c))
= \tilde{\partial}_1^\sharp(D_0(c)) = \eta(c) - \Id(c)
\]
for all $\displaystyle c \in \tilde{C}_0^\sharp(\VEC{p} \ast K;R)$.
If $\displaystyle c \in \tilde{C}_0^\sharp(Q;R)$ for some simplicial
subcomplex $Q$ of $\VEC{p} \ast K$, then $c$ is the sum (after
simplification) of oriented $0$-simplices $\sigma$ in $Q$.  Since
$D_0(\sigma) \in \tilde{C}_1^\sharp(L_\sigma;R) \subset \tilde{C}_1^\sharp(Q,R)$
for all of these oriented $0$-simplices because $L_\sigma \subset Q$
for each of them, we have that $D_0(c) \in \tilde{C}_1^\sharp(Q,R)$.
In particular, $\displaystyle D_0 \big(\tilde{C}_0^\sharp(K;R)\big) \subset
\tilde{C}_1^\sharp(K;R)$.

\stage{$\mathbf{k>0}$}
We assume for $q <k$ that
$\displaystyle \tilde{\partial}_{q+1}^\sharp \circ D_q
+ D_{q-1} \circ \tilde{\partial}_q^\sharp = \eta - \Id$ on
$\displaystyle \tilde{C}_q^\sharp(\VEC{p} \ast K;R)$
and
$\displaystyle D_q \big(\tilde{C}_q^\sharp(Q;R)\big) \subset
\tilde{C}_{q+1}^\sharp(Q;R)$ for all simplicial
subcomplexes $Q$ of $\VEC{p} \ast K$.

Suppose that $\sigma$ is an ordered $k$-simplex of $\VEC{p} \ast K$.
Let $\displaystyle c_\sigma = \eta(\sigma) - \Id(\sigma) -
D_{k-1}(\tilde{\partial}_k^\sharp(\sigma))$.
We have that $\displaystyle c_\sigma \in
\tilde{C}_k^\sharp(L_\sigma;R)$ because
$\displaystyle D_{k-1}\big(\tilde{C}_{k-1}^\sharp(L_\sigma;R)\big)
\subset \tilde{C}_k^\sharp(L_\sigma;R)$ by hypothesis of induction.

From our hypothesis of induction, we also have that
\begin{align*}
\tilde{\partial}_k^\sharp(c_\sigma)
&= \tilde{\partial}_k^\sharp\big( \eta(\sigma) \big)
- \tilde{\partial}_k^\sharp\big(\Id(\sigma)\big) -
\tilde{\partial}_k^\sharp\big(D_{k-1}(\tilde{\partial}_k^\sharp(\sigma))\big) \\
&= \tilde{\partial}_k^\sharp\big( \eta(\sigma) \big)
- \tilde{\partial}_k^\sharp\big(\Id(\sigma)\big)
- \big( - D_{k-2}\big(\tilde{\partial}_{k-1}^\sharp
(\tilde{\partial}_k^\sharp(\sigma))\big)
+ \eta(\tilde{\partial}_k^\sharp(\sigma))
- \Id(\tilde{\partial}_k^\sharp(\sigma))\big) \\
&= \eta(\tilde{\partial}_k^\sharp(\sigma))
- (\Id)(\tilde{\partial}_k^\sharp(\sigma))
+ D_{k-2}\big(\tilde{\partial}_{k-1}^\sharp
(\tilde{\partial}_k^\sharp(\sigma))\big)
- \eta(\tilde{\partial}_k^\sharp(\sigma))
+ \Id(\tilde{\partial}_k^\sharp(\sigma)) \\
&= 0
\end{align*}
because
$\displaystyle \tilde{\partial}_{k-1}^\sharp(\tilde{\partial}_k^\sharp(\sigma))
= 0$.
Thus $\displaystyle c_\sigma \in \tilde{Z}_k^\sharp(\VEC{p} \ast K;R)$.
Since $\displaystyle \tilde{H}_{k+1}^\sharp(\VEC{p} \ast K;R) = 0$,
we have that
$\displaystyle c_\sigma \in \tilde{B}_k^\sharp(\VEC{p} \ast K;R)$.
Therefore, there exists
$\displaystyle b_\sigma \in \tilde{C}_{k+1}^\sharp(\VEC{p} \ast K;R)$
such that $\displaystyle \tilde{\partial}_{k+1}^\sharp(b_\sigma) = c_\sigma$.
We may assume that
$\displaystyle b_\sigma \in \tilde{C}_{k+1}^\sharp(L_\sigma;R)$
because $\displaystyle c_\sigma \in \tilde{C}_k^\sharp(L_\sigma;R)$.

As expected, we set $D_k(\sigma) = b_\sigma$ for all ordered
$k$-simplices $\sigma$ of $\VEC{p} \ast K$, and extend $D_k$ by
linearity to $\displaystyle \tilde{C}_k^\sharp(\VEC{p} \ast K;R)$.

By linearity, we have that
\begin{align*}
\tilde{\partial}_{k+1}^\sharp(D_k(c)) + D_{k-1}(\tilde{\partial}_k^\sharp(c))
&= \big( \eta(c) - \Id(c) -
D_{k-1}(\tilde{\partial}_k^\sharp(c)) \big)
+ D_{k-1}(\tilde{\partial}_k^\sharp(c)) \\
&= \eta(c) - \Id(c)
\end{align*}
for all $\displaystyle c \in \tilde{C}_k^\sharp(\VEC{p} \ast K;R)$.
If $\displaystyle c \in \tilde{C}_k^\sharp(Q;R)$ for some simplicial
subcomplex $Q$ of $\VEC{p} \ast K$, then $c$ is the sum (after
simplification) of oriented $k$-simplices $\sigma$ in $Q$.  Since,
$D_k(\sigma) \in \tilde{C}_{k+1}^\sharp(L_\sigma;R) \subset
\tilde{C}_{k+1}^\sharp(Q,R)$
for all of these oriented $k$-simplices because $L_\sigma \subset Q$
for each of them, we have that $D_k(c) \in \tilde{C}_{k+1}^\sharp(Q,R)$.
In particular, $\displaystyle D_0 \big(\tilde{C}_k^\sharp(K;R)\big) \subset
\tilde{C}_{k+1}^\sharp(K;R)$.

The operator $D_k$ that we are looking for in the statement of the
proposition is $D_K$ defined above restricted to
$\displaystyle \tilde{C}_k^\sharp(K;R)$.
\end{proof}

We have that $G = F \circ \phi$.  Therefore,
$\displaystyle H_k^\sharp(G) = \tilde{H}_k^\sharp(F) \circ
\tilde{H}_k^\sharp(\phi) : H_k^\sharp(K;R) \to H_k^\sharp([K];R)$ is
an homomorphism.  Since
$\displaystyle \tilde{H}_k^\sharp(\phi): H_k^\sharp(K;R) \to
\tilde{H}_k^\sharp(K;R)$ is an isomorphism and
$\displaystyle \tilde{H}_k^\sharp(F): \tilde{H}_k^\sharp(K;R) \to
H_k^\sharp([K];R)$ is an isomorphism according to the
Proposition~\ref{propHkEHkv1}, we have that
$\displaystyle H_k^\sharp(G) : H_k^\sharp(K;R) \to H_k^\sharp([K];R)$ is an
isomorphism.  For similar reasons,
$\displaystyle H_k(G) : H_k(K;R) \to H_k([K];R)$ is an
isomorphism.

We can repeat the previous discussion in the context of relative
homology.
Since $\displaystyle \phi:C_k(L;R) \to \tilde{C}_k(L;R)$,
we may define the homomorphism
$\displaystyle \phi:C_k(K,L;R) \to \tilde{C}_k(K,L;R)$ by
$\phi(\relC[K,L]{c}) = \relC[K,L]{\phi(c)}$ for all
$\displaystyle \relC[K,L]{c} \in C_k(K,L;R)$, where as usual
$\relC[K,L]{\cdot}$ represents the
equivalence classes in both $\displaystyle C_k(K,L;R)$,
and $\displaystyle \tilde{C}_k(K,L;R)$.
Similarly,
Since $\displaystyle \psi:\tilde{C}_k(L;R) \to C_k(L;R)$,
we may define the homomorphism
$\displaystyle \psi:\tilde{C}_k(K,L;R) \to C_k(K,L;R)$ by
$\psi(\relC[K,L]{c}) = \relC[K,L]{\psi(c)}$ for all
$\displaystyle \relC[K,L]{c} \in \tilde{C}_k(K,L;R)$.
We have the following commutative diagram 
\[
\xymatrix@C+2ex{
\ar[r]^(0.2){\overline{\partial}_{k+2}}
& C_{k+1}(K,L;R)  \ar[r]^-{\overline{\partial}_{k+1}}
\ar@/^/[d]^-{\phi}
& C_k(K,L;R) \ar[r]^-{\overline{\partial}_k} \ar@/^/[d]^-{\phi}
& C_{k-1}(K,L;R) \ar[r]^(0.8){\overline{\partial}_{k-1}}
\ar@/^/[d]^-{\phi} & \\
\ar[r]_(0.2){\overline{\tilde{\partial}}_{k+2}}
& \tilde{C}_{k+1}(K,L;R) \ar[r]_-{\overline{\tilde{\partial}}_{k+1}}
\ar@/^/[u]^-{\psi}
& \tilde{C}_k(K,L;R) \ar[r]_-{\overline{\tilde{\partial}}_k}
 \ar@/^/[u]^-{\psi}
& \tilde{C}_{k-1}(K,L;R) \ar[r]_(0.8){\overline{\tilde{\partial}}_{k-1}}
\ar@/^/[u]^-{\psi} &
}
\]
Namely, $\{\phi_k\}_{k\in \ZZ}$ with $\phi_k = \phi$ for all $k$ is a
chain map from the chain complex
$\displaystyle \C = \{(C_k(K,L;R),
\overline{\partial}_k)\}_{k\in \ZZ}$
to the chain complex
$\tilde{\C} = \{(\tilde{C}_k(K,L;R),
\overline{\tilde{\partial}}_k)\}_{k\in \ZZ}$ and
$\{\psi_k\}_{k\in \ZZ}$ with $\psi_k = \psi$ for all $k$ is a
chain map from $\displaystyle \tilde{\C}$ to $\displaystyle \C$.

We may define the homomorphism
$\displaystyle \tilde{H}_k(\phi):H_k(K,L;R) \to \tilde{H}_k(K,L;R)$ by\\
$\displaystyle \tilde{H}_k(\phi)([c]_{K,L}) = [\phi(c)]_{K,L}$ for all
$\displaystyle [c]_{K,L} \in H_k(K,L;R)$,
where as usual $[\cdot]_{K,L}$ represents the equivalence classes in both
$\displaystyle H_k(K,L;R)$, and $\displaystyle \tilde{H}_k(K,L;R)$
As well, we may define the homomorphism
$\displaystyle \tilde{H}_k(\psi):\tilde{H}_k(K,L;R) \to H_k(K,L;R)$ by
$\displaystyle \tilde{H}_k(\psi)([c]_{K,L}) = [\psi(c)]_{K,L}$ for all
$\displaystyle [c]_{K,L} \in \tilde{H}_k(K,L;R)$.

We can repeat all the previous discussion for the reduced
homology and define
$\displaystyle \tilde{H}_k^\sharp(\phi):H_k^\sharp(K,L;R) \to
\tilde{H}_k^\sharp(K,L;R)$ and
$\displaystyle \tilde{H}_k^\sharp(\psi):\tilde{H}_k^\sharp(K,L;R) \to
H_k^\sharp(K,L;R)$.

As we did for the regular homology, we can use
Proposition~\ref{propCsCtfauHsH} to conclude that
$\displaystyle \tilde{H}_k^\sharp(\phi)$ is an isomorphism if and only
if $\displaystyle \tilde{H}_k(\phi)$ is an isomorphism.
To prove that $\displaystyle \tilde{H}_k^\sharp(\phi)$ is an
isomorphism, we prove that $\displaystyle \tilde{H}_k^\sharp(\psi)$ is
the inverse of $\displaystyle \tilde{H}_k^\sharp(\phi)$.  The proof is almost
identical to the proof that we gave above for the reduced homology.
The only difference is that we use the following proposition instead
of Proposition~\ref{propOOhomCM}.

\begin{prop}
There exists $\DD = \{ D_k \}_{k\in \ZZ}$ with
$\displaystyle D_k : \tilde{C}_k^\sharp(K,L;R) \to
\tilde{C}_{k+1}^\sharp(K,L;R)$ such that
$\displaystyle \tilde{\partial}_{k+1}^{\,\sharp} \circ D_k
+ D_{k-1} \circ \tilde{\partial}_k^{\,\sharp} =
\phi\circ \psi - \Id$ on $\displaystyle \tilde{C}_k^\sharp(K,L;R)$.
\end{prop}

\begin{proof}[Proof (Sketch)]
We can use the chain homotopy $\DD = \{ D_k \}_{k\in \ZZ}$ obtained in
Proposition~\ref{propOOhomCM} to induce a
chain homotopy between $\F = \{ f_k\}_{k\in \ZZ}$ with
$\displaystyle f_k = \Id:\tilde{C}_k^\sharp(K,L;R) \to \tilde{C}_k^\sharp(K,L;R)$
and $\GG = \{ g_k\}_{k\in \ZZ}$ with
$\displaystyle g_k = \phi\circ \psi :\tilde{C}_k^\sharp(K,L;R) \to
\tilde{C}_k^\sharp(K,L;R)$.

It suffices to note that $\displaystyle D_k: \tilde{C}_k^\sharp(K;R)
\to \tilde{C}_{k+1}^\sharp(K;R)$ was defined in such a way to have
that $\displaystyle D_k: \tilde{C}_k^\sharp(L;R)
\to \tilde{C}_{k+1}^\sharp(L;R)$.  Thus we can define
$\displaystyle D_k: \tilde{C}_k^\sharp(K,L;R)
\to \tilde{C}_{k+1}^\sharp(K,L;R)$ in a natural way as we have done
with $\displaystyle \tilde{\partial}_k^{\,\sharp}$ for instance to get
$\displaystyle \overline{\tilde{\partial}}_k^{\,\sharp}$.
\end{proof}

We get from $G = F \circ \phi$ that
$\displaystyle H_k(G) = \tilde{H}_k(F) \circ
\tilde{H}_k(\phi) : H_k(K,L;R) \to H_k([K],[L];R)$ is
an homomorphism.  Since
$\displaystyle \tilde{H}_k(\phi): H_k(K,L;R) \to \tilde{H}_k(K,L;R)$
is an isomorphism and
$\displaystyle \tilde{H}_k(F): \tilde{H}_k(K,L;R) \to H_k([K],[L];R)$
is an isomorphism according to the Proposition~\ref{propHkEHkv1}, we have that
$\displaystyle H_k(G) : H_k(K,L;R) \to H_k([K],[L];R)$ is an isomorphism.

In summary, we get the following result.

\begin{prop}  \label{propIsoSShom}
Let $L$ be a subcomplex of a simplicial complex $K$ and $R$ is an
integral domain. Then
$H_k(G): H_k(K;R) \to H_k([K];R)$
and $H_k(G) : H_k(K,L;R) \to H_k([K],[L];R)$ are 
isomorphisms.  We have a similar result for the reduced homology.
\end{prop}

To conclude this subsection, we should mention the following result.
Suppose that $f:([K_1],[L_1]) \to ([K_2],[L_2])$ is a continuous maps
where $L_i$ is a simplicial subcomplex of the simplicial complex
$K_i$ for $i =1,2$.  Then we have the following commutative
diagram.
\[
\xymatrix@C+2em{
H_k(K_1,L_1;R) \ar[r]^{H_k(f)}
\ar[d]_{H_k(G)} & H_k(K_2,L_2;R) \ar[d]^{H_k(G)} \\
H_k([K_1],[L_1];R) \ar[r]^-{{H}_k(f)} & H_k([K_2],[L_2];R)
}
\]
The proof of this result is interesting.  It makes use of simplicial
approximation.  This is a naturality result for those who may know
about category theory.

\subsection{Relation Between Simplicial Cohomology and Singular
Cohomology}

Suppose that $L$ is a simplicial subcomplex of a simplicial complex
$K$, and that $R$ is an integral domain.  We have defined in the
previous subsection an homomorphism
$\displaystyle G: C_k(K;R) \to S_k([K];R)$ and shown that it
induces isomorphism $H_k(G):H_k(K;R) \to H_k([K];R)$ and 
$H_k(G):H_k(K,L;R) \to H_k([K],[L];R)$.

By duality, we get the following result.

\begin{prop}
Let $L$ be a subcomplex of a simplicial complex $K$ and $R$ is an
integral domain. Then
$\displaystyle H^k(G): H^k([K];R) \to H^k(K;R)$
and $\displaystyle H^k(G) : H^k([K],[L];R) \to H^k(K,L;R)$ are 
isomorphisms.
\end{prop}

As for $H_k(G)$, suppose that $f:([K_1],[L_1]) \to ([K_2],[L_2])$ is a
continuous maps where $L_i$ is a simplicial subcomplex of the
simplicial complex $K_i$ for $i =1,2$.  Then we have the
following commutative diagram.
\[
\xymatrix@C+2em{
H^k([K_2],[L_2];R) \ar[r]^{H^k(f)}
\ar[d]_{H^k(G)} & H^k([K_1],[L_1];R) \ar[d]^{H^k(G)} \\
H^k(K_2,L_2;R) \ar[r]^-{{H}^k(f)} & H^k(K_1,L_1;R)
}
\]

%%% Local Variables:
%%% mode: latex
%%% TeX-master: "notes"
%%% End:



%%% Local Variables:
%%% mode: latex
%%% TeX-master: "notes"
%%% End:
