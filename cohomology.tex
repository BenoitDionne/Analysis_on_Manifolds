\chapter{Cohomology}  \label{chaptCohom}

We have seen in Theorem~\ref{closedexact} that closed differential
$k$-forms defined on a star-shaped domain are exact.  In this chapter,
we show that the property that closed differential $k$-forms are exact
is determined by the domain of the differential $k$-form only.

Depending of the researcher to whom we talk, the topic of cohomology
is either part of algebraic topology, differential topology or
differential geometry.  This is another illustration that the
boundaries between different fields of mathematics are often fluid and
blur.

The presentation in this section is inspired by the presentation of
cohomology in \cite{GP,Sv1}.  The first section introduce some of the
important tools that we will use in our study of cohomology.

To simplify the presentation in this chapter, we will only consider
smooth manifolds; namely, manifolds of class $\displaystyle C^\infty$.
Similarly, all the differential forms will be of class
$\displaystyle C^\infty$.  These restrictions may often be reduced to
$\displaystyle C^j$ with $j$ larger than the dimension of the manifold.

\section{Preliminaries}  \label{sectPrelCohom}

\begin{defn}  \label{defnCtoP}
A $k$-dimensional smooth manifold $S$ is
{\bfseries (smoothly) contractible to a point}\index{Contractible to a
Point}\index{Smoothly Contractible to a Point|see{Contractible to a
Point}} if there exist a map $H:S\times [0,1] \to S$ of class
$\displaystyle C^\infty$ \footnotemark and a point $\tilde{\VEC{u}}
\in S$ such that $H(\VEC{u},1) = \VEC{u}$ and
$H(\VEC{u},0) = \tilde{\VEC{u}}$ for all $\VEC{u} \in S$.
\end{defn}

\footnotetext{To follow our convention about function of class
$\displaystyle C^\infty$ on a closed interval, we assume that
there exists an open set $V \supset [0,1]$ such that $H$ can be
extended to a $\displaystyle C^\infty$ function from
$S \times V$ to $S$.}

It is easy to show that $\displaystyle \RR^n$ is contractible to the origin.
It suffices to use the map $\displaystyle H:\RR^n \times [0,1] \to \RR^n$
defined by $H(\VEC{x},t) = t \VEC{x}$.

\begin{rmk}
Recall that the {\bfseries direct sum}\index{Direct Sum}
of two linear spaces $M$ and $N$ over an integral domain $R$
(e.g.\ $\RR$), denoted $M \oplus N$, is the linear
space $M \times N$ with the addition defined by
$(m_1,n_1) + (m_2,n_2) = (m_1 + m_2,n_1 + n_2)$ for 
$m_1,m_2 \in M$ and $n_1,n_2 \in N$, and the product by a scalar
defined by $c(m,n) = (c m, c n)$ for $m \in M$, $n \in N$ and $c \in R$.
In particular, if $M$ and $N$ are two subspaces of a vector space
$E$ such that $\dim(M) + \dim(N) = \dim(E)$ and
$M \cap N = \{\VEC{0}\}$, then we may write $E = M \oplus N$
by identifying $(m,n)$ to $m+n$ because every element of $E$ has a
unique representation of the form $m+n$ with $m \in M$ and $n \in N$.
If $M$, $N$, $\tilde{M}$ and $\tilde{N}$ are four linear spaces, and
$f:M \to \tilde{M}$ and $g:N\to \tilde{N}$ are two linear maps, then
$(f,g): M\oplus N \to \tilde{M} \oplus \tilde{N}$, defined by
$(f,g)(m,n) = (f(m),g(n))$ for all $(m,n) \in M \times N$, is a linear map.
\end{rmk}

Let $S$ be a $k$-dimensional smooth manifold and $J \subset \RR$ be an
open interval.  Then $S \times J$ with is a $(k+1)$-dimensional smooth
manifold.  In particular, if
$\displaystyle \{(W_\alpha,U_\alpha,\phi_\alpha)\}_{\alpha\in A}$ is
an atlas for $S$, then
$\displaystyle \left\{\big(W_\alpha \times J, U_\alpha \times J,
(\phi_\alpha,\Id)\big)\right\}_{\alpha\in A}$
is an atlas for $S\times J$ \footnote{The map
$(\phi_\alpha,\Id):(W_i\times J) \to (U_i \times J)$
is obviously defined as
$(\phi_i,\Id)(\VEC{w},x) = (\phi_i(\VEC{w}),x)$ for all
$(\VEC{w},x) \in W_i \times J$.}.  We also have that
$\TS_{(\VEC{u},x)} (S \times J) = \TS_{\VEC{u}} S \oplus \TS_x J$
for all $(\VEC{u},x) \in S\times J$, where\\
$\TS_{\VEC{u}} S \cong \left\{ \big((\VEC{u},x),(\VEC{y},0)\big) :
(\VEC{u},\VEC{y}) \in \TS_{\VEC{u}} S \right\}$ and
$\TS_x J \cong \left\{ \big((\VEC{u},x),(\VEC{0},z)\big) :
(x,z) \in \TS_x J \right\}$.

\begin{prop} \label{propO1dxO2}
Suppose that $S$ is a $k$-dimensional smooth manifold and $J\subset \RR$ is
an open interval.  Any differential $q$-form $\omega$ on
$S \times J$ can be expressed as
$\omega = \omega_1 + \df{x} \wedge \omega_2$ where
\begin{enumerate}
\item $\omega_1$ is a differential $q$-form on $S \times J$ such that \\
$\displaystyle \omega_1(\VEC{w},x)
\big(\big((\VEC{u},x),(\VEC{v}_1,y_1)\big),
\ldots, \big((\VEC{u},x),(\VEC{v}_q,y_q)\big)\big) = 0$
if $\VEC{v}_i = 0$ for some $1\leq i \leq q$,
\item $\omega_2$ is a differential $(q-1)$-form on $S\times J$ such that \\
$\displaystyle \omega_2(\VEC{w},x)
\big(\big((\VEC{u},x),(\VEC{v}_1,y_1)\big),
\ldots, \big((\VEC{u},x),(\VEC{v}_{q-1},y_{q-1})\big)\big) = 0$
if $\VEC{v}_i = 0$ for some $1\leq i \leq q-1$, and
\item $\df{x}$ is the differential $1$-form on $S \times J$ defined by
$\displaystyle \df{x} (\VEC{u},x) \big( (\VEC{u},x),(\VEC{v},y)\big) = y$
for all $(\VEC{u},x) \in S \times J$ and
$\displaystyle \big( (\VEC{u},x),(\VEC{v},y)\big) \in
\TS_{(\VEC{u},x)} (S \times J)$.
\end{enumerate}
\end{prop}

\begin{proof}
We consider a local chart $\big(W \times J,U \times J, (\phi,\Id)\big)$
of $S\times J$.

Let $\displaystyle \left\{ \psi_j(\VEC{u}) \right\}_{1\leq j \leq k}
\subset \left(\TS_{\VEC{u}} S\right)^\ast$ be the dual basis
associated to the basis \\
$\displaystyle \left\{ \big( \VEC{u}, (\diff \phi)(\phi^{-1}(\VEC{u})) \VEC{e}_j
\big) \right\}_{1\leq j \leq k}$ of $\TS_{\VEC{u}} S$.
Consider
\begin{align*}
\tilde{\psi}_j(\VEC{u},x) : \TS_{(\VEC{u},x)} (S\times J) & \to \RR \\
\big( (\VEC{u},x),(\VEC{v},y) \big) &\mapsto 
\psi_j(\VEC{u}) (\VEC{u},\VEC{v})
\end{align*}
for $1 \leq j \leq k$ and
\begin{align*}
\tilde{\psi}_{k+1}(\VEC{u},x) : \TS_{(\VEC{u},x)} (S\times J) & \to \RR \\
\big( (\VEC{u},x),(\VEC{v},y) \big) &\mapsto \df{x}(x) (x,y) = y
\end{align*}
We have that
$\displaystyle \left\{ \tilde{\psi}_j(\VEC{u},x) \right\}_{1\leq j \leq k+1}
\subset \left(\TS_{(\VEC{u},x)} (S\times J) \right)^\ast$ is the dual basis
associated to the basis\\
$\displaystyle \left\{ \big( (\VEC{u},x), \big(
\diff \phi(\phi^{-1}(\VEC{u}))\, \VEC{e}_j,0\big) \big) \right\}_{1\leq j \leq k}
\cup \left\{ \big( (\VEC{u},x),(\VEC{0},1) \big)\right\}$ of
$\TS_{(\VEC{u},x)} (S\times J) = \TS_{\VEC{u}} S \oplus \TS_x J$.

It follows from Theorem~\ref{thOkbasis} that
\begin{align*}
\omega &= \sum_{1 \leq i_1< i_2< \ldots< i_q \leq k+1}
\omega_{i_1,i_2,\ldots,i_q}\,
\tilde{\psi}_{i_1} \wedge \tilde{\psi}_{i_2} \wedge \ldots \wedge
\tilde{\psi}_{i_q} \\
&= \underbrace{\sum_{1 \leq i_1< i_2< \ldots< i_q \leq k}
\omega_{i_1,i_2,\ldots,i_q}\,
\tilde{\psi}_{i_1} \wedge \tilde{\psi}_{i_2} \wedge \ldots \wedge
\tilde{\psi}_{i_q}}_{=\, \omega_1} \\
&\qquad + \tilde{\psi}_{k+1} \wedge \underbrace{\left(
(-1)^{q-1}\sum_{1 \leq i_1< i_2< \ldots< i_{q-1} \leq k}
\omega_{i_1,i_2,\ldots,i_{q-1},k+1}\,
\tilde{\psi}_{i_1} \wedge \tilde{\psi}_{i_2} \wedge \ldots \wedge
\tilde{\psi}_{i_{q-1}} \right)}_{= \, \omega_2}
\end{align*}
on $U\times J$, where the functions
$\omega_{i_1,i_2,\ldots,i_q}:U\times J \to J$ and
$\omega_{i_1,i_2,\ldots,i_{q-1},k+1}:U\times J \to J$ are of class
$\displaystyle C^\infty$.
From the definition of
$\tilde{\psi}_j$ for $1\leq j \leq k$, it is clear that
$\omega_1$ satisfies the condition (1) in the statement of the
proposition and $\omega_2$ satisfies the condition (2).
$\tilde{\psi}_{k+1}$ is the differential $1$-form $\df{x}$ defined in
the statement of the proposition.

Note that the local representation of $\omega_1$ and $\omega_2$ may change
for a different local chart but their action will not.  As for
$\df{x}$, it is invariant with respect to the local charts
$\big(W \times J,U \times J, (\phi,\Id)\big)$ chosen.
\end{proof}

The inclusion $\iota_x:S \to S\times J$ is defined by
$\iota_x(\VEC{u}) = (\VEC{u},x)$ for all $\VEC{u} \in S$ and $x \in J$.
If $\omega$ is a differential $q$-form on $S \times J$, then
$\displaystyle \iota_x^\ast(\omega)$ is the differential $q$-form on $S$
defined by
\begin{align*}
\left(\iota_x^\ast(\omega)\right)(\VEC{u}) \big((\VEC{u},\VEC{v}_1),
\ldots, (\VEC{u},\VEC{v}_q)\big)
&= \iota_x^\ast( \omega(\VEC{u},x) )\big((\VEC{u},\VEC{v}_1),
\ldots, (\VEC{u},\VEC{v}_q)\big) \\
&= \omega(\VEC{u},x)\Big(\big((\VEC{u},x),(\VEC{v}_1,0)\big),
\ldots, \big((\VEC{u},x),(\VEC{v}_q,0)\big)\Big)
\end{align*}
for all $x \in J$, $\VEC{u} \in S$ and
$(\VEC{u},\VEC{v}_i) \in \TS_{\VEC{u}} S$ with $1\leq i \leq q$.

\begin{rmk}
Consider a local chart            \label{rmkistis}
$\displaystyle \big(W \times J, U \times J, (\phi,\Id)\big)$
of $S\times J$ where $\displaystyle \big(W, U, \phi\big)$ is a local
chart of $S$.  We have that $\iota_x:S \to S \times J$ for $x \in J$
and its local representation $\tilde{\iota}_x:W \to W \times J$
satisfy the relation
$\displaystyle (\phi,\Id) \circ \tilde{\iota}_x = \iota_x \circ \phi$;
namely, we have the following commutative diagram.
\[
\xymatrix{
S \ar[r]^-{\iota_x} & S \times J \\
W \ar[r]_-{\tilde{\iota}_x} \ar[u]^{\phi} & W \times J \ar[u]_{(\phi,\Id)} 
}
\]
Since
$(\iota_x \circ \phi)(\VEC{w}) = (\phi(\VEC{w}), x)$, we get that
$\tilde{\iota}_x(\VEC{w}) = (\VEC{w},x)$ for all $\VEC{w} \in W$.
Since $\displaystyle \diff \tilde{\iota}_x(\VEC{w}) (\VEC{y}) =
(\VEC{y},0)$ and
\[
(\tilde{\iota}_x)_\ast(\VEC{w},\VEC{y})
= \big( \tilde{\iota}_x(\VEC{w}), \diff \tilde{\iota}_x(\VEC{w})(\VEC{y})\big)
= \big( (\VEC{w},x), (\VEC{y}, 0)\big) \in \TS_{\VEC{w},x} (W \times J) \ ,
\]
we get from
\[
(\iota_x)_\ast \circ \phi_\ast
= (\iota_x \circ \phi)_\ast
= ((\phi,\Id) \circ \tilde{\iota}_x)_\ast
= (\phi,\Id)_\ast \circ (\tilde{\iota}_x)_\ast
\]
that
\begin{align*}
(\iota_x)_\ast \big( \phi_\ast ( \VEC{w},\VEC{y})\big)
&= (\phi,\Id)_\ast\big( (\tilde{\iota}_x)_\ast( \VEC{w},\VEC{y}) \big)
= (\phi,\Id)_\ast\big( (\VEC{w},x) , (\VEC{y}, 0) \big) \\
&= \big( (\phi(\VEC{w}),x) , (\diff \phi(\VEC{w}) \VEC{y}, 0) \big)
\end{align*}
for all $\VEC{w} \in W$ and $\displaystyle \VEC{y} \in \RR^k$, 
This translates into
$\displaystyle (\iota_x)_\ast (\VEC{u},\VEC{v}) =
\big((\VEC{u},x),(\VEC{v},0)\big) \in \TS_{\VEC{u},x} (S\times J)$
for $\VEC{u} \in U$ and $(\VEC{u},\VEC{v}) \in \TS_{\VEC{u}} S$.
\end{rmk}

From now on, we assume that $J \subset \RR$ is an open interval
containing $[0,1]$.

We define an operator $F$ that maps differential $q$-forms on $S\times
J$ to differential $(q-1)$-forms on $S$ as it follows.  Suppose that
$\omega$ is a differential $q$-form on $S \times J$.  From
Proposition~\ref{propO1dxO2}, we may express $\omega$ as
$\omega = \omega_1 + \df{x}\wedge \omega_2$.  We define $F(\omega)$ as
\begin{align}
&F(\omega)(\VEC{u})\big( (\VEC{u},\VEC{v}_1), \ldots, (\VEC{u},\VEC{v}_{q-1})
\big)
= \int_0^1 (\iota_x)^\ast(\omega_2)(\VEC{u})
\big( (\VEC{u},\VEC{v}_1), \ldots, (\VEC{u},\VEC{v}_{q-1})\big) \dx{x}
\nonumber \\
&\hspace{7em} = \int_0^1 \omega_2(\VEC{u},x)
\big( (\iota_x)_\ast(\VEC{u},\VEC{v}_1),
\ldots, (\iota_x)_\ast(\VEC{u},\VEC{v}_{q-1}) \big) \dx{x} \nonumber \\
&\hspace{7em} = \int_0^1 \omega_2(\VEC{u},x)
\Big( \big((\VEC{u},x),(\VEC{v}_1,0)\big), \ldots,
\big( (\VEC{u},x),(\VEC{v}_{q-1},0)\big) \Big)\dx{x}   \label{defFSItoS}
\end{align}
for all $\VEC{u} \in S$ and $(\VEC{u},\VEC{v}_1) \in \TS_{\VEC{w}} S$.

\begin{prop} \label{propi1i0dffd}
Suppose that $S$ is a $k$-dimensional smooth manifold and that
$J \subset \RR$ is an open interval containing $[0,1]$.
If $\omega$ is a differential $q$-form on $S \times \RR$,
then $\displaystyle \iota_1^\ast(\omega) - \iota_0^\ast(\omega)
= \df{(F(\omega))} + F(\df{\omega})$.
\end{prop}

\begin{proof}
From Proposition~\ref{propO1dxO2}, we may express $\omega$ as
$\omega = \omega_1 + \df{x}\wedge \omega_2$, where $\omega_1$
satisfies the condition (1) in the statement of the proposition and
$\omega_2$ satisfies the condition (2).

Let $\big(W \times J,U \times J, (\phi,\Id)\big)$ be a local chart of
$S\times J$.  We have
\begin{equation} \label{i1i0dffdEq1}
\begin{split}
(\phi,\Id)^\ast (\omega) &= (\phi,\Id)^\ast(\omega_1)
+ (\phi,\Id)^\ast(\df{x}) \wedge (\phi,\Id)^\ast(\omega_2) \\
&= \underbrace{\sum_{1\leq i_1<i_2<\ldots<i_q \leq k} g_{i_1,i_2,\ldots,i_q}
\, \df{\tilde{w}_{i_1}} \wedge \df{\tilde{w}_{i_2}} \wedge \ldots \wedge
\df{\tilde{w}_{i_q}}}_{=\, \tilde{\omega}_1} \\
&\qquad + \df{x}\wedge \underbrace{ \sum_{1\leq i_1<i_2<\ldots<i_{q-1} \leq k}
h_{i_1,i_2,\ldots,i_{q-1}} \, \df{\tilde{w}_{i_1}} \wedge \df{\tilde{w}_{i_2}}
\wedge \ldots \wedge \df{\tilde{w}_{i_{q-1}}}}_{=\, \tilde{\omega}_2}
\end{split}
\end{equation}
for some continuously differentiable functions
$g_{i_1,i_2,\ldots,i_q}:W \to \RR$ and
$h_{i_1,i_2,\ldots,i_{q-1}}:W \to \RR$, where
$\displaystyle \df{\tilde{w}_j}(\VEC{w},x)
\big((\VEC{w},x),(\VEC{v},y)\big) = \df{w_j}(\VEC{w})\big(\VEC{w},\VEC{v}\big)
= v_j$ and
$\displaystyle \df{x}(\VEC{w},x) \big((\VEC{w},x),(\VEC{v},y)\big) = y$
for $1\leq j \leq k$ and $\displaystyle \big((\VEC{w},x),(\VEC{v},y)\big) \in
\TS_{(\VEC{w},x)} (W \times J) \cong \TS_{\VEC{w}} W \oplus \TS_xJ$.

Moreover,  we have
\begin{align*}
&(\phi^\ast(F(\omega)))(\VEC{w})\big( (\VEC{w},\VEC{v}_1), \ldots,
(\VEC{w},\VEC{v}_{q-1})\big) \\
&\quad = F(\omega)(\phi(\VEC{w}))\big( (\phi(\VEC{w}),\diff
\phi(\VEC{w}) \, \VEC{v}_1), \ldots,
(\phi(\VEC{w}),\diff \phi(\VEC{w})\,\VEC{v}_{q-1})\big) \\
&\quad = \int_0^1 \omega_2(\phi(\VEC{w}),x)
\Big( \big((\phi(\VEC{w}),x),(\diff \phi(\VEC{w})\,\VEC{v}_1,0)\big), \ldots,
\big( (\phi(\VEC{w}),x),(\diff \phi(\VEC{w})\,\VEC{v}_{q-1},0)\big)
\big) \dx{x} \\
&\quad = \int_0^1 (\phi,\Id)^\ast(\omega_2)(\VEC{w},x)
\Big( \big((\VEC{w},x),(\VEC{v}_1,0)\big), \ldots,
\big( (\VEC{w},x),(\VEC{v}_{q-1},0)\big) \Big) \dx{x} \\
&\quad =F\big((\phi,\Id)^\ast(\omega)\big)
(\VEC{w})\big( (\VEC{w},\VEC{v}_1), \ldots,(\VEC{w},\VEC{v}_{q-1})\big)
\end{align*}
for all $\VEC{w} \in W$ and $(\VEC{w},\VEC{v}_i) \in \TS_{\VEC{w}} W$
with $1\leq i \leq q-1$ \footnote{We should probably have used
$\tilde{F}$ to denote the local representation of $F$.  Hopefully, the
context should be enough to determine when we are referring to the
local representation of $F$.}.  Similarly.
\begin{align*}
&\big(\phi^\ast(\iota_x^\ast(\omega))\big)
(\VEC{w})\big( (\VEC{w},\VEC{v}_1), \ldots,
(\VEC{w},\VEC{v}_{q-1})\big) \\
&\quad = \iota_x^\ast(\omega)(\phi(\VEC{w}))\big( (\phi(\VEC{w}),\diff
\phi(\VEC{w})\,\VEC{v}_1), \ldots,
(\phi(\VEC{w}),\diff \phi(\VEC{w})\,\VEC{v}_{q-1})\big) \\
&\quad = \omega(\phi(\VEC{w}),x)
\Big( \big((\phi(\VEC{w}),x),(\diff \phi(\VEC{w})\,\VEC{v}_1,0)\big), \ldots,
\big( (\phi(\VEC{w}),x),(\diff \phi(\VEC{w})\,\VEC{v}_{q-1},0)\big)
\Big) \\
&\quad = (\phi,\Id)^\ast(\omega)(\VEC{w},x)
\Big( \big( (\VEC{w},x),(\VEC{v}_1,0)\big), \ldots,
\big( (\VEC{w},x),(\VEC{v}_{q-1},0)\big) \Big) \\
&\quad = \tilde{\iota}_x^\ast\big((\phi,\Id)^\ast(\omega)\big)
(\VEC{w})\big( (\VEC{w},\VEC{v}_1), \ldots,
(\VEC{w},\VEC{v}_{q-1})\big)
\end{align*}
for all $\VEC{w} \in W$ and $(\VEC{w},\VEC{v}_i) \in \TS_{\VEC{w}} W$
with $1\leq i \leq q-1$, where
$\displaystyle \tilde{\iota}_x(\VEC{w}) = (\VEC{w},x)$ for all $\VEC{w} \in W$
and $x \in J$ as we have seen in Remark~\ref{rmkistis}.  Therefore, it
suffices to prove that
\[
\tilde{\iota}_1^\ast\big((\phi,\Id)^\ast(\omega)\big)
- \tilde{\iota}_0^\ast\big((\phi,\Id)^\ast(\omega)\big)
= \df{(F((\phi,\Id)^\ast(\omega)))} + F(\df{((\phi,\Id)^\ast(\omega))}) \ .
\]
to get
\begin{align*}
\phi^\ast(\iota_1^\ast(\omega)) - \phi^\ast(\iota_0^\ast(\omega))
&= \tilde{\iota}_1^\ast\big((\phi,\Id)^\ast(\omega)\big)
- \tilde{\iota}_0^\ast\big((\phi,\Id)^\ast(\omega)\big) \\
&= \df{(F((\phi,\Id)^\ast(\omega)))} + F(\df{((\phi,\Id)^\ast(\omega))}) \\
&= \df{(\phi^\ast(F(\omega)))} + F((\phi,\Id)^\ast(\df{\omega}))
= \phi^\ast(\df{F(\omega)}) + \phi^\ast(F(\df{\omega})) \ .
\end{align*}

Using (\ref{i1i0dffdEq1}) and the linearity of the maps $F$ and
$i_x$, we have to prove that
\begin{equation} \label{i1i0dffdEq2}
\tilde{\iota}_1^\ast(\tilde{\omega}_1) - \tilde{\iota}_0^\ast(\tilde{\omega}_1)
= \df{(F(\tilde{\omega}_1))} + F(\df{(\tilde{\omega}_1)})
\end{equation}
and
\begin{equation} \label{i1i0dffdEq3}
\tilde{\iota}_1^\ast(\df{x} \wedge \tilde{\omega}_2)
- \tilde{\iota}_0^\ast(\df{x} \wedge \tilde{\omega}_2)
= \df{(F(\df{x} \wedge \tilde{\omega}_2))}
+ F(\df{(\df{x} \wedge \tilde{\omega}_2)}) \ .
\end{equation}

\stage{i} We first prove (\ref{i1i0dffdEq2}).  Since
$F(\tilde{\omega}_1) = 0$, we have that
$\df{(F(\tilde{\omega}_1))} = 0$.  Since
\begin{align*}
\df{(\tilde{\omega}_1)}
&= \sum_{1\leq i_1<i_2<\ldots<i_q \leq k}
\left( \sum_{j=1}^k \pdydx{g_{i_1,i_2,\ldots,i_q}}{w_j} \df{\tilde{w}_j}
\right)\, \df{\tilde{w}_{i_1}} \wedge \df{\tilde{w}_{i_2}} \wedge \ldots \wedge
\df{\tilde{w}_{i_q}} \\
&\qquad + \df{x} \wedge
\sum_{1\leq i_1<i_2<\ldots<i_q \leq k} \pdydx{g_{i_1,i_2,\ldots,i_q}}{x}
\, \df{\tilde{w}_{i_1}} \wedge \df{\tilde{w}_{i_2}} \wedge \ldots \wedge
\df{\tilde{w}_{i_q}} \ ,
\end{align*}
we get
\begin{align*}
&F(\df{(\tilde{\omega}_1)})(\VEC{w})
\big((\VEC{w},\VEC{v}_1), \ldots,(\VEC{w},\VEC{v}_q)\big)\\
&\qquad = \sum_{1\leq i_1<i_2<\ldots<i_q \leq k}
\bigg( \int_0^1 \pdydx{g_{i_1,i_2,\ldots,i_q}}{x}(\VEC{w},x) 
\left( \df{\tilde{w}_{i_1}} \wedge \df{\tilde{w}_{i_2}} \wedge \ldots \wedge
\df{\tilde{w}_{i_q}}\right) (\VEC{w},x) \\
&\hspace{17em} \Big( \big((\VEC{w},x),(\VEC{v}_1,0)\big), \ldots,
\big( (\VEC{w},x),(\VEC{v}_q,0)\big) \Big) \dx{x} \bigg) \\
&\qquad = \sum_{1\leq i_1<i_2<\ldots<i_q \leq k}
\bigg( \int_0^1 \pdydx{g_{i_1,i_2,\ldots,i_q}}{x}(\VEC{w},x) 
\left( \df{w_{i_1}} \wedge \df{w_{i_2}} \wedge \ldots \wedge
\df{w_{i_q}}\right) (\VEC{w}) \\
&\hspace{17em} \big((\VEC{w},\VEC{v}_1), \ldots,
(\VEC{w},\VEC{v}_q) \big) \dx{x} \bigg) \\
&\qquad = \sum_{1\leq i_1<i_2<\ldots<i_q \leq k}
\left( \int_0^1 \pdydx{g_{i_1,i_2,\ldots,i_q}}{x}(\VEC{w},x) \dx{x} \right)
\left( \df{w_{i_1}} \wedge \df{w_{i_2}} \wedge \ldots \wedge
\df{w_{i_q}}\right) (\VEC{w}) \\
&\hspace{17em} \big((\VEC{w},\VEC{v}_1), \ldots, (\VEC{w},\VEC{v}_q) \big) \\
&\qquad = \sum_{1\leq i_1<i_2<\ldots<i_q \leq k}
\left( g_{i_1,i_2,\ldots,i_q}(\VEC{w},1) -
g_{i_1,i_2,\ldots,i_q}(\VEC{w},0) \right)
\left( \df{w_{i_1}} \wedge \df{w_{i_2}} \wedge \ldots \wedge
\df{w_{i_q}}\right) (\VEC{w}) \\
&\hspace{17em} \big((\VEC{w},\VEC{v}_1), \ldots, (\VEC{w},\VEC{v}_q) \big) \\
&\qquad = \sum_{1\leq i_1<i_2<\ldots<i_q \leq k}
\left( g_{i_1,i_2,\ldots,i_q}(\VEC{w},1) -
g_{i_1,i_2,\ldots,i_q}(\VEC{w},0) \right)
\left( \df{\tilde{w}_{i_1}} \wedge \df{\tilde{w}_{i_2}} \wedge \ldots \wedge
\df{\tilde{w}_{i_q}} \right)(\VEC{w},x) \\
&\hspace{17em} \Big( \big((\VEC{w},x),(\VEC{v}_1,0)\big), \ldots,
\big( (\VEC{w},x),(\VEC{v}_q,0)\big) \Big)\\
&\qquad = (\tilde{\iota}_1^\ast(\tilde{\omega}_1))(\VEC{w})
\big( (\VEC{w},\VEC{v}_1), \ldots, (\VEC{w},\VEC{v}_q) \big)
- (\tilde{\iota}_0^\ast(\tilde{\omega}_1))(\VEC{w})
\big( (\VEC{w},\VEC{v}_1), \ldots, (\VEC{w},\VEC{v}_q) \big) \ .
\end{align*}
Thus (\ref{i1i0dffdEq2}) is satisfied.

\stage{ii} We now prove (\ref{i1i0dffdEq3}).   We have that 
$\displaystyle \tilde{\iota}_1^\ast(\df{x} \wedge \tilde{\omega}_2) =
\tilde{\iota}_0^\ast(\df{x} \wedge \tilde{\omega}_2) = 0$ because\\
$\df{x} \big((\VEC{w},x),(\VEC{v},0)\big) = 0$ for all
$\big((\VEC{w},x),(\VEC{v},0)\big) \in \TS_{(\VEC{w},x)} (W \times J)$.
Since
\begin{align*}
&F(\df{x} \wedge \tilde{\omega}_2)(\VEC{w})
\big( (\VEC{w},\VEC{v}_1), \ldots, (\VEC{w},\VEC{v}_q) \big) \\
&\qquad =\sum_{1\leq i_1<i_2<\ldots<i_{q-1} \leq k}
\bigg( \int_0^1 h_{i_1,i_2,\ldots,i_{q-1}}(\VEC{w},x)
\big( \df{\tilde{w}_{i_1}} \wedge \df{\tilde{w}_{i_2}}
\wedge \ldots \wedge \df{\tilde{w}_{i_{q-1}}} \big) (\VEC{w},x)  \\
&\hspace{17em} \big( \big((\VEC{w},x),(\VEC{v}_1,0)\big), \ldots,
\big( (\VEC{w},x),(\VEC{v}_{q-1},0)\big) \big) \dx{x} \bigg) \\
&\qquad =\sum_{1\leq i_1<i_2<\ldots<i_{q-1} \leq k}
\left( \int_0^1 h_{i_1,i_2,\ldots,i_{q-1}}(\VEC{w},x) \dx{x} \right)
\left( \df{w_{i_1}} \wedge \df{w_{i_2}}
\wedge \ldots \wedge \df{w_{i_{q-1}}} \right) (\VEC{w}) \\
&\hspace{17em}
\big( (\VEC{w},\VEC{v}_1), \ldots, (\VEC{w},\VEC{v}_{q-1}) \big)
\end{align*}
for all $\VEC{w} \in W$ and
$(\VEC{w},\VEC{v}_i) \in \TS_{\VEC{w}} W$ with $1\leq i \leq k$,
we get
\[
F(\df{x} \wedge \tilde{\omega}_2)
= \sum_{1\leq i_1<i_2<\ldots<i_{q-1} \leq k}
\left( \int_0^1 h_{i_1,i_2,\ldots,i_{q-1}}(\cdot,x) \dx{x} \right)
\df{w_{i_1}} \wedge \df{w_{i_2}} \wedge \df{w_{i_{q-1}}} \ .
\]
It follows that
\begin{equation} \label{i1i0dffdEq4}
\begin{split}
\df{(F(\df{x} \wedge \tilde{\omega}_2))}
& = \sum_{1\leq i_1<i_2<\ldots<i_{q-1} \leq k}
\sum_{j=1}^k \left( \int_0^1 \pdydx{h_{i_1,i_2,\ldots,i_{q-1}}}{w_j}
  (\cdot,x) \dx{x} \right)\\
&\hspace{10em} \df{w_j} \wedge
\df{w_{i_1}} \wedge \df{w_{i_2}} \wedge \ldots \wedge \df{w_{i_{q-1}}} \ .
\end{split}
\end{equation}

Since
\begin{align*}
& \df{(\df{x} \wedge \omega_2)} \\
&\qquad = - \df{x} \wedge \left( \sum_{1\leq i_1<i_2<\ldots<i_{q-1} \leq k}
\sum_{j=1}^k \left( \pdydx{h_{i_1,i_2,\ldots,i_{q-1}}}{w_j}
\right)\, \df{\tilde{w}_j} \wedge
\df{\tilde{w}_{i_1}} \wedge \df{\tilde{w}_{i_2}} \wedge
\ldots \wedge \df{\tilde{w}_{i_{q-1}}} \right) \ ,
\end{align*}
we get
\begin{align*}
&F{(\df{(\df{x} \wedge \omega_2)})} (\VEC{w})
\big((\VEC{w},\VEC{v}_1), \ldots,(\VEC{w},\VEC{v}_q)\big) \\
&= -\sum_{1\leq i_1<i_2<\ldots<i_{q-1} \leq k}
\sum_{j=1}^k \bigg( \int_0^1 \pdydx{h_{i_1,i_2,\ldots,i_{q-1}}}{w_j}(\VEC{w},x)
\big( \df{\tilde{w}_j} \wedge
\df{\tilde{w}_{i_1}} \wedge \df{\tilde{w}_{i_2}} \wedge
\ldots \wedge \df{\tilde{w}_{i_{q-1}}} \big) (\VEC{w},x) \\
&\hspace{13em} \Big( \big((\VEC{w},x),(\VEC{v}_1,0)\big), \ldots,
\big( (\VEC{w},x),(\VEC{v}_q,0)\big) \Big) \dx{x} \bigg) \\
&= -\sum_{1\leq i_1<i_2<\ldots<i_{q-1} \leq k}
\sum_{j=1}^k \left( \int_0^1 \pdydx{h_{i_1,i_2,\ldots,i_{q-1}}}{w_j}(\VEC{w},x)
\dx{x} \right)\\
&\hspace{13em} \big( \df{w_j} \wedge \df{w_{i_1}} \wedge \df{w_{i_2}} \wedge
\ldots \wedge \df{w_{i_{q-1}}} \big)(\VEC{w})
\big( (\VEC{w},\VEC{v}_1), \ldots, (\VEC{w},\VEC{v}_q) \big)
\end{align*}
for all $\VEC{w} \in W$ and
$(\VEC{w},\VEC{v}_i) \in \TS_{\VEC{w}} W$ with $1\leq i \leq k$.
Thus
\begin{equation} \label{i1i0dffdEq5}
\begin{split}
F{(\df{(\df{x} \wedge \omega_2)})} &= -\sum_{1\leq i_1<i_2<\ldots<i_{q-1} \leq k}
\sum_{j=1}^k \left( \int_0^1 \pdydx{h_{i_1,i_2,\ldots,i_{q-1}}}{w_j}(\cdot,x)
\dx{x} \right) \\
&\hspace{7em} \df{w_j} \wedge \df{w_{i_1}} \wedge \df{w_{i_2}} \wedge
\ldots \wedge \df{w_{i_{q-1}}} \ .
\end{split}
\end{equation}
It follows from (\ref{i1i0dffdEq4}) and (\ref{i1i0dffdEq5}) that
$\df{(F(\df{x} \wedge \tilde{\omega}_2))} + 
F{(\df{(\df{x} \wedge \omega_2)})} = 0$.  This completes the proof of
(\ref{i1i0dffdEq3}).
\end{proof}

We may now generalize Theorem~\ref{closedexact}.

\begin{theorem}[Poincaré Lemma] \label{closedexactCntrct}
Let $S$ be a $k$-dimensional smooth manifold which is smoothly
contractible to the point $\tilde{\VEC{u}} \in S$.
Suppose that $\omega$ is a closed differential $q$-form on $S$ with $q>0$.
Then there exists a differential $(q-1)$-form $\eta$ on $S$ such that
$\df{\eta} = \omega$.
\end{theorem}

\begin{proof}
Suppose that $H:S \times [0,1] \to S$ is a map of class
$\displaystyle C^\infty$ such that $H(\VEC{u},1) = \VEC{u}$ and
$H(\VEC{u},0) = \tilde{\VEC{u}}$ for all $\VEC{u} \in S$.

Since $\Id = H \circ \iota_1:S \to S$ and
$H\circ \iota_0:S \to \{\tilde{\VEC{u}} \}$. we get that
$\displaystyle \iota_1^\ast(H^\ast(\omega))
= (H \circ \iota_1)^\ast(\omega) = \omega$ and
$\displaystyle \iota_0^\ast(H^\ast(\omega))
= (H \circ \iota_0)^\ast(\omega) = 0$.
Since $\df{\omega} = 0$, we also get that
$\displaystyle \df{(H^\ast(\omega))} = H^\ast(\df{\omega}) = 0$.
It follows from Proposition~\ref{propi1i0dffd} that
\[
\omega = \iota_1^\ast(H^\ast(\omega)) - \iota_0^\ast(H^\ast(\omega))
= \df{(F(H^\ast(\omega)))} + F(\df{(H^\ast(\omega))})
= \df{(F(H^\ast(\omega)))} \ .
\]
We may take $\displaystyle \eta = F(H^\ast(\omega))$.
\end{proof}

The following result coming out of Proposition~\ref{propi1i0dffd}
will be quite useful for our presentation of cohomology in the next
section.

\begin{prop} \label{propDeEo}
Suppose that
$\omega = g \df{x_{i_1}} \wedge \df{x_{i_2}} \wedge \ldots \wedge \df{x_{i_q}}$ 
with $1 \leq i_1 < i_2 < \ldots < i_q \leq k$ and $q>0$ is a differential
$q$-form on $\displaystyle \RR^k$, and that $\eta$ is the differential
$(q-1)$-form on $\displaystyle \RR^k$ defined by
\[
\eta(\VEC{x}) = \left(\sum_{j=1}^q (-1)^{j-1}
\left(\int_0^1 r^{q-1} g(r \VEC{x}) \dx{r}\right) x_{i_j} \right)
\, \big(\df{x_{i_1}} \wedge \ldots \wedge \widehat{\df{x_{i_j}}} \wedge \ldots
\wedge \df{x_{i_q}}\big)(\VEC{x})
\]
for $\displaystyle \VEC{x} \in \RR^k$.  Then $\df{\eta} = \omega$.
\end{prop}

\begin{proof}
The reader should review Remark~\ref{rmkWarning1} before studying this
proof.

Let $\displaystyle H: \RR^k \times \RR \to \RR^k$ be the map
defined by $H(\VEC{x},r) = r \VEC{x}$ for
$\displaystyle (\VEC{x},r) \in \RR^k \times \RR$.
If we apply Proposition~\ref{propi1i0dffd} to
$\displaystyle H^\ast(\omega)$, then we get
\begin{equation} \label{detaEomEq1}
i_1^\ast(H^\ast(\omega)) - i_0^\ast(H^\ast(\omega)) =
\df{(F(H^\ast(\omega)))} + F(\df{(H^\ast(\omega))}) \ .
\end{equation}

Let $\displaystyle Z:\RR^k \to \RR^k$ be the function defined by
$Z(\VEC{x}) = \VEC{0}$ for all $\displaystyle \VEC{x} \in \RR^k$.  We
have
\begin{equation} \label{detaEomEq2}
\iota_1^\ast(H^\ast(\omega))
= (H \circ \iota_1)^\ast(\omega) = \Id^\ast(\omega) = \omega
\end{equation}
and
\begin{equation}  \label{detaEomEq3}
\iota_0^\ast(H^\ast(\omega))
= (H \circ \iota_0)^\ast(\omega) = Z^\ast(\omega) = 0
\end{equation}
on $\displaystyle \RR^k$.

Since
\begin{align*}
&H^\ast(\df{x_j})(\VEC{x},r)\big( (\VEC{x},r), (\VEC{v}, z)\Big)
= \df{x_j} (H(\VEC{x},r)) \big( H_\ast\big((\VEC{x},r),(\VEC{v},z)
\big)\big) \\
&\qquad
= \df{x_j} (r\VEC{x}) \big(H(\VEC{x},r),\diff H(\VEC{x},r) (\VEC{v},z)\big)
= \df{x_j} (r\VEC{x}) \big(r\VEC{x}),r \VEC{v} + z \VEC{x}\big)
= r v_j + z x_j \\
&\qquad = r \df{x_j}(\VEC{x})(\VEC{x},\VEC{v}) + x_j \df{r}(r)(r,z)
\end{align*}
for all $\displaystyle \big( (\VEC{x},r), (\VEC{v}, z)\big)
\in \TS_{(\VEC{x},r)} (\RR^k \times \RR)$ and
$\displaystyle (\VEC{x},r) \in \RR^k \times \RR$ with
$1 \leq j\leq k$, we may write that
\[
H^\ast(\df{x_j})(\VEC{x},r) = r \df{\tilde{x}_j}(\VEC{x},r)
+ x_j \df{\tilde{r}}(\VEC{x},r) \ ,
\]
where $\df{\tilde{x}_j} (\VEC{x},r) \big( (\VEC{x},r), (\VEC{v}, z)\big)
= \df{x_j} (\VEC{x}) \big(\VEC{x},\VEC{v}\big) = \VEC{v}_j$ and \\
$\df{\tilde{r}} (\VEC{x},r) \big( (\VEC{x},r), (\VEC{v}, z)\big)
= \df{r} (r)\big(r,z\big) = z$ for all
$\displaystyle \big( (\VEC{x},r), (\VEC{v}, z)\big)
\in \TS_{(\VEC{x},r)}(\RR^k \times \RR)$ and
$\displaystyle (\VEC{x},r) \in \RR^k \times \RR$ with
$1 \leq j\leq k$.
Hence
\begin{align*}
&H^\ast(\omega)
= (g\circ H) \left( r \df{\tilde{x}_{i_1}}
+ x_{i_1} \df{\tilde{r}} \right) \wedge \ldots \wedge
\left( r \df{\tilde{x}_{i_q}} + x_{i_q} \df{\tilde{r}} \right) \\
&\ = r^q g(r\VEC{x}) \, \df{\tilde{x}_{i_1}} \wedge \ldots \df{\tilde{x}_{i_q}}
+ r^{q-1} g(r\VEC{x}) \left( \sum_{j=1}^q
{-1}^{j-1} x_{i_j} \, \df{\tilde{r}} \wedge \df{\tilde{x}_{i_1}} \wedge
\ldots \wedge \widehat{\df{\tilde{x_{i_j}}}} \wedge \ldots \wedge
\df{\tilde{x}_{i_q}} \right) \\
&\ = r^q g(r\VEC{x}) \, \df{\tilde{x}_{i_1}} \wedge \ldots \df{\tilde{x}_{i_q}}
+ \df{\tilde{r}} \wedge \left(
\sum_{j=1}^q {-1}^{j-1} r^{q-1} g(r\VEC{x})  x_{i_j} \,\df{\tilde{x}_{i_1}} \wedge
\ldots \wedge \widehat{\df{\tilde{x_{i_j}}}} \wedge \ldots \wedge
\df{\tilde{x}_{i_q}} \right)
\end{align*}
on $\displaystyle \RR^k \times \RR$.  It follows from
(\ref{defFSItoS}) that
\begin{equation} \label{detaEomEq4}
F(H^\ast(\omega))
= \sum_{j=1}^q {-1}^{j-1} \left(\int_0^1 r^{q-1} g(r\VEC{x}) \dx{r}\right) x_{i_j}
\, \df{x_{i_1}} \wedge \ldots \wedge \widehat{\df{x_{i_j}}} \wedge \ldots \wedge
\df{x_{i_q}} = \eta
\end{equation}
on $\displaystyle \RR^k$.
Moreover
$\displaystyle \df{(H^\ast(\omega))} = H^\ast(\df{\omega}) = H^\ast(0)
= 0$ because $\omega$ is a differential $k$-form on $\displaystyle \RR^k$.
Thus
\begin{equation}  \label{detaEomEq5}
F(\df{(H^\ast(\omega))}) = 0 \ .
\end{equation}

If we substitute (\ref{detaEomEq2}), (\ref{detaEomEq3}),
(\ref{detaEomEq4}) and (\ref{detaEomEq5}) into (\ref{detaEomEq1}),
then we get $\df{\eta} = \omega$.
\end{proof}

\section{Cohomology Modules}

We will start with a more restrictive type of cohomology before moving
on to the more general type.  This will prepare us for the more general
type of cohomology by giving us some useful results for the study of
the more general type of cohomology.  This more restrictive type of
cohomology is really important in its own right.

\subsection{Cohomology Modules with Compact Supports}

\begin{defn} \label{defnCohomRelCompact}
Let $S$ be a $k$-dimensional smooth manifold.
Two closed differential $q$-forms with compact support, $\omega_1$ and
$\omega_2$, on $S$ are {\bfseries cohomologous}\index{Cohomologous Forms}
if there exists a differential $(q-1)$-form with compact support
$\eta$ on $S$ such that 
$\omega_1 - \omega_2 = \df{\eta}$.  We write $\omega_1 \sim \omega_2$.
\end{defn}

In the previous definition, the condition that the support of the
differential $(q-1)$-form $\eta$ be compact is stronger than the
condition that the support of $\df{\eta}$ be compact.  The
differential $q$-form $\df{\eta}$ may have a compact support without the
differential $(q-1)$-form $\eta$ having a compact support.

The relation defined above is an equivalence relation.
\begin{enumerate}
\item If $\omega$ is a closed differential $q$-form with compact
support, then obviously $\omega \sim \omega$.
\item If $\omega_1$ and $\omega_2$ are two closed differential
$q$-forms with compact support, then again it is obvious that
$\omega_1 \sim \omega_2$ if and only if $\omega_2 \sim \omega_1$.
\item If $\omega_i$ for $1\leq i \leq 3$ are three closed differential
$q$-forms with compact support such that $\omega_1 \sim \omega_2$.
and $\omega_2 \sim \omega_3$, then $\omega_1 \sim \omega_3$.  To
verify this claim, let $\eta_1$ be a differential $(q-1)$-form with
compact support such that $\omega_2 - \omega_1 = \df{\eta_1}$
and $\eta_2$ be a differential $(q-1)$-form with compact support such that
$\omega_3 - \omega_2 = \df{\eta_2}$.  Then
$\omega_3 - \omega_1 = (\omega_3 - \omega_2) + (\omega_2 - \omega_1)
= \df{\eta_2} + \df{\eta_1} = \df{(\eta_2+\eta_2)}$ where
$\eta_1 + \eta_2$ is a differential $(q-1)$-form with compact support
because the union of two compact sets is a compact set.
\end{enumerate}

\begin{defn}
Let $S$ be a $k$-dimensional smooth manifold.
The {\bfseries $\displaystyle \mathbf{q^{th}}$ (de Rham) Cohomology Module
with compact support}\index{Cohomology Module with Compact
Support}\index{(de Rham) Cohomology Module with Compact
Support|see{Cohomology Module with Compact Support}} of $S$,
denoted $\displaystyle H_c^q(S)$, is the set of all equivalent
classes of closed differential $q$-forms on $S$ with compact support
with respect to the equivalence relation given in
Definition~\ref{defnCohomRelCompact}.
The equivalence class for a closed differential $q$-form with compact
support $\omega$ on $S$ is called a
{\bfseries cohomology class}\index{Cohomology Class} and
is denoted $[\omega]$.
\end{defn}

\begin{egg}
We have that the dimension of $\displaystyle H^k_c(\RR^k)$ is greater
than $0$.

To prove this statement, let
$\displaystyle \omega = f \df{x_1} \wedge
\df{x_2} \wedge \ldots \wedge \df{x_k}$ where
$\displaystyle f:\RR^k \to [0,\infty[$ is
a continuous function with compact support such that $f(\VEC{x}) > 0$
for some $\displaystyle \VEC{x} \in \RR^k$.  Then
$\displaystyle \int_{\RR^k} \omega = \int_{\RR^k} f(\VEC{x}) \dx{\VEC{x}} > 0$.
Suppose that there is a differential $(k-1)$-form $\eta$ on $S$ with
compact support such that $\df{\eta} = \omega$.  Given $r >0$ such
that $\supp \eta \subset B_r(\VEC{0})$, we then have from
Stokes' theorem that
$\displaystyle \int_{\RR^k} \omega = \int_{B_r(\VEC{0})} \omega 
= \int_{B_r(\VEC{0})} \df{\eta} = \int_{\partial B_r(\VEC{0})} \eta = 0$.
This is a contradiction.
\end{egg}

\begin{egg}
If $S$ is an oriented $k$-dimensional         \label{eggOCSpos}
smooth manifold, then the dimension of $\displaystyle H_c^k(S)$ is
greater than $0$.

To prove this statement, let $(W,U,\phi)$ be an orientation preserving
local chart of $S$ such that $U \cap \partial S = \emptyset$.
Let $\omega$ be a differential $k$-form on $S$ with compact support in $U$
such that its local representation is
$\displaystyle \phi^\ast(\omega) = f \df{w_1} \wedge
\df{w_2} \wedge \ldots \wedge \df{w_k}$ where $f:W \to [0,\infty[$ is
a continuous function with compact support and $f(\VEC{w}) > 0$
for some $\VEC{w} \in W$.  Then
$\displaystyle \int_S \omega = \int_W f(\VEC{w}) \dx{\VEC{w}} > 0$.
Suppose that there is a differential $(k-1)$-form $\eta$ on $S$ with
compact support such that $\df{\eta} = \omega$.  We must have that
$\supp \eta \subset \supp \omega \subset U$ because
$\df{\eta}(\VEC{u}) = \omega(\VEC{u}) = 0$ for
$\VEC{u} \in S \setminus \supp \omega$.  Thus
$\eta(\VEC{u})$ is constant for $\VEC{u} \in S \setminus \supp \omega$.
Since $\eta$ has a compact support, we must have that
$\eta(\VEC{u}) = 0$ for $\VEC{u} \in S \setminus \supp \omega$.
In particular, we have that
$\supp \eta \cap \partial S \subset U \cap \partial S = \emptyset$.
It follows from Stokes' theorem that $\displaystyle \int_S \omega 
= \int_S \df{\eta} = \int_{\partial S} \eta = 0$.  This is a
contradiction.
\end{egg}

\begin{prop}  \label{HCqVectSpac}
Suppose that $S$ is a $k$-dimensional smooth manifold.
Then $\displaystyle H_c^q(S)$ is a vector space.
\end{prop}

\begin{proof}
\stage{i} We prove that $[\omega_1 + \omega_2] =
[\omega_1]+[\omega_2]$ is well defined for all closed differential $q$-forms
with compact support $\omega_1$ and $\omega_2$ on $S$.

We first note that $\omega_1 + \omega_2$ is closed because
$\df{(\omega_1 + \omega_2)} = \df{\omega_1} + \df{\omega_2} = 0$.
Moreover, $\omega_1 + \omega_2$ has a compact support because the
union of two compact sets is a compact set.

Suppose that $\tilde{\omega}_i \sim \omega_i$ for $i=1,2$.
Then there exist two differential $(q-1)$-forms with compact support,
$\eta_1$ and $\eta_2$, on $S$ such that
$\tilde{\omega}_i - \omega_i = \df{\eta_i}$ for $i=1,2$.  It follows that
$(\tilde{\omega}_1 + \tilde{\omega}_2) - (\omega_1 + \omega_2)
= (\tilde{\omega}_2 - \omega_2) + (\tilde{\omega}_1 - \omega_1)
= \df{\eta_2} + \df{\eta_1} = \df{(\eta_2+\eta_1)}$.
Moreover, $\eta_1 + \eta_2$ has a compact support because the union of two
compact sets is a compact set.
Thus $(\tilde{\omega}_1 + \tilde{\omega}_2) \sim (\omega_1 + \omega_2)$
and $[\tilde{\omega}_1 + \tilde{\omega}_2] =[\omega_1 + \omega_2]$.

\stage{ii} We prove that $a [\omega] = [a\omega]$ is well defined for
all differential $q$-form with compact support $\omega$ on $S$ and
$a \in \RR$.

We first note that $a \omega$ is closed because
$\df{(a\omega)} = a\df{\omega} = 0$.  Moreover, $a \omega$ has a
compact support because $\supp (a\omega) = \supp \omega$ if $a \neq 0$.

Suppose that $\tilde{\omega} \sim \omega$.
Then there exists a differential $(q-1)$-form with compact support
$\eta$ on $S$ such that $\tilde{\omega} - \omega = \df{\eta}$.  It
follows that $a\,\tilde{\omega} + a\,\omega = a(\tilde{\omega} - \omega)
= a \df{\eta} = \df{(a\eta)}$ where $a\eta$ is a differential
$(q-1)$-form with compact support. Thus
$a\,\tilde{\omega} \sim a\,\omega$ and $[a\,\tilde{\omega}] =[a\,\omega]$.
\end{proof}

\begin{rmk}
Let $S$ be a $k$-dimensional smooth manifold.
The null element $[0]$ in $\displaystyle H_c^q(S)$ is the collection of
all differential $q$-forms with compact support $\omega$ on $S$ such
that $\omega = \df{\eta}$ for some differential $(q-1)$-form with
compact support $\eta$ on $S$.  This follows from the fact that 
$\omega$ in the equivalence class $[0]$ means that
$\omega = \omega - 0 = \df{\eta}$ for some differential $(q-1)$-form
with compact support on $S$.
\end{rmk}

We want to compute the $\displaystyle q^{th}$ cohomology modules with
compact support for the connected and oriented smooth manifolds.  To
do that will however require some work.

\begin{prop} \label{propHkSMequiv}
Suppose that $S$ is a connected and oriented $k$-dimensional smooth
manifold without boundary.  Then $\displaystyle H_c^k(S) \cong \RR$ if
and only if the map $\displaystyle M : H_c^k(S) \to \RR$ defined by
$M([\omega]) = \int_S \omega$ is an isomorphism.
\end{prop}

\begin{proof}
We first have to prove that the map $M$ is well defined.
Suppose that $[\omega_1] = [\omega_2]$ for two differential $k$-forms
$\omega_1$ and $\omega_2$ on $S$ with compact support.  Then there
exists a differential $(k-1)$-form on $S$ with compact support such
that $\omega_1 - \omega_2 = \df{\eta}$.  It follows from Stokes'
theorem that
$\displaystyle \int_S \omega_1 - \int_S \omega_2 = \int_S (\omega_1 - \omega_2)
= \int_S \df{\eta} = \int_{\partial S} \eta = 0$.  So
$\displaystyle \int_S \omega_1 = \int_S \omega_2 =0 $ and thus
$M([\omega_1]) = M([\omega_2])$.

Suppose that $M$ is an isomorphism between $\displaystyle H_c^k(S)$
and $\RR$, and that $\omega_1$ and $\omega_2$ are two differential
$k$-forms on $S$ with compact support.  Choose $a \in \RR$ such that
$M([\omega_1]) = a M([\omega_2])$.  Then
$M([\omega_1] - a [\omega_2]) = M([\omega_1]) - a M([\omega_2]) = 0$
Since $M$ is an isomorphism, we have that
$[\omega_1] - a [\omega_2] = [0]$; namely,
$[\omega_1] = a [\omega_2]$.

Conversely, suppose that $\displaystyle H_c^k(S) \cong \RR$ and that
$M([\omega_1]) = 0$ with $[\omega_1] \neq [0]$, where $\omega_1$
is a differential $k$-forms on $S$ with compact support.
For every differential $k$-forms $\omega_2$ on $S$ with compact
support, we have that $[\omega_2] = a [\omega_1]$ for some $a \in \RR$.
Thus $M([\omega_2]) = a M([\omega_1]) = 0$.
It follows that $M$ is null on $\displaystyle H_c^k(S)$ which is not
the case.  An example of a differential $k$-form $\omega$ with
$M([\omega])>0$ is given in Example~\ref{eggOCSpos}.
\end{proof}

\begin{lemma} \label{ConnSequ}
Suppose that $S$ is a connected set and that
$\U = \{ U_\alpha \}_{\alpha \in A}$ is an open cover of $S$ with
$U_\alpha \subset S$ for all $\alpha \in I$.
Given any two sets $U_a, U_b \in \U$, there
exists a finite collection $\{ U_{\alpha_j} \}_{0 \leq j \leq J}$ such that
$U_{\alpha_0} = U_a$, $U_{\alpha_J} = U_b$ and
$U_{\alpha_j} \cap U_{\alpha_{j-1}} \neq \emptyset$ for $1 \leq j \leq J$.
\end{lemma}

\begin{proof}
Let $\BB$ be the set of all $U_c \in \U$ such that there exists a
finite collection $\{ U_{\alpha_j} \}_{0 \leq j \leq J_c}$ satisfying
$U_{\alpha_0} = U_a$, $U_{\alpha_{J_c}} = U_c$ and
$U_{\alpha_j} \cap U_{\alpha_{j-1}} \neq \emptyset$ for $1 \leq j \leq J_c$.

Let $\displaystyle C = \bigcup_{U_c \in \BB} U_c$.  We have that $C$ is
an open set because it is the union of open sets.  We now show that
$S \setminus C = \emptyset$ and thus that $U_b \in \BB$.

Suppose that $S \setminus C \neq \emptyset$.  Given any
$\VEC{x} \in S \setminus C$, there exists $U_d \in \U$ such that
$\VEC{x} \in U_d$ because $\U$ is an open cover of $S$.  We have that
$U_d \cap C = \emptyset$ otherwise $W_d$ would be in $\BB$ and
$\VEC{x}$ in $C$.  To be more precise, 
$U_d \cap C \neq \emptyset$ implies that $U_d \cap U_c \neq \emptyset$
for some $U_c \in \BB$.  Thus $U_d$ may be added to the finite collection
$\{ U_{\alpha_j} \}_{0 \leq j \leq J_c}$ joining $U_a$ to $U_c$ 
with $U_{\alpha_j} \cap U_{\alpha_{j-1}} \neq \emptyset$ for $1 \leq j \leq J_c$
to create a finite collection $\{ U_{\alpha_j} \}_{0 \leq j \leq J_c+1}$
joining $U_a$ to $U_d$ with $U_{\alpha_{J_c+1}} = U_d$ and
$U_{\alpha_j} \cap U_{\alpha_{j-1}} \neq \emptyset$ for $1 \leq j \leq J_c+1$.
Thus $S\setminus C$ is a non-empty open set.
Since $S$ is connected, this contradict the fact that $S$ cannot be
the union of two disjoints non-empty open sets.
\end{proof}

\begin{theorem} \label{theoHckCR}
If $S$ is an connected and oriented $k$-dimensional smooth manifold without
boundary, then $\displaystyle H_c^k(S) \cong \RR$.
\end{theorem}

\begin{proof}
\stage{i} We first prove that $\displaystyle H_c^1(\RR) \cong \RR$.
To do so, we prove that the map $\displaystyle M:H_c^1(\RR) \to \RR$
defined in Proposition~\ref{propHkSMequiv} is an isomorphism.
Suppose that $\omega = g \df{x}$ is a differential $1$-form on $S$
with compact support such that $\displaystyle \int_{\RR} \omega = 0$.
In particular, $g:\RR\to \RR$ is a continuous function with compact
support.

There exists a function $f:\RR\to \RR$ such that $\omega = \df{f}$.  In
fact, it suffices to take $\displaystyle f(x) = \int_0^x g(y)\dx{y}$ for
$x \in \RR$.

Since $\omega = \df{f} = f' \dx{x}$ has a compact support, there exists an
interval $[a,b] \subset \RR$ such that $f'(x) = 0$ for
$x \in \RR \setminus [a,b]$.
Thus $f$ is constant on $\RR \setminus [a,b]$.  Suppose that 
$f(x) = c$ for $x <a$ and $f(x) = d$ for $x>b$.  Then $c = d$ because
$\displaystyle 0 = \int_{\RR} \omega = \int_{\RR} f'(x) \dx{x}
= \int_a^b f'(x) \dx{x} = f(b) - f(a) = d - c$.

We get that $\omega = \df{h}$ for $h = f - c$ with
$\supp h \subset [a,b]$.  Thus $[\omega] = [0]$ in 
$\displaystyle H^1(\RR)$.

\stage{ii} We prove that $\displaystyle H_c^{k-1}(S) \cong \RR$
for all connected and oriented $(k-1)$-dimensional smooth manifolds without
boundary implies that $\displaystyle H_c^k(\RR^k) \cong \RR$.  Again, we
do so by proving that $\displaystyle M:H_c^k(\RR) \to \RR$
defined in Proposition~\ref{propHkSMequiv} is an isomorphism.

Suppose that $\omega$ is a differential $k$-form on $\displaystyle \RR^k$
with compact support and that $\displaystyle \int_{\RR^k} \omega = 0$.
Since $\supp \omega$ is compact, there exists $b>0$ such that
$\supp \omega \subset B_b(\VEC{0})$.

Let $\displaystyle h:\RR^k \to \RR^k$ be the diffeomorphism defined by
$h(\VEC{x}) = b\, \VEC{x}$ for $\displaystyle \VEC{x} \in \RR^k$.  We have that
$\displaystyle h^\ast(\omega)$ is a differential $k$-form on
$\displaystyle \RR^k$ with
$\displaystyle \supp\, h^\ast(\omega) \subset B_1(\VEC{0})$.
If we prove that there exists a differential $(k-1)$-form $\eta$ on
$\displaystyle \RR^k$ with compact support such that
$\displaystyle h^\ast(\omega) = \df{\eta}$, then
$\displaystyle \omega = (h^{-1})^\ast(\df{\eta})
= \df{\left( (h^{-1})^\ast(\eta)\right)}$ where
$\displaystyle (h^{-1})^\ast(\eta)$ is a differential $(k-1)$-form $\eta$
on $\displaystyle \RR^k$ with compact support.  Thus
$[\omega] = [0]$ in $\displaystyle H_c^k(\RR^k)$.

Therefore, we may assume without loss of generality that
$\omega = g \df{x_1} \wedge \df{x_2} \wedge \ldots \wedge \df{x_k}$ 
with $\supp g \subset B_1(\VEC{0})$.

Let $\eta$ be the differential $(k-1)$-form on $\displaystyle \RR^k$ defined by
\begin{equation} \label{theoHckCREq1}
\eta(\VEC{x}) = \sum_{j=1}^k (-1)^{j-1}
\left(\int_0^1 r^{k-1} g(r \VEC{x}) \dx{r}\right) 
x_j \left( \df{x_1} \wedge \ldots \wedge \widehat{\df{x_j}} \wedge \ldots
\wedge \df{x_k}\right)(\VEC{x})
\end{equation}
for $\displaystyle \VEC{x} \in \RR^k$.
We have from Proposition~\ref{propDeEo} that $\df{\eta} = \omega$.  However,
$\eta$ may not have a compact support.  We have to slightly modify
$\eta$ to get a differential $(k-1)$-form on $S$ with compact support
whose derivative is $\omega$.

If we use the substitution $t = \|\VEC{x}\| r$ in the integral in
(\ref{theoHckCREq1}) and use Proposition~\ref{propRanEntn}, then we get
\begin{align*}
\eta(\VEC{x}) &= \sum_{j=1}^k (-1)^{j-1}
\left(\int_0^{\|\VEC{x}\|} t^{k-1} g\left(\frac{t}{\|\VEC{x}\|} \VEC{x}\right)
\dx{t}\right) \|\VEC{x}\|^{-k} x_j \left(\df{x_1}
\wedge \ldots \wedge \widehat{\df{x_j}} \wedge \ldots
\wedge \df{x_k}\right)(\VEC{x}) \\
&= \left(\int_0^1 t^{k-1} g\left(\frac{t}{\|\VEC{x}\|} \VEC{x}\right)
\dx{t}\right)
\left( \sum_{j=1}^k (-1)^{j-1} \|\VEC{x}\|^{-k} x_j \left(\df{x_1}
\wedge \ldots \wedge \widehat{\df{x_j}} \wedge \ldots
\wedge \df{x_k} \right)(\VEC{x}) \right) \\
&= \left( \int_0^1 t^{k-1} g\left(\frac{t}{\|\VEC{x}\|} \VEC{x}\right)
\dx{t}\right) \, \rho^\ast(\nu)(\VEC{x})
\end{align*}
for $\displaystyle \VEC{x} \in \RR^k \setminus \{\VEC{0}\}$ where
$\displaystyle \rho:\RR^k \setminus \{\VEC{0}\} \to \SS^{k-1}$ is the
retraction defined by $\displaystyle \rho(\VEC{x}) = \|\VEC{x}\|^{-1} \VEC{x}$ 
and $\nu$ is the volume element on $\displaystyle S^{k-1}$.
We have used the fact that $g(\VEC{x}) = 0$ for $\|\VEC{x}\| > 1$ to
modify the upper limit of integration.

Let $\displaystyle G:\RR^k \to \RR$ be the function defined by
$\displaystyle G(\VEC{x}) = \int_0^1 t^{k-1} g(t \VEC{x}) \dx{t}$
for $\displaystyle \VEC{x} \in \RR^k$.  We have that
$\displaystyle \eta(\VEC{x}) = G(\rho(\VEC{x})) \, \rho^\ast(\nu)(\VEC{x})
= \rho^\ast(G \, \nu)(\VEC{x})$ for all
$\displaystyle \VEC{x} \in \RR^k \setminus \{\VEC{0}\}$.  Moreover, we get from
Proposition~\ref{propPolCoords} that
\[
\int_{S^{k-1}} G\, \nu =
\int_{B_1(\VEC{0})} g \, \df{x_1} \wedge \df{x_2} \wedge \ldots
\wedge \df{x_n} = \int_{\RR^k} \omega = 0 \ .
\]

Using the induction hypothesis with $\displaystyle S^{k-1}$, we have
that $\displaystyle M: H_c^{k-1}(S^{k-1}) \to \RR$ defined by
$\displaystyle M([\omega]) = \int_{S^{k-1}} \omega$ is an isomorphism.
So $[G\, \nu] = [0]$ in $\displaystyle H_c^{k-1}(S^{k-1})$; namely,
there exists a differential $(k-2)$-form $\xi$ on
$\displaystyle S^{k-1}$ with compact support such that $G \, \nu = \df{\xi}$
on $\displaystyle S^{k-1}$.  It follows that
$\displaystyle \eta = \rho^\ast(G\, \nu) = \rho^\ast(\df{\xi}) =
\df{\left(\rho^\ast(\xi)\right)}$ on
$\displaystyle \RR^k \setminus \{\VEC{0}\}$.

Choose $0 < \epsilon < \delta <1$ and let
$\displaystyle h:\RR^k \to \RR$ be a function of class
$\displaystyle C^\infty$ such that $h(\VEC{x}) = 1$ for
$\|\VEC{x}\| \geq \delta$ and $h(\VEC{x}) = 0$ for $\|\VEC{x}\| < \epsilon$.
Then $\displaystyle h \rho^\ast(\xi)$ is a differential $(k-2)$-form on
$\displaystyle \RR^k\setminus \{\VEC{0}\}$.
Since $\displaystyle h \rho^\ast(\xi)$ is null in a neighbourhood of the
origin, we may assume that $\displaystyle h \rho^\ast(\xi)$ is defined on
$\displaystyle \RR^k$ by setting $\displaystyle h \rho^\ast(\xi)(\VEC{0}) = 0$.

We have that $\displaystyle \eta - \df{(h \rho^\ast(\xi))}$ 
is a differential $(k-2)$-form on $\displaystyle \RR^k$ such that\\
$\displaystyle \df{\left(\eta - \df{(h \rho^\ast(\xi))}\right)}
= \df{\eta} = \omega$ and
\begin{align*}
\left(\eta - \df{(h \rho^\ast(\xi))}\right)(\VEC{x})
&= \eta(\VEC{x}) - \df{h}(\VEC{x}) \wedge (h \rho^\ast(\xi))(\VEC{x})
- h(\VEC{x}) \df{(\rho^\ast(\xi))}(\VEC{x}) \\
&= (1 - h(\VEC{x})) \eta(\VEC{x}) - \df{h}(\VEC{x}) \wedge (h
\rho^\ast(\xi))(\VEC{x})  = 0
\end{align*}
for $\|\VEC{x}\| >\delta$ because $h(\VEC{x}) = 1$ for $\|\VEC{x}\|>\delta$.
So $\displaystyle \eta - \df{(h \rho^\ast(\xi))}$ has a compact support
in $B_1(\VEC{0})$.

This shows that $[\omega] = [0]$ in $\displaystyle H_c^k(\RR^k)$.

\stage{iii} We prove that $\displaystyle H_c^k(\RR^k) \cong \RR$
implies $\displaystyle H_c^k(S) \cong \RR$
for all connected and oriented $k$-dimensional smooth manifolds without
boundary.

Let $\displaystyle \A = \left\{ (W_\alpha,U_\alpha,
\phi_\alpha)\right\}_{\alpha\in A}$ be
an atlas of orientation preserving local charts on $S$ and
$\displaystyle \{\psi_j\}_{j \in \NNp}$ be a partition of
unity subordinate to $\A$.  Without any loss of generality, we may
assume that the $W_i$ are open ball in $\displaystyle \RR^k$ centred
at the origin
\footnote{For each $\VEC{x}$, there exists an orientation preserving
local chart $(W,U,\phi)$ on $S$ with $\VEC{x} \in U$.  Choose a small ball
$\displaystyle W_{\VEC{x}} = B_r(\phi^{-1}(\VEC{x})) \subset W$.  Let
$U_{\VEC{x}} = \phi(B_{\VEC{x}})$ and $\phi_{\VEC{x}} = \phi$.  We
get a new local chart $(W_{\VEC{x}},U_{\VEC{x}},\phi_{\VEC{x}})$ on $S$ 
about $\VEC{x} \in S$.  The collection
$\displaystyle \{(W_{\VEC{x}},U_{\VEC{x}},\phi_{\VEC{x}})\}_{\VEC{x} \in S}$
is an atlas on $S$ which orientation preserving local charts such
that the $W_{\VEC{x}}$ are open ball in $\displaystyle \RR^k$.}.

Choose a local chart $(W_{\tau_0},U_{\tau_0},\phi_{\tau_0})$ with
$\tau_0 \in A$ and a differential $k$-form $\omega$ on $S$ with
compact support in $U_{\tau_0}$ such that $[\omega] \neq [0]$ in
$\displaystyle H_c^k(S)$; namely, $\omega$ is not exact.  Given
another differential
$k$-form $\tilde{\omega}$ on $S$ with compact support, we want to show
that there exists $a \in \RR$ such that $[\tilde{\omega}] = a [\omega]$;
namely, there exist $a \in \RR$ and a differential $(k-1)$ form $\eta$
on $S$ with compact support such that
$\tilde{\omega} - a \omega = \df{\eta}$.

Since $\tilde{\omega}$ has a compact support, it follows from
Proposition~\ref{cov4forM} that there exists a finite subset $J$ of
$\displaystyle \NNp$ such that
$\displaystyle \tilde{\omega} = \sum_{j\in J} \psi_j\, \tilde{\omega}$.
If we prove that for each $j \in J$ we can find $a_j \in \RR$ and
a differential $(k-1)$-form $\eta_j$ on $S$ with compact support such that
$\psi_j\, \tilde{\omega} - a_j \omega = \df{\eta_j}$, then we will have
\[
\tilde{\omega} = \sum_{j\in J} \psi_j \tilde{\omega}
= \bigg(\underbrace{\sum_{j\in J} a_j}_{= a}\bigg) \omega + \df{
\bigg(\underbrace{\sum_{j\in J} \eta_j}_{= \eta}\bigg)} \ ,
\]
where $\eta$ is a differential $(k-1)$-form on $S$ with compact support
because each $\eta_j$ has a compact support and the finite union of
compact sets is a compact set.

Therefore, we only have to prove that for each $j \in J$ we can find
$a_j \in \RR$ and a differential $(k-1)$-form $\eta_j$ on $S$ with compact
support such that $\psi_j\, \tilde{\omega} - a_j \omega = \df{\eta_j}$.
If $[\psi_j\, \tilde{\omega}] = [0]$ in $\displaystyle H_c^k(S)$, then
$\psi_j\, \tilde{\omega} = \df{\eta_j}$ for some differential
$(k-1)$-form $\eta_j$ on $S$ with compact support.  Thus, there is
nothing to prove.  We may take $a_j=0$.  We assume that
$[\psi_j\, \tilde{\omega}] \neq [0]$ in $\displaystyle H_c^k(S)$.

Using Lemma~\ref{ConnSequ}, we can find a finite sequence
$\displaystyle \{U_{\tau_m} \}_{0 \leq m \leq M}$ with $\tau_m \in A$
and $U_{\tau_{m-1}} \cap U_{\tau_m} \neq \emptyset$ for $1 \leq m \leq M$
where $U_{\tau_0}$ was defined above and $U_{\tau_M} = U_{\alpha_j}
\supset \supp \psi_j \supset \supp \psi_j\, \tilde{\omega}$.
Such an index $\alpha_j \in A$ exists by definition of a partition of
unity.
\pdfbox{cohomology/sequSets}

Let $\omega_0 = \omega$.  For $0 \leq m < M$,
choose a differential $k$-form $\omega_{m+1}$ on $S$ with compact support in
$U_{\tau_m} \cap U_{\tau_{m+1}}$ and such that
$M([\omega_{m+1}]) = \int_S\omega_{m+1} > 0$ (see
Example~\ref{eggOCSpos}).  This implies that 
$\displaystyle [\phi_{\tau_m}^\ast(\omega_{m+1})] \neq [0]$ in
$\displaystyle H_c^k(\RR^k)$ \footnote{Since 
$\displaystyle \supp \phi_{\tau_m}^\ast(\omega_{m+1}) \subset W_{\tau_m}$,
we may assume that $\displaystyle \phi_{\tau_m}^\ast(\omega_{m+1})$ is
defined on $\displaystyle \RR^k$ by setting it to $0$ outside $W_{\tau_m}$.}
Since we assume that
$\displaystyle H_c^k(\RR^k) \cong \RR$, there exists $a_{m+1} \in \RR$
and a differential $(k-1)$-form $\tilde{\xi}_m$ on
$\displaystyle \RR^k$ with compact
support such that $\displaystyle \phi_{\tau_m}^\ast(\omega_{m+1}) - a_{m+1}
\phi_{\tau_m}^\ast(\omega_m) = \df{\tilde{\xi}_m}$.  Proceeding as in the
prove of (ii), since $\displaystyle \supp \phi_{i_m}^\ast(\omega_{m+1})$
is a subset of the open ball $W_{\tau_m}$ centred at the origin, we may modify
$\tilde{\xi}_m$ to get that $\supp \tilde{\xi}_m \subset W_{\tau_m}$.
We may therefore define a differential $(k-1)$-form $\xi_m$ on $S$
with compact support in $W_{\tau_m}$ by $\xi_m = 0$ on
$S \setminus U_{\tau_m}$ and
$\displaystyle \phi_{\tau_m}^\ast(\xi_m) = \tilde{\xi}_m$ on $U_{\tau_m}$.
Hence, $\displaystyle \df{\tilde{\xi}_m} =
\df{\big(\phi_{\tau_m}^\ast(\xi_m)\big)}
= \phi_{\tau_m}^\ast(\df{\xi_m})$ and we get
\begin{equation} \label{theoHckCREq2}
\omega_{m+1} - a_{m+1} \omega_m = \df{\xi_m}
\end{equation}
for $0\leq m < M$.

For $m = M$, we consider the differential $k$-form
$\displaystyle \omega_{M+1} =  \psi_j \tilde{\omega}$
which has a compact support in $U_{\tau_M} = U_{\alpha_j}$.  We proceed
exactly as in the previous paragraph to 
find $a_{M+1} \in \RR$ and a differential $(k-1)$-form $\xi_M$ on $S$
with compact support in $U_{\tau_M}$ such that
\begin{equation} \label{theoHckCREq3}
\omega_{M+1} - a_{M+1} \omega_M = \df{\xi_M} \ .
\end{equation}

Using induction, it follows from (\ref{theoHckCREq2}) and
(\ref{theoHckCREq3}) that
\begin{align*}
\psi_j \tilde{\omega} &= \omega_{M+1} = a_{M+1} \omega_M + \df{\xi_M}
= a_{M+1} \left(a_M \omega_{M-1} + \df{\xi_{M-1}} \right) + \df{\xi_M} \\
&= a_{M+1} a_M \omega_{M-1} + \left( a_{M+1} \df{\xi_{M-1}} + \df{\xi_M} \right)
\\
&= a_{M+1} a_M \left(a_{M-1} \omega_{M-2} + \df{\xi_{M-2}}\right)
+ \left( a_{M+1} \df{\xi_{M-1}} + \df{\xi_M} \right) \\
&= a_{M+1} a_M a_{M-1} \omega_{M-2} + \left( a_{M+1}a_M \df{\xi_{M-2}}
+ a_{M+1} \df{\xi_{M-1}} + \df{\xi_M} \right) \\
&= \ldots \\
&= \left(\prod_{i=1}^{M+1} a_i\right) \omega_0 +
\df{\left(\xi_M + \sum_{i=0}^{M-1} \left(\prod_{m=i+2}^{M+1} a_m\right)
\xi_i \right)} \\
&= \underbrace{\left(\prod_{i=1}^{M+1} a_i\right)}_{=a_j} \omega +
\df{\underbrace{
\left(\xi_M + \sum_{i=0}^{M-1} \left(\prod_{m=i+2}^{M+1} a_m\right)
\xi_i \right)}_{=\eta_j}} \ ,
\end{align*}
where $\eta_j$ a differential $(k-1)$-form on $S$ with compact support
because each $\xi_j$ has a compact support and the finite union of
compact sets is a compact set.

\stage{iv} We finally have that
(i) and (iii) implies that $\displaystyle H_c^1(S) \cong \RR$
for all connected and oriented $1$-dimensional smooth manifolds without
boundary.  This result combined with (ii) implies that
$\displaystyle H_c^2(\RR^2) \cong \RR$.  Then, this result combined with (iii)
implies that $\displaystyle H_c^2(S) \cong \RR$
for all connected and oriented $2$-dimensional smooth manifolds without
boundary.  And so on.
\end{proof}

\begin{egg}
It is interesting to note that
$\displaystyle H_c^1(S^1) \cong \RR$ could be proved directly.

We first note that all differential $1$-form on $\displaystyle S^1$
are closed because there are no non-null differential $2$-form on
$\displaystyle S^1$.  They also all have compact support because
$\displaystyle S^1$ is a compact set.  Moreover, every exact
differential $1$-form $\omega$ on $\displaystyle S^1$ is of the form
$\omega  = \df{f}$ for some $\displaystyle f:S^1 \to \RR$ of class
$\displaystyle C^\infty$.

We have seen in Example~\ref{CnotEpart2} that 
\[
\omega = \frac{-x_2}{x_1^2+x_2^2} \df{x_1} + \frac{x_1}{x_1^2+x_2^2} \df{x_2}
\]
is a differential $1$-form which is not exact.  In fact, $\omega$ is
the volume (or length) element on $\displaystyle S^1$.

Suppose that $\mu$ is a differential $1$-form.  Let
$\displaystyle \phi:\RR  \to \RR^2$ be the function defined by
$\phi(\theta) = \big(\cos(\theta), \sin(\theta)\big)$ for all
$\theta \in \RR$.  We have that
$\displaystyle \phi^\ast(\mu) = f \df{\theta}$ for some
$2\pi$-periodic function $f:\RR \to \RR$.

There are two cases to consider.  If
$\displaystyle \int_{[0,2\pi]} \phi^\ast(\mu) = \int_0^{2\pi}
f(\theta) \dx{\theta}  = 0$, then $\tilde{f}:\RR\to \RR$ defined by
$\displaystyle \tilde{f}(\theta) = \int_0^\theta f(t)\dx{t}$ is a
$2\pi$-periodic function of class $\displaystyle C^\infty$ such that
$\df{\tilde{f}} = f \df{\theta}$.  Consider
$\displaystyle F: S^1 \to \RR$ defined by
$\displaystyle F = \tilde{f} \circ (\phi\big|_J)^{-1}$ on $\phi(J)$
for all open interval of length less that $2\pi$ to ensure that
$\phi\big|_J$ be one-to-one.  This function is of class
$\displaystyle C^\infty$ on $\displaystyle S^1$ and $\df{F} = \mu$ because
$\displaystyle \phi^\ast(\df{F}) = \df{(\phi^\ast(F))}
= \df{\tilde{f}} = f \df{\theta} = \phi^\ast(\mu)$.  Thus $\mu$ is exact.
Hence $[\mu] = [0]$ in $\displaystyle H_c^1(S^1)$.

If $\displaystyle c = \int_{[0,2\pi]} \phi^\ast(f) = \int_0^{2\pi}
f(\theta) \dx{\theta} \neq 0$, then
$\mu - c/(2\pi) \, \omega$ is exact.  To prove this claim, we first
note that $\displaystyle \phi^\ast(\omega) = \df{\theta}$.  Hence
$\displaystyle \int_{[0,2\pi]} \phi^\ast\big( \mu - c/(2\pi)\, \omega\big)
= \int_0^{2\pi} \big(f(\theta) - c/(2\pi) \big) \dx{\theta} = 0$.
Let $\tilde{f}:\RR\to \RR$ be the function of period $2\pi$ defined by
$\displaystyle \tilde{f}(\theta) = \int_0^\theta f(t)\dx{t} -
c\theta/(2\pi)$ for $0 \leq \theta \leq 2\pi$.  This is a
$2\pi$-periodic function of class $\displaystyle C^\infty$ such that
$\df{\tilde{f}} = f \df{\theta} - c/(2\pi)$.  Hence
$F$ defined as above is a function of class
$\displaystyle C^\infty$ on $\displaystyle S^1$
such that $\df{F} = \mu - c/(2\pi)\, \omega$ because
$\phi^\ast(\df{F}) = \df{(\phi^\ast(F))} = \df{\tilde{f} -c\theta/(2\pi)}
= f \df{\theta} - c/(2\pi)\,\df{\theta}
= \phi^\ast(\mu -c/(2\pi)\, \omega)$.  Thus $[\mu] = [c/(2\pi)\,\omega]$
in $\displaystyle H_c^1(S^1)$.

We have an isomorphism between
$\displaystyle H_c^1(S^1)$ and $\RR$ given by
$[a\, \omega] \mapsto a$ for all $a \in \RR$.
\end{egg}

The next step should normally be to compute $\displaystyle H_c^q(S)$
for the connected and oriented $k$-dimensional smooth manifolds $S$
and $0 \leq q <k$.  We will prove the following result
at the end of the next section.

\begin{theorem}  \label{thrmHqRkE0lk}
$\displaystyle H_c^q(\RR^k) = \{[0]\}$ for $0 \leq q < k$.  
\end{theorem}

\begin{focus}{Note}
Suppose that $S$ is an oriented $k$-dimensional smooth manifold and
$0 \leq q \leq k$.  It is the tradition to write
$\displaystyle H_c^q(S) = 0$ when
$\displaystyle H_c^q(S) = \{[0]\}$.  We will adopt this
tradition from now on.
\end{focus}

\subsection{Cohomology Modules (General Case)} \label{ssubCMGcase}

We now drop the condition that the differential $q$-forms have a
compact support.

\begin{defn} \label{defnCohomRel}
Let $S$ be a $k$-dimensional smooth manifold.
Two closed differential $q$-forms, $\omega_1$ and $\omega_2$, on $S$ are
{\bfseries cohomologous}\index{Cohomologous Forms} if there exists a
differential $(q-1)$-form $\eta$ on $S$ such that
$\omega_1 - \omega_2 = \df{\eta}$.  We write $\omega_1 \sim \omega_2$
\footnotemark.
\end{defn}

\footnotetext{We use the same symbol to denote cohomologous
differential forms with or without compact supports.  The context
should be enough to determine which cohomologous definition should be
used.  The readers may wonder why we did not use a different symbol.
They will understand when reading the next chapter why we choose to
not introduce a new symbol at this time.  There will be enough
different similarity symbols in the next chapter to drive someone crazy.}

As we had for the relation defined in
Definition~\ref{defnCohomRelCompact}, the relation defined in
Definition~\ref{defnCohomRel} is an equivalence relation.  The
justification is basically identical.  We have two definitions of
cohomologous forms.  The context will determine which definition is to
be used.

\begin{defn}  \label{defnqdeRhamCM}
Let $S$ be a $k$-dimensional smooth manifold.  The
{\bfseries $\displaystyle \mathbf{q^{th}}$ (de Rham) Cohomology Module}
\index{Cohomology Module}\index{(de Rham) Cohomology
Module|see{Cohomology Module}}
of $S$, denoted $\displaystyle H^q(S)$, is the set of all equivalent
classes of closed differential $q$-forms on $S$ with respect to the
equivalence relation given in Definition~\ref{defnCohomRel}.
An equivalence class for a closed differential $q$-form $\omega$ on $S$
is called a {\bfseries cohomology class}\index{Cohomology Class} and
is denoted $[\omega]$.
\end{defn}

\begin{rmk}
$\displaystyle H^q(S)$ is also called the
{\bfseries $\displaystyle \mathbf{q^{th}}$ cohomology
group}\index{Cohomology Group|see{Cohomology Module}}
because the space of differential $q$-form on $S$ can be view as a
free abelian group generated by a finite basis of differential
$q$-forms.  Moreover, since we are using $\RR$ for the scalars, we may
also have talked about vector spaces instead of module.  However, the
use of the word module is justified by the fact that the theory can be
expanded to modules.
\end{rmk}

A proof almost identical (and simpler) to the proof of
Proposition~\ref{HCqVectSpac} yields the following result.

\begin{prop}
Suppose that $S$ is a $k$-dimensional smooth manifold.
Then $\displaystyle H^q(S)$ is a vector space.
\end{prop}

\begin{rmk}
Let $S$ be a $k$-dimensional smooth manifold.
The null element $[0]$ in $\displaystyle H^q(S)$ is the collection of
all exact differential $q$-forms on $S$ because $\omega$ in the class
$[0]$ means that $\omega = \omega - 0 = \df{\eta}$ for some
differential $(q-1)$-form on $S$.
\end{rmk}

\begin{egg}
If $S$ is a compact and oriented $k$-dimensional smooth manifold
without boundary, then the dimension of
$\displaystyle H^k(S)$ is larger than $0$.  To prove this claim, we
only have to show that there exists a closed and non-exact
differential $k$-form on $S$.  This is what we did in
Example~\ref{eggHkDim}.

We have a more precise result.  Since $S$ is compact, we get from
Theorem~\ref{theoHckCR} that $\displaystyle H^k(S) = H^k_c(S) \cong \RR$.
\end{egg}

\begin{egg}
We have from Example~\ref{HR0dim} that
$\displaystyle H^{n-1}\left(\RR^n \setminus \{ \VEC{0}\}\right)$
is of dimension greater than $0$ for $n\geq 2$.
\end{egg}

\begin{prop} \label{H0kdimS}
Suppose that $S$ is a $k$-dimensional smooth manifold.  The dimension of
$\displaystyle H^0(S)$ is equal to the number of (connected)
components of $S$.
\end{prop}

\begin{proof}
To justify this claim, we first note that a differential $0$-form on
$S$ is a function.  Let $f:S \to \RR$ be smooth function.

If $(W,U,\phi)$ is a local chart of $S$, then
$\df{f}=0$ implies that
$\displaystyle 0 = \phi^\ast(\df{f}) = \df{(\phi^\ast f)}
= \df{(f\circ \phi)}
= \sum_{j=1}^k \pdfdx{(f \circ \phi)}{w_j} \df{w_j} = 0$ on $W$ and so
$\displaystyle \pdfdx{(f \circ \phi)}{w_j} = 0$ on $W$ for $1\leq j \leq k$.
Since $\phi$ is one-to-one, this implies that $f$ is constant on $U$.
Thus $f$ is constant on open subsets of $S$ and therefore on the
components of $S$.  Suppose that $S$ is the union of the disjoint
components $S_i$ for $i \in I$.  Let
\[
\Chi_i(\VEC{u}) = \begin{cases}
1 & \quad \text{if} \ \VEC{u} \in S_i \\
0 & \quad \text{if} \ \VEC{u} \in S \setminus S_i
\end{cases}
\]
Then $\displaystyle \{ [\Chi_i] \}_{i\in I}$ is a basis of
$\displaystyle H^0(S)$.

It is important to note that each equivalent class $[f]$ of
$\displaystyle H^0(S)$ consists in one element only, the function $f$.
If $g$ is an element of the class $[f]$, then we must have that
$g - f = 0$ because there are no such thing as a differential
$-1$-forms $\eta$ such that $\df{\eta} = g - f$.
\end{proof}

The following result is a consequence of the previous proposition.

\begin{cor} \label{corHqe0e1}
$\displaystyle H^0(\RR^k) \cong \RR$.
\end{cor}

It follows from Poincaré Lemma, Theorem~\ref{closedexactCntrct}, that
if $S$ is a $k$-dimensional smooth manifold which is smoothly
contractible to the point of $S$, then
$\displaystyle H^q(S) = 0$ for $0 <q \leq k$.
In particular, we get the following result.

\begin{cor} \label{corHqg0e0}
$\displaystyle H^q(\RR^k) = 0$ for $0 <q \leq k$.
\end{cor}

Now that we have computed the $\displaystyle q^{th}$ cohomology module
of $\displaystyle \RR^k$ for $0 \leq q \leq k$, we would like to do
the same for other smooth manifolds than $\displaystyle \RR^k$.  In
fact, we can easily
obtain the $\displaystyle q^{th}$ cohomology module of the smooth manifolds
that are diffeomorphic to $\displaystyle \RR^k$.

Suppose that $S_1$ and $S_2$ are two $k$-dimensional smooth manifolds.  Let
$f:S_1 \to S_2$ be a function of class $\displaystyle C^\infty$.  We
have that $\displaystyle f^\ast$ maps differential $q$-forms on $S_2$ to 
differential $q$-forms on $S_1$.  The following map is well
defined.
\begin{align*}
f^\sharp : H^q(S_2) &\to H^q(S_1) \\
[\omega] &\mapsto [f^\ast(\omega)]
\end{align*}
Suppose that
$\omega_1$ and $\omega_2$ are two differential $q$-forms on $S_2$
such that $\omega_1 \sim \omega_2$.  Then there exists a differential
$(q-1)$-form $\eta$ on $S_2$ such that
$\omega_2 - \omega_1 = \df{\eta}$.  Using
Proposition~\ref{manifSDFitem4}, we get that
$\displaystyle f^\ast(\omega_2) - f^\ast(\omega_1) = f^\ast(\omega_2- \omega_1)
= f^\ast (\df{\eta})  = \df{(f^\ast(\eta))}$.  Since
$\displaystyle f^\ast(\eta)$ is
a differential $(q-1)$-form on $S_1$, we have that
$\displaystyle f^\ast(\omega_2) \sim f^\ast(\omega_1)$ and
$\displaystyle [f^\ast(\omega_2)] = [f^\ast(\omega_1)]$.

\begin{prop}  \label{propFsharpP}
\begin{enumerate}
\item Suppose that $S_1$ and $S_2$ are two $k$-dimensional smooth
manifolds and that $f:S_1 \to S_2$ is a function of class
$\displaystyle C^\infty$.  Then
$\displaystyle f^\sharp : H^q(S_2) \to H^q(S_1)$ is a linear map.
\item  Suppose that $S_i$ for $1\leq i \leq 3$ are three
$k$-dimensional smooth manifolds and that $f_i:S_i \to S_{i+1}$ for
$1\leq i \leq 2$ are two function of class $\displaystyle C^\infty$.  Then
$\displaystyle (f_2 \circ f_1)^\sharp = f_1^\sharp \circ f_2^\sharp$.
\end{enumerate}
\end{prop}

We leave the proof of this proposition to the reader.

If $f:S_1 \to S_2$ is a diffeomorphism of class $\displaystyle C^\infty$
between two $k$-dimensional smooth manifolds $S_1$ and $S_2$, then
$\displaystyle f^\sharp: H^q(S_2) \to H^q(S_1)$ is a linear isomorphism.
Therefore $\displaystyle H^q(S_2) \cong H^q(S_1)$.

Our next goal is to compute the $\displaystyle q^{th}$ cohomology modules
of $\displaystyle \RR^k \setminus \{\VEC{0}\}$ for $0 \leq q \leq k$.
This will also give us the $\displaystyle q^{th}$ cohomology module of
$\displaystyle S^{k-1}$ for $0 \leq q \leq k-1$ as we will show shortly.

\begin{defn} \label{defnSmoothHomot}
Let $S_1$ and $S_2$ be two $k$-dimensional smooth manifolds
and $f,g:S_1 \to S_2$ be two functions of class
$\displaystyle C^\infty$.  We say that $f$ and $g$ are
{\bfseries (smoothly) homotopic}\index{Homotopic Functions}
if there exists a function $H:S_1 \times [0,1] \to S_2$ of class
$\displaystyle C^\infty$ such that $H(\VEC{u},0) = f(\VEC{u})$ and
$H(\VEC{u},1) = g(\VEC{u})$ for all $\VEC{u} \in S_1$.  The function
$H$ is called an {\bfseries homotopy}\index{Homotopy} between $f$ and $g$.
\end{defn}

We can reformulate Definition~\ref{defnCtoP} by saying that
a smooth manifold $S$ is contractible to a point $\tilde{\VEC{u}} \in S$
if the identity map $\Id:S \to S$ and the map $g:S \to S$
defined by $g(\VEC{u}) = \tilde{\VEC{u}}$ for all $\VEC{u} \in S$
are homotopic.

\begin{prop} \label{propfsEgs}
Suppose that $S_1$ and $S_2$ are two $k$-dimensional smooth manifolds
and that $f,g:S_1 \to S_2$ are two homotopic functions of class
$\displaystyle C^\infty$.  Then $\displaystyle f^\sharp = g^\sharp$
as maps from $\displaystyle H^q(S_2)$ to $\displaystyle H^q(S_1)$.
\end{prop}

\begin{proof}
Let $F$ be the map defined in (\ref{defFSItoS}) that
maps differential $q$-forms on $S_1\times I$ to differential
$(q-1)$-forms on $S_1$.  Moreover, let $\iota_s:S_1 \to S_1 \times [0,1]$
for $s \in [0,1]$ be the inclusion defined by
$\iota_s(\VEC{u}) = (\VEC{u},s)$.

Since $f$ and $g$ are homotopic, there exists a smooth function
$H:S_1 \times [0,1] \to S_2$ such that
$f = H\circ \iota_0$ and $g = H \circ \iota_1$.

Given $\omega$, a closed differential $q$-form on $S_2$, we have from
Proposition~\ref{propi1i0dffd} that
\begin{align*}
&g^\ast(\omega) - f^\ast(\omega)
= (H \circ \iota_1)^\ast(\omega) - (H \circ \iota_0)^\ast(\omega)
= \iota_1^\ast(H^\ast(\omega)) - \iota_0^\ast(H^\ast(\omega)) \\
&\qquad = \df{\left(F(H^\ast(\omega))\right)}
+ F\left(\df{(H^\ast(\omega))}\right)
= \df{\left(F(H^\ast(\omega))\right)} + F\left(H^\ast(\df{\omega})\right)
= \df{\left(F(H^\ast(\omega))\right)} \ .
\end{align*}
Thus $\displaystyle g^\ast(\omega) \sim f^\ast(\omega)$.  it follows that
$\displaystyle f^\sharp([\omega]) = [f^\ast(\omega)] = [g^\ast(\omega)]
=g^\sharp([\omega])$.
\end{proof}

\begin{prop} \label{propHqSkm1cHqRk}
$\displaystyle H^q(S^{k-1}) \cong H^q(\RR^k \setminus \{\VEC{0}\})$
for $0 \leq q \leq k$.
\end{prop}

\begin{proof}
The inclusion $\displaystyle \iota:S^{k-1} \to \RR^k\setminus \{\VEC{0}\}$
defined by $\iota(\VEC{u}) = \VEC{u}$ and the retraction
$\displaystyle \rho:\RR^k\setminus \{\VEC{0}\}$ defined by
$\displaystyle \rho(\VEC{u}) = \|\VEC{u}\|^{-1} \VEC{u}$ satisfy
$\displaystyle \rho \circ \iota = \Id : S^{k-1} \to S^{k-1}$.
However,
$\displaystyle \iota \circ \rho : \RR^k \setminus \{\VEC{0}\} \to
\RR^k \setminus \{\VEC{0}\}$ is not the identity but it is homotopic to
the identity $\displaystyle \Id: \RR^k \setminus \{\VEC{0}\} \to
\RR^k \setminus \{\VEC{0}\}$.  An homotopy
$\displaystyle H: (\RR^k \setminus \{\VEC{0}\})\times [0,1] \to
\RR^k \setminus \{\VEC{0}\}$ between $\iota \circ \rho$
and $\Id$ is given by
$H(\VEC{u},t) = t \VEC{u} + (1-t) (\iota\circ \rho)(\VEC{u})$ for all
$\displaystyle (\VEC{u},t) \in (\RR^k \setminus \{\VEC{0}\})\times [0,1]$.
It follows from Proposition~\ref{propfsEgs} that
$\displaystyle (\iota \circ \rho)^\sharp = \Id^\sharp : H^q(\RR^k \setminus
\{\VEC{0}\}) \to H^q(\RR^k \setminus \{\VEC{0}\})$ for $0 \leq q \leq k$.

Hence,
$\displaystyle \iota^\sharp \circ \rho^\sharp = (\rho \circ \iota)^\sharp = 
\Id^\sharp : H^q(S^{k-1}) \to H^q(S^{k-1})$ and
$\displaystyle \rho^\sharp \circ \iota^\sharp =
(\iota \circ \rho)^\sharp = \Id^\sharp : H^q(\RR^k \setminus
\{\VEC{0}\}) \to H^q(\RR^k \setminus \{\VEC{0}\})$ for $0 \leq q \leq k$.
Thus
$\displaystyle \rho^\sharp : H^q(S^{k-1}) \to H^q(\RR^k \setminus \{\VEC{0}\})$
is the inverse of
$\displaystyle \iota^\sharp : H^q(\RR^k \setminus \{\VEC{0}\}) \to
H^q(S^{k-1})$.  Therefore $\displaystyle \rho^\sharp$ provides an
isomorphism between $\displaystyle H^q(S^{k-1})$ and
$\displaystyle H^q(\RR^k \setminus \{\VEC{0}\})$.
\end{proof}

The following result follows from Proposition~\ref{H0kdimS} and is
consistent with the previous proposition.

\begin{cor}
$\displaystyle H^0(S^{k-1}) \cong H^0(\RR^k \setminus \{\VEC{0}\})
\cong \RR$.
\end{cor}

The next result follows directly from Proposition~\ref{propHqSkm1cHqRk}
because differential $q$-forms on a $(k-1)$-dimensional manifold are null
if $q>k-1$.

\begin{cor}
$\displaystyle H^k(\RR^k \setminus \{\VEC{0}\}) \cong H^k(S^{k-1}) = 0$.
\end{cor}

Obviously, if $S$ is a compact $k$-dimensional smooth manifold, then
$\displaystyle H^k(S) = H_c^k(S)$.  In particular, if $S$ is a
compact, connected and oriented $k$-dimensional smooth manifold
without boundary, then we get from Theorem~\ref{theoHckCR} that
$\displaystyle H^k(S) = H_c^k(S) \cong \RR$.
The next result then follows from Proposition~\ref{propHqSkm1cHqRk}.

\begin{cor}
$\displaystyle H^{k-1}(\RR^k \setminus \{\VEC{0}\} )\cong H^{k-1}(S^{k-1}) 
\cong \RR$.  
\end{cor}

To compute $\displaystyle H^q(\RR^k \setminus \{\VEC{0}\}) \cong H^q(S^{k-1})$
for $0 < q < k-1$, we need to compute
$\displaystyle H^k(S)$ for $S$ a non-compact, connected and oriented
$k$-dimensional smooth manifold without boundary.  The proof is
similar to the proof of Theorem~\ref{theoHckCR}.  However, the fact
that we are no longer limited to differential forms with compact
supports complicates the proofs.

Many of the topological results that are presented below could be
simplified and even skipped if we consider that all our manifolds are
submanifolds of the metric space $\displaystyle \RR^n$ and if we use
all the properties of a metric space with a countable dense set.
However, with just a little bit more work, we can present a general
theory that may be useful to the reader if they wish to study more
advanced topics of differential and topological geometry later.

We need the following preliminary information.  

\begin{defn}
A topological space $S$ is a
{\bfseries Hausdorff space}\index{Hausdorff Space} if, for every two
distinct elements $x_1$ and $x_2$ in $S$, there exist disjoint open
neighbourhoods $U_1$ and $U_2$ of $x_1$ and $x_2$ respectively.
\end{defn}

All the manifolds that we consider are Hausdorff spaces because they are
subsets of some Euclidean space $\displaystyle \RR^n$ and their
topology is induced from the topology on $\displaystyle \RR^n$.

\begin{defn}
A Hausdorff space $S$ is a {\bfseries Lindelöf space}\index{Lindelöf Space}
if every open cover of $S$ admit a countable open subcover.
\end{defn}

Again, every manifold $S$ that we consider is a Lindelöf space.  This
result follows from the next proposition because $S$ is a subset of
some Euclidean space $\displaystyle \RR^n$.  Therefore, it admits a
countable basis.

\begin{prop}
If $S$ is a Hausdorff space with a countable basis, then $S$ is
a Lindelöf space.
\end{prop}

\begin{proof}
Let $\displaystyle \{B_j\}_{j\in \NN}$  be a countable basis of $S$;
namely, every open set in $S$ can be expressed as the union of some
$B_j$.

Suppose that $\displaystyle \{U_\alpha\}_{\alpha\in A}$ is an open
cover of $S$.  We have that each $U_\alpha$ is the union of some
$\displaystyle B_j$ for
$j \in J_\alpha \subset \NN$.  For each set $B_j$ with
$\displaystyle j \in N = \bigcup_{\alpha\in \alpha} J_\alpha \subset \NN$,
we associate one of the $U_\alpha$ containing $B_j$, say $U_{\alpha_j}$.
Since $\displaystyle S = \bigcup_{\alpha \in A} U_\alpha
= \bigcup_{j \in N} B_j$, we have that
$\displaystyle \{U_{\alpha_j}\}_{j\in N}$ is a (at most)
countable open cover of $S$.
\end{proof}

\begin{defn}
A Hausdorff space $S$ is a
{\bfseries locally compact space}\index{Locally Compact Space} if, for
every element $x$ in $S$, there exists an open neighbourhood $U$
of $x$ in $S$ such that $\overline{U}$ is compact.
\end{defn}

Every manifold $S$ is a locally compact space because, for every
$\VEC{u} \in S$, there exists a local chart $(W,U,\phi)$ such that
$\VEC{u} \in U$ and $U$ is homeomorphic to
$\displaystyle W \subset H_k$ which is locally compact.

\begin{prop} \label{topolProp1}
If $S$ is a locally compact and Lindelöf space, then $S$ is
{\bfseries $\sigma$-compact}\index{$\sigma$-compact}; namely, $S$ can
be expressed as the union of at most countably many compact sets.
\end{prop}

\begin{proof}
For each $x \in S$, there exists an open set $V_x$ such that
$\overline{V}_x$ is compact.  The collection
$\displaystyle \{ V_x\}_{x\in S}$ is an open cover of $S$.  Since $S$
is a Lindelöf space, there exists a countable open subcover
$\displaystyle \{ V_{x_i}\}_{i \in \NN}$ of $S$.
Hence $\displaystyle S = \bigcup_{i\in \NN} \overline{V}_{x_i}$
where $\overline{V}_{x_i}$ is compact for all $i \in \NN$.
\end{proof}

An elementary proof of this result for connected and locally compact
metric spaces is also given in \cite{Sv1}.

\begin{prop}  \label{topolProp2}
Suppose that $S$ is a locally compact space.  Given any compact set 
$K$ and open set $U$ of $S$ such that $K \subset U$, there exist an
open set $V$ such that $K \subset V \subset \overline{V} \subset U$ 
and $\overline{V}$ is compact.
\end{prop}

A proof of this proposition can be found in \cite{Du}.

\begin{prop} \label{BjUilocfin}
Suppose that $\displaystyle \{ B_\tau \}_{\tau \in T}$ is an open cover of
a manifold $S$.  There exists an atlas
$\displaystyle \JJ = \{ (W_j,U_j,\phi_j) \}_{j \in \NN}$ on $S$ 
such that each $U_j$ is a subset of a $B_{\tau_j}$ for some $\tau_j \in T$ and 
$\displaystyle \{ U_j \}_{j\in \NN}$ is
{\bfseries locally finite}\index{Locally Finite}; namely, for every
$\VEC{u} \in S$, there exists an open neighbourhood $V_{\VEC{u}}$ of
$\VEC{u}$ such that $U_j \cap V_{\VEC{u}} = \emptyset$ for all but at
most a finite number of $j \in \NN$.
\end{prop}

\begin{proof}
Since $S$ is a locally compact and Lindelöf space, we may use
Proposition~\ref{topolProp1}.  So, we can express $S$ as
$\displaystyle S = \bigcup_{i\in \NN} K_i$ where the $K_i$ are compact
sets.

From Proposition~\ref{topolProp2}, there exists an open set $V_0$ such
that $K_0 \subset V_0 \subset \overline{V}_0 \subset S$ and
$\overline{V}_0$ is compact.  It follows that $\overline{V_0} \cup K_1$ is 
a compact set.  Again, we have from Proposition~\ref{topolProp2} that
there exists an open set $V_1$ such $\overline{V}_0 \cup K_1 \subset
V_1 \subset \overline{V}_1 \subset S$ and $\overline{V}_1$ is compact.
In general, given $\overline{V}_{i-1}$ with $i >0$ such that
$\overline{V}_{i-1}$ is compact, we have that
$\overline{V}_{i-1} \cup K_i$ is compact.  We again use
Proposition~\ref{topolProp2} to get an open set $V_i$ such
$\overline{V}_{i-1} \cup K_i \subset V_i \subset \overline{V}_i \subset S$
and $\overline{V}_i$ is compact.
We get a sequence of open sets $V_i$ such that
$V_1 \subset \overline{V}_1 \subset V_2 \subset \overline{V}_2 \subset
V_ 3 \subset \ldots$ and the $\overline{V}_i$ are compact sets
and $K_ i \subset V_i$ for all $i \in \NN$.

Since $\displaystyle S = \bigcup_{i\in \NN} K_i \subset \bigcup_{i\in \NN} V_i
\subset S$, we get that $\displaystyle S = \bigcup_{i \in \NN} V_i$.
If we set $V_{-1} = \emptyset$, then we have that 
$\displaystyle S = \bigcup_{i\in \NN} (\overline{V}_i \setminus V_{i-1})$
where each $\overline{V}_i \setminus V_{i-1}$ is a compact set because it
is a closed set of the compact set $\overline{V}_i$.

Given $i \in \NN$ and
$\displaystyle \VEC{u} \in \overline{V}_i \setminus V_{i-1}$,
there exists a local chart $(W,U,\phi)$ and a set $B_\tau$ such that
$\VEC{u} \in U \cap B_\tau$.  Let $V_{-2} = \emptyset$.
If we set $\displaystyle
U_{\VEC{u}} = U\cap B_\tau \cap (V_{i+1} \setminus \overline{V}_{i-2})$,
$\displaystyle W_{\VEC{u}} = \phi^{-1}(U_{\VEC{u}})$ and
$\phi_{\VEC{u}} = \phi$, then we get a local chart
$(W_{\VEC{u}},U_{\VEC{u}},\phi_{\VEC{u}})$ with
$\displaystyle \VEC{u} \in U_{\VEC{u}} \subset (V_{i+1} \setminus
\overline{V}_{i-2})$ and $\displaystyle U_{\VEC{u}} \subset B_\tau$ for
some $\tau \in T$.
Repeating this for all $\VEC{x} \in S$, we get an atlas
$\displaystyle \{(W_{\VEC{u}},U_{\VEC{u}},\phi_{\VEC{u}}) \}_{\VEC{u} \in S}$
on $S$ such that $\displaystyle
U_{\VEC{u}} \subset (V_{i+1} \setminus \overline{V}_{i-2})$
for $\VEC{u} \in \overline{V}_i \setminus V_{i-1}$ and
$\displaystyle U_{\VEC{u}} \subset B_\tau$ for some $\tau \in T$.

Since $\displaystyle \{U_{\VEC{u}}\}_{\VEC{u} \in \overline{V}_0}$ is
an open cover of the compact set $\overline{V}_0$, there is a finite
subcover $\displaystyle \{U_{\VEC{u}_j}\}_{0 \leq j \leq J_0}$
for some $J_0 \in \NN$.  This yields a finite number of local charts
$\displaystyle \{ (W_j,U_j,\phi_j) \}_{0 \leq j \leq J_0}$
such that $\displaystyle \{U_j\}_{0\leq j \leq J_0}$ is an open cover
of $\overline{V}_0$ and $U_j \subset V_1$ for $1 \leq j \leq J_0$.

Let $I_{-1} = 0$.  Suppose that we have a finite number of local charts\\
$\displaystyle \{ (W_j,U_j,\phi_j) \}_{J_{m-1}+1 \leq j \leq J_m}$
for some $J_{i-1} < J_i$ and $i\geq 0$ such that
$\displaystyle \{ U_j \}_{J_{i-1}+1 \leq j \leq J_i}$
is an open cover of $\displaystyle \overline{V}_i \setminus V_{i-1}$ and
$U_j \subset (V_{i+1} \setminus \overline{V}_{i-2})$ for
$J_{i-1}+1 \leq j \leq J_i$.

Since $\displaystyle \{U_{\VEC{u}}\}_{\VEC{u} \in \overline{V}_{i+1}\setminus V_i}$
is an open cover of the compact set
$\displaystyle \overline{V}_{i+1}\setminus V_i$, there is a finite
subcover $\displaystyle \{U_{\VEC{u}_j}\}_{J_i+1 \leq j \leq J_{i+1}}$
for some $J_{i+1} \in \NN$.  This yields a finite number of local charts
$\displaystyle \{ (W_j,U_j,\phi_j) \}_{J_i+1 \leq j \leq J_{i+1}}$
such that
$\displaystyle \{ U_j \}_{J_i+1 \leq j \leq J_{i+1}}$
is an open cover of $\displaystyle \overline{V}_{i+1} \setminus V_i$ and
$U_j \subset (V_{i+2} \setminus \overline{V}_{i-1})$ for
$J_i+1 \leq j \leq J_{i+1}$.
\pdfbox{cohomology/locfinAtl}

By induction, we get the atlas $\JJ = \{ (W_j,U_j,\phi_j) \}_{j \in \NN}$
that we were looking for.  By construction, we have that
$U_j$ is a subset of a $B_{\tau_j}$ for some $\tau_j \in T$.
To show that $\{ U_j \}_{j\in \NN}$ is
locally finite, we consider $\VEC{u} \in S$.  Let $I_{-3} = 0$.
We have that $\displaystyle \VEC{u} \in \overline{V}_i \setminus V_{i-1}$ for
some $i \in \NN$.  If we choose an open neighbourhood
$\displaystyle V_{\VEC{u}} \subset V_{i+1} \setminus \overline{V}_{i-2}$
of $\VEC{u}$, then $U_j \cap V_{\VEC{u}} = \emptyset$ for all
$j \leq J_{i-3}$ and $j > J_{i+2}$.
\end{proof}

\begin{lemma} \label{locfinIfinnumb}
Suppose that $\{ U_\alpha \}_{\alpha\in A}$ is a locally finite open
cover of a topological space $S$.  Given any compact set
$K \subset S$, there exists an open set $V \supset K$ such that
$V\cap U_\alpha = \emptyset$ for all but a finite number of $\alpha \in A$.
\end{lemma}

\begin{proof}
Since the local cover is locally finite, there exists for   
each $\VEC{x} \in K$ an open set $V_{\VEC{x}}$ such that
$U_\alpha \cap V_{\VEC{x}} = \emptyset$ for all but a finite number of
$U_\alpha$.
Since $\displaystyle \{ V_{\VEC{x}} \}_{\VEC{x} \in K}$ is an open
cover of the compact set $K$, there exists a finite subcover
$\displaystyle \{ V_{\VEC{x}_i} \}_{1 \leq i \leq I}$ of $K$.

Let $\displaystyle V = \bigcup_{1 \leq i \leq I}V_{\VEC{x}_i}$.
We have that $V$ is an open set such that $K \subset V$ and
there exists only a finite number of $U_\alpha$ that intersect $V$ because
a set $U_\alpha$ that intersects $V$ must also intersects one of the
$V_{\VEC{x}_i}$ with $1 \leq i \leq I$ and there are only a finite
number of $U_\alpha$ that intersect each $V_{\VEC{x}_i}$.
\end{proof}

\begin{theorem}
If $S$ is a non-compact, connected and oriented $k$-dimensional
smooth manifold without boundary, then $\displaystyle H^k(S) = 0$.
\end{theorem}

\begin{proof}
Since $S$ is not compact, there exists an open cover
$\displaystyle \{B_\tau\}_{\tau\in T}$ of $S$ such that no finite subcover
can cover $S$.  Using Proposition~\ref{BjUilocfin}, we get an atlas
$\displaystyle \A = \{ (W_i,U_i,\phi) \}_{i \in \NN}$ on $S$
such that each $U_i$ is a subset of a $B_{\tau_i}$ for some $\tau_i \in T$ and 
$\displaystyle \{ U_i \}_{i\in \NN}$ is locally finite.

We have that $\displaystyle \{U_i\}_{i \in \NN}$ is an
open cover of $S$ with the property that no finite subcover can cover $S$.
If there were a finite subcover
$\displaystyle \{U_{i_m} \}_{1 \leq m \leq M}$ of $S$, then
$\displaystyle \{B_{\tau_{i_m}} \}_{1 \leq m \leq M}$ would be a
finite subcover of $S$, contradicting our hypothesis.

Let $\displaystyle \{\psi_m\}_{m \in \NNp}$ be a partition of
unity subordinate to $\A$.  Given a differential $k$-form $\omega$ on
$S$, we may express $\omega$ as
$\displaystyle \omega = \sum_{m \in \NNp} \psi_m \omega$.
Recall that the sum is well define because it is finite on any compact
subset of $S$.  If we prove for each $\displaystyle m \in \NNp$ that
there exists a differential $(k-1)$-form of $S$ such that
$\psi_m \omega = \df{\eta_m}$, then we will have that
$\displaystyle \omega = \sum_{m \in \NNp} \df{\eta_m} 
= \df{\Big(\sum_{m \in \NNp} \eta_m\Big)}$.  Hence $[\omega] = [0]$.
This is what we plan to do.

Suppose that $\omega$ is a differential $k$-form $\omega$ on
$S$ with compact support in $U_i$ for some $i \in \NN$.
Since no finite subcover of $\displaystyle \{U_i\}_{i \in \NN}$ can cover
$S$, we have that $U_i \subsetneqq S$.  Let $U_{i_0} = U_i$.  Since $S$ is
connected, there is a set $U_{i_1}$ such that 
$U_{i_0} \cap U_{i_1} \neq \emptyset$.  Given
$\displaystyle \{ U_{i_m} \}_{0 \leq m \leq M}$ such that
$U_{i_m} \cap U_{i_{m-1}} \neq \emptyset$ for $1 \leq m \leq M$.
we can again use the facts that no finite subcover of
$\displaystyle \{U_i\}_{i \in \NN}$ can cover $S$ and that $S$ is
connected, to find a set $U_{i_{M+1}}$ such that
$U_{i_{M+1}} \cap U_{i_M} \neq \emptyset$.  Hence, by induction, we
get an infinite sequence of sets
$\displaystyle \{ U_{i_m} \}_{m \in \NN}$ such that
$U_{i_m} \cap U_{i_{m-1}} \neq \emptyset$ for $m > 0$.

Choose a differential $k$-form $\omega_{i_1}$ with compact support in 
$U_{i_0} \cap U_{i_1}$.  Since $\omega$ has a compact support in
$U_{i_0}$, we may proceed as we did in the proof of
Theorem~\ref{theoHckCR} to get a differential $(k-1)$-form
$\eta_{i_0}$ with support in $U_{i_0}$ and $b_{i_1} \in \RR$ such that 
$b_{i_1} \omega_{i_1} - \omega = \df{\eta_{i_0}}$.
For $m>0$, choose a differential $k$-form $\omega_{i_{m+1}}$ with
compact support in $U_{i_m} \cap U_{i_{m+1}}$.  Since $\omega_{i_m}$
has a compact support in $U_{i_m}$, we may again proceed as we did in
the proof of Theorem~\ref{theoHckCR} to get a differential $(k-1)$-form
$\eta_{i_m}$ with support in $U_{i_m}$ and $b_{i_{m+1}} \in \RR$ such that 
$b_{i_{m+1}}\omega_{i_{m+1}} - \omega_{i_m} = \df{\eta_{i_m}}$.

We have
\begin{align}
\omega &= b_{i_1} \omega_{i_1} - \df{\eta_{i_0}} \nonumber \\
&= b_{i_1} b_{i_2} \omega_{i_2} - b_{i_1}
\df{\eta_{i_1}} - \df{\eta_{i_0}} \nonumber \\
&= b_{i_1} b_{i_2} b_{i_3} \omega_{i_3}
- b_{i_2} b_{i_1} \df{\eta_{i_2}} - b_{i_1} \df{\eta_{i_1}} - \df{\eta_{i_0}}
\nonumber \\
&= \quad \ldots \nonumber \\
&= \left(\prod_{s=1}^m b_{i_s}\right) \omega_{i_m}
- \left(\prod_{s=1}^{m-1} b_{i_s}\right) \df{\eta_{i_{m-1}}}
- \left(\prod_{s=1}^{m-2} b_{i_s}\right) \df{\eta_{i_{m-2}}}
- b_{i_1} \df{\eta_{i_1}} - \df{\eta_{i_0}} \nonumber \\
&= \left(\prod_{s=1}^m b_{i_s}\right) \omega_{i_m}
+\df{\left( - \left(\prod_{s=1}^{m-1} b_{i_s}\right) \eta_{i_{m-1}}
- \left(\prod_{s=1}^{m-2} b_{i_s}\right) \eta_{i_{m-2}}
- b_{i_1} \eta_{i_1} - \eta_{i_0}\right)}   \label{theoHkCzeroEq1}
\end{align}
for $m>0$.  Let
\[
\eta = -\sum_{m=1}^\infty \left(\prod_{s=1}^m b_{i_s}\right) \eta_{i_m}
- \eta_{i_0} \ .
\]
We first note that this sum is finite on every compact subset $K$ of
$S$.  We have that $\supp \eta_{i_m} \subset U_{i_m}$ for all $m \in \NN$.
Since $\displaystyle \{U_i\}_{i \in \NN}$ is a locally finite open cover of
$S$, we get from Lemma~\ref{locfinIfinnumb} that there exists $M_K \geq 0$
such that $U_{i_m} \cap K = \emptyset$ for $m>M_K$.  Thus
$\eta_{i_m}\big|_K = 0$ for $m > M_K$.  We get
\[
\eta\big|_K = -\sum_{m=1}^{M_K} \left(\prod_{s=1}^m b_{i_s}\right)
\eta_{i_m}\big|_K - \eta_{i_0}\big|_K \ .
\]
Moreover, since $\supp \omega_{i_m} \subset U_{i_m}$, we get from
(\ref{theoHkCzeroEq1}) that
\[
\omega\big|_K  = \left(\prod_{s=1}^{M_K+1} b_{i_s}\right)
\underbrace{\omega_{i_{M_K+1}}\big|_K}_{= 0} +\df{\left( \eta\big|_K\right)}
= \df{\left( \eta\big|_K\right)}
\]
Since $S$ is $\sigma$-compact according to
Proposition~\ref{topolProp1}, we have that
$\displaystyle S = \bigcup_{i\in \NN} K_i$ where the $K_i$ are compact
sets.  The previous discussion applied to
$\displaystyle K = \bigcup_{0 \leq i \leq I} K_i$ for all $I$ shows
that $\omega = \df{\eta}$.  This completes the proof.
\end{proof}

\begin{lemma}  \label{lemScStimesR}
Suppose that $S$ is a $k$-dimensional smooth manifold.  Then
$\displaystyle H^q(S) \cong H^q(S \times \RR)$.
\end{lemma}

\begin{proof}
This proof is similar to the proof of Proposition~\ref{propHqSkm1cHqRk}.

Consider the injection $\iota : S \to S \times \{0\}$ defined by
$\iota(\VEC{u}) = (\VEC{u}, 0)$ and the restriction
$\rho:S \times \RR \to S$ defined by $\rho(\VEC{u},x) = \VEC{u}$.

We have that $\rho \circ \iota = \Id: S \to S$.  We also have that
$\iota \circ \rho : S \times \RR \to S \times \RR$ and
$\Id : S \times \RR \to S \times \RR$ are homotopic.  An homotopy
$H: (S \times \RR) \times [0,1] \to S \times \RR$ between
$\iota \circ \rho$ and $\Id$ is given by
$H\big((\VEC{u},x),t\big) = t (\VEC{u},x) + (1-t)
(\iota\circ \rho)(\VEC{u},x))$ for all
$(\VEC{u},x) \in S \times \RR$.  It follows from 
Proposition~\ref{propfsEgs} that
$\displaystyle (\iota \circ \rho)^\sharp = \Id^\sharp : H^q(S\times \RR)
\to H^q(S\times \RR)$.

Hence,
$\displaystyle \iota^\sharp \circ \rho^\sharp = (\rho \circ \iota)^\sharp = 
\Id^\sharp : H^q(S) \to H^q(S)$ and
$\displaystyle \rho^\sharp \circ \iota^\sharp =
(\iota \circ \rho)^\sharp = \Id^\sharp : H^q(S\times \RR) \to H^q(S\times \RR)$
for $0 \leq q \leq k$.
Thus
$\displaystyle \rho^\sharp : H^q(S) \to H^q(S \times \RR)$
is the inverse of
$\displaystyle \iota^\sharp : H^q(S\times \RR) \to H^q(S)$.
Therefore $\displaystyle \rho^\sharp$ provides an
isomorphism between $\displaystyle H^q(S)$ and
$\displaystyle H^q(S \times \RR)$.
\end{proof}

We are now ready to complete the computations of
$\displaystyle H^q(\RR^k \setminus \{\VEC{0}\})$ for the missing cases
of $0 < q < k-1$.

\begin{prop}
$\displaystyle H^q(\RR^k \setminus \{\VEC{0}\}) \cong H^q(S^{k-1}) = 0$
for $0 < q < k-1$.
\end{prop}

\begin{proof}
We prove by induction on $k >2$ that
$\displaystyle H^q(\RR^k\setminus \{\VEC{0}\}) = 0$ for $0 < q < k-1$.

\stage{i}
We first prove the hypothesis of induction for $k=3$.
Suppose that $\omega$ is a differential $1$-form on
$\displaystyle \RR^3 \setminus \{\VEC{0}\} = R_+ \cup R_-$ where
$\displaystyle R_+ = \RR^3 \setminus \{ (0,0,x_3) : x_3 \geq 0 \}$ and
$\displaystyle R_- = \RR^3 \setminus \{ (0,0,x_3) : x_3 \leq 0 \}$.
Since $R_+$ and $R_-$ are star-shaped (one may use $(0,0,-1)$
as the centre of the star for $R_+$ and $(0,0,1)$ as the
centre of the star for $R_-$), we get from Poincaré Lemma,
Theorem~\ref{closedexact}, that there exist two differential
$0$-forms $\eta_+$ and $\eta_-$ on $R_+$ and $R_-$ respectively such that
$\omega = \df{\eta_+}$ on $R_+$ and $\omega = \df{\eta_-}$ on $R_-$.
Since $\eta_+$ and $\eta_-$ are functions such that
$\df{(\eta_+-\eta_-)} = \df{\eta_+}- \df{\eta_-} = 0$ on
$R_+ \cap R_-$, we have that $\eta_+ -\eta_- = c$, a constant, on
$R_+ \cap R_-$.  Let
\[
\eta(\VEC{u}) =
\begin{cases}
\eta_+(\VEC{u}) - c & \quad \text{if} \ \VEC{u} \in R_+ \\
\eta_-(\VEC{u}) & \quad \text{if} \ \VEC{u} \in R_-\setminus R_+
\end{cases}
\]
This is a well defined continuous function on
$\displaystyle \RR^3 \setminus \{\VEC{0}\} = R_+ \cup R_-$ because
$\eta_+(\VEC{u}) - c = \eta_-(\VEC{u})$ for $\VEC{u} \in R_-\cap R_+$.
We have that $\df{\eta} = \omega$ on
$\displaystyle \RR^3 \setminus \{\VEC{0}\}$.  Thus $[\omega] = [0]$ in
$\displaystyle H^1(\RR^3 \setminus \{\VEC{0}\})$.
This proves that  $\displaystyle H^q(\RR^3 \setminus \{\VEC{0}\}) = 0$
for $0 < q < 2$.

\stage{ii} We assume that the hypothesis of induction is true for
$k=j-1$ with $j>3$; namely,
$\displaystyle H^q(\RR^{j-1}\setminus \{\VEC{0}\}) = 0$ for $0 < q < j-2$.

\stage{iii} We now prove that the hypothesis of induction is true for $k=j$.
Let\\
$I_+ = \{ (0,\ldots,0,x_j) \in \RR^j : x_j \geq 0 \}$ and
$I_- = \{ (0,\ldots,0,x_j) \in \RR^j : x_j \leq 0 \}$.
Suppose that $\omega$ is a differential $1$-form on
$\displaystyle \RR^j \setminus \{\VEC{0}\} = R_+ \cup R_-$ where
$\displaystyle R_+ = \RR^j \setminus I_+$ and
$\displaystyle R_- = \RR^j \setminus I_-$.
As in (i), since $R_+$ and $R_-$ are star-shaped, we get from Poincaré Lemma,
Theorem~\ref{closedexact}, that there exist two differential
$0$-forms $\eta_+$ and $\eta_-$ on $R_+$ and $R_-$ respectively such that
$\omega = \df{\eta_+}$ on $R_+$ and $\omega = \df{\eta_-}$ on $R_-$.
Since $\eta_+$ and $\eta_-$ are functions such that
$\df{(\eta_+-\eta_-)} = \df{\eta_+}- \df{\eta_-} = 0$ on
$R_+ \cap R_-$, we have that $\eta_+ -\eta_- = c$, a constant, on
$R_+ \cap R_-$.  Let
\[
\eta(\VEC{u}) =
\begin{cases}
\eta_+(\VEC{u}) - c & \quad \text{if} \ \VEC{u} \in R_+ \\
\eta_-(\VEC{u}) & \quad \text{if} \ \VEC{u} \in R_-\setminus R_+
\end{cases}
\]
This is a well defined continuous function on
$\displaystyle \RR^j \setminus \{\VEC{0}\} = R_+ \cup R_-$ because
$\eta_+(\VEC{u}) - c = \eta_-(\VEC{u})$ for $\VEC{u} \in R_-\cap R_+$.
We have that $\df{\eta} = \omega$ on
$\displaystyle \RR^j \setminus \{\VEC{0}\}$.  Thus $[\omega] = [0]$ in
$\displaystyle H^1(\RR^j \setminus \{\VEC{0}\})$.
This proves that  $\displaystyle H^1(\RR^j \setminus \{\VEC{0}\}) = 0$.

Suppose that $\omega$ is a differential $q$-form on
$\displaystyle \RR^j \setminus \{\VEC{0}\} = R_+ \cup R_-$ with $1<q<j-1$.
Again, we get from Poincaré Lemma that there exist two
differential $(q-1)$-form $\eta_+$ and $\eta_-$ on $R_+$ and $R_-$
respectively such that $\omega = \df{\eta_+}$ on $R_+$ and
$\omega = \df{\eta_-}$ on $R_-$.
We have that
$\df{(\eta_+-\eta_-)} = \df{\eta_+}- \df{\eta_-} = 0$ on
$\displaystyle R_+ \cap R_- = \{\VEC{x} \in \RR^j : x_j = 0 \}
= (\RR^{j-1} \setminus \{\VEC{0}\}) \times \RR$.
However, we get from Lemma~\ref{lemScStimesR} and our hypothesis of induction
that $\displaystyle H^{q-1}((\RR^{j-1} \setminus \{\VEC{0}\}) \times \RR)
\cong H^{q-1}(\RR^{j-1} \setminus \{\VEC{0}\}) = 0$
for $1<q<j-1$.  Therefore, there exists a differential $(q-2)$-form
$\mu$ on $R_+ \cap R_-$ such that
$\eta_+-\eta_- = \df{\mu}$ on $R_+ \cap R_-$.

Let $\{ \psi_j \}_{j\in \NNp}$ be a partition of unity subordinate to the
open cover $\{ R_+,R_-\}$ of $\displaystyle \RR^j\setminus \{\VEC{0}\}$,
and let $\displaystyle \psi_+ = \sum_{\supp \psi_j \subset R_+} \psi_j$ and
$\displaystyle \psi_- = \sum_{\supp \psi_j \subset R_-} \psi_j$.  Note
that the two sums are finite on compact sets according to
Proposition~\ref{cov4}.  We have that
$\psi_+$ and $\psi_-$ are two functions of class $\displaystyle C^\infty$
such that $\supp \psi_+ \subset R_+$ and $\supp \psi_- \subset R_-$.
Let
\[
\tilde{\mu}_-(\VEC{u}) = \begin{cases}
\psi_+(\VEC{u}) \mu(\VEC{u}) &\quad \text{if} \ \VEC{u} \in R_+\cap R_- \\
0 &\displaystyle
\quad \text{if} \ \VEC{u} \in I_+
\end{cases}
\]
and
\[
\tilde{\mu}_+(\VEC{u}) = \begin{cases}
\psi_-(\VEC{u}) \mu(\VEC{u}) &\quad \text{if} \ \VEC{u} \in R_+\cap R_- \\
0 &\displaystyle
\quad \text{if} \ \VEC{u} \in I_-
\end{cases}
\]
By construction, $\tilde{\mu}_-$ is a differentiable $(q-2)$-form on
$R_-$ because $\supp \tilde{\mu}_- \cap I_+ = \emptyset$.
Similarly, $\tilde{\mu}_+$ is a differentiable $(q-2)$-form on $R_+$
because $\supp \tilde{\mu}_+ \cap I_- = \emptyset$.

Since we have on $R_+ \cap R_-$ that
\begin{align*}
\eta_+ - \df{\tilde{\mu}_+}
&= \eta_+ - \psi_- \df{\mu} - \df{\psi_-}\wedge \mu
= \eta_+ - (1- \psi_+) \df{\mu} - \df{(1-\psi_+)}\wedge \mu \\
&= \eta_+ - \df{\mu} - \psi_+ \df{\mu} - \df{\psi_+}\wedge \mu
= \eta_+ - \df{\mu} - \df{\tilde{\mu}_-}
= \eta_- - \df{\tilde{\mu}_-} \ ,
\end{align*}
we may define the following differential $(q-1)$-form on
$\displaystyle \RR^j \setminus \{\VEC{0}\}$.
\[
\eta(\VEC{u}) =
\begin{cases}
\eta_+(\VEC{u}) - \df{\tilde{\mu}_+}(\VEC{u}) & \quad \text{if} \
\VEC{u} \in R_+ \\
\eta_-(\VEC{u}) - \df{\tilde{\mu}_-}(\VEC{u}) & \quad \text{if} \
\VEC{u} \in R_-\setminus R_+ 
\end{cases}
\]
We then have that $\df{\mu} = \df{(\eta_+ - \df{\tilde{\mu}_+})}
= \df{\eta_+} = \omega$ on $R_+$ and
$\df{\mu} = \df{(\eta_- - \df{\tilde{\mu}_-})}
= \df{\eta_-} = \omega$ on $R_-$.  Thus $[\omega] = [0]$ in
$\displaystyle H^q(\RR^j \setminus \{\VEC{0}\}$.  This proves 
that $\displaystyle H^q(\RR^j \setminus \{\VEC{0}\}) = 0$.
\end{proof}

We are now in a position to prove Theorem~\ref{thrmHqRkE0lk}

\begin{proof}[Proof (of Theorem~\ref{thrmHqRkE0lk})]
\stage{i} We first compute $\displaystyle H^0_c(\RR^k)$.  
Let $\displaystyle f:\RR^k \to \RR$ be smooth function.  We can proceed
as in the proof of Proposition~\ref{H0kdimS} to show that $f$ is
constant on $\displaystyle \RR^k$.  Namely, we have that
$\displaystyle \df{f} = \sum_{i=1}^k \pdydx{f}{x_i} \df{x_i} = 0$ on
$\displaystyle \RR^k$ implies $\displaystyle \pdydx{f}{x_i} =0$ for
$1\leq i \leq k$ on $\displaystyle \RR^k$.  Thus $f$ is constant.  But
since $f$ has a compact support, we must have that $f = 0$ on
$\displaystyle \RR^k$.  Thus $\displaystyle H_c^0(\RR^k) = 0$.

\stage{ii} We compute $\displaystyle H^1_c(\RR^k)$.  
Suppose that $\omega$ is a differential $1$-form with compact support
on $\displaystyle \RR^k$.  We have from Proposition~\ref{propDeEo} or
Poincaré Lemma (Theorem~\ref{closedexact} or
Theorem~\ref{closedexactCntrct}) that
there exists a differential $0$-form $f$ on $\displaystyle \RR^k$ such that
$\omega = \df{f}$.

Since $\omega$ has a compact support, there exists $r>0$ such that
$\supp \omega \subset B_r(\VEC{0})$.  It follows from $\df{f} = \omega = 0$ on 
$\displaystyle \RR^k \setminus B_r(\VEC{0})$ that
$f$ is constant on $\displaystyle \RR^k \setminus B_r(\VEC{0})$.
Suppose that $f(\VEC{x}) = c$ for
$\displaystyle \VEC{x} \in \RR^k \setminus B_r(\VEC{0})$.
Then $f -c$ is a differential $0$-form on $\displaystyle \RR^k$ with
compact support in $B_r(\VEC{0})$ such that $\df{(f-c)} = \df{f} = \omega$.
Thus, $\omega \sim 0$ and we obtain that $[\omega] = [0]$ in
$\displaystyle H^1_c(\RR^k)$.
Since $\omega$ is an arbitrary $1$-form with compact support, we
get that $\displaystyle H^1_c(\RR^k) = 0$.

\stage{iii} We now assume that $1 < q <k$.  Suppose that $\omega$ is a
differential $q$-form with compact support on $\displaystyle \RR^k$.
We have from Proposition~\ref{propDeEo} or Poincaré Lemma
(Theorem~\ref{closedexact} or Theorem~\ref{closedexactCntrct}) that
there exists a differential $(q-1)$-form $\eta$ on
$\displaystyle \RR^k$ such that $\omega = \df{\eta}$.

Since $\omega$ has a compact support, there exists $r>0$ such that
$\supp \omega \subset B_r(\VEC{0})$.  Since
$\displaystyle \RR^k \setminus \overline{B_r(\VEC{0})}$ is diffeomorphic to
$\displaystyle \RR^k \setminus \{\VEC{0}\}$ \footnote{A diffeomorphism
is given by $\displaystyle g:\RR^k \setminus \overline{B_r(\VEC{0})} \to
\RR^k \setminus \{\VEC{0}\}$ defined by
$\displaystyle g(\VEC{x}) = \VEC{x} - r\|\VEC{x}\|^{-1} \VEC{x}$.},
we get from the previous proposition that
$\displaystyle H^{q-1}\big(\RR^k \setminus \overline{B_r(\VEC{0})}\,\big) \cong
H^{q-1}(\RR^k \setminus \{\VEC{0}\}) \cong 0$ (see comments after
Proposition~\ref{propFsharpP}).  Thus, there exists a differential
$(q-2)$-form $\mu$ on $\displaystyle \RR^k \setminus \overline{B_r(\VEC{0})}$
such that $\eta = \df{\mu}$

Choose a function $\displaystyle \psi:\RR^k \to \RR$ of class
$\displaystyle C^\infty$
such that $\psi(\VEC{x}) = 0$ for $\VEC{x} \in B_{5r/4}(\VEC{0})$ and
$\psi(\VEC{x}) = 1$ for
$\displaystyle \VEC{x} \in \RR^k \setminus B_{2r}(\VEC{0})$.
Let $\tilde{\mu}$ be the differential $(q-2)$-form on
$\displaystyle \RR^k$ defined by
\[
\tilde{\mu}(\VEC{x}) = \begin{cases}
\psi(\VEC{x}) \mu(\VEC{x}) & \quad \text{if} \ \VEC{x} \in \RR^k
\setminus \overline{B_r(\VEC{0})} \\
0 & \quad \text{if} \ \VEC{x} \in \overline{B_r(\VEC{0})}
\end{cases}
\]
We have that $\supp (\eta - \df{\tilde{\mu}}) \subset B_{2r}(\VEC{0})$ because
$\eta - \df{\tilde{\mu}} = \eta - \df{\mu} = 0$ on
$\displaystyle \RR^k \setminus B_{2r}(\VEC{0})$.  Thus 
$\eta - \df{\tilde{\mu}}$ is a differential $(q-1)$-form with
compact support on $\displaystyle \RR^k$.  Moreover,
$\df{(\eta - \df{\tilde{\mu}})} = \df{\eta} = \omega$.
Hence, $\omega \sim 0$ and we get that $[\omega] = [0]$ in
$\displaystyle H^q_c(\RR^k)$.
Since $\omega$ is an arbitrary $q$-form with compact support, we
conclude that $\displaystyle H^q_c(\RR^k) = 0$.
\end{proof}

\subsection{Mayer-Vietoris Argument} \label{subsectMVarg}

The Mayer-Vietoris argument refers to a standard technique used in
algebraic topology to compute the homology modules of some manifolds.
We will have more to say on this technique in
Section~\ref{sectSingHom} on singular homology. As the
reader will realize in the discussion below, we have kind
of used this technique in our presentation of cohomology.
We will use Mayer-Vietoris argument to compute the $\displaystyle q^{th}$ 
cohomology module of the sphere $\displaystyle S^k$ from scratch.

Let $\displaystyle S_n^{k-1} = S^{k-1}\setminus \{-\VEC{e}_k\}$ and
$\displaystyle S_s^{k-1} = S^{k-1}\setminus \{\VEC{e}_k\}$ be,
respectively, the sphere in $\displaystyle \RR^k$ minus the
``south pole'' and the sphere in $\displaystyle \RR^k$ minus the
``north pole''.

The following information about $\displaystyle S_n^{k-1}$ and
$\displaystyle S_s^{k-1}$ will be useful later.

\begin{lemma} \label{lemSnsContr}
$\displaystyle S_n^{k-1}$ and $\displaystyle S_s^{k-1}$ are smoothly
contractible to a point.
\end{lemma}

\begin{proof}
We prove that $\displaystyle S_n^{k-1}$ is smoothly
contractible to $\VEC{e}_k$ and leave to the reader the task of
proving that $\displaystyle S_s^{k-1}$ is smoothly
contractible to $-\VEC{e}_k$.

Let $\displaystyle p:S_n^{k-1} \to \RR^{k-1}$ be the stereographic
projection defined in the following figure.
\pdfbox{cohomology/stereoP}
We have represented $\displaystyle \RR^{k-1}$ as the classical tangent
plane $\VEC{e}_k + \{ (x_1,x_2, \ldots, x_{k-1},0) : x_i \in \RR \
\text{and}\ 1 \leq i \leq k-1 \}$ to the sphere $\displaystyle
S^{k-1}$ at $\VEC{e}_k$.  In particular, the origin in $\displaystyle
\RR^{k-1}$ is associated to $\VEC{e}_k$ in the tangent plane.

Let $\displaystyle H:S^{k-1}_n \times [0,1] \to S^{k-1}_n$ be the
map defined by $\displaystyle H(\VEC{u},t) = p^{-1}(t\,p(\VEC{u}))$ for
$\displaystyle (\VEC{u},t) \in S^{k-1} \times [0,1]$.  $H$ is a smooth
function such that $H(\VEC{u},0) = \VEC{e}_k$ and
$H(\VEC{u},1) = \VEC{u}$ for all $\displaystyle \VEC{u} \in S^{k-1}_n$.
\end{proof}

\begin{prop}  \label{propHqSeHqSns}
$\displaystyle H^q(S^{k-1}) = H^q(S^{k-1}_n \cup S^{k-1}_s)
= H^{q-1}(S^{k-1}_n \cap S^{k-1}_s)$ for $q>1$.  
\end{prop}

\begin{proof}
The goal is to build a linear isomorphism between
$\displaystyle H^q(S^{k-1}_n \cup S^{k-1}_s)$ and
$\displaystyle H^{q-1}(S^{k-1}_n \cap S^{k-1}_s)$.

\stage{i} We first define a function that maps closed differential
$q$-form on $\displaystyle S^{k-1}_n \cup S^{k-1}_s$ to closed 
differential $(q-1)$-form on $\displaystyle S^{k-1}_n \cap S^{k-1}_s$.

Let $\omega$ be a closed differential $q$-form on
$\displaystyle S^{k-1}_n \cup S^{k-1}_s$.  Since
$\displaystyle S^{k-1}_n$ and $\displaystyle S^{k-1}_s$ are smoothly
contractible to a point according to Lemma~\ref{lemSnsContr}, it
follows from Poincaré lemma, Theorem~\ref{closedexactCntrct}, that
there exist a differential $(q-1)$-form $\eta_n$ on
$\displaystyle S^{k-1}_n$ such that $\omega = \df{\eta_n}$ and a
differential $(q-1)$-form $\eta_s$ on $\displaystyle S^{k-1}_s$ such
that $\omega = \df{\eta_s}$.

Let $\eta = \eta_n - \eta_s$.  we have that $\eta$ is a closed
differential $(q-1)$-form on $\displaystyle S^{k-1}_n \cap S^{k-1}_s$
because $\df{\eta} = \df{\eta_n} - \df{\eta_s} = 0$ on
$\displaystyle S^{k-1}_n \cap S^{k-1}_s$.

We now prove that the map $\displaystyle F: H^q(S^{k-1}_n \cup S^{k-1}_s)
\to H^{q-1}(S^{k-1}_n \cap S^{k-1}_s)$ defined by $F([\omega]) = [\eta]$
is well defined.

We first note that $[\eta]$ is independent of the choice of $\eta_n$
and $\eta_s$.  If $\tilde{\eta}_n$ and $\tilde{\eta}_s$ are two other
differential $(q-1)$-forms on $\displaystyle S^{k-1}_n$ and
$\displaystyle S^{k-1}_s$ respectively such that
$\omega = \df{\tilde{\eta}_n}$ on $\displaystyle S^{k-1}_n$ and
$\omega = \df{\tilde{\eta}_s}$ on $\displaystyle S^{k-1}_s$,
then $\df{(\eta_n - \tilde{\eta}_n)} = \df{\eta_n} - \df{\tilde{\eta}_n} = 0$
on $\displaystyle S^{k-1}_n$.  It follows from Poincaré lemma that
there exists a differential $(q-2)$-form $\mu_n$ on $\displaystyle S^{k-1}_n$
such that $\eta_n - \tilde{\eta}_n = \df{\mu_n}$.  Similarly,
there exists a differential $(q-2)$-form $\mu_s$ on $\displaystyle S^{k-1}_s$
such that $\eta_s - \tilde{\eta}_s = \df{\mu_s}$.  Thus
$(\eta_n - \eta_s) -(\tilde{\eta}_n - \tilde{\eta}_s) 
= \df{(\mu_n - \mu_s)}$ on $\displaystyle S^{k-1}_n \cap S^{k-1}_s$.
Hence $\eta_n - \eta_s \sim \tilde{\eta}_n - \tilde{\eta}_s$.

We also have that $[\eta]$ is independent of the chosen representative
of the cohomology class $[\omega]$.  Suppose that
$\tilde{\omega} \sim \omega$.  Then there exists a differential
$(q-1)$-form $\mu$ on $\displaystyle S^{k-1}$ such that
$\displaystyle \tilde{\omega} = \omega + \df{\mu}$.
Hence, the differential $(q-1)$-form $\eta_n+\mu$ on
$\displaystyle S^{k-1}_n$ is such that $\tilde{\omega} = \df{(\eta_n+\mu)}$ and
the differential $(q-1)$-form $\eta_s+\mu$ on
$\displaystyle S^{k-1}_s$ is such that $\tilde{\omega} = \df{(\eta_s+\mu)}$.
Thus, according to our procedure to define $F$, the closed differential
$q$-form $\tilde{\omega}$ yields $(\eta_n + \mu) - (\eta_s + \mu) =
\eta_n - \eta_s$.

\stage{ii} We define a function that maps closed
differential $(q-1)$-form on $\displaystyle S^{k-1}_n \cap S^{k-1}_s$
to closed differential $q$-form on $\displaystyle S^{k-1}_n \cup S^{k-1}_s$.

Let $\{ \psi_n, \psi_s\}$ be a partition of unity subordinate to the
open cover $\displaystyle \{ S^{k-1}_n, S^{k-1}_s\}$ of
$\displaystyle S^{k-1}$ such that
$\displaystyle \supp \psi_n \subset S^{k-1}_n$ and
$\displaystyle \supp \psi_s \subset S^{k-1}_s$.
Given a closed differential $(q-1)$-form $\eta$ on
$\displaystyle S^{k-1}_n \cap S^{k-1}_s$, we define two closed
differential $(q-1)$-form $\eta_n$ and $\eta_s$ on $\displaystyle S^{k-1}_n$
and $\displaystyle S^{k-1}_s$ respectively as it follows.
\[
\eta_n(\VEC{u}) = \begin{cases}
\psi_s(\VEC{u}) \eta(\VEC{u}) &
\displaystyle \quad \text{if} \ \VEC{u} \in S^{k-1}_n \setminus
\{\VEC{e}_k \} \\
0 &\quad \text{if} \ \VEC{u} = \VEC{e}_k
\end{cases}
\]
and
\[
\eta_s(\VEC{u}) = \begin{cases}
-\psi_n(\VEC{u}) \eta(\VEC{u}) &
\displaystyle \quad \text{if} \ \VEC{u} \in S^{k-1}_s \setminus
\{-\VEC{e}_k \} \\
0 &\quad \text{if} \ \VEC{u} = -\VEC{e}_k
\end{cases}
\]
These two differential $(q-1)$-forms are well defined and smooth because
$\psi_s$ is null in a neighbourhood of $\VEC{e}_k$ since
its compact support is in $\displaystyle S^{k-1}_s$ and
$\psi_n$ is null in a neighbourhood of $-\VEC{e}_k$ since
its compact support is in $\displaystyle S^{k-1}_n$.
We define a differential $q$-form $\omega$ on $\displaystyle S^{k-1}$
as it follows.
\[
\omega(\VEC{u}) = \begin{cases}
\df{\eta_n}(\VEC{u}) & \displaystyle \quad \text{if} \ \VEC{u} \in S^{k-1}_n \\
\df{\eta_s}(\VEC{u}) & \quad \text{if} \ \VEC{u} = -\VEC{e}_k
\end{cases}
\]
This is a smooth differential $q$-form on
$\displaystyle S^{k-1}$ because
$\df{\eta_n} - \df{\eta_s} = \df{(\eta_n + \eta_s)} = \df{\eta} = 0$
on $\displaystyle S^{k-1}_n \cap S^{k-1}_s$.  It is obviously closed
because $\df{\,\omega} = \df[2]{\eta_n} = 0$ on $\displaystyle S^{k-1}_n$
and $\df{\,\omega} = -\df[2]{\eta_s} = 0$ on $\displaystyle S^{k-1}_s$.

We prove that the map $\displaystyle G: H^{q-1}(S^{k-1}_n \cap S^{k-1}_s)
\to H^q(S^{k-1}_n \cup S^{k-1}_s)$ defined by $G([\eta]) = [\omega]$
is well defined.

Suppose that $\tilde{\eta}$ is another closed differential $(q-1)$-form on
$\displaystyle S^{k-1}_n \cap S^{k-1}_s$ such that $\eta \sim \tilde{\eta}$.
We then have that $\tilde{\eta} = \eta + \df{\tilde{\mu}}$ for some
differential $(q-2)$-form $\tilde{\mu}$ on
$\displaystyle S^{k-1}_n \cap S^{k-1}_s$.  We follow the procedure
describe above.  We set
\[
\tilde{\eta}_n(\VEC{u}) = \begin{cases}
\psi_s(\VEC{u}) \eta(\VEC{u}) + \psi_s(\VEC{u}) \df{\tilde{\mu}} &
\displaystyle \quad \text{if} \ \VEC{u} \in S^{k-1}_n \setminus
\{\VEC{e}_k \} \\
0 &\quad \text{if} \ \VEC{u} = \VEC{e}_k
\end{cases}
\]
and
\[
\tilde{\eta}_s(\VEC{u}) = \begin{cases}
-\psi_n(\VEC{u}) \eta(\VEC{u}) - \psi_n(\VEC{u}) \df{\tilde{\mu}}&
\displaystyle \quad \text{if} \ \VEC{u} \in S^{k-1}_s \setminus
\{-\VEC{e}_k \} \\
0 &\quad \text{if} \ \VEC{u} = -\VEC{e}_k
\end{cases}
\]
to get the differential $q$-form on $\displaystyle S^{k-1}$ defined by
\[
\tilde{\omega}(\VEC{u}) = \begin{cases}
\df{\tilde{\eta}_n}(\VEC{u})
& \displaystyle \quad \text{if} \ \VEC{u} \in S^{k-1}_n \\
\df{\tilde{\eta}_s}(\VEC{u}) & \quad \text{if} \ \VEC{u} = -\VEC{e}_k
\end{cases}
\]
Let $\tilde{\mu}_n$ be the differential $(q-2)$-form on 
$\displaystyle S^{k-1}_n$ defined by
\[
\tilde{\mu}_n(\VEC{u}) = \begin{cases}
\psi_s(\VEC{u}) \tilde{\mu}(\VEC{u}) &
\displaystyle \quad \text{if} \ \VEC{u} \in S^{k-1}_n \setminus \{\VEC{e}_k\} \\
0 &\quad \text{if} \ \VEC{u} = \VEC{e}_k
\end{cases}
\]
and $\tilde{\mu}_s$ be the differential $(q-2)$-form on 
$\displaystyle S^{k-1}_s$ defined by
\[
\tilde{\mu}_s(\VEC{u}) = \begin{cases}
-\psi_n(\VEC{u}) \tilde{\mu}(\VEC{u}) &
\displaystyle \quad \text{if} \ \VEC{u} \in S^{k-1}_s \setminus
\{-\VEC{e}_k\} \\
0 &\quad \text{if} \ \VEC{u} = -\VEC{e}_k
\end{cases}
\]
Since $\psi_s\tilde{\mu} = -\psi_n \tilde{\mu}$ on
$\displaystyle S^{k-1}_n \cap S^{k-1}_s$, we may define a
differential $(q-2)$-form $\mu$ on $\displaystyle S^{k-1}$ by
\[
\mu(\VEC{u}) = \begin{cases}
\tilde{\mu}_n(\VEC{u}) & \displaystyle \quad \text{if} \ \VEC{u} \in
S^{k-1}_n \setminus \{-\VEC{e}_k\} \\
\tilde{\mu}_s(\VEC{u}) &\quad \text{if} \ \VEC{u} = -\VEC{e}_k
\end{cases}
\]
We have that
$\tilde{\omega} = \omega + \df{\tilde{\mu}}$ because
$\tilde{\omega} - \omega = \df{\tilde{\eta}_n}$ on 
$\displaystyle S^{k-1}_n$ and
$\tilde{\omega} - \omega = \df{\tilde{\eta}_s}$ on 
$\displaystyle S^{k-1}_s$.  Thus $\tilde{\omega} \sim \omega$ on
$\displaystyle S^{k-1}$.

\stage{iii}  We have that $\displaystyle F = G^{-1}$.  It is follows
directly from the definitions of $F$ and $G$ that
$\displaystyle F\circ G = \Id: H^{q-1}(S^{k-1}_n \cap S^{k-1}_s) \to 
H^{q-1}(S^{k-1}_n \cap S^{k-1}_s)$.  To prove that 
$\displaystyle G\circ F = \Id: H^q(S^{k-1}_n \cup S^{k-1}_s) \to 
H^q(S^{k-1}_n \cup S^{k-1}_s)$, we consider a differential $q$-form
$\omega$ on $\displaystyle H^q(S^{k-1}_n \cup S^{k-1}_s)$.

According to the definition of $F$, starting with $\omega$, we get a
differential $(q-1)$-form $\eta_n$ on $\displaystyle S^{k-1}_n$ such
that $\omega = \df{\eta_n}$ and a differential $(q-1)$-form $\eta_s$
on $\displaystyle S^{k-1}_s$ such that $\omega = \df{\eta_s}$, and set
$\eta = \eta_n - \eta_s$.  Thus $F([\omega]) = [\eta]$.

According to the definition of $G$, starting with $\eta$, we first
define two differential $(q-1)$-form $\tilde{\eta}_n$ and
$\tilde{\eta}_s$ on $\displaystyle S^{k-1}_n$ and $\displaystyle S^{k-1}_s$ by
\[
\tilde{\eta}_n(\VEC{u}) = \begin{cases}
\psi_s(\VEC{u}) \eta(\VEC{u}) &
\displaystyle \quad \text{if} \ \VEC{u} \in S^{k-1}_n \setminus
\{\VEC{e}_k \} \\
0 &\quad \text{if} \ \VEC{u} = \VEC{e}_k
\end{cases}
\]
and
\[
\tilde{\eta}_s(\VEC{u}) = \begin{cases}
-\psi_n(\VEC{u}) \eta(\VEC{u}) &
\displaystyle \quad \text{if} \ \VEC{u} \in S^{k-1}_s \setminus
\{-\VEC{e}_k \} \\
0 &\quad \text{if} \ \VEC{u} = -\VEC{e}_k
\end{cases}
\]
We then define a differential $q$-form $\omega$ on $\displaystyle S^{k-1}$ by
\[
\tilde{\omega}(\VEC{u}) = \begin{cases} \df{\tilde{\eta}_n}(\VEC{u})
& \displaystyle \quad \text{if} \ \VEC{u} \in S^{k-1}_n \\
\df{\tilde{\eta}_s}(\VEC{u}) & \quad \text{if} \ \VEC{u} = -\VEC{e}_k
\end{cases}
\]
Since $\eta_s + \psi_n\eta = \eta_n - \psi_s\eta$ on
$\displaystyle S^{k-1}_n \cap S^{k-1}_s$, we may define the
differential $(q-1)$-form $\mu$ on $\displaystyle S^{k-1}$ by
\[
\mu(\VEC{u}) = \begin{cases}
\eta_s(\VEC{u}) + \psi_n(\VEC{u})\eta(\VEC{u})
& \displaystyle \quad \text{if} \ \VEC{u} \in S^{k-1}_n \\
\eta_n(\VEC{u}) - \psi_s(\VEC{u})\eta(\VEC{u})
& \quad \text{if} \ \VEC{u} = -\VEC{e}_k
\end{cases}
\]
We have that $\tilde{\omega} = \omega - \df{\mu}$ on
$\displaystyle S^{k-1}$ because
$\tilde{\omega} = \df{(\psi_s \eta)} = \df{(\eta - \psi_n\eta)}
= \df{(\eta_n - \eta_s - \psi_n\eta)}
= \omega - \df{(\eta_s + \psi_n\eta)}$ on $\displaystyle S^{k-1}_n$ and
$\tilde{\omega} = -\df{(\psi_n \eta)} = -\df{(\eta - \psi_s\eta)}
= -\df{(\eta_n - \eta_s - \psi_s\eta)}
= \omega - \df{(\eta_n - \psi_s\eta)}$ on $\displaystyle S^{k-1}_s$.
Hence $\tilde{\omega} \sim \omega$.  Thus $G([\eta]) = [\omega]$ and
$G(F([\omega]) = [\omega]$ as desired.
\end{proof}

Using the stereographic projection $p$ defined in the proof of
Lemma~\ref{lemSnsContr}, we have that
$\displaystyle S^{k-1}_n \cap S^{k-1}_s$ is diffeomorphic to
$\displaystyle \RR^{k-1} \setminus \{\VEC{0}\}$.  Using spherical coordinates 
defined by $\VEC{x} \mapsto (\|\VEC{x}\|^{-1}\VEC{x},\|\VEC{x}\|)$
for $\displaystyle \VEC{x} \in \RR^{k-1} \setminus \{\VEC{0}\}$, we
have that $\displaystyle \RR^{k-1} \setminus \{\VEC{0}\}$ is
diffeomorphic to $\displaystyle S^{k-2} \times ]0,\infty[$.  Therefore,
$\displaystyle S^{k-1}_n \cap S^{k-1}_s$ is diffeomorphic to
$\displaystyle S^{k-2} \times ]0,\infty[$.  It then follows from
Lemma~\ref{lemScStimesR} (slightly modified) that
$\displaystyle H^{q-1}(S^{k-1}_n \cap S^{k-1}_s) \cong
H^{q-1}(S^{k-2} \times ]0,\infty[) \cong H^{q-1}(S^{k-2})$.  Combining this
result with Proposition~\ref{propHqSeHqSns}, we get
\begin{equation}  \label{relHqSk1Hq1Sk2}
H^q(S^{k-1}) \cong H^{q-1}(S^{k-2})
\end{equation}
for $k>1$ and $q>1$.

\begin{prop} \label{propH1Sk}
$\displaystyle H^1(S^1) \cong \RR$ and
$\displaystyle H^1(S^k) = 0$ for $k>1$.
\end{prop}

\begin{proof}
\stage{i} We get from Example~\ref{CnotEpart2} that the dimension of
$\displaystyle H^1(S^1)$ is non null.  Since $\displaystyle S^1$ is a
compact manifold without boundary, we may use
Proposition~\ref{propHkSMequiv} to prove that 
$\displaystyle H^1(S^1) \cong \RR$.

Suppose that $\omega$ is a differential $1$-form on $\displaystyle S^1$
such that $\displaystyle \int_{S^1} \omega = 0$.

Consider the path $\displaystyle \sigma : \RR \to S^1$ defined by
$\sigma(s) = (\cos(2\pi s),\sin(2\pi s))$ for $s \in \RR$.  Let
$\displaystyle g(t) = = \int_0^t \sigma^\ast(\omega)$ for $t \in \RR$.
We have by
hypothesis that $\displaystyle g(0) = g(1) = \int_0^1 \sigma^\ast(\omega)
\int_{\sigma(]0,1[)} \omega = \int_{S^1} \omega = 0$.
Thus $\displaystyle g = f \circ \sigma = \sigma^\ast(f)$
for some functions $\displaystyle f:S^1 \to \RR$.
If we derive both sides of
$\displaystyle \sigma^\ast(f)(t) = g(t) = \int_0^t \sigma^\ast(\omega)$
with respect to $t$, then we get
$\displaystyle \sigma^\ast(\df{f}) = \df{(\sigma^\ast(f))}
= \sigma^\ast(\omega)$.
Thus $\df{f} = \omega$ and therefore $[\omega]= [0]$.  This proves that the
map $M$ in Proposition~\ref{propHkSMequiv} is an isomorphism.

\stage{ii}
The proof is very similar to the proof of Proposition~\ref{DGconserveA}.
The context is a little more general.

Suppose that $\omega$ is a closed differential $1$-form on
$\displaystyle S^k$ with $k>1$.

Choose $\displaystyle \VEC{u}_0 \in S^k$.  Since $\displaystyle S^k$
is path connected, we can select for each
$\displaystyle \VEC{u} \in S^k$ a path $\displaystyle \sigma:[0,1] \to S^k$
between $\VEC{u}_0$ and $\VEC{u}$.
Let
\[
f(\VEC{u}) = \int_\sigma \omega  = \int_{[0,1]}\sigma^\ast(\omega) \ .
\]
This function is well defined because the value of the integral is
independent of the path chosen between $\VEC{u}_0$ and $\VEC{u}$.
To prove this claim, suppose that
$\displaystyle \tilde{\sigma}:[0,1] \to S^k$
is another path between $\VEC{u}_0$ and $\VEC{u}$.  Then
$\sigma - \tilde{\sigma}$ is a closed path in $\displaystyle S^k$
which is the boundary of a region $\displaystyle R \subset S^k$.
Thus, according to Stokes' theorem,
\begin{equation} \label{propH1SkEq1}
\int_\sigma \omega - \int_{\tilde{\sigma}} \omega
= \int_{\sigma - \tilde{\sigma}} \omega
= \int_{\partial R} \omega = \int_R \df{\omega} = 0
\end{equation}
because $\omega$ is closed.  Thus
$\displaystyle \int_{\sigma} \omega = \int_{\tilde{\sigma}} \omega$.
As a little note of caution, we have assumed in (\ref{propH1SkEq1})
that $\sigma([0,1]) \cap \tilde{\sigma}([0,1]) =
\{\VEC{u}_0,\VEC{u}\}$ and that the induced orientation on
$\partial R = \sigma([0,1]) \cup \tilde{\sigma}([0,1])$ from $R$ (and
so $\displaystyle S^k$) matches the orientation provided by
$\sigma - \tilde{\sigma}$.
\pdfbox{cohomology/path1}
We still get that
$\displaystyle \int_{\sigma} \omega = \int_{\tilde{\sigma}} \omega$
even if $\sigma([0,1])$ and $\tilde{\sigma}([0,1])$ intersect at
points other than $\VEC{u}_0$ and $\VEC{u}$ as in the following
figure.
\pdfbox{cohomology/path2}
Stokes' theorem can be used by spitting the integral.
Nevertheless, the conclusion is the same because $\df{\omega} = 0$ on $R$.

We now shows that $\df{f} = \omega$.  Since $\omega$ is arbitrary,
this will prove that $[\omega] = [0]$ and so that
$\displaystyle H^1(S^k) = 0$.

To compute the derivative of $f$ at $\displaystyle \VEC{u}_1 \in S^k$,
we only need to know the value of $f$ in an open neighbourhood $U$ of
$\VEC{u}_1$.  We may assume that $(W,U,\phi)$ is a local chart about
$\VEC{u}_1$.  Let $\sigma_1$ be a path between $\VEC{u}_0$
and $\VEC{u}_1$.  Given $\VEC{u}_2 \in U$, let $\sigma_2$ be a path
between $\VEC{u}_1$ and $\VEC{u}_2$ in $U$.  Then $\sigma_1 + \sigma_2$ is a
path between $\VEC{u}_0$ and $\VEC{u}_2$ that can be used to compute
$f(\VEC{u}_2)$.  Hence
\[
f(\VEC{u}_2) - f(\VEC{u}_1) = \int_{\sigma_1 + \sigma_2} \omega
- \int_{\sigma_1} \omega = \int_{\sigma_2} \omega
= \int_{[0,1]} \sigma_2^\ast(\omega) \ .
\]
Since $\sigma_2([0,1]) \subset U$, we get
\begin{equation} \label{propH1SkEq2}
f(\VEC{u}_2) - f(\VEC{u}_1) = \int_{[0,1]} \sigma_2^\ast(\omega)
= \int_{[0,1]} (\phi^{-1} \circ \sigma_2)^\ast \phi^\ast(\omega) \ .
\end{equation}
Suppose that the local representation of $\omega$ in $W$ is given by
$\displaystyle \phi^\ast \omega = \sum_{j=1}^k \tilde{\omega}_j \df{w_j}$.
If we show that
$\displaystyle \phi^\ast(\df{f}) = \df{(\phi^\ast(f))} = \phi^\ast(\omega)$
on $W$, then we will have that $\df{f} = \omega$ because $\VEC{u}_1$
and the local chart $(W,U,\phi)$ about $\VEC{u}_1$ are arbitrary.

Suppose that $\VEC{u}_1 = \phi(\VEC{w}_1)$.  To prove that
$\displaystyle \df{(\phi^\ast(f))}(\VEC{w}_1) = \phi^\ast(\omega)(\VEC{w}_1)$,
we need to prove that
$\displaystyle \pdfdx{\big(\phi^\ast(f)\big)}{w_j}(\VEC{w}_1) =
\tilde{\omega}_j(\VEC{w}_1)$ for $1\leq j \leq k$.    We have that
\[
\frac{1}{h} \big(( \phi^\ast(f)(\VEC{w}_1 + h\VEC{e}_j) - 
    \phi^\ast(f)(\VEC{w}_1) \big)
= \frac{1}{h} \big( f(\phi(\VEC{w}_1 + h\VEC{e}_j)) -
f(\phi(\VEC{w}_1)) \big)
= \frac{1}{h} \int_{[0,1]} \sigma_h^\ast \phi^\ast(\omega)
\]
where $\sigma_h:[0,1] \to W$ is the path from $\VEC{w}_1$ to
$\VEC{w}_1 + h \VEC{e}_j$ defined by $\sigma_h(s) = \VEC{w}_1 + s h \VEC{e}_j$
for $0\leq s \leq 1$.  We obviously assume that $h$ is small enough.
The path $\sigma_2$ in (\ref{propH1SkEq2}) is given by
$\sigma_2 = \phi \circ \sigma_h$.  We get
\begin{align*}
&\pdfdx{\big(\phi^\ast(f)\big)}{w_j}
= \lim_{h\to 0} \frac{1}{h} \left( \phi^\ast(f)(\VEC{w}_1 + h\VEC{e}_j) - 
    \phi^\ast(f)(\VEC{w}_1) \right)
= \lim_{h\to 0} \frac{1}{h} \int_{[0,1]} \sigma_h^\ast \phi^\ast(\omega) \\
&\qquad = \lim_{h\to 0} \frac{1}{h} \int_{[0,1]} \left( \sum_{j=1}^k
\tilde{\omega}_j(\sigma_h(s)) \left(\pdfdx{(\sigma_h(s))_j}{s}\right)
\dx{s} \right)
= \lim_{h\to 0} \frac{1}{h} \int_{[0,1]} \tilde{\omega}_j(\VEC{w}_1 +
s h\VEC{e}_j) \, h \dx{s} \\
&\qquad = \lim_{h\to 0} \frac{1}{h} \int_0^h \tilde{\omega}_j(\VEC{w}_1 +
t\VEC{e}_j) \dx{t}
= \tilde{\omega}_j(\VEC{w}_1) \ .  \qedhere
\end{align*}
\end{proof}

It follows from (\ref{relHqSk1Hq1Sk2}) and Proposition~\ref{propH1Sk} that
$\displaystyle H^q(S^{k-1}) = 0$ for $1 < q+1 < k$ because
$\displaystyle H^q(S^{k-1}) \cong H^{q-1}(S^{k-2}) \cong H^{q-2}(S^{k-3}) 
\cong \ldots \cong H^1(S^{k-q}) = 0$ for $k-q > 1$
and $\displaystyle H^{k-1}(S^{k-1}) \cong \RR$ for $k >1$ because
$\displaystyle H^{k-1}(S^{k-1}) \cong H^{k-2}(S^{k-2}) \cong \ldots
\cong H^1(S^1) \cong \RR$ for $k>1$.  For the $\displaystyle 0^{th}$
cohomology module on $\displaystyle S^{k-1}$, we have that
$\displaystyle H^0(S^{k-1}) \cong \RR$ for $k>1$ according to
Proposition~\ref{H0kdimS}.

\section{Applications}

There are many very important results that can be obtained using
cohomology.  We present a couple of them in this section.

Let $S_1$ and $S_2$ be two connected and oriented compact
$k$-dimensional smooth manifolds without boundary.  Suppose moreover
that $f:S_1 \to S_2$ is a function of class $\displaystyle C^\infty$.
Is there a relation between $\displaystyle \int_{S_2} \omega$ and
$\displaystyle \int_{S_1} f^\ast(\omega)$ for all
differential $k$-forms $\omega$ on $S_2$?

Choose a differential $k$-form $\omega_0$ on $S_2$ such that
$\displaystyle \int_{S_2} \omega_0 \neq 0$.  Such a differential
$k$-form exists according to Proposition~\ref{propHkSMequiv}
because $\displaystyle H^k(S_2) \cong \RR$ according to
Theorem~\ref{theoHckCR}.  There is a number $n$ such that
\begin{equation} \label{defnDergree}
\int_{S_2} f^\ast(\omega_0) = n \int_{S_2} \omega_0 \ .
\end{equation}

For any other differential $k$-form $\omega$ on $S_2$, we have that
$\omega = a_\omega \omega_0 + \df{\mu_\omega}$ for some differential
$(k-1)$-form $\mu_{\omega}$ on $S_2$ and $a_{\omega} \in \RR$.
Hence, using Stokes' theorem, we find that
\[
\int_{S_2} \omega = a_{\omega} \int_{S_2} \omega_0 + \int_{S_2}\df{\mu}
= a_{\omega} \int_{S_2} \omega_0 + \int_{\partial S_2}\mu
= a_{\omega} \int_{S_2} \omega_0
\]
and
\begin{align*}
\int_{S_1} f^\ast(\omega) &= a_{\omega} \int_{S_1} f^\ast(\omega_0)
+ \int_{S_1}f^\ast(\df{\mu})
= a_{\omega} \int_{S_1} f^\ast(\omega_0) + \int_{S_1}\df{(f^\ast(\mu))} \\
&= a_{\omega} \int_{S_1} f^\ast(\omega_0) + \int_{\partial S_1}f^\ast(\mu)
= a_{\omega} \int_{S_1} f^\ast(\omega_0)
\end{align*}
because $\partial S_i = \emptyset$ for $1\leq i \leq 2$.
Thus
\[
\int_{S_1} f^\ast(\omega) = a_{\omega} \int_{S_1} f^\ast(\omega_0)
= n a_{\omega} \int_{S_2} \omega_0 = n \int_{S_2} \omega
\]
for all differential $k$-form $\omega$ on $S_2$.
The constant $n$ depends only on $f$.  This in itself is pretty
amazing.  But there is something more amazing to come.

The previous discussion is also valid if we consider connected and
oriented $k$-dimensional smooth manifolds $S_1$ and $S_2$ without
boundary, and assume that the function $f:S_1 \to S_2$ is of class
$\displaystyle C^\infty$ and {\bfseries proper}\index{Proper Function}; namely,
$\displaystyle f^{-1}(K)$ is compact if $K$ is compact.  Obviously, we
have in that context to consider $\displaystyle H_c^k(S_i) \cong \RR$
for $1\leq i \leq 2$ instead of $\displaystyle H^k(S_i) \cong \RR$
for $1\leq i \leq 2$.

\begin{defn}
The number $n$ defined in (\ref{defnDergree}) is called the
{\bfseries degree}\index{Degree of a Function} of $f$ and is denoted
$\deg(f)$.
\end{defn}

The surprising fact is that $\deg(f)$ is an integer number.

\begin{theorem} \label{thmDegree}
Let $S_1$ and $S_2$ be two connected and oriented $k$-dimensional
smooth manifolds without boundary, and $f:S_1 \to S_2$ be a
proper function of class $\displaystyle C^\infty$.  Suppose that $\VEC{v}$ is a
regular value of $f$ and set, for each
$\displaystyle \VEC{u} \in f^{-1}(\{\VEC{v}\})$,
$\sgn_{\VEC{u}}(f)= 1$ if $f_\ast: \TS_{\VEC{u}} S_1 \to \TS_{\VEC{v}} S_2$
is orientation preserving and
$\sgn_{\VEC{u}}(f)= -1$ if $f_\ast: \TS_{\VEC{u}} S_1 \to \TS_{\VEC{v}} S_2$
is not orientation preserving.  Then
\[
\deg(f) = \sum_{\VEC{u} \in f^{-1}(\{\VEC{v}\})} \sgn_{\VEC{u}}(f) \ .
\]
\end{theorem}

\begin{proof}
It follows from Proposition~\ref{regValCoIm} that
$\displaystyle f^{-1}(\{\VEC{v}\})$ is the union of isolated points
\footnote{We leave it to the reader to generalize
Proposition~\ref{regValCoIm} to a map between manifolds.
Note that $\VEC{u}_2 \in S_2$ is a regular value of a continuously
differentiable function $f:S_1 \to S_2$ between two $k$-dimensional
smooth manifolds $S_1$ and $S_2$ if
$\displaystyle \phi_2^{-1}(\VEC{u}_2)$ is a regular value of
$\displaystyle \phi_2^{-1} \circ f \circ \phi_1$ for all local charts
$(W_1,U_1,\phi_1)$ of $S_1$ and local charts $(W_2,U_2,\phi_2)$ of
$S_2$ with $\displaystyle \phi_2^{-1}(\VEC{u}_2) \in W_2$.
The fact that countable atlases can be used according to 
Proposition~\ref{BjUilocfin} is also needed to generalize
Proposition~\ref{regValCoIm}.}.
Since $f$ is proper, $\displaystyle f^{-1}(\{\VEC{v}\})$ is also a compact
set.  Therefore, $\displaystyle f^{-1}(\{\VEC{v}\})$ is the finite
union of isolate points.

Suppose that $\displaystyle f^{-1}(\{\VEC{v}\}) = \{ \VEC{u}_1,
\VEC{u}_2, \ldots, \VEC{u}_q\}$.  We now present one method
(suggested in \cite{Sv1}) to select
local charts $\{ (W_i,U_i,\phi_i)\}_{1\leq i \leq q}$ of $S_1$ and a
local chart $(W, U,\phi)$ of $S_2$ such that
$U_i \cap U_j = \emptyset$ for $i\neq j$, $\VEC{u}_i \in U_i$ for
$1\leq i \leq q$, $\VEC{v} \in U$, and
$\displaystyle f^{-1}(U) = \bigcup_{1\leq i \leq q} U_i$

Let $\{ (\tilde{W}_i,\tilde{U}_i,\tilde{\phi}_i)\}_{1\leq i \leq q}$
be local charts of $S_1$ with $\VEC{u}_i \in \tilde{U}_i$ for
$1\leq i \leq q$ and $(\tilde{W}, \tilde{U},\tilde{\phi})$ be a local
chart of $S_2$ such that $\VEC{v} \in \tilde{U}$.  Since the points 
$\VEC{u}_i$ are isolated, we may assume by shrinking the sets
$\tilde{U}_i$ if needed that
$\tilde{U}_i \cap \tilde{U}_j = \emptyset$ if $i \neq j$.

According to Proposition~\ref{topolProp2}, we may choose an open
neighbourhood $\breve{U}$ of $\VEC{v}$ with compact closure
$K_1 \subset \tilde{U}$.
Since $f$ is a proper function, $\displaystyle f^{-1}(K_1)$ is a
compact subset of $S_1$.  Thus $\displaystyle K_2 = f^{-1}(K_1) \setminus
\big( \bigcup_{1\leq i \leq q} \tilde{U}_i\big) \subset f^{-1}(\tilde{U})$ is
a compact subset of $S_1$.  Since $f$ is continuous, $K_3 = f(K_2)$ is a
compact subset of $\tilde{U}$.  So $K_3$ is a closed subset of $\tilde{U}$ that
does not contain $\VEC{v}$.

The local charts that we were looking for are defined as it follows.
Let $U = \breve{U} \setminus K_3$ and $\displaystyle W = \phi^{-1}(U)$.
The local chart of $S_2$ that contains $\VEC{v}$ is given by $(W,U,\phi)$.
Let $\displaystyle U_i = \tilde{U}_i \cap f^{-1}(U)$ for $1\leq i \leq q$.
Note that $\VEC{u}_i \in U_i$ for $1\leq i \leq q$ because
$\VEC{v} \in U$.
The local charts of $S_1$ are given by
$\displaystyle (W_i,U_i,\phi_i) = (\phi_i^{-1}(U_i),U_i,\phi_i)$
for $1\leq i \leq q$.
\pdfbox{cohomology/degree1}

Let $g:W \to \RR$ be a non null smooth function with compact support
in $W$ and such that $g(\VEC{w}) \geq 0$ for all $\VEC{w} \in W$.
We consider the differential $k$-form $\omega$ on $S_2$ with the local
representation $\displaystyle \phi^\ast(\omega) = g \df{w_1} \wedge
\df{w_2} \wedge \ldots \wedge \df{x_k}$ on $U$ and null on
$S_2 \setminus U$.  Since
$\displaystyle \supp f^\ast(\omega) \subset f^{-1}(\supp \omega)
\subset f^{-1}(U) = \bigcup_{1\leq i \leq q} U_i$ and
$U_i \cap U_j = \emptyset$ for $i \neq j$, we get
\[
\int_{S_1} f^\ast(\omega) = \sum_{i=1}^q \int_{U_i} f^\ast(\omega)
= \sum_{i=1}^q \left(\sgn_{\VEC{u}_i}(f) \int_U \omega \right)
= \Big(\underbrace{\sum_{i=1}^q \sgn_{\VEC{u}_i}(f)}_{=n}\Big) \int_U \omega
= n \int_{S^2} \omega
\]
because
$\displaystyle \int_{U_i} f^\ast(\omega) = \int_U \omega$ if $f$ is
orientation preserving and
$\displaystyle \int_{U_i} f^\ast(\omega) = -\int_U \omega$ if $f$ does
not preserve the orientation (i.e.\ reverse the orientation)
according to Proposition~\ref{propCVforDF}.

Note that $\phi$ and $\phi_i$ for $1\leq i \leq q$ are
orientation preserving, and $\det (\phi\circ f \circ \phi_i) \neq 0$
on $W_i$ for $1\leq i \leq q$.  Therefore, either
$f_\ast: \TS_{\VEC{u}} S_1 \to \TS_{\VEC{v}} S_2$
is orientation preserving for all $\VEC{u} \in U_i$ or
$f_\ast: \TS_{\VEC{u}} S_1 \to \TS_{\VEC{v}} S_2$
reverses the orientation for all $\VEC{u} \in U_i$.
To change the orientation on $W_i$,
$\det (\phi\circ f \circ \phi_i)$
would have to change sign and so be null at some point in $W_i$.
\end{proof}

\begin{cor}  \label{cordegFdegG}
Let $S_1$ and $S_2$ be two connected and oriented compact $k$-dimensional
smooth manifolds without boundary.  If $f,g:S_1 \to S_2$ are two
homotopic functions of class $\displaystyle C^\infty$, then $\deg(f) = \deg(g)$.
\end{cor}

\begin{proof}
Since $f$ and $g$ are homotopic, there exists a smooth map
$H:S_1 \times [0,1] \to S_2$ such that $H(\VEC{u},0) = f(\VEC{u})$ and
$H(\VEC{u},1) = g(\VEC{u})$ for all $\VEC{u} \in S_1$.
Let $H_t(\VEC{u}) = H(\VEC{u},t)$ for all $(\VEC{u},t) \in S_1 \times [0,1]$.
If $\omega$ is a differential $k$-form on $S_2$, then we have that
\[
\int_{S_1} H_t^\ast(\omega) = n_t \int_{S_2} \omega
\]
for each $t \in [0,1]$ where $n_t \in \ZZ$.  Since the value of the
integral on the left side varies continuously with respect to $t$, we
have that $t \mapsto n_t$ is an integer valued continuous function.
Therefore $n_t$ is constant.  Hence $\deg(f) = n_0 = n_1 = \deg(g)$.
\end{proof}

We have defined vector fields on a $k$-dimensional manifold at the
beginning of Section~\ref{sectVFandDF} but we have not referred to
them until now.  The next result is very interesting.  It is often
humorously stated as saying that there is no way to comb our hair and
not have one hair standing perpendicular.  This is true if we assume
that one's head is a ball fully covered with hair.  But, other than
for cartoon characters, we have not seen any such head yet.

\begin{prop} \label{propNNVFodd}
There does not exist a nowhere null vector field on $\displaystyle S^{k-1}$ 
if $k$ is odd.
\end{prop}

\begin{proof}
If $\displaystyle g_0:S^{k-1} \to S^{k-1}$ is the restriction to
$\displaystyle S^{k-1}$ of the identity map
$\displaystyle \Id:\RR^k \to \RR^k$,
then $\deg(g_0) = 1$ because $\det \Id = 1$.
  
If $\displaystyle g_1:S^{k-1} \to S^{k-1}$ is the restriction to
$\displaystyle S^{k-1}$ of the map $\displaystyle -\Id:\RR^k \to \RR^k$,
so $g_1$ is the antipodal map defined by $g(\VEC{u}) =-\VEC{u}$
for all $\displaystyle \VEC{u} \in S^{k-1}$,
then $\deg(g_1) = -1$ because $\det (-\Id) = -1$ for $k$ odd.
\pdfbox{cohomology/nnVF}

We have from Corollary~\ref{cordegFdegG} that $g_1$ cannot be
smoothly homotopic to $g_0$.

Suppose that there is a nowhere null vector field
$\displaystyle F : S^{k-1} \to \bigcup_{\VEC{u}\in S} \TS_{\VEC{u}} S^{k-1}$.
Let $\displaystyle F(\VEC{u}) = (\VEC{u},f(\VEC{u})) \in
\TS_{\VEC{u}} S^{k-1}$ for all $\displaystyle \VEC{u} \in S^{k-1}$.
We have that $\displaystyle f:S^{k-1} \to \RR^k$ is a smooth function
such that $f(\VEC{u}) \neq \VEC{0}$ for all
$\displaystyle \VEC{u} \in S^{k-1}$ because $F$ is nowhere null.

Let $\displaystyle H:S^{k-1} \times [0,1] \to S^{k-1}$ be the
function defined by
\[
H(\VEC{u},t) = \cos(\pi t) \VEC{u} + \sin(\pi t)
\, \|f(\VEC{u}\|^{-1} f(\VEC{u})
\]
for all $\displaystyle (\VEC{u},t) \in S^{k-1}$.
The function $H$ is an homotopy between $g_0$ and $g_1$.  This is a
contradiction.
\pdfbox{cohomology/nnVF2}
Note that $\| H(\VEC{u},t)\| = 1$ for all
$\displaystyle (\VEC{u},t) \in S^{k-1}$ because
$f(\VEC{u})$ is perpendicular to $\VEC{u}$ by definition of
$\displaystyle \TS_{\VEC{u}} S^{k-1}$.  Hence, if we set
$\displaystyle \VEC{y} = \|f(\VEC{u})\|^{-1} f(\VEC{u})$ for
$\displaystyle \VEC{u} \in S^{k-1}$, then we get
\begin{align*}
\| H(\VEC{u},t\|^2 &= \sum_{j=1}^k (\cos(\pi t) u_j + \sin(\pi t) y_j)^2 \\
&= \cos^2(\pi t) \underbrace{\|\VEC{u}\|^2}_{=1}
+ 2 \cos(\pi t) \sin(\pi t) \underbrace{\VEC{u} \cdot \VEC{y}}_{=0}
+ \sin^2(\pi t) \underbrace{\|\VEC{y}\|^2}_{=1} = 1
\end{align*}
for $t \in [0,1]$.  In fact, $t \mapsto H(\VEC{u},t)$ is a parametric
representation of half the great circle from $\VEC{u}$ to $-\VEC{u}$
that goes through $\displaystyle \|f(\VEC{u})\|^{-1}f(\VEC{u})$.  
\end{proof}

If $k$ is even, then it is always possible to define a nowhere null
vector field of $\displaystyle S^{k-1}$.  For instance,
$\displaystyle F:S^{k-1} \to \bigcup_{\VEC{u}\in S} \TS_{\VEC{u}} S^{k-1}$
defined by $F(\VEC{u}) = (\VEC{u}, f(\VEC{u})$ for
$\displaystyle \VEC{u} \in S^{k-1}$ where
$f(\VEC{u}) = (-u_2,u_1,-u_4,u_3, \ldots, -u_k,u_{k-1})$ is such a
vector field.  It is represented in the figure below for $\displaystyle S^1$.
\pdfbox{cohomology/nnVF3}

%%% Local Variables:
%%% mode: latex
%%% TeX-master: "notes"
%%% End:
