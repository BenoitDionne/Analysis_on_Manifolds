\chapter{The Riemann Integral of Real Valued Functions of
Several Variables} \label{chapMultInt}

\section{Definition and Properties}

As usual, we start with some basic definitions.

\begin{defn}
A {\bfseries rectangle}\index{Rectangle}
$\DS R \subset \RR^n$ is a set of
the form $\DS R = \prod_{i=1}^n I_i$ where the $I_i$ are
bounded intervals. The {\bfseries volume}\index{Volume} of a rectangle
$\DS R = \prod_{i=1}^n I_i$
is the value $\DS V(R) = \prod_{i=1}^n |I_i|$ where $|I_i|$
is the length of the interval $I_i$.  We may use the notation 
$A(R)$ instead of $V(R)$ when $n=2$.
\end{defn}

The definition of sets of zero content, sets of measure zero and
(Jordan) measurable sets in $\RR$ (Definitions~\ref{measureR}
and \ref{JordanMS}) can be extended to $\DS \RR^n$.

\begin{defn} \label{measure}
A set $\DS S\in \RR^n$ has
{\bfseries zero content}\index{Zero Content} if
for every $\epsilon>0$ there exists a finite collection of closed rectangles
$\DS \{ R_i \}_{0\leq i \leq M}$ such that
$\DS S \subset \bigcup_{i=0}^M R_i$ and
$\DS \sum_{i=0}^M V(R_i) < \epsilon$.

A set $\DS S\in \RR^n$ has
{\bfseries measure zero}\index{Measure Zero} if
for every $\epsilon>0$ there exists a countable collection of closed
rectangles $\DS \{ R_i \}_{i\geq 0}$ such that
$\DS S \subset \bigcup_{i=0}^\infty R_i$ and
$\DS \sum_{i=0}^\infty V(R_i) < \epsilon$.
\end{defn}

\begin{defn}
A set $\DS S \subset \RR^n$ is
{\bfseries (Jordan) measurable}\index{Jordan Measurable} if $S$ is bounded
and $\partial S$ has measure zero.
\end{defn}

\begin{rmk}
In the definition of sets of measure zero or zero content, we
may use open rectangles instead of closed rectangles.  Suppose that
$\DS S \subset \RR^n$ has measure zero according to the
definition using closed rectangles.  Given $\epsilon > 0$, there exists a
countable collection
$\DS \{R_i\}_{i\geq 0}$ of closed rectangles such that
$\DS S \subset \bigcup_{i=0}^\infty R_i$ and
$\DS \sum_{i=0}^\infty V(R_i) < \epsilon/2$.
For each $i$, choose an open rectangle $V_i$ such that $R_i \subset V_i$
and $\DS V(V_i) \leq V(R_i) + \epsilon/2^{i+2}$.  Then
$\DS S \subset \bigcup_{i=0}^\infty R_i \subset
\bigcup_{i=0}^\infty V_i$ and
\[
\sum_{i=0}^\infty V(V_i) \leq \sum_{i=0}^\infty V(R_i) + \frac{\epsilon}{4}
\sum_{i=0}^\infty \frac{1}{2^i} = \sum_{i=0}^\infty V(R_i) +
\frac{\epsilon}{2} < \epsilon \ .
\]
Since $\epsilon$ is arbitrary, $S$ has measure zero according to the
definition using open rectangles.

Conversely, suppose that $\DS S \subset \RR^n$ has measure
zero according to the definition using open rectangles.  Given
$\epsilon > 0$, there exists a countable collection
$\DS \{V_i\}_{i\geq 0}$ of open rectangles such that
$\DS S \subset \bigcup_{i=0}^\infty V_i$ and
$\DS \sum_{i=0}^\infty V(V_i) < \epsilon$.
For each $i$, let $R_i = \overline{V_i}$.  Then $V(R_i) = V(V_i)$ for all
$i$.  Hence $\DS S \subset \bigcup_{i=0}^\infty V_i \subset
\bigcup_{i=0}^\infty R_i$ and
$\DS \sum_{i=0}^\infty V(R_i) = \sum_{i=0}^\infty V(V_i) < \epsilon$.
Since $\epsilon$ is arbitrary, $S$ has measure zero according to the
definition using closed rectangles.
\end{rmk}

Given a bounded set $\DS V\subset \RR^n$, we define the diameter of
$V$ as the number
\[
  \diam V = \sup \{ \|\VEC{x} - \VEC{y}\| : \VEC{x}, \VEC{y} \in V \} \ .
\]

\begin{prop}
If, for every $\epsilon>0$, there exists a countable collection of
compact sets $\DS \{ K_i \}_{i\geq 0}$ such that
$\DS S \subset \bigcup_{i=0}^\infty K_i$ and
$\DS \sum_{i=0}^\infty \diam K_i < \epsilon$,
then the set $\DS S\in \RR^n$ has measure zero.
\end{prop}

\begin{proof}
Suppose that $S$ is a set satisfying the hypotheses of the
proposition.  Given $\epsilon >0$, choose a countable collection of
compact sets $\DS \{ K_i \}_{i\geq 0}$ such that
$\DS S \subset \bigcup_{i=0}^\infty K_i$ and
$\DS \sum_{i=0}^\infty \diam K_i < \epsilon^{1/n}$.

Each $K_i$ is a subset of a rectangle $R_i$ with edges of length
$d_i = \diam K_i$.  Thus $\DS V(R_i) = d_i^n$.  Hence
$\DS S \subset \bigcup_{i=0}^\infty K_i
\subset \bigcup_{i=0}^\infty R_i$ and
\[
\sum_{i=0}^\infty V(R_i) = \sum_{i=0}^\infty d_i^n \leq
\left( \sum_{i=0}^\infty d_i \right)^n < \left( \epsilon^{1/n}\right)^n
= \epsilon \ .
\]
Since $\epsilon$ is arbitrary, the definition of sets of measure zero
is satisfied.
\end{proof}

\begin{prop} \label{PropMziZC}
Compact subsets of a set of measure zero have zero content.
\end{prop}

\begin{proof}
Suppose that $Z$ is a set of measure zero and that $K\subset Z$ is a
compact set.  We prove that $K$ has zero content.  Given
$\epsilon >0$, since $Z$ has measure zero, there exists a countable
collection $\DS \{V_i\}_{i\geq 0}$ of open rectangles
such that
$\DS K \subset Z \subset \bigcup_{i=0}^\infty V_i$
and $\DS \sum_{i=0}^\infty V(V_i) < \epsilon$.
Since $K$ is compact, there exists a finite subcover
$\DS \{V_{i_j}\}_{0\leq j \leq N}$ such that
$\DS K \subset \bigcup_{j=0}^N V_{i_j}$.  We obviously
have
$\DS \sum_{j=0}^N V(V_{i_j}) \leq \sum_{i=0}^\infty V(V_i) <
\epsilon$.  Since $\epsilon$ is arbitrary, this shows that $K$ has zero
content.
\end{proof}

\begin{defn}
A {\bfseries partition}\index{Partition} $P$ of a rectangle
$\DS R = \prod_{i=1}^n I_i \subset \RR^n$, where each $I_i$
is an interval, is a collection $P = \{ P_i : 1 \leq i \leq n\}$
where $P_i = \{p_{k,i} : 0 \leq k \leq N_i\}$ is a partition
of the interval $I_i$ for $1 \leq i \leq n$.

Rectangles of the form
$\DS R_{k_1,k_2,\ldots,k_n} = \prod_{i=1}^n [p_{k_i-1,i},p_{k_i,i}]$,
where $1 \leq k_i \leq N_i$, are called
{\bfseries subrectangles of the partition}\index{Subrectangles of the
Partition} $P$. 
\end{defn}

We define the integral of a bounded function on a rectangle
in basically the same way that we defined the integral of a bounded
function on an interval.

\begin{defn}
Let $\DS R \subset \RR^n$ be a rectangle as defined above,
$f:R\to \RR$ be a bounded function, and $P$ be a partition of $[a,b]$.
Moreover, let $\DS m_{k_1,k_2,\ldots,k_n}
= \inf\{ f(\VEC{x}) : \VEC{x} \in  R_{k_1,k_2,\ldots,k_n}\}$
and $\DS M_{k_1,k_2,\ldots,k_n}
= \sup\{ f(\VEC{x}) : \VEC{x} \in  R_{k_1,k_2,\ldots,k_n}\}$
for $1 \leq k_i \leq N_i$ and $1 \leq i \leq n$.  The
{\bfseries lower Riemann sum}\index{Lower Riemann Sum} of $f$ with
respect to the partition $P$ is
\[
  \LL_P(f) = \sum_{\substack{1 \leq k_i \leq N_i\\1 \leq i \leq n}}
m_{k_1,k_2,\ldots,k_n} V(R_{k_1,k_2,\ldots,k_n}) \ .
\]
The {\bfseries upper Riemann sum}\index{Upper Riemann Sum} of $f$ with
respect to the partition $P$ is
\[
\U_P(f) = \sum_{\substack{1 \leq k_i \leq N_i\\1 \leq i \leq n}}
M_{k_1,k_2,\ldots,k_n} V(R_{k_1,k_2,\ldots,k_n}) \ .
\]
\end{defn}

\begin{defn}
A {\bfseries refinement}\index{Refinement} for a partition
$P = \{P_i : 1 \leq i \leq n\}$ of a rectangle $R$ is another
partition $Q = \{Q_i : 1 \leq i \leq n\}$ of $R$ such that
$Q_i$ is a refinement of $P_i$ for $1 \leq i \leq n$.
\end{defn}

The lower and upper Riemann sums for functions of several variables
have the same properties than the lower and upper Riemann sums for
functions of one variable.

\begin{defn}
Let $\DS R \subset \RR^n$ be a rectangle as defined above and 
$f:R \to \RR$ be a bounded function.  The 
{\bfseries lower Riemann integral}\index{Lower Riemann Integral} of
$f$ on $R$ is defined as
\[
\LL(f) = \sup \{ \LL_P(f) : P \text{ is a partition of } R \} \ .
\]
The {\bfseries upper Riemann integral}\index{Upper Riemann Integral}
of $f$ on $R$ is defined as
\[
\U(f) = \inf \{ \U_P(f) : P \text{ is partition of } R \} \ .
\]
\end{defn}

\begin{defn}
Let $\DS R \subset \RR^n$ be a rectangle as defined above and
$f:R\to \RR$ be a bounded function.  If $\U(f) = \LL(f)$, then we say that
{\bfseries $\mathbf{f}$ is (Riemann) integrable}\index{Riemann Integrable}
on the rectangle $R$.  The
{\bfseries Riemann integral of $\mathbf{f}$}\index{Riemann
Integral of a Function} on $R$ is defined as
\[
\int_R f = \int_R f(\VEC{x})\dx{\VEC{x}} = \LL(f) = \U(f) \ .
\]
\end{defn}

Most of the results that we have stated about the Riemann integral of a
bounded function on an interval remain true for the Riemann integral of a
bounded function on a rectangle.  The obvious exception is
Theorem~\ref{thCVin1D} that we will reformulate shortly.

Suppose that $R$ is a rectangle in $\DS \RR^n$ and
$f:R\to \RR$ is a bounded function.  Given $\VEC{x} \in R$, let
\begin{align*}
M(\VEC{x},f,\delta) &= \sup \{ f(\VEC{y}) : \VEC{y}\in R \ \text{and}
\ \|\VEC{y}-\VEC{x}\|<\delta\} \ , \\
m(\VEC{x},f,\delta) &= \inf \{ f(\VEC{y}) : \VEC{y} \in R \ \text{and}
\ \|\VEC{y}-\VEC{x}\|<\delta\}
\end{align*}
and
\[
o(\VEC{x},f) = \lim_{\delta \to 0^+}
\left(M(\VEC{x},f,\delta) - m(\VEC{x},f,\delta) \right) \ .
\]
The limit always exists because
$M(\VEC{x},f,\delta) - m(\VEC{x},f,\delta) > 0$
is decreasing as $\delta$ decreases to $0$.  It is also easy to show that
$f$ is continuous at $\VEC{x}$ if and only if $o(\VEC{x},f)=0$.

The following result will be useful shortly.

\begin{lemma} \label{closedLemma}
Suppose that $\DS R \subset \RR^n$ is a closed set.  The set
$\DS
Z_\epsilon = \left\{ \VEC{x} \in R : o(\VEC{x},f) \geq \epsilon \right\}$
is a closed subset of $R$.
\end{lemma}

\begin{proof}
If $z_\epsilon = \emptyset$, then $Z_\epsilon$ is closed.

Suppose that $Z_\epsilon \neq \emptyset$.  We prove that the
complement of $Z_\epsilon$ in $R$ is open relative to the topology
induced on $R$ from $\DS \RR^n$.
Given $\VEC{x} \in R \setminus Z_\epsilon$, there exists $\delta >0$
such that $M(\VEC{x},f,\delta) - m(\VEC{x},f,\delta) < \epsilon$ since
$o(\VEC{x},f) < \epsilon$.  It suffices to show that
$B_\delta(\VEC{x}) \cap R \subset R \setminus Z_\epsilon$ to
prove that $R \setminus Z_\epsilon$ is open.

Given $\VEC{y} \in B_\delta(\VEC{x}) \cap R$, choose
$\delta_y < \delta - \|\VEC{x} - \VEC{y}\|$.  We have
$B_{\delta_y}(\VEC{y}) \subset B_\delta(\VEC{x})$.  Therefore,
\[
  M(\VEC{y},f,\delta_y) - m(\VEC{y},f,\delta_y) <
  M(\VEC{x},f,\delta) - m(\VEC{x},f,\delta) < \epsilon \ .
\]
Since $M(\VEC{y},f,\delta_y) - m(\VEC{y},f,\delta_y) \geq 0$ decreases
as $\delta_y$ decreases, we have
$o(\VEC{y},f) < \epsilon$.  Thus $\VEC{y} \in R \setminus Z_\epsilon$.
\end{proof}

\begin{lemma}\label{zeroLemma}
Let $R$ be a closed rectangle in $\DS \RR^n$ and $f:R\to \RR$
be a bounded function such that $o(\VEC{x},f) < \epsilon$ for all
$\VEC{x} \in R$.  Then there exists a partition $P$ of $R$ such that
$\U_P(f) - \LL_P(f) < \epsilon V(R)$.
\end{lemma}

\begin{proof}
Since $o(\VEC{x},f) < \epsilon$ for all $\VEC{x} \in R$, there exists
for every $\VEC{x} \in R$ a radius $\delta_x$ such that
$M(\VEC{x},f,\delta_x) - m(\VEC{x},f,\delta_x) < \epsilon$.
For each $\VEC{x} \in R$, choose an open rectangle
$R_{\VEC{x}}$ such that $\VEC{x} \in R_{\VEC{x}} \subset B_{\delta_x}(\VEC{x})$.
The collection $\DS \{ R_{\VEC{x}} : \VEC{x} \in R \}$
is an open cover of the closed and bounded set $R$.  Since $R$ is
compact, there exists a finite subcover
$\DS \{ R_{\VEC{x}_i} \}_{1 \leq i \leq N}$ of $R$.

Suppose that $\DS
R_{\VEC{x}_i} = \prod_{j=1}^n ]a_{i,j},b_{i,j}[$ for $1 \leq i \leq N$.
Choose a partition $P$ of $R$ such that
$a_{i,j},b_{i,j} \in P_i$ for all $1 \leq j \leq N_i$ and $1 \leq i \leq N$
such that $(a_{i,j},b_{i,j}) \in R$.
For each $n$-tuple $(k_1,k_2,\ldots,k_n)$, there exists at least one $\VEC{x}_i$
such that $\DS R_{k_1,k_2,\ldots,k_n} \subset R_{\VEC{x}_i}$.
\pdfbox{mult_integrals/coverR}
Therefore
\[
M_{k_1,k_2,\ldots,k_n} - m_{k_1,k_2,\ldots,k_n}
< M(\VEC{x}_i,f,\delta_{x_i}) - m(\VEC{x}_i,f,\delta_{x_i}) < \epsilon
\]
for all $n$-tuples $(k_1,k_2,\ldots,k_n)$.  It follows that
\begin{align*}
\U_P(f) - \LL_P(f)
&= \sum_{\substack{1 \leq k_i \leq N_i\\1 \leq i \leq n}}
(M_{k_1,k_2,\ldots,k_n} - m_{k_1,k_2,\ldots,k_n})
V(R_{k_1,k_2,\ldots,k_n}) \\
&< \epsilon \sum_{\substack{1 \leq k_i \leq N_i\\1 \leq i \leq n}}
V(R_{k_1,k_2,\ldots,k_n})
= \epsilon V(R) \ . \qedhere
\end{align*}
\end{proof}

\begin{theorem}\label{Rexists}
Let $R$ be a rectangle in $\DS \RR^n$ and $f:R\to \RR$ be
a bounded function.  The set of all points in $R$ where $f$ is
discontinuous has measure zero if and only if $f$ is Riemann
integrable on $R$.
\end{theorem}

\begin{proof}
Suppose that $Z \subset R$ is the set of points where $f$ is discontinuous.
Let \\
$\DS Z_\nu = \left\{ \VEC{x} \in R : o(\VEC{x},f) \geq \nu \right\}$
for $\nu > 0$.

Since $f$ is bounded, there exists $M>0$ such that $|f(\VEC{x})|<M$ for all
$\VEC{x} \in R$.

\stage{$\Rightarrow$} If $Z = \emptyset$, then $f$ is continuous on
$R$ and therefore Riemann integrable on $R$.  Suppose that
$Z \neq \emptyset$.

Given $\epsilon >0$, we have $Z_{\epsilon/(2V(R))} \subset Z$.
We may assume that $\epsilon$ is small enough to have
$Z_{\epsilon/(2V(R))} \neq \emptyset$ because $Z \neq \emptyset$.  Since
$Z_{\epsilon/(2V(R))}$ is closed
according to Lemma~\ref{closedLemma} and bounded, it
is compact.  Since $Z_{\epsilon/(2V(R))}$ is a compact subset of $Z$ which
has measure zero, it follows from Proposition~\ref{PropMziZC} that
$Z_{\epsilon/(2V(R))}$ has zero content.  Thus, there exists a
finite collection $\DS \{R_i\}_{0 \leq i \leq N}$ of closed
rectangles in $R$ such that
$\DS Z_{\epsilon/(2V(R))} \subset \bigcup_{i=0}^N R_i$ and
\begin{equation} \label{zeroEqu4}
\DS \sum_{i=0}^N V(R_i) < \frac{\epsilon}{4M} \ .
\end{equation}

As in the proof of Lemma~\ref{zeroLemma}, we choose a partition $P$ of
$R$ such that each subrectangle $R_{k_1,k_2,\ldots,k_n}$ of the 
partition $P$ with
$R_{k_1,k_2,\ldots,k_n} \cap Z_{\epsilon/(2V(R))} \neq \emptyset$ satisfies
$R_{k_1,k_2,\ldots,k_n} \subset R_i$ for some $i$.  We group the
subrectangles of the partition $P$ into two collections: $S_1$ is the
collection of all subrectangles $R_{k_1,k_2,\ldots,k_n}$ of the
partition $P$ such that
$R_{k_1,k_2,\ldots,k_n} \cap Z_{\epsilon/(2V(R))} \neq \emptyset$, and
$S_2$ is the collection of all the other subrectangles.

For each subrectangle $R_{k_1,k_2,\ldots,k_n}$ of the partition $P$, we
define as before
$\DS M_{k_1,k_2,\ldots,k_n}
= \sup \{ f(\VEC{x}) : \VEC{x} \in R_{k_1,k_2,\ldots,k_n} \}$ and
$\DS m_{k_1,k_2,\ldots,k_n}
= \inf \{ f(\VEC{x}) : \VEC{x} \in R_{k_1,k_2,\ldots,k_n} \}$.
We have $M_{k_1,k_2,\ldots,k_n} - m_{k_1,k_2,\ldots,k_n} < 2M$
for every subrectangle $R_{k_1,k_2,\ldots,k_n}$ of the partition $P$. 
Hence
\begin{equation}\label{zeroEqu1}
\begin{split}
&\sum_{R_{k_1,k_2,\ldots,k_n} \in S_1} (M_{k_1,k_2,\ldots,k_n}
- m_{k_1,k_2,\ldots,k_n}) V(R_{k_1,k_2,\ldots,k_n}) \\
&\qquad \qquad
\leq 2M \sum_{R_{k_1,k_2,\ldots,k_n} \in S_1} V(R_{k_1,k_2,\ldots,k_n})
\leq 2M \sum_{i=1}^N V(R_i) < \frac{\epsilon}{2}
\end{split}
\end{equation}
because of (\ref{zeroEqu4}).

Given a subrectangle $R_{k_1,k_2,\ldots,k_n} \in S_2$, we have
$o(\VEC{x},f) < \epsilon/(2V(R))$ for all
$\VEC{x}\in R_{k_1,k_2,\ldots,k_n}$ because
$R_{k_1,k_2,\ldots,k_n} \cap Z_{\epsilon/(2V(R))} = \emptyset$.
Hence, it follows from Lemma~\ref{zeroLemma} that there exists 
a refinement $\tilde{P}$ of the partition $P$ such that
\begin{equation}\label{zeroEqu2}
\sum_{\tilde{R}_{j_1,j_2,\ldots,j_n}\subset R_{k_1,k_2,\ldots,k_n}}
\left(\tilde{M}_{j_1,j_2,\ldots,j_n} - \tilde{m}_{j_1,j_2,\ldots,j_n}
\right) V(\tilde{R}_{j_1,j_2,\ldots,j_n}) <
\frac{\epsilon V(R_{k_1,k_2,\ldots,k_n})}{2V(R)}
\end{equation}
where the $\tilde{R}_{j_1,j_2,\ldots,j_n}$ are
subrectangles of the partition $\tilde{P}$ and,
as expected,
$\DS \tilde{M}_{j_1,j_2,\ldots,j_n}
= \sup \{ f(\VEC{x}) : \VEC{x} \in \tilde{R}_{j_1,j_2,\ldots,j_n} \}$ and
$\DS \tilde{m}_{j_1,j_2,\ldots,j_n}
= \inf \{ f(\VEC{x}) : \VEC{x} \in \tilde{R}_{j_1,j_2,\ldots,j_n} \}$.

Repeating inductively this procedure of refinement of the partition for
each $R_{k_1,k_2,\ldots,k_n} \in S_2$, we end up with a refinement of
$P$ that we still call $\tilde{P}$ such that (\ref{zeroEqu2}) is true
for all $R_{k_1,k_2,\ldots,k_n} \in S_2$.  Note that a version of
Proposition~\ref{propRef} in $\DS \RR^n$ is used to justify the last 
statement.

The relation (\ref{zeroEqu1}) reminds true for the refinement
$\tilde{P}$ because
\begin{align}
&\sum_{\substack{\tilde{R}_{j_1,j_2,\ldots,j_n}\subset
R_{k_1,k_2,\ldots,k_n}\\R_{k_1,k_2,\ldots,k_n} \in S_1}}
\left(\tilde{M}_{j_1,j_2,\ldots,j_n} - \tilde{m}_{j_1,j_2,\ldots,j_n}\right)
V(\tilde{R}_{j_1,j_2,\ldots,j_n}) \nonumber \\
&\qquad \qquad \leq 2M \sum_{\substack{\tilde{R}_{j_1,j_2,\ldots,j_n}\subset
R_{k_1,k_2,\ldots,k_n}\\R_{k_1,k_2,\ldots,k_n} \in  S_1}}
V(\tilde{R}_{j_1,j_2,\ldots,j_n})
= 2M \sum_{R_{k_1,k_2,\ldots,k_n} \in S_1} V(R_{k_1,k_2,\ldots,k_n})
\nonumber \\
&\qquad \qquad \leq 2M \sum_{i=1}^N V(R_i) < \frac{\epsilon}{2} \ ,
\label{zeroEqu3} 
\end{align}
where (\ref{zeroEqu4}) was used to get the last inequality and
$\tilde{R}_{j_1,j_2,\ldots,j_n}$ are subrectangles
of the partition $\tilde{P}$.

Finally, it follows from (\ref{zeroEqu2}) and (\ref{zeroEqu3}) that
\begin{align*}
\U_{\tilde{P}}(f) - \LL_{\tilde{P}}(f) &=
\sum_{\substack{\tilde{R}_{j_1,j_2,\ldots,j_n}\subset
R_{k_1,k_2,\ldots,k_n}\\R_{k_1,k_2,\ldots,k_n} \in S_1}}
\left(\tilde{M}_{j_1,j_2,\ldots,j_n} - \tilde{m}_{j_1,j_2,\ldots,j_n}\right)
V(\tilde{R}_{j_1,j_2,\ldots,j_n}) \\
&\qquad
+ \sum_{\substack{\tilde{R}_{j_1,j_2,\ldots,j_n}\subset
R_{k_1,k_2,\ldots,k_n}\\R_{k_1,k_2,\ldots,k_n} \in S_2}} 
\left(\tilde{M}_{j_1,j_2,\ldots,j_n} - \tilde{m}_{j_1,j_2,\ldots,j_n}\right)
V(\tilde{R}_{j_1,j_2,\ldots,j_n}) \\
&< \frac{\epsilon}{2} + \frac{\epsilon}{2 V(R)}
\sum_{R_{k_1,k_2,\ldots,k_n} \in S_2} V(R_{k_1,k_2,\ldots,k_n})
\leq \frac{\epsilon}{2} + \frac{\epsilon}{2} = \epsilon \ .
\end{align*}
Since $\epsilon$ is arbitrary, it follows from the version of 
Theorem~\ref{intCauchyCR} in $\DS \RR^n$ that $f$ is integrable.

\stage{$\Leftarrow$} We have
\[
  Z = \bigcup_{m=1}^\infty Z_{1/m} \ .
\]
We prove that each $Z_{1/m}$ has measure zero.  Hence $Z$ will also
be of measure zero since the countable union of sets of measure zero
is a set of measure zero.

Choose $\epsilon > 0$.  Let $P$ be a partition of $R$ such that
$\DS \U_P(f) - \LL_P(f) < \epsilon/(2m)$.  Moreover, let $S_1$
be the collection of all subrectangles $R_{k_1,k_2,\ldots,k_n}$ of the
partition $P$ such that
$\DS R_{k_1,k_2,\ldots,k_n}^\circ \cap Z_{1/n} \neq \emptyset$.
For each $R_{k_1,k_2,\ldots,k_n} \in S_1$, there is at least one
point of $Z_{1/m}$ which is in the interior of $R_{k_1,k_2,\ldots,k_n}$.
Therefore, we get
$\DS  M_{k_1,k_2,\ldots,k_n} - m_{k_1,k_2,\ldots,k_n} \geq 1/m$
for each $R_{k_1,k_2,\ldots,k_n} \in S_1$ \footnote{The condition
$\DS R_{k_1,k_2,\ldots,k_n} \cap Z_{1/m} \neq \emptyset$ is
not sufficient to ensure that
$\DS  M_{k_1,k_2,\ldots,k_n} - m_{k_1,k_2,\ldots,k_n}
\geq 1/m$.  We may have that
$\DS \VEC{x} \in R_{k_1,k_2,\ldots,k_n} \cap Z_{1/m}
\subset \partial R_{k_1,k_2,\ldots,k_n}$ with
$M_{k_1,k_2,\ldots,k_n} - m_{k_1,k_2,\ldots,k_n} < 1/m$ 
but $o(\VEC{x},f) \geq 1/m$ because of the values of $f$ in 
$Z_{1/m} \setminus R_{k_1,k_2,\ldots,k_n}$.}.
Since
\begin{align*}
\frac{1}{m} \sum_{R_{k_1,k_2,\ldots,k_n}\in S_1} V(R_{k_1,k_2,\ldots,k_n})
&\leq \sum_{R_{k_1,k_2,\ldots,k_n} \in S_1}
\left(M_{k_1,k_2,\ldots,k_n} - m_{k_1,k_2,\ldots,k_n}\right)
V(R_{k_1,k_2,\ldots,k_n}) \\
&\leq \U_P(f) - \LL_P(f) < \frac{\epsilon}{2m} \ ,
\end{align*}
we get
\begin{equation}\label{multintzeroA}
  \sum_{R_{k_1,k_2,\ldots,k_n}\in S_1} V(R_{k_1,k_2,\ldots,k_n})
< \frac{\epsilon}{2} \ .
\end{equation}

Let $\DS B
= \bigcup_{R_{k_1,k_2,\ldots,k_n}} \partial R_{k_1,k_2,\ldots,k_n}$.
The union is over all rectangles $R_{k_1,k_2,\ldots,k_n}$ generated by
the partition $P$ of $R$.  The set $B$ is
compact and has zero content since it is composed of a finite number of
line segments.  Choose a finite collection $S_2$ of closed rectangles
such that $\DS B \subset \bigcup_{T\in S_2} T$ and
\begin{equation}\label{multintzeroB}
  \sum_{T\in S_2} V(T) < \frac{\epsilon}{2}
\end{equation}

We have by construction that
\[
  Z_{1/m} \subset \bigcup_{R_{k_1,k_2,\ldots,k_n}\in S_1} R_{k_1,k_2,\ldots,k_n}
\cup \bigcup_{T\in S_2} T \ .
\]
Moreover, we get from (\ref{multintzeroA}) and (\ref{multintzeroB}) that
\[
\sum_{R_{k_1,k_2,\ldots,k_n}\in S_1} V(R_{k_1,k_2,\ldots,k_n})
+ \sum_{T\in S_2} V(T) < \frac{\epsilon}{2} +
  \frac{\epsilon}{2} = \epsilon \ .
\]
Since $\epsilon$ is arbitrary, we obtain that $Z_{1/m}$ has zero
content and so is of measure zero.
\end{proof}

Given $\DS W \subset \RR^n$,
let $\DS \Chi_W:\RR^n \to \RR$ be
the function defined by
\[
\Chi(\VEC{x}) = \begin{cases}
1 & \quad \text{if} \ \VEC{x} \in W \\
0 & \DS \quad \text{if} \ \VEC{x} \in \RR^n\setminus W
\end{cases}
\]
The only points where $\Chi_W$ is discontinuous are the points on
$\partial W$.

\begin{defn}
Let $\DS W \subset \RR^n$ be a Jordan measurable set.  Given
a bounded function $f:W\to \RR$, we define the
{\bfseries Riemann integral of $f$ on $W$}\index{Riemann
Integral of a Function}, denoted $\DS \int_W f$ or
$\DS \int_W f(\VEC{x})\dx{\VEC{x}}$, as the integral
$\DS \int_R \Chi_W f$ if this integral exists
\footnotemark\ where $R$ is a closed rectangle containing $W$.
We say that $f$ is {\bfseries Riemann integrable}\index{Riemann
Integrable} on $W$.
\end{defn}

\footnotetext{It is understood that $f$ is extended to $R$.  For
instance, we may set $f(\VEC{x}) = 0$ for $\VEC{x} \in R\setminus W$.}

This definition of the integral of $f:W \to \RR$ is independent of the
rectangle $R$ containing $W$ since $\Chi_W(\VEC{x}) f(\VEC{x}) = 0$ for
$\VEC{x} \in R \setminus W$.  Moreover, we have that $\partial W$ is
of measure zero because $W$ is a Jordan measurable set.  Hence, the
set of points where $\Chi_W f:R \to \RR$ is discontinuous has measure
zero if and only if the set of points where $f:W \to \RR$ is
discontinuous has measure zero for the same reasons that we have used
to prove this result for functions of one variable.

\begin{prop}
Let $R$ be a rectangle in $\DS \RR^n$ and suppose that
$Z\subset R$ has zero content.  If $h:Z \to \RR$ is a bounded
function, then $h$ is Riemann integrable on $Z$ and
$\DS \int_Z h = 0$.
\end{prop}

\begin{proof}
Let $m = \inf \{ h(\VEC{x}) : \VEC{x} \in Z\}$ and
$M = \sup \{ h(\VEC{x}) : \VEC{x} \in Z\}$.

\stage{i} We first prove that $h$ is integrable on $R$.
We give two proofs of this property.

The first proof uses Theorem~\ref{Rexists}.  The set of points where
$\DS \Chi_Z h: R\to \RR$ is discontinuous has zero content
because it is a subset of $\overline{Z}$ which has zero content
\footnote{Given $\epsilon >0$, if $\{R_i\}_{0\leq i \leq N}$ is a
collection of closed rectangles such that
$\DS Z \subset \bigcup_{i=0}^N R_i$ and
$\DS \sum_{i=0}^N V(R_i) < \epsilon$, then
$\DS \overline{Z} \subset \bigcup_{i=0}^N R_i$ and we still
have $\DS \sum_{i=0}^N V(R_i) < \epsilon$.}.
Hence $\overline{Z}$ has measure zero.  It then follows from
Theorem~\ref{Rexists} that $\Chi_Z h$ is integrable on $R$.

The second proof does not use Theorem~\ref{Rexists}.
Given $\epsilon > 0$, there exists a finite collection of
open rectangles $\DS \{ R_i\}_{0 \leq i \leq N}$ such that
$\DS Z \subset \bigcup_{i=0}^N R_i$ and
$\DS \sum_{i=0}^N V(R_i) < \frac{\epsilon}{M-m}$.

Choose a partition $P$ of $R$ such that the vertices of all the
rectangles $R_i$ are included in $P$.  We then have
$\DS M_{k_1,k_2,\ldots,k_n} = \sup \{ \Chi_Z(\VEC{x}) h(\VEC{x}) :
 \VEC{x} \in R_{k_1,k_2,\ldots,k_n} \} =0$ and
$\DS m_{k_1,k_2,\ldots,k_n} = \inf \{ \Chi_Z(\VEC{x}) h(\VEC{x}) :
 \VEC{x} \in R_{k_1,k_2,\ldots,k_n} \} = 0$ for all
subrectangles $\DS R_{k_1,k_2,\ldots,k_n}$ of the partition
$P$ such that $\DS R_{k_1,k_2,\ldots,k_n} \cap R_i = \emptyset$
for all $i$.  Hence
\begin{align*}
\U_P(\Chi_Z h) - \LL_P(\Chi_Z h)
&= \sum_{\substack{R_{k_1,k_2,\ldots,k_n} \cap R_i \neq \emptyset\\
\text{for some } 0 \leq i \leq N}} \big( M_{k_1,k_2,\ldots,k_n}
- m_{k_1,k_2,\ldots,k_n} \big) V(R_{k_1,k_2,\ldots,k_n}) \\
&\leq (M-m) \sum_{\substack{R_{k_1,k_2,\ldots,k_n} \cap R_i \neq \emptyset\\
\text{for some } 0 \leq i \leq N}} V(R_{k_1,k_2,\ldots,k_n})
\leq (M-m) \sum_{i=0}^N V(R_i)
< \epsilon \ .
\end{align*}
Note that the last equality comes from the fact that
$\DS R_{k_1,k_2,\ldots,k_n} \cap R_i \neq \emptyset$ implies
that $\DS R_{k_1,k_2,\ldots,k_n}^\circ \subset R_i$ because
that $R_i$ are open.  Hence
\[
\bigcup_{\substack{R_{k_1,k_2,\ldots,k_n} \cap R_i \neq \emptyset\\
\text{for some } 0 \leq i \leq N}} R_{k_1,k_2,\ldots,k_n}^\circ
= \bigcup_{i=0}^N R_i \ .
\]

It follows from the version of Theorem~\ref{intCauchyCR} in
$\DS \RR^n$ that $\Chi_Z h$ is integrable on $R$.

\stage{ii} To prove that $\DS \int_Z h = 0$, we consider
$\DS h = h^+ - h^-$ where
\[
h^+(\VEC{x}) = \begin{cases}
h(\VEC{x}) & \quad \text{if} \ h(\VEC{x}) > 0 \\
0 & \quad \text{otherwise}
\end{cases}
\qquad \text{and} \qquad
h^-(\VEC{x}) = \begin{cases}
-h(\VEC{x}) & \quad \text{if} \ h(\VEC{x}) < 0 \\
0 & \quad \text{otherwise}
\end{cases}
\]
It follows from (i) that $\DS h^+$ and $\DS h^-$
are integrable on $R$ and
\[
\int_Z h  = \int_R \Chi_Z h 
= \int_R \Chi_Z h^+ - \int_R \Chi_Z h^-
= \int_Z h^+ - \int_Z h^- \ .
\]
It suffices to prove that $\DS \int_Z h^+$ and
$\DS \int_Z h^-$ are null.

To prove that $\DS \int_Z h^+$ is null, we prove that
$\DS \LL_P(\Chi_Z h^+) = 0$ for all partitions $P$ of $R$.
Let $P$ be a partition of $R$ and let
$\DS \delta = \min_{\substack{0<k_i\leq N_i\\1 \leq i \leq n}}
V(R_{k_1,k_2,\ldots,k_n})$.  We have that $\delta >0$ because there
is a finite number of subrectangles.
Since $Z$ has measure zero, there exists a finite collection of
open rectangles $\DS \{ R_i\}_{0 \leq i \leq N}$ such that
$\DS Z \subset \bigcup_{i=0}^N R_i$ and
$\DS \sum_{i=0}^N V(R_i) < \delta$.
We have
\[
m_{k_1,k_2,\ldots,k_n} = \inf \{ \Chi_Z(\VEC{x})h^+(\VEC{x}) :
 \VEC{x} \in R_{k_1,k_2,\ldots,k_n} \} = 0
\]
for every subrectangle $R_{k_1,k_2,\ldots,k_n}$ of the partition $P$
because
$\DS R_{k_1,k_2,\ldots,k_n} \setminus Z \supset
R_{k_1,k_2,\ldots,k_n} \setminus \bigcup_{i=1}^\infty R_i \neq \emptyset$
since
\[
V\left(\bigcup_{i=1}^\infty R_i\right)
\leq \sum_{i=1}^\infty V(R_i) < \delta \leq V(R_{k_1,k_2,\ldots,k_n}) \ .
\]
Thus $\DS \LL_P(\Chi_Z h^+ ) = 0$.  The proof that
$\DS \LL_P(\Chi_Z h^-) = 0$ for all partitions $P$ of $R$ is similar.
\end{proof}

\begin{rmk}
The previous proposition is also true if $Z$           \label{rmkC0meas}
is a closed set of measure zero.  We have that the set of points where
$\DS \Chi_Z h: R\to \RR$ is discontinuous has measure zero 
because it is a subset of $\overline{Z} = Z$ which has measure zero.
Therefore, it follows from Theorem~\ref{Rexists} that
$\Chi_Z h$ is integrable on $R$.  The proof that
$\DS \int_Z h = 0$ does not change.

It follows that the next corollary is also true if $Z$ is a closed set
of measure zero.

In fact, it is not hard to prove that the previous proposition and the
next corollary are true if $Z$ is a subset of a closed set of measure
zero.
\end{rmk}

\begin{cor} \label{corIfeIgZ0}
Let $R$ be a rectangle in $\DS \RR^n$.  Suppose that
$f,g:R \to \RR$ are two Riemann integrable functions such that
the set $Z = \{ \VEC{x} \in R : f(\VEC{x}) \neq g(\VEC{x}) \}$ has
zero content.  Then $\DS \int_R f = \int_R g$.
\end{cor}

\begin{proof}
Let $h=f-g$.  We have $h = \Chi_Z h$.  From the previous
proposition, we get
\[
\int_R f - \int_R g = \int_R(f-g) = \int_R h = \int_R \Chi_Z h = 0 \ . \qedhere
\]
\end{proof}

\section{Bounded Curves in $\DS \mathbf{\RR^n}$}

A {\bfseries curve}\index{Curve} in $\DS \RR^n$ is the image
of a continuous function $\DS \sigma:[a,b]\to \RR^n$.
To simplify the discussion, we will freely refer to $\sigma$ as a
curve though the curve is its image.  A curve $\sigma$ is
{\bfseries closed}\index{Closed Curve} if $\sigma(a)=\sigma(b)$. 
The domain of $\sigma$ could also be a semi-open or open interval.

A bounded curve in $\DS \RR^n$ should have zero content
(measure zero), should it not?  Unfortunately, the following example
due to Peano shows that this is not true.

\begin{egg}
The {\bfseries Peano's curve}\index{Peano's Curve} is the following
closed curve $\sigma:[0,1]\to [0,1]\times[0,1]$ that fills up the
entire unite square $[0,1]\times[0,1]$.  The curve $\sigma$ is the
uniform limit of the closed curves $\sigma_i$ for $i\geq 1$ defined as
it follows.
\pdfbox{mult_integrals/peano}
The curve $\sigma_i$ contains the centre of all the rectangles 
$\DS \left[\frac{j_1-1}{2^i}, \frac{j_1}{2^i}\right]\times
\left[\frac{j_2-1}{2^i}, \frac{j_2}{2^i}\right]$ for
$\DS 1\leq j_1, j_2 \leq 2^i$ and $i \geq 1$.
By choosing a proper parametric representation $\sigma_i$ on $[0,1]$
for each $i$, one can show that
$\DS \sup_{0\leq x\leq 1} \|\sigma_i(x) - \sigma_{i+1}(x)\| \leq
\frac{1}{2^{i+1}}$ for $i \geq 1$ as it is illustrated in green in the
figure above for $i=1$.
\end{egg}

However, if we require that the curve
$\DS \sigma:[a,b]\to \RR^n$ be continuously differentiable,
then the image of the curve $\sigma$ must have zero content.

\begin{prop}\label{curveC1zero}
Suppose that $\DS \sigma:[a,b]\to \RR^n$ belongs to
$\DS C^1([a,b])$.  Then $\sigma([a,b])$ has zero content.
\end{prop}

\begin{proof}
We prove the proposition for $\DS \sigma:[a,b]\to \RR^2$
only.  The proof is (almost) identical for
$\DS \sigma:[a,b]\to \RR^n$.

Let $\DS C = \sup \{ \|\sigma'(t)\| : t \in [a,b] \}$.
Note that $C < \infty$ because $\sigma'$ is continuous on the compact set
$[a,b]$.  Moreover,
\[
| \sigma_i'(t)| \leq \|\sigma'(t) \| =
\sqrt{(\sigma_1'(t))^2 + (\sigma_2(t))^2} \leq C
\]
for all $t\in [a,b]$ and $i=1$ or $2$.

Given $\epsilon>0$, choose $K>0$ large enough such that
$\DS 2C^2(b-a)^2/K < \epsilon$.  Let
$T=\{t_i: 0 \leq i \leq K\}$ be the partition of $[a,b]$ with
$t_i = a+i(b-a)/K$ for $0 \leq i \leq K$.

Given $0<i\leq K$ and $1 \leq j \leq 2$, we have for
every $s_1, s_2 \in [t_{i-1},t_i]$ that there exists
$\xi_{i,j} = \xi_{i,j}(s_1,s_2)$ between $s_1$ and $s_2$ such that
\[
| \sigma_j(s_1) - \sigma_j(s_2)| = |\sigma_j'(\xi_{i,j})|\,|s_1-s_2|
\leq C(b-a)/K \ .
\]
Hence
\[
\|\sigma(s_1) - \sigma(s_2)\|
= \sqrt{ (\sigma_1(s_1) - \sigma_1(s_2))^2 + (\sigma_2(s_1) - \sigma_2(s_2))^2}
\leq \sqrt{2} C(b-a)/K
\]
for every $s_1$ and $s_2$ in $[t_{i-1},t_i]$ and $0 < i \leq K$.
Therefore, for each $0 < i \leq K$, we can choose a rectangle $R_i$
with sides parallel to the axis and
$\DS A(R_i) \leq \left( \sqrt{2} C(b-a)/K\right)^2
= 2C^2(b-a)^2/K^2$ such that
$\sigma([t_{i-1},t_i]) \subset R_i$.  Thus
$\DS \sigma([a,b]) \subset \bigcup_{i=1}^K R_i$ with
\[
\sum_{i=1}^K A(R_i) \leq \sum_{i=1}^K \frac{2C^2(b-a)^2}{K^2}
= \frac{2C^2(b-a)^2}{K} < \epsilon \ .
\]
Since $\epsilon>0$ is arbitrary, this proves that $\sigma([a,b])$ is
of zero content.
\end{proof}

\section{Iterated Integrals}

The next theorem is one of the fundamental results about the
integration of functions of several variables.

\begin{theorem}[Fubini]
Let $I_1$ and $I_2$ be two intervals in $\RR$, and
$f:I_1\times I_2\to \RR$ be a Riemann integrable function.
\begin{enumerate}
\item For $x_1\in I_1$, let $f_{x_1}:I_2\to \RR$ be the function
defined by $f_{x_1}(x_2) = f(x_1,x_2)$ for $x_2\in I_2$.  Let
$h(x_1) = \LL(f_{x_1})$ (the lower
Riemann integral of $f_{x_1}$ on $I_2$) and $H(x_1) = \U(f_{x_1})$
(the upper Riemann integral of $f_{x_1}$ on $I_2$) for $x_1\in I_1$.
Then $h$ and $H$ are integrable on $I_1$ and
\[
\int_{I_1\times I_2} f  = \int_{I_1} h(x_1)\dx{x_1}
= \int_{I_1} H(x_1) \dx{x_1} \ .
\]
\item For $x_2\in I_2$, let $f_{x_2}:I_1\to \RR$ be the function
defined by $f_{x_2}(x_1) = f(x_1,x_2)$ for $x_1\in I_1$.  Let
$g(x_2) = \LL(f_{x_2})$ (the lower
Riemann integral of $f_{x_2}$ on $I_1$) and $G(x_2) = \U(f_{x_2})$
(the upper Riemann integral of $f_{x_2}$ on $I_1$) for $x_2\in I_2$.
Then $g$ and $G$ are integrable on $I_2$ and
\[
\int_{I_1\times I_2} f = \int_{I_2} g(x_2)\dx{x_2}
= \int_{I_2} G(x_2) \dx{x_2} \ .
\]
\end{enumerate}
\end{theorem}

\begin{proof}
We prove only (1), the proof of (2) is similar.

Let $\DS P_j=\{ p_{i,j} : 0 \leq i \leq N_j\}$ be a partition
of $I_j$ for $j=1$ and $2$.  Then $P= \{P_1,P_2\}$
is a partition of $R = I_1\times I_2$.  The subrectangles $R_{i,j}$ of the
partition $R$ are of the form
$R_{i,j} = [p_{i-1,1},p_{i,1}]\times [p_{j-1,2},p_{j,2}]$.  Let
$A(R_{i,j})$ be the area of the subrectangle $R_{i,j}$.

\stage{i} Let $m_{i,j} = \inf \{ f(x_1,x_2) : (x_1,x_2) \in R_{i,j} \}$.
We have
\begin{align}
\LL_P(f) &=
\sum_{\substack{1\leq i\leq N_1\\1\leq j \leq N_2}} m_{i,j} A(R_{i,j}) =
\sum_{\substack{1\leq i\leq N_1\\1\leq j \leq N_2}} m_{i,j}
 (p_{j,2}-p_{j-1,2})(p_{i,1}-p_{i-1,1})
\nonumber \\
&= \sum_{i=1}^{N_1} \left( \sum_{j=1}^{N_2} m_{i,j} (p_{j,2}-p_{j-1,2})
\right)(p_{i,1}-p_{i-1,1}) \ .  \label{fub1}
\end{align}

If $m_j(x_1) = \inf \{ f_{x_1}(x_2) : p_{j-1,2} \leq x_2 \leq p_{j,2}\}$
for $0 < j \leq N_2$, then
$m_j(x_1) \geq m_{i,j}$ for all $x_1 \in [p_{i-1,1},p_{i,1}]$.  Hence
\[
\sum_{j=1}^{N_2} m_{i,j} (p_{j,2}-p_{j-1,2}) \leq 
\sum_{j=1}^{N_2} m_j(x_1) (p_{j,2}-p_{j-1,2}) = \LL_{P_2}(f_{x_1})
\leq \LL(f_{x_1}) = h(x_1)
\]
for all $x_1\in [p_{i-1,1},p_{i,1}]$.
It follows from (\ref{fub1}) that
\begin{equation} \label{fub2}
\LL_P(f) \leq \sum_{i=1}^{N_1} h(x_1^{[i]}) (p_{i,1}-p_{i-1,1})
\end{equation}
for all $\DS x_1^{[i]} \in [p_{i-1,1},p_{i,1}]$
with $0 < i \leq N_1$.  If $\DS
m_i = \inf \{ h(x_1^{[i]}) : p_{i-1,1}\leq x_1^{[i]} \leq p_{i,1} \}$ for all $i$,
then it follows from (\ref{fub2}) that
\begin{equation} \label{fub3}
\LL_P(f) \leq \sum_{i=1}^N m_i (p_{i,1}-p_{i-1,1}) = \LL_{P_1}(h) \ .
\end{equation}

\stage{ii} Let $M_{i,j} = \sup \{ f(x_1,x_2) : (x_1,x_2) \in R_{i,j} \}$.
We have
\begin{align}
\U_P(f) &=
\sum_{\substack{1\leq i\leq N_1\\1\leq j \leq N_2}} M_{i,j} A(R_{i,j}) =
\sum_{\substack{1\leq i\leq N_1\\1\leq j \leq N_2}} M_{i,j}
 (p_{j,2}-p_{j-1,2})(p_{i,1}-p_{i-1,1})
\nonumber \\
&= \sum_{i=1}^{N_1} \left( \sum_{j=1}^{N_2} M_{i,j} (p_{j,2}-p_{j-1,2})
\right)(p_{i,1}-p_{i-1,1}) \ .  \label{fub1v2}
\end{align}

If $M_j(x_1) = \sup \{ f_{x_1}(x_2) : p_{j-1,2} \leq x_2 \leq p_{j,2}\}$
for $0 < j \leq N_2$, then
$M_j(x_1) \leq M_{i,j}$ for all $x_1 \in [p_{i-1,1},p_{i,1}]$.  Hence
\[
\sum_{j=1}^{N_2} M_{i,j} (p_{j,2}-p_{j-1,2}) \geq
\sum_{j=1}^{N_2} M_j(x_1) (p_{j,2}-p_{j-1,2}) = \U_{P_2}(f_{x_1})
\geq \U(f_{x_1}) = H(x_1)
\]
for all $x_1\in [p_{i-1,1},p_{i,1}]$.
It follows from (\ref{fub1v2}) that
\begin{equation} \label{fub2v2}
\U_P(f) \geq \sum_{i=1}^{N_1} H(x_1^{[i]}) (p_{i,1}-p_{i-1,1})
\end{equation}
for all $\DS x_1^{[i]} \in [p_{i-1,1},p_{i,1}]$
with $0 < i \leq N_1$.  If $\DS
M_i = \sup \{ H(x_1^{[i]}) : p_{i-1,1}\leq x_1^{[i]} \leq p_{i,1} \}$ for all $i$,
then it follows from (\ref{fub2v2}) that
\begin{equation} \label{fub3v2}
\U_P(f) \geq \sum_{i=1}^N M_i (p_{i,1}-p_{i-1,1}) = \U_{P_1}(H) \ .
\end{equation}

\stage{iii} We have
\begin{equation}\label{fub7}
\LL_P(f) \leq \LL_{P_1}(h) \leq \U_{P_1}(h) \leq \U_{P_1}(H) \leq \U_P(f) \ .
\end{equation}
The first inequality is (\ref{fub3}), the second inequality follows from the
definition of the lower and upper Riemann sums, the third inequality is true
because $h(x_1) = \LL(f_{x_1}) \leq \U(f_{x_1}) = H(x_1)$ for all $x_1 \in I_1$,
and the last inequality follows from (\ref{fub3v2}).

Choose $\epsilon>0$.  Since $f$ is integrable on $R$, there exists a
partition $P=(P_1,P_2)$ of $R$ such that $\U_P(f)-\LL_P(f) < \epsilon$.
We get from (\ref{fub7}) that
\[
\U_{P_1}(h) - \LL_{P_1}(h) \leq \U_P(f) - \LL_P(f) < \epsilon \ .
\]
Since $\epsilon$ is arbitrary, it follows from the version of
Theorem~\ref{intCauchyCR} in $\DS \RR^n$ that
$h:I_1\to \RR$ is integrable.  Moreover, we get from (\ref{fub7}) that
$\LL_P(f) \leq \LL_{P_1}(h) \leq \LL(h)$ for all partitions $P$ of
$R$.  Thus $\LL(f) \leq \LL(h)$.  Similarly, we get from (\ref{fub7})
that $\U(h) \leq \U_{P_1}(h) \leq \U_P(f)$ for all partitions $P$ of $R$.
Thus $\U(h) \leq \U(f)$.  We have shown that
$\LL(f) \leq \LL(h) \leq \U(h) \leq \U(f)$.  Since $f$ is integrable on
$R$, we have $\DS \LL(f) = \U(f) = \int_R f$.  Hence
\[
\int_{I_1} h(x_1) \dx{x_1} = \LL(h) = \U(h) = \LL(f) = \U(f) = \int_R f \ .
\]

\stage{iv} Similarly to (iii), we have
\begin{equation}\label{fub7v2}
\LL_P(f) \leq \LL_{P_1}(h) \leq \LL_{P_1}(H) \leq \U_{P_1}(H) \leq \U_P(f) \ .
\end{equation}
The first inequality is (\ref{fub3}), the second inequality is true
because $h(x_1) = \LL(f_{x_1}) \leq \U(f_{x_1}) = H(x_1)$ for all $x_1 \in I_1$,
the third inequality follows from the
definition of the lower and upper Riemann sums, and the last inequality follows
from (\ref{fub3v2}).

Choose $\epsilon>0$.  Since $f$ is integrable on $R$, there exists a
partition $P=(P_1,P_2)$ of $R$ such that $\U_P(f)-\LL_P(f) < \epsilon$.
We get from (\ref{fub7v2}) that
\[
\U_{P_1}(H) - \LL_{P_1}(H) \leq \U_P(f) - \LL_P(f) < \epsilon \ .
\]
Since $\epsilon$ is arbitrary, it follows from the version of
Theorem~\ref{intCauchyCR} in $\DS \RR^n$ that
$H:I_1\to \RR$ is integrable.  Moreover, we get from (\ref{fub7v2}) that
$\U_P(f) \geq \U_{P_1}(H) \geq \U(H)$ for all partitions $P$ of
$R$.  Thus $\U(f) \geq \U(H)$.  Similarly, we get from (\ref{fub7v2})
that $\LL(H) \geq \LL_{P_1}(H) \geq \LL_P(f)$ for all partitions $P$ of $R$.
Thus $\LL(H) \geq \LL(f)$.  We have shown that
$\LL(f) \leq \LL(H) \leq \U(H) \leq \U(f)$.  Since $f$ is integrable on
$R$, we have $\DS \LL(f) = \U(f) = \int_R f$.  Hence
\[
\int_{I_1} H(x_1) \dx{x_1} = \LL(H) = \U(H) = \LL(f) = \U(f) = \int_R f \ .
\qedhere
\]
\end{proof}

\begin{rmkList}
\begin{enumerate}
\item If $f:I_1 \times I_2 \to \RR$ is Riemann integrable and
$f_{x_1}$ is Riemann integrable for all $x_1$, then
$\DS h(x_1) = H(x_1) = \int_{I_2} f(x_1,x_2) \dx{x_2}$
and we get from Fubini's Theorem that
\[
\int_{I_1 \times I_2} f = \int_{I_1}
\left(\int_{I_2} f(x_1,x_2)\dx{x_2}\right)\dx{x_1} \ .
\]
This is called an {\bfseries iterated integral}\index{Iterated Integral}.
\item Similarly to (1), if $f:I_1 \times I_2 \to \RR$ is Riemann integrable and
$f_{x_2}$ is Riemann integrable for all $x_2$, then
$\DS g(x_2) = G(x_2) = \int_{I_1} f(x_1,x_2) \dx{x_1}$ and
we get from Fubini's Theorem that
\[
\int_{I_1\times I_2} f = \int_{I_2}
\left(\int_{I_1} f(x_1,x_2)\dx{x_1}\right)\dx{x_2} \ .
\]
This is another {\bfseries iterated integral}\index{Iterated Integral}.
\item (1) and (2) are satisfied if $f:I_1 \times I_2 \to \RR$ is
continuous because $f_{x_1}:I_2 \to \RR$ and $f_{x_2}:I_1 \to \RR$ are
then also continuous for all $x_1$ and $x_2$.  It follows from
Theorem~\ref{Rexists} that $f$ is Riemann integrable on $I_1 \times I_2$, and
from Theorem~\ref{thRIiffMZ} that $f_{x_1}$ and $f_{x_2}$ are
Riemann integrable on $I_2$ and $I_1$ respectively for all $x_1$ and $x_2$.
\item There are cases where (1) and (2) above cannot be used but
Fubini's Theorem can be used.  The following example is from \cite{S}.

Consider the function $f:[0,1]\times [0,1]\to \RR$ defined by
\[
f(x_1,x_2) =
\begin{cases}
\DS 1-\frac{1}{q} &
\qquad \DS \text{if} \ x_2 \in [0,1]\cap \QQ 
\text{ and } x_1 = \frac{p}{q}\ \text{where} \\
& \qquad \qquad 0 < p \leq q \ \text{are relatively prime.} \\
1 & \qquad \text{otherwise}
\end{cases}
\]

\stage{i} We first prove that $f$ is Riemann integrable on $S =
[0,1]\times [0,1]$ and $\DS \int_S f = 1$.  Given
$\epsilon>0$, we choose $q_0 \in \NN$ large enough such that
$1/q_0 < \epsilon/2$.  We also choose a partition
$P = \{ P_1, P_2 \}$ of $S$ where
$P_j = \{p_{i,j} : 0 \leq i \leq N_j\}$ for $1 \leq j \leq 2$ and
\[
\frac{q_0(q_0+1)}{2}
\max_{1\leq i \leq N_1} (p_{i,1} - p_{i-1,1}) < \frac{\epsilon}{2} \ .
\]
Note that there are less than $q_0(q_0+1)/2$ values of $x$ of the form
$p/q$ with $0 < p\leq q \leq q_0$ relatively prime (See footnote
page~\pageref{fnQChap1}).  Let $\{a_k : 1 \leq k \leq K\}$ be these values.

As usual, let $R_{i,j} = [p_{i-1,1},p_{i,1}]\times [p_{j-1,2},p_{j,2}]$ for
$1\leq i \leq N_1$ and $1\leq j \leq N_2$,
$m_{i,j} = \inf \{ f(x_1,x_2) : (x_1,x_2) \in R_{i,j}\}$
and $M_{i,j} = \sup \{ f(x_1,x_2) : (x_1,x_2) \in R_{i,j}\}$.
We have $M_{i,j} - m_{i,j} \leq 1$ for all $i$ and $j$.
Moreover, $M_{i,j} - m_{i,j} \leq 1/q \leq 1/q_0$ if
$a_k \not\in [p_{i-1},p_i]$ for all $k$.  Hence
\begin{align}
&\U_P(f) - \LL_P(f) = \sum_{i,j} (M_{i,j}-m_{i,j} )
(p_{i,1}-p_{i-1,1})(p_{j,2}-p_{j-1,2})
\nonumber \\
&\quad \leq \sum\nolimits' \frac{1}{q_0} (p_{i,1}-p_{i-1,1})(p_{j,1}-p_{j-1,1})
+ \sum\nolimits'' (p_{i,1}-p_{i-1,1})(p_{j,2}-p_{j-1,2})
\ , \label{fub4}
\end{align}
where the first sum is over all subrectangles $R_{i,j}$ such that
$a_k \not\in [x_{i-1,1},x_{i,1}]$ for all $k$, and the second sum is over all
subrectangles $R_{i,j}$ such that $a_k \in [x_{i-1,1},x_{i,1}]$ for some $k$.

We have
$\DS \sum\nolimits' \frac{1}{q_0} (p_{i,1}-p_{i-1,1})(p_{j,2}-p_{j-1,2})
< \frac{\epsilon}{2}$ because
$\DS \frac{1}{q_0} < \frac{\epsilon}{2}$ and
\[
\sum\nolimits' (p_{i,1}-p_{i-1,1})(p_{j,2}-p_{j-1,2})
\leq \sum_{\substack{0<i\leq N_1\\0<j\leq N_2}}
(p_{i,1}-p_{i-1,1})(p_{j,2}-p_{j-1,2}) = V(S) = 1 \ ,
\]
the area of the rectangle $S$.  Moreover, since there are at most
$q_0(q_0+1)/2$ intervals $[p_{i-1,1},p_{i,1}]$ containing some $a_k$,
we also have
\begin{align*}
\sum\nolimits'' (p_{i,1}-p_{i-1,1})(p_{j,2}-p_{j-1,2}) &\leq
\frac{q_0(q_0+1)}{2} \max_{1\leq i \leq N_1} (p_{i,1} - p_{i-1,1})
\underbrace{\sum_{j=1}^{N_2} (p_{j,2}-p_{j-1,2})}_{=1} \\
&= \frac{q_0(q_0+1)}{2} \max_{1\leq i \leq N_1} (p_{i,1} - p_{i-1,1})
< \frac{\epsilon}{2} \ .
\end{align*}
We get from (\ref{fub4}) that $\U_P(f) - \LL_P(f) < \epsilon$.
Since $\epsilon$ is arbitrary, it follows that $f$ is integrable on
$S$ according to the version of Theorem~\ref{intCauchyCR} in
$\DS \RR^n$.
Since $\U_P(f) = 1$ for all partition $P$ of $S$, we have
$\DS \int_S f = \U(f) = 1$.

\stage{ii}
However, $\DS \int_0^1 f_{x_1}(x_2) \dx{x_2} = 1$ for
$x_1\in \{0\} \cup [0,1]\setminus \QQ$ but $f_{x_1}$ is not Riemann
integrable for $x_1\in \QQ \cap ]0,1]$.  To prove that $f_{x_1}$ is
not Riemann integrable for $x_1\in \QQ \cap ]0,1]$, suppose that
$x_1 = p/q$ where $0 < p \leq q$ are relatively prime.  Then
\[
f_{x_1}(x_2) =
\begin{cases}
1 & \quad \text{if} \ x_2 \in [0,1]\setminus \QQ \\
\DS 1-\frac{1}{q} & \quad \text{if} \ x_2 \in \QQ \cap [0,1]
\end{cases}
\]
This function is not Riemann integrable according to
Theorem~\ref{thRIiffMZ} because $f_{x_1}$ is discontinuous at all
points of $[0,1]$ which is not a set of measure zero.  Since
$Q \cap ]0,1]$ does not have measure zero, 
$\DS x_1 \mapsto \DS \int_0^1 f(x_1,x_2)\dx{x_2}$
cannot define a Riemann integrable function on $[0,1]$.

\stage{iii} We cannot use (1) above to define an iterated integral for
$\DS \int_S f$.  However, Fubini's Theorem can be used.
We have
\[
h(x_1) =
\begin{cases}
\DS 1 - \frac{1}{q} & \DS \quad \text{if} \
x_1 = \frac{p}{q} \ \text{where} \ 0 < p \leq q \
\text{are relatively prime}. \\
1 & \quad \text{if} \ x_1 \in \{0\} \cup [0,1]\setminus \QQ
\end{cases}
\]
and $\DS \int_0^1 h(x_1) \dx{x_1} = 1$.  The proof that this last
integral is equal to $1$ is similar to (in fact, is a stripped version
of) the proof that $\DS \int_S f = 1$ above.
\item It is possible for both iterated integrals in (1) and (2) to
exist, and even be equal, but that the function
$f:I_1 \times I_2 \to \RR$ not be Riemann integrable.

To illustrate this situation, we first construct a set
$Q \subset S = I_1 \times I_2$ where $I_1 =I_2 = I = [0,1]$ such that
there is at most one point of $Q$ on each horizontal and vertical lines in
$S$, and such that $\partial Q = S$.
Let
\[
I_{i,j,k} = \left[ \frac{j-1}{2^i}, \frac{j}{2^i}\right] \times
\left[ \frac{k-1}{2^i}, \frac{k}{2^i}\right]
\]
for $\DS 0 <j,k \leq 2^i$.  For each $i>0$, we have
$\DS S = \bigcup_{0<j,k\leq 2^i} I_{i,j,k}$.
For $i=1$, the set $S$ is the union of four squares; namely,
\[
S = \bigcup_{0<j,k\leq 2} I_{1,j,k}
= \left[0,\frac{1}{2}\right] \times \left[0,\frac{1}{2}\right]  
\cup \ldots \cup
\left[\frac{1}{2},1\right] \times \left[\frac{1}{2},1\right] \ .
\]
Let $Q_1$ be the set containing one point from each square such that
each of these points is the unique point of $Q_1$ on the horizontal
and vertical lines on which it belongs.
For $i=2$, the set $S$ is the union of sixteen squares.  Let $Q_2$ be
the set containing one point from each square such that
each of these points is the unique point of $Q_1 \cup Q_2$ on the
horizontal and vertical lines on which it belongs.  We have by
construction that $Q_1 \cap Q_2 = \emptyset$.
Suppose that we have constructed $Q_1$, $Q_2$, \ldots, $Q_{s-1}$.  For
$i=s$, the set $S$ is the union of $\DS 2^{2s}$ squares.  Let
$Q_s$ be the set containing one point from each square such that
each of these points is the unique point of
$\DS \bigcup_{i=1}^{s} Q_i$ on the horizontal and vertical
lines on which it belongs.  We have by construction that
$\DS Q_s \cap \left( \bigcup_{i=1}^{s-1} Q_i \right) = \emptyset$.
We have an inductive method to define all the sets $Q_s$.

The set $Q$ is the set $\DS Q = \bigcup_{s=1}^{\infty} Q_s$.
We have that $\overline{Q} = S$ because every open subset of $S$
contains squares of the form $I_{i,j,k}$ for $i$ large enough and
each of these squares contains a point of $Q$.  We also have that
$\DS Q^\circ = \emptyset$ because there is at most one point
of $Q$ on each horizontal and vertical lines.  Therefore, every open
subset contains points that are not in $Q$.  Thus $\partial Q = S$.

The function $\Chi_Q$ is not Riemann integrable on $S$ because the set of
points where $\Chi_Q$ is discontinuous is $\partial Q = S$ which is
not of measure zero.
However $x_2 \mapsto \chi_Q(x_1,x_2)$ is Riemann integrable on $I$ for all
$x_1\in I$ because $\Chi_Q(x_1,x_2) = 0$ for all $x_2\in I$ but at
most one point.  Since
$\DS \int_I \chi_Q(x_1,x_2) \dx{x_2} = 0$ for all $x_1\in I$, we get
$\DS \int_I \left( \int_I \chi_Q(x_1,x_2) \dx{x_2} \right)
\dx{x_1}= 0$.  Similarly, we get
$\DS \int_I \left( \int_I \chi_Q(x_1,x_2) \dx{x_1} \right)
\dx{x_2}= 0$.
\end{enumerate}
\end{rmkList}

\section{Partition of Unity}

\begin{theorem}\label{cov1}
Let $\DS \{U_\alpha\}_{\alpha \in A}$ be an open cover of
$\DS W \subset \RR^n$; namely,
$U_\alpha$ is an open set for every $\alpha$ and
$\DS W \subset \bigcup_{\alpha \in A} U_\alpha$.  There exist
an open set $V \supset W$ and a collection
$\DS \{\phi_j\}_{j \in \NNp}$ of
functions in $\DS C^\infty(V)$ such that:
\begin{enumerate}
\item $0 \leq \phi_j(\VEC{x}) \leq 1$ for all
$\DS \VEC{x} \in V$ and all $\DS j \in \NNp$.
\item For each $\VEC{x}\in W$, there exists an open neighbourhood
$W_{\VEC{x}} \subset V$ of $\VEC{x}$ such that
$\DS \phi_j\big|_{W_\VEC{x}} = 0$ for all but a finite
number of $\phi_j$.
\item For each $\VEC{x} \in W$, we have
$\DS \sum_{j \in \NNp} \phi_j(\VEC{x}) = 1$.  This sum
is finite according to (2).
\item For each $\DS j \in \NNp$, we have that
$\supp \phi_j$ is a compact set and
$\supp \phi_j \subset U_\alpha$ for some $\alpha \in A$.
\end{enumerate}
\end{theorem}

\begin{proof}
\stage{i} If $W$ is compact, then there exists a finite subcover 
$\DS \{U_{\alpha_i}: 1 \leq i \leq N\}$ of $W$.
For each $\VEC{x} \in W$, there exist $0 < r_{\VEC{x},1} < r_{\VEC{x},2}$
such that
\[
  B_{r_{\VEC{x},1}}(\VEC{x}) \subset B_{r_{\VEC{x},2}}(\VEC{x})
  \subset \overline{B_{r_{\VEC{x},2}}(\VEC{x})} \subset U_{\alpha_i}
\]
for $\alpha_i \in A$.
Since $\DS \{B_{r_{\VEC{x},1}}(\VEC{x}) \}_{\VEC{x} \in W}$
is an open cover of the compact set $W$, there exists a finite
subcover
$\DS \{B_{r_{\VEC{x}_k,1}}(\VEC{x}_k) : 1 \leq k \leq M \}$
of $W$.  For each $1 \leq k \leq M$, let
\begin{align*}
&\psi_k(\VEC{x}) = \\
&\begin{cases}
1 & \quad \text{if} \ \|\VEC{x} - \VEC{x}_k\| \leq r_{\VEC{x}_k,1} \\
e^{1/(r_{\VEC{x}_k,2}^2 - r_{\VEC{x}_k,1}^2)}
e^{-1/(r_{\VEC{x}_k,2}^2 - \|\VEC{x} - \VEC{x}_k\|^2)}
\left( 1 - e^{-1/(\|\VEC{x} - \VEC{x}_k\|^2-r_{\VEC{x}_k,1}^2)}\right)
& \quad \text{if} \ r_{\VEC{x}_k,1} < \|\VEC{x} - \VEC{x}_k\| < r_{\VEC{x}_k,2} \\
0 & \quad \text{if} \ \|\VEC{x} - \VEC{x}_k\| \geq r_{\VEC{x}_j,2}
\end{cases}
\end{align*}
for $\DS \VEC{x} \in \RR^n$.
We have $\DS \psi_k \in C^\infty(\RR^n)$,
$0 \leq \psi_k(\VEC{x}) \leq 1$ for all $\DS \VEC{x} \in \RR^n$,
$\psi_k(\VEC{x}) = 1$ for all $\VEC{x} \in B_{r_{\VEC{x}_k,1}}(\VEC{x}_k)$,
and
$\supp \psi_k \subset \overline{B_{r_{\VEC{x}_k,2}}(\VEC{x}_k)}
\subset U_{\alpha_i}$ for some $\alpha_i \in A$.

Let $\phi_1 = \psi_1$ and
$\DS \phi_{j+1} = \psi_{j+1}\prod_{k=1}^j(1-\psi_k)$
for $1 <j < M$.

\begin{enumerate}
\item We have
$0 \leq \phi_j(\VEC{x}) \leq 1$ for all
$\DS \VEC{x} \in \RR^n$ and $1 \leq j \leq M$ because
$\phi_j$ is the product of functions whose image is in $[0,1]$.
\item For each $\VEC{x}\in W$, it is obvious that there exists an
open neighbourhood $\DS W_{\VEC{x}} \subset \RR^n$ of
$\VEC{x}$ such that
$\DS \phi_j\big|_{W_\VEC{x}} = 0$ for all but a finite
number of $\phi_j$ because there is only a finite number of $\phi_j$.
\item To prove that $\DS \sum_{j=1}^M\phi_j(\VEC{x}) = 1$
for all $\VEC{x} \in W$, we first prove by induction that
\begin{equation} \label{thPofUEq1}
\sum_{j=1}^m \phi_j = 1 - \prod_{j=1}^m(1-\psi_j)
\end{equation}
for $1 \leq m \leq M$.  We obviously have that (\ref{thPofUEq1}) is
true for $m=1$.  Assume that (\ref{thPofUEq1}) is true for $m<M$.
Using the hypothesis of induction, we get
\begin{align*}
\sum_{j=1}^{m+1} \phi_j &= \sum_{j=1}^m \phi_j + \phi_{m+1}
= 1 - \prod_{j=1}^m(1-\psi_j) + \psi_{m+1}\prod_{j=1}^m(1-\psi_j) \\
&= 1 - \prod_{j=1}^m(1-\psi_j) (1-\psi_{m+1})
= 1 - \prod_{j=1}^{m+1}(1-\psi_j) \ .
\end{align*}
Thus (\ref{thPofUEq1}) is true with $m$ replaced by $m+1$.  This completes
the proof by induction.  It follows from (\ref{thPofUEq1}) that
\[
\sum_{j=1}^M \phi_j(\VEC{x}) = 1 - \prod_{j=1}^M(1-\psi_j(\VEC{x})) = 1
\]
for all $\VEC{x} \in W$ because every $\VEC{x} \in W$ is element
of $B_{r_{\VEC{x}_k,1}}(\VEC{x}_k)$ for at least one $k$ and
$\psi_k(\VEC{y}) = 1$ for all $\VEC{y} \in B_{r_{\VEC{x}_k,1}}(\VEC{x}_k)$.
Therefore, $\DS \sum_{j=1}^M \phi_j(\VEC{x}) = 1$
for all $\VEC{x} \in W$.
\item Since $\supp \phi_j \subset \supp \psi_j \subset
\overline{B_{r_{\VEC{x}_j,2}}(\VEC{x}_j)}$, we have that
$\supp \phi_j$ is a compact set because it is closed and bounded.
We also have $\supp \phi_j \subset U_{\alpha_j}$.
\end{enumerate}
All the conclusions of the theorem are satisfied by the collection
$\DS \{\phi_j\}_{1 \leq j \leq M}$.

\stage{ii} Suppose that $\DS W = \bigcup_{j=0}^\infty W_j$
where each $W_j$ is compact and
$\DS W_j \subset W_{j+1}^\circ$ for all $j$.
We also assume that $W_0 = \emptyset$ to simplify the discussion.
Let $V_1 = W_1$ and
$\DS V_j = W_j \setminus W_{j-1}^\circ$ for $j \geq 2$.

Let $\DS U_{\alpha,1} = U_\alpha \cap W_2^\circ$ for all $\alpha \in A$.
Since $V_1$ is a compact set and
$\DS \{U_{\alpha,1}\}_{\alpha \in A}$ is an open cover of $V_1$,
we may use (i) to find $\DS \{\psi_{k,1}: 1 \leq k \leq N_1\}$
such that $0 \leq \psi_{k,1}(\VEC{x}) \leq 1$ for all
$\DS \VEC{x} \in \RR^n$ and $1\leq k \leq N_1$,
$\DS \sum_{k=1}^{N_1} \psi_{k,1}(\VEC{x}) = 1$ for all
$\VEC{x} \in V_1$, and $\supp \psi_{k,1}$ is a compact set with
$\supp \psi_{k,1} \subset U_{\alpha_{k,1}}$ for some $\alpha_{k,1} \in A$.

For $j \geq 2$, let $\DS U_{\alpha,j} = U_\alpha \cap
\left(W_{j+1}^\circ\setminus W_{j-2}\right)$ for all $\alpha \in A$.
Since $V_j$ is a compact set and
$\DS \{U_{\alpha,j}\}_{\alpha\in A}$ is an open cover of $V_j$,
we may use (i) to find $\DS \{\psi_{k,j}: 1 \leq k \leq N_j\}$
such that
$0 \leq \psi_{k,j}(\VEC{x}) \leq 1$ for all
$\DS \VEC{x} \in \RR^n$ and $1\leq k \leq N_j$,
$\DS \sum_{k=1}^{N_j} \psi_{k,j}(\VEC{x}) = 1$ for all
$\VEC{x} \in V_j$, and
$\supp \psi_{k,j}$ is a compact set with
$\supp \psi_{k,j} \subset U_{\alpha_{k,j}}$ for some $\alpha_{k,j} \in A$.

\pdfbox{mult_integrals/partunit}

Let $\DS \psi(\VEC{x})
= \sum_{j=1}^\infty \sum_{k=1}^{N_j} \psi_{k,j}(\VEC{x})$ for
$\DS \VEC{x} \in \RR^n$.  We first note that the sum is
finite for each $\VEC{x}$ because $\psi_{k,j}(\VEC{x}) = 0$ if
$\DS \VEC{x} \not\in W_{j+1}^\circ\setminus W_{j-2}$ by construction.
Moreover, if $\VEC{x} \in W$, then $\VEC{x} \in V_q$ for some $q \geq 1$
because $\DS \VEC{x} \in W = \bigcup_{j=1}^\infty V_j$.  Hence
\[
\psi(\VEC{x})
= \sum_{\substack{j \text{ such that}\\
\VEC{x} \in W_{j+1}^\circ\setminus W_{j-2}}}
\sum_{k=1}^{N_j} \psi_{k,j}(\VEC{x})
\geq \sum_{k=1}^{N_q} \psi_{k,q}(\VEC{x}) = 1 \ .
\]
Choose an open set $V \supset W$ such that
$\psi(\VEC{x}) \neq 0$ for all $\VEC{x} \in V$ \footnote{
Since $\psi(x) = 1$ for all $\VEC{x} \in W$ and $\psi$ is continuous,
we can find for each $\VEC{x} \in W$ an open neighbourhood
$V_{\VEC{x}}$ of $\VEC{x}$ such that $\psi(\VEC{y}) > 1/2$ 
for all $\VEC{y} \in V_{\VEC{x}}$.  We may take
$\DS V = \bigcup_{\VEC{x}\in W} V_{\VEC{x}}$.}.
Let
\[
\rho_{k,j}(\VEC{x}) = \frac{\psi_{k,j}(\VEC{x})}{\psi(\VEC{x})}
\]
for $\VEC{x} \in V$, $j \geq 1$ and $1 \leq k \leq N_j$.
We have that $\DS \rho_{k,j} \in C^\infty(V)$ because
$\psi_{k,j}$ and $\psi$ are in $\DS C^\infty(V)$, and
$\psi(\VEC{x}) \neq 0$ for all $\DS \VEC{x} \in V$.

\begin{enumerate}
\item We have by construction that
$0 \leq \rho_{k,j}(\VEC{x}) \leq 1$ for all
$\DS \VEC{x} \in V$, $j \geq 1$ and $1 \leq k \leq N_j$.
\item There exists an open neighbourhood
$W_{\VEC{x}} \subset V$ of $\VEC{x} \in W$
such that $\rho_{k,j}\big|_{W_\VEC{x}} = 0$ for all but a finite
number of the $\rho_{k,j}$.  More precisely, given
$\VEC{x} \in W$, we have $\VEC{x} \in V_j$ for some $j$.
Since $\rho_{k,q}(\VEC{y}) = 0$ for all
$\DS \VEC{y} \in W_{j+1}^\circ \setminus W_{j-2}$
if $q<j-2$ and $q>j+1$, it suffices to choose an open neighbourhood
$\DS W_{\VEC{x}} \subset W_{j+1}^\circ \setminus W_{j-2}$ of $\VEC{x}$.
\item We also have by construction that
$\DS \sum_{j=1}^\infty \sum_{k=1}^{N_j}
\rho_{k,j}(\VEC{x}) = 1$ for all $\VEC{x} \in W$.
\item Since $\supp \rho_{k,j} = \supp \psi_{k,j}$, we
have that $\rho_{k,j}$ has a compact support because
$\psi_{k,j}$ has a compact support.  Moreover,
$\supp \rho_{k,j} = \supp \psi_{k,j}  
\subset U_{\alpha_{k,j}}$ for some $\alpha_{k,j} \in A$
for each $j \geq 1$ and $1 \leq k \leq N_j$.
\end{enumerate}
The collection $\DS \{\phi_j\}_{j \in \NNp}$ that satisfies
all the conclusions of the theorem is defined as it follows.
$\phi_j = \rho_{j,1} $ for $1 \leq j \leq N_1$ and
$\phi_j = \rho_{k,q} $ for
$\DS j = \sum_{i=1}^{q-1} N_i + k$ and
$1 \leq k \leq N_q$ with $q >1$.

\stage{iii} If $W$ is an open set, then
$\DS W = \bigcup_{j=j}^\infty W_j$ where
$\DS W_j = \{\VEC{x} : \dist{\VEC{x}}{\RR^n \setminus W}
\geq 1/j\} \cap \{\VEC{x} : \|\VEC{x}\| \leq j\}$.  The sets $W_j$ are
compact and $\DS W_j \subset W_{j+1}^\circ$ for all $j\geq
1$.  We may therefore used (ii).  If $W$ is not open, then we use (ii) with
the open set $\DS \bigcup_{\alpha \in A} U_\alpha\supset W$
instead of $W$.
\end{proof}

\begin{rmk}
In the previous theorem, if $\{U_i\}_{1\leq i \leq N}$    \label{rmkFPUcase}
is a finite open cover of a compact set
$\DS W \subset \RR^n$, then the
first part of the proof proves that we can choose a finite partition of unity
$\{ \phi_i \}_{1 \leq i \leq N}$ subordinate to the open cover
$\{U_i\}_{1\leq i \leq N}$ of $W$ such that $\supp \phi_i \subset U_i$
for $1\leq i \leq N$.  We may have to add together the $\phi_j$'s
given in the proof that have their support in the same $U_i$ to get a
new $\phi_i$, and to reorder the $\phi_i$.
\end{rmk}

\begin{defn}
The collection $\DS \{\phi_j\}_{j \in \NNp}$ of the
previous theorem is called a
{\bfseries partition of unity}\index{Partition of Unity}
subordinated to the open cover $\DS \{U_\alpha\}_{\alpha\in A}$
of $W$.
\end{defn}

\begin{prop} \label{cov4}
Suppose that $\DS \{\phi_j\}_{i \in \NNp}$ is a partition of
unity subordinated to the open cover $\DS \{U_\alpha\}_{\alpha\in A}$
of $W$.  If $K\subset W$ is a compact set, then there exist only a finite
number of functions $\phi_j$ such that $\phi_j\big|_K \neq 0$.
\end{prop}

\begin{proof}
For each $\VEC{x} \in K$, let $W_{\VEC{x}}$ be the open set given in
(2) of Theorem~\ref{cov1}.  Then
$\DS \{W_{\VEC{x}}\}_{\VEC{x} \in K}$ is an open cover
of the compact set $K$.  Therefore, there exists a finite subcover
$\DS \{W_{\VEC{x}_j}: 1\leq j \leq N\}$ of $K$.

A function $\phi_j$ such that $\phi_j\big|_K \neq 0$ must
also be one of the functions such that
$\DS \phi_j\big|_{W_{\VEC{x}_j}} \neq 0$ for some $j$.
There are only a finite number of them according to (2) of Theorem~\ref{cov1}.
\end{proof}

\begin{defn}
An open cover $\DS \{U_\alpha\}_{\alpha\in A}$ of an open
set $W$ is called {\bfseries admissible}\index{Admissible Open Cover}
if $U_\alpha \subset W$ for all $\alpha$.
\end{defn}

\section{General Definition of the Integral}

\begin{defn}
Let $W$ be a Jordan measurable subset of $\DS \RR^n$.  In
this and the next sections, we denote the Riemann integral of a Riemann
integrable function $h:W\to \RR$ by $\DS \rint_W h $.
\end{defn}

The reason for this new notation is that we will shortly generalize
the definition of integrable functions.  One of our goal is to
generalize to $\DS \RR^n$ the concept of improper integral
for functions of one variable that we have seen in the previous
chapter.

\begin{lemma} \label{lemBZimplR}
Let $W$ be an open subset of $\DS \RR^n$ and
$f:W\to \RR$ be a function which is bounded on bounded open
subsets of $W$ and continuous on $W$ except possibly on a set of
measure zero.  If $\DS \psi:W \to \RR$ is a continuous
function with compact support in $W$, then $\psi\, f:W \to \RR$ is
Riemann integrable on $W$.
\end{lemma}

\begin{proof}
Since $\supp \psi \, f \subset \supp \psi$ and $\supp \psi$ is a
compact subset of $W$, we get that $\supp \psi \, f \subset W$ is
compact.  Thus $\supp \psi \, f \subset W$ is bounded and so it is
contained in a bounded open subset of $W$.  Moreover, there exists a
bounded rectangle $\DS R \subset \RR^n$ such that
$\supp \psi\,f \subset R$.  We may assume that
$\psi(\VEC{x}) = f(\VEC{x})$ for $\VEC{x} \in R \setminus W$.

The set of points where $\psi \, f$ is discontinuous is a subset of the
set of points where $f$ is discontinuous.  Thus the set of points
where $\psi \, f$ is discontinuous is of measure zero.  We also have
that $\psi \, f$ is bounded on $R$ because $f$ is bounded on bounded
open subsets of $W$ by hypothesis and $\psi$ is bounded because it is a
continuous function with compact support.

Therefore, we have from Theorem~\ref{Rexists} that
$\psi \, f$ is Riemann integrable on $R$.  Moreover,
$\DS \int_W \psi \, f  = \int_R \psi \, f$.
\end{proof}

\begin{prop} \label{cov3}
Let $W$ be an open subset of $\DS \RR^n$ and
$f:W\to \RR$ be a function which is bounded on bounded open
subsets of $W$ and continuous on $W$ except possibly on a set of
measure zero.
Suppose that $\DS \{\phi_j\}_{j \in \NNp}$ and
$\DS \{\tilde{\phi}_k\}_{k \in \NNp}$
are two partitions of unity subordinated respectively to the admissible
open covers $\DS \{U_{\alpha}\}_{\alpha\in A}$
and
$\DS \{\tilde{U}_{\tilde{\alpha}}\}_{\tilde{\alpha}\in \tilde{A}}$
of the open set $W$.
Then $\DS \sum_{j \in \NNp} \rint_W \phi_j |f|$
converges if and only if
$\DS \sum_{k \in \NNp} \rint_W \tilde{\phi}_k |f|$
converges.  Moreover, the two sums are equal when they converges.  
\end{prop}

\begin{proof}
According to the previous lemma, all the functions $\phi_j |f|$,
$\tilde{\phi}_k|f|$ and $\phi_j \tilde{\phi}_k|f|$ are Riemann
integrable on $W$.

Let $K_j = \supp \phi_j |f|$ for $\DS j \in \NNp$.
The set $K_j$ is compact according to (4) of Theorem~\ref{cov1}.

We then have from Proposition~\ref{cov4} that there are only a finite
number of functions $\tilde{\phi}_k$ such that
$\DS \tilde{\phi}_k\big|_{K_j} \neq 0$.
It follows that the sum
$\DS \sum_{k \in \NNp} \tilde{\phi}_k \phi_j |f|$ is a
finite sum for each $j$.  A similar argument proves that 
$\DS \sum_{j \in \NNp} \phi_j\tilde{\phi}_k|f|$
is a finite sum for each $k$.  Hence,
\begin{equation} \label{cov5}
\begin{split}
\sum_{j \in \NNp} \rint_W \phi_j |f|
&= \sum_{j \in \NNp} \rint_W \big(\sum_{k \in \NNp}
\tilde{\phi}_k \big) \phi_j |f|
= \sum_{j \in \NNp} \sum_{k \in \NNp}
\rint_W \tilde{\phi}_k \phi_j |f| \\
&= \sum_{k \in \NNp} \sum_{j \in \NNp} \rint_W
\tilde{\phi}_k \phi_j |f| 
= \sum_{k \in \NNp} \rint_W \big( \sum_{j \in \NNp}
\phi_j \big) \tilde{\phi}_k |f| \\
&= \sum_{k \in \NNp} \rint_W \tilde{\phi}_k |f| \ .
\end{split}
\end{equation}
The first equality comes from
$\DS \sum_{k \in \NNp} \tilde{\phi}_k(\VEC{x}) = 1$ for all
$\VEC{x} \in W$.  Since the summation with respect to $k$ is finite on
$\supp \phi_j$, we may pull the summation with respect to $k$ outside
the integral to get the second equality.  The third equality is true
because we may change the order of summation of positive series.
Since the summation with respect to $j$ is finite on
$\supp \tilde{\phi}_k$, we may bring the summation with respect to
$j$ inside the integral to get the fourth equality.  The last equality
comes from $\DS \sum_{j\in \NNp} \phi_j(\VEC{x}) = 1$ for all
$\VEC{x} \in W$.
\end{proof}

The previous proposition is used to justify the following definition.

\begin{defn}\label{cov2}
Suppose that $\DS \{\phi_j\}_{j \in \NNp}$
is a partition of unity subordinated to an admissible open cover
$\DS \{U_{\alpha}\}_{\alpha\in A}$ of the open set $W$.
Let $f:W\to \RR$ be a function which is bounded on bounded open
subsets of $W$ and continuous on $W$ except possibly on a set of
measure zero.  We say that $f$ is {\bfseries integrable}\index{Integrable} 
on $W$ if
$\DS \sum_{j \in \NNp} \rint_W \phi_j |f|$ converges.
\end{defn}

As a consequence of the previous proposition, this definition is independent
of the admissible open cover and partition of unity.

In this section and the next section, when we refer to an integrable
function, we will mean an integrable function in the new sense given
in Definition~\ref{cov2}.  To refer to a function integrable in the
sense of Riemann, we will say that the function is Riemann integrable.

\begin{prop}
Suppose that $\DS \{\phi_j\}_{j \in \NNp}$ and
$\DS \{\tilde{\phi}_j\}_{j \in \NNp}$ are two partitions of
unity subordinated respectively to the admissible open cover 
$\DS \{U_{\alpha}\}_{\alpha\in A}$ and
$\DS \{\tilde{U}_{\tilde{\alpha}}\}_{\tilde{\alpha}\in \tilde{A}}$
of the open set $W$.
Let $f:W\to \RR$ be a function which is bounded on bounded open
subsets of $W$ and continuous on $W$ except possibly on a set of
measure zero.
If $f:W\to \RR$ is an integrable function, then
$\DS \sum_{j \in \NNp} \rint_W \phi_j f 
= \sum_{j \in \NNp} \rint_W \tilde{\phi}_j f$.
\end{prop}

\begin{proof}
Since $f$ is integrable,
$\DS \sum_{j \in \NNp} \rint_W \phi_j |f|$ and
$\DS \sum_{k \in \NNp} \rint_W \tilde{\phi}_k |f|$
converge (and are equal).  Thus
$\DS \sum_{j \in \NNp} \rint_W \phi_j f$ and
$\DS \sum_{k \in \NNp} \rint_W \tilde{\phi}_k f$
converge absolutely because
$\DS \left| \rint_W \phi_j f\right|
\leq \rint_W \phi_j |f|$ and for all $\DS j \in \NNp$ and
$\DS \left| \rint_W \tilde{\phi}_k f\right|
\leq \rint_W \tilde{\phi}_k |f|$ for all $\DS k \in \NNp$.

As in the proof of Proposition~\ref{cov3}, we have that
$\DS \sum_{k \in \NNp} \tilde{\phi}_k \phi_j f$ is a finite
sum for each $j$ and
$\DS \sum_{j \in \NNp} \phi_j \tilde{\phi}_k f$
is a finite sum for each $k$.  It also follows from
(\ref{cov5}) in the proof of Proposition~\ref{cov3} that
$\DS \sum_{j \in \NNp} \sum_{k \in \NNp}
\rint_W \tilde{\phi}_k \phi_j f$ converges absolutely.  Hence,
\begin{align*}
\sum_{j \in \NNp} \rint_W \phi_j f
&= \sum_{j \in \NNp} \rint_W \big( \sum_{k \in \NNp}
\tilde{\phi}_k\big) \phi_j f
= \sum_{j \in \NNp} \sum_{k \in \NNp} \rint_W \tilde{\phi}_k \phi_j f \\
&= \sum_{k \in \NNp} \sum_{j \in \NNp} \rint_W \tilde{\phi}_k \phi_j f
= \sum_{k \in \NNp} \rint_W \big(
\sum_{j \in \NNp}  \phi_j \big) \tilde{\phi}_k f
= \sum_{k \in \NNp} \rint_W \tilde{\phi}_k f
\end{align*}
The first equality comes from
$\DS \sum_{k \in \NNp} \tilde{\phi}_k(\VEC{x}) = 1$
for all $\VEC{x}$.
Since the summation with respect to $k$ is finite, we may
pull the summation with respect to $k$ outside of the
integral to get the second equality.  The third equality is true
because we may change the order of summation of absolutely convergent
series.  Since the summation with respect to $j$ is
finite, we may bring the summation with respect to $j$ inside the
integral to get the fourth equality.  The last equality comes from
$\DS \sum_{j \in \NNp} \phi_j(\VEC{x}) = 1$ for all
$\VEC{x}$.
\end{proof}

The previous proposition proves that the following definition is
independent of the admissible open cover and partition of unity.

\begin{defn} \label{cov6}
Suppose that $\DS \{\phi_j\}_{j \in \NNp}$
is a partition of unity subordinated to an admissible open cover
$\DS \{U_{\alpha}\}_{\alpha\in A}$ of the open set $W$.
Let $f:W\to \RR$ be a function which is bounded on bounded open
subsets of $W$ and continuous on $W$ except possibly on a set of
measure zero.
If $f:W\to \RR$ is integrable, then the
{\bfseries integral}\index{Integral of a Function} of $f$ on $W$ is
defined as
\[
\int_W f  = \sum_{j \in \NNp} \rint_W \phi_j f \ .
\]
\end{defn}

\begin{rmk}
We stated Definitions~\ref{cov2} and \ref{cov6} using the   \label{cov38}
$\DS C^\infty$ functions $\phi_j$ provided by the partition
of unity because they came for free from Theorem~\ref{cov1}.  However,
we may only assumed that the $\phi_j$ are continuous.   This will not
change the class of integrable functions as none of the previous
propositions required the smoothness of the $\phi_j$.
\end{rmk}

\begin{theorem}\label{RIIinNS}
Let $W$ be an open and Jordan measurable subset of $\DS \RR^n$ and
$f:W\to \RR$ be a Riemann integrable function on $W$.  Then $f$ is
integrable on $W$ in the new sense.
\end{theorem}

\begin{proof}
Suppose that $\DS \{U_{\alpha}\}_{\alpha\in A}$ is an
admissible open cover of $W$ and that
$\DS \{\phi_j\}_{j \in \NNp}$ is a partition of
unity subordinate to $\DS \{U_{\alpha}\}_{\alpha\in A}$.

Since $W$ is bounded, $W$ is contained in a closed and bounded rectangle $K$.
Since $f$ is Riemann integrable, $f$ is bounded and thus
$M = \sup \{ f(\VEC{x}) : \VEC{x} \in W \} < \infty$.  For every finite subset
$\DS Q \subset \NNp$, we have
\[
\sum_{j \in Q} \rint_W \phi_j |f| \leq
M \sum_{j \in Q} \rint_W \phi_j = M \rint_W \sum_{j\in Q} \phi_j \leq
M \rint_W 1 \leq M V(K) < \infty \ ,
\]
where the second inequality comes from
$\DS \sum_{j\in Q} \phi_j(\VEC{x})
\leq \sum_{j \in \NNp} \phi_j(\VEC{x}) = 1$
for all $\VEC{x} \in W$ and the third inequality comes from
$\DS \rint_W \Chi_W = V(W)$ and $\DS W \subset K$.  Since
$\DS \sum_{j \in \NNp} \rint_W \phi_j |f|$ is a
series of positive terms, we get
\[
\sum_{j \in \NNp} \rint_W \phi_j |f| =
\sup_{\substack{Q\subset \NN\\Q\text{ finite}}}
\sum_{j \in Q} \rint_W \phi_j|f| \leq M V(K) \ .
\]
Therefore $f$ is integrable.
\end{proof}

The next proposition shows that if $f:W\to \RR$ is Riemann integrable, then
the Riemann integral of $f$ is equal to the integral of $f$ as we have
defined in Definition~\ref{cov6}.

\begin{theorem}
Suppose that $\DS W \subset \RR^n$ is an open and Jordan
measurable set.  Moreover, suppose that $f:W\to \RR$ is Riemann
integrable (and so integrable in the new sense according to
Theorem~\ref{RIIinNS}).  Then the Riemann integral of $f$ on $W$ is
equal to the integral of $f$ as defined in Definition~\ref{cov6}.
\end{theorem}

\begin{proof}
Since $f$ is Riemann integrable, it is bounded on $W$.
Choose $M>0$ such that $|f(\VEC{x})|<M$ for all $\VEC{x} \in W$.

\stage{i}
Given $\epsilon>0$, there exists a compact subset $K\subset W$ such that
$\DS \rint_{W\setminus K} 1 < \epsilon/M$.  More precisely,
to find $K$, we use a partition $P$ of a rectangle $R$ containing $W$
such that $\DS \LL(\Chi_W) - \LL_P(\Chi_W) < \epsilon/M$.  Note that
$\Chi_W$ is integrable because $W$ is a Jordan measurable set and so
the set of points $\partial W$ where $\Chi_W$ is discontinuous has
measure zero.   We have
\[
  \LL_P(\Chi_W) = \sum_{R_{k_1,k_2,\ldots,k_n} \subset W}
  V(R_{k_1,k_2,\ldots,k_n})
\]
where as usual $R_{k_1,k_2,\ldots,k_n}$ is a subrectangle of $R$
generated by the partition $P$ and\\
$V(R_{k_1,k_2,\ldots,k_n})$ is the
volume of $R_{k_1,k_2,\ldots,k_n}$.  Note that\\
$m_{k_1,k_2,\ldots,k_n}
= \inf \{ \Chi_W(\VEC{x}) : \VEC{x} \in R_{k_1,k_2,\ldots,k_n}\} = 0$
if $R_{k_1,k_2,\ldots,k_n} \cap (R\setminus W) \neq \emptyset$.

Let $\DS K = \bigcup_{R_{k_1,k_2,\ldots,k_n} \subset W}
R_{k_1,k_2,\ldots,k_n}$.  Then
\[
\rint_K 1 = \rint_R \Chi_K = \sum_{R_{k_1,k_2,\ldots,k_n} \subset W}
V(R_{k_1,k_2,\ldots,k_n}) = \LL_P(\Chi_W) \ .
\]
Moreover
\[
\rint_W 1 = \rint_R \Chi_W  = \LL(\Chi_W)
\]
because $\Chi_W$ is integrable.  It follows that
\[
\rint_{W\setminus K} 1 = \rint_W 1 - \rint_K 1 
= \LL(\Chi_W) - \LL_P(\Chi_W) < \epsilon/M \ .
\]

\stage{ii} Suppose that $\DS \{U_{\alpha}\}_{\alpha\in A}$ is an
admissible open cover of $W$ and that
$\DS \{\phi_j\}_{j \in \NNp}$ is a partition of
unity subordinate to $\DS \{U_{\alpha}\}_{\alpha\in A}$.

According to Proposition~\ref{cov4}, there are only finitely many
$\phi_j$ such that $\DS \phi_j\big|_K \neq 0$.
Let $\DS Q\subset \NNp$ be the set of
indices $\DS j \in \NNp$ such that
$\phi_j\big|_K \neq 0$.  Then, for any finite set
$\DS G \subset \NNp$ such that $Q\subset G$, we have
\begin{align*}
&\left| \rint_W f  - \sum_{j\in G} \rint_W \phi_j f  \right|
= \left| \rint_W f \left( 1 - \sum_{j \in G} \phi_j \right) \right|
\leq \rint_W |f| \left( 1 - \sum_{j \in G} \phi_j \right) \\
&\qquad \leq M \rint_W \left( 1 - \sum_{j \in G} \phi_j \right)
\leq M \rint_W \left( 1 - \sum_{j \in Q} \phi_j \right)
 \leq M \rint_{W\setminus K} 1 < \epsilon \ .
\end{align*}
To get the last two inequalities, we have used the relation
\[
\sum_{j\in Q} \phi_j(\VEC{x})
\leq \sum_{j\in G} \phi_j(\VEC{x})
\leq \sum_{j\in \NNp} \phi_j(\VEC{x}) = 1
\]
for all $\VEC{x} \in W$ and
$\DS \sum_{j\in Q} \phi_j(\VEC{x}) = 1$
for all $\VEC{x}\in K$ because the other $\phi_j$ are null on
$K$.  In particular, this last inequality implies that
$\DS 1 - \sum_{j\in Q} \phi_j(\VEC{x})$ may be
non zero only if $\VEC{x} \in W\setminus K$.  Since
$\DS G\subset \NNp$ with $Q \subset G$ is arbitrary, we get
\[
\left| \rint_W f  - \sum_{j \in \NNp} \rint_W \phi_j f  \right|
\leq \epsilon \ .
\]
Since $\epsilon$ is arbitrary, we get
\[
\rint_W f  = \sum_{j \in \NNp} \rint_W \phi_j f  = \int_W f \ .
\qedhere
\]
\end{proof}

\section{Change of Variables}

The main result of this section is the following theorem.

\begin{theorem} \label{cov7}
Let $\DS W\subset \RR^n$ be an open set, and
$\DS g:W\to \RR^n$ be a
one-to-one and continuously differentiable function such that
$\det \diff g(\VEC{x}) \neq 0$ for all $\VEC{x} \in W$.  If
$f:g(W)\to \RR$ is integrable, then
\begin{equation}\label{cov25}
\int_{g(W)} f  = \int_W (f \circ g)\, |\det \diff g| \ .
\end{equation}
\end{theorem}

\begin{rmk}
The notation used in the statement of (\ref{cov25}) is not the
notation generally used in standard calculus textbooks.  If
$\VEC{x} = g(\VEC{w})$, they use the
{\bfseries Jacobian}\index{Jacobian} defined by
\[
  \frac{\partial(x_1,x_2,\ldots,x_n)}{\partial(w_1,w_2,\ldots,w_n)}
  = \det \diff g ,
\]
So (\ref{cov25}) is stated as
\[
\int_{g(W)} f  = \int_W f(g(\VEC{w}))
\left|\frac{\partial(x_1,x_2,\ldots,x_n)}{\partial(w_1,w_2,\ldots,w_n)}
(\VEC{w})\right| \dx{\VEC{w}} \ .
\]
\end{rmk}

The following proposition is at the heart of the proof of
Theorem~\ref{cov7}.

\begin{prop}\label{cov8}
Let $\DS W\subset \RR^n$ be an open set, and
$\DS g:W\to \RR^n$ be a one-to-one and continuously
differentiable function such that
$\det \diff g(\VEC{x}) \neq 0$ for all $\VEC{x} \in W$.  If $f:g(W)\to
\RR$ is Riemann integrable with compact support in $g(W)$, then
$(f \circ g)\, |\det \diff g| : W \to \RR$ is Riemann integrable with
compact support in $W$ and
\[
\rint_{g(W)} f  = \rint_W (f \circ g)\, |\det \diff g| \ .
\]
\end{prop}

To proof this proposition, we first give several lemmas which will
reduce the proof to a simple situation.

\begin{lemma} \label{lemC1Meas0}
Let $\DS W\subset \RR^n$ be an open set and
$\DS f:W\to \RR^n$ be continuously differentiable function.
If $A \subset W$ has measure zero, then $f(A)$ has measure zero.
\end{lemma}

\begin{proof}
We can write $W$ as the countable union of compact sets
$Q_j$ for $\DS j \in \NNp$ such that
$\DS Q_j \subset Q_{j+1}^\circ \subset Q_{j+1} \subset W$
for all $j$.  For instance, we may write
$\DS W = \bigcup_{j \in \NNp} Q_j$ where $Q_j$ is the compact set
$\DS Q_j = \{ \VEC{x} \in W :
\|\VEC{x}\| \leq j \ \text{and} \ \dist{\partial W}{\VEC{x}} \leq 1/j \}$.
Since
\[
f(A) = f(A\cap W) = f\big(A \cap \bigcup_{j \in \NNp} Q_j\big)
= f\big(\bigcup_{j \in \NNp} (A \cap Q_j)\big)
\subset \bigcup_{j \in \NNp} f(A \cap Q_j) \ ,
\]
it suffices to prove that $f(A \cap Q_j)$ is of measure zero.  Recall
that a subset of a set of measure zero has measure zero and the
countable union of sets of measure zero has measure zero.

We consider an arbitrary but fixed $\DS j \in \NNp$.
Let $M$ be the maximum of $\|\DS \diff f\|$ on $Q_{j+1}$.
We have that $M < \infty$ because $\diff f$ is continuous
on the compact set $Q_{j+1}$, so $\|\diff f\|$ reaches
its maximum at a point of $Q_{j+j}$.

Given $\epsilon >0$, since $Z = A \cap Q_j$ has measure zero, there
exists a collection of rectangles $\DS \{ R_i \}_{i\geq 0}$ such
that $\DS Z \subset \bigcup_{i=0}^\infty R_i$ and
$\DS \sum_{i=0}^\infty V(R_i) < \frac{\epsilon}{n^{n/2}M^n}$.
By subdividing the rectangles $R_i$ if necessary we may assume that
the $R_i$ are squares with edges of length
$\DS d_i < \frac{1}{j(j+1)\sqrt{n}}$.
Hence $\DS \diam R_i < \frac{1}{j(j+1)}$.
Thus $R_i \subset Q_{j+1}^\circ$ for all $i$.
\pdfbox{mult_integrals/measurefC1}
Using the Mean Value Theorem for functions from
$\DS \RR^n$ to $\DS \RR^n$ (Theorem 9.19 in \cite{R}),
we find that
$\| f(\VEC{x}) - f(\VEC{y}) \| \leq M \|\VEC{x} - \VEC{y} \|$
for all $\VEC{x}, \VEC{y} \in R_i$ and all $i$.  It follows that
\[
  \diam f(R_i) \leq M \diam R_i = \sqrt{n}\, M d_i
\]
for all $i$.  Thus, each $\DS f(R_i)$ is a subset of a
square $\DS S_i \subset \RR^n$ with edges of length
$\sqrt{n}\, M d_i$.  The volume of $S_i$ is
$\DS V(S_i) = n^{n/2} M^n d_i^n$.  We have
\[
f(Z) \subset f\left(\bigcup_{i=1}^\infty R_i\right)
= \bigcup_{i=1}^\infty f(R_i) \subset \bigcup_{i=1}^\infty S_i
\]
with
\[
\sum_{i=1}^\infty V(S_i) = \sum_{i=1}^\infty n^{n/2} M^n d_i^n
  =  n^{n/2} M^n \sum_{i=1}^\infty d_i^n < \epsilon
\]
because $\DS V(R_i) = d_i^n$.
Since $\epsilon$ is arbitrary, this proves that
$\DS f(Z)$ has measure zero.
\end{proof}

\begin{lemma}\label{cov27}
Let $\DS W\subset \RR^n$ be an open set, and
$\DS g:W\to \RR^n$ be a one-to-one and continuously
differentiable function such that
$\det \diff g(\VEC{x}) \neq 0$ for all $\VEC{x} \in W$.  If $f:g(W)\to \RR$
is Riemann integrable with compact support, then
$(f \circ g)\, |\det \diff g|: W \to \RR$ is Riemann integrable with
compact support.
\end{lemma}

\begin{proof}
Let $K = \supp f \subset g(W)$.  The set $K$ is compact and
$f(\VEC{x}) =0$ for $\VEC{x} \in W\setminus K$.

\stage{i} The support of $(f \circ g)\, |\det \diff g|: W \to \RR$ is
a compact subset of $W$.

We have that $\DS g:W\to \RR^n$ is a homeomorphism of $W$
onto $g(W)$ because $g$ one-to-one, $g$ continuously differentiable, and
$\det \diff g(\VEC{x}) \neq 0$ for all $\VEC{x}\in W$ implies that $g$
is an open mapping \footnote{Namely, $g(V)$ is an open subset of
$\DS \RR^n$ for all open sets $V \subset W$.  Note that open
subsets of $W$ with the induced topology from $\DS \RR^n$
are open subsets of $\RR^n$ because $W$ is open.} thanks to the
Inverse Function Theorem.
Thus $\DS g^{-1}(K) \subset W$ is a compact set because
continuous functions map compact sets onto compacts sets.

Since $(f \circ g)(\VEC{x}) = 0$ for
$\DS \VEC{x} \in W \setminus g^{-1}(K)$,
the support of $(f \circ g)\, |\det \diff g|: W \to \RR$ is a subset
of the compact set $\DS g^{-1}(K)$.  Therefore the support
is compact because a closed subset of a compact set is compact.

\stage{ii} $(f \circ g)\, |\det \diff g|: W \to \RR$ is bounded.

Let $M = \sup \{ |(f(\VEC{x})| : \VEC{x} \in g(W) \}$.
We have that $M < \infty$ because $f$ is bounded on $g(W)$.

Since $|\det \diff g|: W \to \RR$ is a continuous
function because $g$ is continuously differentiable,
the function $|\det \diff g|$ restricted to the compact set
$\DS g^{-1}(K)$ reaches it maximum a point in
$\DS g^{-1}(K)$.  Therefore
$L = \max \{ |\det \diff g(\VEC{x})| : \VEC{x} \in g^{-1}(K)\} < \infty$.
It follows that
\[
\big|\big((f \circ g)\, |\det \diff g|\big)(\VEC{x})\big| =
|f(g(\VEC{x}))| \, |\det \diff g(\VEC{x})| \leq M L
\]
for all $\VEC{x} \in W$ because $f(g(\VEC{x})) = 0$ for all
$\DS \VEC{x} \in W \setminus g^{-1}(K)$.

\stage{iii} $(f \circ g)\, |\det \diff g|: W \to \RR$ is continuous
except possibly on a set of measure zero.  In fact, if $Z \subset K$ is the
set of measure zero where $f$ is discontinuous, then we show that
$\DS g^{-1}(Z) \subset g^{-1}(K) \subset W$ is the set of
measure zero where $(f \circ g)\, |\det \diff g|: W \to \RR$ is
discontinuous.

As we say in (i), since $g :W \to g(W)$ is one-to-one and onto $g(W)$,
$g$ is continuously differentiable and
$\det \diff g(\VEC{x}) \neq 0$ for all $\VEC{x} \in W$, we get from the
Inverse Function Theorem that $g$ is an open mapping.  Moreover, we
also get that $\DS \diff g^{-1}(\VEC{y})$ exists for all
$\VEC{y} \in g(W)$.  In particular, $\DS \diff g^{-1}(\VEC{y})
= (\diff g)^{-1}(g^{-1}(\VEC{y}))$ for all $\VEC{y} \in g(W)$ according
to the Inverse Function Theorem.  Thus $g:W \to g(W)$ is a diffeomorphism.

It follows from Lemma~\ref{lemC1Meas0} that $\DS g^{-1}(Z)$
has measure zero.

\stage{iv} We get from (ii) and (iii) that
$(f\circ g) |\det \diff g|:W \to \RR$ is Riemann integrable according to
Theorem~\ref{Rexists}.
\end{proof}

\begin{lemma}\label{cov10}
Let $\DS W\subset \RR^n$ be an open set, and
$\DS g:W\to \RR^n$ be a
one-to-one and continuously differentiable function such that
$\det \diff g(\VEC{x}) \neq 0$ for all $\VEC{x} \in W$.
If Proposition~\ref{cov8} is true for Riemann integrable functions
$f:g(W) \to [0,\infty[$ with compact support, then it is true for
Riemann integrable functions $f:g(W) \to \RR$ with compact support.
\end{lemma}

\begin{proof}
Suppose that Proposition~\ref{cov8} is true for Riemann integrable functions
$f:g(W) \to [0,\infty[$ with compact support.

Given a Riemann integrable function $h: g(W) \to \RR$ with compact support, let
$\DS h^+ = (h + |h|)/2$ and $\DS h^- = (|h|-h)/2$.
We have $\DS h = h^+ - h^-$
with $\DS h^+(\VEC{x}) \geq 0$ and
$\DS h^-(\VEC{x}) \geq 0$ for all
$\VEC{x} \in g(W)$.  Moreover, since $h:g(W)\to \RR$ and
$|h|:g(W)\to \RR$ are Riemann integrable functions with compact support
according to the version of Proposition~\ref{propAIFlstIAF} for
functions of several variables, we have that
$\DS h^+$ and $\DS h^-$ are Riemann integrable 
function on $g(W)$ with compact support.
Hence
\begin{align*}
\rint_{g(W)} h
&= \rint_{g(W)} h^+ - \rint_{g(W)} h^-
= \rint_{W} (h^+ \circ g)\, |\det \diff g| 
- \rint_{W} (h^- \circ g)\, |\det \diff g| \\
&= \rint_{W} (h\circ g)\, |\det \diff g|
\end{align*}
by linearity of the Riemann integral, where the second equality comes
from our assumption that Proposition~\ref{cov8} is true for
Riemann integrable functions $f:g(W) \to [0,\infty[$ with compact support.
\end{proof}

\begin{lemma} \label{cov11}
Let $\DS W\subset \RR^n$ be an open set, and
$\DS g:W\to \RR^n$ be a
one-to-one and continuously differentiable function such that
$\det \diff g(\VEC{x}) \neq 0$ for all $\VEC{x} \in W$.
If Proposition~\ref{cov8} is true for $f = \Chi_Q$ where $Q \subset g(W)$ is a
closed rectangle, then Proposition~\ref{cov8}
is true for all Riemann integrable function $f:g(W) \to \RR$
with compact support.
\end{lemma}

\begin{proof}
We assume that
\[
\rint_{g(W)} \Chi_Q  =
\rint_W (\Chi_Q\circ g)\, |\det \diff g|
= \rint_W \Chi_{g^{-1}(Q)} |\det \diff g|
\]
for all closed rectangles $Q \subset g(W)$.  By linearity of the
Riemann integral,  we also have 
\begin{equation} \label{lemCVEq1}
\int_{g(W)} \alpha \Chi_Q  = \alpha \int_{g(W)} \Chi_Q 
= \alpha \int_W (\Chi_Q\circ g)\, |\det \diff g|
= \int_W \alpha (\Chi_Q\circ g)\, |\det \diff g| \ .
\end{equation}
for all closed rectangles $Q \subset g(W)$ and $\alpha \in \RR$.

According to Lemma~\ref{cov10}, we only have to consider Riemann
integrable functions $f:g(W) \to [0,\infty[$ with compact support.
From Lemma~\ref{cov27}, we have that
$(f \circ g)\,|\det\diff g|:W\to [0,\infty[$ is Riemann
integrable and has compact support.  In the following discussion, we
assume without loss of generality that
$\DS f: \RR^n \to [0,\infty[$ by
setting $f(\VEC{x})=0$ for $\DS \VEC{x} \in \RR^n \setminus g(W)$.

\stage{i}
Let $R$ be a rectangle containing the support of $f$ and $P$ be a
partition of $R$.  As usual, the subrectangles generated by the
partition are denoted $R_{k_1,k_2,\ldots,k_n}$ and
$\DS m_{k_1,k_2,\ldots,k_n}
= \inf \{ f(\VEC{y}) : \VEC{y} \in R_{k_1,k_2,\ldots,k_n}\}$.
Note that $m_{k_1,k_2,\ldots,k_n} = 0$ if
$R_{k_1,k_2,\ldots,k_n} \cap (R \setminus \supp f) \neq \emptyset$.

We have
\begin{align}
\LL_P(f) &= \sum_{R_{k_1,k_2,\ldots,k_n} \subset g(W)}
m_{k_1,k_2,\ldots,k_n} V(R_{k_1,k_2,\ldots,k_n}) \nonumber \\
&= \sum_{R_{k_1,k_2,\ldots,k_n} \subset g(W)} \rint_{g(W)}
m_{k_1,k_2,\ldots,k_n} \Chi_{R_{k_1,k_2,\ldots,k_n}} \nonumber \\
&= \sum_{R_{k_1,k_2,\ldots,k_n} \subset g(W)} \rint_{W}
(m_{k_1,k_2,\ldots,k_n} \Chi_{R_{k_1,k_2,\ldots,k_n}} \circ g)\,
|\det\diff g| \nonumber \\
& = \rint_{W}
\bigg( \bigg(\sum_{R_{k_1,k_2,\ldots,k_n} \subset g(W)}
m_{k_1,k_2,\ldots,k_n} \Chi_{R_{k_1,k_2,\ldots,k_n}}\bigg) 
\circ g\bigg)\, |\det\diff g| \nonumber \\
& \leq \rint_{W} (f \circ g)\,|\det\diff g|  \ ,  \label{cov12}
\end{align}
where the third equality comes from (\ref{lemCVEq1}) with $Q$ replaced
by $R_{k_1,k_2,\ldots,k_n}$, and the fourth equality is a consequence
of the linearity of the Riemann integral.  Note that the sum is
finite.  The inequality is true because
$\DS 0 \leq \sum_{R_{k_1,k_2,\ldots,k_n} \subset g(W)}
m_{k_1,k_2,\ldots,k_n} \Chi_{R_{k_1,k_2,\ldots,k_n}} \leq f$.

Since (\ref{cov12}) is true for all partition $P$ of $R$, we get
\begin{equation} \label{cov13}
\rint_{g(W)} f  = \LL(f) \leq
\rint_{W} (f \circ g)\, |\det\diff g|  \ ,
\end{equation}

\stage{ii}
We now prove that
\begin{equation} \label{cov14}
\rint_{g(W)} f \geq \rint_{W} (f \circ g)\, |\det\diff g| \ ,
\end{equation}
following the approach used in (i).

Let $R$ be a rectangle containing $K = \supp f$ and $P$ be a partition of
$R$.  As before, the subrectangles generated by the partition are
denoted $R_{k_1,k_2,\ldots,k_n}$ and
$\DS M_{k_1,k_2,\ldots,k_n}
= \sup \{ f(\VEC{y}) : \VEC{y} \in R_{k_1,k_2,\ldots,k_n} \}$.
Since $K$ is a compact subset of the open set $g(W)$, we may select
the partition $P$ fine enough such that the
$R_{k_1,k_2,\ldots,k_n} \subset g(W)$ if
$R_{k_1,k_2,\ldots,k_n} \cap \supp f \neq \emptyset$.  
For instance, we may assume that the closed rectangle $R$ is such that
$K \subset R$ with $K \cap \partial R = \emptyset$ and let
\[
\delta = \min \{ \|\VEC{x} - \VEC{y}\| : \VEC{x}\in  K\ \text{and}
\ \VEC{y} \in \partial (R \cap g(W)) \} \ .
\]
We have that $\delta >0$ because $\partial (R \cap g(W))$ is a compact
set that does not intersect $K$.  Therefore, there is a pair
$(\VEC{x},\VEC{y}) \in K \times \partial (R \cap g(W))$
where the minimum $\delta$ is reached.  Note that
$\partial\,g(W) \cap K = \emptyset$ 
because $g(W)$ is open.
\pdfbox{mult_integrals/chofv1}
It suffices to take the partition $P$ such that
$\diam R_{k_1,k_2,\ldots,k_n} < \delta/2$ for all indices
$k_1$, $k_2$,\ldots, $k_n$.  Therefore
$R_{k_1,k_2,\ldots,k_n} \subset g(W)$ for all indices
such that $R_{k_1,k_2,\ldots,k_n} \cap K \neq \emptyset$.

We have
\begin{align}
\U_P (f) &= \sum_{R_{k_1,k_2,\ldots,k_n} \cap \supp f \neq \emptyset}
M_{k_1,k_2,\ldots,k_n} V(R_{k_1,k_2,\ldots,k_n}) \nonumber \\
&= \sum_{R_{k_1,k_2,\ldots,k_n} \cap \supp f \neq \emptyset}
                         \rint_{g(W)} 
M_{k_1,k_2,\ldots,k_n} \Chi_{R_{k_1,k_2,\ldots,k_n}} \nonumber \\
&= \sum_{R_{k_1,k_2,\ldots,k_n} \cap \supp f \neq \emptyset} \rint_{W}
(M_{k_1,k_2,\ldots,k_n} \Chi_{R_{k_1,k_2,\ldots,k_n}} \circ g)\, |\det\diff g|
\nonumber \\
&= \rint_{W}
\bigg( \bigg(\sum_{R_{k_1,k_2,\ldots,k_n} \cap \supp f \neq \emptyset}
M_{k_1,k_2,\ldots,k_n}\Chi_{R_{k_1,k_2,\ldots,k_n}}\bigg) \circ g \bigg)\,
|\det\diff g| \nonumber \\
&\geq \rint_{W} (f \circ g)\, |\det\diff g| \ ,   \label{cov15}
\end{align}
where the third equality comes from (\ref{lemCVEq1}) with $Q$ replaced
by $R_{k_1,k_2,\ldots,k_n}$, and the fourth equality is a consequence
of the linearity of the Riemann integral.  The inequality is true
because $\DS \supp f \subset
\bigcup_{R_{k_1,k_2,\ldots,k_n}\cap \supp f \neq \emptyset}
R_{k_1,k_2,\ldots,k_n}$ and
$\DS f \leq \sum_{R_{k_1,k_2,\ldots,k_n} \cap \supp f \neq \emptyset}
M_{k_1,k_2,\ldots,k_n} \Chi_{R_{k_1,k_2,\ldots,k_n}}$.

Since (\ref{cov15}) is true for all fine enough partition $P$ of $R$,
we get
\[
\rint_{g(W)} f  = \U(f) \geq \rint_{W} (f \circ g) |\det\diff g| \ .
\]
This proves (\ref{cov14}).

It follows from (\ref{cov13}) and (\ref{cov14}) that
\[
\rint_{g(W)} f  = \rint_{W} (f\circ g)\, |\det \diff g| \ . \qedhere
\]
\end{proof}

\begin{lemma}\label{cov16}
Let $\DS W\subset \RR^n$ be an open set.
Proposition~\ref{cov8} is true if $\DS g:W\to \RR^n$ is an
invertible linear map.
\end{lemma}

\begin{proof}
Note that $g$ is a continuous differentiable function and
$\det \diff g(\VEC{x}) = \det g \neq 0$ for al $\VEC{x} \in W$
because $g$ is an invertible linear map.

According to Lemma~\ref{cov11}, we only have to prove that
\[
\rint_{g(W)}\Chi_Q = \rint_W \Chi_{g^{-1}(Q)} \, |\det \diff g|
\]
for all closed rectangles $Q \subset g(W)$.  We have used the fact
that $g$ is one-to-one to get $\Chi_Q \circ g = \Chi_{g^{-1}(Q)}$.
This equation can be restated as
\begin{equation} \label{cov17}
\rint_Q  = \rint_{g^{-1}(Q)} |\det g|  \ .
\end{equation}
Note that $\diff g (\VEC{x}) = g$ for all $\VEC{x} \in W$ because $g$ is
a linear map.  The equation (\ref{cov17}) simply states that the
volume of $Q$ is equal to $|\det g|$ times the volume of
$\DS g^{-1}(Q)$ because $|\det g|$ is constant.

Since $\DS g:\RR^n \to \RR^n$ is a linear map such that
$\det g \neq 0$, we may express $g$ as $g = G_1 G_2 \ldots G_s$ where
the $G_i$ are linear maps $\DS G:\RR^n \to \RR^n$ of the
following forms.
\begin{enumerate}
\item For some $1 \leq k \leq n$ and $a \neq 1$, the components
$g_{i,j}$ of the matrix $G$ are
\[
g_{i,j} =
\begin{cases}
0 & \quad \text{if} \quad i \neq j \\
1 & \quad \text{if} \quad i = j \neq k \\
a & \quad \text{if} \quad i = j = k \\
\end{cases}
\]
$G \VEC{x}$ represents a stretching of $\VEC{x}$ in the direction of
$\VEC{e}_k$ by a factor of $|a|$ with a possible reflection through 
the space orthogonal to $\VEC{e}_k$ if $a<0$.
\item For some $1 \leq k_1 < k_2 \leq n$, the components
$g_{i,j}$ of the matrix $G$ are
\[
g_{i,j} =
\begin{cases}
0 & \quad \text{if} \quad i \neq j \ , \ (i,j) \neq (k_1,k_2)
\text{ and } (i,j) \neq (k_2,k_1) \\
1 & \quad \text{if} \quad (i,j) = (k_1,k_2) \text{ or } (i,j) = (k_2,k_1) \\
0 & \quad \text{if} \quad i = j = k_1 \text{ or } i = j = k_2  \\
1 & \quad \text{if} \quad i = j \neq k_1 \text{ and } i = j \neq k_2
\end{cases}
\]
$G \VEC{x}$ is a permutation of the $k_1$ and $k_2$ coordinates of
$\VEC{x}$.
\item For some $1 \leq k_1, k_2 \leq n$ with $k_1 \neq k_2$, the components
$g_{i,j}$ of the matrix $G$ are
\[
g_{i,j} =
\begin{cases}
0 & \quad \text{if} \quad i \neq j \text{ and } (i,j) \neq (k_1,k_2) \\
1 & \quad \text{if} \quad (i,j) = (k_1,k_2) \\
1 & \quad \text{if} \quad i = j
\end{cases}
\]
$G\VEC{x}$ adds the $k_2$-component of $\VEC{x}$ to the $k_1$-component
of $\VEC{x}$ if $k_1<k_2$ or the $k_1$-component of $\VEC{x}$ to
the $k_2$-component of $\VEC{x}$ if $k_2<k_1$. 
\end{enumerate}

One learns in Linear algebra that transformations of the form (2) and
(3) preserve volume and that transformations of the form (1) change
the volume by a factor of $|a|$.  We have that $|\det G| = |a|$ for
transformations $G$ of the form (1) and $|\det G| = 1$ for
transformation $G$ of the form (2) and (3).  Hence, for any set
$\DS U \subset \RR^n$, the volume of $G(U)$ is the volume of $U$
multiplied by $|\det G|$ for all the linear matrices of the form (1),
(2) and (3).

The volume of $Q$ is therefore the product of the volume of
$\DS g^{-1}(Q)$ by\\
$|\det G_1| \, |\det G_2| \ldots |\det G_s|= |\det g|$ as claimed by
(\ref{cov17}).
\end{proof}

\begin{lemma} \label{cov18}
Let $\DS W\subset \RR^n$ be an open set.
If Proposition~\ref{cov8} is true for the one-to-one and continuously
differentiable functions $\DS g:W\to \RR^n$ such that
$\det \diff g(\VEC{x}) \neq 0$ for all $\VEC{x} \in W$ and
$\diff g(\VEC{a}) = \Id$ for some $\VEC{a} \in W$,
then Proposition~\ref{cov8} is true in general.
\end{lemma}

\begin{proof}
Suppose that Proposition~\ref{cov8} is true for
one-to-one and continuously differentiable functions
$\DS g:W\to \RR^n$ such that
$\det \diff g(\VEC{x}) \neq 0$ for all $\VEC{x}\in W$ and
$\diff g(\VEC{a}) = \Id$ for some $\VEC{a} \in W$.

Suppose that $\DS h:W\to \RR^n$ is a one-to-one and
continuously differentiable function such that
$\det \diff h(\VEC{x}) \neq 0$ for all $\VEC{x}\in W$.
Given $\VEC{a} \in W$, let $T= \diff h(\VEC{a})$.  We have that
$\DS T:\RR^n \to \RR^n$ is a linear invertible map
because $\det \diff h(\VEC{a}) \neq 0$.

Suppose that $f: h(W) \to \RR$ is a Riemann integrable function with
compact support in $h(W)$.  Since
$\DS T: T^{-1}(h(W)) \to \RR^n$ is a
one-to-one and continuously differentiable function such that
$\det \diff T(\VEC{x}) = \det T \neq 0$ for all $\VEC{x} \in W$, we
get from Lemma~\ref{cov27} that
$\DS (f \circ T) |\det T| : T^{-1}(h(W)) \to \RR$ is a
Riemann integrable function with compact support.
Note that $\DS T^{-1}(h(W)) \subset \RR^n$ is an open set
and recall that $\diff T(\VEC{x}) = T$ for all $\VEC{x}$ because $T$
is a linear map.
Furthermore, since $T$ is an invertible linear map, we get
from Lemma~\ref{cov16} that
\begin{equation} \label{cov28}
\rint_{h(W)} f = \rint_{T^{-1}(h(W))} (f \circ T) |\det T| \ .
\end{equation}

Since $\DS (f \circ T) |\det T| : T^{-1}(h(W)) \to \RR$ is a Riemann
integrable function with compact support
and $\DS g = T^{-1}\circ h: W \to \RR^n$ is a
one-to-one and continuously differentiable function such that
\[
\det (\diff g(\VEC{x}))
= \det \left( \diff \left(T^{-1}\circ h\right)(\VEC{x}) \right)
= \det \left( T^{-1} \diff h(\VEC{x}) \right)
= \det(T^{-1}) \det (\diff h(\VEC{x})) \neq 0
\]
for all $\VEC{x} \in W$, we get from Lemma~\ref{cov27} that
\begin{align*}
\big(((f \circ T) |\det T|) \circ g\big) \, |\det \diff g|
&= \left( \left( f \circ T \circ \big(T^{-1}\circ h\big)\right)
\, |\det T | \right)\, \left|\det \diff \big( T^{-1} \circ h\big) \right| \\
&= \left( \left( f \circ T \circ \big(T^{-1}\circ h\big)\right)
\, |\det T | \right)\, |\det T^{-1}| \, |\det \diff h | \\
&= (f \circ h) |\det \diff h|
\end{align*}
is a Riemann integrable function with compact support.  Moreover
\[
 \diff g(\VEC{a}) = \diff (T^{-1} \circ h)(\VEC{a})
 = T^{-1} \diff h(\VEC{a}) = T^{-1} \circ T = \Id \ .
\]
Therefore, we get from our assumption at the beginning of the proof
that
\begin{align}
&\rint_{T^{-1}(h(W))} (f \circ T) |\det T|
= \rint_{g(W)} (f \circ T) |\det T| \nonumber  \\
&\qquad \qquad
= \rint_W \big(((f \circ T) |\det T|) \circ g\big) \, |\det \diff g|
= \rint_W (f \circ h) |\det \diff h| \ .  \label{cov29}
\end{align}

It follows from (\ref{cov28}) and (\ref{cov29}) that
\[
  \rint_{h(W)} f = \rint_W (f \circ h) |\det \diff h| \ . \qedhere
\]
\end{proof}

\begin{proof}[Proof (of Proposition~\ref{cov8})]
We proceed by induction on the dimension $n$.

\stage{i} We prove that Proposition~\ref{cov8} is true for $n=1$.

We may assume that $W$ is connected.
If $W$ is not connected and so is the union of non-intersecting
open intervals, then we may use the linearity of the integral to reduce the
integral on $W$ to a sum of integrals on intervals.

We have from Lemma~\ref{cov27} that $(f\circ g) g'$ is Riemann
integrable with compact support in $W$.  Since
$\DS g^{-1}(\supp f) \subset W$ is a compact set because $\supp f$ 
is compact and $\DS g^{-1}:g(W) \to \RR$ is continuous as we
have shown in the proof of Lemma~\ref{cov27}, and
since $W$ is connected, there exists a closed interval $[\alpha,\beta]
\subset W$ such that $\DS g^{-1}(\supp f) \subset [\alpha,\beta]$.  
Therefore $\supp f \subset g([\alpha,\beta])$.

The condition that $g'(x) \neq 0$ for all $x \in W$ implies that $g$
is strictly increasing or decreasing on $W$.
If $g'(x)>0$ for all $x$, then $g(b)>g(a)$ and it follows from
Theorem~\ref{thCVin1D} that
\begin{equation} \label{cov19}
\int_{g(\alpha)}^{g(\beta)} f(x) \dx{x}
= \int_{\alpha}^{\beta} (f\circ g)(x) g'(x) \dx{x} \ .
\end{equation}
We may rewrite (\ref{cov19}) as
\[
\int_{g([\alpha,\beta])} f = \int_{[\alpha,\beta]} (f\circ g)\, |g'| \ .
\]
If $g'(x)<0$ for all $x$, then $g(b)<g(a)$ and it follows from
Remark~\ref{rmThCVin1D} that
\begin{equation} \label{cov30}
\int_{g(\beta)}^{g(\alpha)} f(x) \dx{x}
= -\int_{\alpha}^\beta (f\circ g)(x) g'(x) \dx{x}
= \int_{\alpha}^{\beta} (f\circ g)(x) |g'(x)| \dx{x}
\end{equation}
because $-g'(x) = | g(x)|$ for all $x$.  Again, we can rewrite
(\ref{cov30}) as
\[
\int_{g([\alpha,\beta])} f = \int_{[\alpha,\beta]} (f\circ g)\, |g'| \ .
\]
In either cases, we get Proposition~\ref{cov8} for $n=1$.

\stage{ii} We assume that Proposition~\ref{cov8} is true for $n<k$.

\stage{iii} We prove that Proposition~\ref{cov8} is true for $n=k$.

We first prove that for every $\VEC{a} \in W$, there exists an open
set $U_{\VEC{a}} \subset W$ containing $\VEC{a}$ such that
\begin{equation}\label{cov22}
\rint_{g(U_{\VEC{a}})} f
= \rint_{U_{\VEC{a}}} (f \circ g)\, |\det \diff g| \ .
\end{equation}
for all Riemann integrable functions $f:g(U_{\VEC{a}}) \to \RR$ with
compact support in $g(U_{\VEC{a}})$.

Given $\VEC{a} \in W$, let $T= \diff g(\VEC{a})$ and
$\DS \tilde{g} = T^{-1} \circ g$.  The map
$\DS \tilde{g}:W \to \RR^k$ is a
one-to-one and continuously differentiable map such that
$\det \diff \tilde{g} (\VEC{x}) \neq 0$ for all $\VEC{x} \in W$ and
$\diff \tilde{g}(\VEC{a}) = \Id$.

Consider
\begin{align*}
Q_1:W & \to \RR^k \\
\VEC{x} & \mapsto
\begin{pmatrix}
\tilde{g}_1(\VEC{x}) & \tilde{g}_2(\VEC{x}) & \ldots
& \tilde{g}_{k-1}(\VEC{x}) & x_k
\end{pmatrix}^\top
\end{align*}
We have
\begin{equation} \label{cov20}
\diff Q_1(\VEC{x}) =
\begin{pmatrix}
\DS \pdydx{\tilde{g}_1}{x_1}(\VEC{x}) &
\DS \pdydx{\tilde{g}_1}{x_2}(\VEC{x}) & \ldots &
\DS \pdydx{\tilde{g}_1}{x_k}(\VEC{x}) \\
\vdots & \vdots & \ddots & \vdots \\
\DS \pdydx{\tilde{g}_{k-1}}{x_1}(\VEC{x}) &
\DS \pdydx{\tilde{g}_{k-1}}{x_2}(\VEC{x}) & \ldots &
\DS \pdydx{\tilde{g}_{k-1}}{x_k}(\VEC{x}) \\[1em]
0 & 0 & \ldots & 1
\end{pmatrix} \ .
\end{equation}
In particular, we have
\begin{equation} \label{cov20Eq1}
\diff Q_1(\VEC{a}) =
\begin{pmatrix}
\Id_{k-1} & 0 \\
0 & 1
\end{pmatrix} = \Id \ .
\end{equation}
Since $\diff Q_1(\VEC{a})$ is invertible and
$\DS \diff Q_1:W\to \RR^k$
is continuous, we get that $\diff Q_1(\VEC{x})$ is invertible for all
$\VEC{x}$ in a neighbourhood of $\VEC{a}$.  Hence, we may use the
Inverse Function Theorem to find an open neighbourhood $U_1$ of $\VEC{a}$ such  
that $\DS Q_1:U_1\to \RR^k$ is a diffeomorphism of the open
set $U_1$ onto the open set $Q_1(U_1)$.

Consider
\begin{align*}
Q_2:Q_1(U_1) & \to \RR^k \\
\VEC{x} &\mapsto
\begin{pmatrix}
x_1 & x_2 & \ldots & x_{k-1} & \tilde{g}_k(Q_1^{-1}(\VEC{x}))
\end{pmatrix}^\top
\end{align*}
We have
\begin{equation} \label{cov35}
\diff Q_2(\VEC{x}) =
\begin{pmatrix}
1 & \ldots & 0 & 0 \\
\vdots & \ddots & \vdots & \vdots \\
0 & \ldots & 1 & 0 \\
\DS \pdfdx{(\tilde{g}_k \circ Q_1^{-1})}{x_1}(\VEC{x}) &
\ldots &
\DS \pdfdx{(\tilde{g}_k \circ Q_1^{-1})}{x_{k-1}}(\VEC{x}) &
\DS \pdfdx{(\tilde{g}_k \circ Q_1^{-1})}{x_k}(\VEC{x})
\end{pmatrix} .
\end{equation}
Using the Inverse Function Theorem for $Q_1$, we get
\[
\diff (\tilde{g}_k \circ Q_1^{-1}) (Q_1(\VEC{a})) = \diff \tilde{g}_k(\VEC{a})
\diff Q_1^{-1}(Q_1(\VEC{a})) = \diff \tilde{g}_k(\VEC{a})
(\diff Q_1(\VEC{a}))^{-1} = \diff \tilde{g}_k(\VEC{a})
\]
because $\diff Q_1(\VEC{a}) = \Id$.  It follows that
\[
\pdfdx{(\tilde{g}_k\circ Q_1^{-1})}{x_j}(Q_1(\VEC{a}))
= \pdydx{\tilde{g}_k}{x_j}(\VEC{a}) = \delta_{j,k}
\]
for $1 \leq j \leq k$ because $\diff \tilde{g}(\VEC{a}) = \Id_k$,
where $\delta_{j,k}$ is the {\bfseries Kronecker delta
function}\index{Kronecker Delta Function} defined by
$\delta_{j,k} = 1$ when $j=k$ and $0$ otherwise.
Hence $\diff Q_2(Q_1(\VEC{a})) = \Id_k$.

Since $\diff Q_2(Q_1(\VEC{a}))$ is invertible and
$\DS \diff Q_2:Q_1(U_1)\to \RR^k$ is continuous, we get that
$\diff Q_2(\VEC{x})$ is invertible for all $\VEC{x}$ in a
neighbourhood of $\VEC{b} = Q_1(\VEC{a})$.  Hence, we may use the Inverse
Function Theorem to find an open neighbourhood
$U_2 \subset Q_1(U_1)$ of $\VEC{b}$ such  
that $\DS Q_2:U_2\to \RR^k$ is a diffeomorphism of the open
set $U_2$ onto the open set $Q_2(U_2)$.
\pdfbox{mult_integrals/cov2}

\stage{iii.a} We first prove that
\begin{equation}\label{cov22A}
\rint_{Q_1(U_1)} f
= \rint_{U_1} (f \circ Q_1)\, |\det \diff Q_1| \ .
\end{equation}
for all Riemann integrable functions $f:Q_1(U_1) \to \RR$ with
compact support in $Q_1(U_1)$.  According to Lemma~\ref{cov11}, it 
suffices to prove (\ref{cov22A}) for $f = \Chi_R$ where
$R$ is a closed rectangle in $Q_1(U_1)$; namely, it suffices to prove
that
\[
\rint_R = \rint_{Q_1(U_1)} \Chi_R
= \rint_{U_1} (\Chi_R \circ  Q_1)\, |\det \diff Q_1|
= \rint_{Q_1^{-1}(R)} |\det \diff Q_1| \ .
\]
Suppose that $R = R_1 \times [a,b]$ where
$\DS R_1 \subset \RR^{k-1}$ is a closed
rectangle such that $R_1\times [a,b] \subset Q_1(U_1)$.

From Fubini's Theorem, we get
\[
\rint_R  = \rint_{[a,b]} \left(
\rint_{R_1}  \dx{\VEC{y}}\right) \dx{x_k} \ .
\]
Let
\[
Q_{1,x_k}(\VEC{y}) =
\begin{pmatrix}
\tilde{g}_1(\VEC{y},x_k) & \tilde{g}_2(\VEC{y},x_k) & \ldots &
\tilde{g}_{k-1}(\VEC{y},x_k)
\end{pmatrix}^\top \ .
\]
for all $(\VEC{y},x_k) \in U_1$.  The important observation is that
\[
Q_1^{-1}(R) = \bigcup_{x_k \in [a,b]} \left( Q_{1,x_k}^{-1}(R_1) \times \{x_k\}
\right) \ .
\]
Note that $Q_{1,x_k}$ is one-to-one because
$\DS Q_1:U_1 \to \RR^k$ is
one-to-one.  Moreover
\[
\det \diff_{\VEC{y}} Q_{1,x_k}(\VEC{y})
= \det \diff Q_1(\VEC{y},x_k) \neq 0
\]
for all $\DS (\VEC{y},x_k)\in U_1$ follows
from (\ref{cov20}) and $\det \diff Q_1(\VEC{x}) \neq 0$ for all
for all $\VEC{x} \in U_1$.  Therefore $Q_{1,x_k}$ satisfies
Proposition~\ref{cov8} in dimension $k-1$ for all $x_k \in [a,b]$.
Hence, by our hypothesis of induction, we have
\[
\rint_{R_1} \dx{\VEC{y}}
= \rint_{Q_{1,x_k}^{-1}(R_1)} \left|\det \diff_{\VEC{y}}
Q_{1,x}(\VEC{y})\right| \dx{\VEC{y}}
= \rint_{Q_{1,x_k}^{-1}(R_1)} \left|\det \diff Q_1(\VEC{y},x) \right|
\dx{\VEC{y}}
\]
for all $x_k \in [a,b]$.  Thus
\[
\rint_R = \rint_{[a,b]} \left( \rint_{Q_{1,x_k}^{-1}(R_1)}
\left|\det \diff Q_1(\VEC{y},x_k) \right| \dx{\VEC{y}}
\right) \dx{x_k}
= \rint_{Q_1^{-1}(R)} \left|\det \diff Q_1 \right| \ ,
\]
where the last equality comes from Fubini's Theorem since
$\det \diff Q_1$ is continuous on $U_1$.  This last statement
needs some clarifications.  Since
$\DS Q_1:U_1 \to \RR^k$ is a one-to-one and continuously
differentiable function with $\det \diff Q_1 (\VEC{x}) \neq 0$ for all
$\VEC{x} \in U_1$, we have that $Q_1 : U_1 \to Q(U_1)$ is a
diffeomorphism.  It follows from Lemma~\ref{lemC1Meas0} that
$\DS Q_1^{-1}(R)$ is Jordan measurable because
$\DS \partial Q_1^{-1}(R) = Q_1^{-1}(\partial R)$ is of
measure zero since $\partial R$ is of measure zero.  We also have that
$\DS Q_1^{-1}(R)$ is bounded because $R$ is a compact set and thus
$\DS Q_1^{-1}(R)$ is a compact set in $\DS \RR^k$.
The Fubini's Theorem is applied to the function
$\DS \Chi_{Q_1^{-1}(R)} |\det \diff Q_1|$ on a rectangle
containing $\DS {Q_1^{-1}(R)}$.

\stage{iii.b} We prove that
\begin{equation}\label{cov22B}
\rint_{Q_2(U_2)} f
= \rint_{U_2} (f \circ Q_2)\, |\det \diff Q_2| \ .
\end{equation}
for all Riemann integrable functions $f:Q_2(U_2) \to \RR$ with
compact support in $Q_2(U_2)$.  According to Lemma~\ref{cov11}, it 
suffices to prove (\ref{cov22B}) for $f = \Chi_R$ where
$R$ is a closed rectangle in $Q_2(U_2)$; namely, it suffices to prove
that
\[
\rint_R = \rint_{Q_2(U_2)} \Chi_R
= \rint_{U_2} (\Chi_R \circ  Q_2)\, |\det \diff Q_2| \ .
= \rint_{Q_2^{-1}(R)} |\det \diff Q_2| \ .
\]
Suppose that $R = R_1 \times [a,b]$ where
$\DS R_1 \subset \RR^{k-1}$ is a closed
rectangle such that $R_1\times [a,b] \subset Q_2(U_2)$.

From Fubini's Theorem, we have
\[
\rint_R = \rint_{R_1} \left(
\rint_{[a,b]} \dx{x_k} \right) \dx{\VEC{y}} \ .
\]
Let
$\DS  Q_{2,\VEC{y}}(x_k) = \tilde{g}_k(Q_1^{-1}(\VEC{y},x_k))$
for all $(\VEC{y},x_k) \in U_2$.  The important observation is that
\[
Q_2^{-1}(R) = \bigcup_{\VEC{y} \in R_1}
\left( \{\VEC{y}\} \times Q_{2,\VEC{y}}^{-1}([a,b]) \right) \ .
\]
Note that $Q_{2,\VEC{y}}$ is one-to-one because
$\DS Q_2:U_2 \to \RR^k$ is one-to-one.  Moreover
\[
\diff_x Q_{2,\VEC{y}}(x_k) =
\dfdx{\tilde{g}_k(Q_1^{-1}(\VEC{y},x))}{x}\Big|_{x = x_k}
\neq 0
\]
for all $x_k \in [a,b]$ follows from (\ref{cov35}) and
$\det \diff Q_2(\VEC{x}) \neq 0$ for all $\VEC{x} \in U_2$.
Therefore $Q_{2,\VEC{y}}$ satisfies
Proposition~\ref{cov8} in dimension $1$ for all $\VEC{y} \in R_1$.
Hence, by our hypothesis of induction, we have
\[
\rint_{[a,b]} \dx{x_k}
= \rint_{Q_{2,\VEC{y}}^{-1}([a,b])} \left| \det \diff_{x_k} Q_{2,\VEC{y}}(x_k)
\right| \dx{x_k}
= \rint_{Q_{2,\VEC{y}}^{-1}([a,b])} 
\left| \det \diff Q_2 (\VEC{y},x_k) \right| \dx{x_k}
\]
for all $\VEC{y} \in R_1$ because
$\DS \det \diff Q_2 (\VEC{y},x_k) =
\diff_{x_k} Q_{2,\VEC{y}}(x_k)$ for all $(\VEC{y},x_k) \in U_2$
according to (\ref{cov35}).  Hence
\[
\rint_R h  = \rint_{R_1} \left(
\rint_{Q_{2,\VEC{y}}^{-1}([a,b])} 
\left| \det \diff Q_2 (\VEC{y},x_k) \right| \dx{x_k} \right) \dx{\VEC{y}}
= \rint_{q_2^{-1}(R)} \left| \det \diff Q_2\right| \ ,
\]
where the last equality comes from Fubini's Theorem since
$\det \diff Q_2$ is continuous on $U_2$.  As in (iii.a), this last statement
needs some clarifications.  Since
$\DS Q_2:U_2 \to \RR^k$ is a one-to-one and continuously
differentiable function with $\det \diff Q_2 (\VEC{x}) \neq 0$ for all
$\VEC{x} \in U_2$, we have that $Q_2: U_2 \to Q(U_2)$ is a diffeomorphism.
It follows as in (iii.a) that $\DS Q_2^{-1}(R)$ is a bounded
and Jordan measurable set.  The Fubini's Theorem is applied to the function
$\DS \Chi_{Q_2^{-1}(R)} |\det \diff Q_2|$ on a rectangle
containing $\DS {Q_2^{-1}(R)}$.

\stage{iii.c}
We now prove (\ref{cov22}) with $U_{\VEC{a}} = U_1$.
We consider the Riemann integrable function $h:\tilde{g}(U_1) \to \RR$
with compact support in $\tilde{g}(U_1)$ defined by $h = f \circ T$
where $f:g(U_1) = (T \circ \tilde{g})(U_1) \to \RR$ is a
Riemann integrable functions with compact support in $g(U_1)$.
Recall that $\DS \tilde{g} = Q_2\circ Q_1 = T^{-1} \circ g$ on $U_1$.

We get from Lemma~\ref{cov27} that
$(h \circ \tilde{g}) |\det \diff \tilde{g}|:U_1 \to \RR$ is a Riemann
integrable function with compact support.

Let $\tilde{h}:U_2 = Q_1(U_1) \to \RR$ be the function defined by
$\tilde{h}(\VEC{x}) = (h \circ Q_2)(\VEC{x}) |\det \diff Q_2(\VEC{x})|$
for $\VEC{x} \in U_2$.  As usual, according to Lemma~\ref{cov27},
we have that $\tilde{h}$ is a Riemann integrable function with
compact support in $U_2$ because $h:\tilde{g}(U_1) = Q_2(U_2) \to \RR$
is a Riemann integrable function with compact support.

We may apply (iii.a) to $\tilde{h}$ to get
\begin{align*}
\rint_{U_1} (h\circ \tilde{g})\, |\det \tilde{g}|
&= \rint_{U_1} h(Q_2(Q_1(\VEC{x})))\, \big|\det (\diff Q_2)(Q_1(\VEC{x}))\big|\,
\big|\det \diff Q_1(\VEC{x})\big| \dx{\VEC{x}} \\
&= \rint_{U_1} (\tilde{h} \circ Q_1) |\det \diff Q_1|
= \rint_{Q_1(U_1)} \tilde{h}
= \rint_{U_2} (h \circ Q_2) |\det \diff Q_2| \ .
\end{align*}
We may also apply (iii.b) to $h$ to get
\[
\rint_{U_2} (h \circ Q_2) |\det \diff Q_2|
= \rint_{Q_2(U_2)} h = \rint_{Q_2(Q_1(U_1))} h = \rint_{\tilde{g}(U_1)} h \ .
\]
Combining the last two equations, we get
\begin{equation} \label{covXEq1}
\rint_{U_1} (h\circ \tilde{g})\, |\det \tilde{g}| = \rint_{\tilde{g}(U_1)} h \ .
\end{equation}

Since $T$ is an invertible linear map, we get from
Lemma~\ref{cov16} and (\ref{covXEq1}) that
\begin{align*}
\rint_{g(U_1)} f
&= \rint_{T^{-1}(\tilde{g}(U_1))} (f \circ T) |\det T|
= \rint_{\tilde{g}(U_1)} h |\det T|
= \rint_{U_1} (h\circ \tilde{g})\, |\det T| \,|\det \tilde{g}| \\
&= \int_{U_1} (f\circ g)\, |\det g| \ .
\end{align*}
Therefore, we may take $U_1$ as the set $U_{\VEC{a}}$ that we were
looking for at the beginning of (iii).

\stage{iii.d} Since $f$ in the statement of Proposition~\ref{cov8} has
a compact support, there exists a finite collection
$\DS \{ U_{\VEC{a}_i} \}_{1 \leq i \leq N}$ of open subsets of $W$
such that $\DS \supp f \subset \bigcup_{i=1}^N g(U_{\VEC{a}_i})$
and (\ref{cov22}) is true for all Riemann integrable functions
with compact support in $g(U_{\VEC{a}_i})$.
Do not forget that $\DS g:W\to \RR^n$ is an open mapping and so
$g(U_{\VEC{a}_i})$ is an open set for all $i$.

Let $\DS \{\phi_i\}_{1 \leq i \leq N}$ be a partition of unity
subordinate to the open cover
$\DS \{ g(U_{\VEC{a}_i}) \}_{1 \leq i \leq N}$ of $\supp f$
as described in Remark~\ref{rmkFPUcase}.  In particular,
$\supp \phi_i \subset g(U_{\VEC{a}_i})$ for all $i$.
We have
$\DS f = \sum_{i=1}^N \phi_i f$ on $g(W)$ and
\begin{align*}
\rint_{g(W)} f & = \rint_{g(W)} \sum_{i=1}^N \phi_i f
= \sum_{i=1}^N \rint_{g(W)}  \phi_i f 
= \sum_{i=1}^N \rint_{g(U_{\VEC{a}_i})}  \phi_i f \\
&= \sum_{i=1}^N \rint_{U_{\VEC{a}_i}} ((\phi_i f) \circ g)\, |\det \diff g|
= \sum_{i=1}^N \rint_{W} ((\phi_i f) \circ g)\, |\det \diff g| \\
& =  \rint_{W} \left(\big(\sum_{i=1}^N \phi_i f\big)
\circ g\right)\, |\det \diff g|
= \rint_{W} (f \circ g)\, |\det \diff g| \ ,
\end{align*}
where the fourth equality comes from (\ref{cov22}) since, for each
$i$, $\phi_i f$ is a Riemann integrable function with compact
support in $g(U_{\VEC{a}_i})$.  We have also used several times the
linearity of the Riemann integral.

This complete the proof by induction and the proof of
Proposition~\ref{cov8}.
\end{proof}

\begin{proof}[Proof (of Theorem~\ref{cov7})]
Since $f:g(W) \to \RR$ is integrable, there exist an admissible open
cover $\DS \{\tilde{U}_{\alpha}\}_{\alpha\in A}$ of $g(W)$ and a
partition of unity $\DS \{\phi_j\}_{j \in \NNp}$
subordinate to $\DS \{\tilde{U}_{\alpha}\}_{\alpha\in A}$ such that
\[
  \int_{g(W)} f = \sum_{j \in \NNp} \rint_{g(W)} \phi_j f \ .
\]
In particular, the sum converges absolutely.
For each $\DS j \in \NNp$, we have that
$\supp \phi_j \subset \tilde{U}_{\alpha_j}$ for some $\alpha_j \in A$.

Since $g:W \to g(W)$ is a diffeomorphism as we have shown in (iii) of
the proof of Lemma~\ref{cov27}, the collection
$\DS \{U_{\alpha}\}_{\alpha\in A}$ where
$\DS U_\alpha = g^{-1}(\tilde{U}_{\alpha})$ for all $\alpha$
is an admissible open cover of $W$ and
$\DS \{\phi_j\circ g\}_{j \in \NNp}$ is
a partition of unity subordinate to
$\DS \{U_{\alpha}\}_{\alpha\in A}$ of class $\DS C^1$
(see Remark~\ref{cov38}).  We have that
$\DS \sum_{j \in \NNp} (\phi_j\circ g)(\VEC{x}) = 1$
for all $\VEC{x} \in W$ and
$\DS \supp (\phi_j \circ g)
\subset U_{\alpha_j} = g^{-1}(\tilde{U}_{\alpha_j})$
for all $j \in \NNp$.

Since $\phi_j f: \tilde{U}_{\alpha_j} \to \RR$ is Riemann integrable
according to Lemma~\ref{lemBZimplR} and has a
compact support in $g(W)$ because $\phi_j$ has a compact support in
$\tilde{U}_{\alpha_j}$, we may use Proposition~\ref{cov8} to conclude that
\[
\rint_{g(U_{\alpha_j})} \phi_j h
= \rint_{U_{\alpha_j}} ((\phi_j h)\circ g)\, |\det \diff g|
\]
for $h=|f|$ and $h=f$, and all $\DS j \in \NNp$.  Thus
\begin{align*}
\int_{g(W)} h &= \sum_{j \in \NNp} \rint_{g(W)} \phi_j h
= \sum_{j \in \NNp} \rint_{g(U_{\alpha_j})} \phi_j h
= \sum_{j \in \NNp} \rint_{U_{\alpha_j}}
((\phi_j h)\circ g)\, |\det \diff g| \\
&= \sum_{j \in \NNp} \rint_W (\phi_j\circ g)(h\circ g)\, |\det \diff g|
= \int_W (h\circ g) \, |\det \diff g|
\end{align*}
for $h=|f|$ and $h=f$.  Thus, $(f \circ g)\, |\det \diff g|$ is
integrable on $W$ and (\ref{cov25}) is satisfied.
\end{proof}

\section{Sard's Theorem}

If $f:U \to \RR$ is an integrable function on an open set $U$, then is
follows from Remark~\ref{rmkC0meas} that the value of the integral
of $f$ on $U$ is equal to the value of the integral of $f$ on
$U \setminus A$ where $A$ is a closed set of measure zero (just consider
$g(\VEC{x}) = f(\VEC{x})$ for $\VEC{x} \in U \setminus A$ and
$g(\VEC{x}) = 0$ for $\VEC{x} \in A$).  It follows from the next
theorem that Theorem~\ref{cov7} is also true if the set
$A = \{ \VEC{x} \in W : \det \diff g(\VEC{x}) = 0 \}$ is a closed set
of measure zero because the set $g(A)$ is always of measure zero.

\begin{defn}
Let $U$ be an open subset of $\DS \RR^n$ and
$\DS f:U \to \RR^n$ be a continuously differentiable
function.

A point $\DS \VEC{x} \in U$ is 
a {\bfseries critical point}\index{Critical Point} of $f$ if
$\det(\diff f(\VEC{x})) = 0$ or, equivalently, $\diff f(\VEC{x})$ has
rank less than $n$ or $\diff f(\VEC{x})$ is not invertible.

A point $\DS \VEC{y} \in \RR^n$ is
a {\bfseries regular value}\index{Regular value} of $f$ if
none of the points $\DS \VEC{x} \in f^{-1}(\{\VEC{y}\})$
is a critical point.
\end{defn}

Before stating the main result of this section, we state a result
about regular points that will be useful later.

\begin{prop}  \label{regValCoIm}
Let $U$ be an open subset of $\DS \RR^n$ and
$\DS f:U \to \RR^n$ be a continuously differentiable
function.  If $\DS \VEC{y} \in \RR^n$ is
a regular value of $f$, then $\DS f^{-1}(\{\VEC{y}\})$ is
the union of isolated points.
\end{prop}

\begin{proof}
Suppose that $\VEC{x} \in \DS f^{-1}(\{\VEC{y}\})$.  Then
$\det (\diff f(\VEC{x}))\neq 0$ because $\VEC{y}$ is regular.  We get
from the Inverse Function Theorem that $f$ is an isomorphism from an
open neighbourhood $U_{\VEC{x}} \subset U$ of $\VEC{x}$ to an open
neighbourhood $V_{\VEC{y}}$ of $\VEC{y}$.  In particular,
$\DS U_{\VEC{x}} \cap f^{-1}(\{\VEC{y}\}) = \{\VEC{x}\}$.
\end{proof}

The next theorem says that almost all points in the image of a
continuously differentiable function are regular values.  The proof
that we give closely follows the proof given in \cite{S}.  Another
proof is given in \cite{GP} which, according to the authors, is
``virtually verbatim'' from \cite{Mi}.

\begin{theorem}[Sard's Theorem]
Let $U$ be an open subset of $\DS \RR^n$ and
$\DS f:U \to \RR^n$ be a continuously differentiable
function.  Moreover, let
$A = \{\VEC{x} \in U : \det(\diff f(\VEC{x})) = 0\}$,
Then $f(A)$ is a set of measure zero.
\end{theorem}

\begin{proof}
\stage{i} Suppose that $R \subset U$ is a closed rectangle with sides
of length $L$.  Since $\diff f$ is continuous on the compact set $R$,
we have that $\DS \pdydx{f_i}{x_j}$ is uniformly continuous
on $R$ for $1\leq i, j \leq n$.   Given $\epsilon > 0$, there exists
$\delta > 0$ such that
$\DS \left| \pdydx{f_i}{x_j}(\VEC{y}) - \pdydx{f_i}{x_j}(\VEC{x})
\right| < \frac{\epsilon}{n}$ for $\|\VEC{y} - \VEC{x}\| < \delta$
and $1\leq i, j \leq n$.
Choose $N$ large enough such that $\DS n (L/N)^2 < \delta^2$
and partition the rectangle $R$ into $\DS N^n$ subrectangles
with sides of length $L/N$.  It follows from our choice of $N$ that
the diameter $(L\sqrt{n})/N$ of the subrectangles is less than
$\delta$.   Hence, if $R_{k_1,k_2,\ldots,k_n}$ is one of the subrectangles, then
$\DS \left| \pdydx{f_i}{x_j}(\VEC{y}) - \pdydx{f_i}{x_j}(\VEC{x})
\right| < \frac{\epsilon}{n}$ for all
$\VEC{x}, \VEC{y} \in R_{k_1,k_2,\ldots,k_n}$
and $1\leq i, j \leq n$ because $\|\VEC{y} - \VEC{x}\| < \delta$
for all $\VEC{x}, \VEC{y} \in R_{k_1,k_2,\ldots,k_n}$.

\stage{ii} Suppose that $R_{k_1,k_2,\ldots,k_n}$ is one of the
subrectangles of $R$ and choose $\VEC{x} \in R_{k_1,k_2,\ldots,k_n}$.  Let
$\DS g:R_{k_1,k_2,\ldots,k_n} \to \RR^n$ be
the function defined by $g(\VEC{y}) = (\diff f(\VEC{x}))\VEC{y} - f(\VEC{y})$
for $\VEC{y} \in R_{k_1,k_2,\ldots,k_n}$.  Since
$\DS \left| \pdydx{g_i}{x_j}(\VEC{y}) \right|
= \left| \pdydx{f_i}{x_j}(\VEC{x}) - \pdydx{f_i}{x_j}(\VEC{y})
\right| < \frac{\epsilon}{n}$ for $\VEC{y} \in R_{k_1,k_2,\ldots,k_n}$, we
get from the Mean Value Theorem and Schwarz inequality that
\begin{align}
|g_i(\VEC{y}) - g_i(\VEC{x})|
&= \left| \sum_{j=1}^n \big( g_i(y_1,y_2,\ldots, y_j,x_{j+1}, \ldots, x_n)  
- g_i(y_1,y_2,\ldots, y_{j-1},x_j, \ldots, x_n) \big) \right| \nonumber \\
&= \left| \sum_{j=1}^n \pdydx{g_i}{x_j}(\VEC{\xi}_j)(y_j - x_j) \right|
\leq \left\| \left(\pdydx{g_i}{x_1}(\VEC{\xi}_j),
\pdydx{g_i}{x_2}(\VEC{\xi}_j) , \ldots,
\pdydx{g_i}{x_k}(\VEC{\xi}_j) \right) \right\| \, \| \VEC{y} - \VEC{x} \|
\nonumber \\
&\leq \sqrt{n} \left(\frac{\epsilon}{n}\right) \| \VEC{y} - \VEC{x} \|
\label{sardEq2}
\end{align}
for $\VEC{y} \in R_{k_1,k_2,\ldots,k_n}$, where
$\VEC{\xi}_j \in R_{k_1,k_2,\ldots,k_n}$
on the segment joining $(y_1,y_2,\ldots, y_j,x_{j+1}, \ldots, x_n)$
and
$(y_1,y_2,\ldots, y_{j-1},x_j, \ldots, x_n)$ is given by the
Mean Value Theorem for $1 \leq j \leq n$.  Hence
\begin{equation} \label{sardEq3}
\left\| g(\VEC{y}) - g(\VEC{x}) \right\|
= \left( \sum_{i=1}^n |g_i(\VEC{y}) - g_i(\VEC{x})|^2 \right)^{1/2}
\leq \epsilon \|\VEC{y} -\VEC{x}\|
\end{equation}
for all $\VEC{y} \in R_{k_1,k_2,\ldots,k_n}$.  We get
\begin{equation}  \label{sardEq1}
\left\| (\diff f(\VEC{x})) (\VEC{y} - \VEC{x}) - (f(\VEC{y}) -
f(\VEC{x})) \right\| = \left\| g(\VEC{y}) - g(\VEC{x}) \right\|
\leq \epsilon \|\VEC{y} - \VEC{x} \| < \frac{\epsilon L \sqrt{n}}{N}
\end{equation}
for $\VEC{x}, \VEC{y} \in R_{k_1,k_2,\ldots,k_n}$.

\stage{iii} Suppose that $A \cap R_{k_1,k_2,\ldots,k_n}\neq \emptyset$ for some
subrectangle $R_{k_1,k_2,\ldots,k_n}$.  Choose
$\VEC{x} \in A \cap R_{k_1,k_2,\ldots,k_n}$.
Since $\diff f(\VEC{x})$ is not onto $\DS \RR^n$, its image
is a subspace $\DS S \subset \RR^n$ of dimension at most
$n-1$.  In particular, $(\diff f(\VEC{x})) (\VEC{y} - \VEC{x}) \in S$ for all
$\VEC{y} \in R_{k_1,k_2,\ldots,k_n}$.  It follows from (\ref{sardEq1}) that
$\DS \dist{S+f(\VEC{x})}{f(\VEC{y})}
= \dist{S}{f(\VEC{y}) - f(\VEC{x})} \leq (\epsilon L \sqrt{n})/N$
for all $\VEC{y} \in R_{k_1,k_2,\ldots,k_n}$.

Moreover, since $\DS \pdydx{f_i}{x_j}$ is continuous on the
compact set $R$ for $1\leq i, j \leq n$, they all reach their maximum
on $R$.  Thus, there exists $M>0$ such that
$\DS \left| \pdydx{f_i}{x_j}(\VEC{z})\right| \leq M$ for all
$\VEC{z} \in R$ and $1\leq i, j \leq n$.  Proceeding as we did for
(\ref{sardEq2}) and (\ref{sardEq3}) with $f$ instead of $g$, we get
$|f_i(\VEC{x}) - f_i(\VEC{y})| \leq M\sqrt{n}\, \|\VEC{x} - \VEC{y}\|$ and
$\DS \|f(\VEC{x})- f(\VEC{y})\| \leq M n \|\VEC{x} - \VEC{y}\|
\leq (M n^{3/2} L)/N$ for all $\VEC{y} \in R$.

Therefore, when $A \cap R_{k_1,k_2,\ldots,k_n} \neq \emptyset$, we have that
$f(A\cap R_{k_1,k_2,\ldots,k_n})$ is contained in a cylinder
$C$ of radius
$\DS (M n^{3/2} L)/N$ and height $2 (\epsilon L \sqrt{n})/N$
with an axis perpendicular to the space $S$.
\pdfbox{mult_integrals/sard}
The volume of $C$ is
\begin{align*}
&\left(\frac{2 \pi^{(n-1)/2}}{(n-1)\Gamma((n-1)/2)}\right)
\left(\frac{M n^{3/2}L}{N}\right)^{n-1}
\left(\frac{2\epsilon L \sqrt{n}}{N}\right) \\
&\qquad
= \epsilon \Big( \underbrace{\frac{4 \pi^{(n-1)/2} n^{(3n-1)/2} M^{n-1}}
{(n-1)\Gamma((n-1)/2)}}_{\equiv B} \Big) \left(\frac{L}{N}\right)^n
= \epsilon B \left(\frac{L}{N}\right)^n \ ,
\end{align*}
where we have used the fact that the ``area'' of the cross section
perpendicular to the axis of the cylinder is the volume of a sphere of 
radius $\DS (M n^{3/2} L)/N$ in the subspace $S$ of
dimension $n-1$.  The volume of a sphere of radius $r$ in
$\DS \RR^m$ is
$\DS (2 \pi^{m/2} r^m)/(m \Gamma(m/2))$ \footnote{
Integrating $\DS e^{-\pi \|\VEC{x}\|_2^2}$ over
$\DS \RR^m$ using spherical coordinates (i.e.\ coordinates of the
form $\VEC{x} = r \VEC{y}$ with $\|\VEC{y}\|=1$), the reader can verify that
the volume of $B_1(\VEC{0})$ is $\DS 2 \pi^{m/2}/(m\Gamma(m/2))$.
In particular, the volume of $B_1(\VEC{0})$ in
$\DS \RR^2$ and $\DS \RR^3$ are respectively $\pi$
and $4\pi/3$.}

Hence $f(A \cap R)$ is contained in the union of $\DS N^n$
cylinders whose total volume is at most
$\DS \epsilon B (L/N)^n N^n = \epsilon B L^n$.  Since $\epsilon$ is
arbitrary, we have that $f(A\cap R)$ is of zero content.

\stage{iv} Since $U$ is the union of closed rectangles $\{R_i\}_{i\in \NN}$ 
(see remark below) and the countable union of sets of zero content is
a set of measure zero, we have that
$\DS f(A) = \bigcup_{i\in \NN} f(A \cap R_i)$
is of measure zero.
\end{proof}

\begin{rmk}
There are several methods to prove that and open set
$\DS V \subset \RR^n$ is the countable union of closed rectangles.
We present two methods.  The first one is quick and simple but lack
the constructive approach which is often really useful.

\stage{$\DS \mathbf{1^{st}}$ Method}
For each $\VEC{x} \in U$, choose a closed rectangle
$R_{\VEC{x}}=[a_1(\VEC{x}),b_1(\VEC{x})]\times[a_2(\VEC{x}),b_2(\VEC{x})]
\times\cdots\times[a_n(\VEC{x}),b_n(\VEC{x})] \subset U$
such that $\VEC{x} \in R_{\VEC{x}}$ and
$a_j(\VEC{x}),b_j(\VEC{x}) \in \QQ$ for $1\leq j \leq n$.  Then
$\DS U = \bigcup_{\VEC{x}\in V} R_{\VEC{x}}$.  Surprisingly,
the number of rectangles is countable because\\
$\DS \{ (a_1(\VEC{x}),b_1(\VEC{x}),a_2(\VEC{x}),b_2(\VEC{x}),\ldots,
a_n(\VEC{x}),b_n(\VEC{x})) : \VEC{x} \in U \} \subset \QQ^{2n}$
and a subset of a countable set is a countable set.

\stage{$\DS \mathbf{2^{nd}}$ Method}
Let $U_j = \{ \VEC{x} \in U : \|\VEC{x}\| < j \ \text{and} \
\dist{\VEC{x}}{\partial U} > 1/j \}$ for $j>0$ and
$V_0 = \emptyset$.   Note that many of the sets $U_j$ may be empty but
that does not mater.
Moreover, let $K_j = \overline{U}_j \setminus U_{j-1}$ for $j>0$.
We have that the $K_j$ are compacts subsets of $U$ and
$\DS U = \bigcup_{j>0} K_j$.
For each $\VEC{x} \in K_j$, choose a closed rectangle
$R_{\VEC{x}} \subset U$ such that $\VEC{x} \in R_{\VEC{x}}$.
Since $\DS \{ R_{\VEC{x}}^\circ \}_{\VEC{x}\in K_j}$ is an open
cover of the compact set $K_j$, there exists a finite subcover
$\DS \{ R_{j,i}^\circ \}_{1 \leq i \leq I_j}$ of $K_j$.
The union of the $\DS \{ R_{j,i} \}_{1 \leq i \leq I_j}$ for
$j>0$ is a countable collection of closed rectangles such that
$\DS U = \bigcup_{1 \leq i \leq I_j,\, j>0} R_{j,i}$.
\end{rmk}

\section{Exercises}

\begin{question}
Suppose that $f:[a,b]\to \RR$ is Riemann integrable.  Prove that the
graph of $f$ has zero content.       \label{graphcontzero}
\end{question}

\begin{sol}
Given $\epsilon>0$, we may choose a partition $P=\{p_i: 0 \leq i \leq N\}$ of
$[a,b]$ such that $\DS \U_P(f) - \LL_P(f) < \epsilon$
because $f$ is Riemann integrable.

As usual, let $m_i = \inf\{f(x) : p_{i-1}\leq x \leq p_i \}$ and
$M_i = \sup\{f(x):p_{i-1}\leq x \leq p_i \}$ for $0<i\leq N$.
Consider the rectangles
$E_i = \{ (x,y) : p_{i-1} \leq x \leq p_i \ \text{and}
\ m_i \leq y \leq M_i \}$ for $0<i\leq N$.  We have that the graph of
$f$ is a subset 
of $\DS E = \bigcup_{i=1}^N E_i$ and the area of $E$ is
\[
  A(E) = \sum_{i=1}^N (M_i - m_i)(p_i - p_{i-1})
  = \U_P(f) - \LL_P(f) < \epsilon \ .
\]
Since $\epsilon$ is arbitrary, this proves that the graph of $f$ has
content zero.
\end{sol}

\begin{question}
Suppose that $f:[a,b]\to [0,\infty[$ is Riemann integrable.  Let\\
$\DS
S = \left\{ (x,y) : a \leq x \leq b \ \text{and} \ 0 \leq y \leq f(x)
\right\}$.
Prove that $S$ is Jordan measurable and the area $A(S)$ of $S$ is equal to
$\DS \int_a^b f(x) \dx{x}$.
\end{question}

\begin{sol}
\stage{i}
Since $f$ is Riemann integrable, $f$ is bounded.  Thus, there exists $C>0$
such that $\sup \{ f(x) : a \leq x \leq b \} < C$.  Since
$\DS S\subset R
= \left\{ (x,y) : a \leq x \leq b \ \text{and} \ 0 \leq y \leq C \right\}$,
$S$ is bounded.  Moreover, $\partial S$ is the union of four sets of
zero content and so of measure zero.
The graph of $f$ (including the vertical lines joining
points of discontinuity) has content zero according to
Question~\ref{graphcontzero}.
The sets $\{ (x,y) : x = a \ \text{and} \ 0 \leq y \leq f(a) \}$,
$\{(x,y) : x=b \ \text{and} \ 0 \leq y \leq f(b)\}$ and
$\{(x,y) : y=0 \ \text{and} \ a \leq x \leq b\}$ have all zero content
according to Proposition~\ref{curveC1zero}.
This proves that $S$ is Jordan measurable.

\stage{ii}
Since $S$ is Jordan measurable, we have that the characteristic function
$\Chi_S$ is Riemann integrable on $R$ defined above.
Given $\epsilon >0$, choose a partition $\{P,Q\}$ of $R$ such that
\begin{equation}\label{airunderf}
  \U_{\{P,Q\}}(\Chi_S) - \LL_{\{P,Q\}}(\Chi_S) < \epsilon \ .
\end{equation}
We assume that $P=\{p_i : 0 \leq i \leq N\}$ and
$Q=\{q_j: 0 \leq j \leq M\}$.  Let
$m_i = \inf\{ f(x) : p_{i-1} \leq x \leq p_i \}$,
$M_i = \sup\{ f(x) : p_{i-1} \leq x \leq p_i \}$ and
$R_{i,j} = \{(x,y) : p_{i-1} \leq x \leq p_i \ \text{and}
\ q_{j-1} \leq y \leq q_j\}$.
Then
\begin{align*}
\U_P(f) &= \sum_{i=1}^N M_i (p_i-p_{i-1}) \leq \sum_{i=1}^N
\left( \sum_{R_{i,j}\cap S \neq \emptyset} (q_j - q_{j-1})\right) (p_i - p_{i-1})
\\
&= \sum_{R_{i,j}\cap S \neq \emptyset} (q_j - q_{j-1}) (p_i - p_{i-1})
= \U_{\{P,Q\}}(\Chi_S) \ .
\end{align*}
The following figure shows the rectangle $R_{i,j}$ included in the
previous summation.
\pdfbox{mult_integrals/area1}

Similarly,
\begin{align*}
\LL_P(f) &= \sum_{i=1}^N m_i (p_i-p_{i-1}) \geq \sum_{i=1}^N
\left( \sum_{R_{i,j}\subset S} (q_j - q_{j-1})\right) (p_i - p_{i-1}) \\
&= \sum_{R_{i,j}\subset S} (q_j - q_{j-1}) (p_i - p_{i-1})
= \LL_{\{P,Q\}}(\Chi_S) \ .
\end{align*}
Thus
\[
  \LL_{\{P,Q\}}(\Chi_S) \leq \LL_P(f) \leq \int_a^b f(x) \dx{x} \leq
  \U_P(f) \leq \U_{\{P,Q\}}(\Chi_S)
\]
and
\[
  \LL_{\{P,Q\}}(\Chi_S) \leq A(S) = \int_R \Chi_S
  \leq \U_{\{P,Q\}}(\Chi_S) \ .
\]
It follows from (\ref{airunderf}) that
$\DS \left| A(S) - \int_a^b f(x)\dx{x} \right| < \epsilon$.
Since $\epsilon$ is arbitrary, we get
$\DS A(S) = \int_a^b f(x)\dx{x}$.
\end{sol}

\begin{question}
Let $S$ be a bounded subset of $\DS \RR^2$.   \label{innerarea}
Prove that $S$ and $\DS S^\circ$ have the same inner area;
namely, show that $\LL(\Chi_S) = \LL(\Chi_{S^\circ})$.
Note that we do not require $S$ to be Jordan measurable.  If it was,
then the question would be easy to answer because
$\DS S \setminus S^\circ \subset \partial S$ would be a
subset of the closed set of measure zero $\partial S$ and we will be
able to use Remark~\ref{rmkC0meas}.
\end{question}

\begin{sol}
\stage{i} We first prove that $\LL(\Chi_{S^\circ}) \leq \LL(\Chi_S)$.
Let $R$ be a rectangle containing $S$ and $\{P,Q\}$ be a partition
of $R$.  We assume that $P=\{p_i:0 \leq i \leq N\}$ and
$Q=\{q_j: 0 \leq j \leq M\}$, and let
$R_{i,j} = \{(x,y) : p_{i-1}\leq x \leq p_i \ \text{and}
\ q_{j-1} \leq y \leq q_j\}$ for $0 < i \leq N$ and $0 < j \leq M$.
Given $\DS V \subset \RR^2$, let
\[
m_{i,j}(V) = \inf \{ \Chi_V(x,y) : (x,y) \in R_{i,j}\}
= \begin{cases}
1 & \quad \text{if} \ R_{i,j} \subset V \\
0 & \quad \text{otherwise}
\end{cases}
\]
for $0 < i \leq N$ and $0 < j \leq M$.
We have that $\DS m_{i,j}(S^\circ) \leq m_{i,j}(S)$ for
$0 < i \leq N$ and $0 < j \leq M$.
Hence
\begin{align*}
\LL_{\{P,Q\}}(\Chi_{S^\circ})
&= \sum_{R_{i,j} \subset S^\circ} (p_i-p_{i-1})(q_j- q_{j-1}) 
\leq \sum_{R_{i,j} \subset S} (p_i-p_{i-1})(q_j- q_{j-1}) \\
&= \LL_{\{P,Q\}}(\Chi_S) \leq \LL(\Chi_S) \ .
\end{align*}
Since this is true for all partition $\{P,Q\}$ of $R$, we get
\[
\LL(\Chi_{S^\circ}) = \sup \left\{\LL_{\{P,Q\}}(\Chi_{S^\circ}) :
\{P,Q\} \ \text{is a partition of}\  R\right\} \leq \LL(\Chi_S) \ .
\]

\stage{ii} We prove that $\LL(\Chi_S) \leq \LL(\Chi_{S^\circ})$.
Let $R$ be a rectangle containing $S$ and $\{P,Q\}$ be a partition of $R$.
We assume that $P$, $Q$ and $R_{i,j}$ are as defined in (i).

Given $\epsilon >0$, choose
$\DS \delta < \frac{\epsilon}{\LL(\Chi_S)}$.
We assume that $\LL(\Chi_S) > 0$ otherwise
$0 = \LL(\Chi_S) \leq \LL(\Chi_{S^\circ})$ and there is nothing
to prove.  For each rectangle $R_{i,j} \subset S$, we construct a rectangle
$\DS \breve{R}_{i,j} \subset R_{i,j}^\circ \subset S^\circ$ by
shrinking the rectangle $R_{i,j}$ by a factor of
$\sqrt{1-\delta}$ in both directions.
We have $A(\breve{R}_{i,j}) = (1-\delta) A(R_{i,j})$ and
$\breve{m}_{i,j}= \inf \{ \Chi_{S^\circ}(x,y) : (x,y) \in \breve{R}_{i,j}\}
= m_{i,j}(S) = 1$ for all $R_{i,j} \subset S$.

We choose a refinement $\{\tilde{P},\tilde{Q}\}$ of the partition
$\{P,Q\}$ that includes all the vertices (i.e.\ the corners) of the
rectangles $\breve{R}_{i,j}$.  Let $\tilde{R}_{r,s}$ be the rectangles
associated to the partition $\{\tilde{P},\tilde{Q}\}$.
Note that each rectangle $\breve{R}_{i,j}$ is represented by one and
only one of the $\tilde{R}_{r,s}$.  Therefore,
\begin{align*}
\LL_{\{\tilde{P},\tilde{Q}\}}(\Chi_{S^\circ})
&= \sum_{\tilde{R}_{r,s} \subset S^\circ} A(\tilde{R}_{r,s}) 
\geq \sum_{R_{i,j} \subset S} A(\breve{R}_{i,j})
= (1-\delta) \sum_{R_{i,j} \subset S} m_{i,j} A(R_{i,j}) \\
&= (1 - \delta) \LL_{\{P.Q\}}(\Chi_S)
\geq \left( 1 - \frac{\epsilon}{ \LL(\Chi_S)}\right)
\LL_{\{P.Q\}}(\Chi_S) \ .
\end{align*}
Since this relation is true for all refinements
$\{\tilde{P},\tilde{Q}\}$ of the partitions $\{P,Q\}$ containing the
$\breve{R}_{i,j}$, we get
\[
  \LL(\Chi_{S^\circ}) \geq \left( 1 - \frac{\epsilon}{ \LL(\Chi_S)}\right)
\LL_{\{P.Q\}}(\Chi_S) \ .
\]
Since this relation is true for all partitions $\{P,Q\}$ of $R$, we get
\[
  \LL(\Chi_{S^\circ}) \geq \left( 1 - \frac{\epsilon}{ \LL(\Chi_S)}\right)
\LL(\Chi_S) = \LL(\Chi_S) - \epsilon \ .
\]
Finally, since $\epsilon$ is arbitrary, we get
$\LL(\Chi_{S^\circ}) \geq \LL(\Chi_S)$.
\end{sol}

\begin{question}
Let $S$ be a bounded subset of $\DS \RR^2$.    \label{outerarea}
Prove that $S$ and $\overline{S}$ have the same outer area; namely, show that
$\U(\Chi_S) = \U(\Chi_{\overline{S}})$.  As for
question~\ref{innerarea}, we do not require $S$ to be Jordan measurable.
\end{question}

\begin{sol}
The proof is similar to the proof given for Question~\ref{innerarea}.
\end{sol}

\begin{question}
Let $S$ be a bounded subset of $\DS \RR^2$.  Prove that outer
area of $S$ is equal to the inner area of $S$ plus the outer area of
$\partial S$; namely, prove that
$\U(\Chi_S) = \LL(\Chi_{\overline{S}}) + \U(\Chi_{\partial S})$.
Again, as for question~\ref{innerarea}, we do not require $S$ to be Jordan
measurable.
\end{question}

\begin{sol}
We have that $\DS \partial S = \overline{S} \setminus S^\circ$.

\stage{i} We first prove that
$\U(\Chi_S) \leq \LL(\Chi_S) + \U(\Chi_{\partial S})$.
According to the result of Question~\ref{innerarea}, it is enough to
prove that $\U(\Chi_S) \leq \LL(\Chi_{S^\circ}) + \U(\Chi_{\partial S})$.

Let $\{P,Q\}$ be a partition of a rectangle $R \supset \overline{S}$.  As
usual, we assume that $P=\{p_i: 0 \leq i \leq N\}$ and
$Q=\{q_j:0 \leq j \leq M\}$, and let
$R_{i,j} = \{(x,y) : p_{i-1}\leq x \leq p_i\ \text{and}
\ q_{j-1} \leq y \leq q_j\}$ for $0 < i \leq N$ and $0 < j \leq M$.
We have
\[
M_{i,j}(V) = \sup \{ \Chi_V(x,y) : (x,y) \in R_{i,j} \}
= \begin{cases}
1 & \quad \text{if} \ R_{i,j} \cap V \neq \emptyset \\
0 & \quad \text{otherwise}
\end{cases}
\]
and
\[
m_{i,j}(V) = \inf \{ \Chi_V(x,y) : (x,y) \in R_{i,j}\}
= \begin{cases}
1 & \quad \text{if} \ R_{i,j} \subset V \\
0 & \quad \text{otherwise}
\end{cases}
\]
for all $\DS V \subset \RR^2$.
Note that $m_{i,j}(V) = M_{i,j}(V)$ for $R_{i,j} \subset V$.
Hence
\begin{align*}
\U(\Chi_S) &= \U(\Chi_{\overline{S}}) \leq \U_{\{P,Q\}}(\Chi_{\overline{S}})
= \sum_{R_{i,j} \cap \overline{S} \neq \emptyset} A(R_{i,j})
= \sum_{R_{i,j} \subset S^\circ} A(R_{i,j})
+ \sum_{R_{i,j} \cap \partial S \neq \emptyset} A(R_{i,j}) \\
&= \LL_{\{P,Q\}}(\Chi_{S^\circ}) + \U_{\{P,Q\}}(\Chi_{\partial S})
\leq \LL(\Chi_{S^\circ}) + \U_{\{P,Q\}}(\Chi_{\partial S}) \ ,
\end{align*}
where the first equality is a consequence of the result of
Question~\ref{outerarea}.  Since this is true for all
partitions $\{P,Q\}$ of $R$, we get
$\DS \U(\Chi_S) \leq \LL(\Chi_{S^\circ}) + \U(\Chi_{\partial S})$.

\stage{ii} We now prove that
$\DS \U(\Chi_S) \geq \LL(\Chi_S) + \U(\Chi_{\partial S})$.
Again, according to the result of Question~\ref{innerarea}, it is
enough to prove that
$\DS \U(\Chi_S) \geq \LL(\Chi_{S^\circ}) + \U(\Chi_{\partial S})$.

Let $R$ be rectangle containing $\overline{S}$.  Given $\epsilon >0$,
choose a partition $\{\tilde{P},\tilde{Q}\}$ of $R$ such that
$\DS \LL_{\{\tilde{P},\tilde{Q}\}}(\Chi_{S^\circ})
+ \epsilon > \LL(\Chi_{S^\circ})$.
Let $\{P,Q\}$ be a refinement of $\{\tilde{P},\tilde{Q}\}$.
We assume that $P$, $Q$ and $R_{i,j}$ are as defined in (i).
Then
\begin{align*}
\U_{\{P,Q\}}(\Chi_S) &= \U_{\{P,Q\}}(\Chi_{\overline{S}})
= \sum_{R_{i,j} \cap \overline{S} \neq \emptyset} A(R_{i,j})
= \sum_{R_{i,j} \subset S^\circ} A(R_{i,j}) 
+ \sum_{R_{i,j} \cap \partial S \neq \emptyset} A(R_{i,j}) \\
& = \LL_{\{P,Q\}}(\Chi_{S^\circ}) + \U_{\{P,Q\}}(\Chi_{\partial S})
\geq \LL_{\{\tilde{P},\tilde{Q}\}}(\Chi_{S^\circ}) + \U(\Chi_{\partial S}) \\
&> \LL(\Chi_{S^\circ}) - \epsilon + \U(\Chi_{\partial S}) \ ,
\end{align*}
where the first equality is again a consequence of the result of
Question~\ref{outerarea}.
Since this is true for any refinement $\{P,Q\}$ of
$\{\tilde{P},\tilde{Q}\}$, we get
$\DS \U(\Chi_S) \geq \LL(\Chi_{S^\circ}) - \epsilon +
\U(\Chi_{\partial S})$.
Since this last inequality is true for all $\epsilon >0$, we finally
get
$\DS \U(\Chi_S) \geq \LL(\Chi_{S^\circ}) + \U(\Chi_{\partial S})$.
\end{sol}

\begin{question}
Prove the                         \label{MeanValueTHInt}
{\bfseries Mean Value Theorem for integrals}\index{Mean
Value Theorem for Integrals}.  Namely, let
$\DS S\subset \RR^n$ be a compact, convex and Jordan
measurable set.  If $g:S\to \RR$ is Riemann integrable,
$g(\VEC{x}) \geq 0$ for all $\VEC{x} \in S$, and $f:S\to \RR$ is
continuous, then there exists $\VEC{a}\in S$ such that
$\DS \int_S f(\VEC{x}) g(\VEC{x}) \dx{\VEC{x}} =
f(\VEC{a}) \int_S g(\VEC{x}) \dx{\VEC{x}}$.
\end{question}

\begin{sol}
Since $f, g:S \to \RR$ are Riemann integrable, we have that
$fg:S\to \RR$ is Riemann integrable (Question~\ref{intfg}).
Let
$\DS m = \inf\{ f(\VEC{x}) : \VEC{x} \in S \}$ and
$\DS M = \sup\{ f(\VEC{x}) : \VEC{x} \in S \}$.
$m$ and $M$ are real number because $f$ is continuous on the compact set
$S$.  In fact, $m=f(\VEC{x}_1)$ and $M=f(\VEC{x}_2)$ for some $\VEC{x}_1$ and
$\VEC{x}_2$ in $S$.

Since $m \leq f(\VEC{x}) \leq M$ for all $\VEC{x} \in S$, we get
\[
m \int_S g(\VEC{x}) \dx{\VEC{x}} \leq 
\int_S g(\VEC{x})f(\VEC{x}) \dx{\VEC{x}} \leq
M \int_S g(\VEC{x}) \dx{\VEC{x}} \ .
\]

The result is trivial if
$\DS \int_S g(\VEC{x}) \dx{\VEC{x}} = 0$.  We therefore
assume that $\DS \int_S g(\VEC{x}) \dx{\VEC{x}} >0$.
Thus
\[
m \leq \frac{\int_S g(\VEC{x})f(\VEC{x}) \dx{\VEC{x}}}
{\int_S g(\VEC{x}) \dx{\VEC{x}} } \leq M \ .
\]
Since $f$ is continuous on $S$ and the line
$\DS \ell = \{\VEC{x}_1 + t(\VEC{x}_2-\VEC{x}_1) : 0 \leq t \leq 1\}$
between $\VEC{x}_1$ and $\VEC{x}_2$ is completely inside $S$ because $S$ is
convex, we may use the Intermediate Value Theorem to find
$t_1 \in [0,1]$ such that
\[
f\big(\VEC{x}_1 + t_1(\VEC{x}_2-\VEC{x}_1)\big)
= \frac{\int_S g(\VEC{x})f(\VEC{x}) \dx{\VEC{x}}}
{\int_S g(\VEC{x}) \dx{\VEC{x}} } \ .
\]
Hence, we may take $\VEC{a} = \VEC{x}_1 + t_1(\VEC{x}_2-\VEC{x}_1)$.
\end{sol}

\begin{question}
Let
\[
f(x,y) = \begin{cases}
1 & \qquad \text{if} \quad \VEC{x} \in \QQ \\
2y & \qquad \text{if} \quad \VEC{x} \in \RR\setminus \QQ
\end{cases}
\]
Prove that $\DS \int_0^1 \int_0^1 f(x,y) \dx{y}\dx{x} =1$ but
$\DS \int_0^1 \int_0^1 f(x,y) \dx{x}\dx{y}$ does not exist.
\end{question}

\begin{question}
Consider the function
\[
f(x,y) = \begin{cases} \DS \frac{1}{q} &
\quad \text{if}\ x = p/q \ \text{where $0< p \leq q$ are relatively prime} \\
0 & \quad \text{otherwise}
\end{cases}
\]
Determine if $f$ is Riemann integrable on $[0,1]\times [0,1]$.
\end{question}

\begin{question}
Consider the function
\[
f(x,y) =
\begin{cases}
y^{-2} & \quad \text{if}\ 0 < x < y < 1 \\
-x^{-2} & \quad \text{if}\ 0 < y < x < 1 \\
0 & \quad \text{otherwise}
\end{cases}
\]
Let $S$ be the square $S = [0,1]\times[0,1]$.\\
\subQ{a} Prove that $f$ is not integrable on $S$.\\
\subQ{b} Prove that $x \to f(x,y)$ is integrable on $[0,1]$ for all
$y \in [0,1]$ and compute the integral.\\
\subQ{c} Prove that $y \to f(x,y)$ is integrable on $[0,1]$ for all
$x \in [0,1]$ and compute the integral.\\
\subQ{d} Prove that
$\DS \int_0^1 \int_0^1 f(x,y) \dx{x}\dx{y}$ and
$\DS \int_0^1 \int_0^1 f(x,y) \dx{y}\dx{x}$ exist but are
not equal.
\end{question}

\begin{sol}
\subQ{a} The function $f$ is not integrable on $S$ because $f$ is not
bounded on $S$.  For instance,
$\DS f(z/2,z) = 1/z^2 \to \infty$ as $z \to 0$
and $\DS f(z,z/2) = -1/z^2 \to -\infty$ as $z \to 0$.

\subQ{b} For $y \in ]0,1[$ fixed, we have
\[
f_y(x) = f(x,y) =
\begin{cases}
y^{-2} & \quad \text{if}\ 0 < x < y < 1 \\
-x^{-2} & \quad \text{if}\ 0 < y < x < 1 \\
0 & \quad \text{if}\ x = 0, y\ \text{or}\ 1
\end{cases}
\]
The function $f_y$ is integrable because it is a bounded function and
the set of points where $f_y$ is discontinuous is $\{0, y,1\}$ which
is a set of zero content.  We have
\begin{align*}
\int_0^1 f(x,y) \dx{x} &= \int_0^1 f_y(x) \dx{x}
= \int_0^y f_y(x) \dx{x} + \int_y^1 f_y(x) \dx{x} 
= \int_0^y \frac{1}{y^2} \dx{x} - \int_y^1 \frac{1}{x^2} \dx{x} \\
&= \frac{x}{y^2} \bigg|_{x=0}^y + \frac{1}{x}\bigg|_{x=y}^1
= \frac{1}{y} + 1 - \frac{1}{y} = 1 \ .
\end{align*}

For $y = 0$ or $y = 1$, we have that
$f_y(x) = f(x,y) = 0$ for $0 \leq x \leq 1$.
Hence $f_y$ is integrable and $\DS \int_0^1 f_y(x) \dx{x} = 0$.

\subQ{c} For $x \in ]0,1[$ fixed, we have
\[
f_x(y) = f(x,y) =
\begin{cases}
-x^{-2} & \quad \text{if}\ 0 < y < x < 1 \\
y^{-2} & \quad \text{if}\ 0 < x < y < 1 \\
0 & \quad \text{if}\ y = 0, x\ \text{or}\ 1
\end{cases}
\]
The function $f_x$ is integrable because it is a bounded function and
the set of points where $f_x$ is discontinuous is the set $\{0, x,1\}$
of zero content. We have
\begin{align*}
\int_0^1 f(x,y) \dx{y} &= \int_0^1 f_x(y) \dx{y}
= \int_0^x f_x(y) \dx{y} + \int_x^1 f_x(y) \dx{y} 
= -\int_0^x \frac{1}{x^2} \dx{y} + \int_x^1 \frac{1}{y^2} \dx{y} \\
&= -\frac{y}{x^2} \bigg|_{y=0}^x - \frac{1}{y}\bigg|_{y=x}^1
= -\frac{1}{x} - 1 + \frac{1}{x} = -1 \ .
\end{align*}

For $x = 0$ or $x = 1$, we have that
$f_x(y) = f(x,y) = 0$ for $0 \leq y \leq 1$.
Hence $f_x$ is integrable and $\DS \int_0^1 f_x(y) \dx{y} = 0$.

\subQ{d}
Let $\DS g(x) = \int_0^1 f_x(y) \dx{y}$ for $0 \leq x \leq 1$
and $\DS h(y) = \int_0^1 f_y(x) \dx{x}$ for $0 \leq y \leq 1$.
We have
\[
g(x) = \begin{cases}
-1 & \quad \text{if} \ 0 < x <1 \\
0 & \quad \text{if} \ x = 0, 1
\end{cases}
\qquad \text{and} \qquad 
h(y) =
\begin{cases}
1 & \quad \text{if} \ 0 < y <1 \\
0 & \quad \text{if} \ y = 0, 1  
\end{cases}
\]
Since they are two continuous functions on $]0,1[$ and
$\{0,1\}$ is a set of zero content, they are both Riemann integrable on
$[0,1]$.  In fact, 
$\DS
\int_0^1 \int_0^1 f(x,y) \dx{y}\dx{x} =\int_0^1 g(x) \dx{x} = -1$
and
$\DS
\int_0^1 \int_0^1 f(x,y) \dx{x}\dx{y} =\int_0^1 h(y) \dx{y} = 1$.
\end{sol}

\subsection{Supplementary Topics}

The following questions are traditional questions that can be found
in advanced calculus textbooks.  Nevertheless, we include them here
for the benefit of the readers who may not have had a solid advanced
calculus course.  They are problems that any mathematics students
should be able to solve.

\subsubsection{Multiple Integrals}

The most common double integrals are illustrated below.
The reader should have not problem to generalize this presentation to
multiple integrals in higher dimensions.  If the domain of integration
is $R = \left\{ (x,y) : a\leq x \leq b\ \text{and}\ g_1(x) \leq y \leq
g_2(x) \right\}$ (Figure~\ref{VOLUME3}), then it follows from Fubini's
theorem on a rectangle containing $R$ that
\[
\iint_R f = \int_a^b \int_{g_1(x)}^{g_2(x)} f(x,y) \dx{y} \dx{x} \ .
\]
Likewise, if the domain of integration is
$R = \left\{ (x,y) : c\leq y \leq d \ \text{and}\ h_1(y) \leq x \leq
h_2(y) \right\}$ (Figure~\ref{VOLUME3}), it follows from Fubini's
theorem on a rectangle containing $R$ that
\[
\iint_R f = \int_c^d \int_{h_1(y)}^{h_2(y)} f(x,y) \dx{x} \dx{y} \ .
\]
\pdfF{mult_integrals/volume3}{Double integrals}{We have drawn two of
the most frequent domains of integration for double integrals.}{VOLUME3}

In the questions below, we will often use Theorem~\ref{cov7} in
situations that do not perfectly meet the hypothesis of this theorem
for a change of variables.  If $f:U \to \RR$ is an integrable function
on an open set $U$ and the change of variables meet all the hypothesis
on $U$ except on a closed set $A$ of measure zero, then we may use 
Theorem~\ref{cov7} on $U\setminus A$.  It follows from
Remark~\ref{rmkC0meas} that the value of the integral
of $f$ on $U$ is equal to the value of the integral of $f$ on
$U \setminus A$.  This is particularly useful when
using polar or spherical coordinates because these changes of
variables are not properly defined at the origin.

\begin{question}
Consider the region $D$ enclosed by the curve $\DS y=x^3$,
and the lines $x=1$ and $y=0$.  Give two integrals to compute the mass
of this thin region if the density is given by
$\DS \rho(x,y) = \sqrt{x^4+1}$.  Compute the mass of this
region using one of the integrals that you have given.
\end{question}

\begin{sol}
We have to compute the mass of the region represented in the following
figure.
\pdfbox{mult_integrals/extra11}
The mass is given by the formula
$\DS m = \iint_D \rho = \int_0^1 \int_0^{x^3} \rho(x,y) \dx{y}\dx{x} =
\int_0^1 \int_{y^{1/3}}^1 \rho(x,y) \dx{x}\dx{y}$.
It is easier to compute the mass of the region $D$ with the first
double integral than the second one (the reader should try to
compute the second integral.)\ \  Hence, the mass is
\begin{align*}
m = \int_0^1 \int_0^{x^3} \sqrt{x^4+1} \dx{y}\dx{x} &=
\int_0^1 y\sqrt{x^4+1} \bigg|_{y=0}^{x^3} \dx{x} 
= \int_0^1 x^3 \sqrt{x^4+1} \dx{x} \\
&=\frac{1}{6}(x^4+1)^{3/2} \bigg|_0^1
= \frac{1}{6}(2\sqrt{2}-1)
\end{align*}
\end{sol}

\begin{question}
Evaluate $\DS I = \iint_S (x^2 - \sqrt{y})$, where $S$
is the region enclosed by the parabola $\DS x=y^2$ and the
line $x=2y$.
\end{question}

\begin{sol}
The ordinates ($y$ values) of the points of intersection of
$\DS x=y^2$ and $x=2y$ are given by
\[
y^2 = 2y \Rightarrow y(y-2) = 0 \Rightarrow y=0\ \text{or}\ y= 2 \ .
\]
We find the two intersection points $(0,0)$ and $(4,2)$.  We have
drawn the region below.
\pdfbox{mult_integrals/question1}
We have
$\DS I = \int_0^4 \int_{x/2}^{\sqrt{x}} (x^2 - \sqrt{y})\dx{y}\dx{x}
  = \int_0^2 \int_{y^2}^{2y} (x^2 - \sqrt{y}) \dx{x} \dx{y}$.
We evaluate the second integral.
\begin{align*}
I &= \int_0^2 \left( \frac{x^3}{3} - x \sqrt{y}\right)\bigg|_{y^2}^{2y} \dx{x}
= \int_0^2 \left( \frac{8 y^3}{3} - 2 y^{3/2} - \frac{y^6}{3}
+ y^{5/2}\right) \dx{y} \\
&= \left( \frac{2y^4}{3} - \frac{4y^{5/2}}{5} - \frac{y^7}{21}
+ \frac{2 y^{7/2}}{7} \right)\bigg|_0^2
= \frac{32}{7} - \frac{32 \sqrt{2}}{35}  \ .
\end{align*}
\end{sol}

\begin{question}
Express the following double integrals $\DS \iint_S f$
in terms of iterated integrals in two different ways.\\
\subQ{a} $S$ is the region enclosed by $\DS y=x^{1/3}$ and
$\DS y=x^2$ for $y\geq 0$.\\
\subQ{b} $S$ is the triangle with vertices at $(0,0)$, $(-1,1)$ and
$(2,1)$.\\
\subQ{c} $S$ is the region enclosed by the $x$ axis, the line $x=2$
and the curve $y=\ln(x)$.
\end{question}

\begin{sol}
\subQ{a} The abscissas ($x$ values) of the points of intersection of
$\DS y=x^2$ with $\DS y=x^{1/3}$ are given by
\[
x^2 = x^{1/3} \Rightarrow x(x^5- 1) = 0 \Rightarrow x=0\ \text{or}
\ x = 1 \ .
\]
The points of intersection are $(0,0)$ and $(1,1)$.
The domain of integration is shown in the following figure.
\pdfbox{mult_integrals/question6}
Hence
\[
\iint_S f = \int_0^1\int_{x^2}^{x^{1/3}} f(x,y) \dx{y} \dx{x}
= \int_0^1 \int_{y^3}^{\sqrt{y}} f(x,y) \dx{x}\dx{y} \ .
\]

\subQ{b} The domain of integration is shown in the following figure.
\pdfbox{mult_integrals/question7}
Hence
\[
\iint_S f = \int_0^1 \int_{-y}^{2y} f(x,y) \dx{x}\dx{y}
= \int_{-1}^0 \int_{-x}^1 f(x,y) \dx{y}\dx{x}
+ \int_0^2\int_{x/2}^1 f(x,y) \dx{y}\dx{x} \ .
\]

\subQ{c} The domain of integration is shown in the following figure.
\pdfbox{mult_integrals/question8}
Hence
\[
\iint_S f = \int_1^2\int_0^{\ln(x)} f(x,y) \dx{y}\dx{x}
= \int_0^{\ln(2)} \int_{e^y}^2 f(x,y) \dx{x}\dx{y} \ .
\]
\end{sol}

\begin{question}
Express the following double integrals $\DS \iint_S f$
in terms of iterated integrals in two different ways.\\
\subQ{a} $S$ is the region enclosed by $\DS y=x^3$ and
$y=4x$ for $y\leq 0$.\\
\subQ{b} $S$ is the triangle with vertices at $(0,0)$, $(2,2)$ and
$(3,1)$.\\
\subQ{c} $S$ is the region enclosed by the parabolas $\DS y=x^2$ and
$\DS y=6 - 4x -x^2$.
\end{question}

\begin{sol}
\subQ{a} The abscissas (x values) of the points of intersection of
$\DS y=x^3$ with $y=4x$ are given by
\[
x^3 = 4x \Rightarrow x(x^2-4) = 0 \Rightarrow x=0, x= -2 \ \text{or}
\ x = 2 \ .
\]
The points of intersection are $(-2,-8)$, $(0,0)$ and $(2,8)$.
The domain of integration is shown in the following figure.
\pdfbox{mult_integrals/question3}
Hence
\[
\iint_S f = \int_{-2}^0 \int_{4x}^{x^3} f(x,y) \dx{y}\dx{x}
= \int_{-8}^0 \int_{y^{1/3}}^{y/4} f(x,y) \dx{x}\dx{y} \ .
\]

\subQ{b} The domain of integration is shown in the following figure.
\pdfbox{mult_integrals/question4}
Hence
\begin{align*}
\iint_S f &= \int_0^2 \int_{x/3}^{x} f(x,y) \dx{y}\dx{x}
+ \int_2^3 \int_{x/3}^{4-x} f(x,y) \dx{y}\dx{x} \\
&= \int_0^1 \int_{y}^{3y} f(x,y) \dx{x}\dx{y} 
+ \int_1^2 \int_{y}^{4-y} f(x,y) \dx{x}\dx{y} \ .
\end{align*}

\subQ{c} The abscissas of the points of intersection of the two
parabolas $\DS y=x^2$ and
$\DS y=6 - 4x -x^2$ are given by
\[
x^2 = 6- 4x - x^2 \Rightarrow 2x^2 + 4x -6 = 0
\Rightarrow 2(x-1)(x+3)=0 \Rightarrow x= -3\ \text{or}
\ x = 1 \ .
\]
The points of intersection are $(-3,9)$ and $(1,1)$.
The domain of integration is shown in the following figure.
\pdfbox{mult_integrals/question5}
Hence
\begin{align*}
\iint_S f &= \int_{-3}^1 \int_{x^2}^{6-4x-x^2} f(x,y) \dx{y}\dx{x}
= \int_0^1 \int_{-\sqrt{y}}^{\sqrt{y}} f(x,y) \dx{x}\dx{y} \\
&\qquad + \int_1^9 \int_{-\sqrt{y}}^{-2+\sqrt{10-y}} f(x,y) \dx{x}\dx{y}
+ \int_9^{10} \int_{-2-\sqrt{10-y}}^{-2+\sqrt{10-y}} f(x,y) \dx{x}\dx{y} \ .
\end{align*}
To get the limits of integration, we have use the relation
\[
y = 6 - 4x -x^2 \Leftrightarrow  x^2 + 4x + (y-6) = 0
\Leftrightarrow x = \frac{-4 \pm \sqrt{4^2 - 4(y-6)}}{2}
= -2 \pm \sqrt{10 - y}
\]
for $y\leq 10$.
\end{sol}

\begin{question}
Evaluate the integral $\DS \iint_S f$ in each of the
following cases.\\
\subQ{a} $\DS f(x,y) = ye^{2x}$ and $S$ is the region
enclosed by $y=x$, $y=1$, $y=3$ and the $y$ axis.\\
\subQ{b} $\DS f(x,y) = \cos(y^3+1)$ and $S$ is the region enclosed by
$y=\sqrt{x}$, $y=1$ and the $y$ axis.\\
\subQ{c} $\DS f(x,y) = y e^{xy}$ and $S$ is the region enclosed by
$y=1/x$, $x=2$ and $y=1$.
\end{question}

\begin{sol}
\subQ{a} The domain of integration is shown in the following figure.
\pdfbox{mult_integrals/question9}
To avoid having to split the domain of integration in two, it
is preferable to integrate with respect to $x$ first than with respect
to $y$.  Thus
\begin{align*}
\iint_S y e^{2x} &= \int_1^3 \int_0^y y e^{2x} \dx{x}\dx{y}
= \int_1^3 \left( \frac{y e^{2x}}{2}\bigg|_{x=0}^y\right)\dx{y} \\
& = \frac{1}{2} \int_1^3 \left(y e^{2y} - y\right)\dx{y}
= \frac{1}{2} \left(\frac{y e^{2y}}{2} - \frac{e^{2y}}{4} -
\frac{y^2}{2} \right)\bigg|_1^3
= \frac{5e^6}{8} - \frac{e^2}{8} - 2 \ .
\end{align*}

\subQ{b} The domain of integration is shown in the following figure.
\pdfbox{mult_integrals/question10}
Hence
\[
\iint_S \cos(y^3+1) = \int_0^1 \int_{\sqrt{x}}^1 \cos(y^3+1) \dx{y}\dx{x}
= \int_0^1 \int_0^{y^2} \cos(y^3+1) \dx{x}\dx{y} \ .
\]
It is clear that the last integral is the simplest to compute (The
reader should try to compute the middle integral.)\ \  Thus
\begin{align*}
\iint_S \cos(y^3+1)
&= \int_0^1 \int_0^{y^2} \cos(y^3+1) \dx{x}\dx{y}
= \int_0^1 \left( x \cos(y^3+1)\right)\bigg|_{x=0}^{y^2}\dx{y} \\
&= \int_0^1 y^2 \cos(y^3+1) \dx{y}
= \frac{1}{3} \sin(y^3+1)\bigg|_{y=0}^1 = \frac{\sin(2) - \sin(1)}{3} \ .
\end{align*}

\subQ{c} The domain of integration is shown in the following figure.
\pdfbox{mult_integrals/question11}
Hence
\[
\iint_S y e^{xy}
= \int_1^2 \int_{1/x}^1 y e^{x y} \dx{y}\dx{x}
= \int_{1/2}^1 \int_{1/y}^2 y e^{x y} \dx{x}\dx{y} \ .
\]
The last integral is simpler to compute because only a simple
substitution is required.  The middle integral requires integration by
parts.  Thus
\begin{align*}
\iint_S y e^{xy}
&= \int_{1/2}^1 \int_{1/y}^2 y e^{x y} \dx{x}\dx{y}
= \int_{1/2}^1 \left( e^{x y}\bigg|_{x=1/y}^2\right)  \dx{y} \\
&= \int_{1/2}^1 \left( e^{2y} - e \right)  \dx{y}
= \left( \frac{e^{2y}}{2} - e y\right)\bigg|_{y=1/2}^1
= \frac{e^2}{2} - e \ .
\end{align*}
\end{sol}

\begin{question}
Let $f:[0,1]\to \RR$ be a continuous function.  Is the integral
$\DS \int_0^1 \int_x^{1-x} f(y) \dx{y} \dx{x}$
equal to the integral of $f$ over the region delimited by the limits
of integration?  Write the integral of $f$ over the 
region delimited by the boundary of integration as the sum of two
integrals of one variable.
\end{question}

\begin{sol}
The region delimited by the limits of integration is shown below.
\pdfbox{mult_integrals/question12}
The integral over this domain is
\[
  \iint_S f = \int_0^{1/2} \int_{x}^{1-x} f(y)\dx{y}\dx{x}
  + \int_{1/2}^1 \int_{1-x}^x f(y) \dx{y}\dx{x} \ .
\]
This is not equal to
$\DS \int_0^1 \int_x^{1-x} f(y) \dx{y}\dx{x}$ because
\begin{align*}
\int_0^1 \int_x^{1-x} f(y) \dx{y}\dx{x}
&= \int_0^{1/2} \int_{x}^{1-x} f(y)\dx{y}\dx{x}
  + \int_{1/2}^1 \int_x^{1-x} f(y) \dx{y}\dx{x} \\
&= \int_0^{1/2} \int_{x}^{1-x} f(y)\dx{y}\dx{x}
  - \int_{1/2}^1 \int_{1-x}^x f(y) \dx{y}\dx{x} \ .
\end{align*}
To express $\DS \iint_S f$ as the sum of two
integrals of one variable, we change the order of integration.
\begin{align*}
\iint_S f &= \int_0^{1/2} \int_0^y f(y)\dx{x} \dx{y}
+ \int_{1/2}^1 \int_0^{1-y} f(y)\dx{x} \dx{y} \\
&\qquad \qquad + \int_0^{1/2} \int_{1-y}^1 f(y)\dx{x} \dx{y}
+ \int_{1/2}^1 \int_y^1 f(y)\dx{x} \dx{y} \\
&= \int_0^{1/2} y f(y) \dx{y}
+ \int_{1/2}^1 (1-y) f(y) \dx{y}
+ \int_0^{1/2} y f(y) \dx{y}
+ \int_{1/2}^1 (1-y) f(y) \dx{y} \\
&= 2\int_0^{1/2} y f(y) \dx{y}
+ 2\int_{1/2}^1 (1-y) f(y) \dx{y} \ .
\end{align*}
The fact that we get $2$ times each integral should not be surprising
to the reader because $f$ is independent of $x$.  Thus
\[
 (b-a) f(y) = \int_a^b f(y) \dx{x} = \int_c^d f(y) \dx{x} = (d-c) f(y)
\]
as long as $b-a = d-c$.  In the present case, we have that
$a=0$, $b=y$, $c=1-y$ and $d=1$.
\end{sol}

\begin{question}
Reverse the order of integration of the integral
$\DS \int_1^3 \int_1^{\sqrt{x^2-1}} f(x,y)\dx{y}\dx{x}$.
Is this integral equal to the integral of $f$ over the region enclosed
by the limits of integration?  Justify your answer.
\end{question}

\begin{sol}
The region delimited by the limits of integration is shown below.
\pdfbox{mult_integrals/question13}
We have
\begin{align*}
\int_1^3 \int_1^{\sqrt{x^2-1}} f(x,y)\dx{y}\dx{x}
&= \int_0^{\sqrt{2}} \int_1^{\sqrt{x^2+1}} f(x,y) \dx{y}\dx{x} +
  \int_{\sqrt{2}}^3 \int_1^{\sqrt{x^2-1}} f(x,y) \dx{y}\dx{x} \\
&= - \int_0^{\sqrt{2}} \int_{\sqrt{x^2+1}}^1 f(x,y) \dx{y}\dx{x} +
  \int_{\sqrt{2}}^3 \int_1^{\sqrt{x^2-1}} f(x,y) \dx{y}\dx{x} \\
&= -\int_0^1 \int_1^{\sqrt{y^2+1}} f(x,y) \dx{x}\dx{y} +
\int_1^{2\sqrt{2}} \int_{\sqrt{y^2+1}}^3 f(x,y) \dx{x}\dx{y} \ .
\end{align*}
This integral is not equal to the integral of $f$ over the region $S$
enclosed by the limits of integration.  The integral of
$f$ over $S$ is given by
\begin{align*}
\iint_S f &=
\int_0^1 \int_1^{\sqrt{y^2+1}} f(x,y) \dx{x}\dx{y} +
\int_1^{2\sqrt{2}} \int_{\sqrt{y^2+1}}^3 f(x,y) \dx{x}\dx{y} \\
& = \int_1^{\sqrt{2}} \int_{\sqrt{x^2-1}}^1 f(x,y) \dx{y}\dx{x}
+ \int_{\sqrt{2}}^3 \int_1^{\sqrt{x^2-1}} f(x,y)\dx{y}\dx{x} \; .
\end{align*}
\end{sol}

\begin{question}
Reverse the order of integration of the integral
$\DS \int_{-6}^8 \int_{x^{1/3}}^{(x+6)/7} xy \dx{y}\dx{x}$.
\end{question}

\begin{question}
Reverse the order of integration in the integral
$\DS \int_0^a \int_0^{\sqrt{2ay-y^2}} f(x,y) \dx{x}\dx{y}$.
\end{question}

\begin{sol}
From $\DS x=\sqrt{2ay-y^2}$, we get that
$\DS x^2=2ay-y^2$.  By completing the square with respect to
the variable $y$, we get that $\DS x^2 - a^2 = -(y - a)^2$.
Thus, the boundary of the domain of integration $S$ is an arc of the
circle $\DS x^2+ (y - a)^2 = a^2$.  The following figure
shows the domain of integration.
\pdfbox{mult_integrals/midterm_1}
Solving $\DS x^2+ (y - a)^2 = a^2$ for $y$, we get that
$\DS y = a - \sqrt{a^2 - x^2}$ on the lower part of the circle.
Hence
\[
\int_0^a \int_0^{\sqrt{2ay-y^2}} f(x,y) \dx{x}\dx{y}
= \int_0^a \int_{a-\sqrt{a^2-x^2}}^a f(x,y) \dx{y}\dx{x} \ .
\]
\end{sol}

\begin{question}
Find the volume of the solid whose base is the triangle with vertices
$(0,0)$, $(1,0)$ and $(0,1)$, and which is bounded above by the
surface $z = 6xy(1-x-y)$.
\end{question}

\begin{sol}
The domain of integration $R$ is shown in the figure below.
\pdfbox{mult_integrals/question2}
Since $6xy(1-x-y) \geq 0$ for all $(x,y) \in R$, the volume is
\begin{align*}
V &= \int_0^1 \int_0^{1-x}  6xy(1-x-y) \dx{y}\dx{x}
= 6 \int_0^1 \int_0^{1-x}  (xy - x^2y -xy^2) \dx{y}\dx{x} \\
&= 6 \int_0^1 \left(\frac{xy^2}{2} - \frac{x^2y^2}{2} -
\frac{xy^3}{3}\right)\bigg|_{y=0}^{1-x} \dx{x}
= \int_0^1 \left( x - 3x^2 + 3x^3 - x^4\right) \dx{x} \\
&= \left( \frac{x^2}{2} - x^3 + \frac{3x^4}{4}
- \frac{x^5}{5}\right)\bigg|_{x=0}^1 = \frac{1}{20} \ .
\end{align*}
\end{sol}

\begin{question}
Find the value of the integral of $f(x,y)=2xy$ over the region $D$
enclosed by the lines $y=x-1$, $y=-x$, $y=0$ and $y=1$.
\end{question}

\begin{sol}
The region $D$ is shown in the figure below.
\pdfbox{mult_integrals/extra9}
It is advantageous to integrate with respect to $x$ first and then
with respect to $y$ to avoid having to split the domain of integration.
\begin{align*}
\iint_D f &= \int_0^1 \int_{-y}^{1+y} 2xy \dx{x}\dx{y}
= \int_0^1 x^2y\bigg|_{x=-y}^{1+y} \dx{y}
= \int_0^1 \left(y + 2 y^2\right) \dx{y}
= \left(\frac{y^2}{2} + \frac{2y^3}{3}\right)\bigg|_{y=0}^1
= \frac{7}{6} \ .
\end{align*}
\end{sol}

\begin{question}
Compute the integral of $\DS f(x,y)=x^2+y^2$ over the region $D$
enclosed by the parabola $\DS x=y^2-2y$ and the line $y=x$.
\end{question}

\begin{sol}
The region $D$ is shown in the figure below.
\pdfbox{mult_integrals/extra10}
To find the points of intersection of the parabola with the line
$y=x$, we first find the ordinates of these points of intersection.
\[
y = y^2-2y \Rightarrow y(y-3) =0 \Rightarrow  y=0\ \text{or}\ 3 \ .
\]
The points of intersection are $(0,0)$ and $(3,3)$.  We integrate with
respect to $x$ first and then with respect to $y$ to avoid having to
split the domain of integration.
\begin{align*}
\iint_D f &= \int_0^3 \int_{y^2-2y}^y (x^2 + y^2) \dx{x}\dx{y} 
= \int_0^3 \left(\frac{x^3}{3} + xy^2\right)\bigg|_{x=y^2-2y}^y\dx{y} \\
&= \int_0^3 \left(-\frac{y^6}{3} + 2y^5 - 5 y^4 + 6y^3\right) \dx{y}
= \left(-\frac{y^7}{21} + \frac{y^6}{3} - y^5
+ \frac{3y^4}{2}\right)\bigg|_{y=0}^3 = \frac{3^5}{14} \ .
\end{align*}
\end{sol}

\begin{question}
Evaluate $\DS I = \iint_S (x + 3y^3)$ where $S$ is
the upper half of the disk $\DS x^2 + y^2  \leq 4$.
\end{question}

\begin{sol}
Because of the shape of the domain of integration, it is advantageous
to use polar coordinates to compute this integral.
Let $x = r \cos(\theta)$ and $y= r\sin(\theta)$ for $0 \leq r \leq 2$
and $0 \leq \theta \leq \pi$.  Since
$\DS \left| \frac{\partial(x,y)}{\partial(r,\theta)} \right| = r$,
we get
\begin{align*}
I &= \int_0^2 \int_0^\pi \left(r\cos(\theta) + 3 r^3
\sin^3(\theta)\right) r \dx{\theta} \dx{r} \\
&= \int_0^2 \int_0^\pi r^2\cos(\theta) \dx{\theta} \dx{r}
+ 3 \int_0^2 \int_0^\pi r^4 \left(\sin(\theta)
- \cos^2(\theta)\sin(\theta) \right) \dx{\theta} \dx{r} \\
&= \int_0^2 r^2 \sin(\theta)\bigg|_{\theta=0}^\pi \dx{r}
+ 3 \int_0^2 r^4 \left( -\cos(\theta) + \frac{1}{3}
  \cos^3(\theta) \right)\bigg|_{\theta=0}^\pi \dx{r} \\
&= 0 + 3 \int_0^2 r^4 \left(\frac{4}{3} \right) \dx{r}
= 4 \, \frac{r^5}{5}\bigg|_0^2 = \frac{128}{5} \ ,
\end{align*}
where the famous trigonometric identity
$\DS \sin^2(\theta) = 1 - \cos^2(\theta)$
has been used to get the second equality above.
\end{sol}

\begin{question}
Find the area of the region $S$ enclosed by the cardioid
$r = 1 + \cos(\theta)$.
\end{question}

\begin{sol}
The region $S$ is shown in the following figure.
\pdfbox{mult_integrals/question18}
We need to compute $\DS A = \iint_S 1$.  Since $S$ is
given using polar coordinates, this suggests that polar coordinates should
be used to compute the integral.  Let $x = r \cos(\theta)$ and
$y = r\sin(\theta)$ for $0 \leq r \leq 1 + \cos(\theta)$ and
$0 \leq \theta \leq 2\pi$.  We have that
$\DS \left| \frac{\partial(x,y)}{\partial(r,\theta)} \right| = r$.
Thus
\begin{align*}
A &= \iint_S 1
= \int_0^{2\pi} \int_0^{1+\cos(\theta)} r \dx{r}\dx{\theta}
= \int_0^{2\pi} \left( \frac{r^2}{2}\bigg|_{r=0}^{1+\cos(\theta)} \right)
\dx{\theta} \\
&= \int_0^{2\pi} \left( \frac{1}{2} + \cos(\theta) +
  \frac{\cos^2(\theta)}{2} \right) \dx{\theta}
= \int_0^{2\pi} \left( \frac{3}{4} + \cos(\theta) +
  \frac{\cos(2\theta)}{4} \right) \dx{\theta} \\
&= \left( \frac{3\theta}{4} + \sin(\theta) +
  \frac{\sin(2\theta)}{8} \right)\bigg|_{\theta=0}^{2\pi} 
= \frac{3\pi}{2}  \ ,
\end{align*}
where the famous trigonometric identity $\cos^2(\theta) = (1+ \cos(2\theta))/2$
has been used to get the fifth equality.
\end{sol}

\begin{question}
Evaluate $\DS \iint_S e^{-(x^2+xy+y^2)}$ where
$\DS S = \{ (x,y) : x^2 + xy + y^2 \leq 1 \}$.
\end{question}

\begin{sol}
Before computing the integral, it may be useful to review a little bit
of linear algebra.  We have the quadratic form
\[
x^2 + xy + y^2 = \begin{pmatrix} x & y \end{pmatrix}
\begin{pmatrix} 1 & 1/2 \\ 1/2 & 1 \end{pmatrix}
\begin{pmatrix} x \\ y \end{pmatrix} \ .
\]
Let $\DS A = \begin{pmatrix} 1 & 1/2 \\ 1/2 & 1 \end{pmatrix}$.
Since $A$ is symmetric, it has only real eigenvalues and an orthogonal matrix
may be used to diagonalize $A$.  More precisely, suppose that
$\lambda_1$ and $\lambda_2$ are the eigenvalues of $A$.
If $\lambda_1 \neq \lambda_2$, and $\VEC{v}_1$ and $\VEC{v}_2$ are
eigenvectors of norm $1$ associated to $\lambda_1$ and $\lambda_2$
respectively, then $\VEC{v}_1$ and $\VEC{v}_2$ are orthonormal.
If $\lambda_1 = \lambda_2$, then we can always find two linearly
independent eigenvectors $\VEC{v}_1$ and $\VEC{v}_2$ associated to
$\lambda_1 = \lambda_2$.  We may select them (using Gram-Schmidt
orthogonalization) such that they are orthonormal.
Hence, with
$R = \begin{pmatrix} \VEC{v}_1 & \VEC{v}_2\end{pmatrix}$, we have
$\DS R^\top A R = \begin{pmatrix} \lambda_1 & 0 \\ 0 &
\lambda_2 \end{pmatrix}$.

It is easy to see that
$\DS \VEC{v}_1 =
\begin{pmatrix} 1/\sqrt{2} \\ 1/\sqrt{2} \end{pmatrix}$ and
$\DS \VEC{v}_2 =
\begin{pmatrix} -1/\sqrt{2} \\ 1/\sqrt{2} \end{pmatrix}$ are
eigenvectors of $A$ associated to the eigenvalues $\lambda_1=3/2$ and
$\lambda_2=1/2$ of $A$.  Thus, with
$\DS R = \begin{pmatrix} 1/\sqrt{2} & -1/\sqrt{2} \\
1/\sqrt{2} & 1/\sqrt{2} \end{pmatrix}$, we get
$\DS R^\top A R = \begin{pmatrix} 3/2 & 0 \\ 0 & 1/2 \end{pmatrix}$.
Note that $\DS R = \begin{pmatrix}
\cos(\theta) & -\sin(\theta) \\ \sin(\theta) &
\cos(\theta) \end{pmatrix}\bigg|_{\theta=\pi/4}$ is a counterclockwise
rotation by $\pi/4$ about the origin.

We may use this information to choose a change of variables 
$g:D \to S$ to compute the integral.  We choose
$\DS \begin{pmatrix} x \\ y \end{pmatrix} =
g(u,v) = R \begin{pmatrix} u \\ v \end{pmatrix}$.  Thus
$\DS \left| \det \diff g \right| = \left| \det
\begin{pmatrix} 1/\sqrt{2} & -1/\sqrt{2} \\
1/\sqrt{2} & 1/\sqrt{2} \end{pmatrix} \right| = 1$.
This was to be expected because rotations do not change the volume of
objects.

To determine the domain $D$ of $g$, we note that
\begin{align*}
&x^2 + xy + y^2 = \begin{pmatrix} x & y  \end{pmatrix} A
\begin{pmatrix} x \\ y \end{pmatrix}
= \begin{pmatrix} x & y \end{pmatrix} R
\begin{pmatrix} 3/2 & 0 \\ 0 & 1/2 \end{pmatrix} R^\top
\begin{pmatrix} x \\ y \end{pmatrix} \\
&\qquad = \left( R^\top \begin{pmatrix} x \\ y \end{pmatrix} \right)^\top
\begin{pmatrix} 3/2 & 0 \\ 0 & 1/2 \end{pmatrix}
\left( R^\top \begin{pmatrix} x \\ y \end{pmatrix} \right)
= \begin{pmatrix} u & v \end{pmatrix}
\begin{pmatrix} 3/2 & 0 \\ 0 & 1/2 \end{pmatrix}
\begin{pmatrix} u \\ v \end{pmatrix}
= \frac{3}{2} \ u^2 + \frac{1}{2} \ v^2 
\end{align*}
Hence $g$ maps
$\DS D = \left\{ (u,v) : \frac{3u^2}{2} + \frac{v^2}{2} \leq 1
\right\}$ onto $S$.  We get
\[
\iint_S e^{-(x^2+xy+y^2)} = \iint_D e^{-(3u^2/2) - (v^2/2)} \ .
\]
We now use a second change of variables; namely,
$\DS
\begin{pmatrix} u \\ v \end{pmatrix} = h(r,s)
= \begin{pmatrix} \sqrt{2/3}\, r \cos(s) \\ \sqrt{2}\, r\sin(s) \end{pmatrix}$
for $0\leq r \leq 1$ and $0 \leq \theta \leq 2\pi$.  Thus
$|\det \diff h| = 2r /\sqrt{3}$ and
\[
\iint_D e^{-(3u^2/2) - (v^2/2)}
= \frac{2}{\sqrt{3}} \int_0^{2\pi} \int_0^1 e^{-r^2} r \dx{r}\dx{\theta}
= -\frac{2\pi}{\sqrt{3}} e^{-r^2}\bigg|_{r=0}^1
= \frac{2\pi}{\sqrt{3}}\left(1-e^{-1}\right) \ .
\]
\end{sol}

\begin{question}
Let $D$ be the region bounded by $y=x/2$, $y=(x/2) + 2$, $y=3x$ and $y=3x-4$.
Use an appropriate change of variables to compute
$\DS \iint_D xy$.
\end{question}

\begin{sol}
\pdfbox{mult_integrals/supp1}
Let $u= 3x-y$ and $v= y - x/2$. We get that
$x = 2(u+v)/5$ and $y= (u+6v)/5$.  We consider the change of variables
$\DS \begin{pmatrix} x \\ y \end{pmatrix} = g(u,v)
= \begin{pmatrix} 2(u+v)/5 \\ (u+6v)/5 \end{pmatrix}$.  We have that
$\DS |\det \diff g(u,v)| = \left|\det
\begin{pmatrix} 2/5 & 2/5 \\ 1/5 & 6/5 \end{pmatrix}\right| = \frac{2}{5}$.
Moreover, $u = 3x-y$ varies from $0$ to $4$, and
$v= y - x/2$ varies from $0$ to $2$.  Thus
\begin{align*}
\iint_D xy &=
\int_0^4\int_0^2 \left(\frac{2(u+v)}{5}\right) \left( \frac{u+6v}{5} \right)
\frac{2}{5} \dx{v}\dx{u}
= \frac{4}{5^3} \int_0^4\int_0^2 \left( u^2 + 7uv + 6 v^2\right)
\dx{v}\dx{u} \\
&= \frac{4}{5^3} \int_0^4\left( u^2v + \frac{7uv^2}{2} + 2
v^3\right)\bigg|_{v=0}^2 \dx{u}
= \frac{8}{5^3} \int_0^4\left( u^2 + 7u + 8\right) \dx{u} \\
&= \frac{8}{5^3}\left( \frac{u^3}{3} + \frac{7u^2}{2} + 8u\right)\bigg|_{u=0}^4
= \frac{2624}{375} \ .
\end{align*}
\end{sol}

\begin{question}
Find the coordinates of the controid (namely,   \label{questCentroid1}
the coordinates of the centre of mass) of the parallelogram $R$ enclosed
by the lines $x-3y=0$, $2x +y=0$, $x-3y = 10$ and $2x + y = 15$.
The density $\rho$ of this thin plate is constant.
\end{question}

\begin{sol}
Recall that the coordinates $(\overline{x},\overline{y})$ of the
{\bfseries centroid}\index{Centroid}
of a thin plate $D$ of density $\rho(x,y)$ at $(x,y) \in D$
are found as it follows.  Let $\DS m = \iint_D \rho$ be the
mass of the thin plate.  Then
$\DS \overline{x} = \frac{1}{m} \iint_D x \rho$ and 
$\DS \overline{x} = \frac{1}{m} \iint_D y \rho$.

The parallelogram $R$ is shown in the figure below.
\pdfbox{mult_integrals/question26}
Let $u= x-3y$ and $v= 2x + y$.  We get that
$x = (u+3v)/7$ and $y= (v -2u)/7$.  This is the change of variables
that we use with $0 \leq u \leq 10$ and $0 \leq v \leq 15$.
We have that
$\DS \left| \frac{\partial(x,y)}{\partial(u,v)} \right|
= \left|\det
\begin{pmatrix} 1/7 & 3/7 \\ -2/7 & 1/7 \end{pmatrix}\right| = \frac{1}{7}$.
Thus, the mass is
\[
m = \iint_R \rho = \rho \int_0^{10} \int_0^{15} \frac{1}{7} \dx{v}\dx{u}
= \frac{150 \rho}{7} \ .
\]
The coordinates $(\overline{x}, \overline{y})$ of the centroid are
\[
\overline{x} = \frac{1}{m} \iint_R x \rho
= \frac{7}{150} \int_0^{10} \int_0^{15} \left( \frac{u+3v}{7}\right)
\frac{1}{7} \dx{v}\dx{u}
= \frac{55}{14}
\]
and
\[
\overline{y} = \frac{1}{m} \iint_R y \rho
= \frac{7}{150} \int_0^{10} \int_0^{15} \left( \frac{v-2u}{7}\right)
\frac{1}{7} \dx{v}\dx{u}
= -\frac{5}{14} \ .
\]
\end{sol}

\begin{question}
Compute the integral $\DS \iint_R (2x+y)^2(3y-x)^{-2}$
where $R$ is the parallelogram given by $-1 \leq 2x+y \leq 6$ and
$-3 \leq 3y -x \leq 4$.
\end{question}

\begin{sol}
The parallelogram $R$ is shown in the figure below.
\pdfbox{mult_integrals/question25}
Let $u= 2x+y$ and $v= 3y - x$.  We get that
$x = (3u-v)/7$ and $y= (u+2v)/7$.  This is the change of variables
that we use with $-1 \leq u \leq 6$ and $-3 \leq v \leq 4$.
We have that
$\DS \left|\frac{\partial(x,y)}{\partial(u,v)} \right|
= \left|\det
\begin{pmatrix} 3/7 & -1/7 \\ 1/7 & 2/7 \end{pmatrix}\right| = \frac{1}{7}$.
Thus
\begin{align*}
\iint_R (2x+y)^2(3y-x)^{-2}
&= \int_{-1}^6 \int_{-3}^4  \frac{u^2 v^{-2}}{7} \dx{v} \dx{u}
= \frac{1}{7} \int_{-1}^6 \left( -\frac{u^2}{v} \right)\bigg|_{v=-3}^4 \dx{u} \\
&= -\frac{1}{12} \int_{-1}^6 u^2 \dx{u}
= -\frac{1}{12} \left( \frac{u^3}{3} \right)\bigg|_{u=-1}^6
= -\frac{217}{36} \ .
\end{align*}
\end{sol}

\begin{question}
Let $R$ be the region enclosed by the curves $xy=1$ and $xy=9$, and the
lines $y=x$ and $y=4x$ for $x, y > 0$.  Find the coordinates of the
centroid of this region assuming that the density $\rho$ is constant.
For those who may have forgotten the formulae used to compute the
coordinates of a centroid or may not have seen them before, they are
given in Question~\ref{questCentroid1}.
\end{question}

\begin{sol}
The region $R$ is shown in the figure below.
\pdfbox{mult_integrals/question27}
Let $u= xy$ and $v=y/x$.  We get that
$x = \sqrt{u/v}$ and $y= \sqrt{uv}$.  This is the change of variables
that we use with $1 \leq u \leq 9$ and $1 \leq v \leq 4$.
We have that
$\DS
\left|\frac{\partial(x,y)}{\partial(u,v)} \right|
= \left|\det
\begin{pmatrix} \DS \frac{1}{2}u^{-1/2}v^{-1/2} &
\DS -\frac{1}{2} u^{1/2} v^{-3/2} \\[0.7em]
\DS \frac{1}{2} u^{-1/2}v^{1/2} &
\DS \frac{1}{2} u^{1/2} v^{-1/2}
\end{pmatrix}\right| = \frac{1}{2 v}$.
Thus, the mass is
\[
m = \iint_R \rho = \rho \int_1^9 \int_1^4 \frac{1}{2v} \dx{v}\dx{u} 
  = 9 \ln(2) \rho \ .
\]
The coordinates $(\overline{x}, \overline{y})$ of the centroid are
\[
\overline{x} = \frac{1}{m} \iint_R x \rho
= \frac{1}{9\ln(2)} \int_1^9 \int_1^4 \frac{1}{2} u^{1/2}v^{-1/2} \dx{v}\dx{u}
= \frac{52}{27\ln(2)}
\]
and
\[
\overline{y} = \frac{1}{m} \iint_R y \rho
= \frac{1}{9\ln(2)} \int_1^9 \int_1^4 \frac{1}{2} u^{1/2}v^{1/2} \dx{v}\dx{u}
= \frac{364}{81\ln(2)} \ .
\]
\end{sol}

\begin{question}
Let $R$ be the region enclosed by the hyperbolas $xy=1$, $xy=4$,
$\DS x^2 -y^2 = 1$ and $\DS x^2-y^2= 9$ for $x$,
$y > 0$.  Compute the mass of the thin plate represented by this
region if the density is given by $\DS \rho(x,y) = (x^2+y^2)^3$.
\end{question}

\begin{sol}
The region $R$ is represented in the figure below.
\pdfbox{mult_integrals/question28}

We want to compute the mass $\DS m = \iint_S (x^2+y^2)^3$.
For this question, we apply the change of variables in
the opposite direction.  Namely, we assume that
\[
\iint_R (x^2+y^2)^3 = \iint_R f(g(x,y))\,
\left|\det \diff g(x,y)\right| = \iint_S f(u,v)
\]
for some function $f:S \to \RR$ where
$\DS (x^2+y^2)^3 = f(g(x,y))\, \left|\det \diff g(x,y)\right|$
and $g:R \to S$ is our change of variables.  Let $u = xy$ and
$\DS v=x^2 - y^2$.  If we try to express $x$ and $y$ in
terms of $u$ and $v$, then we get ugly formulae.  Therefore, we proceed
slightly differently than usual.  Let
$\DS \begin{pmatrix} u \\ v \end{pmatrix} = g(x,y)
= \begin{pmatrix} xy \\ x^2 - y^2 \end{pmatrix}$
with $S = \{ (u,v) : 1 \leq u \leq 4 \text{ and } 1 \leq v \leq 9 \}$.
We have that
$\DS \left| \det \diff g(x,y) \right)
=\left| \frac{\partial(u,v)}{\partial(x,y)} \right|
= \left| \det \begin{pmatrix} y & x \\ 2x & -2y \end{pmatrix} \right|
= 2(x^2 + y^2)$ and
$\DS f(u,v) = \frac{v^2 + 4 u^2}{2} = \frac{(x^2+y^2)^22}{2}$.
Hence, the mass is
\begin{align*}
m &= \iint_R (x^2+y^2)^3
= \iint_R \underbrace{\frac{(x^2+y^2)^2}{2}\,
2(x^2+y^2)}_{=f(g(x,y))\, \left|\det \diff g(x,y)\right|}
= \iint_S \frac{v^2+4u^2}{2} \\
&= \frac{1}{2} \int_1^4 \int_1^9 (v^2 + 4u^2) \dx{v}\dx{u}
= 700 \ .
\end{align*}
\end{sol}

\begin{question}
Evaluate the integral $\DS \iint_S \frac{1}{x+y}$
where $S$ is the region bounded by the $x$ and $y$ axes, and the lines
$x+y=1$ and $x+y=4$.
\end{question}

\begin{sol}
The domain of integration $S$ is shown in the following figure.
\pdfbox{mult_integrals/extra5}

\stage{First method}
The obvious change of variables comes from $u= x+y$ and $v= y/x$.  We
then get that $x = u/(1+v)$ and $y= uv/(1+v)$.  This is the change of variables
that we use with $1 \leq u \leq 4$ and $0 < v < \infty$.  We have to
ignore $v=0$ to ensure that the change of variables is injective.

Unfortunately, this change of variables requires the computation of
an improper integral.  We have that
$\DS \left|\frac{\partial(x,y)}{\partial(u,v)} \right|
= \left|\det
\begin{pmatrix} 1/(1+v) & -u/(1+v)^2 \\
v/(1+v) & u/(1+v)^2 \end{pmatrix}\right| = \frac{u}{(1+v)^2}$.
Thus
\begin{align*}
\iint_S \frac{1}{x+y}
&= \lim_{R\to \infty} \int_1^4 \int_0^R
\frac{1}{u}\left(\frac{u}{(1+v)^2}\right) \dx{v} \dx{u}
= \lim_{R\to \infty} \int_1^4 \int_0^R \frac{1}{(1+v)^2} \dx{v} \dx{u} \\
& = \lim_{R\to \infty} \int_1^4 \left(\frac{-1}{1+v}\right)\bigg|_{v=0}^R \dx{u}
= \lim_{R\to \infty} \int_1^4 \left(1 - \frac{1}{1+R}\right) \dx{u} \\
&= \lim_{R\to \infty} 3\left(1 - \frac{1}{1+R}\right) = 3 \ .
\end{align*}

\stage{Second method}
If we do not want to introduce an improper integral, then we have to
slightly modify the previous change of variables.

We still set $u = x +y$ for $1 \leq u \leq 4$.
Instead of using $y = v x$ for $0 < v < \infty$ as we did before,
we use $\DS y = \left(\frac{v}{1-v}\right)x$ for $0 < 0 < 1$.
This substitution yields $y=0$ when $v=0$ and approaches $x=0$ when
$\DS v \to 1^-$.  If we solve for $v$, then we get
$\DS v = \frac{y}{x+y}$.  Thus, the change of variables that
we use is $x=u-uv$ and $y=uv$ for $1 \leq u \leq 4$ and $0 < v < 1$.
We have to ignore $v=0$ to ensure that the change of variables is injective.
As $u$ varies from $1$ to $4$, we have all the lines $x+y=u$ that
cover $S$.  When $v$ varies from $0$ to $1$ (excluding the end
points), we have that $\DS y = \left(\frac{v}{1-v}\right)x$
yields all the lines through the origin that cover $S$ except for the
segments
$\{(x,0) : 1 \leq x \leq 4\}$ and $\{(0,y) : 1 \leq y \leq 4\}$.

We have that $\DS \left|\frac{\partial(x,y)}{\partial(u,v)} \right|
= \left|\det \begin{pmatrix} 1-v & -u \\ v & u \end{pmatrix}\right| = u$.
Thus
\[
\iint_S \frac{1}{x+y} = \int_1^4 \int_0^1 \frac{1}{u}\ u \dx{v}\dx{u}
= \int_1^4 \int_0^1 \dx{v}\dx{u} = 3 \ .
\]
\end{sol}

\begin{question}
Compute the area of the region in the first quadrant enclosed by the
curves $r=\theta$ and $r=3$ for $0\leq \theta \leq \pi/2$, and the $x$
and $y$ axes.
\end{question}

\begin{sol}
The region is shown in the following figure.
\pdfbox{mult_integrals/extra6}
We use polar coordinates to compute the area of this region.
Let $x = r \cos(\theta)$ and $y=r\sin(\theta)$ with $0 \leq \theta < 2\pi$
and $\theta \leq r \leq 3$.  We have that
$\DS \left|\frac{\partial(x,y)}{\partial(r,\theta)}\right| = r$
and the area is
\begin{align*}
A &= \iint_D 1 = \int_0^{\pi/2} \int_{\theta}^3 r \dx{r}\dx{\theta}
= \int_0^{\pi/2} \left( \frac{r^2}{2}\bigg|_{r=\theta}^3 \right)\dx{\theta}\\
&= \int_0^{\pi/2} \left( \frac{9}{2} - \frac{\theta^2}{2}\right)\dx{\theta}
= \left( \frac{9\theta}{2} - \frac{\theta^3}{6}\right)\bigg|_{\theta=0}^{\pi/2}
= \frac{9\pi}{4} - \frac{\pi^3}{48} \ .
\end{align*}
\end{sol}

\begin{question}
Compute the volume of the right cylinder whose base is the region of
the $x,y$ plane bounded by the curve $\DS x^2+y^2 = x$, and
whose height is given by $z=4-y$.
\end{question}

\begin{sol}
One can rewrite $\DS x^2+y^2 = x$ as
$\DS \left(x - \frac{1}{2}\right)^2 + y^2 = \frac{1}{4}$.
This is a circle of radius $1/2$ centred at $(1/2, 0)$.  The bounded
region is shown in the following figure.
\pdfbox{mult_integrals/extra7}
We use polar coordinates $x= 1/2 + r\cos(\theta)$ and $y=r\sin(\theta)$
for $0 \leq \theta < 2\pi$ and $0< r < 1/2$.
We have that $\DS
\left|\frac{\partial(x,y)}{\partial(r,\theta)}\right| = r$
and the volume of the cylinder is
\begin{align*}
V &= \iint_D (4-y)
= \int_0^{2\pi} \int_0^{1/2} (4- r\sin(\theta)) r \dx{r}\dx{\theta}
= \int_0^{2\pi} \left( 2 r^2 - \frac{r^3}{3}\, \sin(\theta)
\right)\bigg|_{r=0}^{1/2} \dx{\theta}\\
& = \int_0^{2\pi} \left( \frac{1}{2} - \frac{1}{8}\sin(\theta)\right)\dx{\theta}
= \left( \frac{\theta}{2} + \frac{1}{8}\cos(\theta)
\right)\bigg|_{\theta=0}^{2\pi} = \pi \ .
\end{align*}
\end{sol}

\begin{question}
Compute the volume of the solid obtained if we cut out the ellipsoid
given by the $\DS \frac{x^2}{16}+\frac{y^2}{16}+z^2 \leq 1$ with a
cylinder of radius $3$ whose axis is the $z$ axis.
\end{question}

\begin{sol}
We use polar coordinates $x=r\cos(\theta)$ and $y=r\sin(\theta)$ for
$0\leq \theta < 2\pi$ and $0\leq r \leq 3$ to compute the volume of
the region.  The following figure shows half of the solid, the section
above the $x,y$ plane.  Since the solid is symmetric with respect to
the plane $z=0$, the volume of the solid is twice the volume of this
section.
\pdfbox{mult_integrals/extra8}
We have that
$\DS \left|\frac{\partial(x,y)}{\partial(r,\theta)}\right| = r$
and the volume is
\begin{align*}
V& = 2\iint_D \sqrt{1-\frac{x^2}{16} - \frac{y^2}{16}}
= 2 \int_0^{2\pi} \int_0^3 \sqrt{1-\frac{r^2}{16}}\ r \dx{r}\dx{\theta} \\
&= 2\int_0^{2\pi} \left(-\frac{16}{3}
\left(1-\frac{r^2}{16}\right)^{3/2}\right)\bigg|_{r=0}^3 \dx{\theta}
= 2\left(-\frac{16}{3} \left(\frac{7}{16}\right)^{3/2}
+\frac{16}{3}\right) \, \int_0^{2\pi} \dx{\theta} \\
&= \frac{\pi}{3} \left( 64 -7\sqrt{7}\right)  \ .
\end{align*}
\end{sol}

\begin{question}
Compute the volume of the ellipsoid $E$ given by
$\DS (2x+y +z)^2 + (2x-y)^2  + 4z^2 \leq 4$.
\end{question}

\begin{sol}
Let $u = 2x+y+z$, $v=2x-y$ and $w = 2z$.  We get that
$\DS x = \frac{u}{4} + \frac{v}{4} - \frac{w}{8}$,
$\DS y = \frac{u}{2} - \frac{v}{2} - \frac{w}{4}$ and
$\DS z = \frac{w}{2}$.  This is the change of variables that
we use.  We have that
$\DS \left|\frac{\partial(x,y,z)}{\partial(u,v,w)} \right|
= \left|\det
  \begin{pmatrix} 1/4 & 1/4 & -1/8 \\ 1/2 & -1/2 & -1/4 \\
  0 & 0 & 1/2 \end{pmatrix}\right| = \frac{1}{8}$.
Thus, the volume of $E$ is
\[
  V = \iiint_E 1
  = \frac{1}{8} \iiint_{u^2+v^2+w^2 \leq 4} \dx{u}\dx{v}\dx{w} \ .
\]
We could compute this integral using spherical coordinates but it is
easier to use the formula $\DS \frac{4r^3\pi}{3}$ to
compute the volume of a sphere of radius $r$; a proof of this formulae
is given in Question~\ref{questVolSphere} for the benefit of the
reader.  We get that $\DS V = \frac{4\pi}{3}$.
\end{sol}

\begin{question}
If $S$ is the solid enclosed by the paraboloid
$\DS z=x^2+y^2$ and the plane $z=1$, write the iterated
integrals of a function $f$ over the region $S$ in the orders
$\dx{z}\dx{y}\dx{x}$, $\dx{y}\dx{z}\dx{x}$ and $\dx{x}\dx{y}\dx{z}$.
\end{question}

\begin{sol}
The solid $S$ is represented in the following figure.
\pdfbox{mult_integrals/question14}  
We have
\begin{align*}
\iiint_S f &=
\int_{-1}^1 \int_{-\sqrt{1-x^2}}^{\sqrt{1-x^2}}
\int_{x^2+y^2}^1 f(x,y,z) \dx{z} \dx{y} \dx{x}
= \int_{-1}^1 \int_{x^2}^1 \int_{-\sqrt{z-x^2}}^{\sqrt{z-x^2}}
f(x,y,z) \dx{y} \dx{z} \dx{x} \\
& = \int_0^1 \int_{-\sqrt{z}}^{\sqrt{z}}
\int_{-\sqrt{z-y^2}}^{\sqrt{z-y^2}} f(x,y,z) \dx{x} \dx{y} \dx{z} \ .
\end{align*}
\end{sol}

\begin{question}
For each of the following integrals, sketch its domain of integration 
and express the iterated integral in the requested orders.\\
\subQ{a}
$\DS \int_0^1 \int_0^{1-y^2}\int_0^y f(x,y,z) \dx{z}\dx{x}\dx{y}$
in the orders $\dx{z}\dx{y}\dx{x}$ and $\dx{y}\dx{z}\dx{x}$.\\
\subQ{b}
$\DS \int_0^2 \int_0^{1-x/2} \int_0^{x/2} f(x,y,z)\dx{z}\dx{y}\dx{x}$
in the orders $\dx{y}\dx{x}\dx{z}$ and $\dx{x}\dx{y}\dx{z}$.\\
\subQ{c}
$\DS \int_0^1 \int_0^z \int_0^y f(x,y,z)\dx{x}\dx{y}\dx{z}$
in the orders $\dx{z}\dx{y}\dx{x}$ and $\dx{y}\dx{x}\dx{z}$.\\
\subQ{d}
$\DS \int_0^1 \int_0^y \int_0^{x/\sqrt{3}} f(x,y,z)\dx{z}\dx{x}\dx{y}$
in the orders $\dx{z}\dx{y}\dx{x}$ and $\dx{y}\dx{z}\dx{x}$.\\
\subQ{e}
$\DS \int_1^2 \int_1^z \int_{1/y}^2 f(x,y,z)\dx{x}\dx{y}\dx{z}$
in the orders $\dx{z}\dx{y}\dx{x}$ and $\dx{y}\dx{z}\dx{x}$.\\
\subQ{f}
$\DS
\int_{0}^{4} \int_{\sqrt{x}}^{2} \int_{0}^{2-y} f(x,y,z) \dx{z} \dx{y}\dx{x} $
in the orders
$\text{d}x\,\text{d}z\,\text{d}y$ and $\text{d}y\,\text{d}x\,\text{d}z$.\\
\subQ{g}
$\DS \int_0^2 \int_1^{2-x/2} \int_x^2 f(x,y,z)\dx{z}\dx{y}\dx{x}$
in the orders $\dx{y}\dx{x}\dx{z}$ and $\dx{x}\dx{y}\dx{z}$.
\end{question}

\begin{sol}
\subQ{a} The solid $S$ represented by the region enclosed by the
limits of integration is shown in the following figure.
\pdfbox{mult_integrals/question16}
$S$ is the solid in the quadrant $x,y,z \geq 0$ which is enclosed by
the planes $x=0$, $y=0$ and $z=y$, and the surface
$\DS x=1-y^2$.  We have
\[
\iiint_S f =
\int_0^1 \int_0^{\sqrt{1-x}} \int_0^y f(x,y,z) \dx{z} \dx{y} \dx{x}
= \int_0^1 \int_0^{\sqrt{1-x}} \int_z^{\sqrt{1-x}}
f(x,y,z) \dx{y} \dx{z} \dx{x} \ .
\]

\subQ{b}
The domain of integration is represented in the following figure.
\pdfbox{mult_integrals/supp5}
We have
\begin{align*}
\DS \int_0^2 \int_0^{1-x/2} \int_0^{x/2} f(x,y,z)\dx{z}\dx{y}\dx{x}
&= \DS \int_0^1 \int_{2z}^2 \int_0^{1-x/2} f(x,y,z)\dx{y}\dx{x}\dx{z}
\\
&= \DS \int_0^1 \int_0^{1-z} \int_{2z}^{2-2y}
f(x,y,z)\dx{x}\dx{y}\dx{z} \ .
\end{align*}

\subQ{f}
The domain of integration is represented in the following figure.
\pdfbox{mult_integrals/finalA}
We have
\begin{align*}
\int_{0}^{4} \int_{\sqrt{x}}^{2} \int_{0}^{2-y} f(x,y,z) \dx{z} \dx{y}\dx{x}
&= \int_{0}^{2} \int_0^{2-y} \int_{0}^{y^2} f(x,y,z) \dx{x} \dx{z}\dx{y} \\
&= \int_{0}^{2} \int_0^{(2-z)^2} \int_{\sqrt{x}}^{2-z}
f(x,y,z) \dx{y}\dx{x}\dx{z} \ .
\end{align*}

\subQ{g}
The domain of integration is represented in the figure below.
\pdfbox{mult_integrals/supp2}
We have
\[
\int_0^2 \int_1^{2-x/2} \int_x^2 f(x,y,z)\dx{z}\dx{y}\dx{x}
= \int_0^2 \int_0^z \int_1^{2-x/2} f(x,y,z) \dx{y}\dx{x}\dx{z} \ .
\]
The projection on the $y,z$ plane of the line of intersection of the planes
$z=x$ and $y=2 - x/2$ is the line given by $z = 4 -2y$.  Hence, we get
\begin{align*}
& \int_0^2 \int_1^{2-x/2} \int_x^2 f(x,y,z)\dx{z}\dx{y}\dx{x}
= \int_0^2 \int_1^{2-z/2} \int_0^z f(x,y,z) \dx{x}\dx{y}\dx{z} \\
&\qquad \qquad
+ \int_0^2 \int_{2-z/2}^2 \int_0^{4-2y} f(x,y,z) \dx{x}\dx{y}\dx{z} \ .
\end{align*}
\end{sol}

\begin{question}
A solid $V$ is defined by the region enclosed by the surfaces
$\DS z=x^2-3$,
$z=x-1$, $y=0$ and $y=2$.  If the density of the solid is given by
$\DS \rho(a,y,z) = x^2y$, compute the mass of the solid.
\end{question}

\begin{sol}
The abscissas of the points of intersection of the curve
$\DS z=x^2-3$ and the line $z=x-1$ are given by
\[
  x^2 - 3 = x - 1 \Leftrightarrow x^2 -x  -2 = (x-2)(x+1) = 0
  \Leftrightarrow x= -1\ \text{or}\ x=2 \ .
\]
Thus the points of intersection are $(x,z) = (-1,-2)$ and $(x,z) = (2,1)$.
The two surfaces in $\DS \RR^3$ given by
$\DS z=x^2-3$ and $z=x-1$
intersect each along the lines $\{ (-1,y,2) : y \in \RR\}$
and $\{ (2,y,1) : y \in \RR\}$.
The solid $V$ is represented in the following figure.
\pdfbox{mult_integrals/question17}
The mass $m$ of the solid is
\begin{align*}
m &= \int_0^2 \int_{-1}^2 \int_{x^2-3}^{x-1} x^2y \dx{z}\dx{x}\dx{y}
= \int_0^2 \int_{-1}^2  \left( x^2yz \right) \bigg|_{z=x^2-3}^{x-1}
\dx{x}\dx{y} \\
&= \int_0^2 \int_{-1}^2 y(-x^4 + x^3 + 2 x^2) \dx{x}\dx{y}
= \int_0^2  y\left( -\frac{x^5}{5} + \frac{x^4}{4} + \frac{2 x^3}{3}
\right)\bigg|_{x=-1}^{2} \dx{y} \\
&= \frac{63}{20} \int_0^2  y \dx{y}
= \frac{63}{20} \frac{y^2}{2} \bigg|_0^2
= \frac{63}{10} \ .
\end{align*}                                       
\end{sol}

\begin{question}
Compute the volume of the solid $D$ in the region $x,y,z \geq 0$ bounded
by the cylinder $\DS x^2+z^2=9$ and the planes $y = -x/2 +1$,
$x=0$, $y=0$ and $z=0$.
\end{question}

\begin{sol}
The solid $D$ is represented in the following figure.
\pdfbox{mult_integrals/extra12}
The volume $V$ is
\begin{align*}
V &= \iiint_D 1 = \int_0^2 \int_0^{-x/2+1}\int_0^{\sqrt{9-x^2}}\dx{z}\dx{y}\dx{x}
= \int_0^2 \int_0^{-x/2+1} z\bigg|_{y=0}^{\sqrt{9-x^2}} \dx{y}\dx{x}\\
&= \int_0^2 \int_0^{-x/2+1} \sqrt{9-x^2} \dx{y}\dx{x}
= \int_0^2  \left(y \sqrt{9-x^2}\right)\bigg|_0^{-x/2+1} \dx{x} \\
&= \int_0^2  \left(1-\frac{x}{2}\right) \sqrt{9-x^2} \dx{x}
= \left( \frac{x\sqrt{9-x^2}}{2} + \frac{9}{2}
\arcsin\left(\frac{x}{3}\right)  + \frac{(9-x^2)^{3/2}}{6}
\right)\bigg|_{x=0}^2 \\
&= \frac{11\sqrt{5}}{6} + \frac{9}{2}\arcsin\left(\frac{2}{3}\right)
- \frac{9}{2} \ .
\end{align*}
We have used the substitution $\DS u = 9-x^2$ to compute the
integral $\DS \int_0^2 \frac{x}{2}\sqrt{9-x^2} \dx{x}$, and the
trigonometric substitution $x=3\sin(\theta)$ to compute the integral
$\DS \int_0^2 \sqrt{9-x^2} \dx{x}$.
\end{sol}

\begin{question}
Let $R$ be the solid obtained from the intersection of the cylinders
$\DS x^2+y^2 = 4$ and $\DS x^2+z^2=4$.  Compute the
volume of $R$ and its mass if the density is given by
$\DS \rho(x,y,z) = x^2$.
\end{question}

\begin{sol}
The portion of the solid $R$ in the region $x,y,z \geq 0$ is
represented by the solid $D$ in the figure below.
\pdfbox{mult_integrals/extra13}
The solid $D$ is one eighth of the solid $R$.  The volume of $R$ is
\begin{align*}
V &= \iiint_R 1 = 8 \iiint_D 1
= 8 \int_0^2 \int_0^{\sqrt{4-x^2}} \int_0^{\sqrt{4-x^2}} \dx{z} \dx{y}\dx{x}\\
&= 8 \int_0^2 \int_0^{\sqrt{4-x^2}} z \bigg|_{z=0}^{\sqrt{4-x^2}} \dx{y}\dx{x}
= 8 \int_0^2 \int_0^{\sqrt{4-x^2}} \sqrt{4-x^2} \dx{y}\dx{x} \\
&= 8 \int_0^2 (y \sqrt{4-x^2})\bigg|_{y=0}^{\sqrt{4-x^2}} \dx{x}
= 8 \int_0^2 (4-x^2) \dx{x} = 8\left(4x - \frac{x^3}{3}\right)\bigg|_{x=0}^2
= \frac{128}{3} \ .
\end{align*}
Since the density is independent of $y$ and $z$, and even with respect
to $x$, the mass of $R$ is
\begin{align*}
m &= \iiint_R \rho = 8 \iiint_D x^2
= 8 \int_0^2 \int_0^{\sqrt{4-x^2}} \int_0^{\sqrt{4-x^2}}
x^2 \dx{z} \dx{y} \dx{x}\\
& = 8 \int_0^2 x^2(4-x^2) \dx{x}
= 8\left(\frac{4x^3}{3} - \frac{x^5}{5}\right)\bigg|_{x=0}^2
= \frac{512}{15} \ .
\end{align*}
\end{sol}

\begin{question}
If $D$ is the pyramid with vertices at $(0,0,0)$, $(1,1,1)$, $(0,1,0)$
and $(0,1,1)$, compute $\DS \iiint_D xz$.
\end{question}

\begin{sol}
The pyramid $D$ is represented in the following figure.
\pdfbox{mult_integrals/extra15}
There are several possibilities for the order of integration, each as
good as the other.  We choose $\dx{x}\dx{z}\dx{y}$.
Hence
\begin{align*}
\iiint_D xz &= \int_0^1 \int_0^y\int_0^{z} xz \dx{x}\dx{z}\dx{y}
= \int_0^1 \int_0^y \frac{x^2z}{2} \bigg|_{x=0}^z \dx{z}\dx{y} \\
&= \int_0^1 \int_0^y \frac{z^3}{2} \dx{z}\dx{y}
= \int_0^1 \frac{z^4}{8} \bigg|_{z=0}^y \dx{y}
= \int_0^1 \frac{y^4}{8} \dx{y}
= \frac{y^5}{40}\bigg|_{y=0}^1 = \frac{1}{40} \; .
\end{align*}
\end{sol}

\begin{question}
If $D$ is the pyramid with vertices at $(0,0,0)$, $(1,1,0)$, $(0,1,0)$
and $(0,1,1)$, compute $\DS \iiint_D xz$.
\end{question}

\begin{sol}
The pyramid $D$ is represented in the following figure.
\pdfbox{mult_integrals/extra16}
To find the equation of the plane through the points $(0.0.0)$,
$(1,1,0)$ and $(0,1,1)$, we consider the vectors $\VEC{v}_1 = (1,1,0)$ and
$\VEC{v}_2=(0,1,1)$ in the plane.  The vector
$\VEC{n} = \VEC{v}_1 \times \VEC{v}_2 = (1,-1,1)$
is perpendicular to the plane.  The equation of the plane is of the
form $\ps{\VEC{n}}{(x,y,z)} = x - y +z = C$ where $C$ is a constant.
Since the point $(0,0,0)$ is on that plane, 
we must have $C=0$.  Thus, the equation of the plane is $z = y - x$.
Hence
\begin{align*}
\iiint_D xz &= \int_0^1 \int_x^1\int_0^{y-x} xz
\dx{z}\dx{y}\dx{x}
= \int_0^1 \int_x^1 \frac{xz^2}{2} \bigg|_{z=0}^{y-x} \dx{y}\dx{x} \\
&= \int_0^1 \int_x^1 \frac{x (y-x)^2}{2} \dx{y}\dx{x}
= \frac{1}{6} \int_0^1 x (y-x)^3\bigg|_{y=x}^1 \dx{x}\\
&= \frac{1}{6} \int_0^1 \left( -x^4 + 3x^3 -3x^2 + x\right) \dx{x}
= \frac{1}{6} \left(-\frac{x^5}{5} +\frac{3x^4}{4}- x^3 +
  \frac{x^2}{2}\right) \bigg|_{x=0}^1 = \frac{1}{120} \; .
\end{align*}
\end{sol}

\begin{question}
If $D$ is the pyramid with vertices at $(1,0,0)$, $(0,1,0)$, $(0,0,1)$ and
$(0,0,0)$, compute\\ $\DS \iiint_D (xy+z)$.
\end{question}

\begin{question}
If $D$ is the pyramid with vertices at $(0,0,0)$, $(2,0,0)$, $(1,1,0)$ and
$(0,0,3)$, compute $\DS \iiint_D xy$.
\end{question}

\begin{sol}
We have drawn the pyramid $D$ in the following figure.
\pdfbox{mult_integrals/extra14}
To find the equation of the plane through the points $(2.0.0)$,
$(1,1,0)$ and $(0,0,3)$, we consider the vectors $\VEC{v}_1 = (2,0,-3)$ and
$\VEC{v}_2=(1,1,-3)$ that are parallel to the plane.  The vector
$\VEC{n} = \VEC{v}_1 \times \VEC{v}_2 = (3,3,2)$
is perpendicular to the plane.  The equation of the plane is of the
form $\ps{\VEC{n}}{(x,y,z)} = 3x + 3y +2z = C$
where $C$ is a constant.  Since the point $(2,0,0)$ is on that plane,
we must have $C=6$.  Thus, the equation of the plane is
$\DS z = 3\left(1 -\frac{x}{2} - \frac{y}{2}\right)$.
Hence
\begin{align*}
\iiint_D xy &= \int_0^1 \int_y^{2-y}\int_0^{3(1-x/2-y/2)}xy
\dx{z}\dx{x}\dx{y}
= \int_0^1 \int_y^{2-y} xyz \bigg|_{z=0}^{3(1-x/2-y/2)} \dx{x}\dx{y} \\
&= \int_0^1 \int_y^{2-y}
\left(3xy-\frac{3x^2y}{2}-\frac{3xy^2}{2}\right) \dx{x}\dx{y}
= \int_0^1 \left(\frac{3x^2y}{2} - \frac{x^3y}{2} - \frac{3x^2y^2}{4}\right)
\bigg|_{x=y}^{2-y} \dx{y}\\
&= \int_0^1 \left( y^4- 3 y^2 + 2y\right)\dx{y}
= \left( \frac{y^5}{5} - y^3 + y^2 \right)\bigg|_{y=0}^1 = \frac{1}{5}\ .
\end{align*}
\end{sol}

\begin{question}
Find the coordinates of the centroid of the half-cone given by
$\DS \sqrt{x^2+y^2} \leq z \leq 1$ and $x \geq 0$.  We
assume that the density $\rho$ is constant to $1$; namely,
$\rho(x,y,z) = 1$ for all $(x,y,z) \in S$.    
\end{question}

\begin{sol}
We have drawn the half-cone $S$ in the figure below.
\pdfbox{mult_integrals/question19}
We use the cylindrical coordinates
$x=r\cos(\theta)$, $y=r\sin(\theta)$ and $z=z$ for
$-\pi/2 \leq \theta \leq \pi/2$, $0 \leq r \leq 1$ and
$\DS r = \sqrt{x^2+y^2} \leq z \leq 1$.  We have that
$\DS
\left|\frac{\partial(x,y,z)}{\partial(r,\theta,z)}\right| = r $.
The mass is
\[
m = \iiint_S \rho
= \int_{-\pi/2}^{\pi/2} \int_0^1 \int_r^1 r \dx{z} \dx{r} \dx{\theta}
= \frac{\pi}{6} \ .
\]
The coordinates $(\overline{x}, \overline{y}, \overline{z})$ of the
centroid are
\[
\overline{x} = \frac{1}{m} \iiint_S x \rho
= \frac{6}{\pi}
\int_{-\pi/2}^{\pi/2} \int_0^1 \int_r^1 r^2\cos(\theta) \dx{z} \dx{r}
\dx{\theta}
= \frac{1}{\pi} \ ,
\]
\[
\overline{y} = \frac{1}{m} \iiint_S y \rho
= \frac{6}{\pi}
\int_{-\pi/2}^{\pi/2} \int_0^1 \int_r^1 r^2\sin(\theta) \dx{z} \dx{r}
\dx{\theta}
= 0
\]
and
\[
\overline{z} = \frac{1}{m} \iiint_S z \rho
= \frac{6}{\pi}
\int_{-\pi/2}^{\pi/2} \int_0^1 \int_r^1 z r \dx{z} \dx{r} \dx{\theta}
= \frac{3}{4} \ .
\]
That $\overline{y}=0$ should not be a surprise the reader because the
half-cone above is symmetric with respect to the plane $y=0$ and the
density is constant.
\end{sol}

\begin{question}
Find the coordinates of the centroid of the portion $B$ of the
ball $\DS x^2 + y^2 + z^2 \leq 4$ for $x$, $y$, $z \geq 0$.
The density $\rho$ of the ball is constant.
\end{question}

\begin{sol}
The portion $B$ of the ball is represented in the figure below.
\pdfbox{mult_integrals/question24}
We use the spherical coordinates
$x=r\cos(\theta)\sin(\phi)$, $y=r\sin(\theta)\sin(\phi)$ and
$z=r\cos(\phi)$ for $0 \leq \theta \leq \pi/2$, $0 \leq \phi \leq
\pi/2$ and $0 \leq r \leq 2$.  We have that
$\DS \left|\frac{\partial(x,y,z)}{\partial(r,\theta,\phi}\right|
= r^2\sin(\phi)$.  The mass is
\[
m = \iiint_B \rho
= \rho \int_0^{\pi/2} \int_0^{\pi/2} \int_0^2 r^2\sin(\phi) \dx{r}
\dx{\phi} \dx{\theta}
= \frac{4\pi \rho}{3} \ .
\]
The coordinates $(\overline{x}, \overline{y}, \overline{z})$ of the
centroid are:
\[
\overline{x} = \frac{1}{m} \iiint_B x \rho
= \frac{3}{4\pi}
\int_0^{\pi/2} \int_0^{\pi/2} \int_0^2 r^3\cos(\theta)\sin^2(\phi)
\dx{r} \dx{\phi} \dx{\theta}
= \frac{3}{4} \ ,
\]
\[
\overline{y} = \frac{1}{m} \iiint_B y \rho
= \frac{3}{4\pi}
\int_0^{\pi/2} \int_0^{\pi/2} \int_0^2 r^3\sin(\theta)\sin^2(\phi)
\dx{r} \dx{\phi} \dx{\theta}
= \frac{3}{4}
\]
and
\[
\overline{z} = \frac{1}{m} \iiint_B z \rho
= \frac{3}{4\pi}
\int_0^{\pi/2} \int_0^{\pi/2} \int_0^2 r^3 \cos(\phi)\sin(\phi) \dx{r}
\dx{\phi} \dx{\theta}
= \frac{3}{4} \ .
\]
It is not surprising that all three coordinates of the centroid are
equal since the density is constant and the solid $B$ is invariant
under permutations of the axes.
\end{sol}

\begin{question}
Compute the volume $V$ of the region $R$ enclosed   \label{questVolSphere}
by the sphere $\DS x^2 + y^2 + z^2 = 4$ and the cylinder
$\DS y^2 + z^2 =1$.
\end{question}

\begin{sol}
The easiest way to compute the volume $V$ is to subtract the volume
$V_1$ of the part of the cylinder inside the sphere from the volume
$V_2$ of the sphere.  The region $R$ is represented in the following
figure.
\pdfbox{mult_integrals/question20}

\stage{i} We compute he volume $V_1$ of the part of the cylinder
inside the sphere.  We use the cylindrical coordinates $x=x$,
$y = r\cos(\theta)$ and
$z=r\sin(\theta)$ for $\DS -\sqrt{4-r^2} \leq x \leq \sqrt{4-r^2}$,
$0 \leq r \leq 1$ and $0 \leq \theta \leq 2\pi$.  We have that
$\DS \left|\frac{\partial(x,y,z)}{\partial(x,r,\theta)}\right| = r$.
Hence
\begin{align*}
V_1 &= \int_0^{2\pi} \int_0^1 \int_{-\sqrt{4-r^2}}^{\sqrt{4-r^2}} r
\dx{x}\dx{r}\dx{\theta}
= \int_0^{2\pi} \int_0^1
\left( r x \right)\bigg|_{x=-\sqrt{4-r^2}}^{\sqrt{4-r^2}}\dx{r}\dx{\theta}
= \int_0^{2\pi} \int_0^1 2 r \sqrt{4-r^2} \dx{r}\dx{\theta} \\
&= \int_0^{2\pi} \left(-\frac{2}{3} (4-r^2)^{3/2}
\right)\bigg|_{r=0}^1 \dx{\theta}
= \frac{2(8-3\sqrt{3})}{3} \int_0^{2\pi} \dx{\theta}
= \frac{4 (8-3\sqrt{3})\pi}{3} \ .
\end{align*}
 
\stage{ii} Assuming that we have forgotten the formula to compute the
volume of a sphere and that we do not have access to tables of
formulae, we compute the volume $V_2$ of the sphere using integrals.
This also provides a proof of the formulae to compute the volume of
a sphere.  We use the spherical coordinates $x=r\cos(\theta)\sin(\phi)$,
$y = r\sin(\theta)\sin(\phi)$ and
$z=r\cos(\phi)$ for $0 \leq r \leq 2$, $0 \leq \theta \leq 2\pi$ and
$0 \leq \phi \leq \pi$.  We have that
$\DS \left|\frac{\partial(x,y,z)}{\partial(r,\theta,\phi)}\right|
= r^2 \sin(\phi)$.
Hence
\begin{align*}
V_2 &= \int_0^{2\pi} \int_0^{\pi} \int_0^2 r^2 \sin(\phi)
\dx{r}\dx{\phi}\dx{\theta}
= \int_0^{2\pi} \int_0^{\pi}
\left( \frac{r^3}{3} \sin(\phi) \right)\bigg|_{r=0}^2\dx{\phi}\dx{\theta} \\
&= \frac{8}{3} \int_0^{2\pi} \int_0^{\pi} \sin{\phi} \dx{\phi}\dx{\theta}
= \frac{8}{3} \int_0^{2\pi} \left(-\cos(\phi)
\right)\bigg|_{\phi=0}^{\pi} \dx{\theta} = \frac{16}{3} \int_0^{2\pi} \dx{\theta}
= \frac{32\pi}{3} \ .
\end{align*}

\stage{iii} Therefore, the volume of the region is
$\DS
V = \frac{32\pi}{3} - \frac{4 (8-3\sqrt{3})\pi}{3} = 4 \sqrt{3}\, \pi$.
\end{sol}

\begin{question}
Find the volume $V$ of the solid $R$ enclosed between the upper part of
the sphere $\DS x^2+y^2+z^2=25$ and the plane $z=4$.\\
\subQ{a} Setup but do not compute the integral in Cartesian coordinates.\\
\subQ{b} Setup but do not compute the integral in cylindrical coordinates.\\
\subQ{c} Setup but do not compute the integral in polar coordinates.\\
\subQ{d} Compute the easiest integral to get the volume.
\end{question}

\begin{sol}
The intersection of the sphere $\DS x^2 + y^2 + z^2 = 25$
with the plane $z=4$ is the circle $\DS x^2+y^2=9$.
We have drawn the solid $R$ in the figure below.
\pdfbox{mult_integrals/question22}

\subQ{a}
In Cartesian coordinates, the volume is
$\DS
V = \int_{-3}^3 \int_{-\sqrt{9-x^2}}^{\sqrt{9-x^2}} \int_4^{\sqrt{25-x^2-y^2}}
\dx{z}\dx{y}\dx{x}$.

\subQ{b}
We use the cylindrical coordinates $x=r\cos(\theta)$,
$y=r\sin(\theta)$ and $z=z$ for $\DS 4 \leq z \leq \sqrt{25 -r^2}$,
$0 \leq r \leq 3$ and $0 \leq \theta \leq 2\pi$.  We have that
$\DS \left|\frac{\partial(x,y,z)}{\partial(r,\theta,z)}\right| = r$.
The volume is
$\DS V = \int_0^{2\pi} \int_0^3  \int_{4}^{\sqrt{25-r^2}}
r \dx{z}\dx{r}\dx{\theta}$.

\subQ{c} 
We use the spherical coordinates $x=r\cos(\theta)\sin(\phi)$,
$y=r\sin(\theta)\sin(\phi)$ and $z=r\cos(\phi)$ for
$4/\cos(\phi) \leq r \leq 5$, $0 \leq \phi \leq \pi/3$ and
$0 \leq \theta \leq 2\pi$.
\pdfbox{mult_integrals/question22b}
We have that
$\DS \left|\frac{\partial(x,y,z)}{\partial(r,\theta,\phi)}\right|
= r^2 \sin(\phi)$.  The volume is\\
$\DS V = \int_0^{2\pi} \int_0^{\arccos(4/5)}
\int_{4/\cos(\phi)}^5 r^2 \sin(\phi) \dx{r}\dx{\phi}\dx{\theta}$.

\subQ{d}  The easiest integral to compute is in cylindrical coordinates
though the integral in spherical coordinates is not too bad either.
The integral in Cartesian coordinates requires a lot more work.
The integral in (b) gives
\begin{align*}
V &= \int_0^{2\pi} \int_0^3  \int_{4}^{\sqrt{25-r^2}} r \dx{z}\dx{r}\dx{\theta}
= \int_0^{2\pi} \int_0^3  (r z)\bigg|_{z=4}^{\sqrt{25-r^2}} \dx{r}\dx{\theta} \\
&= \int_0^{2\pi} \int_0^3 \left( r\sqrt{25-r^2} - 4 r \right)\dx{r}\dx{\theta}
= \int_0^{2\pi} \left( -\frac{1}{3} (25-r^2)^{3/2} - 2 r^2 \right)\bigg|_{r=0}^3
\dx{\theta} \\
&= \left( -\frac{64}{3} - 18 + \frac{125}{3} \right)
\int_0^{2\pi} \dx{\theta} = \frac{14\pi}{3} \ .
\end{align*}

The integral in (c) gives
\begin{align*}
V &= \int_0^{2\pi} \int_0^{\arccos(4/5)}
\int_{4/\cos(\phi)}^5 r^2 \sin(\phi) \dx{r}\dx{\phi}\dx{\theta} \\
&= \int_0^{2\pi} \int_0^{\arccos(4/5)}
\left( \frac{r^3}{3} \sin(\phi)\right)\bigg|_{r = 4/\cos(\phi)}^5
\dx{\phi}\dx{\theta} \\
&= \int_0^{2\pi} \int_{0}^{\arccos(4/5)}
\left( \frac{125}{3}\sin(\phi) - \frac{64 \sin(\phi)}{3\cos^3(\phi)} \right)
\dx{\phi}\dx{\theta} \\
&= \int_0^{2\pi}\left(-\frac{125}{3}\cos(\phi)
  -\frac{64}{6\cos^2(\phi)} \right)\bigg|_{\phi=0}^{\arccos(4/5)} \dx{\theta}
% &= 2\pi \left( -\frac{100}{3} -\frac{50}{3} + \frac{125}{3}
% + \frac{32}{3} \right)
= \frac{14\pi}{3} \ .
\end{align*}
\end{sol}

\begin{question}
Compute the volume $V$ of the bounded region $R$ enclosed part the
cylinder $\DS (x-1)^2+y^2 =1$, the cone
$\DS x^2 + y^2 - (z-2)^2 = 0$ and the $x,y$ plane.
\end{question}

\begin{sol}
We have drawn the region $R$ in the figure below.
\pdfbox{mult_integrals/question21}
Because of the cylinder, we are first inclined to use the cylindrical
coordinates $x = 1 + r\cos(\theta)$, $y=r\sin(\theta)$ and $z=z$ for
$\DS 0 \leq z \leq 2 - \sqrt{x^2+y^2}
= 2 - \sqrt{r^2 + 2r \cos(\theta) +  1}$,
$0 \leq r \leq 1$ and $0 \leq \theta \leq 2 \pi$.  Unfortunately, the
reader may verify that this leads to a difficult integral to compute.

We are still going to use cylindrical coordinates but in a non-usual
way.  We consider $x = r\cos(\theta)$, $y=r\sin(\theta)$ and $z=z$ for
$0 \leq z \leq 2 - r$, $0 \leq r \leq 2 \cos(\theta)$ and
$-\pi/2 \leq \theta \leq \pi/2$.
\pdfbox{mult_integrals/question21b}
To determine the range of $r$, we use
\begin{align*}
(x-1)^2 + y^2 = 1 & \Rightarrow
(r\cos(\theta) -1)^2 + r^2\sin^2(\theta) = 1
\Rightarrow r^2 -2 r \cos(\theta) = 0 \\
&\Rightarrow 
r = 0 \ \text{or} \ r = 2 \cos(\theta) \ .
\end{align*}
To determine the range of $z$, we use
\[
x^2 + y^2 - (z-2)^2 = 0 \Rightarrow
r^2 - (z-2)^2=0 \Rightarrow z = 2 - r
\]
for $z-2 ,r \geq 0$.  Since
$\DS \left|\frac{\partial(x,y,z)}{\partial(r,\theta.z)}\right| = r$,
the volume is
\begin{align*}
V &= \int_{-\pi/2}^{\pi/2} \int_0^{2\cos(\theta)} \int_0^{2 -r}
r \dx{z} \dx{r} \dx{\theta}
= \int_{-\pi/2}^{\pi/2} \int_0^{2\cos(\theta)}
(z r) \Big|_{z=0}^{2-r}\dx{r} \dx{\theta} \\
&= \int_{-\pi/2}^{\pi/2} \int_0^{2\cos(\theta)} (2r - r^2) \dx{r} \dx{\theta}
= \int_{-\pi/2}^{\pi/2} \left(r^2 - \frac{r^3}{3}
\right)\bigg|_0^{2\cos(\theta)}  \dx{\theta} \\
&= \int_{-\pi/2}^{\pi/2} \left(4\cos^2(\theta) - \frac{8 \cos^3(\theta)}{3}
\right)  \dx{\theta} \\
&= \int_{-\pi/2}^{\pi/2} \left(2(1+\cos(2\theta))
- \frac{8 \cos(\theta)}{3} +  \frac{8 \sin^2(\theta)\cos(\theta)}{3}
\right)  \dx{\theta} \\
&= \left(2\theta + \sin(2\theta)
- \frac{8 \sin(\theta)}{3} +  \frac{8 \sin^3(\theta)}{9}
\right)\bigg|_{-\pi/2}^{\pi/2} = 2\pi - \frac{32}{9} \ .
\end{align*}
\end{sol}

\begin{question}
Set up an integral in cylindrical coordinates to compute the volume of the
solid $S$ enclosed by the sphere $\DS x^2+y^2+z^2=12$ and the cone
$\DS 3z^2 = x^2 + y^2$ for $z\geq 0$.  Answer the same question in
spherical coordinates.
\end{question}

\begin{sol}
The solid $S$ is represented in the following figure.
\pdfbox{mult_integrals/midterm_2}
On the intersection of the sphere $\DS x^2+y^2+z^2=12$ and the cone
$\DS 3z^2 = x^2 + y^2$ where $z \geq 0$, we have
\[
3z^2 = x^2 + y^2 = 12 - z^2 \Rightarrow 
4z^2 = 12 \Rightarrow z= \pm \sqrt{3} \ .
\]
Thus, the intersection of the two surfaces is the circle
$\DS x^2+y^2 = 9$ with $z = \sqrt{3}$.

In cylindrical coordinates, we have that
$x = r\cos(\theta)$, $y = r\sin(\theta)$ and $z =z$
for $0 \leq \theta \leq 2\pi$, $0\leq r \leq 3$ and
$\DS r/\sqrt{3} \leq z \leq \sqrt{12-r^2}$.  Since
$|\det \diff g(r,\theta,z) | = r$, the volume of the
solid is
\[
V = \int_S 1 = \int_0^{2\pi} \int_0^3 \int_{r/\sqrt{3}}^{\sqrt{12-r^2}}
r \dx{z}\dx{r}\dx{\theta} \ .
\]

In spherical coordinates, we have that
$x = r\cos(\theta)\sin(\phi)$, $y= r\sin(\theta)\sin(\phi)$ and
$z=r\cos(\phi)$ for $0 \leq \theta \leq 2\pi$, $0 \leq \phi \leq \pi/3$ and
$0 \leq r \leq 2\sqrt{3}$.  Since
$\DS \left| \det \diff g \right| = r^2\sin(\phi)$, the
volume of the solid is
\[
V = \int_S 1 = \int_0^{2\pi} \int_0^{\pi/3} \int_0^{2\sqrt{3}}
r \sin^2(\phi) \dx{r}\dx{\phi}\dx{\theta} \ .
\]
\end{sol}

\begin{question}
Set up an integral in cylindrical coordinates to compute the volume of the
solid $S$ enclosed by the sphere $\DS x^2+y^2+z^2=5$ and the
paraboloid $\DS 4z = x^2 + y^2$.   Answer the same question
in spherical coordinates.
\end{question}

\begin{sol}
The solid $S$ is represented in the following figure.
\pdfbox{mult_integrals/extra2}
On the intersection of the sphere $\DS x^2+y^2+z^2=5$ and
the paraboloid $\DS 4z = x^2 + y^2$ where $z \geq 0$, we
have
\[
4z = x^2 + y^2 = 5 - z^2 \Rightarrow  z^5+ 4z- 5= (z+5)(z-1) = 0
\Rightarrow z= -5\ \text{or}\ z = 1 \ .
\]
Thus, the intersection of the two surfaces is the circle
$\DS x^2+y^2= 4$ with $z=1$.

In cylindrical coordinates, we have that
$x =r\cos(\theta)$, $y=r\sin(\theta)$ and $z=z$
for $0 \leq \theta \leq 2\pi$, $0 \leq r \leq 2$ and
$\DS r^2/4 \leq z \leq \sqrt{5-r^2}$.  Since
$\DS \left| \det \diff g \right| = r$, the volume of the
solid is
\[
V =
\int_0^{2\pi} \int_0^2 \int_{r^2/4}^{\sqrt{5-r^2}} r \dx{z}\dx{r}\dx{\theta} \ .
\]

A cross section of the region through the $z$ axis gives the following
figure.
\pdfbox{mult_integrals/extra2b}
In spherical coordinate, we have that 
$x = r\cos(\theta)\sin(\phi)$, $y= r\sin(\theta)\sin(\phi)$ and
$z = r\cos(\phi))$ for $0 \leq \theta \leq 2\pi$,
$0 \leq \phi \leq \tilde{\phi}$ and $0 \leq r \leq \sqrt{5}$, and
for $0 \leq \theta \leq 2\pi$, $\tilde{\phi} \leq \phi \leq \pi/2$ and
$\DS 0 \leq r \leq 4 \cos(\phi)/\sin^2(\phi)$.
We have that $\tilde{\phi} = \arctan(2)$.  To determine the range of
$r$ for $\tilde{\phi} \leq \phi \leq \pi/2$, we used
\begin{align*}
4 z = x^2 + y^2 &\Rightarrow
4 r \cos(\phi) = r^2 \cos^2(\theta) \sin^2(\phi) + r^2 \sin^2(\theta)
\sin^2(\phi) = r^2 \sin^2(\phi) \\
&\Rightarrow r = 0 \ \text{or}\ r = 4 \cos(\phi)/\sin^2(\phi) \ .
\end{align*}
Since $\DS \left| \det \diff g \right| = r^2\sin(\phi)$,
the volume of the solid is
\begin{align*}
V &= \int_S 1 =
\int_0^{2\pi} \int_0^{\arctan(2)} \int_0^{\sqrt{5}} r^2 \sin(\phi)
\dx{r}\dx{\phi}\dx{\theta} \\
&\hspace{5em}
+ \int_0^{2\pi} \int_{\arctan(2)}^{\pi/2} \int_0^{4 \cos(\phi)/\sin^2(\phi)}
r^2 \sin(\phi) \dx{r}\dx{\phi}\dx{\theta} \ .
\end{align*}
\end{sol}

\subsubsection{Improper integrals}

In this subsection, we provide another approach to expand the Riemann
integral to unbounded domain or to domain where the function is
unbounded.  It is important to note that this approach is not
equivalent to our general definition of integrable functions given in
Definition~\ref{cov2}.  Our general definition requires that the
functions be absolutely integrable on bounded open sets and that the sums
obtained from partitions of unity be absolutely convergent.  This is not
required for the approach in this subsection.  Do not forget that we
now work only with functions which are continuous except on a set o
measure zero.  Functions integrable according to our general
definition will be integrable according to the approach of this
subsection but the converse is not true.

The approach in this subsection is a generalization of the concept of
improper integrals for functions of one variable seen at the end of
the previous chapter.  The approach of this subsection introduces limits in
the computations of the integrals but does not require absolute
convergence of the limits.  We could not ignore this approach because
it is often used in engineering applications, often without proper
justifications.

The improper integral is formally defined as
$\DS \int_S f = \lim_{q\to \infty} \int_{S(q)} f$ where 
$S(q_1) \subsetneqq S(q_2) \subset S$ for $q_1 < q_2$ and $S(q) \to S$ as
$q \to \infty$, or $S(q_1) \subsetneqq S(q_2) \subset S$ for $q_2 < q_1$ and
$S(q) \to S$ as $\DS q \to 0^+$.  We will not provide any
rigorous definition of convergence for sets (though this could be
done) since we will not use improper integrals in higher dimensions in
these lecture notes.

As an example of the difference between the definition of improper
integrals given above and the general definition of integrals given in
Definition~\ref{cov2}, we consider the integral
$\DS \int_1^\infty \frac{\sin(x)}{x} \dx{x}$.  We have seen
in Question~\ref{questsinoxCD} that this improper integral converges
but $\DS \int_1^\infty \frac{|\sin(x)|}{x} \dx{x}$
diverges.  Thus $\DS \frac{\sin(x)}{x}$ is a converging
improper integral on $[1,\infty[$ but it is not integrable on
$[1,\infty[$ according to Definition~\ref{cov2}.

The next questions will illustrate the general idea about improper
integrals in higher dimensions.

\begin{question}
Determine if the following improper integrals converge or
diverge.  If they converge, then compute their value.
\begin{center}
\begin{tabular}{*{1}{l@{\hspace{0.5em}}l@{\hspace{3em}}}l@{\hspace{0.5em}}l}
\subQ{a} & $\DS \iiint_{\RR^3} \frac{1}{1+x^2+y^2+z^2}$  &
\subQ{b} & $\DS \iint_{x,y>0} \frac{1}{(1+x^2+y^2)^3}$ \\[0.8em]
\subQ{c} & $\DS \iiint_{x^2+y^2+z^2<1}
\frac{z^2}{(x^2+y^2+z^2)^{3/2}}$ &
\subQ{d} & $\DS \iint_{x>0} x e^{-x^2-y^2}$ \\[0.8em]
\subQ{e} & $\DS \iint_{x^2+y^2<1} \frac{x^2}{(x^2+y^2)^2}$ &  &
\end{tabular}
\end{center}
\end{question}

\begin{sol}
\subQ{a}
We use spherical coordinates $x= r\cos(\theta)\sin(\phi)$,
$y = r \sin(\theta)\sin(\phi)$ and $z = r \cos(\phi)$ for
$0 \leq \theta < 2\pi$, $0 \leq \phi < \pi$ and $0 \leq r < q$ where
$q\to \infty$.  We have that $\DS S(q) = B_q(\VEC{0}) \subset \RR^3$
in our formal definition of improper integrals.
Since
$\DS \left| \frac{\partial(x,y,z)}{\partial(r,\theta,\phi)} \right|
  = r^2 \sin(\phi)$,
we get
\begin{align*}
&\iiint_{\RR^3} \frac{1}{1+x^2+y^2+z^2}
= \lim_{q\to \infty} \int_0^q \int_0^\pi \int_0^{2\pi} 
\frac{r^2}{1+r^2}\sin(\phi) \dx{\theta}\dx{\phi}\dx{r} \\
&\qquad = \lim_{q\to \infty} 2\pi \int_0^q \int_0^\pi
\frac{r^2}{1+r^2}\sin(\phi) \dx{\phi}\dx{r}
= \lim_{q\to \infty} 2\pi \int_0^q
\frac{r^2}{1+r^2}\left(-\cos(\phi)\right)\bigg|_{\phi=0}^\pi\dx{r} \\
&\qquad = \lim_{q\to \infty} 4\pi \int_0^q
\left( 1 - \frac{1}{1+r^2}\right)\dx{r}
= \lim_{q\to \infty} 4\pi \left( r - \arctan(r)\right)\bigg|_{r=0}^q \\
&\qquad = \lim_{q\to \infty} 4\pi \left( q - \arctan(q)\right)
= +\infty \ .
\end{align*}
The integral diverges. 

\subQ{b} 
We use polar coordinates $x= r\cos(\theta)$ and $y = r \sin(\theta)$ for
$0 \leq \theta < \pi/2$ and $0 \leq r < q$ where $q\to \infty$.
We have that $\DS S(q) = B_q(\VEC{0}) \cap \{ (x,y): x,y \geq 0\} 
\subset \RR^2$ in our formal definition of improper integrals.
Since
$\DS \left| \frac{\partial(x,y)}{\partial(r,\theta)} \right| = r$,
we get
\begin{align*}
\iint_{x,y>0} \frac{1}{(1+x^2+y^2)^3}
&= \lim_{q\to \infty} \int_0^q  \int_0^{\pi/2} \frac{r}{(1+r^2)^3}
\dx{\theta} \dx{r}
= \lim_{q\to \infty} \frac{\pi}{2} \int_0^q \frac{r}{(1+r^2)^3} \dx{r} \\
& = \lim_{q\to \infty} -\frac{\pi}{8} \left(\frac{1}{(1+r^2)^2}
\right)\bigg|_{r=1}^q
= \lim_{q\to \infty} -\frac{\pi}{8}\left(\frac{1}{(1+q^2)^2} - 1 \right)
= \frac{\pi}{8} \ .
\end{align*}
The integral converges.

\subQ{c}
We use spherical coordinates $x= r\cos(\theta)\sin(\phi)$,
$y = r \sin(\theta)\sin(\phi)$ and $z = r \cos(\phi)$ for
$0 \leq \theta < 2\pi$, $0 \leq \phi < \pi$ and $q \leq r \leq 1$ where
$q\to 0$.  We have that $\DS S(q) = B_1(\VEC{0}) \setminus
B_q(\VEC{0}) \subset \RR^3$ in our formal definition of improper
integrals.  Since
$\DS \left| \frac{\partial(x,y,z)}{\partial(r,\theta,\phi)} \right|
= r^2 \sin(\phi)$, we get
\begin{align*}
&\iiint_{x^2+y^2+z^2<1} \frac{z^2}{(x^2+y^2+z^2)^{3/2}}
= \lim_{q\to 0^+} \int_q^1 \int_0^\pi \int_0^{2\pi} r\cos^2(\phi)\sin(\phi)
\dx{\theta}\dx{\phi}\dx{r} \\
&\quad = \lim_{q\to 0^+} 2\pi \int_q^1 \int_0^\pi r\cos^2(\phi)\sin(\phi)
\dx{\phi}\dx{r}
= \lim_{q\to 0^+} 2\pi \int_q^1 r
\left(\frac{-\cos^3(\phi)}{3}\right)\bigg|_{\phi=0}^\pi\dx{r} \\
&= \lim_{q\to 0^+} \frac{4\pi}{3} \int_q^1 r \dx{r}
= \lim_{q\to 0^+} \frac{4\pi}{3} \left(\frac{r^2}{2}\right)\bigg|_{r=q}^1
= \lim_{q\to 0^+} \frac{4\pi}{3}\left(\frac{1}{2} - \frac{q^2}{2}\right)
= \frac{2\pi}{3} \ .
\end{align*}
The integral converges.

\subQ{d}
We use polar coordinates $x= r\cos(\theta)$ and $y = r \sin(\theta)$ for
$-\pi/2 \leq \theta \leq \pi/2$ and $0 \leq r < q$ where $q\to \infty$.
We have that $\DS S(q) = B_q(\VEC{0}) \cap \{(x,y): x \geq 0\}
\subset \RR^2$ in our formal definition of improper integrals.
Since
$\DS \left| \frac{\partial(x,y)}{\partial(r,\theta)} \right| = r$,
we get
\begin{align*}
\iint_{x>0} x e^{-x^2-y^2}
&= \lim_{q\to\infty} \int_0^q \int_{-\pi/2}^{\pi/2}
r^2\cos(\theta) e^{-r^2} \dx{\theta}\dx{r}
=\lim_{q\to\infty} \int_0^q r^2 e^{-r^2}
\sin(\theta)\bigg|_{\theta=-\pi/2}^{\pi/2} \dx{r} \\
& = \lim_{q\to\infty} 2\int_0^q r^2 e^{-r^2} \dx{r} 
= \lim_{q\to\infty} 2 \left( -\frac{r e^{-r^2}}{2}\bigg|_0^q
+ \int_0^q \frac{e^{-r^2}}{2}\dx{r}\right) \\
&= \lim_{q\to\infty} 2 \left( -\frac{q e^{-q^2}}{2}
+ \int_0^q \frac{e^{-r^2}}{2}\dx{r}\right)
= \int_0^\infty e^{-r^2} \dx{r} = \sqrt{\pi}
\end{align*}
because $\DS \lim_{q\to\infty} q e^{-q^2}= 0$ as can be
shown with l'Hospital Rule.  We have use integration by parts to get
the fourth equality above.

\subQ{e}
We use polar coordinates $x= r\cos(\theta)$ and $y = r \sin(\theta)$ for
$0 \leq \theta < 2 \pi$ and $q \leq r \leq 1$ where $q\to 0$.
We have that $\DS S(q) = B_1(\VEC{0}) \setminus
B_q(\VEC{0}) \subset \RR^2$ in our formal definition of improper
integrals.  Since
$\DS \left| \frac{\partial(x,y)}{\partial(r,\theta)} \right| = r$,
we get
\begin{align*}
&\iint_{x^2+y^2<1} \frac{x^2}{(x^2+y^2)^2}
= \lim_{q\to 0^+} \int_q^1 \int_0^{2\pi} \frac{1}{r} \cos^2(\theta)
\dx{\theta}\dx{r}
= \lim_{q\to 0^+} \frac{1}{2} \int_q^1 \int_0^{2\pi} \frac{1}{r} (1 +
\cos(2\theta)) \dx{\theta}\dx{r} \\
&\qquad = \lim_{q\to 0^+} \frac{1}{2} \int_q^1  \frac{1}{r} \left(\theta +
\frac{\sin(2\theta)}{2}\right)\bigg|_0^{2\pi}\dx{r}
= \lim_{q\to 0^+} \pi \int_q^1  \frac{1}{r} \dx{r}
=\lim_{q\to 0^+} \pi \ln(r)\bigg|_q^1  \\
&\qquad = - \lim_{q\to 0^+} \pi \ln(q) = \infty \ .
\end{align*}
The integral diverges.
\end{sol}


%%% Local Variables: 
%%% mode: latex
%%% TeX-master: "notes"
%%% End: 
