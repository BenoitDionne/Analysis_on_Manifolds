\chapter{Manifolds}

Intuitively, if $\displaystyle V \subset \RR^3$ is a small enough open
neighbourhood of a point $\VEC{x}$ on a nice surface $S$ in
$\displaystyle \RR^3$, the set $V \cap S$ looks like a piece of
$\displaystyle \RR^2$.  How can we give a meaningful sense to this
last sentence?  The answer to this question comes from the
introduction of a local system of coordinates at each point of $S$.

\section{Preliminaries}\label{manifold_prep}

Before defining what is a local system of coordinates, we need to define what
is an open set in the half space
\[
H_k = \left\{ \VEC{u} \in \RR^k : u_k \geq 0 \right\}
\]
and what is a differentiable function $\displaystyle f:H_k \to \RR^n$.

\begin{defn}
Let $S$ be a subset of $\displaystyle \RR^n$.  An open set in $S$ is a
set of the form $U = S \cap V$ where $V$ is an open set in
$\displaystyle \RR^n$.  The collections of
all the open sets on $S$ defines a topology on $S$ called the
{\bfseries induced topology}\index{Induced Topology} on $S$ by
$\displaystyle \RR^n$.
\end{defn}

We equip $H_k$ with the induced topology from $\displaystyle \RR^k$.
It follows from the previous definition that a set $U \subset H_k$ is an
open subset of $H_k$ if there exists an open set
$\displaystyle V \subset \RR^k$ such that $U = H_k \cap V$.

\begin{defn}
The points $\VEC{x} \in H_k$ such that $x_k>0$ are the
{\bfseries interior points}\index{Interior Point} of
$H_k$ and those with $x_k=0$ are the
{\bfseries boundary points}\index{Boundary Point} of $H_k$.
The set of all interior points of $H_k$ is the
{\bfseries interior}\index{Interior of a Set} of $H_k$ and is denoted
$H_k^\circ$.  The set of all boundary points of $H_k$ is the
{\bfseries boundary}\index{Boundary of a Set} of $H_k$ and
is denoted $\partial H_k$.
\end{defn}

The interior $H_k^\circ$ is open in the induced topology because
$H_k^\circ = \left\{ \VEC{x}\in \RR^k : x_k>0\right\} \cap H_k$.
The definition of interior and boundary points of $H_k$ correspond to
the topological definition of interior and boundary points.  This
correspondence will be broken for manifolds as we will see shortly.

We need to extend our definition of differentiable functions to functions
that may not be defined on an open subset of $\displaystyle \RR^n$.

\begin{defn}
Suppose that $U$ is an open subset of $H_k$.  A function
$\displaystyle f:U \to \RR^m$ is
{\bfseries continuously differentiable}\index{Continuously Differentiable} at
$\VEC{x} \in U$ if there exist an open neighbourhood $V_{\VEC{x}}$ of
$\VEC{x}$ in $\displaystyle \RR^k$ and a continuously differentiable function
$\displaystyle F:V_{\VEC{x}} \to \RR^m$ such that $\displaystyle F = f$ on
$V_{\VEC{x}}\cap U$.  The function $F$ is an
{\bfseries extension}\index{Extension of a Function} of $f$ to an open
neighbourhood of $\VEC{x}$ in $\displaystyle \RR^k$.  We define
$\diff f(\VEC{x})$ as $\diff F(\VEC{x})$.
We say that $\displaystyle f:U \to \RR^m$ is 
{\bfseries continuously differentiable}\index{Continuously Differentiable}
on $U$ if $f$ is continuously differentiable at every $\VEC{x} \in U$.
\end{defn}

At the interior points of $H_k$, our previous definition of open
neighbourhood and differentiable functions correspond to the classical
definitions.

\begin{prop} \label{manifWEX}
If $U$ is an open subset of $H_k$ and $\displaystyle f:U \to \RR^m$ is
continuously differentiable at $\VEC{x} \in U$, then the definition of
$\diff f(\VEC{x})$ is independent of the extension chosen.
\end{prop}

\begin{proof}
For $i=1,2$, suppose that $\displaystyle V_{\VEC{x}}^{[i]}$ is an open
neighbourhood of $\VEC{x}$ in $\displaystyle \RR^k$ and that
$\displaystyle F^{[i]}:V_{\VEC{x}}^{[i]} \to \RR^m$ is a continuously
differentiable function such that $\displaystyle F^{[i]} = f$ on
$\displaystyle V_{\VEC{x}}^{[i]}\cap H_k$.  Let
$\displaystyle V = V_{\VEC{x}}^{[1]} \cap V_{\VEC{x}}^{[2]} \cap U \cap H_k^\circ$.

If $\displaystyle \VEC{x}\in H_k^\circ$, then $V$ is an open neighbourhood of
$\VEC{x}$ in $\displaystyle \RR^k$.
We have that $\displaystyle F^{[1]} = F^{[2]} = f$ in $V$.  Since the
derivative at a point is a local property, we get
$\displaystyle \diff F^{[1]}(\VEC{x}) = \diff F^{[2]}(\VEC{x}) = \diff
f(\VEC{x})$, where $\diff f$ denotes the standard derivative of $f$.

Suppose that $\VEC{x} \in U \cap \partial H_k$.  Choose
$\displaystyle \{ \VEC{x}_i \}_{i=1}^\infty \subset V$ such that
$\displaystyle \{ \VEC{x}_i \}_{i=1}^\infty$ converges to
$\VEC{x}$.  Since $\displaystyle F^{[1]} = F^{[2]} = f$ in $V$, we
have as in the previous paragraph that
$\displaystyle \diff F^{[1]}(\VEC{x}_i) = \diff F^{[2]}(\VEC{x}_i)$
for all $i$.  Thus $\displaystyle \diff F^{[1]}(\VEC{x}) = \lim_{i\to
  \infty} \diff F^{[1]}(\VEC{x}_i) =
\lim_{i\to \infty} \diff F^{[2]}(\VEC{x}_i) = \diff F^{[2]}(\VEC{x})$
by continuity of $\displaystyle \diff F^{[1]}$ and
$\displaystyle \diff F^{[2]}$ on $V$.
\end{proof}

\begin{prop} \label{manifDGFDGDF}
Let $U$, $V$ and $W$ be open subsets of $H_k$.  Suppose that $f: U \to V$ and
$g: V \to W$ are two differentiable functions, then
$\displaystyle \diff (g\circ f)(\VEC{x}) = \diff g(f(\VEC{x})) \diff f(\VEC{x})$
for all $\VEC{x} \in U$.
\end{prop}

\begin{proof}
Given $\VEC{x} \in U$, suppose that $U_1$ is an open neighbourhood of
$\VEC{x}$ in $\displaystyle \RR^k$ and that
$\displaystyle F:U_1 \to \RR^k$ is a continuously differentiable
function such that $\displaystyle F = f$ on $U_1 \cap U$.  Similarly,
suppose that $V_1$ is an open neighbourhood of $f(\VEC{x})$ in
$\displaystyle \RR^k$ and that $\displaystyle G:V_1 \to \RR^k$ is a
continuously differentiable function such that $\displaystyle G = g$ on
$V_1 \cap V$.

Since $F$ is continuous, $\displaystyle U_2 = F^{-1}(V_1) \cap U_1$ is an open
neighbourhood of $\VEC{x}$ in $\displaystyle \RR^k$.  Hence
$\displaystyle G\circ F: U_2 \to \RR^k$ is a
continuously differentiable function such that $G\circ F = g\circ f$ on
$U_2 \cap U$.

We have that $\displaystyle \diff (G\circ F)(\VEC{x}) = \diff G( F(\VEC{x}))
\diff F(\VEC{x})$.  Since
$\diff f(\VEC{x})$, $\diff g(f(\VEC{x}))$ and
$\diff (g\circ f)(\VEC{x})$ are defined by $\diff F(\VEC{x})$, $\diff
G(F(\VEC{x}))$ and $\diff (G\circ f)(\VEC{x})$ respectively, 
we get that
$\displaystyle \diff (g\circ f)(\VEC{x}) = \diff g(f(\VEC{x})) \diff
f(\VEC{x})$.
\end{proof}

The next definition is well known but it is worth repeating it here
since it is going to play a major role from now on.

\begin{defn}
Suppose that $V$ and $W$ are two open subsets of $H_k$.  The function
$h:V \to W$ is a {\bfseries diffeomorphism}\index{Diffeomorphism} if
$h$ is a homeomorphism, and $h$ and $\displaystyle h^{-1}$ are continuously
differentiable.
\end{defn}

\section{Local Charts, Atlases and Manifolds}

\begin{defn}\label{manifold_charts}
Let $S$ be a subset of $\displaystyle \RR^n$.  We assume that the
topology on $S$ is the induced topology from $\displaystyle \RR^n$.  A
{\bfseries local chart}\index{Local Chart} is defined by three
objects.
\begin{enumerate}
\item An open subset $U$ of $S$.
\item An open subset $W$ of $H_k$ for some integer $k$ such that
$0 \leq k\leq n$.
\item A homeomorphism $\phi:W\rightarrow U$.
\end{enumerate}
This local chart is denoted $(W, U, \phi)$
(Figures~\ref{MANIFOLD1} and \ref{MANIFOLD3}).
\end{defn}

\pdfF{manifolds/manifold1}{A local chart about an interior point of
a manifold}{$(W,U,\phi)$ is a local chart about an interior point
of the manifold $S$.}{MANIFOLD1}

\pdfF{manifolds/manifold3}{A local chart about a boundary point of a manifold}
{$(W, U,\phi)$ is a local chart about a boundary point of
the manifold $S$.}{MANIFOLD3}

\begin{rmk}
For $k=0$, we define $H_k$ and $\displaystyle \RR^k$ as $\{ 0 \} \subset \RR$.
Hence, a local chart is given by $W = \{0\}$ and the function
$\phi:W\rightarrow U$ maps $0$ to a single point $\phi(0) \in S$.  The
conditions for continuity and differentiability are meaningless in
this case and therefore they are ignored.  This special notation for $k=0$
will be useful later to simplify the statements of some results.
\end{rmk}

\begin{defn}\label{manifold_atlas}
An {\bfseries atlas}\index{Atlas} of class $\displaystyle C^j$,
denoted $\A(k,j)$, for a set $\displaystyle S \subset \RR^n$ is a
collection of local charts
$\displaystyle \left\{ (W_i, U_i,\phi_i)\right\}_{i\in I}$, where
$I$ is an index set, such that the following conditions are satisfied.
\begin{enumerate}
\item $\displaystyle S = \bigcup_{i\in I} U_i$.
\item $W_i \subset H_k$ for all $i \in I$. 
\item If $(W_{i_1}, U_{i_1},\phi_{i_1})$ and $(W_{i_2}, U_{i_2},\phi_{i_2})$
are two local charts such that $U_{i_1} \cap U_{i_2} \neq \emptyset$, then
\begin{equation} \label{manifold_comp}
\phi_{i_2}^{-1} \circ \phi_{i_1}: M_1 = \phi_{i_1}^{-1}(U_{i_1} \cap U_{i_2})
\rightarrow M_2 = \phi_{i_2}^{-1}(U_{i_1} \cap U_{i_2})
\end{equation}
is a $C^j$-diffeomorphism (Figure~\ref{MANIFOLD2}).
\end{enumerate}
\end{defn}

For $j>0$, the condition that
$\phi_{i_2}^{-1} \circ \phi_{i_1}: M_1 \rightarrow M_2$
be a $C^j$-diffeomorphism for all indices $i_1,i_2 \in I$ implies that
$\diff \phi_i(\VEC{w})$ is of rank $k$ for all $\VEC{w} \in W_i$ and
all $i \in I$.  We will come back on this subject in
Subsection~\ref{manifRRn}.

Note that the set $M_1$ and $M_2$ in the previous definition are open
subsets of $H_k$ because the functions $\phi_i$ are homeomorphisms.

\pdfF{manifolds/manifold2}{Mapping defined at the intersection of two local
charts}{The map $\displaystyle \phi_2^{-1}\circ\phi_1$ obtained
from two local charts provides a diffeomorphism from
$\displaystyle M_1 =\phi_{i_1}^{-1}(U_{i_1} \cap U_{i_2})$ to
$\displaystyle M_2 = \phi_{i_2}^{-1}(U_{i_1}\cap U_{i_2})$.}{MANIFOLD2}

In the previous definition, we could have considered only homeomorphisms
$\displaystyle \phi_{i_2}^{-1}\circ \phi_{i_1}$ instead of
$C^j$-diffeomorphism.  However, since our goal is integration on manifold,
we will need differentiable local charts anyway.  In most of our
examples, the local charts will
even be of class $\displaystyle C^j$ with $j>1$ or $j=\infty$.

\begin{defn} \label{defnDiffStruc}
Two atlases $\A_1(k,j)$ and $\A_2(k,j)$ for a set
$\displaystyle S \subset \RR^n$ are
{\bfseries equivalent}\index{Atlas!Equivalence} if
$\A_1(k,j) \cup \A_2(k,j)$ yields an atlas $\A_3(k,j)$ for $S$.  For
$k$ and $j$ fixed, a 
{\bfseries differential structure}\index{Atlas!Differential Structure}
$\SS(k,j)$ for a set $S$ is an equivalent class of atlases $\A(k,j)$ for $S$. 
The union of all the atlases of an equivalent class of atlases for $S$ forms
a {\bfseries maximal atlas}\index{Atlas!Maximal} for $S$.  An
{\bfseries admissible local chart}\index{Atlas!Admissible Local Chart}
is a local chart in the chosen maximal atlas.
\end{defn}

\begin{defn} \label{manifold_def}
A {\bfseries k-dimensional manifold}\index{Manifold!k-Dimensional Manifold} of
class $\displaystyle C^j$ is a subset $S$ of $\displaystyle \RR^n$ with a
differential structure $\SS(k,j)$ for $S$.
\end{defn}

From now on, when we talk about a local chart of the manifold $S$, we
mean an admissible local chart from the chosen maximal atlas for $S$.
Moreover, a local chart $(W, U,\phi)$ about a point $\VEC{u}$ of a
manifold $S$ is simply a local chart $(W, U,\phi)$ with $\VEC{u} \in U$.

To determine if a set $\displaystyle S \subset \RR^n$ is a manifold,
it suffices to find one atlas for $S$.

\begin{defn}
Let $S$ be a manifold.  The point $\VEC{u} \in S$ is an
{\bfseries interior point}\index{Interior Point} of $S$ if there
exists a local chart $(W,U,\phi)$ about $\VEC{u}$ such that
$\displaystyle W \subset H_k^\circ$.  The set of all interior points
of $S$ is the {\bfseries interior}\index{Interior} of $S$.
The interior of the manifold $S$ is denoted $\Int S$.
The point $\VEC{u} \in S$ is a
{\bfseries boundary point}\index{Boundary Point} of $S$ if
there exists a local chart $(W,U,\phi)$ about $\VEC{u}$
such that $\displaystyle \phi^{-1}(\VEC{u}) \in \partial H_k$.
The set of all boundary points of $S$ is the
{\bfseries boundary}\index{Boundary} of $S$.
The boundary of the manifold $S$ is denoted $\partial S$.
If $\partial S = \emptyset$, we say that we have a
{\bfseries manifold without boundary}\index{Manifold!Manifold Without Boundary}.
If $\partial S \neq \emptyset$, we say that we have a
{\bfseries manifold with boundary}\index{Manifold!Manifold With Boundary}.
\end{defn}

The interior and boundary for a manifold differ from the definition of
the interior and boundary given in topology.  The context will determine the
meaning given to interior and boundary.

\begin{rmk}
For the manifold without boundary, we may obviously replace   \label{rmkLCRn}
$\displaystyle H_k^\circ$ in the definition of a manifold by
$\displaystyle \RR^k$ without any consequence.  This is often very
convenient when defining local charts to simplify the definition of
the maps $\phi_i$.
\end{rmk}

\begin{egg}
Let $S$ be the circle of radius $2$ centred at the origin in
$\displaystyle \RR^2$.  We have
that $S$ is a $1$-dimensional manifold.  A possible atlas for $S$ is
given by the local charts:
\begin{align*}
&\left(]-2,2[, \{ (x_1,x_2) \in S : x_2>0\}, \phi_1(w)
= (w-2,\sqrt{4-w^2}\right) , \\
&\left(]-2,2[, \{ (x_1,x_2) \in S : x_2<0\}, \phi_2(w)
= (w-2,-\sqrt{4-w^2}\right) , \\
&\left(]-2,2[, \{ (x_1,x_2) \in S : x_1>0\}, \phi_3(w)
= (\sqrt{4-w^2}\,,w-2) \right)
\intertext{and}
&\left((]-2,2[, \{ (x_1,x_2) \in S : x_1<0\}, \phi_4(w)
= (-\sqrt{4-w^2}\,,w-2) \right) \ .
\end{align*}
We have that $\displaystyle ]-2,2[ \subset \RR$.

The open upper-hemisphere
$\displaystyle S_o= \left\{ \VEC{x} \in \RR^3 : \| \VEC{x} \| = 2 \ \text{and} \
x_3> 0 \right\}$
is a $2$-dimension manifold.  A possible atlas for $S_o$ is given by
the single local chart
\[
\left(\{(w_1,w_2):w_1^2+w_2^2<4\}, S_o, \phi(\VEC{w}) = \Big(w_1,w_2-2,
\sqrt{4-w_1^2-w_2^2}\,\Big) \right) \ .
\]
We have that $\displaystyle \{(w_1,w_2):w_1^2+w_2^2<4\} \subset \RR^2$.

Both $S$ and $S_o$ are manifolds of class $\displaystyle C^\infty$
without boundary.

The closed upper-hemisphere
$\displaystyle S_c= \left\{ \VEC{x} \in \RR^3 : \| \VEC{x} \| = 2 \
\text{and} \ x_3 \geq 0 \right\}$
is a $2$-dimension manifold with boundary.
Let $\displaystyle B = \{(w_1,w_2):w_1^2+(w_2-2)^2<4\} \subset H_2^\circ$
and $\displaystyle B_o = \{(w_1,w_2):w_1^2+w_2^2<4, w_2\geq 0\} \subset H_2$.
A possible atlas for
$S_c$ is given by the local charts:
\begin{align*}
&\left( B, \{ (x_1,x_2,x_3) \in S_c :  x_3>0\},
\phi_1(\VEC{w}) = (w_1,w_2-2,\sqrt{4-w_1^2-(w_2-2)^2})\right) \ ,\\
&\left( B_o, \{ (x_1,x_2,x_3) \in S_c : x_2>0\},
\phi_2(\VEC{w}) = (w_1,\sqrt{4-w_1^2-w_2^2}\,,w_2)\right) \ ,\\
&\left( B_o, \{ (x_1,x_2,x_3) \in S_c : x_2<0\},
\phi_3(\VEC{w}) = (w_1,-\sqrt{4-w_1^2-w_2^2}\,,w_2)\right) \ ,\\
&\left( B_o, \{ (x_1,x_2,x_3) \in S_c : x_1>0\},
\phi_4(\VEC{w}) = (\sqrt{4-w_1^2-w_2^2}\,,w_1,w_2)\right)
\intertext{and}
&\left( B_o, \{ (x_1,x_2,x_3) \in S_c : x_1<0\},
\phi_5(\VEC{w}) = (-\sqrt{4-w_1^2-w_2^2}\,,w_1,w_2)\right) \ ,
\end{align*}
where $\displaystyle B = \{(w_1,w_2):w_1^2+(w_2-2)^2<4\} \subset H_2^\circ$
and $\displaystyle B_o = \{(w_1,w_2):w_1^2+w_2^2<4, w_2\geq 0\} \subset H_2$.

Some readers may have thought that a possible atlas for $S_c$ could be given
by the single local chart
\[
\left( \{ (w_1,w_2): w_1^2+(w_2-2)^2 \leq 4\}, S_c, \phi(\VEC{w})
= \Big(w_1,w_2-2,\sqrt{4-w_1^2-(w_2-2)^2}\,\Big) \right) \ .
\]
However, this is not a local chart according to our definition.  The set
$\{ (w_1,w_2) : w_1^2+(w_2-2)^2 \leq 4\}$ is not an open subset of $H_2$.
Moreover, $\phi$ is not differentiable at $(w_1,w_2)$ with
$\displaystyle w_1^2 + (w_2-2)^2 = 4$; the map $\phi$ cannot be
extended to a differentiable function in an open neighbourhood of
these points.

The points on the circle
$\displaystyle S_1 = \left\{ \VEC{x} \in \RR^3 : \| \VEC{x} \| = 2 \
\text{and} \ x_3=0 \right\}$.
form the boundary of $S_c$.  The set $S_1$ is a $1$-dimensional
manifold without boundary.
\end{egg}

If $\VEC{u}$ is a boundary point of a manifold $S$ according to one local
chart $(W_1,U_1,\phi_1)$ with $\VEC{u} \in U_1$, is $\VEC{u}$ a boundary
point according to another equivalent local chart $(W_2,U_2,\phi_2)$ with
$\VEC{u} \in U_2$?  The following propositions will be used to answer
this question affirmatively.  Therefore, our definition of boundary for a
manifold is consistent.

\begin{prop} \label{propPHsbH}
Suppose that $W$ is an open subset of $H_k$ and $\phi:W \to H_k$ is a
differentiable map.  Moreover, suppose that
$\phi(\tilde{\VEC{w}}) \in \partial H_k$ for some
$\displaystyle \tilde{\VEC{w}} \in W \cap H_k^\circ$.  Then
$\diff \phi(\tilde{\VEC{w}}) (\RR^k) \subset \partial H_k$.
\end{prop}

\begin{proof}
Let $\displaystyle \pi_k:\RR^k \to \RR$ be the canonical projection on
the $\displaystyle k^{th}$ coordinate.  Choose
$\displaystyle \VEC{x} \in \RR^k$.  Since $\phi$ is differentiable at
$\displaystyle \tilde{\VEC{w}} \in W \cap H_k^\circ$, we
have that
$\displaystyle \phi(\tilde{\VEC{w}} + t\VEC{x}) = \phi(\tilde{\VEC{w}}) +
\diff \phi(\tilde{\VEC{w}}) (t\VEC{x}) + g(t\VEC{x})$
for $|t|$ small enough where $\displaystyle g:\RR^k \to \RR^k$ a
function such that
$\displaystyle \lim_{\VEC{y}\to \VEC{0}} \frac{\|g(\VEC{y})\|}{\|\VEC{y}\|}=0$.

Since $\phi(\tilde{\VEC{w}}) \in \partial H_0$ and
$\pi_k(\phi(\VEC{w})) \geq 0$
for all $\displaystyle \VEC{w}\in \RR^k$ by hypothesis, we have
\begin{equation} \label{manifold_prop1}
\begin{split}
0 &\leq \pi_k\left( \phi(\tilde{\VEC{w}} + t\VEC{x}) \right) = 
\pi_k \left(\phi(\tilde{\VEC{w}})\right) +
\left(\pi_k \circ \diff \phi(\tilde{\VEC{w}})\right) (t\VEC{x})
+ \left(\pi_k \circ g\right)(t\VEC{x}) \\
& = \left(\pi_k \circ \diff \phi(\tilde{\VEC{w}})\right) (t\VEC{x})
+ \left(\pi_k \circ g\right)(t\VEC{x}) \ .
\end{split}
\end{equation}
Hence, for $t>0$ small enough, we get from (\ref{manifold_prop1}) that
\[
0 \leq \left(\pi_k \circ \diff \phi(\tilde{\VEC{w}})\right) (\VEC{x})
+ \frac{1}{t} \left(\pi_k \circ g\right)(t\VEC{x}) \ .
\]
If we let $t$ converges to $0$, we get
$\displaystyle 0 \leq
\left(\pi_k \circ \diff \phi(\tilde{\VEC{w}})\right) (\VEC{x})$
because
\[
\Big| \frac{1}{t} \left(\pi_k \circ g\right)(t\VEC{x}) \Big|
\leq \|\VEC{x}\|\, \frac{\|g(t\VEC{x})\|}{\|t\VEC{x}\|}
\to 0
\]
as $t \to 0$.  Similarly, for $t<0$ closed to $0$, we get from
(\ref{manifold_prop1}) that
\[
0 \geq \left(\pi_k \circ \diff \phi(\tilde{\VEC{w}})\right) (\VEC{x})
+ \frac{1}{t} \left(\pi_k \circ g\right)(t\VEC{x}) \ .
\]
Again, if we let $t$ converges to $0$, we get
$\displaystyle 0 \geq
\left(\pi_k \circ \diff \phi(\tilde{\VEC{w}})\right) (\VEC{x})$.

Hence $\left(\pi_k \circ \diff \phi(\tilde{\VEC{w}})\right) (\VEC{x}) = 0$.
Since $\displaystyle \VEC{x}\in \RR^k$ is arbitrary, we get the conclusion
of the proposition.
\end{proof}

\begin{prop} \label{manifold_prop2}
Suppose that $W_1$ and $W_2$ are open subsets of $H_k$ and $f:W_1 \to W_2$ is
a diffeomorphism.  Then
$\displaystyle f(W_1 \cap H_k^\circ) \subset W_2 \cap H_k^\circ$ and
$f(W_1 \cap \partial H_k) \subset W_2 \cap \partial H_k$.  Moreover,
$\displaystyle f\big|_{W_1 \cap H_k^\circ}: W_1 \cap H_k^\circ 
\to W_2 \cap H_k^\circ$ and 
$\displaystyle f\big|_{W_1 \cap \partial H_k}:W_1 \cap \partial H_k
\to W_2 \cap \partial H_k$ are diffeomorphisms.
\end{prop}

\begin{proof}
\subQ{i} We consider the case with $\displaystyle W_1 \subset H_k^\circ$.   We
prove that $\displaystyle W_2 \subset H_k^\circ$.   Hence
$\displaystyle f\big|_{W_1 \cap H_k^\circ}:W_1 \cap H_k^\circ \to
W_2 \cap H_k^\circ$ will be a diffeomorphism because
$\displaystyle W_1 \cap H_k^\circ = W_1$ and
$\displaystyle W_2 \cap H_k^\circ = W_2$ are open subsets 
of $\displaystyle \RR^k$, and $f:W_1 \to W_2$ is then a diffeomorphism
in the traditional sense.

Suppose that $W_2 \cap \partial H_k \neq \emptyset$.  Since $f$ is
onto $W_2$, there exists $\tilde{\VEC{w}} \in W_1$ such that
$f(\tilde{\VEC{w}}) \in W_2 \cap \partial H_k$.
It follows from Proposition~\ref{manifDGFDGDF} that
\[
\diff f(\tilde{\VEC{w}}) \ \diff f^{-1}(f(\tilde{\VEC{w}})) =
\diff f(f^{-1}(f(\tilde{\VEC{w}}))) \ \diff f^{-1}(f(\tilde{\VEC{w}}))
= \diff ( f \circ f^{-1} )(f(\tilde{\VEC{w}})) = \Id
\]
and
\[
\diff f^{-1}(f(\tilde{\VEC{w}})) \ \diff f(\tilde{\VEC{w}}) =
\diff (f^{-1} \circ f)(\tilde{\VEC{w}}) = \Id \ .
\]
Thus $\diff f(\tilde{\VEC{w}})$ is invertible; in particular,
$\diff f(\tilde{\VEC{w}})$ maps $\displaystyle \RR^k$ onto
$\displaystyle \RR^k$.  This result contradicts the
conclusion of Proposition~\ref{propPHsbH} which says that
$\displaystyle \diff f(\tilde{\VEC{w}}) (\RR^k) \subset \partial H_k$.  Thus
$W_2 \cap \partial H_k = \emptyset$.

\subQ{ii} We consider the case with $W_1 \cap \partial H_k \neq \emptyset$.  We
must have that $W_2 \cap \partial H_k \neq \emptyset$ because (i) applied to
$\displaystyle f^{-1}$ instead of $f$ would then yield
$W_1 \cap \partial H_k = \emptyset$.

Given $\displaystyle \VEC{w} \in W_1 \cap H_k^\circ$, there exists an
open neighbourhood $\displaystyle V_1 \subset W_1 \cap H_k^\circ$ of
$\VEC{w}$.  If we apply (i) with $W_1$ replaced by $V_1$, then we get that
$\displaystyle f(V_1) \subset H_k^\circ$.  Thus $f(V_1)$ is an open
neighbourhood of $f(\VEC{w})$ in $\displaystyle W_2 \cap H_k^\circ$.
Since $\VEC{w}$ is arbitrary, this shows that
$\displaystyle f(W_1\cap H_k^\circ) \subset W_2 \cap H_k^\circ$.
The same reasoning applied to $\displaystyle f^{-1}$ shows that
$\displaystyle f^{-1}(W_2 \cap H_k^\circ) \subset W_1 \cap H_k^\circ$.
Thus $\displaystyle f(W_1 \cap H_k^\circ) = W_2 \cap H_k^\circ$ and
$\displaystyle f\big|_{W_1\cap H_k^\circ}:W_1 \cap H_k^\circ\to
W_2 \cap H_k^\circ$ is a diffeomorphism.

Since $f:W_1 \to W_2$ is one to one and onto, we then have that
$f(W_1 \cap \partial H_k) = W_2 \cap \partial H_k$ and
$\displaystyle f\big|_{W_1 \cap \partial H_k}:W_1 \cap \partial H_k
\to W_2 \cap \partial H_k$ is a diffeomorphism.
Note that $W_i \cap \partial H_k$ for $1\leq i \leq 2$ are open subsets
of $\displaystyle \partial H_k \cong \RR^{k-1}$.  So
$\displaystyle f\big|_{W_1 \cap \partial H_k}:W_1 \cap \partial H_k
\to W_2 \cap \partial H_k$ is a standard diffeomorphism between open
sets of $\displaystyle \RR^{k-1}$.
\end{proof}

\begin{theorem} \label{manifBSM}
Suppose that $S$ is a $k$-dimensional manifold.  Then $\Int S$ is a 
$k$-dimensional manifold without boundary and $\partial S$ is
a $(k-1)$-dimensional manifold without boundary.
\end{theorem}

\begin{proof}
To proof this theorem, it suffices to apply the results of
Proposition~\ref{manifold_prop2} to the compositions
$\displaystyle \phi_i^{-1} \circ \phi_j$, where $(W_i, U_i, \phi_i)$ and
$(W_j, U_j, \phi_j)$ are two local charts of $S$ such that 
$U_i \cap U_j \neq \emptyset$.

If $\displaystyle \{ (W_i,U_i,\phi_i) \}_{i\in I}$ is a maximal atlas
for $S$ then
\[
\left\{ \left(W_i\cap H_k^\circ, U_i \cap \Int S,
\phi_i\big|_{W_i\cap H_k^\circ}) \right) : i\in I \right\}
\]
is an atlas for $\Int S$.  An atlas for $\partial S$ is given by
\[
\left\{ \left(W_i\cap \partial H_k, U_i \cap \partial S,
\phi_i\big|_{W_i \cap \partial H_k}\right) :
i \in I \ \text{and} \ U_i \cap \partial H_k \neq \emptyset \right\} \ .
\]
Recall that
$\displaystyle W_i \cap \partial H_k$ is an open subset of
$\displaystyle \{ \VEC{x} \in \RR^k : x_k = 0 \} \cong \RR^{k-1}$ for
all $i$.  So $\partial S$ is a manifold without boundary of dimension $k-1$.
\end{proof}

The boundary of a manifold $S$ is in fact a ``submanifold'' of $S$.

\begin{defn}
Let $S$ be a $k$-dimensional manifold.  A
{\bfseries submanifold}\index{Submanifold} of $S$ is a subset $M$ of
$S$ such that there exist a positive integer $q < k$, indices
$1 \leq k_1 <k_2 < \ldots < k_q \leq k$, and an atlas
$\displaystyle \left\{ (W_i, U_i,\phi_i)\right\}_{i\in I}$ for $S$ such
that $\displaystyle U_i \cap M = \phi_i(W \cap H_k^{[k_1,k_2,\ldots,k_q]})$ 
for all $i \in I$, where
$\displaystyle H_k^{[k_1,k_2,\ldots,k_q]}
= \{ \VEC{w} \in H_k : w_{k_i} =0 \ \text{for} \ 1 \leq i \leq q \}$.
\end{defn}

If we set $\displaystyle \tilde{W}_i = W \cap H_k^{[k_1,k_2,\ldots,k_q]}$ and
$\tilde{U}_i = U_i \cap M$ for all $i \in I$, then
$\displaystyle \left\{ (\tilde{W}_i, \tilde{U}_i,\phi_i)\right\}_{i\in I}$
is an atlas for $M$.  Hence $M$ is itself a $q$-dimensional manifold.

\subsection{Manifolds in $\displaystyle \mathbf{\RR^n}$}
\label{manifRRn}

The next theorem gives another method to describe manifolds in
$\displaystyle \RR^n$.

\begin{theorem}\label{manifSPIVAK}
Let $S$ be a subset of $\displaystyle \RR^n$ and $\VEC{u} \in S$.  The
following two statements are equivalent.
\begin{enumerate}
\item There exist an open neighbourhood
$\displaystyle V_1 \subset \RR^n$ of $\VEC{u}$, an open set
$\displaystyle V_2 \subset \RR^n$, and a diffeomorphism $h:V_1\to V_2$
such that
$h(V_1 \cap S) = \{ \VEC{v} \in V_2 : v_{k+1} = v_{k+2} = \ldots = v_n = 0\}$
(Figure~\ref{MANIFOLD4}).
\item There exist an open subset $\displaystyle W \subset \RR^k$, an
open neighbourhood $\displaystyle V \subset \RR^n$ of $\VEC{u}$ and a
function $\phi:W \to V \cap S$ such that $\phi$ is a homeomorphism,
$\phi$ is of class $\displaystyle C^1$ and $\diff \phi(\VEC{w})$ is of
rank $k$ for all $\VEC{w} \in W$.
\end{enumerate}
Moreover, (1) implies that $\displaystyle \phi^{-1}:V\cap S\to W$ in
(2) can be extended to a continuously differentiable function in the
neighbourhood $V_1$ of $\VEC{u}$.
\end{theorem}

\pdfF{manifolds/manifold4}{Local representation of a manifold}
{Local representation of a manifold $S$.}{MANIFOLD4}

\begin{proof}
\subQ{i} Suppose that (1) is true.  By definition of induced topology
$U = V_1 \cap S$ is an open subset of $S$.  Let
\[
W = \left\{ \VEC{w} \in \RR^k :
\begin{pmatrix} \VEC{w} \\ \VEC{0} \end{pmatrix} \in V_2 \right\}
\cong V_2 \cap
\left\{ \begin{pmatrix} \VEC{w} \\ \VEC{0} \end{pmatrix} \in \RR^n :
\VEC{w} \in \RR^k \right\} \ .
\]
By definition of the induced topology, $W$ is an open subset of
$\displaystyle \RR^k$.  Let
$\phi: W \to U$ be the function defined by
$\displaystyle \phi(\VEC{w}) =  h^{-1}(\VEC{w},\VEC{0})$ for
$\VEC{w} \in W$.
We have that $\phi(W) = U$,
$\displaystyle \phi = h^{-1}\big|_W$ and $\phi^{-1} = h\big|_U$ are
continuous, and $\displaystyle \phi \in C^1(W)$ because
$\displaystyle h^{-1} \in C^1(V_2)$ with $W \subset V_2$.
Moreover,  $\displaystyle \tilde{h}:V_1 \to \RR^k$ defined by
$\displaystyle \tilde{h}(\VEC{x}) =
\begin{pmatrix}
h_1(\VEC{x}) & h_2(\VEC{x}) & \ldots & h_k(\VEC{x})
\end{pmatrix}^\top$ for all $\VEC{x} \in V_1$ is a continuously
differentiable extension of $\displaystyle \phi^{-1}$.

It follows from $\tilde{h} \circ \phi = \Id$ on $W$ that
$\displaystyle \diff \tilde{h} (\phi(\VEC{w})) \, \diff \phi(\VEC{w})
= \diff (\tilde{h} \circ \phi) (\VEC{w}) = \Id_k$
for all $\VEC{w} \in W$ where $\Id_k$ is the identity matrix on
$\displaystyle \RR^k$.  Thus, the rank of $\diff \phi(\VEC{w})$ is $k$.

\pdfF{manifolds/manifold5}{Figure associated to the proof of
Theorem~\ref{manifSPIVAK}}{Figure associated to the proof of
Theorem~\ref{manifSPIVAK}.  Only the sets $V_1 = \tilde{V}_1 \cap U$
and $V_2 \subset \tilde{V}_2$ are not represented in the figure.}{Manifold5}

\subQ{ii} Suppose that (2) is true.  The reader may want to refer to
Figure~\ref{Manifold5} for an illustration of the proof.
Choose $\VEC{w} \in W$ such that $\phi(\VEC{w}) = \VEC{u}$.  By
changing the order of the coordinates in $\displaystyle \RR^n$ if
necessary, we may assume that the matrix
\[
M = \begin{pmatrix}
\displaystyle \pdydx{\phi_1}{w_1}(\VEC{w}) &
\displaystyle \pdydx{\phi_1}{w_2}(\VEC{w}) & \ldots &
\displaystyle \pdydx{\phi_1}{w_k}(\VEC{w}) \\[1em] 
\displaystyle \pdydx{\phi_2}{w_1}(\VEC{w}) &
\displaystyle \pdydx{\phi_2}{w_2}(\VEC{w}) & \ldots &
\displaystyle \pdydx{\phi_2}{w_k}(\VEC{w}) \\
\vdots & \vdots & \ddots & \vdots \\
\displaystyle \pdydx{\phi_k}{w_1}(\VEC{w}) &
\displaystyle \pdydx{\phi_k}{w_2}(\VEC{w}) & \ldots &
\displaystyle \pdydx{\phi_k}{w_k}(\VEC{w})
\end{pmatrix}
\]
is invertible.  Let
$\displaystyle g: W \times \RR^{n-k} \to \RR^n$ be the function
defined by
$\displaystyle g\left( \begin{pmatrix} \VEC{z} \\ \VEC{y} \end{pmatrix} \right)
= \phi(\VEC{z}) + \begin{pmatrix} \VEC{0} \\ \VEC{y} \end{pmatrix}$
for $\VEC{z} \in W$ and $\VEC{y} \in \RR^{n-k}$.
Since
$\displaystyle \det \left( \diff g\left(
\begin{pmatrix} \VEC{w} \\ \VEC{0} \end{pmatrix} \right) \right)
= \det \begin{pmatrix} M & 0 \\ * & \Id_{n-k} \end{pmatrix}
= \det(M) \neq 0$, we may use the Inverse Function Theorem to get
an open neighbourhood $\displaystyle \tilde{V}_2 \subset W \times \RR^{n-k}$ of
$\displaystyle \begin{pmatrix} \VEC{w} & \VEC{0}\end{pmatrix}^\top$
and an open neighbourhood
$\tilde{V}_1 \subset V$ of $\displaystyle  g\left(
\begin{pmatrix} \VEC{w} \\ \VEC{0} \end{pmatrix}\right) = \phi(\VEC{w})
= \VEC{u}$
such that $g:\tilde{V}_2 \to \tilde{V}_1$ is a diffeomorphism.  The set
\[
D = \left\{ \VEC{z} \in W
: \begin{pmatrix} \VEC{z} \\ \VEC{0} \end{pmatrix} \in \tilde{V}_2 \right\}
\cong \tilde{V}_2 \cap
\left\{ \begin{pmatrix} \VEC{z} \\ \VEC{0} \end{pmatrix} : \VEC{z} \in W
\right\}
\]
is an open subset of $W$ because the induced topology on
$\displaystyle \RR^k \subset \RR^n$ is the usual topology on $\RR^k$.
Since $\phi:W \to V\cap S$ is a homeomorphism, we
have that $\phi(D)$ is an open subset of $V\cap S$.  Hence, there
exists an open set $\displaystyle U \subset \RR^n$ such that
$U\cap S = \phi(D)$.

Let $V_1 = \tilde{V}_1 \cap U$, $\displaystyle V_2 = g^{-1}(V_1)
\subset \tilde{V}_2$, and set
$\displaystyle h = g^{-1}\big|_{V_1}$.  Since
\[
V_1 \cap S = \left\{ \phi(\VEC{z}) :
\begin{pmatrix} \VEC{z} \\ \VEC{0} \end{pmatrix} \in V_2 \right\}
= \left\{ g\left( \begin{pmatrix} \VEC{z} \\ \VEC{0} \end{pmatrix} \right) :
\begin{pmatrix} \VEC{z} \\ \VEC{0} \end{pmatrix} \in V_2 \right\} \ ,
\]
we get that
\[
h(V_1 \cap S) = g^{-1}(V_1 \cap S) =
g^{-1} \left( \left\{ g\left( \begin{pmatrix} \VEC{z} \\
\VEC{0} \end{pmatrix} \right) :
\begin{pmatrix} \VEC{z} \\ \VEC{0} \end{pmatrix} \in V_2 \right\} \right)
= V_2 \cap 
\left\{ \begin{pmatrix} \VEC{z} \\ \VEC{0} \end{pmatrix} : \VEC{z} \in \RR^k
\right\} \ . \qedhere
\]
\end{proof}

Condition (1) in the previous theorem is often used to define manifolds
in $\displaystyle \RR^n$ (e.g.\ \cite{Fl,S}).

The previous theorem provides local charts for a manifold without boundary.
A slight modification of the previous proof (replace
$\displaystyle \RR^k$ by $H_k$) can
also provide local charts for manifold with boundary.

\begin{theorem}\label{manifSPIVAKB}
Let $S$ be a subset of $\displaystyle \RR^n$ and $\VEC{u} \in S$.  The
following two statements are equivalent
\begin{enumerate}
\item There exist an open neighbourhood
$\displaystyle V_1 \subset \RR^n$ of $\VEC{u}$, an open set
$\displaystyle V_2 \subset \RR^n$, and a diffeomorphism $h:V_1\to V_2$ such that
$h(V_1 \cap S) = \{ \VEC{w} \in V_2 : w_k \geq 0 \ \text{and}
\ w_{k+1} = w_{k+2} = \ldots = w_n = 0\}$.
\item There exist an open subset $W \subset H_k$, an open neighbourhood
$\displaystyle V \subset \RR^n$ of $\VEC{u}$ and a function
$\phi:W \to V \cap S$ such that $\phi$ is a homeomorphism, $\phi$ is
of class $\displaystyle C^1$ and $\diff \phi(\VEC{w})$ is of rank $k$
for all $\VEC{w} \in W$.
\end{enumerate}
Moreover, (1) implies that $\displaystyle \phi^{-1}:V\cap S\to W$ in
(2) can be extended to a continuously differentiable function in the
neighbourhood $V_1$ of $\VEC{u}$.
\end{theorem}

A consequence of the Theorem~\ref{manifSPIVAK} is the following result.

\begin{theorem}  \label{thmIFTmanif}
If $\displaystyle V \subset \RR^n$ is open and
$\displaystyle g:V \to \RR^{n-k}$ is a function of class
$\displaystyle C^1$ such that $\diff g(\VEC{x})$ is of rank $n-k$ for all
$\displaystyle \VEC{x} \in g^{-1}(\{\VEC{0}\})$, then
$\displaystyle S = g^{-1}(\{\VEC{0}\}) \subset V$
is a $k$-dimensional manifold.
\end{theorem}

\begin{proof}
We may apply Theorem~\ref{implLImT} to each $\VEC{v} \in V$ such that
$g(\VEC{v}) = \VEC{0}$.  We get an open set
$\displaystyle W_{\VEC{v}} \subset \RR^n$, an open neighbourhood
$U_{\VEC{v}} \subset V$ of $\VEC{v}$, and a diffeomorphism
$h:W_{\VEC{v}} \to U_{\VEC{v}}$ such that
$g(h(\VEC{x})) = (x_{n-k+1} \ x_{n-k+2} \ \ldots \  x_n)^\top$ for
$\VEC{x} \in W_{\VEC{v}}$.

Permuting the coordinates in $V$ if necessary, we may assume that
\[
P = \begin{pmatrix}
\displaystyle \pdydx{g_1}{x_{k+1}}(\VEC{v}) &
\displaystyle \pdydx{g_1}{x_{k+2}}(\VEC{v}) & \ldots
& \displaystyle \pdydx{g_1}{x_n}(\VEC{v}) \\
\displaystyle \pdydx{g_2}{x_{k+1}}(\VEC{v}) &
\displaystyle \pdydx{g_2}{x_{k+2}}(\VEC{v}) & \ldots
& \displaystyle \pdydx{g_2}{x_n}(\VEC{v}) \\
\vdots & \vdots & \ddots & \vdots \\
\displaystyle \pdydx{g_{n-k}}{x_{k+1}}(\VEC{v}) &
\displaystyle \pdydx{g_{n-k}}{x_{k+2}}(\VEC{v}) &
\ldots & \displaystyle \pdydx{g_{n-k}}{x_n}(\VEC{v})
\end{pmatrix}
\]
has rank $n-k$ and thus is invertible.  Consider the function
$\displaystyle f:W_{\VEC{v}} \to \RR^n$ defined by
$\displaystyle
f(\VEC{x}) = \begin{pmatrix} x_1 & x_2 & \ldots & x_k & g(h(\VEC{x}))
\end{pmatrix}^\top$ for $\VEC{x} \in W_{\VEC{v}}$.
If $\displaystyle \VEC{u} = h^{-1}(\VEC{v})$, we have that
\[
\diff f (\VEC{u}) =
\begin{pmatrix}
\Id_{k\times k} & 0 \\
\ast & P
\end{pmatrix} \diff h(\VEC{u})
% = \begin{pmatrix}
% \Id_{k\times k} & 0 \\
% \ast & P
% \end{pmatrix} \diff h^{-1}(\VEC{v}) \ ,
\]
where the $\ast$ stands for the $(n-k)\times k$ matrix of derivatives
$\displaystyle \pdydx{g_{i}}{x_{j}}(\VEC{v})$ with $1\leq i \leq n-k$
and $1 \leq j \leq k$.  Since
$\det\big(\diff f(\VEC{u})\big) = \det(P) \, \det(\diff h(\VEC{u})) \
\neq 0$, we have that $\diff f (\VEC{u})$ is invertible.  Therefore,
by shrinking $W_{\VEC{v}}$ if necessary, we may assume from the
Inverse Function Theorem that
$f : W_{\VEC{v}} \to V_{\VEC{v}} = f(W_{\VEC{v}})$ is a
diffeomorphism.

The function $f$ maps $\displaystyle h^{-1}(S) \cap W_{\VEC{v}}$ onto
$\left\{ \VEC{x} \in V_{\VEC{v}} : x_{n-k+1} = x_{n-k+2} = \ldots = x_n = 0
\right\}$.  It follows from Theorem~\ref{manifSPIVAK} that there exists a
local chart $(W, U, \phi)$ of $\displaystyle h^{-1}(S) \cap W_{\VEC{v}}$ about
$\displaystyle \VEC{w} = h^{-1}(\VEC{v})$.  Thus $(W, h(U), h\circ \phi)$
is a local chart of $S$ about $\VEC{v}$.
\[
\xymatrix{
U_{\VEC{v}} \cap S \ar[r]^g & \{\VEC{0}\} \subset \RR^{n-k} \\
U \subset W_{\VEC{v}}\cap h^{-1}(S) \ar[u]^{h} \ar[ur]_{g\circ h} & \\
W \ar[u]^{\phi} &
}
\]
Since we can find a local chart about each point of $S$, a structure of
manifold can be defined on $S$.
\end{proof}

\section{Tangent Spaces and Oriented Manifolds}

Let $\displaystyle W \subset \RR^k$ be an open set.  Given a vector
$\VEC{w} \in W$, we
have defined in Section~\ref{stokes_TSVS} the tangent space of $W$ at
$\VEC{w}$ to be the set
$\displaystyle \TS_{\VEC{w}} W = \left\{ (\VEC{w},\VEC{v}) : \VEC{v} \in
\RR^k \right\}$.
We extend our definition of a tangent space of an open subset of
$\displaystyle \RR^k$ to a tangent space of an open subset of
$H_k$ in a natural way.

\begin{defn}
Suppose that $W\subset H_k$ is an open
subset and $\VEC{w} \in W$.  By definition of induced topology, there
exists an open subset $V$ of $\displaystyle \RR^k$ such that
$W = H_k \cap V$.  Then $\TS_{\VEC{w}} W$ is defined as $\TS_{\VEC{w}} V$.
\end{defn}

There are many equivalent ways to define the tangent space of a manifold $S$
at a point $\VEC{w} \in S$.  The approach that we have chosen is more visual
and closer to the definition of the tangent space of an open subset
$W \subset H_k$ at a point $\VEC{w} \in W$ given above.

\begin{defn} \label{manifdefTang}
Let $\displaystyle S\subset \RR^n$ be a $k$-dimensional manifold of
class $\displaystyle C^1$ and $\VEC{u} \in S$.  Suppose that
$(W,U,\phi)$ is a local chart such that $\VEC{u} \in U$ and
$\displaystyle \VEC{w} = \phi^{-1}(\VEC{u})$.  The
{\bfseries tangent space}\index{Tangent Space} $\TS_{\VEC{u}} S$ of $S$
at $\VEC{u}$ is the subspace
$\displaystyle \TS_{\VEC{u}} S
= \phi_\ast(\TS_{\VEC{w}} W) \subset \TS_{\VEC{u}} \RR^n$.
\end{defn}

Recall that $\displaystyle \phi_\ast: \TS_{\VEC{w}} W \to \TS_{\VEC{u}} \RR^n$
is defined by $\displaystyle \phi_\ast (\VEC{w}, \VEC{v})
= \left( \phi(\VEC{w}) , \diff \phi (\VEC{w}) \VEC{v} \right)$
for $\displaystyle \VEC{v} \in \RR^k$.
By Proposition~\ref{manifWEX}, $\diff \phi (\VEC{w})$ is well defined
even if $\VEC{w} \in \partial W$.

If $S$ is a $k$-dimensional manifold with boundary of class
$\displaystyle C^1$, we have that
$\TS_{\VEC{u}} \partial S$ is a $(k-1)$ dimensional subspace of
$\TS_{\VEC{u}} S$.  In fact, if $(W,U,\phi)$ is a local chart
with $\VEC{u} \in U \cap \partial S$ and
$\VEC{w} = \phi(\VEC{u}) \in W \cap \partial H_k$, It follows from the
proof of Theorem~\ref{manifBSM} that
$\TS_{\VEC{u}} \partial S
= \phi_{\ast} \big(\TS_{\VEC{w}} (W \cap \partial H_k)\big)$
where
$\displaystyle
\TS_{\VEC{w}} (W\cap \partial H_k) \cong
\{ (\VEC{w},\VEC{y}) : \VEC{y} \in \RR^k \text{ and }
y_k = 0 \} \cong \RR^{k-1}$.  In particular, 
$\TS_{\VEC{u}} \partial S$ can be seen as a $(k-1)$-dimensional
subspace of $\TS_{\VEC{u}} S$.

The proposition below shows that the definition of the tangent space to a
$k$-dimensional manifold at a point is independent of the local 
chart used.  But before stating this proposition, suppose that $U$ and $V$ are
open subsets of $H_k$, and $f:U \to V$ and
$\displaystyle g:V \to W \subset \RR^k$ are two differentiable
functions.  It follows from Proposition~\ref{manifDGFDGDF} that
\begin{align*}
(g\circ f)_\ast (\VEC{u}, \VEC{x})
&= \left( (g\circ f)(\VEC{u}) , \diff (g\circ f) (\VEC{u}) \VEC{x} \right)
= \left( g(f(\VEC{u})) , \diff g(f(\VEC{u})) \diff f(\VEC{u}) \VEC{x}
\right) \\
& = g_\ast \left( f(\VEC{u}) , \diff f(\VEC{u}) \VEC{x} \right)
= g_\ast\left(f_\ast\left(\VEC{u} , \VEC{x} \right)\right)
= (g_\ast \circ f_\ast)\left(\VEC{u} , \VEC{x} \right)
\end{align*}
for all $(\VEC{u}, \VEC{x}) \in \TS_{\VEC{u}} U$.  Thus
$(g\circ f)_\ast = g_\ast \circ f_\ast$ is still true.  Suppose that
$f:U\to V$ is a diffeomorphism.  Using $W=U$ and $\displaystyle g=f^{-1}$ in
the previous relation, we get that
$\displaystyle (f^{-1})_\ast \circ f_\ast =  (f^{-1} \circ f)_\ast = \Id_\ast$
and $\displaystyle f_\ast \circ (f^{-1})_\ast
=  (f \circ f^{-1})_\ast = \Id_\ast$.
Thus $\displaystyle (f_\ast)^{-1} = (f^{-1})_\ast$.

\begin{prop} \label{manifCTS}
Let $\displaystyle S\subset \RR^n$ be a $k$-dimensional manifold of
class $\displaystyle C^1$ and $\VEC{u} \in S$.  Suppose that
$(W_i,U_i,\phi_i)$ for $i=1,2$ are local charts about $\VEC{u} \in S$.  If
$\displaystyle \VEC{w}_i = \phi_i^{-1}(\VEC{u})$ for $i=1,2$, then
\begin{equation} \label{manifTang}
(\phi_1)_\ast(\TS_{\VEC{w}_1} W_1) = (\phi_2)_\ast(\TS_{\VEC{w}_2} W_2) \ .
\end{equation}
Moreover
$\displaystyle (\phi_2^{-1} \circ \phi_1)_\ast: \TS_{\VEC{w}_1} W_1 \to
\TS_{\VEC{w}_2} W_2$ is a linear isomorphism.
\end{prop}

\begin{proof}
\subQ{i}
We begin by proving that
$\displaystyle
(\phi_2^{-1} \circ \phi_1)_\ast : \TS_{\VEC{w}_1} W_1 \to \TS_{\VEC{w}_2} W_2$
is a linear isomorphism.

Let
$\displaystyle h = \phi_2^{-1} \circ \phi_1: \phi_1^{-1}(U_1 \cap U_2) \to
\phi_2^{-1}(U_1 \cap U_2)$.
By definition of a local chart, $h$ is a diffeomorphism of
$\displaystyle \phi_1^{-1}(U_1 \cap U_2)$ onto
$\displaystyle \phi_2^{-1}(U_1 \cap U_2)$.  Since
$\displaystyle h\circ h^{-1} = \Id$ on
$\displaystyle \phi_2^{-1}(U_1 \cap U_2)$, we get that
$h_\ast \circ h^{-1}_\ast = (h \circ h^{-1})_\ast = \Id_\ast$
on $\TS_{\VEC{w}_2} W_2$.  Similarly, since
$\displaystyle h^{-1} \circ h = \Id$ on
$\displaystyle \phi_1^{-1}(U_1 \cap U_2)$, we get that
$\displaystyle h^{-1}_\ast \circ h_\ast = (h^{-1} \circ h)_\ast = \Id_\ast$
on $\TS_{\VEC{w}_1} W_1$.  Thus
$\displaystyle h_\ast: \TS_{\VEC{w}_1} W_1 \to \TS_{\VEC{w}_2} W_2$ is
invertible.  The linearity of $h$ on $\TS_{\VEC{w}_1} W_1$ comes from the
definition of $h_\ast$; namely,
$\displaystyle h_\ast( \VEC{w}_1, \VEC{v} )
= \left( h(\VEC{w}_1) , \diff h(\VEC{w}_1) \VEC{v} \right)
= \left( \VEC{w}_2 , \diff h(\VEC{w}_1) \VEC{v} \right)$
for all $\displaystyle \VEC{v} \in \RR^k$.

\subQ{ii}
To prove (\ref{manifTang}), we note that $\phi_1 = \phi_2 \circ h $.  Hence,
$(\phi_1)_\ast = (\phi_2 \circ h)_\ast = (\phi_2)_\ast \circ h_\ast$.
Since $h_\ast$ maps $\TS_{\VEC{w}_1} W_1$ onto $\TS_{\VEC{w}_2} W_2$ by
(i), we get (\ref{manifTang}).
\end{proof}

From Proposition~\ref{manifCTS}, we have that
$\displaystyle (\phi_2^{-1} \circ \phi_1)_\ast: \TS_{\VEC{w}_1} W_1 \to
\TS_{\VEC{w}_2} W_2$ is a linear isomorphism.  Namely,
$\displaystyle (\VEC{w}_1,\VEC{v}) \mapsto
(\phi_2^{-1} \circ \phi_1)_\ast( \VEC{w}_1, \VEC{v} )
= \left( \VEC{w}_2 ,
\diff (\phi_2^{-1} \circ \phi_1)(\VEC{w}_1) \VEC{v} \right)$
is a linear isomorphism for $\VEC{w}_1$ fixed.  Hence
$\displaystyle \diff (\phi_2^{-1} \circ \phi_1)(\VEC{w}_1)$
is a linear isomorphism of $\RR^k$ onto itself.  Thus
$\displaystyle \diff (\phi_2^{-1} \circ \phi_1)(\VEC{w}_1)$ has rank
$k$.
Hence $\diff \phi_1(\VEC{w})$ has rank $k$ and it is one-to-one because
its domaine is $\displaystyle \RR^k$.  Thus
$\displaystyle \TS_{\VEC{x}} S = (\phi_1)_\ast \left(\TS_{\VEC{w}_1} W_1\right)$
is a linear subspace of $\displaystyle \TS_{\VEC{x}} \RR^n$ of dimension $k$.

\begin{defn}
Suppose that $S_i$ is a $k_i$-dimensional manifold of class
$\displaystyle C^q$ for $i=1,2$.
A function $f:S_1 \to S_2$ is of class $\displaystyle C^q$ if
$\displaystyle \phi_2^{-1} \circ f \circ \phi_1:
\phi_1^{-1}(f^{-1}(U_2)) \to W_2$ is of class $\displaystyle C^q$ for
all local charts $(W_1,U_1,\phi_1)$ of $S_1$ and $(W_2,U_2,\phi_2)$ of $S_2$.

Suppose that $k_1 = k_2$ and $q>0$.  A function $f:S_1 \to S_2$ is a
{\bfseries diffeomorphism}\index{Diffeomorphism} if
$f^{-1}:S_2 \to S_1$ exists, and both $f:S_1 \to S_2$ and
$f^{-1}:S_2 \to S_1$ are functions of class $\displaystyle C^1$.
\end{defn}

The reader should prove that to determine if a map $f:S_1 \to S_2$
between two manifolds $S_1$ and $S_2$ is of class $\displaystyle C^q$,
it suffices to prove that $\displaystyle \phi_2^{-1} \circ f 
\circ \phi_1: \phi_1^{-1}(f^{-1}(U_2)) \to W_2$ is of class
$\displaystyle C^q$ for local charts $(W_1,U_1,\phi_1)$ of an atlas
$\A_1(k,q)$ for $S_1$ and local charts $(W_2,U_2,\phi_2)$ of an atlas
$\A_2(k,q)$ for $S_2$.

The following theorem is an immediate consequence of
Proposition~\ref{manifold_prop2}.

\begin{theorem} \label{theofSbS}
If $f:S_1 \to S_2$ is a diffeomorphism between two $k$-dimensional
manifolds of class $\displaystyle C^1$, then $f(\Int S_1) \subset \Int S_2$ and 
$f(\partial S_1) \subset \partial S_2$.  Moreover,
$\displaystyle f\big|_{\Int S_1}:\Int S_1 \to \Int S_2$ and 
$\displaystyle f\big|_{\partial S_1}:\partial S_1 \to \partial S_2$ are
diffeomorphisms.
\end{theorem}

Suppose that $S_i$ is a $k_1$-dimensional manifold of class
$\displaystyle C^1$ for $i=1,2$, and that
$f:S_1 \to S_2$ is a map of class $\displaystyle C^1$.  Since
$\displaystyle S_1^\circ = \emptyset$ when
$k <n$, we cannot define the derivative of $f$ from an extension of
$f$ to $\displaystyle \RR^n$ as we did for the derivative of a
function on the boundary points of $H_k$ because there is no unique
extension of $f$ and we cannot use the interior of $S_1$ to ensure
that the value of the derivative is unique.

Given $\VEC{u}_1 \in S_1$, let $\VEC{u}_2 = f(\VEC{u}_1)$.  We define
$\displaystyle f_\ast: \TS_{\VEC{u}_1} S_1 \to \TS_{\VEC{u}_2} S_2$
at it follows.  Let $(W_1,U_1,\phi_1)$ be a local chart of $S_1$ about
$\VEC{u}_1 \in S_1$ and $(W_2,U_2,\phi_2)$ be a local chart of $S_2$
about $\VEC{u}_2 \in S_2$.  We have that
$\displaystyle \phi_2^{-1} \circ f \circ \phi_1
: \phi_1^{-1}(f^{-1}(U_2)) \to W_2$ and
$\displaystyle (\phi_2^{-1} \circ f \circ \phi_1)_\ast
\left(\TS_{\VEC{w}_1} W_1\right) \subset \TS_{\VEC{w}_2} W_2$
where $\displaystyle \VEC{w}_1 = \phi_1^{-1}(\VEC{u}_1)$
and $\displaystyle \VEC{w}_2 = \phi_2^{-1}(\VEC{u}_2)$.
If $\displaystyle (\VEC{u}_1,\VEC{x}) \in \TS_{\VEC{u}_1}S_1$, then
$(\VEC{u}_1,\VEC{x}) = (\phi_1)_\ast(\VEC{w}_1,\VEC{y})$ for some
unique $\displaystyle (\VEC{w}_1,\VEC{y}) \in \TS_{\VEC{w}_1}W_1$.
We set $\displaystyle f_\ast(\VEC{u_1},\VEC{x}) = (\phi_2)_\ast \big(
(\phi_2^{-1} \circ f \circ \phi_1)_\ast(\VEC{w}_1,\VEC{y})\big)$.
We get the following commutative diagram.
\[
\xymatrix@C=2cm
{
\TS_{\VEC{u}_1} S_1 \ar[r]^{f_\ast} & \TS_{\VEC{u}_2} S_2 \\
\TS_{\VEC{w}_1=\phi_1^{-1}(\VEC{u}_1)} W_1 \ar[u]^{(\phi_1)_\ast}
\ar[r]^{(\phi_2^{-1}\circ f \circ \phi_1)_\ast} 
& \TS_{\VEC{w}_2=\phi_1^{-1}(\VEC{u}_2)} W_2 \ar[u]_{(\phi_2)_\ast}
}
\]
We note that $(\phi_i)_\ast: \TS_{\VEC{w}_i} W_i
\to \TS_{\VEC{u}_i} S_i$ is a linear isomorphism because
$\diff \phi_i(\VEC{w}_i)$ is of rank $k_i$.  Thus
$\diff \phi_i(\VEC{w}_i)$ is a linear isomorphism between $\RR^{k_i}$
and a substance of dimension $k_i$ in $\RR^n$.

\begin{defn}
Suppose that $S$ is a $k$-dimensional manifold of class $\displaystyle C^1$.
The {\bfseries tangent bundle of $\mathbf S$}\index{Tangent Bundle} is the set
\[
\TS S = \bigcup_{\VEC{u}\in S} \TS_{\VEC{u}} S \ .
\]
The map $\pi_S : \TS \to S$ defined by $\pi_S(\VEC{u}, \VEC{x}) = \VEC{u}$
for all $(\VEC{u},\VEC{x}) \in \TS S$ is the
{\bfseries tangent bundle projection}\index{Tangent Bundle Projection} of $S$.
\end{defn}

If $S$ is a $k$-dimensional manifold, we can give a structure of
$2k$-dimensional manifold to $\TS S$.  Suppose that $(W,U,\phi)$ is a local
chart of $S$.  Let
$\displaystyle \tilde{W} = W \times \RR^k$ and
$\displaystyle \tilde{U} = \TS\, U = \bigcup_{\VEC{u} \in U} \TS_{\VEC{u}} S$, and
consider the map $\tilde{\phi}:\tilde{W} \to \tilde{U}$ defined by
$\tilde{\phi}(\VEC{w}, \VEC{y}) = \phi_\ast(\VEC{w}, \VEC{y})
= \big(\phi(\VEC{w}) , \diff \phi(\VEC{w}) \VEC{y} \big)$ for all
$(\VEC{w},\VEC{y}) \in \tilde{W}$.
Then $(\tilde{W},\tilde{U}, \tilde{\phi})$ defines a local chart of $\TS S$.
Note that $\displaystyle \tilde{W} = W \times \RR^k$ can be seen as an
open subset of $H_{2k}$ because
$\displaystyle H_k \times \RR^k \cong  H_{2k}$ by permuting coordinates.

\begin{rmk}
The tangent bundle of a manifold $S$ has the structure of what is
called a {\bfseries vector bundle}\index{Vector Bundle}.  The reader
interested in learning more about this subject should consult \cite{A,Sv1}.
\end{rmk}

Suppose that $S$ is a $k$-dimensional manifold.  For each
$\VEC{u} \in S$, the tangent space $\TS_{\VEC{u}} S$ is a subspace of
$\displaystyle \RR^n$ of dimension $k$.  
Recall that if $(W,U,\phi)$ is a local chart with $\VEC{u} \in U$
and $\VEC{w} \in W$ such that $\VEC{u} = \phi(\VEC{w})$, then
$\displaystyle \diff \phi(\VEC{w})(\RR^k) \subset \RR^n$ is a subspace
of dimension $k$ of $\displaystyle \RR^n$.  Therefore, we may associate an
{\bfseries orientation}\index{Orientation}
\[
\mu_{\VEC{u}} =
[(\VEC{u},\VEC{x}_{\VEC{u},1}), (\VEC{u},\VEC{x}_{\VEC{u},2}), \ldots,
(\VEC{u},\VEC{x}_{\VEC{u},k})]
\]
to $\TS_{\VEC{u}} S$ where $\displaystyle \{ \VEC{x}_{\VEC{u},i} \}_{i=1}^k$ is
a basis of the subspace $\displaystyle \diff \phi(\VEC{w})(\RR^k)$.

\begin{defn}
Let $S$ be a $k$-dimensional manifold of class $\displaystyle C^1$.
Suppose that we assign an orientation $\mu_{\VEC{u}}$ on each $\TS_{\VEC{u}} S$
for each $\VEC{u} \in S$.  Such choice of orientations is
{\bfseries consistent}\index{Consistent Orientation} if for every
local chart $(W,U,\phi)$ of $S$, and for every $\VEC{w}_1$ and
$\VEC{w}_2$ in $W$, we have
\[
[ \phi_\ast(\VEC{w}_1,\VEC{e}_1), \phi_\ast(\VEC{w}_1,\VEC{e}_2),
\ldots , \phi_\ast(\VEC{w}_1,\VEC{e}_k) ] = \mu_{\phi(\VEC{w}_1)}
\]
if and only if
\[
[ \phi_\ast(\VEC{w}_2,\VEC{e}_1), \phi_\ast(\VEC{w}_2,\VEC{e}_2),
\ldots , \phi_\ast(\VEC{w}_2,\VEC{e}_k) ] = \mu_{\phi(\VEC{w}_2)} \ .
\]
\end{defn}

As a little warning to the reader, we emphasize that, in the previous
definition, we did not request that
\[
[ \phi_\ast(\VEC{w},\VEC{e}_1), \phi_\ast(\VEC{w},\VEC{e}_2),
\ldots , \phi_\ast(\VEC{w},\VEC{e}_k) ] = \mu_{\phi(\VEC{w})}
\]
for all $\VEC{w} \in W$.  We may as well have that
\[
[ \phi_\ast(\VEC{w},\VEC{e}_1), \phi_\ast(\VEC{w},\VEC{e}_2),
\ldots , \phi_\ast(\VEC{w},\VEC{e}_k) ] \neq \mu_{\phi(\VEC{w})}
\]
for all $\VEC{w} \in W$.  As we have seen in
Section~\ref{sectOrientRn}, we have for each $\VEC{u} \in S$ two
possible orientations on $\TS_{\VEC{u}} S$.  This will play an
important role in the study of integration on manifolds later.

\begin{defn}
Let $S$ be a $k$-dimensional manifold of class $\displaystyle C^1$.
Suppose that the choice of orientation $\mu_{\VEC{u}}$ on each
$\VEC{u} \in S$ is consistent.  A local chart $(W,U,\phi)$ of $S$ is called
{\bfseries orientation preserving}\index{Orientation Preserving} if
\[
[\phi_\ast(\VEC{w},\VEC{e}_1), \phi_\ast(\VEC{w},\VEC{e}_2), \ldots,
\phi_\ast(\VEC{w},\VEC{e}_k)] = \mu_{\phi(\VEC{w})}
\]
for one (and so all \footnotemark) $\VEC{w} \in W$.
\end{defn}

\footnotetext{Because the choice of orientation is consistent.}

If a local chart $(W,U,\phi)$ of a $k$-dimensional manifold $S$
without boundary is not orientation preserving, we can easily produce
a local chart which will be orientation preserving.  Choose a linear map
$\displaystyle T:\RR^k \to \RR^k$ with
$\det(T) = -1$ (e.g.\ a permutation of two coordinates), then
$\displaystyle (T^{-1}(W), V, \phi \circ T)$ is a local chart which is
orientation preserving.

Suppose that $(W_i,U_i,\phi_i)$ for $i=1,2$ are two orientation
preserving local charts of a $k$-dimensional manifold $S$.  If
$\VEC{u} \in U_1 \cap U_2$ and $\VEC{u} = \phi_i(\VEC{w}_i)$ for
$\VEC{w}_i \in W_i$, we have that
\begin{align*}
&[(\phi_1)_\ast(\VEC{w}_1,\VEC{e}_1), (\phi_1)_\ast(\VEC{w}_1,\VEC{e}_2),
\ldots, (\phi_1)_\ast(\VEC{w}_1,\VEC{e}_k)] \\
&\qquad = \mu_{\VEC{u}}
= [(\phi_2)_\ast(\VEC{w}_2,\VEC{e}_1), (\phi_2)_\ast(\VEC{w}_2,\VEC{e}_2),
\ldots, (\phi_2)_\ast(\VEC{w}_2,\VEC{e}_k)] \ .
\end{align*}
Thus
\begin{align*}
&[(\phi_2^{-1}\circ \phi_1)_\ast(\VEC{w}_1,\VEC{e}_1),
(\phi_2^{-1}\circ \phi_1)_\ast(\VEC{w}_1,\VEC{e}_2),
\ldots, (\phi_2^{-1}\circ \phi_1)_\ast(\VEC{w}_1,\VEC{e}_k)] \\
& \qquad = [(\VEC{w}_2,\VEC{e}_1), (\VEC{w}_2,\VEC{e}_2), \ldots,
(\VEC{w}_2,\VEC{e}_k)] \ .
\end{align*}
This implies that
$\displaystyle \diff (\phi_2^{-1} \circ \phi_1)(\VEC{w}_1)$ preserves the
orientation on $\displaystyle \RR^k$.  Hence
$\displaystyle \det( \diff (\phi_2^{-1} \circ \phi_1)(\VEC{w}_1) ) > 0$.

\begin{defn}
A $k$-dimensional manifold $S$ of class $\displaystyle C^1$ is
{\bfseries orientable}\index{Orientable} if there exists a
consistent choice of orientations $\mu_{\VEC{u}}$ on $\TS_{\VEC{u}} S$ for
each $\VEC{u} \in S$.  A consistent choice of orientation is called an
{\bfseries orientation}\index{Orientation} of $S$.  A
$k$-dimensional manifold $S$ with an orientation is called an
{\bfseries oriented manifold}\index{Manifold!Oriented Manifold}.
\end{defn}

\begin{rmk}
We will not consider non-orientable manifolds in this book.  Many of
the results about oriented manifolds presented in this book have
versions for non-orientable manifolds.  However, there are also many
results about oriented manifolds that do not have counterparts for
non-orientable manifolds and vice-versa.
\end{rmk}

\begin{egg}
We cannot mention non-orientable manifolds without providing the best
know example of a non-orientable manifold; namely, the Möbius strip
(Figure~\ref{mobius}).  It is not possible to provide a consistent
orientation on the Möbius strip.  Suppose that $M$ is a Möbius strip
and that $\tilde{\VEC{u}} \in M$.  It is easy to imagine a sequence of local
charts $(W_i,U_i,\phi_i)$ for $1 \leq i \leq I$ such that
$U_{i-1} \cap U_i \neq \emptyset$ for $1 <i \leq I$,
$(W_i,U_i,\phi_i)$ preserves the orientation inherited from
$(W_{i-1},U_{i-1},\phi_{i-1})$ for $1 <i \leq I$ \footnote{Namely,
if $\VEC{u} = \phi_i(\VEC{w}_i) = \phi_{i-1}(\VEC{w}_{i-1})$,
then\\
$[(\phi_i)_\ast(\VEC{w}_i,\VEC{e}_1), (\phi_i)_\ast(\VEC{w}_i,\VEC{e}_2)]
= [(\phi_{i-1})_\ast(\VEC{w}_{i-1},\VEC{e}_1),
(\phi_{i-1})_\ast(\VEC{w}_{i-1},\VEC{e}_2)] = \mu_{\VEC{u}}$.},
and $\tilde{\VEC{u}} \in U_1 \cap U_I$.
If $\tilde{\VEC{u}} = \phi_1(\tilde{\VEC{w}}_1) = \phi_I(\tilde{\VEC{w}}_I)$,
then
$[(\phi_1)_\ast(\tilde{\VEC{w}}_1,\VEC{e}_1),
(\phi_1)_\ast(\tilde{\VEC{w}}_1,\VEC{e}_2)]
\neq [(\phi_I)_\ast(\tilde{\VEC{w}}_I,\VEC{e}_1),
(\phi_I)_\ast(\tilde{\VEC{w}}_I,\VEC{e}_2)]$.
\end{egg}

\pdfF{manifolds/mobius}{Möbius strip}{Möbius strip.}{mobius}

\begin{defn}
Let $S$ and $\tilde{S}$ be two $k$-dimensional manifolds of class
$\displaystyle C^1$.
Suppose that $\mu_{\VEC{u}}$ for each $\VEC{u} \in S$ is the
orientation on $S$ and $\tilde{\mu}_{\VEC{u}}$ for each $\VEC{u} \in
\tilde{S}$ is the orientation on $\tilde{S}$.  We say that a
diffeomorphism $g:S \to \tilde{S}$ is 
{\bfseries orientation preserving}\index{Orientation Preserving}
if
\[
[g_\ast(\VEC{u},\VEC{x}_1), g_\ast(\VEC{u},\VEC{x}_2), \ldots,
g_\ast(\VEC{u},\VEC{x}_k)] = \tilde{\mu}_{g(\VEC{u})}
\]
for all $\displaystyle \{ (\VEC{u}, \VEC{x}_i \}_{1\leq i \leq k}
\subset \TS_{\VEC{u}} S$ such that
$\displaystyle [(\VEC{u},\VEC{x}_1), (\VEC{u},\VEC{x}_2), \ldots,
(\VEC{u},\VEC{x}_k)] = \mu_{\VEC{u}}$ and all $\VEC{u} \in S$.
\end{defn}

The orientation on the $k$-dimensional manifold $S$ with boundary
induces an orientation on the $(k-1)$-dimensional manifold $\partial S$.

Suppose that $S$ is a oriented $k$-dimensional manifold.  To define the
induced orientation on $\partial S$, we begin by defining the
outward unit normal to the
manifold $\partial S$ at $\VEC{u} \in \partial S$. 
Since $\TS_{\VEC{u}} \partial S$ is a $(k-1)$ dimensional subspace of
$\TS_{\VEC{u}} S$, there are two unit vectors orthogonal to
$\TS_{\VEC{u}} \partial S$ in $\TS_{\VEC{u}} S$.  Let
$(\VEC{u}, \VEC{x}_1)$ and $(\VEC{u},\VEC{x}_2)$ be
these two vectors.  One of the two is the outward unit normal.  Choose
an orientation preserving local chart $(W,U,\phi)$ of $S$ such that
$\VEC{u} \in U \cap \partial S$,
and let $\VEC{w} \in \partial W \cap H_k$ be the vector such that
$\VEC{u} = \phi(\VEC{w})$.  Since
$\phi_\ast:\TS_{\VEC{w}} W \to \TS_{\VEC{u}} S$ is a linear isomorphism, there
exists a unique $(\VEC{w},\VEC{q}) \in \TS_{\VEC{w}} W$ such that
$\VEC{q}\in \RR^k \setminus H_k$ and
$\phi_\ast(\VEC{w},\VEC{q})$ is one of $(\VEC{u}, \VEC{x}_1)$ or
$(\VEC{u},\VEC{x}_2)$.

\begin{defn}  \label{manifOutNormal}
The vector $\VEC{n}_{\VEC{u}} = (\VEC{u},\VEC{n}) =
\phi_\ast(\VEC{w},\VEC{q}) \in \TS_{\VEC{u}} S$ is
called the {\bfseries outward unit normal}\index{Outward Unit Normal}
to $\partial S$ at $\VEC{u} \in \partial S$ (Figure~\ref{NORMAL}).
\end{defn}

One can show using Proposition~\ref{manifold_prop2} that the definition of
the outward unit normal is independent of the local chart used.

\pdfF{manifolds/normal}{Representation of the outward unit normal to the
boundary of a manifold}{The outward unit normal
$\VEC{n}_{\VEC{u}} \in \TS_{\VEC{u}} S$ to the manifold
$\partial S$ at the point $\VEC{u} \in \partial S$ is given by
$\VEC{n}_{\VEC{u}} =(\VEC{u},\VEC{n})$ where
$\VEC{n} = \diff \phi(\VEC{w}) (\VEC{q})$.}{NORMAL}

\begin{defn}\label{manifdbOrient}
Suppose that $S$ is a $k$-dimensional manifold of class
$\displaystyle C^1$ with boundary and that
$\mu_{\VEC{u}}$ for $\VEC{u} \in S$ is an orientation on $S$.  For
each $\VEC{u} \in \partial S$, choose
$\displaystyle \left\{ (\VEC{u}, \VEC{x}_{\VEC{u},i})
\right\}_{i=1}^{k-1} \subset \TS_{\VEC{u}} \partial S$ such that
$\displaystyle \left[ \VEC{n}_{\VEC{u}} , (\VEC{u}, \VEC{x}_{\VEC{u},1}),
(\VEC{u}, \VEC{x}_{\VEC{u},2}),
\ldots, (\VEC{u}, \VEC{x}_{\VEC{u},k-1}) \right] = \mu_{\VEC{u}}$.
The {\bfseries induced orientation on
$\mathbf{\partial S}$}\index{Induced Orientation} is defined as
$\displaystyle (\partial \mu)_{\VEC{u}} \equiv
\left[(\VEC{u}, \VEC{x}_{\VEC{u},1}),
(\VEC{u}, \VEC{x}_{\VEC{u},2}), \ldots, (\VEC{u}, \VEC{x}_{\VEC{u},k-1})
\right]$
for $\VEC{u} \in \partial S$.
\end{defn}

Using the local charts of $\partial S$ provided in
Theorem~\ref{manifBSM}, it is easy to prove that if $\mu_{\VEC{u}}$ is
a consistent choice of orientations on $\TS_{\VEC{u}} S$ for $\VEC{u} \in S$,
then $(\partial \mu)_{\VEC{u}}$ is a consistent choice of orientations on
$\TS_{\VEC{u}} \partial S$ for $\VEC{u} \in \partial S$.

\begin{rmk}
For the special case where $S= H_k$ with the standard     \label{rmkHkOrient}
orientation, then the induced orientation on
$\displaystyle \partial S \cong \RR^{k-1}$ is $\displaystyle (-1)^k$ times the
standard orientation on $\displaystyle \RR^{k-1}$.  We have that
$\VEC{n}_{\VEC{w}} = (\VEC{w}, -\VEC{e}_k)$ for $\VEC{w} \in \partial H_k$.
Hence
\begin{align*}
&\left[ (\VEC{w}, -\VEC{e}_k), (\VEC{w},\VEC{e}_1), (\VEC{w},\VEC{e}_2),
\ldots , (\VEC{w},\VEC{e}_{k-1}) \right] \\
&\qquad = (-1)^{k-1} \left[ (\VEC{w},\VEC{e}_1), (\VEC{w},\VEC{e}_2),
\ldots , (\VEC{w},\VEC{e}_{k-1}),(\VEC{w},-\VEC{e}_k) \right] \\
&\qquad =  (-1)^{k} \left[ (\VEC{w},\VEC{e}_1), (\VEC{w},\VEC{e}_2),
\ldots , (\VEC{w},\VEC{e}_{k-1}),(\VEC{w},\VEC{e}_k) \right] \ .
\end{align*}
The two orientations are the same when $k$ is even and are of opposite signs
when $k$ is odd.
\end{rmk}

\begin{egg}
We illustrate the previous concept of induced
orientation and outward unit normal for two of the simplest manifolds
in $\displaystyle \RR^3$.

For the first example, $S$ is a disk in the
$x_1,x_2$ plane.  Thus $S$ is an orientable $2$-dimensional manifold.
\pdfbox{manifolds/outUN1}
The chosen orientation on $\TS_{\VEC{u}} S$ is given by
$\mu_{\VEC{u}} = [(\VEC{u},\VEC{x}_{\VEC{u},1}),
(\VEC{u},\VEC{x}_{\VEC{u},2})]$.  The outward unit normal to $S$ at
$\VEC{u} \in \partial S$ is
$\VEC{n}_{\VEC{u}} = (\VEC{u}, -\VEC{x}_{\VEC{u},2})$.
The orientation on $\TS_{\VEC{u}} \partial S$ is
$(\partial \mu)_{\VEC{u}} = [(\VEC{u},\VEC{x}_{\VEC{u},1})]$ because
$[\VEC{n}_{\VEC{u}}, (\VEC{u},\VEC{x}_{\VEC{u},1})]
=[(\VEC{u}, -\VEC{x}_{\VEC{u},2}),(\VEC{u},\VEC{x}_{\VEC{u},1})]
=-[(\VEC{u},\VEC{x}_{\VEC{u},1}),(\VEC{u}, -\VEC{x}_{\VEC{u},2})]
= \mu_{\VEC{u}}$.

For our second example, $S$ is half a ball.  Thus $S$ is an orientable
$3$-dimensional manifold.
\pdfbox{manifolds/outUN2}
We consider $\VEC{u} \in \partial S$ on the bottom face of $S$.
The chosen orientation on $\TS_{\VEC{u}} S$ is given by
$\mu_{\VEC{u}} = [(\VEC{u},\VEC{x}_{\VEC{u},1}),
(\VEC{u},\VEC{x}_{\VEC{u},2}),(\VEC{u},\VEC{x}_{\VEC{u},3})$.
The outward unit normal to $S$ at
$\VEC{u} \in \partial S$ is
$\VEC{n}_{\VEC{u}} = (\VEC{u}, -\VEC{x}_{\VEC{u},3})$.
The orientation on $\TS_{\VEC{u}} \partial S$ is
$(\partial \mu)_{\VEC{u}}
= [(\VEC{u},\VEC{x}_{\VEC{u},2}),(\VEC{u},\VEC{x}_{\VEC{u},1})]$ because
\begin{align*}
[\VEC{n}_{\VEC{u}}, (\VEC{u},\VEC{x}_{\VEC{u},2}),
(\VEC{u},\VEC{x}_{\VEC{u},1})]
&=[(\VEC{u}, -\VEC{x}_{\VEC{u},3}),(\VEC{u},\VEC{x}_{\VEC{u},2}),
(\VEC{u},\VEC{x}_{\VEC{u},1})] \\
&=-[(\VEC{u},\VEC{x}_{\VEC{u},1}),(\VEC{u},\VEC{x}_{\VEC{u},2}),
-(\VEC{u},\VEC{x}_{\VEC{u},3})] = \mu_{\VEC{u}} \ .
\end{align*}
\end{egg}

For $(n-1)$-dimensional manifold $S$ in $\displaystyle \RR^n$, we may
define an outward unit normal to $S$ at every point of $S$ even if $S$
is not the boundary of a $n$-dimensional manifold.

\begin{defn}\label{manifNormal}
Suppose that $S$ is a $(n-1)$-dimensional manifold of class
$\displaystyle C^1$.  Suppose that
$\displaystyle \mu_{\VEC{u}} =
[(\VEC{u},\VEC{x}_{\VEC{u},1}), (\VEC{u},\VEC{x}_{\VEC{u},2}), \ldots,
(\VEC{u},\VEC{x}_{\VEC{u},n-1})]$
is the selected orientation on $\TS_{\VEC{u}} S$ for $\VEC{u} \in S$.
The {\bfseries outward unit normal}\index{Outward Unit Normal} to $S$ at
$\VEC{u} \in S$ is the unit vector $\VEC{n}_{\VEC{u}} =
(\VEC{u},\VEC{n}(\VEC{u})) \in \TS_{\VEC{u}} \RR^n \cong \RR^n$
such that
$\displaystyle
[(\VEC{u}, \VEC{n}(\VEC{u})), (\VEC{u},\VEC{x}_{\VEC{u},1}),
(\VEC{u},\VEC{x}_{\VEC{u},2}), \ldots, (\VEC{u},\VEC{x}_{\VEC{u},n-1})]
= [(\VEC{u}, \VEC{e}_1), (\VEC{u},\VEC{e}_2), \ldots,
(\VEC{u},\VEC{e}_n)]$,
the standard orientation on $\displaystyle \RR^n$.
\end{defn}

\subsection{Tangent Spaces as Equivalence Classes of Curves}
\label{subSTSequiCC}

We find in the literature several equivalent definitions of the
tangent space to a manifold at a point.  The choice of the definition is
usually determined by how we plan to use them.  We have chosen the
definition which is closer to the idea that the reader may have of a
tangent space.  We briefly present in this subsection and the next
subsection two other definitions of the tangent space to a manifold at
a point.

Suppose that $S$ is a $k$-dimensional manifold of class $\displaystyle C^1$
and $\VEC{u} \in S$.  A differentiable curve at $\VEC{u} \in S$ is a function
$\rho:]-a,a[\to S$ of class $\displaystyle C^1$ such that
$\rho(0)=\VEC{u}$ and $a>0$.  Two differentiable curves
$\rho_i:]-a_i,a_i[\to S$ for $1 \leq i \leq 2$ at
$\VEC{u}$ are {\bfseries equivalent}\index{Equivalent Curves} if
$\rho_1'(0) = \rho_2'(0)$.

The {\bfseries tangent space $\tilde{T}_{\VEC{u}} S$}\index{Tangent Space}
is defined as the set of all equivalent class of differentiable curves
$\rho:]-a,a[\to S$ at $\VEC{u} \in S$.  If $[\rho]_{\VEC{u}}$ denotes
the equivalent class for the differentiable curve $\rho:]-a,a[\to S$
at $\VEC{u} \in S$, then
\[
\tilde{T}_{\VEC{u}} S = \left\{ [\rho]_{\VEC{u}} : \rho:]-a,a[\to S
\ \text{is a differentiable curve at}\ \VEC{u} \right\} \ .
\]

We now explain why this definition of a tangent space is equivalent to
Definition~\ref{manifdefTang}.

Suppose that $(W,U,\phi)$ is a local chart such that $\VEC{u} \in U$ and let
$\displaystyle \VEC{w} = \phi^{-1}(\VEC{u})$.  The function
\begin{align*}
\tilde{\phi} : \tilde{T}_{\VEC{w}} W & \to \tilde{T}_{\VEC{u}} S \\
[\rho]_{\VEC{w}} &\mapsto [\phi\circ \rho]_{\VEC{u}}
\end{align*}
defines a bijective map.

\stage{i} The function $\tilde{\phi}$ is well defined.
\begin{enumerate}
\item If $\rho:]-a,a[\to W$ is a differentiable curve at $\VEC{w}$, then
$\phi\circ\rho:]-a,a[\to S$ is a differentiable curve at $\VEC{u}$
because $\phi\circ \rho$ is of class $\displaystyle C^1$ (composition
of functions of class $\displaystyle C^1$) and
$(\phi\circ\rho)(0)= \phi(\rho(0)) = \phi(\VEC{w}) = \VEC{u}$.
\item Let $\rho_1$ and $\rho_2$ be two equivalent differentiable
curves at $\VEC{w}$.  Then $\phi\circ \rho_1$ and $\phi\circ \rho_2$
are two equivalent differentiable curves at $\VEC{u}$ because
$\rho_1'(0)=\rho_2'(0)$ implies that
\begin{align*}
(\phi\circ\rho_1)'(0) &= \diff \phi(\rho_1(0)) \rho_1'(0)
= \diff \phi(\VEC{w}) \rho_1'(0) = \diff \phi(\VEC{w}) \rho_2'(0) \\
& = \diff \phi(\rho_2(0)) \rho_2'(0) = (\phi\circ\rho_2)'(0) \ .
\end{align*}
\end{enumerate}

\stage{ii} The function $\tilde{\phi} : \tilde{T}_{\VEC{w}} W \to
\tilde{T}_{\VEC{u}}S$ is one-to-one.  If $\rho_1$ and $\rho_2$ are
two differentiable curves at $\VEC{w}$ such that
$\phi\circ \rho_1$ and $\phi\circ\rho_2$ are equivalent differentiable
curves at $\VEC{u}$,  then
\begin{align*}
\diff \phi(\VEC{w}) \rho_1'(0) &= \diff \phi(\rho_1(0)) \rho_1'(0)
= (\phi\circ\rho_1)'(0) = (\phi\circ\rho_2)'(0) \\
&= \diff \phi(\rho_2(0)) \rho_2'(0) = \diff \phi(\VEC{w}) \rho_2'(0)  \ .
\end{align*}
Since $\diff \phi(\VEC{w})$ is one-to-one, $\rho_1'(0) = \rho_2'(0)$ and so
$\rho_1$ and $\rho_2$ are two equivalent differentiable curves at $\VEC{w}$.

\stage{iii} To prove that
$\tilde{\phi} : \tilde{T}_{\VEC{w}} W \to \tilde{T}_{\VEC{u}} S$
is onto, it suffices to note that if $\rho:]-a,a[\to U$ is a
differentiable curve at $\VEC{u}$, then $\eta:]-a,a[\to W$ defined by
$\rho(t) = \phi(\eta(t))$ for $t\in]-a,a[$ is a differentiable curve
at $\VEC{w}$.  To prove that the curve is $\eta$ is of class
$\displaystyle C^1$, we
use the map $\tilde{h}$ provided in the proof of
Theorem~\ref{manifSPIVAK} and Theorem~\ref{manifSPIVAKB}.  We have
that $\displaystyle \tilde{h}: \RR^n  \to \RR^k$ is of class
$\displaystyle C^1$ and
$\displaystyle \tilde{h}\big|_{S \cap U} = \phi^{-1}: U \to W$.
So $\eta = \tilde{h} \circ \rho:]-a,a[\to W$ is a differentiable curve
at $\VEC{w}$.

We define a map $\psi$ from $\TS_{\VEC{w}} W$ to $\tilde{T}_{\VEC{w}} W$
as it follows.  We assign to each
$(\VEC{w},\VEC{y}) \in \TS_{\VEC{w}} W$ a unique
equivalence class of differentiable curves at $\VEC{w}$ such that the
derivative at $0$ is $\VEC{y}$; namely, the class
$[\rho_{\VEC{w},\VEC{y}}]_{\VEC{w}}$ of all differentiable curves
equivalent to $\rho_{\VEC{w},\VEC{y}}(t) = \VEC{w} + t \VEC{y}$ for
$t \in ]-a,a[$ and $a$ small enough.  We leave it to the reader to
convince themselves that the function
\begin{align*}
\psi : \TS_{\VEC{w}} W & \to \tilde{T}_{\VEC{w}} W \\
(\VEC{w},\VEC{y}) & \mapsto [\rho_{\VEC{w},\VEC{y}}]_{\VEC{w}}
\end{align*}
is a bijection.

Finally, since $\diff \phi(\VEC{w})$ is a linear isomorphism onto
$\displaystyle \diff \phi(\VEC{w}) (\RR^k)$ because
$\diff \phi(\VEC{w})$ is of rank $k$, we have that
$\phi_\ast :\TS_{\VEC{w}} W \to \TS_{\VEC{u}} S$ is a linear isomorphism.
Thus
$\displaystyle \tilde{\phi}\circ \psi \circ(\phi_\ast)^{-1}: \TS_{\VEC{u}} S \to
\tilde{T}_{\VEC{u}} S$ is an isomorphism.  Hence $\TS_{\VEC{u}} S$ and
$\tilde{T}_{\VEC{u}} S$ are isomorphic.  In fact, it follows from the
definition of this isomorphism that the equivalent class of curves at
$\VEC{u}$ associated to $(\VEC{u}, \VEC{x}) \in \TS_{\VEC{u}} S$ is the
class of all differentiable curves $\rho:]-a,a[\to S$ such that
$\rho(0)=\VEC{u}$ and $\rho'(0) = \VEC{x}$.  To prove this claim,
suppose that $(\VEC{u},\VEC{x}) \in \TS_{\VEC{u}} S$.  Then
$\displaystyle (\phi_\ast)^{-1}(\VEC{u},\VEC{x}) = (\VEC{w},\VEC{y})$ where
$\phi(\VEC{w}) = \VEC{u}$ and $(\diff \phi(\VEC{w}))(\VEC{y}) = \VEC{x}$.
Hence $\displaystyle (\psi \circ (\phi_\ast)^{-1})(\VEC{u},\VEC{x})
= \psi(\VEC{w},\VEC{y}) = [\rho_{\VEC{w},\VEC{y}}]_{\VEC{w}}$.
Thus
$(\tilde{\phi}\circ \psi \circ(\phi_\ast)^{-1})(\VEC{u},\VEC{x})
= \tilde{\phi}([\rho_{\VEC{w},\VEC{y}}]_{\VEC{w}})
= [(\phi\circ\rho_{\VEC{w},\VEC{y}})]_{\VEC{u}}$ because
$(\phi \circ \rho_{\VEC{w},\VEC{y}})(0)
= \phi(\rho_{\VEC{w},\VEC{y}}(0)) = \phi(\VEC{w}) = \VEC{u}$.  We also
have that $(\phi \circ \rho_{\VEC{w},\VEC{y}})'(0)
= (\diff \phi(\VEC{w}))(\VEC{y}) = \VEC{x}$.

In the previous discussion, we have assumed that $\VEC{u} \in \Int S$.
We leave it to the reader to study the case where $\VEC{u} \in \partial S$.
We then have to consider the intervals $[0,a[$ or $]-a,0]$ to get
a curve in $S$.

\subsection{Tangent Spaces as Spaces of differential Linear
Operators} \label{secTSasDiffop}

Suppose that $S$ is a $k$-dimensional manifold of class
$\displaystyle C^1$ and that $\VEC{u} \in S$.
A {\bfseries tangent vector}\index{Tangent Vector} to $S$ at
$\VEC{u}$ is a map $\displaystyle q: C^\infty(S) \to \RR$ with the
following property.  If $(W,U,\phi)$ is a local chart of $S$ about 
$\VEC{u} \in S$, then there exists $\VEC{y} \in \RR^k$  such that
\[
  \tilde{q} = \sum_{i=1}^k y_i \pdfdx{}{w_i}\Big|_{\VEC{w}}
\]
with $\displaystyle \VEC{w} = \phi^{-1}(\VEC{u})$
is the local representation of $q$; namely,
\begin{equation} \label{TSv2Eq1}
q(g) = \phi^\ast(\tilde{q})(g)
= \sum_{i=1}^k y_i \pdfdx{(g\circ \phi)}{w_i}\Big|_{\VEC{w}=\phi^{-1}(\VEC{u})}
\end{equation}
for all real valued functions $\displaystyle g \in C^\infty(S)$.

The {\bfseries tangent space}\index{Tangent Space}
$\breve{\TS}_{\VEC{u}} S$ is the set of all the tangent vectors to $S$
at $\VEC{u}$.

\subI{Uniqueness of $\VEC{y}$}
The vector $\VEC{y}$ satisfying (\ref{TSv2Eq1}) is unique.
Suppose that $\displaystyle V \subset \RR^n$ is an open set such that 
$U = S \cap V$.  Choose an open set $B$ such that 
$\VEC{u} \in B \subset \overline{B} \subset V$ and a smooth function
$\psi: \RR^n \to R$ such that
\begin{enumerate}
\item $0 \leq \psi(\VEC{x}) \leq 1$ for all $\displaystyle \VEC{x} \in \RR^n$, 
\item $\psi(\VEC{x}) = 1$ for all $\VEC{x} \in B$ and
\item $\psi(\VEC{x}) = 0$ for all $\displaystyle \VEC{x} \in \RR^n \setminus B$.
\end{enumerate}
We can get the function $\psi$ from Theorem~\ref{cov1}, Partition
of Unity, with the open cover
$\displaystyle \{ V, \RR^n \setminus \overline{B}\}$.  If
\[
g(\VEC{u}) = \begin{cases}
\displaystyle \pi_j(\phi^{-1}(\VEC{u})) \psi(\VEC{u})
& \quad \text{if} \ \VEC{u} \in U \\
0 & \displaystyle \quad \text{if}\ \VEC{u} \in \RR^n \setminus U
\end{cases}
\]
for $1 \leq j \leq k$ fixed, then $\displaystyle g \in C^\infty(S)$.
The projection $\displaystyle \pi_j:\RR^k \to \RR$ defined by
$\pi_j(\VEC{w}) = w_j$ is used to obtain the
$\displaystyle j^{th}$ component of $\phi^{-1}$.
Since $\displaystyle
(g\circ \phi)(\VEC{w}) = \pi_j(\phi^{-1}(\phi(\VEC{w}))) \psi(\phi(\VEC{w}))
= w_j$ for all
$\displaystyle \VEC{w} \in \phi^{-1}(S\cap B) \subset W$, an open neighbourhood
of $\displaystyle \phi^{-1}(\VEC{u})$, we get that
\begin{equation} \label{tsdloEq1}
q(g) = \sum_{i=1}^k y_i \pdfdx{(g\circ \phi)}{w_i}(\phi^{-1}(\VEC{u})) = y_j
\ .
\end{equation}
Thus $y_j$ is uniquely determinate by the value of $q(g)$.

\subI{Dependence on the local chart}
The existence of the vector $\VEC{y} \in \RR^k$ is independent of the
local chart used.  Suppose that
$(W_m,U_m,\phi_m)$ for $m=1,2$ are two local
charts of $S$ and $\VEC{u} \in U_1 \cap U_2$.  Moreover, suppose that
there exists $\VEC{y}_1 \in \RR^k$ such that 
\[
q(g) = \sum_{i=1}^k y_{1,i}
\pdfdx{(g\circ \phi_1)}{w_i}\Big|_{\VEC{w}=\phi_1^{-1}(\VEC{u})}
\]
for all $\displaystyle g \in C^\infty(S)$ is the local representation 
of $q$ with respect to the local chart $(W,U,\phi)$.  If
we work on the open subset $U = U_1 \cap U_2$ of $S$ containing
$\VEC{u}$, we have that
\begin{align*}
q(g) & = \sum_{i=1}^k y_{1,i}
\pdfdx{(g\circ \phi_1)}{w_i}\Big|_{\VEC{w}=\phi_1^{-1}(\VEC{u})}
= \sum_{i=1}^k y_{1,i}
\pdfdx{\big((g\circ \phi_2) \circ (\phi_2^{-1} \circ \phi_1)\big)}{w_i}
\Big|_{\VEC{w}=\phi_1^{-1}(\VEC{u})} \\
&= \sum_{i=1}^k y_{1,i} \left( \sum_{j=1}^k \pdfdx{(g\circ \phi_2)}{w_j}
\Big|_{\VEC{w}=\phi_2^{-1}(\VEC{u})}
\, \pdfdx{\big(\pi_j(\phi_2^{-1} \circ \phi_1)\big)}{w_i}
\Big|_{\VEC{w}=\phi_1^{-1}(\VEC{u})} \right) \\
&= \sum_{j=1}^k  \left( \sum_{i=1}^k y_{1,i}
\pdfdx{\big(\pi_j(\phi_2^{-1} \circ \phi_1)\big)}{w_i}
\Big|_{\VEC{w}=\phi_1^{-1}(\VEC{u})} \right)
\pdfdx{(g\circ \phi_2)}{w_j}\Big|_{\VEC{w}=\phi_2^{-1}(\VEC{u})} \\
&= \sum_{j=1}^k  y_{2,j}
\pdfdx{(g\circ \phi_2)}{w_j}\Big|_{\VEC{w}=\phi_2^{-1}(\VEC{u})} 
\end{align*}
for all $\displaystyle g \in C^\infty(S)$ if and only if
\begin{equation} \label{TSv2Eq2}
\VEC{y}_2 = \left( \diff (\phi_2^{-1} \circ \phi_1)
(\phi_1^{-1}(\VEC{u}))\right)  \,\VEC{y}_1 \ .
\end{equation}
The projection $\displaystyle \pi_j:\RR^k \to \RR$ defined by
$\pi_j(\VEC{w}) = w_j$ was used to obtain the
$\displaystyle j^{th}$ component of $\phi_2^{-1} \circ \phi_1$.
This relation (\ref{TSv2Eq2}) is exactly
the relation provided by $\displaystyle \phi_2^{-1} \circ \phi_1$ 
in Proposition~\ref{manifCTS}.

\subI{$\breve{\TS}_{\VEC{u}} S$ is a Vector Space over $\RR$}
If $\displaystyle q_m \in \breve{\TS}_{\VEC{u}} S$ for $m=1,2$
and
\[
q_m(g) = \sum_{i=1}^k y_{m,i} \pdfdx{(g\circ \phi)}{w_i}
\Big|_{\VEC{w}=\phi^{-1}(\VEC{u})}
\]
for $m=1,2$, then it follows from (\ref{tsdloEq1}) that
$\displaystyle q_1 + q_2 \in \breve{\TS}_{\VEC{u}} S$ and
\[
(q_1+q_2)(g) = \sum_{i=1}^k (y_{1,i} +y_{2,i})
\pdfdx{(g\circ \phi)}{w_i}\Big|_{\VEC{w}=\phi^{-1}(\VEC{u})} \ .
\]
If $\displaystyle q \in \breve{\TS}_{\VEC{u}} S$ and (\ref{TSv2Eq1}) is
satisfied, then it again follows from (\ref{tsdloEq1}) that 
$\displaystyle \lambda q \in \breve{\TS}_{\VEC{u}} S$ for all
$\lambda \in \RR$ and
\[
\lambda q(g) = \sum_{i=1}^k (\lambda y_i)
\pdfdx{(g\circ \phi)}{w_i}\Big|_{\VEC{w}=\phi^{-1}(\VEC{u})}
\]
for all $\lambda \in \RR$.

\subI{Product rule}
Let $(W,U,\phi)$ be a local chart of $S$ about $\VEC{u} \in S$.
Suppose that $\displaystyle g_1,g_2 \in C^\infty(S)$ and that
$q \in \breve{\TS}_{\VEC{u}} S$ is given by (\ref{TSv2Eq1}).
Since $g_1 g_2 \in C^\infty(S)$, we may compute
\begin{align*}
q(g_1g_2) &= \sum_{i=1}^k y_i
\pdfdx{\big((g_1\circ \phi)(g_2\circ \phi)\big)}{w_i}
\Big|_{\VEC{w}=\phi^{-1}(\VEC{u})} \\
&= \left(\sum_{i=1}^k y_i
\pdfdx{(g_1\circ \phi)}{w_i}\Big|_{\VEC{w}=\phi^{-1}(\VEC{u})}\right)
(g_2\circ \phi)(\phi^{-1}(\VEC{u})) \\
&\qquad \qquad + (g_1\circ \phi)(\phi^{-1}(\VEC{u})) \sum_{i=1}^k y_i \left(
\pdfdx{(g_2\circ \phi)}{w_i}\Big|_{\VEC{w}=\phi^{-1}(\VEC{u})} \right) \\
&= q(g_1)\, g_2(\VEC{u}) + g_1(\VEC{u})\, q(g_2) .
\end{align*}
Since this relation is independent of the local chart, it is true on
$\breve{\TS}_{\VEC{u}} S$ for all $\VEC{u} \in S$.  

It can be proved that the definition given in (\ref{TSv2Eq1}) of a
tangent vector $q$ to $S$ at $\VEC{u}$ is uniquely determined by the three
conditions: (1) $q(g_1 + g_2) = q(g_1) + q(g_2)$ for all
$\displaystyle g_1,g_2 \in C^\infty(S)$, (2) $q(\lambda g) = \lambda q(g)$ 
for all $\displaystyle g \in C^\infty(S)$ and $\lambda \in \RR$, and
(3) $q(g_1g_2) = q(g_1)\, g_2(\VEC{u}) + g_1(\VEC{u})\, q(g_2)$ for all
$\displaystyle f,g \in C^\infty(S)$ and $\VEC{u} \in S$.

\subI{Isomorphism between $\TS_{\VEC{u}} S$ and $\breve{\TS}_{\VEC{u}} S$}
We may identify $\breve{\TS}_{\VEC{u}} S$ to $\TS_{\VEC{u}} S$ by associating
$\displaystyle \VEC{y} \in \RR^k$ given by (\ref{TSv2Eq1}) 
to $(\VEC{u},\VEC{x}) = \phi_\ast(\VEC{w},\VEC{y}) \in \TS_{\VEC{u}} S$.

Suppose that $(W,U,\phi)$ is a local chart of a $k$-dimensional
manifold $S$ with $\VEC{u} \in U$, the tangent vectors to $S$ at
$\VEC{u}$ with the local representation (\ref{TSv2Eq1}) given by
$\displaystyle \VEC{y} = \VEC{e}_j \in \RR^k$ for $1 \leq j \leq k$ are
denoted $\displaystyle \pdydx{}{w_j}\Big|_{\VEC{w} = \phi^{-1}(\VEC{u})}$.
It follows
that $\displaystyle \left\{ \pdydx{}{w_j}\Big|_{\VEC{w} = \phi^{-1}(\VEC{u})}
\right\}_{1\leq j \leq k}$ is a basis of $\breve{\TS}_{\VEC{u}} S$.

To learn more about this definition of a tangent plane to a manifold
at a point using the approach in this subsection, the reader should
consult \cite{ST,Sv1}.  We will make use of this notation when
studying Riemann manifolds in Chapter~\ref{ChapRGeom} where this
notation proves to be more convenient.

\section{Vector Fields}

\begin{defn}
A {\bfseries vector field}\index{Vector Field} on a $k$-dimensional
manifold $\displaystyle S \subset \RR^n$ of class $\displaystyle C^1$ 
is a map
$\displaystyle F : S \to \TS\, S = \bigcup_{\VEC{u}\in S} \TS_{\VEC{u}} S$
defined by $F(\VEC{u}) = (\VEC{u}, f(\VEC{u})) \in \TS_{\VEC{u}} S$
for all $\VEC{u} \in S$ where $f:S \to \RR^n$.
\end{defn}

Let $(W,U,\phi)$ be a local chart of $S$.  There exists a unique
vector field
$\displaystyle \tilde{F}:W \to \bigcup_{\VEC{w} \in W} \TS_{\VEC{w}} W$
such that $\phi_\ast (\tilde{F}(\VEC{w})) = F(\phi(\VEC{w}))$ for all
$\VEC{w}\in W$.  Moreover, if
$\tilde{F}(\VEC{w}) = (\VEC{w}, \tilde{f}(\VEC{w})) \in \TS_{\VEC{w}} W$
for all $\VEC{w} \in W$, then $\tilde{f}$ satisfies
$\displaystyle (\diff \phi(\VEC{w})) \big(\tilde{f}(\VEC{w})\big)
= f(\phi(\VEC{w}))$ for all $\VEC{w} \in W$.  An equivalent statement
is that $\displaystyle f(\VEC{u})
= \left(\diff_{\VEC{w}} \phi\big(\phi^{-1}(\VEC{u})\big)\right)
\tilde{f}\big(\phi^{-1}(\VEC{u})\big)$ for all $\VEC{u} \in U$.
We say that $F$ is of class $\displaystyle C^j$ on
$S$ if $\tilde{F}$ is of class $\displaystyle C^j$ on $W$ for all
local charts $(W,U,\phi)$ of $S$.

The theory of ordinary differential equations can be applied to vector
fields on manifolds.

\begin{defn}  \label{defnIntCurvV1}
Suppose that $F$ is a vector field on a $k$-dimensional manifold $S$
of class $\displaystyle C^1$.
An {\bfseries integral curve}\index{Integral Curve} of $F$ at
$\VEC{u} \in S$ is a function $\sigma:I \to S$ where $I$ is an open
neighbourhood of the origin such that $\sigma(0) = \VEC{u}$ and
$\sigma_\ast(t,1) = F(\sigma(t))$ for all $t \in I$.
\end{defn}

Suppose that $(W,U,\phi)$ is a local chart of the $k$-dimensional
manifold $S$ about $\VEC{u} \in S$.  Suppose that
$\tilde{\sigma}:I \to W$ is the local representation of
$\sigma:I \to S$; namely,
$\displaystyle \tilde{\sigma} = \phi^{-1} \circ \sigma$
and so $\sigma = \phi \circ \tilde{\sigma}$.
Since
\[
\phi_\ast\big(\tilde{\sigma}_\ast(t,1)\big)
=\sigma_\ast(t,1) = F(\sigma(t))
=\phi_\ast\big(\tilde{F}\big(\phi^{-1}(\sigma(t))\big)\big)
=\phi_\ast\big(\tilde{F}\big(\tilde{\sigma}(t)\big)
\]
for all $t \in I$, we get the following local representation of
$\sigma_\ast(t,1) = F(\sigma(t))$.
\[
\big(\tilde{\sigma}(t), \tilde{\sigma}'(t) \big) = \tilde{\sigma}_\ast(t,1)
= \tilde{F}(\tilde{\sigma}(t)) = \big( \tilde{\sigma}(t),
\tilde{f}(\tilde{\sigma}(t)) \big)
\]
for all $t \in I$.
Therefore, solving $\sigma_\ast(t,1) = F(\sigma(t))$ for $t$ near $0$
is equivalent to solving
$\tilde{\sigma}'(t) = \tilde{f}(\tilde{\sigma}(t))$ for $t$ near $0$.
We have reduce the problem to an ordinary differential equation in
$\RR^k$.

Suppose that $g:S_1 \to S_2$ is a diffeomorphism
between two $k$-dimensional manifolds of class $\displaystyle C^1$.  The
{\bfseries pull-back}\index{Pull-Back} of a vector field
$F:S_2 \to \TS\,S_2$ is the vector field on $S_1$ defined by
$g^\ast(F)(\VEC{u}) = (g^{-1})_\ast(F(g(\VEC{u})))$ for all
$\VEC{u} \in S_1$.

For the readers who would like to learn more about vector fields and
differential equations, they can find a full chapter on these fabulous
topics in \cite{A,Sv1}.

\begin{rmk}
Suppose that $\displaystyle F : S \to \TS S$ is    \label{rmkVFwithOper}
a vector field on a $k$-dimensional manifold
$\displaystyle S \subset \RR^n$ of class $\displaystyle C^1$, and that
$(W,U,\phi)$ is a local chart of $S$.  The local representation of the
vector field $F$ with respect to the description of tangent spaces
given in Subsection~\ref{secTSasDiffop} is given by
\[
\tilde{F}(\VEC{w}) = \sum_{i=1}^k \tilde{f}_i(\VEC{w})
\pdydx{}{w_i}\Big|_{\VEC{w}}
\]
for $\VEC{w} \in W$ where $\tilde{f}_j:W \to \RR$ for $1\leq i \leq k$.
The vector field 
$F$ is of class $\displaystyle C^j$ on $S$ if, for each local chart
$(W,U,\phi)$, the function $\tilde{f}_i :W \to \RR$ are of class
$\displaystyle C^j$.

For all $\displaystyle g \in C^\infty(S)$, we have that
\[
F(\VEC{u})(g) = \tilde{F}(\phi^{-1}(\VEC{u}))(g\circ \phi)
=\sum_{i=1}^k \tilde{f}_i(\phi^{-1}(\VEC{u}))
\pdfdx{(g\circ \phi)}{w_i}\Big|_{\VEC{w}=\phi^{-1}(\VEC{u})}
\]
for all $\VEC{u} \in U$.
\end{rmk}

\section{Differential Forms} \label{sectVFandDF}

\begin{defn}
A {\bfseries differential $\mathbf p$-form}\index{Differential
Form!$\mathbf p$-form} $\omega$ on a $k$-dimensional manifold $S$ 
of class $\displaystyle C^1$ is a map
$\displaystyle \omega: S \to \Omega^p(S) =
\bigcup_{\VEC{u}\in S} \Omega^p\left(\TS_{\VEC{u}} S \right)$ defined by
$\displaystyle \omega(\VEC{u}) \in \Omega^p\left( \TS_{\VEC{u}} S \right)$
for all $\VEC{u} \in S$.
\end{defn}

Let $\omega$ be a differential $p$-form on a $k$-dimensional manifold
$S$ of class $\displaystyle C^1$.  It is important to note that
$\displaystyle \omega(\VEC{u}) \in \Omega^p\left( \TS_{\VEC{u}} S \right)$
for each $\VEC{u} \in S$; namely, for each $\VEC{u} \in S$,
$\omega(\VEC{u})$ is an alternating $p$-tensor on the k-dimensional
subspace $\TS_{\VEC{u}} S$ of $\TS_{\VEC{u}} \RR^n \cong \RR^n$.

If $(W,U,\phi)$ is a local chart of $S$, then
$\displaystyle \phi^\ast (\omega)$ is a differential $p$-form on $W$
defined by
\begin{align*}
&\phi^\ast(\omega)(\VEC{w})\big( (\VEC{w},\VEC{y}_1),
\ldots, (\VEC{w},\VEC{y}_p) \big)
=\phi^\ast(\omega(\phi(\VEC{w})))\big( (\VEC{w},\VEC{y}_1),
\ldots, (\VEC{w},\VEC{y}_p) \big) \\
&\qquad = \omega(\phi(\VEC{w}))\big( \phi_\ast(\VEC{w},\VEC{y}_1),
\ldots, \phi_\ast(\VEC{w},\VEC{y}_p) \big) \\
&\qquad = \omega(\phi(\VEC{w}))\big(
(\phi(\VEC{w}), (\diff \phi(\VEC{w}))\VEC{y}_1),
\ldots, (\phi(\VEC{w}), (\diff \phi(\VEC{w}))\VEC{y}_p) \big)
\end{align*}
for all $\VEC{w} \in W$ and $(\VEC{w},\VEC{y}_i) \in \TS_{\VEC{w}} W$
with $1\leq i \leq p$.  Thus
$\displaystyle \phi^\ast (\omega) : W \to \bigcup_{\VEC{w}\in W}
\Omega^p\left( \TS_{\VEC{w}} W \right)$ and
$\displaystyle \phi^\ast(\omega)(\VEC{w})
\in \Omega^p\left( \TS_{\VEC{w}} W \right)$
for all $\VEC{w} \in W$ as expected.

\begin{rmk}
If $S$ is a $k$-dimensional manifold of class $\displaystyle C^1$,
then the set $\displaystyle \TS^\ast S
= \bigcup_{\VEC{u}\in S} \left(\TS_{\VEC{u}} S\right)^\ast
= \bigcup_{\VEC{u}\in S} \Omega^1\left(\TS_{\VEC{u}} S \right)$
is called the
{\bfseries cotangent bundle of $\mathbf{S}$}\index{Cotangent Bundle}.
As the tangent bundle $\TS S$, it is has the structure of a vector
bundle.
\end{rmk}

\begin{defn}
We say that a differential $p$-form $\omega$ is of class
$\displaystyle C^j$ on a
$k$-dimensional manifold $S$ of class $\displaystyle C^j$
if, for every local chart $(W,U,\phi)$ of $S$, the
differential $p$-form $\displaystyle \phi^\ast(\omega)$ is of class
$\displaystyle C^j$ on $W$.
\end{defn}

More explicitly, let $\omega$ be a differential $p$-form on a
$k$-dimensional manifold $S$ of class $\displaystyle C^j$ and
$(W,U,\phi)$ be a local chart of $S$.  We have seen in
Section~\ref{stokesDefDiffF} that we can express
$\displaystyle \phi^\ast(\omega)$ as
\[
\phi^\ast(\omega) = \sum_{1 \leq i_1< i_2< \ldots< i_p\leq k}
\omega_{i_1,i_2,\ldots,i_p}
\df{w_{i_1}} \wedge \df{w_{i_2}} \wedge \ldots \wedge \df{w_{i_p}}
\]
where $\omega_{i_1,i_2,\ldots,i_p}: W \to \RR$ for all
$1 \leq i_1< i_2< \ldots< i_p\leq k$.  If, for all local
chart $(W,U,\phi)$, the functions $\omega_{i_1,i_2,\ldots,i_p}: W \to \RR$
are of class $\displaystyle C^j$, then $\omega$ is a differential $p$-form 
of class $\displaystyle C^j$ on $S$.

Suppose that $S_1$ and $S_2$ are two manifolds of class $\displaystyle C^1$ 
and that $f :S_1 \to S_2$ is a function of class $\displaystyle C^1$.
Let $\omega$ is a differential $p$-form on $S_2$.  As expected, the
{\bfseries pull-back}\index{Pull-Back} of $\omega$ by $f$ is defined
by 
\begin{align*}
&f^\ast(\omega)(\VEC{u}_1)\big( (\VEC{u}_1,\VEC{x}_1),
\ldots, (\VEC{u}_1,\VEC{x}_p) \big)
=f^\ast(\omega(f(\VEC{u}_1)))\big( (\VEC{u}_1,\VEC{x}_1),
\ldots, (\VEC{u}_1,\VEC{x}_p) \big) \\
&\qquad = \omega(f(\VEC{u}_1))\big( f_\ast(\VEC{u}_1,\VEC{x}_1),
\ldots, f_\ast(\VEC{u}_1,\VEC{x}_p) \big)
\end{align*}
for all $\VEC{u}_1 \in S_1$ and
$(\VEC{u}_1,\VEC{x}_i) \in \TS_{\VEC{u}_1} S_1$, where $f_\ast$ is
defined in the paragraph following Theorem~\ref{theofSbS}.

The properties that we have listed for differential forms on open
subsets of $\displaystyle \RR^n$ in Section~\ref{stokesDefDiffF}
are also true for differential forms on manifolds.   The proofs are
generally almost identical since they are often reduced to computation
with alternating tensors on the tangent space at each point of the
manifold.  For instance, we have the following result that generalize
(4) of Proposition~\ref{propAstoR} to manifolds.

\begin{prop} \label{propfwewf}
Suppose that $S_1$ and $S_2$ are two manifolds of
class $\displaystyle C^1$ and that $f:S_1 \to S_2$ is a function of
class $\displaystyle C^1$.  If $\omega_i$ is a differential $p_i$-form
on $S_2$ for $i = 1,2$, then
$f^\ast(\omega_1 \wedge \omega_2) = f^\ast(\omega_1) \wedge f^\ast(\omega_2)$.
\end{prop}

\begin{proof}
We have that
\begin{align*}
&f^\ast(\omega_1 \wedge\omega_2)(\VEC{u})\big( (\VEC{u},\VEC{x}_1),
\ldots, (\VEC{u},\VEC{x}_{p_1+p_2}) \big) \\
% &\qquad =f^\ast\big(\omega_1(f(\VEC{u}))\wedge \omega_2(f(\VEC{u}))\big)
% \big( (\VEC{u},\VEC{x}_1), \ldots, (\VEC{u},\VEC{x}_{p_1+p_2}) \big) \\
&\qquad = \big(\omega_1(f(\VEC{u})) \wedge \omega_2(f(\VEC{u}))\big)
\big( f_\ast(\VEC{u},\VEC{x}_1), \ldots,
f_\ast(\VEC{u},\VEC{x}_{p_1+p_2}) \big)\\
&\qquad = \frac{1}{p_1!\,p_2!} \sum_{\sigma \in S_{p_1+p_2}}
\big(\omega_1(f(\VEC{u})) \otimes \omega_2(f(\VEC{u}))\big)
\big( f_\ast(\VEC{u},\VEC{x}_{\sigma(1)}), \ldots,
f_\ast(\VEC{u},\VEC{x}_{\sigma(p_1+p_2)}) \big)\\
&\qquad = \frac{1}{p_1!\,p_2!} \sum_{\sigma \in S_{p_1+p_2}}
\omega_1(f(\VEC{u})) \big( f_\ast(\VEC{u},\VEC{x}_{\sigma(1)}), \ldots,
f_\ast(\VEC{u},\VEC{x}_{\sigma(p_1)}) \big) \otimes \\
&\hspace{17em}  \omega_2(f(\VEC{u}))\big)
\big( f_\ast(\VEC{u},\VEC{x}_{\sigma(p_1+1)}), \ldots,
f_\ast(\VEC{u},\VEC{x}_{\sigma(p_1+p_2)}) \big)\\
&\qquad = \frac{1}{p_1!\,p_2!} \sum_{\sigma \in S_{p_1+p_2}}
(f^\ast(\omega_1))(\VEC{u}) \big((\VEC{u},\VEC{x}_{\sigma(1)}), \ldots,
(\VEC{u},\VEC{x}_{\sigma(p_1)}) \big) \otimes \\
&\hspace{17em} (f^\ast(\omega_2))(\VEC{u})
\big( (\VEC{u},\VEC{x}_{\sigma(p_1+1)}), \ldots,
(\VEC{u},\VEC{x}_{\sigma(p_1+p_2)}) \big)\\
&\qquad = \big( f^\ast(\omega_1)\wedge f^\ast(\omega_2) \big)(\VEC{u})
\big( (\VEC{u},\VEC{x}_1), \ldots, (\VEC{u},\VEC{x}_{p_1+p_2}) \big)
\end{align*}
for all $\VEC{u} \in S_1$ and
$(\VEC{u},\VEC{x}_i) \in \TS_{\VEC{u}} S_1$ with $1\leq i \leq p_1+p_2$.
\end{proof}

If $\omega$ is a differential $p$-form is of class $\displaystyle C^j$
on a $k$-dimensional manifold $S$ of class $\displaystyle C^j$,
how are we going to define $\df{\omega}$?  We cannot simply derive the
$\omega$ as we did for differential $p$-forms on $\displaystyle \RR^n$ because
$\omega$ is only defined on $S$ which is not an open subset of
$\displaystyle \RR^n$ when $k<n$.  The following 
theorem will answer this question.

\begin{theorem}\label{manifdOOd}
Let $\omega$ be a $p$-form of class $\displaystyle C^1$ on a
$k$-dimensional manifold $S$ of class $\displaystyle C^1$.
There exists a unique differential $(p+1)$-form $\mu$ on $S$ such that
$\displaystyle \phi^\ast(\mu) = \df{(\phi^\ast(\omega))}$ on $W$ for
every local chart $(W,U,\phi)$ of $S$.  This unique differential
$(p+1)$-form is denoted $\df{\omega}$.
\end{theorem}

\begin{proof}
Let $(W,U,\phi)$ be a local chart of $S$.  The differential $(p+1)$-form
$\mu$ on $S$ is defined locally by
\begin{align*}
&\mu\big|_U (\VEC{u})\big( (\VEC{u}, \VEC{x}_1), (\VEC{u}, \VEC{x}_2), \ldots,
(\VEC{u}, \VEC{x}_{p+1}) \big) \\
&\qquad = \df{(\phi^\ast(\omega))}(\VEC{w})
\big( (\VEC{w}, \VEC{y}_1), (\VEC{w}, \VEC{y}_2), \ldots,
(\VEC{w}, \VEC{y}_{p+1}) \big)
\end{align*}
for $\VEC{u} \in U$ and $(\VEC{u}, \VEC{x}_i) \in \TS_{\VEC{u}} S$,
where $\displaystyle (\VEC{u},\VEC{x}_i) = \phi_\ast(\VEC{w},\VEC{y}_i)$ for
$\VEC{w} \in W$ and $(\VEC{w}, \VEC{y}_i) \in \TS_{\VEC{w}} W$ with
$1\leq i \leq p+1$.   Since
$\phi_\ast : \TS_{\VEC{w}} W \to \TS_{\phi(\VEC{w})} S$ is a linear
isomorphism for every $\VEC{w} \in W$, there is a unique $\VEC{y}_i$ for each
$\VEC{x}_i$.

\subQ{i} The previous definition is independent of the local chart used.
Suppose that $(W_i,U_i,\phi_i)$ for $i = 1,2$ are two local charts
of $S$.  Since
$\displaystyle \phi_2^{-1}\circ \phi_1: \phi_1(U_1\cap U_2) \to
\phi_2(U_1\cap U_2)$
is a diffeomorphism, the map
$\displaystyle (\phi_2^{-1} \circ \phi_1)_\ast : \TS_{\VEC{w}} W_1 \to
\TS_{\phi_2^{-1}(\phi_1(\VEC{w}))} W_2$ is a linear isomorphism for every
$\VEC{w} \in \phi_1(U_1\cap U_2)$.  Suppose that
$\phi_1(\VEC{w}_1) = \phi_2(\VEC{w}_2) = \VEC{u} \in U_1 \cap U_2$.  Then
$\displaystyle \VEC{w}_2 = \phi_2^{-1}(\phi_1(\VEC{w}_1))$.

If $\displaystyle (\phi_1)_\ast(\VEC{w}_1,\VEC{y}_j) = (\VEC{u},\VEC{x}_j)$
for $1 \leq j \leq p+1$, then
\begin{align}
&\mu\big|_{U_1} (\VEC{u})\big( (\VEC{u}, \VEC{x}_1), (\VEC{u}, \VEC{x}_2),
\ldots, (\VEC{u}, \VEC{x}_{p+1}) \big) \nonumber \\
&\quad = \df{(\phi_1^\ast(\omega))}(\VEC{w}_1)
\big( (\VEC{w}_1, \VEC{y}_1), (\VEC{w}_1, \VEC{y}_2), \ldots,
(\VEC{w}_1, \VEC{y}_{p+1}) \big) \nonumber \\
&\quad =
\df{((\phi_2\circ \phi_2^{-1}\circ \phi_1)^\ast(\omega))}(\VEC{w}_1)
\big( (\VEC{w}_1, \VEC{y}_1), (\VEC{w}_1, \VEC{y}_2), \ldots,
(\VEC{w}_1, \VEC{y}_{p+1}) \big) \nonumber \\
&\quad = \df{((\phi_2^{-1}\circ \phi_1)^\ast(\phi_2^\ast(\omega)))}(\VEC{w}_1)
\big( (\VEC{w}_1, \VEC{y}_1), (\VEC{w}_1, \VEC{y}_2), \ldots,
(\VEC{w}_1, \VEC{y}_{p+1}) \big) \nonumber \\
&\quad = (\phi_2^{-1}\circ \phi_1)^\ast(\df{(\phi_2^\ast(\omega))})(\VEC{w}_1)
\big( (\VEC{w}_1, \VEC{y}_1), (\VEC{w}_1, \VEC{y}_2), \ldots,
(\VEC{w}_1, \VEC{y}_{p+1}) \big) \nonumber \\
&\quad = \df{(\phi_2^\ast(\omega))}(\VEC{w}_2)
\big( (\VEC{w}_2, \diff (\phi_2^{-1}\circ \phi_1)(\VEC{w}_1)\VEC{y}_1),
(\VEC{w}_2, \diff (\phi_2^{-1}\circ \phi_1)(\VEC{w}_1)\VEC{y}_2), \ldots,
\nonumber \\
&\hspace{6em}
(\VEC{w}_2, \diff (\phi_2^{-1}\circ \phi_1)(\VEC{w}_1)\VEC{y}_{p+1}) \big)
\nonumber \\
&\quad = \mu\big|_{U_2} (\VEC{u})\big( (\VEC{u}, \VEC{x}_1),
(\VEC{u},\VEC{x}_2),\ldots, (\VEC{u}, \VEC{x}_{p+1}) \big) \label{defdomegaEq1}
\end{align}
because
\begin{align*}
(\phi_2)_\ast\left(\VEC{w}_2,
\diff (\phi_2^{-1}\circ \phi_1)(\VEC{w}_1)\VEC{y}_j \right)
&= \left( \phi_2(\VEC{w}_2) , \diff(\phi_2)(\VEC{w}_2)\;
\diff (\phi_2^{-1}\circ \phi_1)(\VEC{w}_1)\VEC{y}_j \right) \\
&= \left( \VEC{u} , \diff \phi_1(\VEC{w}_1)\VEC{y}_j \right)
= \left( \VEC{u} , \VEC{x}_j \right)
\end{align*}
for $1 \leq j \leq p+1$ as required.  Note that
we have used Theorem~\ref{stokesDF} to get the fourth equality
in (\ref{defdomegaEq1}).

\subQ{ii}  Finally, for each local chart $(W,U,\phi)$ of $S$, we have
by construction that
\begin{align*}
&\phi^\ast(\mu) (\VEC{w})
\big( (\VEC{w}, \VEC{y}_1), (\VEC{w}, \VEC{y}_2), \ldots,
(\VEC{w}, \VEC{y}_{p+1}) \big) \\
&\qquad = \mu\big|_U (\phi(\VEC{w}))\big( \phi_\ast(\VEC{w}, \VEC{y}_1),
\phi_\ast(\VEC{w}, \VEC{y}_2), \ldots, \phi_\ast(\VEC{w}, \VEC{y}_{p+1})
\big) \\
&\qquad = \df{(\phi^\ast(\omega))}(\VEC{w})
\big( (\VEC{w}, \VEC{y}_1), (\VEC{w}, \VEC{y}_2), \ldots,
(\VEC{w}, \VEC{y}_{p+1}) \big)
\end{align*}
for all $\VEC{w} \in W$ and $(\VEC{w}, \VEC{y}_i) \in \TS_{\VEC{w}} W$ with
$1\leq i \leq p+1$.
Thus $\displaystyle \phi^\ast(\mu) = \df{(\phi^\ast(\omega))}$ on $W$.
\end{proof}

Since the derivative of a differential form is defined locally, we may
generalize (4) of Theorem~\ref{stokesDF}.

\begin{prop} \label{manifSDFitem4}
Suppose that $S_1$ and $S_2$ are two manifolds of
class $\displaystyle C^1$ and that $f:S_1 \to S_2$ is a function of
class $\displaystyle C^1$.
If $\omega$ be a $p$-form of class $\displaystyle C^1$ on $S_2$.
then $f^\ast (\df{\omega}) = \df{(f^\ast(\omega))}$.
\end{prop}

\begin{proof}
Let $(W_i,U_i,\phi_i)$ be a local chart of $S_i$ for $1\leq i \leq 2$.
Without loss of generality, we may assume that $f(U_1) \subset U_2$.
We have by definition of the derivative of a differential form given
in Theorem~\ref{manifdOOd} that
$\displaystyle \df{(\phi_1^\ast(f^\ast(\omega)))}
= \phi_1^\ast(\df{(f^\ast(\omega))})$ on $W_1$.  Moreover, we have that
\begin{align*}
\df{(\phi_1^\ast(f^\ast(\omega)))} &= \df{((f\circ \phi_1)^\ast(\omega))}
= \df{\big((\phi_2^{-1} \circ f\circ \phi_1)^\ast(\phi_2^\ast(\omega))\big)} \\
&= (\phi_2^{-1} \circ f\circ \phi_1)^\ast\df{(\phi_2^\ast(\omega))}
= (\phi_2^{-1} \circ f\circ \phi_1)^\ast\big(\phi_2^\ast(\df{\omega})\big) \\
&= (f\circ \phi_1)^\ast(\df{\omega}) = \phi_1^\ast(f^\ast(\df{\omega})) \ ,
\end{align*}
where the third equality comes from (4) of Theorem~\ref{stokesDF} and
the fourth equality from the definition of the derivative of a
differential form given in Theorem~\ref{manifdOOd}.
Thus $\displaystyle \phi_1^\ast(\df{(f^\ast(\omega))})
= \phi_1^\ast(f^\ast(\df{\omega}))$.  Since $(W_1,U_1,\phi_1)$ is an
arbitrary local chart of $S_1$, we get that
$\displaystyle \df{(f^\ast(\omega))} = f^\ast(\df{\omega})$.
\end{proof}

\begin{rmkList}       \label{rmkrestrictdxi}
\begin{enumerate}
\item We would like to warn the reader that a different notation
is often used in the literature on differential geometry.  Some authors
write
\[
\omega = \sum_{1 \leq i_1< i_2< \ldots< i_p\leq k} \omega_{i_1,i_2,\ldots,i_p}
\df{x_{i_1}} \wedge \df{x_{i_2}} \wedge \ldots \wedge \df{x_{i_p}} \ ,
\]
where $\displaystyle \omega_{i_1,i_2,\ldots,i_p}:S\to \RR$ as a
differential $q$-form on a $k$-dimensional manifold $S$ of class
$S^1$.  This in fact represents the local representation of the
differential $q$-form $\omega$.   Their definition of a local chart
$(W,U,\phi)$ of $S$ is slightly different than our definition.
Instead of $\phi:W\to U$, they use $\phi:U \to W$.  The $x_i$ for
$1 \leq i \leq k$ represent the local coordinates in $\displaystyle \RR^k$
instead of the $w_i$ in our definition.   Therefore, to match our
theory to their, all the local formulae that we have presented have to
be reformulated according to this new convention.
\item As we have seen in Section~\ref{stokesDefDiffF},
we may write a differential $p$-form $\omega$ on $\displaystyle \RR^n$ as
\begin{equation} \label{kfromonSeq2}
\omega = \sum_{1\leq i_1< i_2< \ldots< i_p\leq n} \omega_{i_1,i_2,\ldots,i_p}
\df{x_{i_1}} \wedge \df{x_{i_2}} \wedge \ldots \wedge \df{x_{i_p}} \ ,
\end{equation}
where $\displaystyle \omega_{i_1,i_2,\ldots,i_p}:\RR^n\to \RR$ for all
indices.  If $S$ is a $k$-dimensional manifold of class $\displaystyle C^1$
in $\displaystyle \RR^n$, it is also frequent in the literature to use
this notation to talk about $\omega\big|_S$. 
Recall that $\df{x_i}$ for $1\leq i \leq n$ are differential
$1$-form on $\RR^n$ defined by $\df{x_i}(\VEC{u}) \big( (\VEC{u},\VEC{x}) \big)
= \pi_i(\VEC{x}) = x_i$ for all $(\VEC{u},\VEC{x}) \in \TS_{\VEC{u}}\RR^n$,
$\VEC{u} \in \RR^n$ and $1\leq i \leq n$.  Since $\TS_{\VEC{u}} S$ is
a linear subspace of $\TS_{\VEC{u}} \RR^n$ for all $\VEC{u} \in S$, we
have that $\displaystyle \df{x_i}\big|_S$ is a differential $1$-form
on $S$ where $\df{x_i}(\VEC{u})\big( (\VEC{u},\VEC{x}) \big)
= \pi_i(\VEC{x}) = x_i$ for all $(\VEC{u},\VEC{x}) \in \TS_{\VEC{u}} S$,
$\VEC{u} \in S$ and $1\leq i \leq n$.  Thus, when we write
$\df{x_{i_j}}$ in (\ref{kfromonSeq2}), we really mean
$\displaystyle \df{x_{i_j}}\big|_S$.
As a warning, when $k < n$, the set
$\displaystyle \{ \df{x_{i_1}} \wedge \df{x_{i_2}} \wedge \ldots \wedge
\df{x_{i_p}} : 1 \leq i_1 < i_2 < \ldots < i_p \leq n \}$ is a set of
generators for $\displaystyle \Omega^p\left(\TS_{\VEC{u}} S\right)$ but 
not a basis.

The fact that we do not have a precise action of the differential
$1$-form $\df{x_i}$ on $S$ is not going to be an issue in the
integrals that we will consider later because differential $p$-form as in
(\ref{kfromonSeq2}) are reinterpreted in terms of the local
coordinates given by the local charts on $S$ using substitution.
\end{enumerate}
\end{rmkList}

\begin{rmk}
Suppose that $\displaystyle F : S \to \TS S$ is  \label{rmkdyVFwithOper}
a vector field on a $k$-dimensional manifold $\displaystyle S \subset \RR^n$
of class $\displaystyle C^1$.  Moreover, suppose that
$g:S \to \RR$ is a function of class $\displaystyle C^\infty$.
Then $\df{g}$ is a differential $1$-from on $S$.  We may therefore compute
$\df{g}(\VEC{u})\big(F(\VEC{u})\big)$ for all $\VEC{u} \in S$.

Suppose that $(W,U,\phi)$ is a local chart of $S$.
If we work on the local chart $(W,U,\phi)$ and assume that
$\VEC{u} = \phi(\VEC{w})$ for $\VEC{w} \in W$, we get that
\begin{align*}
\df{g}(\VEC{u})\big(F(\VEC{u})\big)
&= \df{g}(\phi(\VEC{w}))\big(\phi_\ast(\tilde{F}(\VEC{w})\big)
= \phi^\ast(\df{g})(\VEC{w})\big(\tilde{F}(\VEC{w})\big)
= \df(\phi^\ast(g))(\VEC{w})\big(\tilde{F}(\VEC{w})\big) \\
&= \df(g \circ \phi)(\VEC{w})\big(\tilde{F}(\VEC{w})\big)
= \diff (g \circ \phi) (\VEC{w}) \big(\tilde{F}(\VEC{w})\big)\\
&= \sum_{i=1}^k \tilde{f}_i(\VEC{w})\,\pdfdx{(g\circ \phi)}{w_i}\Big|_{\VEC{w}}
= \sum_{i=1}^k \tilde{f}_i\big(\phi^{-1}(\VEC{u})\big)\,
\pdfdx{(g\circ \phi)}{w_i}\Big|_{\VEC{w}=\phi^{-1}(\VEC{u})}
\end{align*}
for all $\VEC{u} \in U$, where the fifth equality comes from 
Definition~\ref{defnd01}.  With respect to the description of tangent
spaces given in Subsection~\ref{secTSasDiffop}, we have that $F$ has
the locale representation
$\displaystyle \tilde{F}(\VEC{w}) = \sum_{i=1}^k \tilde{f}_i(\VEC{w})
\pdydx{}{w_i}\Big|_{\VEC{w}}$ for $\VEC{w} \in W$.  Therefore
$\df{g}(\VEC{u})\big(F(\VEC{u})\big) = F(\VEC{u})(g)$ for all
$\VEC{u} \in U$.  The expression
$\df{g}(\VEC{u})\big(F(\VEC{u})\big)$ is called the
{\bfseries directional derivative}\index{Directional Derivative} or
{\bfseries Lie derivative}\index{Lie Derivative} of $g$ along $F$ at
$\VEC{u}$.
\end{rmk}

\section{Integration on Manifolds}

We first expand a little Definition~\ref{defStcInt}.

\begin{defn}
Let $k$ be a positive integer and
$\omega = f \df{x_1}\wedge \df{x_2}\wedge \ldots \wedge \df{x_k}$ be a
differential $k$-form defined on an open set
$\displaystyle V \subset \RR^k$.  The
{\bfseries integral of $\omega$}\index{Integral of a Differential Form}
on $V$ is defined by
$\displaystyle \int_V \omega = \int_V f(\VEC{x}) \dx{\VEC{x}}$.
\end{defn}

\begin{defn}
Let $S$ be a $k$-dimensional oriented manifold of class $\displaystyle C^1$
and $(W,U,\phi)$ be an orientation preserving local chart of $S$.
Suppose that $\omega$ is a differential $k$-form on $S$ with compact
support in $U$.  We define the
{\bfseries integral}\index{Integral of a Differential Form} of
$\omega$ on $S$ as
\[
\int_S \omega = \int_W \phi^\ast(\omega) \ .
\]
\end{defn}

The next proposition shows that this definition is independent of the 
local chart containing the support of $\omega$ that is used.

\begin{prop}
Let $V$ and $W$ be two open subsets of $H_k$ and suppose that
$g:W \to V$ is an orientation preserving diffeomorphism.  If $\omega$ is
a differential $k$-form with compact support in $V$, then
$\displaystyle g^\ast(\omega)$ is a differential $k$-form with compact
support in $W$ and
\[
\int_{V} \omega = \int_{W} g^\ast(\omega) \ .
\]
\end{prop}
 
\begin{proof}
Since $g:W \to V$ is a diffeomorphism, it follows that
$\displaystyle g^\ast(\omega)$ is a
differential $k$-form on $W$.  Since $\supp \omega$ is a compact set
and $\displaystyle g^{-1}:V \to W$ is continuous, we have that
$\displaystyle \supp g^\ast(\omega) = g^{-1}(\supp \omega)$ is a
compact set.

If $\omega = f \df{x_1}\wedge\df{x_2}\wedge\ldots\wedge\df{x_k}$, we have
\begin{align*}
\int_{W} g^\ast(\omega)
&= \int_{W} (f\circ g) (\det \diff g)
\df{x_1}\wedge\df{x_2}\wedge \ldots\wedge\df{x_k} \\
&= \int_{W} (f\circ g) (\det \diff g) \dx{x_1}\dx{x_2}\ldots \dx{x_k}
= \int_{V} f \dx{x_1}\dx{x_2}\ldots \dx{x_k} = \int_V \omega \ .
\end{align*}
The first equality is a consequence of the definition of
$\displaystyle \omega$ combined with Theorem~\ref{stokesCofV}
and the second equality is the definition of the integral of a
differential $k$-form over an open subset of $H_k$.  The third
equality comes from the change of variable theorem and the fact that
$\det \diff g >0$ on $W$ since $g$ is orientation preserving.  Finally, the
last equality is again the definition of the integral of a differential
$k$-form over an open subset of $H_k$.

Note that we used the extended definition of diffeomorphism between open
subsets of $H_k$ when $\supp \omega \cap \partial H_k \neq \emptyset$.
\end{proof}

It follows from the previous proposition that if
$(W_i,U_i,\phi_i)$ for $1\leq i \leq 2$ are two orientation
preserving local charts of a $k$-dimensional oriented manifold $S$ and 
that $\omega$ is a differential $k$-form on $S$ with
$\supp \omega \subset U_1 \cap U_2$, then
$\displaystyle \int_S \omega = \int_{W_1} \phi_1^\ast(\omega) 
= \int_{W_2} \phi_2^\ast(\omega)$.  To be more precise,
let $\tilde{W}_i = \phi_i^{-1}(U_1 \cap U_2)$ for $1=1,2$ and
$g = \phi_2^{-1}\circ \phi_1$.  Then $g$ is an orientation preserving
diffeomorphism between $\tilde{W}_1$ and $\tilde{W}_2$.  Hence
\[
\int_{W_2} \phi_2^\ast(\omega) = \int_{\tilde{W}_2} \phi_2^\ast(\omega) 
= \int_{\tilde{W}_1} g^\ast(\phi_2^\ast(\omega))
= \int_{\tilde{W}_1} (\phi_2\circ g)^\ast(\omega)
= \int_{\tilde{W}_1} \phi_1^\ast(\omega)
= \int_{W_1} \phi_1^\ast(\omega) \ .
\]

To define the integral of a differential form on a manifold, we need
the following definition.

\begin{defn}
Let $S$ be a $k$-dimensional oriented manifold of class
$\displaystyle C^1$ and
$\displaystyle \A
= \left\{ (W_\alpha,U_\alpha,\phi_\alpha)\right\}_{\alpha\in A}$ be an atlas of
orientation preserving local charts for $S$.  A
{\bfseries partition of unity}\index{Partition of Unity} 
subordinate to the atlas $\A$ is a collection
$\displaystyle \{\psi_j\}_{j\in \NNp}$ of functions of class
$\displaystyle C^\infty$ such that:
\begin{enumerate}
\item $0 \leq \psi_j(\VEC{x}) \leq 1$ for all
$\displaystyle \VEC{x} \in S$ and all $\displaystyle j\in \NNp$.
\item For each $\VEC{x}\in S$, there exists an open neighbourhood
$U_{\VEC{x}} \subset S$ of $\VEC{x}$ such that
$\displaystyle \psi_j\big|_{U_\VEC{x}} = 0$ for
all but a finite number of $\psi_j$.
\item For each $\VEC{x} \in S$,
$\displaystyle \sum_{j \in \NNp} \psi_j(\VEC{x}) = 1$.  This sum
is finite according to (2).
\item For each $\displaystyle j \in \NNp$, we have that
$\supp \psi_j$ is a compact set with
$\supp \psi_j \subset U_{\alpha_j}$ for some $\alpha_j \in A$.
\end{enumerate}
\end{defn}

It follows from Theorem~\ref{cov1} that we can always find a
partition of unity subordinate to the atlas $\A$.  It suffices to take
a partition of unity $\displaystyle \{\psi_j\}_{j\in \NNp}$ subordinate 
to the open cover $\displaystyle \{V_\alpha\}_{\alpha\in A}$ of $S$,
where $V_\alpha$ is an open set in $\RR^n$ such that
$U_\alpha = S \cap V_\alpha$ for $\alpha\in A$.

The following result is a direct consequence of Proposition~\ref{cov4}. 

\begin{prop} \label{cov4forM}
Let $S$ be a $k$-dimensional oriented manifold of class
$\displaystyle C^1$.  Let
$\displaystyle \A
= \left\{ (W_\alpha,U_\alpha,\phi_\alpha)\right\}_{\alpha\in A}$ be an
atlas of orientation preserving local charts for $S$ and
$\displaystyle \{\psi_j\}_{j \in \NNp}$ be a partition of
unity subordinate to $\A$.  If $K\subset S$ is a compact set, then
there exist only a finite number of functions $\psi_j$ such that
$\phi_j\big|_K \neq 0$.
\end{prop}

\begin{defn} \label{DefIntMan}
Let $S$ be a $k$-dimensional oriented manifold of class
$\displaystyle C^1$ and $\displaystyle \A = \left\{ (W_\alpha,U_\alpha,
\phi_\alpha)\right\}_{\alpha\in A}$ be an atlas of
orientation preserving local charts for $S$.  Let
$\displaystyle \{\psi_j\}_{j\in \NNp}$ be a partition of unity subordinate 
to $\A$.  Suppose that $\omega$ is a differential $k$-form on $S$ with
compact support.  For each $\displaystyle j\in \NNp$, the
differential $k$-forms
$\omega_j = \psi_j\, \omega$ has a compact support in $U_{\alpha_j}$ for some
$\alpha_j \in A$.  The
{\bfseries integral}\index{Integral of a Differential Form} of
$\omega$ on $S$ is defined by
\begin{equation} \label{manifIntDef}
\int_S \omega = \sum_{j\in \NNp} \int_S \omega_j = \sum_{j\in \NNp}
\int_{W_{\alpha_j}} \phi_j^\ast(\omega_j) \ .
\end{equation}
\end{defn}

The next proposition shows that the definition of the integral of a
differential $k$-form on a $k$-dimensional oriented manifold of class
$\displaystyle C^1$ is not ambiguous.

\begin{prop}
\begin{enumerate}
\item The sum (\ref{manifIntDef}) contains only a finite number of non-zero
terms.
\item The value of the sum in (\ref{manifIntDef}) is independent of the atlas
for $S$ chosen.
\end{enumerate}
\end{prop}

\begin{proof}
The proof used the information provided in Definition~\ref{DefIntMan}.

\stage{1} According to Proposition~\ref{cov4forM}, we have that
$\displaystyle \omega_j = \psi_j\, \omega = 0$ for all
$\displaystyle j \in \NNp$ but a finite number of
$\displaystyle j \in \NNp$ because $\omega$ has a compact support.

\stage{2} Suppose that
$\displaystyle \A
= \big\{ (W_\alpha,U_\alpha,\phi_\alpha)\big\}_{\alpha\in A}$ and
$\displaystyle \tilde{\A} =
\big\{ (\tilde{W}_{\tilde{\alpha}},\tilde{U}_{\tilde{\alpha}},
\tilde{\phi}_{\tilde{\alpha}})\big\}_{\tilde{\alpha}\in \tilde{A}}$
are two atlases of orientation preserving local charts for $S$.
Let $\displaystyle \{\psi_j\}_{j\in \NNp}$ be a partition of unity subordinate 
to $\A$ and $\displaystyle \{\tilde{\psi}_j\}_{j\in \NNp}$ be a partition of
unity subordinate to $\tilde{\A}$.  Then
\begin{align*}
\sum_{j_1\in \NNp} \int_S \psi_{j_1} \omega
&= \sum_{j_1\in \NNp} \int_S \psi_{j_1} \left(\sum_{j_2\in \NNp}
\tilde{\psi}_{j_2}\right) \omega
= \sum_{j_1\in \NNp} \int_S
\left(\sum_{j_2\in \NNp} \tilde{\psi}_{j_2} ( \psi_{j_1} \omega) \right) \\
&= \sum_{j_1\in \NNp} \sum_{j_2\in \NNp} \int_S \tilde{\psi}_{j_2}
\psi_{j_1} \omega
= \sum_{j_2\in \NNp} \sum_{j_1\in \NNp} \int_S \tilde{\psi}_{j_2} \psi_{j_1} \omega
= \sum_{j_2\in \NNp} \int_S \left( \sum_{j_1\in \NNp} \tilde{\psi}_{j_2}
\psi_{j_1} \omega \right) \\
&= \sum_{j_2\in \NNp} \int_S \tilde{\psi}_{j_2}
\left( \sum_{j_1\in \NNp} \psi_{j_1}\right) \omega
= \sum_{j_2\in \NNp} \int_S \tilde{\psi}_{j_2} \omega \ .
\end{align*}
The first equality comes from
$\displaystyle \sum_{j_2\in \NNp} \tilde{\psi}_{j_2}(\VEC{x}) = 1$ for
all $\VEC{x}$.
Since the summation with respect to $j_2$ is finite because of (1), we may pull
the summation with respect to $j_2$ out of the integral to get the third
equality.  Because
$\supp (\psi_{j_1}\tilde{\psi}_{j_2}) \cap \supp \omega \neq \emptyset$
for only a finite number of $\psi_{j_1}\tilde{\psi}_{j_2}$, the
summations may be inverted to give the fourth equality.  Since the
summation with respect to $j_1$ is finite 
because of (1),  we may bring the summation with respect to $j_1$ in the
integral to get the fifth equality.  The last equality comes from
$\displaystyle \sum_{j_1\in \NNp} \psi_{j_1}(\VEC{x}) = 1$ for all $\VEC{x}$.
\end{proof}

There is a change of variable formula for integration of differential
forms on manifolds.

\begin{prop}  \label{propCVforDF}
Let $S$ and $\tilde{S}$ be two $k$-dimensional manifolds of class
$\displaystyle C^1$.  Suppose that $\omega$ is a differential $k$-form
on $\tilde{S}$ with compact support and that $g :S \to \tilde{S}$ is
an orientation preserving diffeomorphism.  Then
$\displaystyle \int_{\tilde{S}} \omega = \int_S g^\ast(\omega)$.
\end{prop}

\begin{proof}
Let $K = \supp\, \omega$.
Then $\displaystyle \supp g^\ast(\omega) = g^{-1}(K)$.
This is a compact set because $\displaystyle g^{-1}$ is continuous and
the image of a compact set by a continuous function is a compact set.
Therefore, our definition of integral of differential forms on manifolds can
be used for both $\omega$ and $\displaystyle g^\ast(\omega)$.

Let $\displaystyle \A
= \left\{ (W_\alpha,U_\alpha,\phi_\alpha)\right\}_{\alpha\in A}$ be
an atlas of orientation preserving local charts for $S$, and
$\displaystyle \{\psi_j\}_{j\in \NNp}$ be a partition of unity subordinate 
to $\A$.  Then $\displaystyle \B
= \left\{ (W_\alpha,g(U_\alpha),g\circ \phi_\alpha)\right\}_{\alpha\in A}$ is
an atlas of orientation preserving local charts for $\tilde{S}$ and
$\displaystyle \{\psi_j \circ g^{-1}\}_{j\in \NNp}$ is a partition of
unity subordinate to $\B$.
If $\alpha_j \in A$ is an index such that
$\supp \psi_j \subset U_{\alpha_j}$, then
$\displaystyle \supp (\psi_j \circ g^{-1}) \subset g(U_{\alpha_j})$.
Hence
\begin{align*}
\int_S g^\ast(\omega)
&= \sum_{j\in \NNp} \int_{U_{\alpha_j}} \psi_j g^\ast(\omega)
= \sum_{j\in \NNp} \int_{W_{\alpha_j}} \phi_{\alpha_j}^\ast(\psi_j g^\ast(\omega))
= \sum_{j\in \NNp} \int_{W_{\alpha_j}} (\psi_j\circ\phi_{\alpha_j})\,
\phi_{\alpha_j}^\ast(g^\ast(\omega)) \\
&= \sum_{j\in \NNp} \int_{W_{\alpha_j}}
\big( \psi_j \circ g^{-1} \circ (g \circ\phi_{\alpha_j})\big) \,
(g\circ \phi_{\alpha_j})^\ast(\omega)
= \sum_{j\in \NNp} \int_{W_{\alpha_j}}
(g \circ \phi_{\alpha_j})^\ast \big( (\psi_j \circ g^{-1})\, \omega\big) \\
&= \sum_{j\in \NNp} \int_{g(U_{\alpha_j})} (\psi_j \circ g^{-1})\, \omega
= \int_{\tilde{S}} \omega \ .  \qedhere
\end{align*}
\end{proof}

Before stating and proving the main theorem of this chapter, we need
to clarify some technical details.

\begin{lemma} \label{stokesLem1}
Suppose that $\upsilon$ is a differential $k$-form on an open set
$W \subset H_k$ and $\supp \upsilon$ is a compact subset of $W$.
There exists a finite collection
$\displaystyle \left\{ \sigma_m \right\}_{1 \leq m \leq M}$
of singular $k$-cubes of class $\displaystyle C^1$ such that:
\begin{enumerate}
\item $\sigma_m(I_k)$ is a closed cube in $W$ with sides parallel to
the axes in $H_k$, 
\item $\displaystyle \sigma_{m_1}(I_k) \cap
\sigma_{m_2}(I_k)$ for $m_1 \neq m_2$ is either the empty set or a
common side of these two closed cubes (namely, 
$(\sigma_{m_1})_{j,1}(I_{k-1}) = (\sigma_{m_2})_{j,0}(I_{k-1})$ or
$(\sigma_{m_1})_{j,0}(I_{k-1}) = (\sigma_{m_2})_{j,1}(I_{k-1})$ for some
$1 \leq j \leq k$), and
\item
{
\renewcommand{\labelenumii}{(\alph{enumii})}
\begin{enumerate}
\item $\displaystyle \supp \upsilon \subset
\left( \bigcup_{m=1}^M \sigma_m(I_k) \right)^\circ$ if
$\supp \upsilon \cap \partial H_k = \emptyset$ or
\item $\displaystyle \supp \upsilon \subset \bigcup_{m=1}^M \sigma_m(I_k)$ and
$\displaystyle \supp \upsilon \cap
\partial \left(\bigcup_{m=1}^M \sigma_m(I_k)\right)
\subset \partial H_k$ (the topological boundaries) if
$\supp \upsilon \cap \partial H_k \neq \emptyset$.
\end{enumerate}
}
\end{enumerate}
\end{lemma}

\begin{proof}
Since $\supp \upsilon$ is a compact subset of the open set
$W$ of $H_k$ with the induced topology from $\displaystyle \RR^k$, there
exists an open set $V \subset H_k$ such that
$\supp \upsilon \subset V \subset \overline{V} \subset W$.
For each $\VEC{w} \in \supp \upsilon$, choose an open cube
$B_{\VEC{w}} \subset V$ in $H_k$ such that $\VEC{w} \in U_{\VEC{w}}$.
The collection $\displaystyle \{ B_{\VEC{w}} \}_{\VEC{w} \in \supp \upsilon}$ is an
open cover of the compact set $\supp \upsilon$.  Therefore, there exists a
finite subcover $\displaystyle \{ B_{\VEC{w}_j} \}_{1 \leq j \leq J}$ of
$\supp \upsilon$.  By subdividing the cubes 
$\displaystyle \left\{\,\overline{B_{\VEC{w}_j}}\, \right\}_{1 \leq j \leq J}$,
we may get a finite collections of closed cubes
$\displaystyle \{ B_m\}_{1 \leq m \leq M}$ such that
$\displaystyle \supp \upsilon \subset \bigcup_{m=1}^m B_m \subset
\overline{V} \subset W$ and
$\displaystyle B_{m_1} \cap B_{m_2}$ for $m_1 \neq m_2$ is either the
empty set or a common side of these two closed cubes
(Figure1~\ref{Cube1} and \ref{Cube2}).
Each cube $B_m$ is the image of an affine transformation
$\sigma_m:I_k \to B_m$.  These $\sigma_m$  are obviously
singular $k$-cubes of class $\displaystyle C^1$.
\end{proof}

\pdfF{manifolds/cube1}{Refining a covering by open cubes}{
The collection of open boxes $\{ B_{\VEC{w}_i}\}_{1\leq i \leq 4}$ on the left 
cover $\supp \upsilon$.  These boxes can be subdivided to give the singular
$k$-cubes $\{ \sigma_i\}_{1\leq i \leq 9}$ on the right whose images cover
$\supp \upsilon$.}{Cube1}

\pdfF{manifolds/cube2}{Covering of the support of a differential
$k$-form by singular $k$-cubes}{Covering of the support of the
differential $k$-form $\upsilon$ defined on $W$ by
singular $k$-cubes whose sides are parallel to the axes in $H_k$.}{Cube2}

\begin{lemma} \label{stokesLem2}
Let $S$ be an orientable $k$-dimensional manifold of class
$\displaystyle C^1$ with boundary.  Suppose that $\mu_{\VEC{u}}$ is
the chosen orientation on $\TS_{\VEC{u}} S$ for every
$\VEC{u} \in S$, that $(W,U,\phi)$ is an orientation preserving local
chart on $S$ such that $U \cap \partial S \neq \emptyset$ and that
$\omega$ is a differential $(k-1)$-form on $S$ with
compact support in $U$ such that
$\supp \omega \cap \partial S \neq \emptyset$ (Figure~\ref{Cube3}).
If $\displaystyle \left\{ \sigma_m \right\}_{1 \leq m \leq M}$ is a finite
collection of singular $k$-cubes satisfying (3(b))
of Lemma~\ref{stokesLem1} with $\upsilon = \phi^\ast(\omega)$, then
$\displaystyle \int_{\partial S} \omega = \int_{\partial \sigma}
\phi^\ast(\omega)$.
\end{lemma}

\begin{proof}
Since $\supp \phi^\ast (\omega)$ is a compact subset of $W$,
we may effectively apply Lemma~\ref{stokesLem1} to get the finite
collection $\displaystyle \left\{ \sigma_m \right\}_{1 \leq m \leq M}$.

We have that
$\displaystyle \sigma = \sum_{m=1}^M \sigma_m$ is a $k$-chain in $W$
whose image contains the support of $\phi^\ast(\omega)$ and such that the
image of the $(k-1)$-chain
$\displaystyle \sum_{\sigma_m(I_k) \cap \partial H_k\neq \emptyset}
(\sigma_m)_{k,0}$ contains the support of the differential $(k-1)$-form
$\displaystyle \phi^\ast \left(\omega\big|_{U\cap \partial S}\right)
= \phi^\ast(\omega)\big|_{W\cap \partial H_k}$ on
$W \cap \partial H_k$.

Since $\phi$ is orientation preserving, we have that
\begin{align*}
\mu_{\VEC{u}} &=
[ \phi_\ast(\VEC{w},-\VEC{e}_k) ,\phi_\ast(\VEC{w},\VEC{e}_1),
\phi_\ast(\VEC{w},\VEC{e}_2), \ldots,
\phi_\ast(\VEC{w},\VEC{e}_{k-1})] \\
&= (-1)^k [\phi_\ast(\VEC{w},\VEC{e}_1), \phi_\ast(\VEC{w},\VEC{e}_2), \ldots, 
\phi_\ast(\VEC{w},\VEC{e}_{k-1}), \phi_\ast(\VEC{w},\VEC{e}_k)]
\end{align*}
for $\VEC{u} \in U \cap \partial S$ and $\VEC{w} \in W \cap \partial H_k$
such that $\VEC{u} = \phi(\VEC{w})$.  Therefore, the induce
orientation on $\partial S$ differs from the orientation provided by
the standard orientation on $H_k$ by a factor of $\displaystyle (-1)^k$.
The two orientations match when $k$ is even and are of opposite signs
when $k$ is odd.  See also Remark~\ref{rmkHkOrient}.  Thus, we have that
\[
\int_{\partial S} \omega = \int_{U \cap \partial S} \omega  
= (-1)^k \int_{W \cap \partial H_k} \phi^\ast(\omega)
= (-1)^k \sum_{\sigma_m(I_k) \cap \partial H_k \neq \emptyset}
\int_{(\sigma_m)_{k,0}} \phi^\ast(\omega) \ .
\]

Since $\displaystyle \partial \sigma =
\sum_{m=1}^M \left( \sum_{i=1}^k (-1)^i \big( (\sigma_m)_{i,0} 
- (\sigma_m)_{i,1}\big) \right)$, we get that
\begin{align*}
\int_{\partial S} \omega
&= \sum_{\sigma_m(I_k) \cap \partial H_k\neq \emptyset}
(-1)^k \int_{(\sigma_m)_{k,0}} \phi^\ast(\omega)
= \sum_{\sigma_m(I_k) \cap \partial H_k \neq \emptyset}
\int_{(-1)^k (\sigma_m)_{k,0}} \phi^\ast(\omega) \\
&= \sum_{m=1}^M \sum_{i=1}^k \left(
\int_{(-1)^i (\sigma_m)_{i,0}} \phi^\ast(\omega)
-\int_{(-1)^i (\sigma_m)_{i,1}} \phi^\ast(\omega)\right)
= \int_{\partial \sigma} \phi^\ast(\omega) \ ,
\end{align*}
where we have used the fact that
$\displaystyle \int_{(\sigma_m)_{i,j}} \phi^\ast(\omega) = 0$ if
$\displaystyle \supp \phi^\ast(\omega) \cap
(\sigma_m)_{i,j}(I_{k-1}^\circ) = \emptyset$ with
$j=0,1$ and $1\leq i \leq k$,
and $\displaystyle \int_{(\sigma_{m_1})_{i,1}}
\phi^\ast(\omega) = -\displaystyle \int_{(\sigma_{m_2})_{i,0}} \phi^\ast(\omega)$
if $(\sigma_{m_1})_{i,1}(I_{k-1}) = (\sigma_{m_2})_{i,0}(I_{k-1})$
for some $m_1\neq m_2$ and $1 \leq i \leq k$.
\end{proof}

\pdfF{manifolds/cube3}{Integration along a chain on a manifold}
{Integration along a chain on a manifold}{Cube3}

The following theorem is the general form of the fundamental theorem of
calculus.

\begin{theorem}[Stokes' Theorem] \label{TheStokesTh}
Suppose that $S$ is a compact and oriented $k$-dimensional manifold of
class $\displaystyle C^1$, and $\omega$ is a differential $(k-1)$-form
on $S$.  Then,
\begin{equation}\label{TheStokesTHEqu}
\int_S \df{\omega} = \int_{\partial S} \omega \ .
\end{equation}
\end{theorem}

\begin{proof}
Let $\displaystyle \A = \{(W_\alpha, U_\alpha, \phi_\alpha)\}_{\alpha\in A}$
be an atlas of orientation preserving local charts for $S$.
Since $S$ is compact, we can find a finite open cover
$\displaystyle \left\{ U_{\alpha_i} \right\}_{1 \leq i \leq I}$ of $S$.
Therefore\\
$\displaystyle \A_I = \{(W_{\alpha_i}, U_{\alpha_i}, \phi_{\alpha_i})
\}_{1\leq i \leq I}$
is a finite atlas of orientation preserving local charts for $S$.  Let
$\displaystyle \left\{\psi_i\right\}_{1 \leq i \leq I}$ be a
partition of unity subordinate to $\A_I$ such that
$\supp \psi_i \subset V_{\alpha_i}$ for $1 \leq i \leq i$ (see
Remark~\ref{rmkFPUcase}).

Let $\omega_i = \psi_i \omega$ for $1 \leq i \leq I$.  Then
\[
\int_S \omega = \sum_{i=1}^I \int_S \omega_i  \ .
\]
There are two cases to consider.

\subQ{i} If $(\supp \omega_i) \cap \partial U_{\alpha_i} = \emptyset$
(topological neighbourhood of $U_{\alpha_i}$), there
exists a finite number of orientation preserving singular $k$-cubes
satisfying (3(a)) of Lemma~\ref{stokesLem1} with $U$ and $W$ replaced
by $U_{\alpha_i}$ and $W_{\alpha_i}$ respectively, and
$\displaystyle \upsilon = \phi_i^\ast(\omega_i)$.
Consider the $k$-chain $\displaystyle \sigma = \sum_{m=1}^M \sigma_m$ 
in $W_{\alpha_i}$.  It follows from Theorems~\ref{stokesStokes} that
\[
\int_S \df{\omega_i} = \int_{U_{\alpha_i}} \df{\omega_i} =
\int_{W_{\alpha_i}} \phi_i^\ast(\df{\omega_i})
= \int_{W_{\alpha_i}} \df{\phi_i^\ast(\omega_i)}
= \int_\sigma \df{\phi_i^\ast(\omega_i)} =
\int_{\partial \sigma} \phi_i^\ast(\omega_i) = 0
\]
because
$\displaystyle \phi_i^\ast(\omega_i)\big|_{\partial \sigma} = 0$.  Since
$\displaystyle \omega_i\big|_{\partial S} = 0$, we also get
$\displaystyle \int_{\partial S} \omega_i = 0$.  Thus
$\displaystyle \int_S \df{\omega_i} = \int_{\partial S} \omega_i$.

\subQ{ii} If $(\supp \omega_i) \cap \partial U_{\alpha_i} \neq \emptyset$,
We may use Lemma~\ref{stokesLem2} with $U$ and $W$ replaced
by $U_{\alpha_i}$ and $W_{\alpha_i}$ respectively, and
$\displaystyle \upsilon = \phi_i^\ast(\omega_i)$, to conclude that
$\displaystyle
\int_{\partial S} \omega_i = \int_{\partial \sigma} \phi_i^\ast(\omega_i)$.
It again follows from Theorems~\ref{stokesStokes} that
\[
\int_{\partial S} \omega_i = \int_{\partial \sigma} \phi_i^\ast(\omega_i)
= \int_\sigma \df{\phi_i^\ast(\omega_i)}
= \int_\sigma \phi_i^\ast(\df{\omega_i})
= \int_{W_{\alpha_i}} \phi_i^\ast(\df{\omega_i})
= \int_{U_{\alpha_i}} \df{\omega_i}
= \int_S \df{\omega_i} \ .
\]

\subQ{iii}
Since $\displaystyle \sum_{i=1}^I \psi_i = 1$,  we get that
$\displaystyle 0 = \df{\left(\sum_{i=1}^I \psi_i\right)} =
\sum_{i=1}^I \df{\psi_i}$.  Therefore,
\begin{equation}\label{stokeszero}
\sum_{i=1}^I \int_S \df{\psi_i} \wedge \omega
= \int_S \left( \sum_{i=1}^I \df{\psi_i} \right)\wedge \omega
= 0 \ .
\end{equation}

\stage{iv} Finally, it follows from (i), (ii) and (iii) that
\begin{align*}
\int_S \df{\omega} &= \sum_{i=1}^I \int_S \psi_i \df{\omega}
=  \sum_{i=1}^I \int_S \df{\psi_i} \wedge \omega
+ \sum_{i=1}^I \int_S  \psi_i \df{\omega}
= \sum_{i=1}^I \int_S \left( \df{\psi_i} \wedge \omega
+ \psi_i \df{\omega} \right) \\
&= \sum_{i=1}^I \int_S \df{(\psi_i \omega)}
= \sum_{i=1}^I \int_S \df{\omega_i}
= \sum_{i=1}^I \int_{\partial S} \omega_i
= \int_{\partial S} \omega \ .
\end{align*}
The first equality comes from he definition of the integral of a
differential $k$-form since
$\displaystyle \left\{ \psi_i\right\}_{1 \leq i \leq I}$
is a partition of unity subordinate to the open cover
$\displaystyle \left\{ V_{\alpha_i} \right\}_{1 \leq i \leq I}$ of $S$, the second
equality comes from (\ref{stokeszero}), the last equality comes from
the definition of the integral of a differential $k-1$-form since
$\displaystyle \left\{ \psi_i \right\}_{1 \leq i \leq I}$
is a partition of unity subordinate to the open cover
$\displaystyle \left\{ V_{\alpha_i} \right\}_{1 \leq i \leq I}$ of
$\partial S$.
\end{proof}

The next proposition is almost a corollary of Stokes' theorem and in
some textbooks it is stated as Stokes' theorem.

\begin{prop} \label{propStokesTh}
Suppose that $S$ is a oriented $k$-dimensional manifold of class
$\displaystyle C^1$ and $\omega$ is a differential
$(k-1)$-form on $S$ with compact support in $S$.  Then,
\[
\int_S \df{\omega} = \int_{\partial S} \omega \ .
\]
\end{prop}

\begin{proof}
Let $\displaystyle \A = \{(W_\alpha, U_\alpha, \phi_\alpha)\}_{\alpha\in A}$
be an atlas of orientation preserving local charts for $S$, and
$\displaystyle \left\{\psi_j\right\}_{j \in \NNp}$ be a
partition of unity subordinate to the atlas $\A$.

Since $\supp \omega$ is compact, it follows from
Proposition~\ref{cov4forM} that there exist only a finite number of
$\psi_j$ that are not null  on $\supp \omega$.  Without loss of
generality, we may assume that they are
$\displaystyle \psi_j$ for $1 \leq j \leq J$.
Suppose that $\alpha_j \in A$ is an index such that
$\supp \psi_j \subset U_{\alpha_j}$.

Let $\omega_j = \psi_j \omega$ for $1 \leq j \leq J$.  Then
\[
\int_S \omega = \sum_{j=1}^J \int_S \omega_j  \ .
\]
The rest of the proof is almost identical to the proof of
Theorem~\ref{TheStokesTh}.
\end{proof}

In Theorem~\ref{TheStokesTh}, we assumed that $S$ was a compact manifold.
The proof of that theorem used the fact that the support of the
differential $(k-1)$-form $\omega$ on $S$ and $\partial S$ were covered
by local charts because $\overline{S} = S$ and
$\partial S = \overline{S} \setminus \Int S$.

In proposition~\ref{propStokesTh}, we did not assume that $S$ was
compact.  Therefore, $\partial S$ might not be equal to 
$\overline{S} \setminus \Int S$. Hence, there might be points of
$\overline{S} \setminus \Int S$ that were not elements of a local chart.
However, we assumed that the support of the differential $(k-1)$-form
$\omega$ on $S$ was a subset of $S$.  This ensures that
$\supp \omega$ and $\partial S$ were covered by local charts.

What can we do if $S$ is not compact and the support of a
differential $(k-1)$-form $\omega$ on $S$ is not a subset of $S$?
Namely, what can we do if $\supp \omega \cap (\overline{S} \setminus S)
\neq \emptyset$?  This is one of the issues that we address in the next
subsection.

\subsection{Generalization of Stokes' Theorem}

The Stokes' theorem stated above does not apply to simple objects like
cones because they are not manifolds.  They have ``corners'' that
cannot be smoothly mapped to some open subsets of $H_k$.
Nevertheless, they are almost like manifolds except at some points.

There are two possible approaches to generalize Stokes' theorem.  We can
either generalize our definition of manifolds to include the
``manifolds with corners.''\  This is the approach adopted in \cite{A}.
The second approach is to show that the set of points where there are
``corners'' is negligible and that the integrals can be computed by
ignoring these points.  This is the approach used in \cite{LaDM}.
Both approach however approximate the integral of a
differential form $\omega$ on a ``manifold with corners'' by the limit
of integrals of differential forms $\omega_j$ for $j\in \NN$ that do
not involve the ``corners.''

We present the approaches in \cite{LaDM} and sketch the
approach in \cite{A}.  Both approaches require some notions of measure
theory and functional analysis that the reader may not have.  So,
the next subsections are for readers with a more advanced background
in mathematics than the background that is required for the rest of
the book.  Nevertheless, we believe that the reader without the proper
background may still be able to follow the main steps of the proofs.

\subsubsection{First Approach}

We need to introduce some new concepts before stating and proving a
generalized version of Stokes' theorem.  The next results heavily
relied on measure theory and functional analysis.

\begin{theorem} \label{functCcSomg}
Let $S$ be a oriented $k$-dimensional manifold of class $\displaystyle C^1$
without boundary and $\omega$ be a continuous differential $k$-form on $S$.
\begin{enumerate}
\item There exists a unique linear functional $\eta$ on
$C_c(S)$, the space of continuous functions on $S$ with compact
supports, such that, if $(W,U,\phi)$ is an orientation preserving
local chart of $S$ and
$\phi^\ast \omega = f \df{w_1}\wedge \df{w_2} \wedge \ldots \wedge \df{w_k}$
is the local representation of $\omega$ on $W$, then
$\displaystyle \eta(g) = \int_W g(\phi(\VEC{w})) \, f(\VEC{w})
\dx{\VEC{w}}$ for all $g \in C_c(S)$ with support in $U$.
\item The linear functional $\eta$ is continuous on $C_c(S)$ with
respect to the supermum norm.
\end{enumerate}
If in addition $\omega$ has compact support, then
\begin{enumerate}  \addtocounter{enumi}{2}
\item $\eta$ can be extended to a continuous linear functional on the
space $C_0(S)$ of continuous functions that vanish at infinity in $S$
\footnotemark, and
\item there exists a unique real measure $\eta_{\omega}$ on $S$ such that 
$\displaystyle \eta(g) = \int_S g \dx{\eta_{\omega}}$ for all
$g \in C_0(S)$.
\end{enumerate}
\end{theorem}

\footnotetext{A continuous function $f$ on $S$ {\bfseries vanish at
infinity}\index{Vanish at Infinity} if for every $\epsilon >0$ there
exists a compact set $K \subset S$ such that $|f(\VEC{x})| < \epsilon$
for all $\VEC{x} \in S \setminus K$.}

\begin{proof}[Proof (Sketch)]
\stage{1} Let $\displaystyle \A = \left\{
(W_\alpha,U_\alpha,\phi_\alpha)\right\}_{\alpha\in A}$
be an atlas of orientation preserving local charts for $S$ and
$\displaystyle \{\psi_j\}_{j\in \NNp}$ be a partition of unity subordinate 
to the atlas $\A$.  Suppose that $\alpha_j \in A$ is an index such that
$\supp \psi_j \subset V_{\alpha_j}$ for $\displaystyle j \in \NNp$.
We also assume that $\phi_\alpha^\ast \omega = f_\alpha \df{w_1}\wedge
\df{w_2} \wedge \ldots \wedge \df{w_k}$ on $W_\alpha$ for all $\alpha \in A$
where $f_\alpha:W_\alpha \to \RR$ is continuous.

Given $g \in C_c(S)$, it follows from Proposition~\ref{cov4forM} that
there exists a finite subset $\displaystyle J \subset \NNp$ such that
$\displaystyle g = \sum_{j\in J} \psi_j g$
because $g$ has compact support.  We define $\eta(g)$ as
\begin{equation} \label{functCcSEq1omg}
\eta(g) = \eta\left(\sum_{j\in J} \psi_j g\right)
= \sum_{j\in J} \int_{W_{\alpha_j}} \psi_j(\phi_{\alpha_j}(\VEC{w}))
g(\phi_{\alpha_j}(\VEC{w}))\, f_{\alpha_j}(\VEC{w}) \dx{\VEC{w}} \ .
\end{equation}
Note that $\psi_j g$ is a continuous function on $S$ with compact
support in $U_{\alpha_j}$.  It is easy to show that this defines a
linear function $\eta$ on $C_c(S)$.

Using the change of variable formula, it is not too difficult to prove
that this definition is independent of the chosen atlas.
Suppose that $\displaystyle \tilde{\A} = \left\{
(\tilde{W}_{\tilde{\alpha}},\tilde{U}_{\tilde{\alpha}},
\tilde{\phi}_{\tilde{\alpha}})\right\}_{\tilde{\alpha}\in \tilde{A}}$
is another atlas of orientation preserving local charts for $S$.
We assume that $\tilde{\phi}_{\tilde{\alpha}}^\ast \omega =
\tilde{f}_{\tilde{\alpha}} \df{w_1}\wedge \df{w_2} \wedge \ldots
\wedge \df{w_k}$ on $\tilde{W}_{\tilde{\alpha}}$ for all
$\tilde{\alpha} \in \tilde{A}$ 
where $\tilde{f}_{\tilde{\alpha}}:\tilde{W}_{\tilde{\alpha}} \to \RR$
is continuous.  It is enough to prove that
\[
\int_{W_{\alpha}} g(\phi_\alpha(\VEC{w}))\, f_\alpha(\VEC{w}) \dx{\VEC{w}} 
= \int_{\tilde{W}_{\tilde{\alpha}}}
g(\tilde{\phi}_{\tilde{\alpha}}(\VEC{w}))\,
\tilde{f}_{\tilde{\alpha}}(\VEC{w}) \dx{\VEC{w}}
\]
for all $g \in C_c(S)$ with
$\supp g \subset U_\alpha \cap \tilde{U}_{\tilde{\alpha}}$.
We get from Theorem~\ref{stokesCofV} that
\begin{align*}
&\tilde{f}_{\tilde{\alpha}} \df{w_1}\wedge \df{w_2} \wedge \ldots \wedge \df{w_k}
= (\phi_\alpha^{-1} \circ \tilde{\phi}_{\tilde{\alpha}})^\ast
f_{\alpha} \df{w_1}\wedge \df{w_2} \wedge \ldots \wedge \df{w_k} \\
&\qquad = f_\alpha\big((\phi_\alpha^{-1} \circ
\tilde{\phi}_{\tilde{\alpha}} )(\VEC{w})\big)
\det \diff (\phi_\alpha^{-1} \circ \tilde{\phi}_{\tilde{\alpha}})
(\VEC{w}) \df{w_1}\wedge \df{w_2} \wedge \ldots \wedge \df{w_k}
\end{align*}
on $\displaystyle
\tilde{\phi}_{\tilde{\alpha}}^{-1}(U_\alpha \cap \tilde{U}_\alpha)$.
It then follows from Theorem~\ref{cov7} about change of variables that
\begin{align*}
&\int_{W_\alpha} g(\phi_\alpha(\VEC{w}))\,
f_\alpha(\VEC{w}) \dx{\VEC{w}} \\
&\qquad = \int_{\tilde{W}_{\tilde{\alpha}}}
g\big(\phi_\alpha\circ (\phi_\alpha^{-1} \circ
\tilde{\phi}_{\tilde{\alpha}} )(\VEC{w})\big)\,
f_\alpha\big((\phi_\alpha^{-1} \circ
\tilde{\phi}_{\tilde{\alpha}} )(\VEC{w})\big)
\big|\det \diff (\phi_\alpha^{-1} \circ \tilde{\phi}_{\tilde{\alpha}})
(\VEC{w}) \big| \dx{\VEC{w}} \\
&\qquad = \int_{\tilde{W}_{\tilde{\alpha}}}
g\big(\tilde{\phi}_{\tilde{\alpha}}(\VEC{w})\big)\,
f_\alpha\big((\phi_\alpha^{-1} \circ
\tilde{\phi}_{\tilde{\alpha}} )(\VEC{w})\big)
\det \diff (\phi_\alpha^{-1} \circ \tilde{\phi}_{\tilde{\alpha}})
(\VEC{w}) \dx{\VEC{w}} \\
&\qquad = \int_{\tilde{W}_{\tilde{\alpha}}}
g\big(\tilde{\phi}_{\tilde{\alpha}}(\VEC{w})\big)\,
\tilde{f}_{\tilde{\alpha}}(\VEC{w}))\dx{\VEC{w}} \ .
\end{align*}
Note that the second identity comes from the fact that the determinant
is positive since we have orientation preserving local charts.
For similar reason, the definition of $\eta$ in (\ref{functCcSEq1omg})
is independent of the chosen partition of unity.

The uniqueness of the linear functional $\eta$ comes from
(\ref{functCcSEq1omg}).

\stage{2} This follows from the definition of $\eta$ in
(\ref{functCcSEq1omg}) and the fact that integration is continuous
with respect to uniform convergence.

\stage{3} It suffices to note that there exists a finite set
$\displaystyle J \subset \NNp$ such that
$\supp \omega \cap \supp \psi_j \neq \emptyset$  because
$\supp \omega$ is compact. We may choose this set $J$
in (\ref{functCcSEq1omg}) for all $g \in C_c(S)$.
It is then easy to prove that $\eta$ is uniformly continuous using
(\ref{functCcSEq1omg}) or equivalently that $\eta$ is a bounded linear
functional.

Since $C_c(S)$ is dense with respect to the supremum norm in the space
$C_0(S)$ (see \cite{Rrca}), we can extend $\eta$ continuously to $C_0(S)$.

\stage{4} It follows from Riesz Representation Theorem (see
Theorem~6.19 in \cite{Rrca})
that there exists a unique real measure $\eta_{\omega}$ on $S$ such that
$\displaystyle \eta(g) = \int_S g \dx{\eta_{\omega}}$ for all
$g \in C_0(S)$.
\end{proof}

Thus, if $\omega$ is a continuously differential $k$-form with compact
support, then we may then define $\displaystyle \int_S g \omega$ for
all (Lebesgue) integrable function $g$ on $S$ as
$\displaystyle \int_S g \dx{\eta_{\omega}}$.
Note that one of the properties of a real measure $\mu$ on a set
$X$ is that $\mu(M) \in \RR$ for all measurable sets $M$ of $X$.  In
particular, $\mu(X) \in \RR$.  Thus, $\eta_{\omega}(S) \in \RR$ and it
makes sense to set $\displaystyle \int_S \omega = \int_S \dx{\eta_{\omega}}$.
In fact, if we take the set $J$ defined in (3) of the previous proof
and a function $g\in C_c(S)$ such that $g=1$ on $K$, then
\[
\int_S \omega = \int_S g \omega = \eta(g)
= \int_S g \dx{\eta_{\omega}} = \int_S \dx{\eta_{\omega}}
\]
because $\supp \eta_{\omega} \subset \supp \omega$ as it can be proved
(see (4) or Proposition~\ref{prop_functCcS} below).  We will see
shortly that Theorem~\ref{functCcSomg} is also true even if
$\supp \omega$ is not a compact subset of $S$.  Thus, defining 
$\displaystyle \int_S \omega = \int_S \dx{\eta_{\omega}}$ is justified
in that case.  In fact, if we use a sequence $\{g_j\}_{j \in \NNp}$ of
functions in $C_c(S)$ such that $g_{j+1} \leq g_j$ for all $j$ and
$\{g_j\}_{j \in \NNp}$ converges pointwise to $\Chi_S$, then we get
from the Lebesgue dominated convergence theorem \footnote{Since we are
working with a real measure, we cannot use the Lebesgue monotone
convergence theorem as if we had a positive measure.} that
\[
\int_S g_j \omega = \eta(g_j) = \int_S g_j \dx{\eta_{\omega}}   
\to \int_S \Chi_S \dx{\eta_{\omega}}
= \int_S \dx{\eta_{\omega}} = \eta_{\omega}(S) < \infty \ .
\]

\begin{defn}
The linear functional $\eta$ defined in Theorem~\ref{functCcSomg} is
called the {\bfseries functional associated to
$\omega$}\index{Functional Associated to a Differential From} on $S$.
The measure $\eta_{\omega}$ is called the {\bfseries measure
associated to $\omega$}\index{Measure Associated to a Differential From}.
\end{defn}

\begin{theorem} \label{functCcS}
Let $S$ be a oriented $k$-dimensional manifold of class $\displaystyle C^1$
without boundary and $\omega$ be a continuous differential $k$-form on $S$.
\begin{enumerate}
\item There exists a unique positive linear functional $\eta$ on
$C_c(S)$  \footnotemark\ such that, if $(W,U,\phi)$ is an orientation preserving
local chart of $S$ and
$\phi^\ast \omega = f \df{w_1}\wedge \df{w_2} \wedge \ldots \wedge \df{w_k}$
is the local representation of $\omega$ on $W$, then
$\displaystyle \eta(g) = \int_W g(\phi(\VEC{w}))\, |f(\VEC{w})|
\dx{\VEC{w}}$ for all $g \in C_c(S)$ with support in $U$.
\item There exists a unique positive (countably additive and Borel) measure
$\eta_{|\omega|}$ on $S$ such that 
$\displaystyle \eta(g) = \int_S g \dx{\eta_{|\omega|}}$ for all
$g \in C_c(S)$.
\end{enumerate}
\end{theorem}

\footnotetext{Namely, $\eta(g) \geq 0$ for all
$g\in C_c(S)$ such that $g \geq 0$ on $S$.}

\begin{proof}[Proof (Sketch)]
\stage{1}
Let $\displaystyle \A = \left\{
(W_\alpha,U_\alpha,\phi_\alpha)\right\}_{\alpha\in A}$
be an atlas of orientation preserving local charts for $S$ and
$\displaystyle \{\psi_j\}_{j\in \NNp}$ be a partition of unity subordinate 
to the atlas $\A$.  Suppose that $\alpha_j \in A$ is an index such that
$\supp \psi_j \subset V_{\alpha_j}$ for $\displaystyle j \in \NNp$.
We also assume that $\phi_\alpha^\ast \omega = f_\alpha \df{w_1}\wedge
\df{w_2} \wedge \ldots \wedge \df{w_k}$ on $W_\alpha$ for all $\alpha \in A$
where $f_\alpha:W_\alpha \to \RR$ is continuous.

Given $g \in C_c(S)$, it follows from Proposition~\ref{cov4forM} that
there exists a finite subset $\displaystyle J \subset \NNp$ such that
$\displaystyle g = \sum_{j\in J} \psi_j g$
because $g$ has a compact support.  We define $\eta(g)$ as
\begin{equation} \label{functCcSEq1}
\eta(g) = \eta\left(\sum_{j\in j} \psi_j g\right)
= \sum_{j\in j} \int_{W_{\alpha_j}} \psi_j(\phi_{\alpha_j}(\VEC{w}))
g(\phi_{\alpha_j}(\VEC{w}))\, |f_{\alpha_j}(\VEC{w})| \dx{\VEC{w}} \ .
\end{equation}
Note that $\psi_j g$ is continuous function on $S$ with compact
support in $U_{\alpha_j}$.  It is easy to show that this defines a
positive linear function $\eta$ on $C_c(S)$.  Using the change of
variable formulae, we can prove as we did in the proof of
Theorem~\ref{functCcSomg} that this definition is independent of the
chosen atlas and satisfies item (1) in the statement of the theorem.
Likewise, the definition of $\eta$ in (\ref{functCcSEq1})
is independent of the chosen partition of unity.

\stage{3} It follows from Riesz Representation Theorem (see
Theorem~2.24 in \cite{Rrca}) that there exists a positive (countably
additive and Borel) measure $\eta_{|\omega|}$ on $S$ such that
$\displaystyle \eta(g) = \int_S g \dx{\eta_{|\omega|}}$ for all
$g \in C_c(S)$.
\end{proof}

In Theorem~\ref{functCcS}, the assumption that $S$ is an oriented
manifold is not needed but we still stated it because we will only
work with oriented manifolds.  There is a version of Stokes' theorem
for non-orientable manifolds but we will not present it.

\begin{defn}
The positive measure $\eta_{|\omega|}$ given in Theorem~\ref{functCcS} is
called the {\bfseries measure associated to $|\omega|$}\index{Measure
Associated to a Differential From}.
\end{defn}

We say that a positive measure $\mu$ on a manifold $S$ is a
{\bfseries finite}\index{Finite Measure} if $\mu(S) < \infty$.

\begin{rmkList}
\begin{enumerate}
\item Since $C_c(S)$ is dense with respect to the supremum norm in the space
$C_0(S)$, we have that
$\displaystyle \eta(g) = \int_S g \dx{\eta_{|\omega|}}$ for all
$g \in C_0(S)$.
\item Theorems~\ref{functCcSomg} is true if $\eta_{|\omega|}$ given in
Theorem~\ref{functCcS} is a finite measure.

In the proof of Theorems~\ref{functCcSomg}, the only issue is to prove
that $\eta$ is uniformly continuous on $C_c(S)$ so that it can be
extended continuously to $C_0(S)$.

Suppose that $g_1$ and $g_2$ are two functions of $C_c(S)$.
Since they both have compact support, there exists a finite set
$\displaystyle J \subset \NNp$ such that
$\supp \phi_j \cap \supp g_1 = \emptyset$ and 
$\supp \phi_j \cap \supp g_2 = \emptyset$ for $j \not\in J$.
Using (\ref{functCcSEq1omg}), we get that
\begin{align*}
|\eta(g_1) - \eta(g_2)| &=
\bigg| \eta\left(\sum_{j\in J} \psi_j (g_1-g_2)\right) \bigg| \\
&\leq \sum_{j\in J} \int_{W_{\alpha_j}} \psi_j(\phi_{\alpha_j}(\VEC{w}))
\big|g_1(\phi_{\alpha_j}(\VEC{w})) - g_1(\phi_{\alpha_j}(\VEC{w}))\big|
 \, \big|f_{\alpha_j}(\VEC{w})\big| \dx{\VEC{w}} \\
&\leq \|g_1-g_2\|_{\infty}
\sum_{j\in J} \int_{W_{\alpha_j}} \psi_j(\phi_{\alpha_j}(\VEC{w}))
 \, \big|f_{\alpha_j}(\VEC{w})\big| \dx{\VEC{w}}
\leq \|g_1-g_2\|_{\infty}\, \eta_{|\omega|}(S) \ .
\end{align*}
It follows that $\eta$ is uniformly continuous on $C_c(S)$.
\end{enumerate}
\end{rmkList}

\begin{prop} \label{prop_functCcS}
We assume that
{\renewcommand{\labelitemi}{-}  
\begin{itemize}
\item $S$ is an oriented $k$-dimensional manifold of class
  $\displaystyle C^1$ without boundary,
\item $\omega$ is a continuous differential $k$-form on $S$,
\item $\eta_{|\omega|}$ is the positive measure provide in
  Theorem~\ref{functCcS},
\item $\displaystyle \A
= \left\{ (W_\alpha,U_\alpha,\phi_\alpha)\right\}_{\alpha\in A}$
is an atlas of orientation preserving local charts for $S$,
\item $\displaystyle \{\psi_j\}_{j\in \NNp}$ is a partition of unity
  subordinate to the atlas $\A$,
\item $\alpha_j \in A$ is an index such that
$\supp \psi_j \subset V_{\alpha_j}$ for $\displaystyle j \in \NNp$, and
\item $\phi_\alpha^\ast \omega = f_\alpha \df{w_1}\wedge \df{w_2} \wedge
\ldots \wedge \df{w_k}$ on $W_\alpha$ for all $\alpha \in A$ where
$f_\alpha:W_\alpha \to \RR$ is continuous.
\end{itemize}
}
Then
\begin{enumerate}
\item $\displaystyle \left|\int_S g \omega \right| \leq \int_S |g|
\dx{\eta_{|\omega|}}$ for all $g \in C_c(S)$,
\item 
$\displaystyle \left|\int_S \omega \right| \leq \int_S \dx{\eta_{|\omega|}}$
if $\omega$ has compact support,
\item given a set $V \subset S$ (open or closed) with a compact closure in $S$,
\[
\eta_{|\omega|}(V) = \sum_{j\in J} \int_{W_{\alpha_j} \cap \phi_{\alpha_j}^{-1}(V)}
\psi_j(\phi_{\alpha_j}(\VEC{w}))\, |f_{\alpha_j}(\VEC{w})| \dx{\VEC{w}}
\]
for some finite subset $\displaystyle J \subset \NNp$, and
\item $\supp \eta_{|\omega|} \subset \supp \omega$ if $\omega$ has
compact support.
\end{enumerate}
\end{prop}

\begin{proof}
\stage{1} Suppose that $g \in C_c(S)$.  It follows from
Proposition~\ref{cov4forM} that there exists a finite subset
$\displaystyle J\subset \NNp$ such that
$\displaystyle g = \sum_{j\in J} \psi_j g$
because $g$ has a compact support.  We then get from
(\ref{functCcSEq1}) that
\begin{align*}
\left| \int_S g\, \omega \right|
&= \left|\sum_{j\in J} \int_S \psi_j g\, \omega \right|
= \left| \sum_{j\in J} \int_{U_{\alpha_j}} \psi_j g\, \omega \right| \\
&= \left| \sum_{j\in J} \int_{W_{\alpha_j}} \psi_j(\phi_{\alpha_j}(\VEC{w}))
g(\phi_{\alpha_j}(\VEC{w})) f_{\alpha_j}(\VEC{w}) \dx{\VEC{w}} \right| \\
&\leq \sum_{j\in J} \int_{W_{\alpha_j}} \psi_j(\phi_{\alpha_j}(\VEC{w}))
|g(\phi_{\alpha_j}(\VEC{w}))|\, |f_{\alpha_j}(\VEC{w})| \dx{\VEC{w}}
= \eta(|g|) = \int_S |g| \dx{\eta_{|\omega|}} \ .
\end{align*}

\stage{2}  This result is true even if $\omega$ does not have a
compact support \footnote{If $\eta_{|\omega|}(S) = \infty$, then it is
obviously true.  If $\eta_{|\omega|}(S) < \infty$, then it follows
from the previous remark that the measure $\eta_\omega$ on $S$ exists.
The result in (2) can be proved using the total variation $|\eta_{\omega}|$
for $\eta_{\omega}$ and $|\eta_{\omega}| = \eta_{|\omega|}$.}.
We only prove (2) for $\omega$ with compact support so that we do not
have to refer to real measure.  Choose $\VEC{u} \in S$.  For $m>0$, choose
$\displaystyle g_m \in C_c(S)$ such that
$0 \leq g_m(\VEC{x}) \leq 1$ for all $\displaystyle \VEC{x} \in S$,
$g_m(\VEC{x}) = 1$ for all $\VEC{x} \in S \cap \overline{B_m(\VEC{u})}$ and
$g_m(\VEC{x}) = 0$ for all $\VEC{x} \in S \cap B_{m+1}(\VEC{u})$
\footnote{Recall that $B_m(\VEC{u})$ is the open ball of radius $m$
centred at $\VEC{u}$.}.
This is a classical result.  In fact, given a compact set $K$ and an open set
$V \supset K$ in $\RR^n$, we can use a partition of unity for instance
to proof that there exists a function $g$ of class
$\displaystyle C^\infty$ with compact support such that
$0\leq g(\VEC{x}) \leq 1$ for all $\VEC{x}$, $g(\VEC{x}) = 1$ for all
$\VEC{x} \in K$ and $g(\VEC{x}) = 0$ for $\VEC{x} \not\in V$.  We
have already used this type of arguments before.

We have that $g_m(\VEC{x}) \to 1$ for all $\VEC{x} \in S$ as $m \to \infty$
and $g_m(\VEC{x}) \leq g_{m+1}(\VEC{x})$ for all $m>0$.
It follows that
\begin{equation} \label{functCcSEq2}
\left| \int_S \omega \right|
= \left| \lim_{m\to \infty} \int_S g_m \omega \right|
\leq \lim_{m\to \infty} \left| \int_S g_m \omega \right|
\leq \lim_{m\to \infty} \int_S |g_m| \dx{\eta_{|\omega|}}
= \int_S \dx{\eta_{|\omega|}} \ ,
\end{equation}
where the Lebesgue dominated convergence theorem (see \cite{Rrca}) is
used to get the first equality and the Lebesgue monotone
convergence theorem (see \cite{Rrca}) is used to get the last equality.
The use of the Lebesgue dominated convergence theorem to get the first
equality in (\ref{functCcSEq2}) is explained in detail below.

Since $K = \supp \omega$ is a compact set, it follows from
Proposition~\ref{cov4forM} that $\psi_j\big|_K = 0$ for all but a finite
number of $\displaystyle j \in \NNp$.  Let $\displaystyle J \subset \NNp$
be the finite set of indices such that $\psi_j\big|_K \neq 0$.
As usual, we assume that $\alpha_j \in A$ is an index such that
$\supp \psi_j \subset U_{\alpha_j}$ for $\displaystyle j \in \NNp$.  Then
$\displaystyle \supp \omega \subset \bigcup_{j\in J} U_{\alpha_j}$.
Hence
\[
\int_S \omega  = \sum_{j\in J} \int_{W_{\alpha_j}}
\psi_j(\phi_{\alpha_j}(\VEC{w}))
f_{\alpha_j}(\VEC{w}) \dx{\VEC{w}} \ .
\]
Since $(\psi_j(\phi_{\alpha_j}(\VEC{w})) g_m(\phi_{\alpha_j}(\VEC{w}))
f_{\alpha_j}(\VEC{w}) \to
\psi_j(\phi_{\alpha_j}(\VEC{w})) f_{\alpha_j}(\VEC{w})$
as $m \to \infty$ for all $\VEC{w} \in W_{\alpha_j}$,
$|\psi_j(\phi_{\alpha_j}(\VEC{w})) g_m(\phi_{\alpha_j}(\VEC{w}))
f_{\alpha_j}(\VEC{w}) | \leq
|\psi_j(\phi_{\alpha_j}(\VEC{w})) f_{\alpha_j}(\VEC{w})|$ for all
$\VEC{w} \in W_{\alpha_j}$ and
$m>0$, and $|(\psi_j\circ \phi_{\alpha_j}) f_{\alpha_j}|$ has
a bounded integral on $W_{\alpha_j}$ for all $j \in J$, we may use the
Lebesgue dominated convergence theorem to conclude that
\[
\int_{W_{\alpha_j}} \psi_j(\phi_{\alpha_j}(\VEC{w}))
f_{\alpha_j}(\VEC{w}) \dx{\VEC{w}}
= \lim_{m\to \infty}
\int_{W_{\alpha_j}} \psi_j(\phi_{\alpha_j}(\VEC{w})) g_m(\phi_{\alpha_j}(\VEC{w}))
f_{\alpha_j}(\VEC{w}) \dx{\VEC{w}}
\]
for all $j \in J$.  Hence
\begin{align*}
\int_S \omega  &= \sum_{j\in J} \left(
\lim_{m\to \infty} \int_{W_{\alpha_j}} \psi_j(\phi_{\alpha_j}(\VEC{w}))
g_m(\phi_{\alpha_j}(\VEC{w})) f_{\alpha_j}(\VEC{w}) \dx{\VEC{w}} \right) \\
&= \lim_{m\to \infty} \left( \sum_{j\in J} \int_{W_{\alpha_j}}
\psi_j(\phi_{\alpha_j}(\VEC{w})) g_m(\phi_{\alpha_j}(\VEC{w})) 
f_{\alpha_j}(\VEC{w}) \dx{\VEC{w}} \right)
= \lim_{m\to \infty} \int_S g_m \omega \ .
\end{align*}

\stage{3} Since $K = \overline{V}$ is a compact subset of $S$, it follows from
Proposition~\ref{cov4forM} that $\psi_j\big|_K = 0$ for all but a finite
number of $\displaystyle j \in \NNp$.  Let $\displaystyle J \subset \NNp$
be the finite set of indices such that $\psi_j\big|_K \neq 0$.

For $m>0$, choose $\displaystyle g_m \in C_c(S)$ such that
$0 \leq g_m(\VEC{x}) \leq 1$ for all $\displaystyle \VEC{x} \in S$,
$g_m(\VEC{x}) = 1$ for all $\displaystyle \VEC{x} \in \{ \VEC{x} \in S :
\dist{\VEC{x}}{S \setminus V} \geq 1/m \}$
and $g_m(\VEC{x}) = 0$ for all $\displaystyle \VEC{x} \in S \setminus V$.
We have that $g_m(\VEC{x}) \to \Chi_V(\VEC{x})$ for all $\VEC{x} \in S$
as $m \to \infty$ and $|g_m(\VEC{x})| \leq \Chi_V(\VEC{x})$ for all $m>0$.
Note that $\supp g_m \subset K \subset \bigcup_{j\in J} U_{\alpha_j}$
for all $m$.

Using the Lebesgue dominated convergence theorem, we get that
\begin{align*}
\eta_{|\omega|}(V) &= \int_S \Chi_V \dx{\eta_{|\omega|}}
= \lim_{m\to \infty} \int_S g_m \dx{\eta_{|\omega|}} \\
&= \lim_{m\to \infty} \left(
\sum_{j\in J} \int_{W_{\alpha_j}} \psi_j(\phi_{\alpha_j}(\VEC{w}))  
g_m(\phi_{\alpha_j}(\VEC{w}))\, |f_{\alpha_j}(\VEC{w})| \dx{\VEC{w}} \right) \\
&= \sum_{j\in J} \int_{W_{\alpha_j}} \psi_j(\phi_{\alpha_j}(\VEC{w}))
\Chi_V(\phi_{\alpha_j}(\VEC{w}))\, |f_{\alpha_j}(\VEC{w})| \dx{\VEC{w}} \\
&= \sum_{j\in J} \int_{W_{\alpha_j} \cap \phi_{\alpha_j}^{-1}(V)}
\psi_j(\phi_{\alpha_j}(\VEC{w})) |f_{\alpha_j}(\VEC{w})| \dx{\VEC{w}} \ .
\end{align*}

\stage{4} The support of $\eta_{|\omega|}$ is the complement of the
largest open set $G$ such that $\eta_{|\omega|}(G) = 0$.

It follows from (3) that $\eta_{|\omega|}(V) = 0$ for all open
subsets $V$ of $S$ such that $\overline{V}$ is a compact subset of $S$ and
$V \cap \supp \omega = \emptyset$.  Since the
countable union of sets of measure zero is a set of measure, we have
that the open set $S \setminus \supp \omega$ has measure zero.
The set $S \setminus \supp \omega$ is open because $\supp \omega$ is
compact and thus closed.  Hence $\supp \eta_{|\omega|} \subset \supp \omega$.
\end{proof}

\begin{defn} \label{defnRegSingPts}
Let $S$ be a $k$-dimensional manifold of class $\displaystyle C^1$
without boundary.  Given $\VEC{u} \in \overline{S} \setminus S$, if
there exists a local chart $(W,U,\phi)$ about $\VEC{u}$ with $U$ an
open subset of $\overline{S}$ according to the induced topology on
$\overline{S}$ from $\displaystyle \RR^n$,
then $\VEC{u}$ is a {\bfseries regular points}\index{Regular Point} of $S$.
The set of regular points is denoted $S_r$.
The points in the set $S_s= \overline{S} \setminus (S \cup S_r)$
are called {\bfseries singular points}\index{Singular Point} of $S$.
\end{defn}

This definition requires some explanation.  We want to emphasize the
difference between the boundary $\partial S$ of a manifold $S$ as we
have defined it and the set $\overline{S} \setminus \Int S$.
We have that $S \cup S_r$ is a manifold with boundary and
$\partial (S \cup S_r) = S_r$.  Moreover
$\overline{S} = S \cup S_r \cup S_s$.  The set $S_r$ is an open subset of
$\overline{S} \setminus S$ because, given a local chart
$(W_{\VEC{u}},U_{\VEC{u}},\phi_{\VEC{u}})$ about $\VEC{u} \in S_r$, all
points in $U_{\VEC{u}} \cap (\overline{S} \setminus S)$ are themselves in
$S_r$ since they belong to the local chart
$(W_{\VEC{u}},U_{\VEC{u}},\phi_{\VEC{u}})$.  
The set $S_s$ is a closed subset of
$\overline{S}\setminus S$ because $S_s = \overline{S}\setminus (S\cup S_r)$. 
Moreover, the set $S_s$ is also a closed subset of $\overline{S}$ because
$\displaystyle S_s = \overline{S} \setminus
\Big( S \cup \bigcup_{\VEC{u} \in S_r} U_{\VEC{u}} \Big)$
with $\displaystyle S = \bigcup_{\alpha \in A} U_\alpha$ where
$\big\{ (W_\alpha,U_\alpha,\phi_\alpha) \big\}_{\alpha\in A}$ is an
atlas for $S$.

\begin{egg}
Consider the cone      \label{coneEgg1}
$\displaystyle C = \{ \VEC{x} \in \RR^3 : x_1^2 + x_2^2 \leq x_3^2
\ \text{with} \ 0 \leq x_3 \leq 1 \}$.  This is not a manifold 
because of the corners.  Let
$\displaystyle S = \{ \VEC{x} \in \RR^3 : x_1^2 + x_2^2 < x_3^2
\ \text{with} \ 0 < x_3 < 1 \}$.
The set $S$ is a manifold without boundary and $C = \overline{S}$.
The set of regular points is
\[
S_r = \{ \VEC{x} \in \RR^3 : x_1^2 + x_2^2 = x_3^2 \ \text{with}
\ 0 < x_3 < 1 \ \text{or} \ x_1^2 + x_2^2 < 1 
\ \text{with} \ x_3 = 0 \} \ .
\]
The set of singular points is the set
\[
S_s= \{ \VEC{x} \in \RR^3 : \VEC{x} = \VEC{e}_3 \ \text{or}
\ x_1^2 + x_2^2 = 1 \ \text{with} \ x_3 = 0 \} \ .
\]
\end{egg}

\begin{defn}
A {\bfseries fundamental sequence}\index{Fundamental Sequence} for a
set $\displaystyle R \subset \RR^n$ is a collection of open sets
$\displaystyle \{V_j\}_{j\in \NNp}$ of $\displaystyle \RR^n$ such that
$R \subset V_j$ for all $\displaystyle j \in \NNp$ and, if
$\displaystyle V \subset \RR^n$ is an open set containing $R$,
then there exists $\displaystyle J\in \NNp$ such that
$V_j \subset V$ for $j\geq J$.
\end{defn}

\begin{defn} \label{defnNegligible}
Let $S$ be a oriented $k$-dimensional manifold of class $\displaystyle C^1$
without boundary.  A closed subset $\displaystyle R \subset \RR^n$ is
{\bfseries negligible}\index{Negligible Set} in $S$ if 
there exist
\begin{enumerate}
\item an open set $\displaystyle G\subset \RR^n$ such that $R \subset G$,
\item a fundamental sequence $\displaystyle \{G_j\}_{j\in \NNp}$
of $R$ in $G$ with $\overline{G_j} \subset G$ for all $j$, and
\item a sequence of functions $\displaystyle \{ g_j\}_{j\in \NNp}$ 
such that
{
\renewcommand{\labelenumii}{\alph{enumii}.}
\begin{enumerate}
\item $g_j : G \to \RR$ is of class $\displaystyle C^1$ for all
$\displaystyle j \in \NNp$,
\item $0 \leq g_j(\VEC{x}) \leq 1$ for all $\VEC{x} \in G$ and
$\displaystyle j \in \NNp$,
\item $g_j(\VEC{x}) = 1$ for all $\VEC{x} \in G\setminus G_j$ and
$\displaystyle j \in \NNp$,
\item there exists for each $\displaystyle j\in \NNp$ an open set
$\tilde{G}_j \subset G_j$ 
such that $R \subset \tilde{G}_j$ and $g_j(\VEC{x}) = 0$ for all
$\VEC{x} \in \tilde{G}_j$, and,
\item for any $(k-1)$-form $\omega$ of class $\displaystyle C^1$
on $G$, the positive measures $\eta_{|\df{g_j}\wedge \omega|}$
associated with $\df{g_j}\wedge \omega$ on $G \cap S$ for
$\displaystyle j \in \NNp$
are finite measures satisfying
$\displaystyle \lim_{j\to \infty} \eta_{|\df{g_j}\wedge \omega|}(G \cap S) = 0$.
\end{enumerate}
}
\end{enumerate}
\end{defn}

This previous definition is scarring but we will shortly demonstrate
how to work with it in most of the situations that we encounter in
practice.

\begin{theorem}[Stokes' Theorem with Singular Points] \label{GenStokesTh}
Let $S$ be an oriented $k$-dimensional manifold of class
$\displaystyle C^1$ without boundary.  Let
$\displaystyle O \subset \RR^n$ be an open set such that
$\overline{S} \subset O$, and $\omega$ be a differential
$(k-1)$-form of class $\displaystyle C^1$ on $O$ 
with compact support.  If the set of singular points $S_s$ of $S$ is
such that $S_s \cap \supp \omega$ is negligible in $S$, and
the measures $\mu_{|\df{\omega}|}$ on $S$ associated to $|\df{\omega}|$
and $\mu_{|\omega|}$ on $S_r$ associated to $|\omega|$ are
finite, then
\begin{equation} \label{GenStokesThEq0}
\int_S \df{\omega} = \int_{S_r} \omega \ .
\end{equation}
\end{theorem}

We will show later that the sets $S_s$ of ``manifolds with corners.''
satisfy the statement of the previous theorem.

Note that the version of Stokes' theorem above is more general than the
previous versions of Stokes' theorem.  First, it does not require that
$S$ be a compact set as in Theorem~\ref{TheStokesTh}.  So, we may not have
that $S_r = \overline{S} \setminus \Int S$ as in
Theorem~\ref{TheStokesTh}.  It also does not require that $\omega$ has
a compact support in $S$ as in Proposition~\ref{propStokesTh}.  Thus,
the intersection of $\supp \omega$ and $S_s$ may not be empty.

\begin{rmk}
Some authors define $\partial S$ as the set      \label{rmkSrSsDNM}
$S_r$ even if the point
$\VEC{u} \in S_r$ does not belong to $S$ as we have for our definition
of $\partial S$.  In that case, the right hand side of 
(\ref{GenStokesThEq0}) is $\displaystyle \int_{\partial S} \omega$.

The previous theorem is also true if $S$ is a manifold with boundary.
It suffices to apply the theorem to $\Int S$ and note that
$\displaystyle \int_S \omega = \int_{\Int S} \omega$.  To justify this
equality, suppose that $S$ is a $k$-dimensional manifold and
$(W,U,\phi)$ is a local chart about $\VEC{u} \in \partial S$.  Then
$W \cap \partial H_k$ is a set of measure zero in $W$ and the integral
over $W \cap \partial H_k$ is equal to the integral over
$\displaystyle W \cap \{\VEC{y} \in \RR^k : y_k>0\}$.

Note that $S_r$ is often the disjoint union of the interior of some
manifolds.  For instance, in Example~\ref{coneEgg1}, we have that
$S_r = \Int M_1 \cup \Int M_2$ where
$\displaystyle M_1 = \{ \VEC{x} \in \RR^3 : x_1^2 + x_2^2 = x_3^2 \ \text{with}
\ 0 < x_3 \leq 1 \}$ and $\displaystyle M_2 =
\{ \VEC{x} \in \RR^3: x_1^2 + x_2^2 \leq 1 \ \text{with} \ x_3 = 0 \}$ are
both manifolds with boundary.  For a reason similar to the one given
in the previous paragraph, if
$\displaystyle \partial S = \bigcup_{m=1}^M \Int M_m$ for some
manifolds $M_m$, then we write $\displaystyle \int_{S_r} \omega
= \sum_{m=1}^M \int_{M_m} \omega$.
\end{rmk}

\begin{egg}
If we consider our cone $C$     \label{coneEgg2}
in Example~\ref{coneEgg1} and assume that $\omega$ is a differential
$(k-1)$-form with compact support defined in an open set containing the cone
$C$.  Our new version of Stokes' theorem states that
$\displaystyle \int_S \df{\omega} = \int_{S_r} \omega$.

Despite the fact that $C$ is not a manifold, we often write
$\displaystyle \int_C \df{\omega}$ instead of
$\displaystyle \int_S \df{\omega}$.  We accept this informal terminology
because $S_s$ is a set of measure zero in $\displaystyle \RR^3$.
Moreover, we write
$\displaystyle \int_{S_r} \omega = \int_{M_1} \omega + \int_{M_2} \omega$
despite the fact that $S_r = \Int M_1 \cup \Int M_2$ as we
explained in the previous remark.  We also informally write
$\displaystyle \int_{\partial C}$ instead of 
$\displaystyle \int_{S_r}$ though $\partial C$ is the topological
boundary of $C$ and is different from the manifold boundary $S_r$.
\end{egg}

\begin{proof}[Proof (of Theorem~\ref{GenStokesTh})]
Since $S_s \cap \supp \omega$ is negligible, there exist an open set
$G \subset O$, a fundamental sequence
$\displaystyle \{G_j\}_{j\in \NNp}$ of $S_s$ in $G$ 
and sequence of functions $\displaystyle \{ g_j\}_{j\in \NNp}$
satisfying Definition~\ref{defnNegligible}.  Moreover, there exists a
collection of open sets $\displaystyle \{\tilde{G}_j\}_{j\in \NNp}$ such that
$S_s \cap \supp\, \omega \subset \tilde{G}_j \subset G_j$ and
$g_j = 0$ on $\tilde{G}_j$ for all $\displaystyle j \in \NNp$.

Let $\omega_j = g_j \omega$ for all $\displaystyle j\in \NNp$ \footnote{Since
$g_j = 1$ on $G\setminus G_j$ we may assume that $g_j = 1$ on
$O \setminus G_j$ and so $\omega_j$ is also defined on $O$.}.  We have
that $\omega_j$ has a compact
support because $\supp\, \omega_j \subset \supp\, \omega$ and $\omega$ has
a compact support.  Since $\omega_j = 0$ on the open set
$\tilde{G}_j \supset S_s \cap \supp\, \omega$, we have that
$S_s \cap \supp\, \omega_j = \emptyset$.   Thus
$(S \cup S_r) \cap \supp\, \omega_j = \overline{S} \cap \supp\, \omega_j$
is a compact set for all $\displaystyle j\in \NNp$ because
$\overline{S} \cap \supp\, \omega_j$ is a closed
subset of the compact set $\supp\, \omega_j$.
We may therefore use Proposition~\ref{propStokesTh} to obtain
\begin{equation} \label{GenStokesThEq1}
\int_{S_r} \omega_j = \int_{S\cup S_r} \df{\omega_j} = \int_S \df{\omega_j}
= \int_S \df{g_j}\wedge \omega + \int_S g_j \df{\omega} \ .
\end{equation}

\stage{i} If $\eta_{\omega}$ is the measure associated to $\omega$
given in Theorem~\ref{functCcSomg} with $S_r$ instead of $S$, then we
get that
\begin{align}
\left| \int_{S_r} \omega_j - \int_{S_r} \omega \right|
&= \left| \int_{S_r}g_j \dx{\eta_{\omega}} - \int_{S_r} \dx{\eta_{\omega}} \right|
= \left| \int_{S_r} (g_j- 1)\dx{\eta_{\omega}}  \right|
= \left| \int_{S_r} (g_j -1) \omega \right| \nonumber \\
&\leq \int_{S_r} |g_j -1| \dx{\eta_{|\omega|}}
= \int_{G_j \cap S_r} |g_j -1| \dx{\eta_{|\omega|}} \nonumber \\
& \leq \int_{G_j \cap S_r} \dx{\eta_{|\omega|}}
= \eta_{|\omega|}(G_j \cap S_r) \ ,  \label{GenStokesThEq2}
\end{align}
where we have used the fact that $g_j - 1 \in C_c(S_r)$ to get
the third equality, Proposition~\ref{prop_functCcS} (2) to get the
first inequality, $g_j(\VEC{x}) - 1 = 0$ for
$\VEC{x} \in O\setminus G_j$ to get the fourth equality, and
$0\leq g_j(\VEC{x}) \leq 1$ for all $\VEC{x} \in O$ to get the last
inequality.

We now show that
$\displaystyle \bigcap_{j\in \NNp} G_j \cap S_r = \emptyset$.
If $\VEC{u} \in S_r$, then there exists a local chart
$(W,U,\phi)$ such that $\VEC{u} \in U$.  Recall that
$U = V \cap \overline{S}$ where $\displaystyle V \in \RR^n$ is an open
set.  We have that $U \cap S_s = \emptyset$.
Choose $\delta$ small enough such that
$\overline{B_\delta(\VEC{u})} \subset V$.
\pdfbox{manifolds/stokesPrf1}
Thus $\overline{B_\delta(\VEC{u})} \cap \overline{S} \subset U$
and $\overline{B_\delta(\VEC{u})} \cap (\overline{S}\setminus S) \subset S_r$.
Hence $O \setminus \overline{B_\delta(\VEC{u})}$ is an open set in
$\displaystyle \RR^n$ containing $S_s$.  By definition of fundamental
sequence for $S_s$, there exists $\displaystyle J\in \NNp$ such that
$G_j \subset O \setminus \overline{B_\delta(\VEC{u})}$ for
$j \geq J$.  Thus $\VEC{u} \not\in G_j \cap S_r$ for $j \geq J$.

Since $\displaystyle \bigcap_{j\in \NNp} G_j \cap S_r = \emptyset$,
we have from measure theory \footnote{Use Theorem~1.19 in \cite{Rrca}
with $\displaystyle A_n = \bigcap_{1 \leq j \leq n} G_j \cap S_r$
for $n\geq 0$.  The fact that $\mu_{|\omega|}$ is
a finite measure is also needed to apply Theorem~1.19.} that
$\eta_{|\omega|}(G_j \cap S_r) \to 0$ as $j \to \infty$.
It follows from (\ref{GenStokesThEq2}) 
that
\begin{equation} \label{GenStokesThEq3}
\lim_{j\to \infty} \int_{S_r} \omega_j = \int_{S_r} \omega \ .
\end{equation}

\stage{ii} The proof that $\displaystyle \int_S g_j \df{\omega}$
converge to $\displaystyle \int_S \df{\omega}$ as $j \to \infty$ is
similar to proof given in (i).

If $\eta_{\df{\omega}}$ is the measure associated to $\df{\omega}$
given in Theorem~\ref{functCcSomg} with $S$, then we
get that
\begin{align}
\left| \int_S g_j \df{\omega} - \int_S \df{\omega} \right|
&= \left| \int_{S_r}g_j \dx{\eta_{\df{\omega}}}
- \int_{S_r} \dx{\eta_{\df{\omega}}} \right|
= \left| \int_{S_r} (g_j- 1)\dx{\eta_{\df{\omega}}}  \right|
= \left| \int_S (g_j -1) \df{\omega} \right| \nonumber \\
&\leq \int_S |g_j -1| \dx{\eta_{|\df{\omega}|}}
= \int_{G_j \cap S} |g_j -1| \dx{\eta_{|\df{\omega}|}} \nonumber \\
&\leq \int_{G_j \cap S} \dx{\eta_{|\df{\omega}|}}
= \eta_{|\df{\omega}|}(G_j \cap S) \ ,  \label{GenStokesThEq4}
\end{align}
where we have used the fact that $g_j - 1 \in C_c(S)$ to get
the third equality, Proposition~\ref{prop_functCcS} (2) to get the
first inequality, $g_j(\VEC{x}) - 1 = 0$ for
$\VEC{x} \in O\setminus G_j$ to get the fourth equality and
$0\leq g_j(\VEC{x}) \leq 1$ for all $\VEC{x} \in O$ to get the last
inequality.

We now show that
$\displaystyle \bigcap_{j\in \NNp} G_j \cap S = \emptyset$.
If $\VEC{u} \in S$, then there exists a local chart
$(W,U,\phi)$ such that $\VEC{u} \in U$.
\pdfbox{manifolds/stokesPrf2}
Since $U \cap (\overline{S} \setminus S) = \emptyset$
We have that $U \cap S_s = \emptyset$.
Choose $\delta$ small enough such that
$\overline{B_\delta(\VEC{u})} \cap (\overline{S} \setminus S) = \emptyset$.
Then $O \setminus \overline{B_\delta(\VEC{u})}$ is an open set in
$\displaystyle \RR^n$ containing $S_s$.
By definition of fundamental sequence for $S_s$, there exists
$\displaystyle J \in \NNp$ such that
$G_j \subset O \setminus \overline{B_\delta(\VEC{u})}$ for $j \geq J$.  Thus
$\VEC{u} \not\in G_j \cap S$ for $j \geq J$.

Since $\displaystyle \bigcap_{j\in \NNp} G_j \cap S = \emptyset$,
we have as in (i) that
$\mu_{|\df{\omega}|}(G_j \cap S) \to 0$ as $j \to \infty$.
It follows from (\ref{GenStokesThEq4}) 
that
\begin{equation} \label{GenStokesThEq5}
\lim_{j\to \infty} \int_S g_j \df{\omega} = \int_S \df{\omega} \ .
\end{equation}

\stage{iii} We get from (3.e) of the definition  of
negligible sets, Definition~\ref{defnNegligible}, and
(2) and (4) of Proposition~\ref{prop_functCcS} that
\begin{equation} \label{GenStokesThEq6}
\left| \int_S \df{g_j}\wedge \omega \right|
\leq \int_S \dx{\eta_{|\df{g_j}\wedge \omega|}}
= \eta_{|\df{g_j}\wedge \omega|}(S)
= \eta_{|\df{g_j}\wedge \omega|}(G \cap S) \to 0
\end{equation}
as $j \to \infty$.  We have used the fact that
$\supp \eta_{|\df{g_j}\wedge \omega|} \subset \supp \df{g_j}\wedge \omega
\subset G$ \footnote{When restricted to measurable subsets of $G \cap S$,
the measure $\eta_{|\df{g_j}\wedge \omega|}$ defined from the $k$-form
$\df{g_j}\wedge \omega$ on the entire manifold $S$ is equal to the
measure $\eta_{|\df{g_j}\wedge \omega|}$ defined from the $k$-form
$\df{g_j}\wedge \omega$ on $G\cap S$ because
$\supp \df{g_j}\wedge \omega \subset G$.  This justifies the use of
the same name for the two measures.}.

\stage{iv} If we let $j \to \infty$ in (\ref{GenStokesThEq1}) and use
(\ref{GenStokesThEq3}), (\ref{GenStokesThEq5}) and
(\ref{GenStokesThEq6}), we get (\ref{GenStokesThEq0}).
\end{proof}

This generalized version of Stokes' theorem was not too hard to prove
but it leaves a big question to answer.  How can we determine if
$S_s \cap \supp \omega$ is negligible in $S$?  The next two propositions
will answer this question.

The next proposition is use to reduce the problem of determining if a
set is negligible to a local problem.

\begin{prop}  \label{propCrit1}
Let $S$ be a oriented $k$-dimensional manifold of class $C^1$.  If
$R_1$ and $R_2$ are negligible in $S$, then $R_1 \cup R_2$ is
negligible in $S$.
\end{prop}

\begin{proof}[Proof (Sketch)]
Suppose that $\displaystyle \{G_{i,j}\}_{j \in \NNp}$ and
$\displaystyle \{g_{i,j}\}_{j \in \NNp}$ are given by the
definition of negligible sets applied to $R_i$ in $S$ for
$i = 1,2$.   It suffices to show that
$\displaystyle \{G_{1,j} \cup G_{2,j}\}_{j \in \NNp}$ and
$\displaystyle \{g_{1,j}\, g_{2,j}\}_{j \in \NNp}$ satisfy the
definition of negligible sets in $S$ for $R_1 \cup R_2$.

Note that $\df{(g_{1,j}\,g_{2,j})} \wedge \omega
= g_{1,j} \df{g_{2,j}} \wedge \omega + g_{2,j} \df{g_{1,j}} \wedge \omega$.
Thus the measure $\displaystyle \eta_{|\df{(g_{1,j}\,g_{2,j})} \wedge \omega|}$
associated to $\df{(g_{1,j}\,g_{2,j})} \wedge \omega$ is the sum of the
measures $\displaystyle \eta_{|g_1 \df{g_{2,j}} \wedge \omega|}$
and $\displaystyle \eta_{|g_2 \df{g_{1,j}} \wedge \omega|}$
associated to $g_{1,j}\df{g_{2,j}} \wedge \omega$ and
$g_{2,j}\df{g_{1,j}} \wedge \omega$ respectively.
\end{proof}

\begin{prop} \label{propCrit2}
Let $S$ be a oriented $n$-dimensional manifold of class $C^1$.
A compact set $R$ is negligible in $S$ if $R$ is a singular $q$-cube
with $q \leq n-2$.
\end{prop}

\begin{egg}
Before proving this proposition, let us use
our two previous propositions to show that the set of singular points
$S_s$ for our famous cone $C$ in Example~\ref{coneEgg1} is negligible
in $S$.

We have that $S_s= R_1 \cup R_2$ where $R_1 = \{ \VEC{e}_3 \}$ and
$R_2 = \{ \VEC{x} \in \RR^3 : x_3 = 0 \ \text{and} \ x_1^2 + x_2^2 = 1 \}$.
Since $R_1 = \sigma_1(\{0\})$
where $\sigma_1$ is defined by $\sigma_1(0) = \VEC{e}_3$, we have that
$R_1$ is the image of a singular $0$-cube.  Since
$0 \leq 3-2 =1$, it follows from Proposition~\ref{propCrit2}
that $R_1$ is negligible in $S$.  Similarly,
since $R_2 = \sigma_2([0,1])$ where $\sigma_2$ is defined by
$\sigma_2(\theta) = (\cos(2\pi\theta),\sin(2\pi\theta),0)$ for
$0 \leq \theta \leq 1$, we have that
$R_2$ is the image of a singular $1$-cube.  Since
$1 \leq 1$, it follows from Proposition~\ref{propCrit2}
that $R_2$ is negligible in $S$.  Finally, it follows from
Proposition~\ref{propCrit1} that $S_s= R_1 \cup R_2$ is negligible in $S$.
\end{egg}

\begin{proof}[Proof (of Proposition~\ref{propCrit2})]
\stage{a} Suppose that $R = \sigma(I_q)$.
We split $I_q$ into $p^q$ cubes of the form
\[
I_{i_1,i_2,\ldots,i_q}
= \left[ \frac{i_1-1}{p}, \frac{i_1}{p} \right]
\times \left[ \frac{i_2-1}{p}, \frac{i_2}{p} \right] \times \cdots
\times \left[ \frac{i_q-1}{p}, \frac{i_q}{p} \right]
\]
for $1 \leq i_j \leq p$ with $1\leq j \leq q$, and $p>0$.
Since $\sigma$ is continuously differentiable on the compact set
$I_k$, we have that $\displaystyle C_0 =
\sup_{\VEC{x} \in I_k} \|\diff \sigma(\VEC{x}) \| < \infty$.  Hence,
from the Mean Value Theorem for functions of several variables, we have that
$\displaystyle
\| \sigma(\VEC{x}) - \sigma(\VEC{y})\| \leq \frac{C_0 \sqrt{q}}{p}$ for
all $\VEC{x},\VEC{y} \in I_{i_1,i_2,\ldots,i_q}$.  Thus
$\sigma(I_{i_1,i_2,\ldots,i_q})$ is a subset of a closed cube
$\tilde{I}_{i_1,i_2,\ldots,i_q} \subset \RR^n$ of
volume $\displaystyle \left(\frac{C_0 \sqrt{q}}{p}\right)^n$ and
$\displaystyle R = \sigma(I_q) \subset
\bigcup_{\substack{1 \leq i_j \leq p\\1\leq j \leq q}}
\tilde{I}_{i_1,i_2,\ldots,i_q}$.

\stage{b} Let $\displaystyle
G_p = \{ \VEC{x} \in \RR^n : \dist{\VEC{x}}{R} < 2/p \}$.  We
construct a function $\displaystyle g_p:\RR^n \to \RR$ of
class $\displaystyle C^\infty$ \footnote{This is more than needed but
it comes for free.} such that:
{\renewcommand{\labelenumi}{\roman{enumi}.}
\begin{enumerate}
\item $0 \leq g_p(\VEC{x}) \leq 1$ for all $\displaystyle \VEC{x} \in \RR^n$,
\item $g_p(\VEC{x}) = 1$ for all
$\displaystyle \VEC{x} \in \RR^n \setminus G_p$,
\item there exists an open set $\tilde{G}_p$ with
$R \subset \tilde{G}_p \subset G_p$ and
$g_p(\VEC{x}) = 0$ for all $\VEC{x} \in \tilde{G}_p$, and
\item there exists a constant $C_1$ that does not depend on $p$ such that
$\displaystyle \left|\pdydx{g_p}{x_s}(\VEC{x})\right| \leq C_1 p$
for all $\displaystyle \VEC{x} \in \RR^n$ and $1\leq s \leq n$.
\end{enumerate}
}

Choose a function $\displaystyle \psi:\RR^n \to \RR$ of class
$\displaystyle C^\infty$ such that
$0 \leq \psi(\VEC{x}) \leq 1$ for all $\displaystyle \VEC{x} \in \RR^n$,
$\psi(\VEC{x}) = 0$ for $\|\VEC{x}\| \leq 1/2$ and
$\psi(\VEC{x}) = 1$ for $\|\VEC{x}\| \geq 1$
\footnote{As we have seen, we can use a partition of unity to
construct such a function.}.
For $p>0$, the function $g_p$ is defined by
\[
g_p(\VEC{x}) = \prod_{\VEC{z} \in R_p}
\psi\left(p\left(\VEC{x} - (2p)^{-1} \VEC{z}\right)\right)
\]
for $\displaystyle \VEC{x} \in \RR^n$, where
$\displaystyle R_p \equiv \{ \VEC{z} \in \ZZ^n :
\dist{(2p)^{-1}\VEC{z}}{R} \leq 1/p\}$.

We prove that the product in the definition of $g_p$ is finite.
Since $\displaystyle R \subset \RR^n$ is compact, there exist
intervals $[a_i,b_i]$ for $1 \leq i \leq n$ such that
$R \subset B \equiv [a_1,b_1]\times[a_2,b_2] \times \cdots \times [a_n,n_n]$.
We have that
$\displaystyle R_p \subset B_p \equiv
\{ \VEC{z} \in \ZZ^n : \dist{(2p)^{-1}\VEC{z}}{B} \leq 1/p\}$.
Since
\[
  \dist{\frac{1}{2p} \,\VEC{z}}{B} \leq \frac{1}{p}
\Rightarrow  a_i - \frac{1}{p} \leq \frac{1}{2p} \, z_i \leq b_i + \frac{1}{p}
\Rightarrow 2p a_i - 2 \leq z_i \leq 2 p b_i + 2
\]
for $1\leq i \leq n$, there is a finite number of points in
$B_p$ and therefore in $R_p$.

Therefore, the functions $g_p$ are of class $\displaystyle C^\infty$ and
$0 \leq g_p(\VEC{x}) \leq 1$ for all $\displaystyle \VEC{x} \in \RR^n$
because $0 \leq \psi(\VEC{x}) \leq 1$ for all
$\displaystyle \VEC{x}\in \RR^n$.  Thus (i) is satisfied.

Given $\VEC{x}$ such that $\displaystyle \dist{\VEC{x}}{R} < (4p)^{-1}$, there
exists at least one $\VEC{z} \in R_p$ such that
$\displaystyle \| \VEC{x} - (2p)^{-1} \VEC{z} \| \leq (2p)^{-1}$.
For this value of $\VEC{z}$, we have that
$\displaystyle \| p\left(\VEC{x} - (2p)^{-1} \VEC{z}\right) \| \leq 1/2 $.
It follows from the definition of $\psi$ that
$\displaystyle \psi\left(p\left(\VEC{x} - (2p)^{-1}
\VEC{z}\right)\right) =0$ and so $g_p(\VEC{x}) = 0$.
Therefore, we may take $\displaystyle \tilde{G}_p = \{ \VEC{x} \in \RR^n :  
\dist{\VEC{x}}{R} < (4p)^{-1}\}$.  Hence (iii) is satisfied.

Given $\VEC{x}$ such that $\displaystyle \dist{\VEC{x}}{R} \geq 2/p$ and
$\VEC{z} \in R_p$, we have that
$\displaystyle \|\VEC{x} - (2p)^{-1}\VEC{z}\|
\geq \|\VEC{x} - \VEC{y}\| - \|\VEC{y} - (2p)^{-1}\VEC{z}\|$ for all
$\VEC{y} \in R$ implies that $\displaystyle \|\VEC{x} - (2p)^{-1}\VEC{z}\|
\geq 2/p - 1/p = 1/p$ because there is at least one $\VEC{y} \in R$
with $\|\VEC{y} - (2p)^{-1}\VEC{z}\| \leq 1/p$ since $\VEC{z} \in R_p$.
Hence $\displaystyle \| p\left(\VEC{x} - (2p)^{-1} \VEC{z}\right) \| \geq 1$
for all $\VEC{z} \in R_p$.
It follows from the definition of $\psi$ that
$\displaystyle \psi\left(p\left(\VEC{x} - (2p)^{-1} \VEC{z}\right)\right) =1$
for all $\VEC{z} \in R_p$ and so $g_p(\VEC{x}) = 1$.
Therefore, we have that $g_p(\VEC{x}) = 1$ for
$\displaystyle \VEC{x} \in \RR^n \setminus G_p$.  Hence (ii) is
satisfied.

Finally, we prove that item (iv) is satisfied.  We note that
\begin{equation}   \label{propCrit2Eq1}
\pdydx{g_p}{x_i}(\VEC{x})
= \sum_{\VEC{z} \in R_p} \bigg(
\bigg( \prod_{\substack{\VEC{w} \in R_p \\
\VEC{w} \neq \VEC{z}}}
\psi\left(p\left(\VEC{x} - (2p)^{-1} \VEC{w}\right)\right) \bigg)
\pdfdx{ \psi\left(p\left(\VEC{x} - (2p)^{-1}\VEC{z}\right)\right)}{x_i}
\bigg) \ .
\end{equation}
Let $\displaystyle M = \max_{1\leq i \leq n}\, \sup_{\VEC{x} \in \RR^n}
\left|\pdydx{\psi}{x_i}(\VEC{x})\right|$.  We have that $M < \infty$
because $\displaystyle \pdydx{\psi}{x_i}(\VEC{x}) = 0$ for
$\|\VEC{x}\| \leq 1/2$ or $\|\VEC{x}\|\geq 1$.  We get that
\[
\left|\pdfdx{ \psi\left(p\left(\VEC{x} - (2p)^{-1}\VEC{z}\right)\right)}{x_i}
\right| = p \left|\pdydx{\psi}{x_i}\left(p\left(\VEC{x} - (2p)^{-1}
\VEC{z}\right)\right) \right| \leq p M
\]
for all $\displaystyle \VEC{x} \in \RR^n$,
$\displaystyle \VEC{z} \in \ZZ^n$ and $1 \leq i \leq n$.  Hence
\begin{align*}
&\bigg| \bigg( \prod_{\substack{\VEC{w} \in R_p\\ \VEC{w} \neq \VEC{z}}}
\psi\left(p\left(\VEC{x} - (2p)^{-1} \VEC{w}\right)\right) \bigg)
\pdfdx{ \psi\left(p\left(\VEC{x} - (2p)^{-1}\VEC{z}\right)\right)}{x_i}\bigg|
\\
&\qquad = \prod_{\substack{\VEC{w} \in R_p\\ \VEC{w} \neq \VEC{z}}}
\bigg| \psi\left(p\left(\VEC{x} - (2p)^{-1} \VEC{w}\right)\right) \bigg| \,
\left| \pdfdx{ \psi\left(p\left(\VEC{x} -
(2p)^{-1}\VEC{z}\right)\right)}{x_i}\right| 
\leq p M
\end{align*}
for all $\VEC{x} \in \RR^n$ and all $\VEC{z} \in R_p$
because $0 \leq \psi(\VEC{y}) \leq 1$ for all $\displaystyle \VEC{y} \in \RR^n$.

Moreover, the number of terms in the sum in
(\ref{propCrit2Eq1}) does not grow with $p$.  There are at most
$\displaystyle 9^n$ terms independently of the value of $p$.  To understand why
it is so, we first note that
$\displaystyle \psi\left(p\left(\VEC{y} - (2p)^{-1}\VEC{z}\right)\right) = 1$
for all $\displaystyle \VEC{y} \in \{ \VEC{y} \in \RR^n :
\|\VEC{y} - (2p)^{-1} \VEC{z}\| > 1/p\}$.  Thus, the partial derivatives
of $\displaystyle \psi\left(p\left(\VEC{y} - (2p)^{-1}\VEC{z}\right)\right)$
are null in the set $\displaystyle \VEC{y} \in \{ \VEC{y} \in \RR^n :
\|\VEC{y} - (2p)^{-1} \VEC{z}\| > 1/p$.
For $\VEC{x}$ given and fixed, the only terms possibly non null in
the sum in (\ref{propCrit2Eq1}) are therefore those associated to the
values of $\VEC{z}$ such that
$\displaystyle \|\VEC{x} - (2p)^{-1} \VEC{z}\| \leq 1/p$.
Suppose that $\displaystyle \VEC{z}_j \in \ZZ^n$ satisfies
$\displaystyle \|\VEC{x} - (2p)^{-1} \VEC{z}_j\| \leq 1/p$ for
$j = 1,2$.
Then $\displaystyle \|(2p^{-1}\VEC{z}_1 - (2p)^{-1}\VEC{z}_2 \| \leq 
\|\VEC{x} - (2p)^{-1} \VEC{z}_1\| + \|\VEC{x} - (2p)^{-1} \VEC{z}_2\| \leq 2/p$
implies that $\|\VEC{z}_1 - \VEC{z}_2 \| \leq 4$.  Thus
$-4 \leq z_{1,i} - z_{2,i} \leq 4$ for $1 \leq i \leq n$.  Since the
$\VEC{z}_j$ are in $\displaystyle \ZZ^n$, there are at most $9$
acceptable values for each coordinates.  In fact, we reach the maximum
of $9$ possible values only when $x_i \in \ZZ$ because we then have
$x_i - 4 \leq  z_{j,i} \leq x_i + 4$ for $j = 1,2$.
Even in that case, these $\VEC{z}$ may not all be included in the sum
because they may not be elements of $R_p$. 

It follows from (\ref{propCrit2Eq1}) that
$\displaystyle \left|\pdydx{g_p}{x_i}(\VEC{x})\right| \leq C_1 p$ for
all $\VEC{x} \in \RR^n$ where $C_1 = 9^n M$.

\stage{c} We show that $G_p$ can be covered by closed cubes whose total
volume is at most $\displaystyle \frac{(4 + C_0 \sqrt{q})^n}{p^2}$.

Given $\VEC{x} \in G_p$,
choose $\displaystyle \VEC{y} \in \tilde{I}_{i_1,i_2,\ldots,i_q}$
for some indices $1 \leq i_1,i_2,\ldots,i_q\leq p$ such that
$\|\VEC{x} - \VEC{y}\| < 2/p$.  Then $\VEC{x}$ is an element of
a closed cube
$\breve{I}_{i_1,i_2,\ldots,i_q}  \supset \tilde{I}_{i_1,i_2,\ldots,i_q}$
whose sides are of length less or equal to
$\displaystyle \frac{4}{p} + \frac{C_0\sqrt{q}}{p}$.  The volume of
this cube is $\displaystyle \left(\frac{4 + C_0 \sqrt{q}}{p}\right)^n$.  Hence
$G_p$ is covered by closed cubes whose total volume is at most
$\displaystyle p^q \left(\frac{4+C_0 \sqrt{q}}{p}\right)^n =
\frac{(4+C_0 \sqrt{q})^n}{p^{n-q}} \leq \frac{(4+C_0 \sqrt{q})^n}{p^2}$.

\stage{d}
Since $G_1$ is bounded because $R$ is compact, we may choose 
a bounded open set $G$ such that $G \supset \overline{G_1}$.
It reminds to prove (3.e) of Definition~\ref{defnNegligible} to
complete the proof that $\displaystyle \{G_j\}_{j > 0}$ and
$\displaystyle \{g_j\}_{j > 0}$ are associated to the
definition of negligible sets in $S$.

Suppose that $\omega$ is a $(n-1)$-form on the open set $G$.
We may assume that
\[
\omega =
\sum_{s=1}^n f_s \df{x_1}\wedge \df{x_2} \wedge \widehat{\df{x_s}}
\wedge \ldots \wedge \df{x_n} \ ,
\]
where $f_s:G \to \RR$ for $1\leq s \leq n$ are continuous.
We also have that
\[
  \df{g_p} = \sum_{s=1}^n \pdydx{g_p}{x_s} \df{x_s}
\]
on $G$.  Hence
\[
\df{g_p} \wedge \omega
= \sum_{s=1}^n (-1)^{s-1} \left( \pdydx{g_p}{x_s}
f_s \right) \df{x_1} \wedge \df{x_2} \wedge \ldots \wedge \df{x_n}
\]
on $S$.  It follows from from (3) of Proposition~\ref{prop_functCcS}
that
\begin{equation} \label{propCrit2Eq2}
\eta_{|\df{g_p}\wedge \omega|}(S \cap G)
= \int_{S \cap G} \left|
\sum_{s=1}^n (-1)^{s-1} \pdydx{g_p}{x_s}(\VEC{x}) f_s(\VEC{x})
\right| \dx{\VEC{x}}
\end{equation}
because $\overline{G}$ is compact.
We have that $\displaystyle \left|\pdydx{g_p}{x_s}(\VEC{x})\right| \leq C_1 p$
for all $\VEC{x} \in G$ and $p>0$.
Since $f_s:G\to \RR$ is continuous on the compact set
$\overline{G_1} \subset G$ for $1\leq s \leq n$, there exists a
constant $C_2$ such that $|f_s(\VEC{x})| \leq C_2$ for all
$\VEC{x} \in S \cap G_1$ and $1 \leq s \leq n$.  Since
$\displaystyle \supp \pdydx{g_p}{x_s} \subset \overline{G_p}
\subset \overline{G}_1$ for all $p>0$, we get from (c) above that
\[
\eta_{|\df{g_p}\wedge \omega|}(S \cap G)
= n C_1 C_2 p \int_{S \cap G_p} \dx{\VEC{x}}
\leq n C_1 C_2 p\, \frac{(4 + C_0 \sqrt{q})^n}{p^2} 
\to 0
\]
as $p \to \infty$.
\end{proof}

\subsubsection{Second Approach}

The first step is to expand the definition of local charts.

\begin{defn}
Let $W$ be an open subset of $\RR^{k-1}$ and $f:W \to \RR$ be a
continuous function.  A point
$\VEC{p} \in G_{f,W} = \{(\VEC{x},f(\VEC{x})) : \VEC{x} \in W \}$
is a {\bfseries regular point}\index{Regular Point} if there exists an
open set $V \subset W$ such that $G_{f,V}$ is a $k-1$-dimensional
manifold.  The set of regular points is denoted $R_{f,W}$.
The points in $S_{f,W} \equiv G_{f,W} \setminus R_{f,W}$ are called
{\bfseries singular points}\index{Singular Point}.

The function $f$ is said to be
{\bfseries piecewise smooth}\index{Piecewise Smooth}
if $f^{-1}(S_{f,W}) \subset \RR^{k-1}$ is a set of measure zero and,
for each compact set $K \subset W$, there exists a constant $L$ such that
$|f(\VEC{x}_1) - f(\VEC{x}_2)| \leq L \|\VEC{x}_1 - \VEC{x}_2\|$ for
all $\VEC{x}_1,\VEC{x}_2 \in K$.
\end{defn}

\begin{defn}\label{manifold_charts_wc}
Let $S$ be a subset of $\displaystyle \RR^n$.  We assume that the
topology on $S$ is the induced topology from $\displaystyle \RR^n$.  A
{\bfseries local chart}\index{Local Chart} is either defined as in
Definition~\ref{manifold_charts} or defined by the following three
objects:
\begin{enumerate}
\item an open subset $U$ of $S$,
\item an open subset $W = W_1 \times W_2 \subset \RR^{k-1} \times \RR$
with a piecewise smooth function $f:W_1 \to W_2$, and
\item a homeomorphism $\phi:W\rightarrow U$
\end{enumerate}
such that $\phi(G_{f,W_1}) = U \cap \partial S$ and
$\phi\big( \{ (\VEC{x},y) \in W : y < f(\VEC{x}) \}\big) = U \cap \Int S$.
This local chart is denoted $(W, U, \phi)$ (Figures~\ref{MANIFOLD6}).
\end{defn}

\pdfF{manifolds/stokesPrf3}{A local chart for a manifold with
piecewise smooth boundary}{$(W,U,\phi)$ is a local chart of a manifold
with a piecewise smooth boundary.}{MANIFOLD6}

We can now defined atlases for a set $S$ exactly as we did in
Definition~\ref{manifold_atlas} except that we now also accept local
charts as defined above.  Equivalent atlases, differential
structures, maximal atlases and admissible local charts are defined as in 
Definition~\ref{defnDiffStruc}.  Finally, the definition of manifold
given in Definition~\ref{manifold_def} stays the same except that, as we
said above, we now also accept local charts as in
Definition~\ref{manifold_charts_wc}.  To distinguish the
manifolds that accept local charts as in Definition~\ref{manifold_charts_wc}
from those that only accept local charts as in
Definition~\ref{manifold_charts}, we call the former
{\bfseries manifolds with piecewise smooth boundary}\index{Manifolds
with Piecewise Smooth Boundary}. 

Theorem~\ref{TheStokesTh} can be generalized to manifolds with
piecewise smooth boundary.

\begin{theorem}[Stokes' Theorem with Piecewise Smooth
Boundary] \label{GenStokesThV2}
Let $S$ be a closed and oriented $k$-dimensional manifold with
piecewise smooth boundary.  Let $\omega$ be a differential
$(k-1)$-form of class $\displaystyle C^1$ defined in an open set of
$\RR^n$ containing $\overline{S}$.  Then
$\displaystyle \int_S \df{\omega} = \int_{\partial S} \omega$.
\end{theorem}

\begin{proof}[Proof (Sketch)]
The proof of the theorem used arguments similar to those introduced in
the proofs of Theorem~\ref{GenStokesTh} and Proposition~\ref{propCrit2}.

Since $S$ is closed in $\RR^n$, we have that 
$\overline{S} \setminus \Int S = \partial S$.

Using an atlas for $S$ and a partition of unity subordinate to this
atlas, the problem is reduced to proving the theorem on a local 
chart $(W,U,\phi)$ of $S$ as in Definition~\ref{manifold_charts_wc}.
Let $K$ be a compact subset of the set $S_{f,W}$ of singular points.
The first step is to construct functions $g_p$ as in the proof of
Proposition~\ref{propCrit2} with $R$ replaced by $K$ and to show
that the volume of $G_p\cap W$ times
$\displaystyle \max_{1\leq i \leq  k}
\sup_{\VEC{x} \in \RR^k} \bigg|\pdydx{g_p}{x_i}(\VEC{x}) \bigg|$ is
bounded by $C/p$ for a constant $C$ independent of $p$.

The second step is to prove the conclusion of the theorem on
this local chart using techniques similar to those used in the proof
of Theorem~\ref{GenStokesTh}.
\end{proof}

\section{The Volume Element}\label{manifVolume}

To talk about the volume element, we have first to expand the
definition of inner product to tangent spaces of manifolds.

\begin{defn}
Let $S$ be a $k$-dimensional manifold of class $\displaystyle C^1$.
Given $\VEC{u} \in S$, an
{\bfseries inner product}\index{Tensor!Inner Product} on $\TS_{\VEC{u}} S$
is a symmetric and positive definite $2$-tensor
$\displaystyle \tau_{\VEC{u}} \in \T^2\left( \TS_{\VEC{u}} S\right)$
\end{defn}

Consider
$\displaystyle \tau : S \to \bigcup_{\VEC{u}\in S} \T^2(\TS_{\VEC{u}} S)$
where $\tau(\VEC{u})$ is an inner product on
$\displaystyle \TS_{\VEC{u}} S$ for all $\VEC{u} \in S$.
If $(W,U,\phi)$ is a local chart of $S$, then
$\displaystyle (\phi^\ast(\tau))(\VEC{w})$ is an inner product on
$\TS_{\VEC{w}} W$ defined by
\begin{align*}
\phi^\ast(\tau)(\VEC{w})\big((\VEC{w}, \VEC{y}_1),(\VEC{w},\VEC{y}_2)\big)
&= \phi^\ast\big(\tau(\phi(\VEC{w}))\big)\big((\VEC{w}, \VEC{y}_1),
(\VEC{w},\VEC{y}_2)\big) \\
&= \tau(\phi(\VEC{w}))\big(\phi_\ast(\VEC{w}, \VEC{y}_1),
\phi_\ast(\VEC{w},\VEC{y}_2)\big)
\end{align*}
for all $\displaystyle (\VEC{w},\VEC{y}_1), (\VEC{w},\VEC{y}_2)
\in \TS_{\VEC{w}} W \cong \RR^k$ and $\VEC{w} \in W$.
It follows from Theorem~\ref{stokesBT} that we can express $\tau$ locally as
$\displaystyle \phi^\ast(\tau) = \sum_{1 \leq i_1 \leq i_2\leq k} \tau_{i_1,i_2}
\df{w_{i_1}} \otimes \df{w_{i_2}}$ 
where $\displaystyle \tau_{i_1,i_2}:W \to \RR$ for all indices.
We say that $\tau$ is of class $\displaystyle C^j$ on $U$
if each $\displaystyle \tau_{i_1,i_2}$ is of class $\displaystyle C^j$
on $W$.  If this is true for all local chart $(W,U,\phi)$, then 
we say that $\tau$ is of class $\displaystyle C^j$ on $S$.

\begin{defn} \label{defnCmInnerP}
Let $S$ be a $k$-dimensional manifold of class $\displaystyle C^j$, and
$\displaystyle \tau : S \to \bigcup_{\VEC{u}\in S} \T^2(\TS_{\VEC{u}} S)$
be a map such that $\tau(\VEC{u})$ is an inner product on
$\displaystyle \TS_{\VEC{u}} S$ for all $\VEC{u} \in S$.
If $\tau$ is of class $\displaystyle C^j$ on $S$, then we say that the
inner product on $\TS_{\VEC{u}} S$ given by
$\ps{}{}_{\VEC{u}} =  \tau(\VEC{u})$ is of class $\displaystyle C^j$ on $S$.
\end{defn}

\begin{prop}  \label{propEIPonS}
Let $S$ be a $k$-dimensional manifold of class $\displaystyle C^j$.
Then there exists a map
$\displaystyle \tau : S \to \bigcup_{\VEC{u}\in S} \T^2(\TS_{\VEC{u}} S)$
of class $\displaystyle C^j$ such that $\tau(\VEC{u})$ is an inner product on
$\displaystyle \TS_{\VEC{u}} S$ for all $\VEC{u} \in S$.
\end{prop}

\begin{proof}
Suppose that
$\displaystyle \A = \left\{ (W_\alpha,U_\alpha,
\phi_\alpha)\right\}_{\alpha\in A}$ is an atlas of
orientation preserving local charts for $S$ and that
$\displaystyle \{\psi_j\}_{j\in \NNp}$ is a partition of unity
subordinate to the atlas $\A$.  Given $\displaystyle j \in \NNp$, we
assume that $\alpha_j \in A$ is an index such that
$\supp \psi_j \subset U_{\alpha_j}$.

For each local chart $(W_\alpha,U_\alpha,\phi_\alpha)$, we define 
$\displaystyle \tau_{U_\alpha} : U_\alpha \to \bigcup_{\VEC{u}\in U_\alpha}
\T^2(\TS_{\VEC{u}} S)$ as it follows.  For each $\VEC{u} \in U_\alpha$,
$\tau_{U_\alpha}(\VEC{u})$ is the inner product on $\TS_{\VEC{u}} S$
defined by
\[
\ps{(\VEC{u},\VEC{x}_1)}{(\VEC{u},\VEC{x}_2)}_{\VEC{u}}
= \tau_{U_\alpha}(\VEC{u})\big((\VEC{u},\VEC{x}_1),(\VEC{u},\VEC{x}_2)\big)
= \ps{(\VEC{w}.\VEC{y}_1)}{(\VEC{w},\VEC{y}_2)}_{\VEC{w}}
= \ps{\VEC{y}_1}{\VEC{y}_2} \ ,
\]
where $(\phi_\alpha)_\ast(\VEC{w}.\VEC{y}_i) = (\VEC{u},\VEC{x}_i)$ for
$i =1,2$, and the last inner product is the standard inner
product on $\displaystyle \RR^k$.  Note that $\tau_{U_\alpha}$ is of class
$\displaystyle C^j$ because it is defined from $\phi_\alpha$.  In
fact, $\displaystyle \phi_\alpha^\ast(\tau_{U_\alpha})
= \sum_{i=1}^k \df{w_i} \otimes \df{w_i}$.

We have that $\displaystyle \tau = \sum_{j\in \NN} \psi_j \tau_{U_{\alpha_j}}$
satisfies the conclusion of the proposition.  Since the sum is
finite on every compact subset of $S$, we have that $tau$ is also of
class $\displaystyle C^j$.  In particular, the sum is finite at
every point $\VEC{u} \in S$.
\end{proof}

\begin{rmk}
A smooth $k$-dimensional manifold $S$ with a smooth inner product on
$\TS_{\VEC{u}} S$ is called a
{\bfseries Riemannian manifold}\index{Riemannian Manifold}.
We will study this type of manifolds in greater detail
in Chapter~\ref{ChapRGeom}.
\end{rmk}

Let $S$ be an oriented $k$-dimension manifold and let $\mu_{\VEC{u}}$ be
the selected orientation on $\TS_{\VEC{u}} S$ for each $\VEC{u} \in S$.  Given
an inner product $\ps{}{}_{\VEC{u}}$ on $\TS_{\VEC{u}} S$ of class
$\displaystyle C^1$, we have seen in Section~\ref{sectOrientRn} that
there exists a unique
$\displaystyle \nu(\VEC{u}) \in \Omega^k(\TS_{\VEC{u}} S)$ such that
\begin{equation} \label{volElmCond}
\nu(\VEC{u})\big((\VEC{u},\VEC{x}_1),(\VEC{u},\VEC{x}_2), \ldots,
(\VEC{u},\VEC{x}_k)\big) = 1
\end{equation}
for all orthonormal basis
$\displaystyle \left\{ (\VEC{u},\VEC{x}_i) \right\}_{i=1}^k$ of
$\TS_{\VEC{u}} S$ such that
$[(\VEC{u},\VEC{x}_1),(\VEC{u},\VEC{x}_2), \ldots,
(\VEC{u},\VEC{x}_k)]=\mu_{\VEC{u}}$.

\begin{defn} \label{defnVolElemD1}
Let $S$ be an oriented $k$-dimensional manifold of class $\displaystyle C^1$.
The unique continuous differential $k$-form on $S$ satisfying
(\ref{volElmCond}) is called the
{\bfseries volume element on $\mathbf S$}\index{Volume Element} and
is often denoted $\dx{V}$.  The {\bfseries volume}\index{Volume} of
the manifold $S$ is defined as $\displaystyle \int_S \dx{V}$.
\end{defn}

The notation $\dx{V}$ is confusing because it designate the
differential $k$-form $\nu$ that may not be the derivative
of any differential $(k-1)$-form.

When $S$ is a surface (a $2$-dimensional manifold), we usually replace the
word volume by {\bfseries surface}\index{Surface} and we write
$\dx{A}$ or $\dx{S}$ instead of $\dx{V}$.  Similarly, when $S$ is a
curve (a $1$-dimensional manifold), we 
replace the word volume by {\bfseries length}\index{Length} and we
write $\dx{s}$ instead of $\dx{V}$.

We will not prove the existence of volume elements for oriented
$k$-dimensional manifolds in general.  Instead, we will provide the
volume element for some of the low dimensional manifolds.

\begin{egg}
Suppose that $S$ is an oriented $1$-dimensional    \label{eggVolElem1D}
manifold of class $\displaystyle C^1$ in $\displaystyle \RR^n$, and
that the orientation on $S$ is given by
$\displaystyle \mu_{\VEC{u}} = [ (\VEC{u}, \VEC{t}(\VEC{u})) ]$ where 
$(\VEC{u}, \VEC{t}(\VEC{u})) \in \TS_{\VEC{u}} S$ is a unit vector with
respect to the inner product $\ps{}{}_{\VEC{u}}$ on $\TS_{\VEC{u}} S$
defined by
$\ps{ (\VEC{u},\VEC{x})}{(\VEC{u},\VEC{y})}_{\VEC{u}}
= \ps{\VEC{x}}{\VEC{y}}$ for all $(\VEC{u},\VEC{x}), (\VEC{u},\VEC{y})
\in \TS_{\VEC{u}} S$ and the standard inner product $\ps{}{}$ on
$\displaystyle \RR^n$.

Let $\displaystyle \nu(\VEC{u}) \in \Omega^1(\TS_{\VEC{u}} S)$ be the
alternating $1$-tensor defined by
\[
\nu(\VEC{u})\big((\VEC{u},\VEC{x})\big)
= \ps{ (\VEC{u},\VEC{x})}{(\VEC{u},\VEC{t}(\VEC{u}))}_{\VEC{u}}
= \ps{\VEC{x}}{\VEC{t}(\VEC{u})}
\]
for all $(\VEC{u},\VEC{x}) \in \TS_{\VEC{u}} S$.
Then $\displaystyle \nu(\VEC{u})\big( (\VEC{u},\VEC{t}(\VEC{u}) ) \big)
= \ps{\VEC{t}(\VEC{u})}{\VEC{t}(\VEC{u})} = 1$
for all $\VEC{u} \in S$.

To prove that $\nu$ is a continuous differential $1$-form on $S$, we consider
an orientation preserving local chart $(W,U,\phi)$ of
$S$.  Since $\phi$ is orientation preserving, we have that
$\displaystyle \VEC{t}(\VEC{u}) = \|\phi'(w)\|^{-1} \phi'(w)$
for $\VEC{u} = \phi(w)$ with $w \in W$.  Hence
\begin{align}
\phi^\ast(\nu)(w)\big( (w,y) \big)
&= \phi^\ast\big(\nu(\phi(w))\big)\big( (w,y) \big)
= \nu(\phi(w))\big( \phi_\ast(w,y)\big) \nonumber \\
&= \nu(\phi(w))\big( \phi(w), y\, \phi'(w)\big)
= \ps{y\, \phi'(w)}{\VEC{t}(\phi(w))} \nonumber \\
&= y \|\phi'(w)\| = \left(\left(\|\phi'\| \df{w}\right)(w)\right) (w,y)
\label{tunEq1}
\end{align}
for all $(w,y) \in \TS_w W \cong \RR$ and $w \in W$.  Thus
$\phi^\ast(\nu)$ is continuous on $W$.  Note that $\phi'(w) \neq \VEC{0}$
for all $w \in W$ since $\phi'(w)$ is a rank one by definition of
local charts.

Therefore, by uniqueness of the volume element, $\dx{s} = \nu$.
It also follows from (\ref{tunEq1}) that
\begin{equation} \label{defn1Dvolelem}
  \phi^\ast \dx{s} = \|\phi'\| \df{w} \ .
\end{equation}

The element $(\VEC{u},\VEC{t}(\VEC{u})) \in \TS_{\VEC{u}} S$, or
simply $\VEC{t}(\VEC{u})$ when there is no confusion about the base
$\VEC{u}$, is called a
{\bfseries tangent unit vector}\index{Tangent Unit Vector} to $S$ at
$\VEC{u}$.
\end{egg}

Before going on with reading the rest of this section, the reader
should review the second part of Remark~\ref{rmkrestrictdxi}.
The next results and examples will be useful for our study of the
classical vector calculus in the next chapter.

\begin{theorem} \label{manifCXYZT}
Let $C$ be a smooth curve in $\displaystyle \RR^3$ (i.e.\ an
oriented $1$-dimensional manifold of class $\displaystyle C^1$) and let
$\mu_{\VEC{u}} = [ (\VEC{u}, \VEC{t}(\VEC{u})) ]$ be the chosen
orientation on $\TS_{\VEC{u}} S$ for all $\VEC{u} \in C$ where
$(\VEC{u},\VEC{t}(\VEC{u}))$ is of norm one; thus
$\VEC{t}(\VEC{u})$ is a tangent unit vector to $C$ at $\VEC{u} \in C$.
Then 
\begin{equation} \label{manifCXYZeq1}
\dx{s} = t_1 \df{x_1} + t_2 \df{x_2} + t_3 \df{x_3} \ .
\end{equation}
Moreover, on $C$ (and only on $C$), we have
\begin{equation} \label{manifCXYZeq2}
t_i \dx{s} = \df{x_i}
\end{equation}
for $1\leq i \leq 3$.
\end{theorem}

\begin{proof}
For all $(\VEC{u}, \VEC{x})$ in $\TS_{\VEC{u}} C$, we have
\begin{align*}
\dx{s}(\VEC{u})(\VEC{u}, \VEC{x})
&= \ps{\VEC{x}}{\VEC{t}(\VEC{u})}
= x_1 t_1(\VEC{u}) + x_2 t_2(\VEC{u}) + x_3 t_3(\VEC{u}) \\
&= t_1(\VEC{u}) \df{x_1}(\VEC{u})(\VEC{u}, \VEC{x}) + t_2(\VEC{u})
\df{x_2}(\VEC{u})(\VEC{u}, \VEC{x}) + t_3(\VEC{u})
\df{x_3}(\VEC{u})(\VEC{u}, \VEC{x})
\end{align*}
where $\ps{}{}$ is the standard inner product on $\displaystyle \RR^3$.
Since $\VEC{u}\in C$ is arbitrary, we get (\ref{manifCXYZeq1}).

To prove (\ref{manifCXYZeq2}), we first note that
$\VEC{x} = \epsilon \|\VEC{x}\|\, \VEC{t}(\VEC{u})$ for all vector
$(\VEC{u}, \VEC{x}) \in \TS_{\VEC{u}} C$ (vectors tangent to the curve $C$ at
$\VEC{u}$) where $\epsilon = 1$ if $\VEC{x}$ points in the same direction
than $\VEC{t}(\VEC{u})$ and $\epsilon = -1$ otherwise.  Hence, 
$\displaystyle 
\dx{s}(\VEC{u}) (\VEC{u}, \VEC{x})
= \ps{ \epsilon \|\VEC{x}\|\,\VEC{t}(\VEC{u})}{\VEC{t}(\VEC{u})}
= \epsilon \|\VEC{x}\|$
for all $(\VEC{u}, \VEC{x})$ in $\TS_{\VEC{u}} C$. We already knew
that from the previous example.  It follows that
\[
\ps{\VEC{q}}{\VEC{t}(\VEC{u})} \dx{s}(\VEC{u})(\VEC{u},\VEC{x}) =
\ps{\VEC{q}}{\VEC{t}(\VEC{u})}\, \epsilon \| \VEC{x}\|
= \ps{\VEC{q}}{ \epsilon \| \VEC{x}\| \VEC{t}(\VEC{u})}
= \ps{\VEC{q}}{\VEC{x}}
\]
for all $(\VEC{u}, \VEC{x})$ in $\TS_{\VEC{u}} C$ and
$\displaystyle \VEC{q} \in \RR^3$.
With $\VEC{q} = \VEC{e}_i$ for $1\leq i \leq 3$, we get that
\[
t_i(\VEC{u}) \dx{s}(\VEC{u})(\VEC{u},\VEC{x})
= \ps{\VEC{e}_i}{\VEC{x}} = x_i
= \df{x_i}(\VEC{u})(\VEC{u},\VEC{x})
\]
for all $(\VEC{u}, \VEC{x})$ in $\TS_{\VEC{u}} C$.
This yields (\ref{manifCXYZeq2}).
\end{proof}

\begin{rmk}
If we use the local chart introduced in Example~\ref{eggVolElem1D}, we
can also prove (\ref{manifCXYZeq2}).  Recall that
$\displaystyle \VEC{t}(\VEC{u}) = \|\phi'(w)\|^{-1} \phi'(w)$
for $\VEC{u} \in U$ and $w\in W$ such that $\phi(w) = \VEC{u}$.

We have from Proposition~\ref{propAstoR} (1) that
$\displaystyle \phi^\ast(\df{x_i})(w) = \phi_i'(w) \df{w}$.
This can also be proved from
\begin{align*}
\phi^\ast(\df{x_i})(w)\big((w,r)\big)
&= \phi^\ast\big( \df{x_i}(\phi(w))\big)\big( (w,r)\big)
= \df{x_i}(\phi(w)) (\phi_\ast(w,r)) \\
&= \df{x_i}(\phi(w)) \big(\phi(w), r \phi'(w)\big)
= r \phi_i'(w) = \phi_i'(w) \df{w}(w)(w,r)
\end{align*}
for all $(w,r) \in \TS_w W$ \footnote{We have that
$\df{w}(w)\big((w,r)\big) = r$ for all $(w,r) \in \TS_w W \cong \RR$
because the projection of $\RR$ on itself is the identity map.}.
Moreover
\begin{align*}
&\phi^\ast\big( t_i \dx{s} \big) (w) \big( (w,r) \big)
= \phi^\ast\big( t_i(\phi(w)) \dx{s}(\phi(w)) \big) \big( (w,r) \big)
= t_i(\phi(w)) \dx{s}(\phi(w)) \big( \phi_\ast(w,r) \big)) \\
&\qquad = t_i(\phi(w)) \dx{s}(\phi(w)) \big( (\phi(w),r\phi'(w)) \big)
= t_i(\phi(w)) \ps{r\phi'(w)}{\VEC{t}(\phi(w))} \\
&\qquad = t_i(\phi(w))\, r\|\phi'(w)\| = r \phi_i'(w) = \phi_i'(w)
\df{w}(w)(w,r) \ .
\end{align*}
Therefore, $t_i \dx{s} = \df{x_i}$ because they have the same local
representation.
\end{rmk}

\begin{rmk}
Suppose that $S$ is an oriented $2$-dimensional   \label{rmkOUNinRR3}
manifold of class $\displaystyle C^1$ in $\displaystyle \RR^3$.  The
outward unit normal to $S$ at $\VEC{u} \in S$ (as defined in
Definition~\ref{manifNormal}) can be computed using the cross product.

If $\displaystyle \mu_{\VEC{u}} =
[(\VEC{u},\VEC{v}_1(\VEC{u})), (\VEC{u},\VEC{v}_2(\VEC{u})) ]$
is the selected orientation on $\TS_{\VEC{u}} S$, where\\
$\displaystyle \{ (\VEC{u},\VEC{v}_1(\VEC{u})),
(\VEC{u},\VEC{v}_2(\VEC{u})) \}$ is an orthonormal basis of
$\TS_{\VEC{u}} S$, then the outward unit normal is
$\displaystyle \VEC{n}_{\VEC{u}}
= \big(\VEC{u}, \VEC{v}_1(\VEC{u}) \times \VEC{v}_2(\VEC{u})\big)$.  The
inner product used to determine orthogonality in the present situation
is the inner product $\ps{}{}_{\VEC{u}}$ on $\TS_{\VEC{u}} S$ defined
above.  If $\VEC{v}_3(\VEC{u})
= \VEC{v}_1(\VEC{u}) \times \VEC{v}_2(\VEC{u})$,
then we have in the following figure
\begin{align*}
[(\VEC{u}, \VEC{v}_3(\VEC{u})),(\VEC{u}, \VEC{v}_1(\VEC{u})),
(\VEC{u}, \VEC{v}_2(\VEC{u}))]
&= [(\VEC{u}, \VEC{v}_1(\VEC{u})),(\VEC{u}, \VEC{v}_2(\VEC{u})),
(\VEC{u}, \VEC{v}_3(\VEC{u}))] \\
&= [(\VEC{u}, \VEC{e}_1),(\VEC{u}, \VEC{e}_2), (\VEC{u}, \VEC{e}_3)] \ .
\end{align*}
Thus $\VEC{n}(\VEC{u}) = \VEC{v}_3(\VEC{u})$ and
$\VEC{n}_{\VEC{u}} = (\VEC{u}, \VEC{v}_3(\VEC{u}))$.
\pdfbox{manifolds/outUN3}
\end{rmk}

\begin{egg}
Suppose that $S$ is a $2$-dimensional oriented    \label{eggVolElem2D}
manifold in $\displaystyle \RR^3$.  Let
$\VEC{n}_{\VEC{u}} = (\VEC{u},\VEC{n}(\VEC{u}))$ be the
outward unit normal to $S$ at $\VEC{u}$ associated to the chosen
orientation $\mu_{\VEC{u}}$ on $\TS_{\VEC{u}} S$ as defined in
Definition~\ref{manifNormal}. 
Let $\nu$ be the differential $2$-form such that
$\displaystyle \nu(\VEC{u}) \in \Omega^2(\TS_\VEC{x} S)$ is the
alternating $2$-tensor defined by
\[
\nu(\VEC{u})\big((\VEC{u},\VEC{x}_1),(\VEC{u},\VEC{x}_2)\big)
= \det \begin{pmatrix} \VEC{x}_1 & \VEC{x}_2 & \VEC{n}(\VEC{u})
\end{pmatrix}
\]
for all $(\VEC{u},\VEC{x}_i) \in \TS_{\VEC{u}} S$ with $i =1,2$.
We have that
$\nu(\VEC{u})\big((\VEC{u},\VEC{v}_1),(\VEC{u},\VEC{v}_2)\big) = 1$
for all orthonormal bases
$\{(\VEC{u},\VEC{v}_1),(\VEC{u},\VEC{v}_2)\}$ of $\TS_{\VEC{u}} S$ with
$[(\VEC{u},\VEC{v}_1),(\VEC{u},\VEC{v}_2)]=\mu_{\VEC{u}}$; namely,
such that
$[\VEC{n}_{\VEC{u}},(\VEC{u},\VEC{v}_1),(\VEC{u},\VEC{v}_2)]
= [(\VEC{u},\VEC{e}_1),(\VEC{u},\VEC{e}_2),(\VEC{u},\VEC{e}_3)]$.
By definition of the cross product, we also have that
\[
\nu(\VEC{u})\big((\VEC{u},\VEC{x}_1),(\VEC{u},\VEC{x}_2)\big)
= \ps{\VEC{x}_1 \times \VEC{x}_2}{\VEC{n}(\VEC{u})}
\]
for all $(\VEC{u},\VEC{x}_i) \in \TS_{\VEC{u}} S$ with for $i =1,2$,
where $\ps{}{}$ is the standard inner product on $\displaystyle \RR^3$.

To prove that $\nu$ is a continuous differential $2$-form on $S$, 
we consider an orientation preserving local chart $(W,U,\phi)$ of $S$. 
Since $\phi$ is orientation preserving, we have that
\[
\left[ \left( \phi(\VEC{w}), \diff \phi(\VEC{w})\, \VEC{e}_1 \right) ,
  \left( \phi(\VEC{w}), \diff \phi(\VEC{w})\, \VEC{e}_2\right) )\right]
= \left[ \left( \phi(\VEC{w}), \pdydx{\phi}{w_1}(\VEC{w}) \right) ,
  \left( \phi(\VEC{w}), \pdydx{\phi}{w_2}(\VEC{w}) \right) \right]
= \mu_{\phi(\VEC{w})} \ .
\]
Moreover
\[
\VEC{n}(\VEC{u})
= \left\| \pdydx{\phi}{w_1}(\VEC{w})
\times \pdydx{\phi}{w_2}(\VEC{w}) \right\|^{-1}
\left( \pdydx{\phi}{w_1}(\VEC{w}) \times \pdydx{\phi}{w_2}(\VEC{w}) \right)
\]
for $\VEC{u} = \phi(\VEC{w})$ and $\VEC{w} \in W$.
Given $\displaystyle (\VEC{w},\VEC{y}_i) \in \TS_{\VEC{w}} W \cong \RR^2$
for $i =1,2$.  We have that
$(\VEC{w}. \VEC{y}_i) = (\VEC{w}. b_{i,1} \VEC{e}_1 + b_{i,2} \VEC{e}_2)$
for $i =1,2$ and some $b_{i,j} \in \RR$.  Hence
\begin{align}
&\phi^\ast(\nu)(\VEC{w})
\big((\VEC{w},\VEC{y}_1),(\VEC{w},\VEC{y}_2)\big)
= \nu(\phi(\VEC{w}))
\big(\phi_\ast(\VEC{w},\VEC{y}_1),\phi_\ast(\VEC{w},\VEC{y}_2)\big) \nonumber
\\
&\quad = \nu(\phi(\VEC{w}))
\big((\phi(\VEC{w}),\diff \phi(\VEC{w})\, \VEC{y}_1),
(\phi(\VEC{w}),\diff \phi(\VEC{w}) \, \VEC{y}_2)\big) \nonumber \\
&\quad = \nu(\phi(\VEC{w}))
\Big(\big(\phi(\VEC{w}), b_{1,1} \diff \phi(\VEC{w})\, \VEC{e}_1
+ b_{1,2} \diff \phi(\VEC{w})\, \VEC{e}_2\big), \nonumber \\
&\hspace{12em} \big(\phi(\VEC{w}), b_{2,1} \diff \phi(\VEC{w})\, \VEC{e}_1
+ b_{2,2} \diff \phi(\VEC{w})\, \VEC{e}_2\big)\Big) \nonumber \\
&\quad = \nu(\phi(\VEC{w}))
\bigg(\left(\phi(\VEC{w}), b_{1,1} \pdydx{\phi}{w_1}(\VEC{w})
+ b_{1,2} \pdydx{\phi}{w_2}(\VEC{w}) \right),
\left(\phi(\VEC{w}), b_{2,1} \pdydx{\phi}{w_1}(\VEC{w})
+ b_{2,2} \pdydx{\phi}{w_2}(\VEC{w})\right)\bigg) \nonumber \\
&\quad = \ps{\left(b_{1,1} \pdydx{\phi}{w_1}(\VEC{w})
+ b_{1,2} \pdydx{\phi}{w_2}(\VEC{w}) \right) \times
\left(b_{2,1} \pdydx{\phi}{w_1}(\VEC{w})
+ b_{2,2} \pdydx{\phi}{w_2}(\VEC{w})\right)}{\VEC{n}(\phi(\VEC{w}))} \nonumber
\\
&\quad = \ps{\left(b_{1,1} b_{2,2} - b_{1,2} b_{2,1} \right)
\pdydx{\phi}{w_1}(\VEC{w}) \times \pdydx{\phi}{w_2}(\VEC{w})}
{\VEC{n}(\phi(\VEC{w}))} \nonumber \\
&\quad = (b_{1,1} b_{2,2} - b_{1,2} b_{2,1})
\left\| \pdydx{\phi}{w_1}(\VEC{w}) \times \pdydx{\phi}{w_2}(\VEC{w}) \right\|
\nonumber \\
&\quad = \left( \left\| \pdydx{\phi}{w_1} \times \pdydx{\phi}{w_2} \right\|
\df{w_1} \wedge \df{w_2}\right) (\VEC{w}) \big( (\VEC{w} , \VEC{y}_1),
(\VEC{w}, \VEC{y}_2) \big) \ . \label{vunEq1}
\end{align}
Thus $\phi^\ast(\nu)$ is continuous on $W$.

Therefore, by uniqueness of the volume element, $\dx{A} = \nu$.
It also follows from (\ref{vunEq1}) that
\begin{equation} \label{defn2Dvolelem}
\phi^\ast\dx{A} =
\left\| \pdydx{\phi}{w_1} \times \pdydx{\phi}{w_2}
\right\| \dx{w_1} \wedge \dx{w_2} \ .
\end{equation}
\end{egg}

A proof similar to the proof of the previous theorem gives the following
result.

\begin{theorem} \label{manifVFXYZT}
Let $S$ be a $2$-dimensional oriented manifold in $\displaystyle \RR^3$ and let
$\VEC{n}_{\VEC{u}} = (\VEC{u}, \VEC{n}(\VEC{u}))$ be the outward unit
normal to $S$ at $\VEC{u}$ (as defined in
Definition~\ref{manifNormal}) associated to the chosen orientation
$\mu_{\VEC{u}}$ on $\TS_{\VEC{u}} S$ for all $\VEC{u} \in S$.  Then
\begin{equation} \label{manifVFXYZ}
\dx{A} = n_1 \df{x_2}\wedge \df{x_3} + n_2 \df{x_3}\wedge
\df{x_1} + n_3 \df{x_1}\wedge\df{x_2} \ .
\end{equation}
Moreover, on $S$ (and only on $S$), we have
\begin{equation} \label{manifVFXYZN}
n_1 \dx{A} = \df{x_2}\wedge\df{x_3} \ ,
\ n_2 \dx{A} = \df{x_3}\wedge\df{x_1}
\quad \text{and} \quad n_3 \dx{A} = \df{x_1}\wedge\df{x_2} \ .
\end{equation}
\end{theorem}

\begin{proof}
For all $(\VEC{u}, \VEC{v}_1)$ and $(\VEC{u}, \VEC{v}_2)$ in
$\TS_{\VEC{u}} S$, we have that
\begin{align*}
&\dx{A}(\VEC{u})\big((\VEC{u}, \VEC{v}_1),(\VEC{u}, \VEC{v}_2)\big)
= \det \begin{pmatrix} \VEC{v}_1 & \VEC{v}_2 & \VEC{n}(\VEC{u})
\end{pmatrix} \\
&\quad = n_1(\VEC{u})(v_{1,2} v_{2,3} - v_{1,3} v_{2,2})
- n_2(\VEC{u})(v_{1,1}v_{2,3}-v_{1,3}v_{2,1})
+n_3(\VEC{u}) (v_{1,1}v_{2,2}-v_{1,2}v_{2,1}) \\
&\quad = n_1(\VEC{u})(\df{x_2}\wedge\df{x_3})(\VEC{u})\big((\VEC{u}, \VEC{v}_1),
(\VEC{u}, \VEC{v}_2)\big)
- n_2(\VEC{u}) (\df{x_1}\wedge\df{x_3})(\VEC{u})\big((\VEC{u}, \VEC{v}_1),
(\VEC{u}, \VEC{v}_2)\big) \\
& \qquad \quad +n_3(\VEC{u})
(\df{x_1}\wedge\df{x_2})(\VEC{u})\big((\VEC{u}, \VEC{v}_1),
(\VEC{u}, \VEC{v}_2)\big) \ .
\end{align*}
Since $\VEC{u} \in S$ is arbitrary, we get (\ref{manifVFXYZ}).

To prove (\ref{manifVFXYZN}), we note that
\begin{align*}
&\ps{\VEC{q}}{\VEC{n}(\VEC{u})}
\dx{A}(\VEC{u})\big((\VEC{u},\VEC{v}_1),(\VEC{u},\VEC{v}_2)\big)
= \ps{\VEC{q}}{\VEC{n}(\VEC{u})} \ps{ \VEC{v}_1 \times \VEC{v}_2}
{\VEC{n}(\VEC{u})} \\
&\qquad  = \ps{\VEC{q}}{\VEC{n}(\VEC{u})}
\| \VEC{v}_1 \times \VEC{v}_2\|
= \ps{\VEC{q}}{ \| \VEC{v}_1 \times \VEC{v}_2\| \VEC{n}(\VEC{u})}
= \ps{\VEC{q}}{\VEC{v}_1 \times \VEC{v}_2}
\end{align*}
for all $(\VEC{x}, \VEC{v}_i)\in \TS_{\VEC{u}} S$ with $i =1,2$
and all $\displaystyle \VEC{q} \in \RR^3$.  With $\VEC{q} = \VEC{e}_i$, we get
\begin{align*}
n_i(\VEC{u})
\dx{A}(\VEC{u})\big((\VEC{u},\VEC{v}_1),(\VEC{x},\VEC{v}_2)\big)
&= \ps{\VEC{e}_i}{\VEC{v}_1 \times \VEC{v}_1}
= \begin{cases}
v_{1,2} v_{2,3} - v_{1,3} v_{2,2} & \quad \text{if} \ i = 1 \\
-v_{1,1} v_{2,3} + v_{1,3} v_{2,1} & \quad \text{if} \ i = 2 \\
v_{1,1} v_{2,2} - v_{1,2} v_{2,1} & \quad \text{if} \ i = 3
\end{cases}
\\
&= \begin{cases}
(\df{x_2} \wedge \df{x_3}) \big( (\VEC{u}, \VEC{v}_1), (\VEC{u},\VEC{v}_2)\big)
& \quad \text{if} \ i = 1 \\
-(\df{x_1}\wedge \df{x_3})\big( (\VEC{u}, \VEC{v}_1), (\VEC{u},\VEC{v}_2)\big)
& \quad \text{if} \ i = 2 \\
(\df{x_1} \wedge \df{x_2})\big( (\VEC{u}, \VEC{v}_1), (\VEC{u},\VEC{v}_2)\big)
& \quad \text{if} \ i = 3
\end{cases}
\end{align*}
for all $(\VEC{x}, \VEC{v}_i) \in \TS_{\VEC{u}} S$ with $i =1,2$.
This proves the relations in (\ref{manifVFXYZN}).
\end{proof}

\begin{egg}
Let $S$ be a compact and oriented $k$-dimensional     \label{eggHkDim}
manifold without boundary.  There exists a closed
and non-exact differential $k$-form on $S$.  One such differential
form is the volume element $\nu$ on $S$.  It is obviously closed because the
derivative of a dimensional $k$-form on a manifold of dimension $k$ is
always null.  Moreover, it follows from Stokes' theorem that $\nu$
cannot be exact.  Suppose that there were a differential $(k-1)$-form
$\eta$ on $S$ such that $\df{\eta} = \nu$.  Then we would have from
Stokes' theorem that $\displaystyle 0 = \int_{\partial S} \eta
= \int_S \df{\eta} = \int_S \nu$.   But $\int_S \nu$ is a
positive real number because $\nu$ is the volume element on a
compact manifold.
\end{egg}

\subsection{Polar Coordinates}

We begin by expressing in a convenient way the volume element on
$\displaystyle S^{n-1} \equiv \{ \VEC{x} \in \RR^n : \|\VEC{x}\|=1\}$.

Suppose that $\mu_{\VEC{u}}$ is the orientation on
$\displaystyle \TS_{\VEC{u}} S^{n-1}$ for $\displaystyle \VEC{u} \in S^{n-1}$
and that
$\displaystyle \big(\VEC{u},\VEC{n}(\VEC{u})\big) \in \TS_{\VEC{u}} \RR^n$
is the outward unit normal to $S$ at $\VEC{u} \in S$ as in
Definition~\ref{manifNormal}.
We note that $\VEC{n}(\VEC{u}) = \VEC{u}$ for all
$\displaystyle \VEC{u} \in S^{n-1}$.
If $\displaystyle \left\{ (\VEC{u},\VEC{v}_{\VEC{u},i})
\right\}_{1 \leq i \leq n-1} \subset \TS_{\VEC{u}} S^{n-1}$ is such that
$\mu_{\VEC{u}} = [(\VEC{u},\VEC{v}_{\VEC{u},1}), \ldots,
(\VEC{u},\VEC{v}_{\VEC{u},n-1})]$, then
$\displaystyle [(\VEC{u},\VEC{n}(\VEC{u})),(\VEC{u},\VEC{v}_{\VEC{u},1}), \ldots,
(\VEC{u},\VEC{v}_{\VEC{u},n-1})] =
[(\VEC{u},\VEC{e}_1),(\VEC{u},\VEC{e}_2), \ldots,
(\VEC{u},\VEC{e}_n)]$.

Let
\[
\nu(\VEC{u})\big((\VEC{u},\VEC{v}_1), \ldots , (\VEC{u}, \VEC{v}_{n-1})\big)
= \det
\begin{pmatrix}
\VEC{u} & \VEC{v}_1 & \ldots & \VEC{v}_{n-1}
\end{pmatrix}
\]
for all $\displaystyle (\VEC{u},\VEC{v}_1) \in \TS_{\VEC{u}} S^{n-1}$
and $\displaystyle \VEC{u} \in S^{n-1}$.
It is easy to prove that $\nu$ is a differential $(n-1)$-form on
$\displaystyle S^{n-1}$ of class $\displaystyle C^\infty$ since the
standard local atlas for $\displaystyle S^1$ is of class
$\displaystyle C^\infty$.  Moreover, if 
$\displaystyle \left\{ (\VEC{u},\VEC{v}_{\VEC{u},i})
\right\}_{1 \leq i \leq n-1} \subset \TS_{\VEC{u}} S^{n-1}$
is an orthonormal basis on $\displaystyle \TS_{\VEC{u}} S^{n-1}$
with respect to the inner product
$\ps{(\VEC{u},\VEC{v}_i)}{(\VEC{u},\VEC{v}_j)}_{\VEC{u}}
= \ps{\VEC{v}_i}{\VEC{v}_j}$,
where the inner product on the right is the standard inner product on
$\displaystyle \RR^n$, then
$\nu(\VEC{u})\big((\VEC{u},\VEC{v}_1), \ldots , (\VEC{u},
\VEC{v}_{n-1})\big) = 1$.  Therefore, $\nu$ is the
volume element on $S$.

\begin{egg}
We have shown in Examples~\ref{CnotEpart1}     \label{HR0dim}
and \ref{CnotEpart2} that
\[
\omega = \frac{-x_2}{x_1^2+x_2^2} \df{x_1} + \frac{x_1}{x_1^2+x_2^2} \df{x_2}
\]
is a closed differential $1$-form on
$\displaystyle V= \RR^2 \setminus \{\VEC{0}\}$ that is not exact.

We now show that we can always find a closed differential $n-1$-form on
$\displaystyle \RR^n \setminus \{ \VEC{0}\}$ for $n\geq 2$ that is not
exact.  For this purpose, we will use the volume element $\nu$ on
$\displaystyle S^{n-1}$.   We have that $\nu$ is closed because
it is a differential $(n-1)$-form on a $(n-1)$-dimensional manifold.

Consider the retraction
$\displaystyle \rho : \RR^n \setminus \{\VEC{0}\} \to S^{n-1}$ defined by
$\displaystyle \rho(\VEC{x}) = \|\VEC{x}\|^{-1} \VEC{x}$, and
the inclusion $\displaystyle \iota : S^{n-1} \to \RR^n \setminus \{\VEC{0}\}$
defined by $\iota(\VEC{u}) = \VEC{u}$.
We have that $\displaystyle \rho \circ \iota = \Id: S^{n-1} \to S^{n-1}$.

The differential $(n-1)$-form $\displaystyle \rho^\ast(\nu)$ on
$\displaystyle \RR^n \setminus \{\VEC{0}\}$ is closed because
$\displaystyle \df{(\rho^\ast(\nu))} = \rho^\ast(\df{\nu}) = 0$
since $\df{\nu} = 0$.
However, $\displaystyle \rho^\ast(\nu)$ is not exact.  This is proved by
contradiction.  Suppose that $\displaystyle \rho^\ast(\nu)$ is exact and
$\displaystyle \rho^\ast(\nu) = \df{\eta}$ for some differential
$(n-2)$-form on $\displaystyle \RR^n \setminus \{\VEC{0}\}$.
Then $\displaystyle \nu = (\rho \circ \iota)^\ast \nu
= \iota^\ast( \rho^\ast(\nu)) = \iota^\ast(\df{\eta})
= \df{(\iota^\ast(\eta))}$.  However, this is a contradiction because
we know from Example~\ref{eggHkDim} that $\nu$ cannot be exact on
$\displaystyle S^{n-1}$ since $\displaystyle \partial S^{n-1} = \emptyset$.
\end{egg}

Let $\nu$ be the volume element on $\displaystyle S^{n-1}$, and let
$\displaystyle \iota:S^{n-1} \to \RR^n \setminus \{\VEC{0}\}$
and $\displaystyle \rho: \RR^n\setminus \{\VEC{0}\} \to S^{n-1}$ be
respectively the inclusion and retraction defined in Example~\ref{HR0dim}.
Consider the following differential $(n-1)$-form on $\displaystyle \RR^n$.
\begin{equation} \label{oemagRn}
\tilde{\nu} = \sum_{j=1}^n (-1)^{j-1} x_j \df{x_1} \wedge \ldots \wedge
\widehat{\df{x_j}} \wedge \ldots \wedge \df{x}_n
\end{equation}
(see Remark~\ref{rmkWarning1}).  The reader should verify that
\[
\tilde{\nu}(\VEC{x})\big((\VEC{x},\VEC{y}_1),
\ldots,(\VEC{x},\VEC{y}_{n-1})\big)
= \det \begin{pmatrix} \VEC{x} & \VEC{y}_1 & \ldots & \VEC{y}_{n-1}
\end{pmatrix}
\]
for all $\displaystyle \VEC{x} \in \RR^n$ and
$\displaystyle (\VEC{x},\VEC{y}_j) \in \TS_{\VEC{x}} \RR^n$
with $1 \leq j \leq n-1$.  Thus $\displaystyle \nu = \iota^\ast(\tilde{\nu})$.
However, $\displaystyle \rho^\ast(\nu)$ is not equal to $\tilde{\nu}$.  Instead,
we have the following result.

\begin{prop} \label{propRanEntn}
Let $\nu$ be the volume element on $\displaystyle S^{n-1}$,
$\tilde{\nu}$ be the differential $(n-1)$-form given in
(\ref{oemagRn}), and
$\displaystyle \rho: \RR^n\setminus \{\VEC{0}\} \to S^{n-1}$ be
the retraction defined by
$\displaystyle \rho(\VEC{x}) = \|\VEC{x}\|^{-1}\VEC{x}$ for
$\displaystyle \VEC{x} \in \RR^n\setminus \{\VEC{0}\}$.
Then $\displaystyle
\rho^\ast(\nu)(\VEC{u}) = \|\VEC{u}\|^{-n} \tilde{\nu}(\VEC{u})$
in $\displaystyle \Omega^{n-1}\left(\TS_{\VEC{u}} \RR^n\right)$
for all $\displaystyle \VEC{u} \in \RR^n\setminus \{\VEC{0}\}$.
\end{prop}

\begin{proof}
An orthonormal basis of $\displaystyle \TS_{\VEC{u}} \RR^n$ is given
by the outward unit normal
$\displaystyle \VEC{n}_{\VEC{u}}=(\VEC{u},\VEC{n}(\VEC{u}))
= (\VEC{u}, \|\VEC{u}\|^{-1}\VEC{u})$ to
$\partial B_{\|\VEC{u}\|}(\VEC{0})$ at $\VEC{u} \in
\partial B_{\|\VEC{u}\|}(\VEC{0})$ and an orthonormal basis
$\displaystyle \{ (\VEC{u},\VEC{v}_i)\}_{1\leq i \leq n-1}$
of $\TS_{\VEC{u}} (\partial B_{\|\VEC{u}\|}(\VEC{0}))$.

Because of the multi-linearity of $\displaystyle \rho^\ast(\nu)$, it
suffices to verify that
\begin{equation} \label{RiOEq1}
\rho^\ast(\nu)(\VEC{u})\big((\VEC{u},\VEC{y}_1),
\ldots,(\VEC{u},\VEC{y}_{n-1})\big)
= \|\VEC{u}\|^{-n} \tilde{\nu}(\VEC{u})\big((\VEC{u},\VEC{y}_1),
\ldots,(\VEC{u},\VEC{y}_{n-1})\big)
\end{equation}
where each $(\VEC{u},\VEC{y}_j)$ is
$\displaystyle (\VEC{u}, \|\VEC{u}\|^{-1}\VEC{u})$ or one of
$\displaystyle (\VEC{u},\VEC{v}_i)$ for $1 \leq i \leq n-1$.

We first note that
$\displaystyle \rho_\ast(\VEC{u}, \|\VEC{u}\|^{-1}\VEC{u})
= (\|\VEC{u}\|^{-1}\VEC{u}, \VEC{0})$
because
$\displaystyle \diff \rho(\VEC{u}) \left(\|\VEC{u}\|^{-1}\VEC{u}\right)$ is the
directional derivative of $\rho$ at $\VEC{u}$ in the radial direction
$\VEC{n}(\VEC{u})$.
But $\displaystyle \rho(\VEC{u}) = \|\VEC{u}\|^{-1}\VEC{u}$ is constant
along the radial direction $\VEC{n}(\VEC{u})$.  Thus, the directional
derivative is null.  It follows that the left side of (\ref{RiOEq1})
is null if one of the $(\VEC{u},\VEC{y}_j)$ is
$\displaystyle (\VEC{u}, \|\VEC{u}\|^{-1}\VEC{u})$.  As for the right
hand side of (\ref{RiOEq1}), if 
$\displaystyle (\VEC{u},\VEC{y}_j) = (\VEC{u}, \|\VEC{u}\|^{-1}\VEC{u})$
for $j=j_0 \in \{1,2,\ldots, n-1\}$, then
\begin{align*}
&\tilde{\nu}(\VEC{u})\big((\VEC{u},\VEC{y}_1),
\ldots,(\VEC{u},\VEC{y}_{n-1})\big)
= \det \begin{pmatrix} \VEC{u} & \VEC{y}_1 & \ldots & \VEC{y}_{n-1}
\end{pmatrix} \\
&\qquad = \det \begin{pmatrix} \VEC{u} & \VEC{y}_1 & \ldots &
\VEC{y}_{j_0-1} & \|\VEC{u}\|^{-1}\VEC{u} & \VEC{y}_{j_0+1} & \ldots &
 \VEC{y}_{n-1}
\end{pmatrix} \\
&\qquad
= \|\VEC{u}\|^{-1} \, \det \begin{pmatrix} \VEC{u} & \VEC{y}_1 & \ldots &
\VEC{y}_{j_0-1} & \VEC{u} & \VEC{y}_{j_0+1} & \ldots &  \VEC{y}_{n-1}
\end{pmatrix} = 0 \ .
\end{align*}

We now consider (\ref{RiOEq1})
when each $(\VEC{u},\VEC{y}_j)$ is one of $\displaystyle (\VEC{u},\VEC{v}_i)$
for $1 \leq i \leq n-1$.  If two $(\VEC{u},\VEC{y}_j)$ are equal
to the same $(\VEC{u},\VEC{v}_i)$, then both sides of (\ref{RiOEq1})
are null.  Therefore, we may assume that the $(\VEC{u},\VEC{y}_j)$ are
distinct.  We have that
\[
(\diff \rho(\VEC{u})) \VEC{v}_j
= \begin{pmatrix}
\displaystyle
\diff \left(\frac{x_1}{\|\VEC{x}\|}\right)\bigg|_{\VEC{x} = \VEC{u}}
\cdot \VEC{v}_j \\[1em]
\displaystyle
\diff \left(\frac{x_2}{\|\VEC{x}\|}\right)\bigg|_{\VEC{x} = \VEC{u}}
\cdot \VEC{v}_j \\
\vdots \\
\displaystyle
\diff \left(\frac{x_n}{\|\VEC{x}\|}\right)\bigg|_{\VEC{x} = \VEC{u}}
\cdot \VEC{v}_j
\end{pmatrix}
= \begin{pmatrix}
\displaystyle
\pdfdx{\left(\frac{x_1}{\|\VEC{x}\|}\right)}{\VEC{v}_j}(\VEC{u}) \\[1em]
\displaystyle
\pdfdx{\left(\frac{x_2}{\|\VEC{x}\|}\right)}{\VEC{v}_j}(\VEC{u}) \\
\vdots \\
\displaystyle
\pdfdx{\left(\frac{x_n}{\|\VEC{x}\|}\right)}{\VEC{v}_j}(\VEC{u})
\end{pmatrix} \ .
\]
Moreover
\begin{align*}
&\pdfdx{\left(\frac{x_i}{\|\VEC{x}\|}\right)}{\VEC{v}_j}(\VEC{u}) -
\frac{v_{j,i}}{\|\VEC{u}\|}
= \lim_{h\to 0} \frac{1}{h}
\left( \frac{u_i + h v_{j,i}}{\|\VEC{u} + h \VEC{v}_j\|}
- \frac{u_i}{\|\VEC{u}\|} - \frac{hv_{j,i}}{\|\VEC{u}\|} \right) \\
&\quad = \lim_{h\to 0} \frac{(u_i + h v_{j,i})(\|\VEC{u}\|
- \|\VEC{u}+h\VEC{v}_j\|)}{h\|\VEC{u}\|\,\|\VEC{u} + h \VEC{v}_j\|}
= \lim_{h\to 0} \frac{(u_i + h v_{j,i})(\|\VEC{u}\|^2
- \|\VEC{u}+h\VEC{v}_j\|^2)}{h\|\VEC{u}\|\,\|\VEC{u} + h \VEC{v}_j\|
(\|\VEC{u}\| + \|\VEC{u} + h \VEC{v}\|)} \\
&\quad = \lim_{h\to 0} \frac{(u_i + h v_{j,i})( -2 h\, \VEC{u}\cdot \VEC{v}_j
- h^2\|\VEC{v}_j\|^2)}{h\|\VEC{u}\|\,\|\VEC{u} + h \VEC{v}_j\|
(\|\VEC{u}\| + \|\VEC{u} + h \VEC{v}\|)}
= \lim_{h\to 0} \frac{-h(u_i + h v_{j,i})\|\VEC{v}_j\|^2}
{\|\VEC{u}\|\,\|\VEC{u} + h \VEC{v}_j\|
(\|\VEC{u}\| + \|\VEC{u} + h \VEC{v}\|)} = 0
\end{align*}
because $\VEC{u}$ is orthogonal to $\VEC{v}_j$ for $1\leq j \leq n-1$.
Therefore $\displaystyle (\diff \rho(\VEC{u})) \VEC{v}_j
= \|\VEC{u}\|^{-1} \VEC{v}_j$ for $1\leq j \leq n-1$.
Hence, for any permutation $\sigma \in S_{n-1}$, we have that
\begin{align*}
& \rho^\ast(\nu)(\VEC{u})\big((\VEC{u},\VEC{v}_{\sigma(1)}),
\ldots,(\VEC{u},\VEC{v}_{\sigma(n-1)})\big)
= \nu(\rho(\VEC{u}))\big(\rho_\ast(\VEC{u},\VEC{v}_{\sigma(1)}),
\ldots,\rho_\ast(\VEC{u},\VEC{v}_{\sigma(n-1)})\big) \\
&\qquad =  \nu\left(\|\VEC{u}\|^{-1}\VEC{u}\right)
\big((\|\VEC{u}\|^{-1}\VEC{u},\|\VEC{u}\|^{-1}\VEC{v}_{\sigma(1)}),
\ldots,(\|\VEC{u}\|^{-1}\VEC{u},\|\VEC{u}\|^{-1}\VEC{v}_{\sigma(n-1)})\big) \\
&\qquad = \det \begin{pmatrix}
\|\VEC{u}\|^{-1}\VEC{u} & \|\VEC{u}\|^{-1}\VEC{v}_{\sigma(1)}) &
\ldots & \|\VEC{u}\|^{-1}\VEC{v}_{\sigma(n-1)} \end{pmatrix} \\
&\qquad = \|\VEC{u}\|^{-n}\det \begin{pmatrix}
\VEC{u} & \VEC{v}_{\sigma(1)}) & \ldots & \VEC{v}_{\sigma(n-1)} \end{pmatrix}
= \|\VEC{u}\|^{-n}
\tilde{\nu}(\VEC{u})\big((\VEC{u},\VEC{v}_{\sigma(1)}),
\ldots,(\VEC{u},\VEC{v}_{\sigma(n-1)})\big)  \ .
\end{align*}
This complete the proof of (\ref{RiOEq1}).
\end{proof}

The use of polar and spherical coordinates can be generalized to
$\displaystyle \RR^n$ with $n > 3$.

\begin{prop}  \label{propPolCoords}
Suppose that $f:B_1(\VEC{0}) \to \RR$ is a continuous function.  Then
\[
\int_{B_1(\VEC{0})} f \, \df{x_1} \wedge \df{x_2} \wedge \ldots
\wedge \df{x_n} = \int_{S^{n-1}} F\, \nu \ .
\]
where $\displaystyle F(\VEC{u}) = \int_0^1 r^{n-1}f(r \VEC{u}) \dx{r}$
for $\displaystyle \VEC{u} \in S^{m-1}$.
\end{prop}

\begin{proof}
To prove the proposition, we use the projections
$\displaystyle \pi_S: S^{n-1} \times \RR \to S^{n-1}$ and
$\displaystyle \pi_I:S^{n-1} \times \RR \to \RR$.  We consider the
differential $n$-form on $\displaystyle S^{n-1} \times \RR$ defined by
$\displaystyle \pi_S^\ast(\nu) \wedge \pi_I^\ast(\df{r})$, where $\df{r}$ is the
differential $1$-form on $\RR$ defined by $\df{r}(r)(r,y) = y$ for all
$r \in \RR$ and $(r,y) \in \TS_r \RR \cong \RR$.

\stage{i}
Let $\displaystyle g:S^{n-1}\times [0,1] \to \RR$ be the function
defined by $\displaystyle g(\VEC{u},r) = r^{n-1} f(r\VEC{u})$ for
$\displaystyle \VEC{u} \in S^{n-1}$ and $r \in [0,1]$.

Since $\displaystyle S^{n-1}$ is a compact manifold, we may choose a
finite atlas $\A = \{ (W_i,U_i,\phi_i) \}_{1\leq i \leq I}$ of
$\displaystyle S^{n-1}$ and a finite partition of unity
$\{\psi_i\}_{1\leq i \leq I}$ subordinate to $\A$ such that
$\supp \psi_i \subset U_i$ for all $i$.  We have that
$\{ (W_i\times ]0,1[,U_i\times]0,1[,\phi_i\times\Id) \}_{1\leq i \leq I}$
is an atlas of $\displaystyle S^{n-1}\times ]0,1[$.
We assume that
$\phi_i^\ast(\nu) = q_i \df{w_1}\wedge \df{w_2} \wedge \ldots
\wedge \df{w_{n-1}}$ for all $i$.
Hence
\begin{align}
\int_{S^{n-1}} F \, \nu
&= \int_{S^{n-1}} \left( \int_{]0,1[} g \df{r}\right) \nu
= \sum_{i=1}^I \int_{S^{n-1}} \psi_i \left( \int_{]0,1[} g \df{r}\right) \nu
\nonumber \\
&= \sum_{i=1}^I \int_{W_i} \psi_i(\phi_i(\VEC{w}))
\left( \int_{]0,1[} g(\phi_i(\VEC{w}),r)\df{r}\right) \phi^\ast(\nu)(\VEC{w})
\nonumber \\
& = \sum_{i=1}^I \int_{W_i} \psi_i(\phi_i(\VEC{w}))
\left( \int_{]0,1[} g(\phi_i(\VEC{w}),r)\df{r}\right)
q_i(\VEC{w}) \df{w_1}\wedge \df{w_2} \wedge \ldots \wedge \df{w_{n-1}}
\nonumber \\
& = \sum_{i=1}^I \int_{W_i} \psi_i(\phi_i(\VEC{w}))
\left( \int_{]0,1[} g(\phi_i(\VEC{w}),r)\df{r}\right)
q_i(\VEC{w}) \dx{w_1} \dx{w_2} \ldots \dx{w_{n-1}} \nonumber \\
& = \sum_{i=1}^I \int_{W_i} \int_{]0,1[} \psi_i(\phi_i(\VEC{w}))
g(\phi_i(\VEC{w}),r)
q_i(\VEC{w}) \dx{r} \dx{w_1} \dx{w_2} \ldots \dx{w_{n-1}} \nonumber \\
&= \sum_{i=1}^I \int_{W_i \times ]0,1[} \psi_i(\phi_i(\VEC{w}))
g(\phi_i(\VEC{w}),r) 
q_i(\VEC{w}) \dx{r} \wedge \df{w_1} \wedge \df{w_2} \wedge \ldots \wedge
\dx{w_{n-1}} \nonumber \\
&= (-1)^{n-1}\sum_{i=1}^I \int_{W_i \times ]0,1[} \psi_i(\phi_i(\VEC{w}))
g(\phi_i(\VEC{w}),r) 
q_i(\VEC{w}) \df{w_1} \wedge \df{w_2} \wedge \ldots \wedge
\dx{w_{n-1}} \wedge \dx{r}  \nonumber \\
&= (-1)^{n-1}\sum_{i=1}^I \int_{W_i \times ]0,1[}
(\phi\times \Id)^\ast\left( \psi_i(\VEC{u}) g(\VEC{u},r) \,
\pi_S^\ast(\nu) \wedge \pi_I^\ast(\df{r}) \right) \nonumber \\
&= (-1)^{n-1}\sum_{i=1}^I \int_{S^{n-1} \times ]0,1[}
\psi_i g\, \pi_S^\ast(\nu) \wedge \pi_I^\ast(\df{r}) \nonumber \\
&= (-1)^{n-1} \int_{S^{n-1} \times ]0,1[} g\, \pi_S^\ast(\nu) \wedge
\pi_I^\ast(\df{r}) \ .
\label{propPolCoordsEq1}
\end{align}

\stage{ii}
Let $\displaystyle h:B_1(\VEC{0})\setminus \{\VEC{0}\}
\to S^{n-1} \times ]0,1]$ be the map defined by
$\displaystyle h(\VEC{x}) = \big(\rho(\VEC{x}), N(\VEC{x})\big)$
for all $\displaystyle \VEC{x} \in B_1(\VEC{0})\setminus \{\VEC{0}\}$,
where $\displaystyle \rho: \RR^n \setminus \{\VEC{0}\} \to S^{n-1}$ is
the retraction previously defined and
$\displaystyle N: \RR^n \setminus \{\VEC{0}\} \to ]0,\infty[$ is defined
by $N(\VEC{x}) = \|\VEC{x}\|$ for
$\displaystyle \VEC{x} \in \RR^n \setminus \{\VEC{0}\}$.  Using the
previous proposition, we have that
\begin{align*}
&h^\ast\left(\pi_S^\ast(\nu) \wedge \pi_I^\ast(\df{r})\right)(\VEC{x})
= (\pi_S\circ h)^\ast(\nu)(\VEC{x}) \wedge (\pi_I\circ h)^\ast(\df{r})(\VEC{x})
= \rho^\ast(\nu)(\VEC{x}) \wedge N^\ast(\df{r})(\VEC{x}) \\
&\qquad = \left( \|\VEC{x}\|^{-n}\sum_{j=1}^n (-1)^{j-1} x_j
\, \df{x_1} \wedge
\ldots \wedge \widehat{\df{x_j}} \wedge \ldots \wedge \df{x}_n \right)
\wedge \left( \sum_{i=1}^n \pdfdx{\|\VEC{x}\|}{x_i} \df{x_i} \right) \\
&\qquad = \left( \|\VEC{x}\|^{-n}\sum_{j=1}^n (-1)^{j-1} x_j
\, \df{x_1} \wedge \ldots \wedge \widehat{\df{x_j}} \wedge \ldots \wedge
\df{x}_n \right)
\wedge \left( \sum_{i=1}^n \frac{x_i}{\|\VEC{x}\|}\df{x_i} \right) \\
&\qquad = (-1)^{n-1} \|\VEC{x}\|^{-n+1} \, \df{x_1} \wedge
\ldots \wedge \df{x_j} \wedge \ldots \wedge \df{x}_n
\end{align*}
for all $\displaystyle \VEC{x} \in \RR^n \setminus \{\VEC{0}\}$.  It
follows that
\begin{align*}
&h^\ast\left(g \,\pi_S^\ast(\nu) \wedge \pi_I^\ast(\df{r})\right)(\VEC{x})
= g(h(\VEC{x})) \,
h^\ast\left(\pi_S^\ast(\nu) \wedge \pi_I^\ast(\df{r})\right)(\VEC{x}) \\
&\qquad
= \|\VEC{x}\|^{n-1} f(\VEC{x})\, (-1)^{n-1} \|\VEC{x}\|^{-n+1}
\, \df{x_1} \wedge \ldots \wedge \df{x_j} \wedge \ldots \wedge \df{x}_n \\
&\qquad
= (-1)^{n-1} f(\VEC{x}) \, \df{x_1} \wedge \ldots \wedge \df{x_j} \wedge
\ldots \wedge \df{x}_n
\end{align*}
for all $\displaystyle \VEC{x} \in \RR^n \setminus \{\VEC{0}\}$.
Using Proposition~\ref{propCVforDF} and the fact that $\{\VEC{0}\}$ is
a set of measure zero where $f$ is continuous, we finally get
from (\ref{propPolCoordsEq1}) that
\begin{align*}
\int_{B_1(\VEC{0})} f \, \df{x_1} \wedge \df{x_2} \wedge \ldots
\wedge \df{x_n}
&=\int_{B_1(\VEC{0})\setminus \{0\}} f \, \df{x_1} \wedge \df{x_2} \wedge \ldots
\wedge \df{x_n} \\
&= (-1)^{n-1} \int_{B_1(\VEC{0})\setminus \{0\}}
h^\ast\left(g \,\pi_S^\ast(\nu) \wedge \pi_I^\ast(\df{r})\right) \\
&= (-1)^{n-1} \int_{S^{n-1} \times ]0,1[}
g \,\pi_S^\ast(\nu) \wedge \pi_I^\ast(\df{r})
= \int_{S^{n-1}} F \, \nu \ .  \qedhere
\end{align*}
\end{proof}

%%% Local Variables:
%%% mode: latex
%%% TeX-master: "notes"
%%% End: 
