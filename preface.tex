\nonumchapter{Preface}

The first part of these lecture notes provides a clear, detailed and
complete presentation of Riemann integrals for functions of several
variables including an introduction to manifolds and integration on
manifolds.

The first part of these lecture notes is a complement to and an expansion
of Spivak's {\bfseries Calculus on Manifolds} \cite{S}.   We review
some of the topics in the book, adding some additional information and
results.  We also reproduce (in some cases with major differences)
some of the proofs that were sketchy.  After having learned
calculus of functions of several variables in their first year at the
university, students study basic analysis.  This usually includes some
concepts of metric spaces and topology, the derivative of functions of
several variables including the inverse and implicit function
theorems, and sometime a rigorous introduction to Riemann integration
of functions of one variable.  The rigorous presentation of Riemann
integration for functions of several variables including integration
on manifolds is completely ignored.  The next time that students may
study integration of functions is at the graduate level when they
learn about Lebesgue integrals and so never cover rigorously vector
calculus and integration on manifolds.  The first part of these
lecture notes fills this cap that we often find in undergraduate
mathematics programs.

A large number of problems, many with detailed solutions, are included
in Chapters~\ref{ChaptSingleInt}, \ref{chapMultInt} and \ref{chaptCVC}.
These problems come from several textbooks listed in the reference, in
particular from Folland's {\bfseries Advanced Calculus} \cite{F}.  For
many of them, they are standard problems that one can find in all
advanced calculus textbooks.  These lecture notes are therefore also
ideal for self study and as a reference for students.

The second part of these lecture notes provides a modern introduction
to differential geometry.  There are several unusual features to the
presentation of this topic in these lecture notes.  One of them is
that the more geometric presentation of the tangent space to a
manifold at a point is used almost everywhere in these notes.  The
only time that the definition of the tangent space of manifold at a
point as a space of linear differential operators is used is to define
the Lie Bracket where it is really the easiest way to define the Lie
Bracket.  Another unusual feature is the inclusion of a chapter on
algebraic geometry.  We feel that this is necessary because algebraic
geometry provides a very nice and efficient technique to reach
conclusions about manifolds that would otherwise require very
technical and convoluted proofs.  The second part begins with an
introduction to de Rham cohomology.  We then provide a solid
introduction to algebraic topology including the simplicial and
singular approaches to homology and cohomology.  We prove the relation
between de Rham cohomology and the general concept of cohomology.

The last chapter of these lecture notes and its second part is an
introduction to Riemannian geometry.  The presentation of this topic
is very geometric and hopefully provides the reader with a good visual
understanding of the properties associated to Riemannian geometry.  In
particular, this chapter includes a proof of the beautiful
Gauss-Bonnet theorem.  We have tried to stay away from formal
computations frequently associated with this subject.

\section*{Some Terminology}

\subsection*{Metric and Topological Spaces}

The {\bfseries $\mathbf{k}$-norm}\index{Norm!$k$-Norm} of
$\DS \VEC{x} \in \RR^n$ is defined by
\[
\|\VEC{x}\|_k =
\begin{cases}
\DS \left(\sum_{i=1}^n |x_i|^k\right)^{1/k} & \quad \text{if}\ k > 0 \\
\DS \max_{1\leq i \leq n} |x_i| & \quad \text{if} \ k = \infty
\end{cases}
\]
We adopt the convention that $\|\cdot\|$ and $\|\cdot\|_2$ will denote
the {\bfseries $\mathbf{2}$-norm}\index{Norm!$2$-Norm} or
{\bfseries Euclidean norm}\index{Norm!Euclidean Norm}.

The {\bfseries diameter}\index{Diameter} of a set
$\DS W \subset \RR^n$ is the number 
$\diam Q = \inf \{ \|\VEC{x} - \VEC{y}\| : \VEC{x}, \VEC{y} \in W \}$.

The {\bfseries distance}\index{Distance} between two sets
$\DS V, W \subset \RR^n$ is the number
$\dist{V}{W} = \inf \{ \|\VEC{x} - \VEC{y}\| : \VEC{x} \in V \
\text{and} \ \VEC{y} \in W \}$.  The distance between a point
$\DS \VEC{x} \in \RR^n$ and a set
$\DS W \subset \RR^n$ is defined as
$\dist{\VEC{x}}{W} = \dist{\{\VEC{x}\}}{W}$.

The {\bfseries open ball}\index{Open Ball} of radius $r$ centred at
$\DS \VEC{y} \in \RR^n$ is denoted $\DS B_r(\VEC{y})$;
namely,
\[
B_r(\VEC{y}) = \left\{ \VEC{x} \in \RR^n :
  \left\|\VEC{x}-\VEC{y} \right\| < r \right\} \ .
\]
We will also use this notation to denote open balls in other metric spaces.

Let $W$ be a subset in $\DS \RR^n$.  The
{\bfseries closure}\index{Closure of a Set} of $W$, denoted
$\DS \overline{W}$, is the smallest closed set containing
$W$.  We have $\DS \overline{W} =
\bigcap_{\substack{W\subset V\\V\text{ closed}}} V$.
One of the must important case in these lecture notes is the closure of a
rectangle $\DS R = \prod_{i=1}^n I_i$ where the $I_i$ are
bounded intervals.  We have
$\DS \overline{R} = \prod_{i=1}^n \overline{I}_i$ where
$\overline{I}_i$ is the closed interval obtained from the interval $I_i$ by
adding the end points.

Let $W$ be a set in $\DS \RR^n$.  The
{\bfseries interior}\index{Interior of a Set} of $W$, denoted
$\DS W^\circ$, is the largest open subset of $W$.  We have
$\DS W^\circ = \bigcup_{\substack{V\subset W\\V\text{ open}}} V$.
One of the must important case in these lecture notes is the interior of a
rectangle $\DS R = \prod_{i=1}^n I_i$ where the $I_i$ are
bounded intervals.  We have
$\DS R^\circ = \prod_{i=1}^n I_i^\circ$ where
$\DS I_i^\circ$ is the open interval obtained from the
interval $I_i$ by removing the end points.

The definitions of the interior and closure of a set
$\DS W \subset V \subset \RR^n$ relative to the set $V$ are
identical to the previous definitions except that $\DS \RR^n$
is replaced by $V$ where $V$ has the induced topology from
$\DS \RR^n$.

Let $W$ be a set in $\DS \RR^n$.  The
{\bfseries boundary}\index{Boundary of a Set} of $W$, denoted
$\DS \partial W$, is the set of all points
$\DS \VEC{x} \in \RR^n$ such that
$B_r(\VEC{x}) \cap W \neq \emptyset$
and $\DS B_r(\VEC{x}) \cap (\RR^n \setminus W) \neq \emptyset$
for all $r>0$.
One of the must important case in these lecture notes is the boundary of a
rectangle $\DS R = \prod_{i=1}^n I_i$ where $I_i= [a_i,b_i]$,
$[a_,b_i[$, $]a_i,b_i]$ or $]a_i,b_i[$ for all $i$.  We have
\begin{align*}
\partial R &= \bigcup_{i=1}^n \Big( \big(
I_1 \times \ldots \times I_{i-1} \times \{ a_i\} \times I_{i+1} \times
\ldots \times I_n \big) \\
&\hspace{7em} \bigcup \big(
I_1 \times \ldots \times I_{i-1} \times \{ b_i\} \times I_{i+1} \times
\ldots \times I_n \big) \Big)
\end{align*}
where $I_1 \times \ldots \times I_{i-1}$ is ignored when $i=1$ and
$I_{i+1} \times \ldots, \times I_n$ is ignored when $i=n$.  The reader
may want to write down the previous formula for the case $n=2$ and
$n=3$ to verify its accuracy.

\subsection*{Function Spaces}

A function $\DS f:[a,b] \to \RR^n$ is
{\bfseries piecewise continuous}\index{Piecewise Continuous Function}
if $f$ is continuous at all but a finite number of points in $[a,b]$
and, at a point $c \in [a,b]$ where $f$ is not continuous, we have
$\DS \lim_{x\to c^-}f(x)$ and 
$\DS \lim_{x\to c^+}f(x)$ exist.

Similarly, given $m \in \NN$, a function $\DS f:[a,b] \to \RR^n$ is
{\bfseries piecewise $\DS \mathbf{C^m}$ continuously
differentiable}\index{Piecewise $C^m$ Continuous
Differentiable Function} if $f$ is $m$-time differentiable at all but
a finite number of points in $[a,b]$ and, at a point $c \in [a,b]$
where $f$ is not $m$-time differentiable, we have
$\DS \lim_{x\to c^-}f^{(j)}(x)$ and
$\DS \lim_{x\to c^+}f^{(j)}(x)$ exist for $0 \leq j \leq m$.
Obviously, $f$ is piecewise $\DS C^0$ continuously
differentiable is another way to say that $f$ is
piecewise continuously differentiable.

Let $V$ be a subset of $\DS \RR^n$ and $f:V \to \RR$.  The
{\bfseries supremum norm}\index{Norm!Supremum Norm of a Function}
of $f$ on $V$ is defined by
$\DS \|f\|_{\infty,V} = \sup_{\VEC{x}\in V} |f(\VEC{x})|$.
We will simply write $\|f\|_\infty$ when the set $V$ is clear from the
context.

Let $V$ be a subset of $\DS \RR^n$.
The {\bfseries support}\index{Support of a Function} of a function
$f:V \to \RR$ is the set
$\DS \supp f = \overline{ \{\VEC{x} \in V : f(\VEC{x}) \neq 0 \}}$.

Let $V$ be an open subset of $\DS \RR^n$ and $f:V \to \RR$
be a sufficiently differentiable function, we define the
{\bfseries differential operator
$\DS \mathbf{\diff^{\alpha}}$}\index{Differential Operator} for
$\DS \alpha \in \NN^n$ by
\[
\diff^\alpha f\left(\VEC{x}\right)
= \frac{\partial^{|\alpha|} f}{\partial x_1^{\alpha_1}
\partial x_2^{\alpha_2} \ldots \partial x_n^{\alpha_n}}\left(\VEC{x}\right)
\]
for $\VEC{x} \in V$, where $\DS |\alpha| = \sum_{j=1}^n \alpha_j$.

Let $V$ be an open subset of $\DS \RR^n$.  For $m \in \NN$,
$\DS C^m(V)$ is the space of all
function $f:V \to \RR$ such that all
$\DS \diff^{\alpha}f:V \to \RR$ exists and is
continuous for all $\DS \alpha \in \NN^n$ with
$|\alpha| \leq m$.  Likewise, $\DS C^\infty(V)$ is the space
of all function $f:V \to \RR$ such that all
$\DS \diff^{\alpha}f:V \to \RR$ exists and is
continuous for all $\DS \alpha \in \NN^n$.

For $m\in \NN$, we say that $\DS f \in C^m(\overline{V})$
if $f \in C(\overline{V})$, and there exist and open set
$W \supset \overline{V}$ and a function $\DS F\in C^m(W)$
such that $F = f$ on $V$.  Likewise, we say that
$\DS f \in C^\infty(\overline{V})$ if $f \in C(\overline{V})$,
and there exist and open set $W \supset \overline{V}$ and a function
$\DS F\in C^\infty(W)$ such that $F = f$ on $V$.

Let $\DS f:\RR^n \rightarrow \RR^m$ and
$\DS g:\RR^n \rightarrow \RR$ be two functions.  We write
$f\left(\VEC{x}\right) = O\left(g\left(\VEC{x}\right)\right)$ near
the origin if there exists a positive constant $K$ such that
$\DS \left\| f\left(\VEC{x}\right) \right\| <
K \left| g\left(\VEC{x}\right) \right|$
for $\VEC{x}$ in a neighbourhood of the origin.

\begin{focus*}{Note}
In these lecture notes, $\NN = \{0,1,2,3, \ldots\}$ and
$\DS \NNp = \{1,2,3,\ldots \}$.
\end{focus*}

\begin{focus*}{Note}
By tradition,  $C(V)$ means the space of
continuous real valued functions on $V$.  Namely,
$\DS C(V) = C^0(V)$.  If we really want to talk about the
space of continuously differentiable real valued functions on $V$,
then we will write $\DS C^1(V)$.
\end{focus*}

%%% Local Variables: 
%%% mode: latex
%%% TeX-master: "notes"
%%% End: 
